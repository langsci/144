%%%%%%%%%%%%%%%%%%%%%%%%%%%%%%%%%%%%%%%%%%%%%%%%%%%%
%%%                                              %%%
%%%     Language Science Press Master File       %%%
%%%         follow the instructions below        %%%
%%%                                              %%%
%%%%%%%%%%%%%%%%%%%%%%%%%%%%%%%%%%%%%%%%%%%%%%%%%%%%
% Everything following a % is ignored% Some lines start with %. Remove the % to include them

\documentclass[output=book,
 nonflat,
 modfonts,
 nobabel
                 ]{langsci/langscibook}
  
   
%%%%%%%%%%%%%%%%%%%%%%%%%%%%%%%%%%%%%%%%%%%%%%%%%%%%
%%%                                              %%%
%%%          additional packages                 %%%
%%%                                              %%%
%%%%%%%%%%%%%%%%%%%%%%%%%%%%%%%%%%%%%%%%%%%%%%%%%%%%

% put all additional commands you need in the 
% following files. {I}f you do not know what this might 
% mean, you can safely ignore this section

\usepackage[ngerman]{babel}
\usepackage{qtree}
\usepackage{amsmath}
\usepackage{pst-jtree}
\usepackage{array} 
\usepackage{mdsymbol}
\usepackage{diagbox}
\usepackage{pst-3d}
\usepackage{graphicx}
\usepackage{hyperref}
\usepackage{arydshln}
\usepackage{tabularx}
\newcolumntype{x}[1]{!{\centering\arraybackslash\vrule width #1}}
\usepackage[demo]{graphicx}
\usepackage{booktabs}
\usepackage{setspace}\usepackage{threeparttable}
\usepackage{multirow}
\usepackage{makecell}


\author{Paul R. Kroeger}
\title{Analyzing meaning}
\BackBody{This book provides an introduction to the study of meaning in human language, from a linguistic perspective. It covers a fairly broad range of topics, including lexical semantics, compositional semantics, and pragmatics. The chapters are organized into six units: (1) Foundational concepts; (2) Word meanings; (3) Implicature (including indirect speech acts); (4) Compositional semantics; (5) Modals, conditionals, and causation; (6) Tense \& aspect.

Most of the chapters include exercises which can be used for class discussion and/or homework assignments, and each chapter contains references for additional reading on the topics covered.

As the title indicates, this book is truly an INTRODUCTION: it provides a solid foundation which will prepare students to take more advanced and specialized courses in semantics and/or pragmatics. It is also intended as a reference for fieldworkers doing primary research on under-documented languages, to help them write grammatical descriptions that deal carefully and clearly with semantic issues. The approach adopted here is largely descriptive and non-formal (or, in some places, semi-formal), although some basic logical notation is introduced. The book is written at level which should be appropriate for advanced undergraduate or beginning graduate students. It presupposes some previous coursework in linguistics, but does not presuppose any background in formal logic or set theory.
}
\subtitle{An introduction to semantics and pragmatics}
\renewcommand{\lsSeries}{tbls}
\renewcommand{\lsSeriesNumber}{5}
\newcommand{\lsID}{144}
\typesetter{Felix Kopecky, Paul Kroeger, Sebastian Nordhoff}
\proofreader{
Aleksandrs Berdicevskis,
Andreas Hölzl,
Anne Kilgus,
Bev Erasmus,
Carla Parra,
Catherine Rudin,
Christian Döhler,
David Lukeš,
David Nash,
Eitan Grossman,
Eugen Costetchi,
Guohua Zhang,
Ikmi Nur Oktavianti,
Jean Nitzke,
Jeroen van de Weijer,
José Poblete Bravo,
Joseph De Veaugh,
Lachlan Mackenzie,
Luigi Talamo,
Martin Haspelmath,
Mike Aubrey,
Monika Czerepowicka,
Myke Brinkerhoff,
Parviz Parsafar,
Prisca Jerono,
Ritesh Kumar,
Sandra Auderset,
Torgrim Solstad,
Vadim Kimmelman,
Vasiliki Foufi}

\renewcommand{\lsAdditionalFontsImprint}{, AR PL UMing}
\BookDOI{10.5281/zenodo.1164112}

\renewcommand{\lsISBNdigital}{978-3-96110-034-7}
\renewcommand{\lsISBNhardcover}{978-3-96110-035-4}
\renewcommand{\lsISBNsoftcover}{978-3-96110-067-5}
% \renewcommand{\lsISBNsoftcoverus}{978-1984976550}

\input{localpackages.tex}
\input{localhyphenation.tex}
\bibliography{localbibliography} 

%%%%%%%%%%%%%%%%%%%%%%%%%%%%%%%%%%%%%%%%%%%%%%%%%%%%
%%%                                              %%%
%%%             Frontmatter                      %%%
%%%                                              %%%
%%%%%%%%%%%%%%%%%%%%%%%%%%%%%%%%%%%%%%%%%%%%%%%%%%%% 
\begin{document}     
%add all your local new commands to this file

\newcommand{\smiley}{:)}

\renewbibmacro*{index:name}[5]{%
  \usebibmacro{index:entry}{#1}
    {\iffieldundef{usera}{}{\thefield{usera}\actualoperator}\mkbibindexname{#2}{#3}{#4}{#5}}}



\newcommand{\appref}[1]{Appendix \ref{#1}}
\newcommand{\fnref}[1]{Appendix \ref{#1}}
\newcommand{\vernacular}[1]{\emph{#1}}
\newcommand{\gloss}[1]{#1}

\newenvironment{stylepoints}{\ea}{\z}
\newcommand{\furtherreading}{\section*{For further reading}}
\newcommand{\tablehead}[1]{#1}
\newcommand{\textstylest}[1]{\textsc{#1}}
\newcommand{\biberror}[1]{{\color{red}#1}}

\renewcommand{\emptyset}{⌀} 


\maketitle                
\frontmatter
% %% uncomment if you have preface and/or acknowledgements

\currentpdfbookmark{Contents}{name} % adds a PDF bookmark
\tableofcontents
\addchap{Preface}
\begin{refsection}

This book provides an introduction to the study of meaning in human language, from a linguistic perspective. It covers a fairly broad range of topics, including lexical semantics, compositional semantics, and pragmatics. The approach is largely descriptive and non-formal, although some basic logical notation is introduced.


The book is written at level which should be appropriate for advanced undergraduate or beginning graduate students. It presupposes some previous coursework in linguistics, including at least a full semester of morpho-syntax and some familiarity with phonological concepts and terminology. It does not presuppose any previous background in formal logic or set theory.



Semantics and pragmatics are both enormous fields, and an introduction to either can easily fill an entire semester (and typically does); so it is no easy matter to give a reasonable introduction to both fields in a single course. However, I believe there are good reasons to teach them together.



In order to cover such a broad range of topics in relatively little space, I have not been able to provide as much depth as I would have liked in any of them. As the title indicates, this book is truly an \textsc{introduction}: it attempts to provide students with a solid foundation which will prepare them to take more advanced and specialized courses in semantics and/or pragmatics. It is also intended as a reference for fieldworkers doing primary research on under-documented languages, to help them write grammatical descriptions that deal carefully and clearly with semantic issues. (This has been a weak point in many descriptive grammars.) At several points I have also pointed out the relevance of the material being discussed to practical applications such as translation and lexicography, but due to limitations of space this is not a major focus of attention.



The book is organized into six Units: (\ref{unit:1}) Foundational concepts; (\ref{unit:2}) Word meanings; (\ref{unit:3}) Implicature (including indirect speech acts); (\ref{unit:4}) Compositional semantics; (\ref{unit:5}) Modals, conditionals, and causation; (\ref{unit:6}) Tense \& aspect. The sequence of chapters is important; in general, each chapter draws fairly heavily on preceding chapters. The book is intended to be teachable in a typical one-semester course module. However, if the instructor needs to reduce the amount of material to be covered, it would be possible to skip Chapters~\ref{sec:6} (Lexical sense relations), \ref{sec:15} (Intensional contexts), \ref{sec:17} (Evidentiality), and\slash or \ref{sec:22} (Varieties of the perfect) without seriously affecting the students’ comprehension of the other chapters. Alternatively, one might skip the entire last section, on tense \& aspect.



Most of the chapters (after the first) include exercises which are labeled as being for “Discussion” or “Homework”, depending on how I have used them in my own teaching. (Of course other instructors are free to use them in any way that seems best to them.) A few chapters have only “Discussion exercises”, and two (Chapters~\ref{sec:15} and \ref{sec:17}) have no exercises at all in the current version of the book. Additional exercises for many of the topics covered here can be found in \citet{Saeed2009} and \citet{Kearns2000}.


The book is available for collaborative reading on the PaperHive platform at \url{https://paperhive.org/documents/remote?type=langsci&id=144}. Suggestions which will help to improve any aspect of the book will be most welcome. \textit{Soli Deo Gloria}.

\printbibliography[heading=subbibliography]
\end{refsection}
% \include{chapters/acknowledgments}
% \addchap{Abbreviations} 




\begin{tabularx}{.45\textwidth}{>{\scshape}lQ} 
acc & accusative\\
aux & auxiliary\\
comp & complementizer\\
cond & conditional\\
conject & conjecture\\
cont & continuous\\
contr & contrast\\
cop & copula\\
cos & Change of State\\
dat & dative\\
decl & declarative\\
deic & deictic\\
dem & demonstrative\\
deon & deontic\\
det & determiner\\
dim & diminutive\\
dir & direct evidence\\
emph & emphatic\\
epis & epistemic\\
erg & ergative\\
excl & exclusive\\
exclam & exclamation\\
exis & existential\\
exper & experiential aspect\\
f & feminine\\
frus & frustrative\\
fut & future\\
gen & genitive\\
hon & honorific\\
imp &  imperative; \\
impf & imparfait (French)\\
int & intimate speech\\
intr & intransitive\\
\end{tabularx}
\begin{tabularx}{.45\textwidth}{>{\scshape}lQ} 
inan & inanimate\\
ind & indicative\\
ipfv & imperfective\\
lnk & linker\\
loc & locative\\
m & masculine\\
nec & necessity\\
neg & negative\\
nom & nominative\\
npst & non-past\\
obj & object\\
pejor & pejorative\\
pfv & perfective\\
pl & plural\\
pol & polite\\
poss & possessive\\
potent & potentive\\
pred & predicative\\
prf & perfect\\
prob & probability\\
prog & progressive\\
prtcl & particle\\
ps & passé simple (French)\\
pst & past\\
ptcp & participle\\
q & question\\
rel & relativizer\\
sbjv & subjunctive\\
sg & singular\\
stat & stative\\
subj & subject\\
tr & transitive\\
\\
\end{tabularx}


 
 
\mainmatter         
  

%%%%%%%%%%%%%%%%%%%%%%%%%%%%%%%%%%%%%%%%%%%%%%%%%%%%
%%%                                              %%%
%%%             Chapters                         %%%
%%%                                              %%%
%%%%%%%%%%%%%%%%%%%%%%%%%%%%%%%%%%%%%%%%%%%%%%%%%%%%
 
\chapter{{1}: The meaning of \textit{meaning}}

\section{Semantics and pragmatics}\label{sec:} %1. /

The American author Mark Twain is said to have described a certain person as “a good man in the worst sense of the word.” The humor of this remark lies partly in the unexpected use of the word \textit{good}, with something close to the opposite of its normal meaning: Twain seems to be implying that this man is puritanical, self-righteous, judgmental, or perhaps hypocritical. Nevertheless, despite using the word in this unfamiliar way, Twain still manages to communicate at least the general nature of his intended message.



Twain’s witticism is a slightly extreme example of something that speakers do on a regular basis: using old words with new meanings. It is interesting to compare this example with the following famous conversation from \textit{Through the Looking Glass}, by Lewis Carroll:


\ea
{}[Humpty Dumpty speaking] “There’s glory for you!”

“I don’t know what you mean by ‘glory’,” Alice said.

Humpty Dumpty smiled contemptuously. “Of course you don’t — till I tell you. I meant ‘there’s a nice knock-down argument for you!’ ”

“But ‘glory’ doesn’t mean ‘a nice knock-down argument’,” Alice objected. 

“When I use a word,” Humpty Dumpty said, in rather a scornful tone, “it means just what I choose it to mean — neither more nor less.”

“The question is,” said Alice, “whether you can make words mean so many different things.”

“The question is,” said Humpty Dumpty, “which is to be master — that’s all.”
\z

Superficially, Humpty Dumpty’s comment seems similar to Mark Twain’s: both speakers use a particular word in a previously unknown way. The results, however, are strikingly different: Mark Twain successfully communicates (at least part of) his intended meaning, whereas Humpty Dumpty fails to communicate; throughout the ensuing conversation, Alice has to ask repeatedly what he means.



Humpty Dumpty’s claim to be the “master” of his words — to be able to use words with whatever meaning he chooses to assign them — is funny because it is absurd. If people really talked that way, communication would be impossible. Perhaps the most important fact about word meanings is that they must be shared by the speech community: speakers of a given language must agree, at least most of the time, about what each word means.



At the same time, Twain’s remark illustrates how word meanings can be stretched or extended in various novel ways, without loss of comprehension on the part of the hearer. The contrast between Mark Twain’s successful communication and Humpty Dumpty’s failure to communicate suggests that the conventions for extending meanings must also be shared by the speech community. In other words, there seem to be rules even for bending the rules. In this book we will be interested both in the rules for “normal” communication, and in the rules for bending the rules.



The term \textsc{semantics} is often defined as the study of meaning. It might be more accurate to define it as the study of the relationship between linguistic form and meaning. This relationship is clearly rule-governed, just as other aspects of linguistic structure are. For example, no one believes that speakers memorize every possible sentence of a language; this cannot be the case, because new and unique sentences are produced every day, and are understood by people hearing them for the first time. Rather, language learners acquire a vocabulary (lexicon), together with a set of rules for combining vocabulary items into well-formed sentences (syntax). The same logic forces us to recognize that language learners must acquire not only the meanings of vocabulary items, but also a set of rules for interpreting the expressions that are formed when vocabulary items are combined. All of these components must be shared by the speech community in order for linguistic communication to be possible. When we study semantics, we are trying to understand this shared system of rules that allows hearers to correctly interpret what speakers intend to communicate.



The study of meaning in human language is often partitioned into two major divisions, and in this context the term \textsc{semantics} is used to refer to one of these divisions. In this narrower sense, semantics is concerned with the inherent meaning of words and sentences as linguistic expressions, in and of themselves, while \textsc{pragmatics} is concerned with those aspects of meaning that depend on or derive from the way in which the words and sentences are used. In the above-mentioned quote attributed to Mark Twain, the basic or “default” meaning of \textit{good} (the sense most likely to be listed in a dictionary) would be its semantic content. The negative meaning which Twain manages to convey is the result of pragmatic inferences triggered by the peculiar way in which he uses the word.



The distinction between semantics and pragmatics is useful and important, but as we will see in \chapref{sec:9}, the exact dividing line between the two is not easy to draw and continues to be a matter of considerable discussion and controversy. Because semantics and pragmatics interact in so many complex ways, there are good reasons to study them together, and both will be of interest to us in this book.


\section{Three “levels” of meaning}\label{sec:} %2. /

In this book we will be interested in the meanings of three different types of linguistic units:


\begin{enumerate}
\item \bfseries
word meaning
\item \bfseries
sentence meaning
\item \textbf{utterance meaning} (also referred to as “speaker meaning”)
\end{enumerate}

The first two units (words and sentences) are hopefully already familiar to the reader. In order to understand the third level, “utterance meaning”, we need to distinguish between sentences vs. utterances. A sentence is a linguistic expression, a well-formed string of words, while an utterance is a speech event by a particular speaker in a specific context. When a speaker uses a sentence in a specific context, he produces an utterance. As hinted in the preceding section, the term \textsc{sentence meaning} refers to the semantic content of the sentence: the meaning which derives from the words themselves, regardless of context.\footnote{As we will see, this is an oversimplification, because certain aspects of sentence meaning do depend on context; see \chapref{sec:9}, sec. 3 for discussion.} The term \textsc{utterance} \textsc{meaning} refers to the semantic content plus any pragmatic meaning created by the specific way in which the sentence gets used. \citet[27]{Cruse2000} defines utterance meaning as “the totality of what the speaker intends to convey by making an utterance.”



\citet[1]{Kroeger2005} cites the following example of a simple question in Teochew (a Southern Min dialect of Chinese), whose interpretation depends heavily on context.


\ea \label{ex:2}
Lɯ  chyaʔ  pa  bɔy?\\
you  eat  full  not.yet\\
‘Have you already eaten?’  (tones not indicated)
\z


The literal meaning (i.e., sentence meaning) of the question is, “Have you already eaten or not?”, which sounds like a request for information. But its most common use is as a greeting. The normal way for one friend to greet another is to ask this question. (The expected reply is: “I have eaten,” even if this is not in fact true.) In this context, the utterance meaning is roughly equivalent to that of the English expressions \textit{hello} or \textit{How do you do?} In other contexts, however, the question could be used as a real request for information. For example, if a doctor wants to administer a certain medicine which cannot be taken on an empty stomach, he might well ask the patient “Have you eaten yet?” In this situation the sentence meaning and the utterance meaning would be essentially the same.


\section{3. Relation between form and meaning}\label{sec:}

For most words, the relation between the form (i.e., phonetic shape) of the word and its meaning is arbitrary. This is not always the case. \textsc{Onomatopoetic} words are words whose forms are intended to be imitations of the sounds which they refer to, e.g. \textit{ding-dong} for the sound of a bell, or \textit{buzz} for the sound of a housefly. But even in these cases, the phonetic shape of the word (if it is truly a part of the vocabulary of the language) is partly conventional. The sound a dog makes is represented by the English word \textit{bow-wow}, the Balinese word \textit{kong-kong}, the Armenian word \textit{haf-haf}, and the Korean words \textit{mung-mung} or \textit{wang-wang}.\footnote{source: \url{http://www.psychologytoday.com/blog/canine-corner/201211/how-dogs-bark-in-different-languages}} This cross-linguistic variation is presumably not motivated by differences in the way dogs actually bark in different parts of the world. On the other hand, as these examples indicate, there is a strong tendency for the corresponding words in most languages to use labial, velar, or labio-velar consonants and low back vowels.\footnote{Labial consonants such as /b,m/; velar consonants such as /g, ng/; or labio-velar consonants such as /w/. Low back vowels include /a,o/.} Clearly this is no accident, and reflects the non-arbitrary nature of the form-meaning relation in such words. The situation with “normal” words is quite different, e.g. the word for ‘dog’: Armenian \textit{shun}, Balinese \textit{cicin}, Korean \textit{gae}, Tagalog \textit{aso}, etc. No common phonological pattern is to be found here.



The relation between the form of a sentence (or other multi-word expression) and its meaning is generally not arbitrary, but \textsc{compositional}. This term means that the meaning of the expression is predictable from the meanings of the words it contains and the way they are combined. To give a very simple example, suppose we know that the word \textit{yellow} can be used to describe a certain class of objects (those that are yellow in color) and that the word \textit{submarine} can be used to refer to objects of another sort (those that belong to the class of submarines). This knowledge, together with a knowledge of English syntax, allows us to infer that when the Beatles sang about living in a \textit{yellow} \textit{submarine} they were referring to an object that belonged to both classes, i.e., something that was both yellow and a submarine.



This \textsc{principle of compositionality} is of fundamental importance to almost every topic in semantics, and we will return to it often. But once again, there are exceptions to the general rule. The most common class of exceptions are \textsc{idioms}, such as \textit{kick the bucket} for ‘die’ or \textit{X’s goose is cooked} for ‘X is in serious trouble’. Idiomatic phrases are by definition non-compositional: the meaning of the phrase is not predictable from the meanings of the individual words. The meaning of the whole phrase must be learned as a unit.



The relation between utterance meaning and the form of the utterance is neither arbitrary nor, strictly speaking, compositional. Utterance meanings are derivable (or “calculable”) from the sentence meaning and the context of the utterance by various pragmatic principles that we will discuss in later chapters. However, it is not always fully predictable; sometimes more than one interpretation may be possible for a given utterance in a particular situation.


\section{4. What does \textit{mean} mean?}\label{sec:}

When someone defines semantics as “the study of meaning”, or pragmatics as “the study of meanings derived from usage”, they are defining one English word in terms of other English words. This practice has been used for thousands of years, and works fairly well in daily life. But if our goal as linguists is to provide a rigorous or scientific account of the relationship between form and meaning, there are obvious dangers in using this strategy. To begin with, there is the danger of circularity: a definition can only be successful if the words used in the definition are themselves well-defined. In the cases under discussion, we would need to ask: What is the meaning of \textit{meaning}? What does \textit{mean} mean?



One way to escape from this circularity is to translate expressions in the \textsc{object language} into a well-defined \textsc{metalanguage}. If we use English to describe the linguistic structure of Swahili, Swahili is the object language and English is the metalanguage. However, both Swahili and English are natural human languages which need to be analyzed, and both exhibit vagueness, ambiguities, and other features which make them less than ideal as a semantic metalanguage.



Many linguists adopt some variety of formal logic as a semantic metalanguage, and later chapters in this book provide a brief introduction to such an approach. Much of the time, however, we will be discussing the meaning of English expressions using English as the metalanguage. For this reason it becomes crucial to distinguish (object language) expressions we are trying to analyze from the (metalanguage) words we are using to describe our analysis. When we write “What is the meaning of \textit{meaning}?” or “What does \textit{mean} mean?”, we use italics to identify object language expressions. These italicized words are said to be \textsc{mentioned}, i.e., referred to as objects of study, in contrast to the metalanguage words which are \textsc{used} in their normal sense, and are written in plain font.



Let us return to the question raised above, “What do we mean by \textit{meaning}?” This is a difficult problem in philosophy, which has been debated for centuries, and which we cannot hope to resolve here; but a few basic observations will be helpful. We can start by noting that our interests in this book, and the primary concerns of linguistic semantics, are for the most part limited to the kinds of meaning that people intend to communicate via language. We will not attempt to investigate the meanings of “body language”, manner of dress, facial expressions, gestures, etc., although these can often convey a great deal of information. (In sign languages, of course, facial expressions and gestures do have linguistic meaning.) And we will not address the kinds of information that a hearer may acquire by listening to a speaker, which the speaker does not intend to communicate.



For example, if I know how your voice normally sounds, I may be able to deduce from hearing you speak that you have laryngitis, or that you are drunk. These are examples of what the philosopher Paul Grice called “natural meaning”, rather than linguistic meaning. Just as smoke “means” fire, and a rainbow “means” rain, a rasping whisper “means” laryngitis. \citet[15]{Levinson1983} uses the example of a detective questioning a suspect to illustrate another type of unintended communication. The suspect may say something which is inconsistent with the physical evidence, and this may allow the detective to deduce that the suspect is guilty, but his guilt is not part of what the suspect intends to communicate. Inferences of this type will not be a central focus of interest in this book.



An approach which has proven useful for the linguistic analysis of meaning is to focus on how speakers use language to talk about the world. This approach was hinted at in our discussion of the phrase \textit{yellow} \textit{submarine}. Knowing the meaning of words like \textit{yellow} or \textit{submarine} allows us to identify the class of objects in a particular situation, or universe of discourse, which those words can be used to refer to. Similarly, knowing the meaning of a sentence will allow us to determine whether that sentence is true in a particular situation or universe of discourse.



Technically, sentences like \textit{It is raining} are neither true nor false. Only an utterance of a certain kind (namely, a statement) can have a truth value. When a speaker utters this sentence at a particular time and place, we can look out the window and determine whether or not the speaker is telling the truth. The statement is true if its meaning corresponds to the situation being described: is it raining at that time and place? This approach is sometimes referred to as the \textsc{correspondence} theory of truth.



We might say that the meaning of a (declarative) sentence is the knowledge or information which allows speakers and hearers to determine whether it is true in a particular context. If we know the meaning of a sentence, the principle of compositionality places an important constraint on the meanings of the words which the sentence contains: the meaning of individual words (and phrases) must be suitable to compositionally determine the correct meaning for the sentence as a whole. Certain types of words (e.g., \textit{if}, \textit{and}, or \textit{but}) do not “refer” to things in the world; the meanings of such words can only be defined in terms of their contribution to sentence meanings.


\section{Saying, meaning, and doing}\label{sec:} %5. /

The Teochew question in \REF{ex:2} illustrates how a single sentence can be used to express two or more different utterance meanings, depending on the context. In one context the sentence is used to greet someone, while in another context the same sentence is used to request information. So this example demonstrates that a single sentence can be used to perform two or more different \textsc{speech acts}, i.e., things that people do by speaking.



In order to fully understand a given utterance, the addressee (= hearer) must try to answer three fundamental questions:


\begin{enumerate}
\item What did the speaker say? i.e., what is the semantic content of the sentence? (The philosopher Paul Grice used the term \textit{“What is said”} as a way of referring to semantic content or sentence meaning.)
\item What did the speaker intend to communicate? (Grice used the term \textsc{implicature} for intended but unspoken meaning, i.e., aspects of utterance meaning which are not part of the sentence meaning.)
\item What is the speaker trying to do? i.e., what speech act is being performed?
\end{enumerate}

In this book we attempt to lay a foundation for investigating these three questions about meaning. We will return to the analysis of speech acts in \chapref{sec:10}; but for a brief example of why this is an important facet of the study of meaning, consider the word \textit{please} in examples (\ref{ex:3}a--b).


\ea \label{ex:3}
\ea \textit{Please} pass me the salt.\\
\ex Can you \textit{please} pass me the salt?
                       \z
\z


What does \textit{please} mean? It does not seem to have any real semantic content, i.e., does not contribute to the sentence meaning; but it makes an important contribution to the utterance meaning, in fact, two important contributions. First, it identifies the speech act which is performed by the utterances in which it occurs, indicating that they are \textsc{requests}. The word \textit{please} does not occur naturally in other kinds of speech acts. Second, this word is a marker of politeness; so it indicates something about the manner in which the speech act is performed, including the kind of social relationship which the speaker wishes to maintain with the hearer. So we see that we cannot understand the meaning of \textit{please} without referring to the speech act being performed.



The claim that the word \textit{please} does not contribute to sentence meaning is supported by the observation that misusing the word does not affect the truth of a sentence. We said that it normally occurs only in requests. If we insert the word into other kinds of speech acts, e.g. \textit{It seems to be raining, please}, the result is odd; but if the basic statement is true, adding \textit{please} does not make it false. Rather, the use of \textit{please} in this context is simply inappropriate (unless there is some contextual factor which makes it possible to interpret the sentence as a request).



The examples in \REF{ex:3} also illustrate an important aspect of how form and meaning are related with respect to speech acts. We will refer to the utterance in (\ref{ex:3}a) as a \textsc{direct} request, because the grammatical form (imperative) matches the intended speech act (request); so the utterance meaning is essentially the same as the sentence meaning. We will refer to the utterance in (\ref{ex:3}b) as an \textsc{indirect} request, because the grammatical form (interrogative) does not match the intended speech act (request); the utterance meaning must be understood by pragmatic inference.


\section{“More lies ahead” (a roadmap)}\label{sec:} %6. /

As you have seen from the table of contents, the chapters of this book are organized into six units. In the first four units we introduce some of the basic tools, concepts, and terminology which are commonly used for analyzing and describing linguistic meaning. In the last two units we use these tools to explore the meanings of several specific classes of words and grammatical markers: modals, tense markers, \textit{if}, \textit{because}, etc.



The rest of this first unit is devoted to exploring two of the foundational concepts for understanding how we talk about the world: reference and truth. \chapref{sec:2} deals with reference and the relationship between reference and meaning. Just as a proper name can be used to refer to a specific individual, other kinds of noun phrase can be used to refer to people, things, groups, etc. in the world. The actual reference of a word or phrase depends on the context in which it is used; the meaning of the word determines what things it can be used to refer to in any given context.



\chapref{sec:3} deals with truth, and also with certain kinds of inference. We say that a statement is true if its meaning corresponds to the situation under discussion. Sometimes the meanings of two statements are related in such a way that the truth of one will give us reason to believe that the other is also true. For example, if I know that the statement in (\ref{ex:4}a) is true, then I can be quite certain that the statement in (\ref{ex:4}b) is also true, because of the way in which the meanings of the two sentences are related. A different kind of meaning relation gives us reason to believe that if a person says (\ref{ex:4}c), he must believe that the statement in (\ref{ex:4}a) is true. These two types of meaning-based inference, which we will call \textsc{entailment} and \textsc{presupposition} respectively, are of fundamental importance to most of the topics discussed in this book.


\ea \label{ex:4}
\ea John killed the wasp.\\
\ex The wasp died.\\
\ex John is proud that he killed the wasp.
                       \z
\z


\chapref{sec:4} introduces some basic logical notation that is widely used in semantics, and discusses certain patterns of inference based on truth values and logical structure.



Unit 2 focuses on word meanings, starting with the observation that a single word can have more than one meaning. One of the standard ways of demonstrating this fact is by observing the ambiguity of sentences like the famous headline in \REF{ex:5}. Many of the issues we discuss in Unit 2 with respect to “content words” (nouns, verbs, adjectives, etc.), such as ambiguity, vagueness, idiomatic uses, co-occurrence restrictions, etc., will turn out to be relevant in our later discussions of various kinds of “function words” and grammatical morphemes as well.


\ea \label{ex:5}
Headline: \textit{Reagan wins on budget, but more lies ahead}.
\z


Unit 3 deals with a pattern of pragmatic inference known as \textsc{conversational implicature}: meaning which is intended by the speaker to be understood by the hearer, but is not part of the literal sentence meaning. Many people consider the identification of this type of inference, by the philosopher Paul Grice in the 1960’s, to be the “birth-date” of pragmatics as a distinct field of study. It is another foundational concept that we will refer to in many of the subsequent chapters. \chapref{sec:10} discusses a class of conversational implicatures that has received a great deal of attention, namely indirect speech acts. As illustrated above in example (\ref{ex:3}b), an indirect speech act involves a sentence whose literal meaning seems to perform one kind of speech act (asking a question: \textit{Can you pass me the salt?}) used in a way which implicates a different speech act (request: \textit{Please pass me the salt}). \chapref{sec:11} discusses various types of expressions (e.g. sentence adverbs like \textit{frankly}, \textit{fortunately}, etc., honorifics and politeness markers, and certain types of “discourse particles”) whose meanings seem to contribute to the appropriateness of an utterance, rather than to the truth of a proposition. Some such meanings were referred to by Grice as a different kind of implicature.



Unit 4 addresses the issue of compositionality: how the meanings of phrases and sentences can be predicted based on the meanings of the words they contain and the way those words are arranged (syntactic structure). It provides a brief introduction to some basic concepts in set theory, and shows how these concepts can be used to express the truth conditions of sentences. One topic of special interest is the interpretation of “quantified” noun phrases such as \textit{every person}, \textit{some animal}, or \textit{no student}, using set theory to state the meanings of such phrases. In Unit 5 we will use this analysis of quantifiers to provide a way of understanding the meanings of modals (e.g. \textit{may}, \textit{must}, \textit{should}) and \textit{if} clauses.



Unit 6 presents a framework for analyzing the meanings of tense and aspect markers. Tense and aspect both deal with time reference, but in different ways. As we will see, the use and interpretation of these markers often depends heavily on the type of situation being described.



Each of these topics individually has been the subject of countless books and papers, and we cannot hope to give a complete account of any of them. This book is intended as a broad introduction to the field as a whole, a stepping stone which will help prepare you to read more specialized books and papers in areas that interest you.



\furtherreading



For helpful discussions of the distinction between semantics vs. pragmatics, see Levinson (1983, ch. 1) and Birner (2013, sec. 1.2). Levinson (1983, ch. 1) also provides a helpful discussion of Grice’s distinction between “natural meaning” vs. linguistic meaning.


\chapter{{2}: Referring, denoting, and expressing}

\section{Talking about the world}\label{sec:} %1. /

In this chapter and the next we will think about how speakers use language to talk about the world. Referring to a particular individual, e.g. by using expressions such as \textit{Abraham Lincoln} or \textit{my father}, is one important way in which we talk about the world. Another important way is to describe situations in the world, i.e., to claim that a certain state of affairs exists. These claims are judged to be true if our description matches the actual state of the world, and false otherwise. For example, if I were to say \textit{It is raining} at a time and place where no rain is falling, I would be making a false statement.



We will focus on truth in the next chapter, but in this chapter our primary focus is on issues relating to reference. We begin in section two with a very brief description of two ways of studying linguistic meaning. One of these looks primarily at how a speaker’s words are related to the thoughts or concepts he is trying to express. The other approach looks primarily at how a speaker’s words are related to the situation in the world that he is trying to describe. This second approach will be assumed in most of this book.



In section three we will think about what it means to “refer” to things in the world, and discuss various kinds of expressions that speakers can use to refer to things. In section four we will see that we cannot account for meaning, or even reference, by looking only at reference. To preview that discussion, we might begin with the observation that people talk about the “meaning” of words in two different ways, as illustrated in \REF{ex:2.1}. In (\ref{ex:2.1}a), the word \textit{meant} is used to specify the reference of a phrase when it was used on a particular occasion, whereas in (\ref{ex:2.1}b-c), the word \textit{means} is used to specify the kind of meaning that we might look up in a dictionary.


\ea \label{ex:2.1}
\ea When Jones said that he was meeting “a close friend” for dinner, he meant his lawyer.\\
\ex \textit{Salamat} means ‘thank you’ in Tagalog.\\
\ex \textit{Usufruct} means ‘the right of one individual to use and enjoy the property\\
  of another.’\footnote{http://legal-dictionary.thefreedictionary.com/usufruct}
\z
\z


We will introduce the term \textsc{sense} for the kind of meaning illustrated in (\ref{ex:2.1}b-c), the kind of meaning that we might look up in a dictionary. One crucial difference between sense and reference is that reference depends on the specific context in which a word or phrase is used, whereas sense does not depend on context in this way.



In section five we discuss various types of \textsc{ambiguity}, that is, ways in which a word, phrase or sentence can have more than one sense. The existence of ambiguity is an important fact about all human languages, and accounting for ambiguity is an important goal in semantic analysis.



In section six we discuss a kind of meaning that does not seem to involve either reference to the world, or objective claims about the world. \textsc{Expressive} meaning (e.g. the meanings of words like \textit{ouch} and \textit{oops}) reflects the speaker’s feelings or attitudes at the time of speaking. We will list a number of ways in which expressive meaning is different from “normal” \textsc{descriptive} meaning.


\section{2. Denotational semantics vs. cognitive semantics}\label{sec:}

Let us begin by discussing the relationships between a speaker’s words, the situation in the world, and the thoughts or concepts associated with those words. These relationships are indicated in the figure in \REF{ex:2.2}, which is a version of a diagram that is sometimes referred to as the Semiotic Triangle.




\todo{drawobject}

\ea \label{ex:2.2}
 \begin{tikzpicture}
  \node[regular polygon, regular polygon sides=3, minimum size=3cm,draw] (polygon3) {};
  \node[shift=(polygon3.corner 1),above] {\sffamily Mind}; 
  \node[shift=(polygon3.corner 3),below right] {\sffamily World}; 
  \node[shift=(polygon3.corner 2),below left] {\sffamily Language}; 
 \end{tikzpicture}
% %  \caption{(one version of) the Semiotic Triangle}
\z


Semiotics is the study of the relationship between signs and their meanings. In this book we are interested in the relationship between forms and meanings in certain kinds of symbolic systems, namely human languages. The diagram is a way of illustrating how speakers use language to describe things, events, and situations in the world. As we will see when we begin to look at word meanings, what speakers actually describe is a particular \textsc{construal}, or way of thinking about, the situation. Moreover, the speaker’s linguistic description rarely if ever includes everything that the speaker knows or believes about the situation. And, of course, what the speaker knows or believes about the situation may not match the actual state of the world. So there is no one-to-one correspondence between the speaker’s mental representation and either the actual situation in the world or the linguistic expressions used to describe that situation. But clearly there are strong links or associations connecting each of these domains with the others.



The basic approach we adopt in this book focuses on the link between linguistic expressions and the world. This approach is often referred to as \textsc{denotational} semantics. (We will discuss what \textsc{denotation} means in \sectref{sec:4} below.) An important alternative approach, \textsc{cognitive semantics}, focuses on the link between linguistic expressions and mental representations. Of course both approaches recognize that all three corners of the Semiotic Triangle are involved in any act of linguistic communication. One motivation for adopting a denotational approach comes from the fact that it is very hard to find direct evidence about what is really going on in a speaker’s mind. A second motivation is the fact that this approach has proven to be quite successful at accounting for compositionality (how meanings of complex expressions, e.g. sentences, are related to the meanings of their parts).



The two foundational concepts for denotational semantics, i.e. for talking about how linguistic expressions are related to the world, are \textsc{truth} and \textsc{reference}. As we mentioned in \chapref{sec:1}, we will say that a sentence is true if it corresponds to the actual situation in the world which it is intended to describe. It turns out that native speakers are fairly good at judging whether a given sentence would be true in a particular situation; such judgments provide an important source of evidence for all semantic analysis. Truth will be the focus of attention in \chapref{sec:3}. In the next several sections of this chapter we focus on issues relating to reference.


\section{3. Types of referring expressions}\label{sec:}

Philosophers have found it hard to agree on a precise definition for \textit{reference}, but intuitively we are talking about the speaker’s use of words to “point to” something in the world; that is, to direct the hearer’s attention to something, or to enable the hearer to identify something. Suppose we are told that Brazilians used to “refer to” Pelé as \textit{o rei} ‘the king’.\footnote{Of course, Pelé rose to fame long after Brazil became a republic, so there was no king ruling the country at that time.} This means that speakers used the phrase \textit{o rei} to direct their hearers’ attention to a particular individual, namely the most famous soccer player of the 20\textsuperscript{th} century. Similarly, we might read that amyotrophic lateral sclerosis (ALS) is often “referred to” as Lou Gehrig’s Disease, in honor of the famous American baseball player who died of this disease. This means that people use the phrase \textit{Lou Gehrig’s Disease} to direct their hearers’ attention to that particular disease.



A \textsc{referring expression} is an expression (normally some kind of noun phrase) which a speaker uses to refer to something. The identity of the referent is determined in different ways for different kinds of referring expressions. A proper name like \textit{King Henry VIII}, \textit{Abraham Lincoln}, or \textit{Mao Zedong}, always refers to the same individual. (In saying this, of course, we are ignoring various complicating factors, such as the fact that two people may have the same name. We will focus for the moment on the most common or basic way of using proper names, namely in contexts where they have a single unambiguous referent.) For this reason, they are sometimes referred to as \textsc{rigid designators}. “Natural kind” terms, e.g. names of species (\textit{camel, octopus, durian}) or substances (\textit{gold, salt, methane}), are similar. When they are used to refer to the species as a whole, or the substance in general, rather than any specific instance, these terms are also rigid designators: their referent does not depend on the context in which they are used. Some examples of this usage are presented in \REF{ex:2.3}.


\ea \label{ex:2.3}
\ea \textit{The octopus} has eight tentacles and is quite intelligent.\\
\ex \textit{Camels} can travel long distances without drinking.\\
\ex \textit{Methane} is lighter than air and highly flammable.
\z
\z


For most other referring expressions, reference does depend on the context of use. \textsc{deictic} elements (sometimes called \textsc{indexicals}) are words which refer to something in the speech situation itself. For example, the pronoun \textit{I} refers to the current speaker, while \textit{you} refers to the current addressee. \textit{Here} typically refers to the place of the speech event, while \textit{now} typically refers to the time of the speech event.



Third person pronouns can be used with deictic reference, e.g. “Who is \textit{he}?” (while pointing); but more often are used anaphorically. An \textsc{anaphoric} element is one whose reference depends on the reference of another NP within the same discourse. (This other NP is called the \textsc{antecedent}.) The pronoun \textit{he} in sentence \REF{ex:2.4} is used anaphorically, taking \textit{George} as its antecedent.


\ea \label{ex:2.4}
Susan refuses to marry George\textsubscript{i} because he\textsubscript{i} smokes.
\z


Pronouns can be used with quantifier phrases, like the pronoun \textit{his} in sentence (\ref{ex:2.5}a); but in this context, the pronoun does not actually refer to any specific individual. So in this context, the pronoun is not a referring expression.\footnote{Pronouns used in this way are functioning as “bound variables”, as described in \chapref{sec:4}.} For the same reason, quantifier phrases are not referring expressions, as illustrated in (\ref{ex:2.5}b). (The symbol “\#” in (\ref{ex:2.5}b) indicates that the sentence is grammatical but unacceptable on semantic or pragmatic grounds.)


\ea \label{ex:2.5}
\ea{} [Every boy]\textsubscript{i} should respect his\textsubscript{i} mother.\\        
\ex{} [Every American male]\textsubscript{i} loves football; \#he\textsubscript{i} watched three games last weekend.
\z
\z

Some additional examples that illustrate why quantified noun phrases cannot be treated as referring expressions are presented in (\ref{ex:2.6}--\ref{ex:2.8}). As example (\ref{ex:2.6}a) illustrates, reflexive pronouns are normally interpreted as having the same reference as their antecedent; but this principle does not hold when the antecedent is a quantified noun phrase (\ref{ex:}b).


\ea \label{ex:2.6}
\ea \textit{John trusts himself}  is equivalent to:  \textit{John trusts John}.\\
\ex \textit{Everyone trusts himself}  is not equivalent to:  \textit{Everyone trusts everyone}.
\z
\z


As we discuss in \chapref{sec:3}, a sentence of the form \textit{X is Estonian and X is not Estonian} is a contradiction; it can never be true, whether X refers to an individual as in (\ref{ex:2.7}b) or a group of individuals as in (\ref{ex:2.7}c). However, when X is replaced by certain quantified noun phrases, e.g. those beginning with \textit{some} or \textit{many}, the sentence could be true. This shows that these quantified noun phrases cannot be interpreted as referring to either individuals or groups of individuals.\footnote{Peters \& Westerståhl (2006: 49–52) present a mathematical proof showing that quantified noun phrases cannot be interpreted as referring to sets of individuals.}


\begin{stylepoints}  \label{ex:2.7}
\ea \#X is Estonian and X is not Estonian.\\
\ex \#John is Estonian and John is not Estonian.\\
\ex \#My parents are Estonian and my parents are not Estonian.\\
\ex Some/many people are Estonian and some/many people are not Estonian.
\z
\end{stylepoints}


As a final example, the contrast in \REF{ex:2.8} suggests that neither \textit{every student} nor \textit{all students} can be interpreted as referring to the set of all students, e.g. at a particular school. There is much more to be said about quantifiers. We will give a brief introduction to this topic in \chapref{sec:3}, and discuss them in more detail in \chapref{sec:14}.


\ea \label{ex:2.8}
\ea The student body outnumbers the faculty.\\                
\ex \#Every student outnumbers the faculty.\\
\ex \#All students outnumbers the faculty.
\z
\z


Common noun phrases may or may not refer to anything. Definite noun phrases (sometimes called \textsc{definite descriptions}) like those in \REF{ex:2.9} are normally used in contexts where the hearer is able to identify a unique referent. But definite descriptions can also be used generically, without referring to any specific individual, like the italicized phrases in \REF{ex:2.10}.


\ea \label{ex:2.9}
\ea this book\\
\ex the sixteenth President of the United States\\
\ex my eldest brother
                       \z
\z

\ea \label{ex:2.10}
Life’s battles don’t always go\\
   To \textit{the stronger or faster man},\\
But sooner or later \textit{the man who wins}\\
   Is \textit{the one who thinks he can}.\\
(from the poem “Thinking” by Walter D. Wintle, first published 1905?)\footnote{This poem is widely copied and often mis-attributed. Authors wrongly credited with the poem include Napoleon Hill, C.W. Longenecker, and the great American football coach Vince Lombardi.}
\z


\textsc{Indefinite descriptions} may be used to refer to a specific individual, like the object NP in (\ref{ex:2.11}a); or they may be non-specific, like the object NP in (\ref{ex:2.11}b). Specific indefinites are referring expressions, while non-specific indefinites are not.


\ea \label{ex:2.11}
\ea My sister has just married \textit{a cowboy}.\\
\ex My sister would never marry \textit{a cowboy}.\\
\ex My sister wants to marry \textit{a cowboy}.
                       \z
\z


In some contexts, like (\ref{ex:2.11}c), an indefinite NP may be ambiguous between a specific vs. a non-specific interpretation. Under the specific interpretation, (\ref{ex:2.11}c) says that my sister wants to marry a particular individual, who happens to be a cowboy. Under the non-specific interpretation, (\ref{ex:2.11}c) says that my sister would like the man she marries to be a cowboy, but doesn’t have any particular individual in mind yet. We will discuss this kind of ambiguity in more detail in \chapref{sec:12}.


\section{Sense vs. denotation}\label{sec:} %4. /

In \sectref{sec:1} we noted that when people talk about what a word or phrase “means”, they may have in mind either its dictionary definition or its referent in a particular context. The German logician Gottlob Frege (1848–1925) was one of the first people to demonstrate the importance of making this distinction. He used the German term \textit{Sinn} (English \textsc{sense}) for those aspects of meaning which do not depend on the context of use, the kind of meaning we might look up in a dictionary.



Frege used the term \textit{Bedeutung} (English \textsc{denotation})\footnote{The term \textit{Bedeutung} is often translated into English as \textit{reference}, but this can lead to confusion when dealing with non-referring expressions which nevertheless do have a denotation.} for the other sort of meaning, which does depend on the context. The denotation of a referring expression, such as a proper name or definite NP, will normally be its referent. The denotation of a content word (e.g. an adjective, verb, or common noun) is the set of all the things in the current universe of discourse which the word could be used to describe. For example, the denotation of \textit{yellow} is the set of all yellow things, the denotation of \textit{tree} is the set of all trees, the denotation of the intransitive verb \textit{snore} is the set of all creatures that snore, etc. Frege proposed that the denotation of a sentence is its truth value. We will discuss his reasons for making this proposal in \chapref{sec:12}; in this section we focus on the denotations of words and phrases.



We have said that denotations are context-dependent. This is not so easy to see in the case of proper names, because they always refer to the same individual. Other referring expressions, however, will refer to different individuals or entities in different contexts. For example, the definite NP \textit{the Prime Minister} can normally be used to identify a specific individual. Which particular individual is referred to, however, depends on the time and place. The denotation of this phrase in Singapore in 1975 would have been Lee Kuan Yew; in England in 1975 it would have been Harold Wilson; and in England in 1989 it would have been Margaret Thatcher. Similarly, the denotation of phrases like \textit{my favorite color} or \textit{your father} will depend on the identity of the speaker and/or addressee.



The denotation of a content word depends on the situation or universe of discourse in which it is used. In our world, the denotation set of \textit{talks} will include most people, certain mechanical devices (computers, GPS systems, etc.) and (perhaps) some parrots. In Wonderland, as described by Lewis Carroll, it will include playing cards, chess pieces, at least one white rabbit, at least one cat, a dodo bird, etc. In Narnia, as described by C.S. Lewis, it will include beavers, badgers, wolves, some trees, etc.



For each situation, the sense determines a denotation set, and knowing the sense of the word allows speakers to identify the members of this set. When Alice first hears the white rabbit talking, she may be surprised. But her response would not be, “What is that rabbit doing?” or “Has the meaning of \textit{talk} changed?” but rather “How can that rabbit be talking?” It is not the language that has changed, but the world. Sense is a fact about the language, denotation is a fact about the world or situation under discussion.



Two expressions that have different senses may still have the same denotation in a particular situation. For example, the phrases \textit{the largest land mammal} and \textit{the African bush elephant} refer to the same organism in our present world (early in the 21\textsuperscript{st} century). But in a fictional universe of discourse (e.g., the movie \textit{King Kong}), or in an earlier time period of our own world (e.g., 30 million BC, when the gigantic \textit{Paraceratherium} —estimated weight about 20,000 kg— walked the earth), these two phrases could have different denotations. If two expressions can have different denotations in any context, they do not have the same sense.



Such examples demonstrate that two expressions which have different senses \textsc{may} have the same denotation in certain situations. However, two expressions that have the same sense (i.e., \textsc{synonymous} expressions) must \textsc{always} have the same denotation in any possible situation. For example, the phrases \textit{my mother-in-law} and \textit{the mother of my spouse} seem to be perfect synonyms (i.e., identical in sense). If this is true, then it will be impossible to find any situation where they would refer to different individuals when spoken by the same (monogamous) speaker under exactly the same conditions.



So, while we have said that we will adopt a primarily “denotational” approach to semantics, this does not mean that we are only interested in denotations, or that we believe that denotation is all there is to meaning. If meaning was just denotation, then phrases like those in \REF{ex:2.12}, which have no referent in our world at the present time, would all either mean the same thing, or be meaningless. But clearly they are not meaningless, and they do not all mean the same thing; they simply fail to refer. 


\ea \label{ex:2.12}
\ea the present King of France\\
\ex the largest prime number\\
\ex the diamond as big as the Ritz\\
\ex the unicorn in the garden
                       \z
\z


Frege’s distinction allows us to see that non-referring expressions like those in \REF{ex:2.12} may not have a referent, but they do have a sense, and that sense is derived in a predictable way by the normal rules of the language.


\section{Ambiguity}\label{sec:} %5. /

It is possible for a single word to have more than one sense. For example, the word \textit{hand} can refer to the body part at the end of our arms; the pointer on the dial of a clock; a bunch of bananas; the group of cards held by a single player in a card game; or a hired worker. Words that have two or more senses are said to be \textsc{ambiguous} (more precisely, \textsc{polysemous}; see \chapref{sec:5}).



A deictic expression such as \textit{my father} will refer to different individuals when spoken by different speakers, but this does not make it ambiguous. As emphasized above, the fact that a word or phrase can have different denotations in different contexts does not mean that it has multiple senses, and it is important to distinguish these two cases. We will discuss the basis for making this distinction in \chapref{sec:5}.



If a phrase or sentence contains an ambiguous word, the phrase or sentence will normally be ambiguous as well, as illustrated in \REF{ex:2.13}.


\ea \label{ex:2.13}
\textsc{Lexical ambiguity}\\
\ea A boiled egg is hard to \textit{beat}.\\
\ex The farmer allows walkers to cross the field for free, but the bull \textit{charges}.\\
\ex I just turned 51, but I have a nice new \textit{organ} which I enjoy tremendously.\footnote{from e-mail newsletter, 2011.}
                       \z
\z


An ambiguous sentence is one that has more than one sense, or “reading”. A sentence which has only a single sense may have different truth values in different contexts, but will always have one consistent truth value in any specific context. With an ambiguous sentence, however, there must be at least one conceivable context in which the two senses would have different truth values. For example, one reading of (\ref{ex:2.13}b) would be true at the same time that the other reading is false if there is a bull in the field which is aggressive but not financially sophisticated.



In addition to lexical ambiguity of the kind illustrated in \REF{ex:2.13}, there are various other ways in which a sentence can be ambiguous. One of these is referred to as \textsc{structural ambiguity}, illustrated in (\ref{ex:2.14}a--d). In such cases, the two senses (or readings) arise because the grammar of the language can assign two different structures to the same string of words, even though none of those words is itself ambiguous. The two different structures for (\ref{ex:2.14}d) are shown by the bracketing in (\ref{ex:2.14}e), which corresponds to the expected reading, and (\ref{ex:2.14}f) which corresponds to the Groucho Marx reading. Of course, some sentences involve both structural and lexical ambiguity, as is the case in (\ref{ex:2.14}c).


\ea \label{ex:2.14}
\textsc{Structural ambiguity}\footnote{These examples are taken from \citet[102]{Pinker1994}. The first three are said to be actual newspaper examples.}\\
\ea Two cars were reported stolen by the Groveton police yesterday.\\
\ex The license fee for altered dogs with a certificate will be \$3 and for pets owned\\
  by senior citizens who have not been altered the fee will be \$1.50.\\
\ex For sale: mixing bowl set designed to please a cook with round bottom for\\
  efficient beating.\\
\ex One morning I shot an elephant in my pajamas. How he got into my pajamas\\
  I’ll never know.  (Groucho Marx, in the movie \textit{Animal Crackers})\\
\ex One morning I [shot an elephant] [in my pajamas].\\
\ex One morning I shot [an elephant in my pajamas].
                       \z
\z


Structural ambiguity shows us something important about meaning, namely that meanings are not assigned to strings of phonological material but to syntactic objects.\footnote{\citet[514]{Kennedy2011}.} In other words, syntactic structure makes a crucial contribution to the meaning of an expression. The two readings for (\ref{ex:2.14}d) involve the same string of words but not the same syntactic object.



A third type of ambiguity which we will mention here is \textsc{referential ambiguity}. (We will discuss additional types of ambiguity in later chapters.) It is fairly common to hear people using pronouns in a way that permits more than one possible antecedent, e.g. \textit{Adams wrote frequently to Jefferson while he was in Paris}. The pronoun \textit{he} in this sentence has ambiguous reference; it could refer either to John Adams or to Thomas Jefferson. It is also possible for other types of NP to have ambiguous reference. For example, if I am teaching a class of 14 students, and I say to the Dean \textit{My student has won a Rhodes scholarship}, there are multiple possible referents for the subject NP.



A famous example of referential ambiguity occurs in a prophecy from the oracle at Delphi, in ancient Greece. The Lydian king Croesus asked the oracle whether he should fight against the Persians. The oracle’s reply was that if Croesus made war on the Persians, he would destroy a mighty empire. Croesus took this to be a positive answer and attacked the Persians, who were led by Cyrus the Great. The Lydians were defeated and Croesus was captured; the empire which Croesus destroyed turned out to be his own.


\section{6. Expressive meaning: \textit{Ouch} and \textit{oops}}\label{sec:}

Words like \textit{ouch} and \textit{oops}, often referred to as \textsc{expressives}, present an interesting challenge to the “denotational” approach outlined above. They convey a certain kind of meaning, yet they neither refer to things in the world, nor help to determine the conditions under which a sentence would be true. In fact, it is hard to claim that they even form part of a sentence; they seem to stand on their own, as one-word utterances. The kind of meaning that such words convey is called \textsc{expressive meaning}, which \citet[44]{Lyons1995} defines as “the kind of meaning by virtue of which speakers express, rather than describe, their beliefs, attitudes and feelings.” Expressive meaning is different from \textsc{descriptive meaning} (also called \textsc{propositional meaning} or \textsc{truth-} \textsc{conditional} \textsc{meaning}), the “normal” type of meaning which determines reference and truth values. If someone says \textit{I just felt a sudden sharp pain}, he is describing what he feels; but when he says \textit{Ouch!}, he is expressing that feeling.



Words like \textit{ouch} and \textit{oops} carry only expressive meaning, and seem to be unique in other ways as well. They may not necessarily be intended to communicate. If I hurt myself when I am working alone, I will very likely say \textit{ouch} (or some other expressive with similar meaning) even though there is no one present to hear me. Such expressions seem almost like involuntary reactions, although the specific forms are learned as part of a particular language. But it is important to be aware of the distinction between expressive vs. descriptive meaning, because many “normal” words carry both types of meaning at once.



For example, the word \textit{garrulous} means essentially the same thing as \textit{talkative}, but carries additional information about the speaker’s negative attitude towards this behavior.\footnote{\citet{Barker2001}.} There are many other pairs of words which seem to convey the same descriptive meaning but differ in terms of their expressive meaning: \textit{father} vs. \textit{dad}; \textit{woman} vs. \textit{broad}; \textit{horse} vs. \textit{nag}; \textit{alcohol} vs. \textit{booze}; etc. In each case either member of the pair could be used to refer to the same kinds of things in the world; the speaker’s choice of which term to use indicates varying degrees of intimacy, respect, appreciation or approval, formality, etc.



The remainder of this section discusses some of the properties which distinguish expressive meaning from descriptive meaning.\footnote{Much of this discussion is based on \citet{Cruse1986,Cruse2000} and \citet{Potts2007c}.} These properties can be used as diagnostics when we are unsure which type of meaning we are dealing with.


\subsection{Independence}\label{sec:} %6.1 /

Expressive meaning is independent of descriptive meaning in the sense that expressive meaning does not affect the denotation of a noun phrase or the truth value of a sentence. For example, the addressee might agree with the descriptive meaning of \REF{ex:2.15} without sharing the speaker’s negative attitude indicated by the expressive term \textit{jerk}. Similarly, the addressee in \REF{ex:2.16} might agree with the descriptive content of the sentence without sharing the speaker’s negative attitude indicated by the pejorative suffix \textit{-aco}.


\ea \label{ex:2.15}
That \textit{jerk} Peterson is the only real economist on this committee.
\z

\ea \label{ex:2.16}
\gll Los  vecinos  tienen  un  pajarr-\textit{aco}  como  mascota.  [Spanish]\\
the  neighbors  have  a  bird-\textsc{pejor}  as  pet\\
\glt Descriptive: The neighbors have a pet bird.\\
Expressive: The speaker has a negative attitude towards the bird.  (\citealt{Fortin2011})
\z

\subsection{Nondisplaceability}\label{sec:} %6.2 /

\citet{Hockett1958,Hockett1960} used the term \textsc{Displacement} to refer to the fact that speakers can use human languages to describe events and situations which are separated in space and time from the speech event itself. Hockett listed this ability as one of the distinctive properties of human language, one which distinguishes it, for example, from most types of animal communication.



\citet[272]{Cruse1986} notes that this capacity for displacement holds only for descriptive meaning, and not for expressive meaning. A person can describe his own feelings in the past or future, e.g. \textit{Last month I felt a sharp pain in my chest}, or \textit{I will probably feel a lot of pain when the dentist drills my tooth tomorrow}; or the feelings of other people, e.g. \textit{She was in} \textit{a lot of pain}. But when a person says \textit{Ouch!}, it must normally express pain that is felt by the speaker at the moment of speaking.


\subsection{Immunity}\label{sec:} %6.3 /

Descriptive meaning can be negated (\ref{ex:2.17}a), questioned (\ref{ex:2.17}b), or challenged (\ref{ex:2.17}c). Expressive meaning is “immune” to all of these things, as illustrated in \REF{ex:2.18}. As we will see in later chapters, negation, questioning, and challenging are three of the standard tests for identifying truth-conditional meaning. The fact that expressive meaning cannot be negated, questioned, or challenged shows that it is not part of the truth-conditional meaning of the sentence.


\ea \label{ex:2.17}
\ea \textit{I am not feeling any pain.}\\
\ex \textit{Are you feeling any pain?}\\
\ex  \textsc{patient}: \textit{I just felt a sudden sharp pain.}\\
  \textsc{dentist}: \textit{That’s a lie — I gave you a double dose of Novocain.}\\
    (\citealt{Cruse1986}:271)
\z

\ea \label{ex:2.18}
\ea \textit{*Not ouch.}\\
                       \z
\ex \textit{*Ouch?}  (can only be interpreted as an elliptical form of the question:\\
    \textit{Did you say “Ouch”?})\\
\ex  \textsc{patient}: \textit{Ouch!}\\
  \textsc{dentist}: \#\textit{That’s a lie.}
                       \z
\z

\subsection{Scalability and repeatability}\label{sec:} %6.4 /

Expressive meaning can be intensified through repetition (as seen in ex. g below), or by the use of intonational features such as pitch, length or loudness. Descriptive meaning is generally expressible in discrete units which correspond to the lexical semantic content of individual words. Repetition of descriptive meaning tends to produce redundancy, though we should note that a number of languages do use reduplication to encode plural number, repeated actions, etc.


\subsection{6.5 Descriptive ineffability}\label{sec:}

“Effability” means ‘expressibility’. The \textsc{Effability} \textsc{Hypothesis} claims that “Each proposition can be expressed by some sentence in any natural language”;\footnote{\citet[209]{Katz1978}.} or in other words, “Whatever can be meant can be said.”\footnote{\citet[18]{Searle1969}; see also \citet[18-24]{Katz1972}; \citet[33]{Carston2002}.}



\citet{Potts2007c} uses the phrase “descriptive ineffability” to indicate that expressive meaning often cannot be adequately stated in terms of descriptive meaning. A paraphrase based on descriptive meaning (e.g. \textit{young dog} for \textit{puppy}) is often interchangeable with the original expression, as illustrated in \REF{ex:2.19}. Whenever (\ref{ex:2.19}a) is true, (\ref{ex:2.19}b) must be true as well, and vice versa. Moreover, this substitution is equally possible in questions, commands, negated sentences, etc. This is not the case with expressives, even where a descriptive paraphrase is possible, as illustrated in (\ref{ex:2.17}--\ref{ex:2.18}) above.


\ea \label{ex:2.19}
\ea \textit{Yesterday my son brought home a puppy.}\\
\ex \textit{Yesterday my son brought home a young dog.}
                       \z
\z


For many expressives there is no descriptive paraphrase available, and speakers often find it difficult to explain the meaning of the expressive form in descriptive terms. For example, most dictionaries do not attempt to paraphrase the meaning of \textit{oops}, but rather “define” it by describing the contexts in which it is normally used:


\ea






  a. “used typically to express mild apology, surprise, or dismay”\\
  (http://www.merriam-webster.com)\\
\ex “an exclamation of surprise or of apology as when someone drops something\\
  or makes a mistake” (Collins English Dictionary – Complete and Unabridged\\
  © HarperCollins Publishers1991 , 1994, 1998, 2000, 2003)
\z


This limited expressibility correlates with limited translatability. The descriptive meaning conveyed by a sentence in one language is generally expressible in other languages as well. (Whether this is always the case, as predicted by strong forms of the Effability Hypothesis, is a controversial issue.) However, it is often difficult to find an adequate translation equivalent for expressive meaning. One well known example is the ancient Aramaic term of contempt \textit{raka}, which appears in the Greek text of Matthew 5:22 (and in many English translations), presumably because there was no adequate translation equivalent in Koine Greek. (Some of the English equivalents which have been suggested include: \textit{good-for-nothing}, \textit{rascal}, \textit{empty head}, \textit{stupid}, \textit{ignorant}.) In 393 AD, St. Augustine offered the following explanation:


\begin{quote}
Hence the view is more probable which I heard from a certain Hebrew whom I had asked about it; for he said that the word does not mean anything, but merely expresses the emotion of an angry mind. Grammarians call those particles of speech which express an affection of an agitated mind \textsc{interjections}; as when it is said by one who is grieved, ‘Alas,’ or by one who is angry, ‘Hah.’ And these words in all languages are proper names, and are not easily translated into another language; and this cause certainly compelled alike the Greek and the Latin translators to put the word itself, inasmuch as they could find no way of translating it.”\footnote{\textit{On the Sermon on the Mount}, Book I, ch. 9, sec. 23; \url{http://www.newadvent.org/fathers/16011.htm}} 
\end{quote}


Whether or not Augustine was correct in his view that \textit{raka} was a pure expressive, he provides an excellent description of this class of words and the difficulty of translating them from one language to another. This quote also demonstrates that the challenges posed by expressives have been recognized for a very long time.



A similar translation problem helped to create an international incident in 1993 when the Malaysian Prime Minister, Dr. Mahathir Mohamad, declined an invitation to attend the first Asia-Pacific Economic Cooperation (APEC) summit. Australian Prime Minister Paul Keating, when asked for a comment, replied: “APEC is bigger than all of us; Australia, the US and Malaysia and Dr Mahathir and any other recalcitrants.” Bilateral relations were severely strained, and both Malaysian government policies and Malaysian public opinion towards Australia were negatively affected for a long period of time. A significant factor in this reaction was the fact that the word \textit{recalcitrant} was translated in the Malaysian press by the Malay idiom \textit{keras kepala}, literally ‘hard headed’. The two expressions have a similar range of descriptive meaning (‘stubborn, obstinate, defiant of authority’), but the Malay idiom carries expressive meaning which makes the sense of insult and disrespect much stronger than in the English original. \textit{Keras kepala} would be appropriate in scolding a child or subordinate, but not in referring to a head-of-government.


\subsection{Case study: Expressive uses of diminutives}\label{sec:} %6.6 /

Diminutives are grammatical markers whose primary or literal meaning is to indicate small size; but diminutives often have secondary uses as well, and often these involve expressive content. Anna \citet{Wierzbicka1985} describes one common use of diminutives in Polish as follows:


\begin{quote}
In Polish, warm hospitality is expressed as much by the use of diminutives as it is by the ‘hectoring’ style of offers and suggestions. Characteristically, the food items offered to the guest are often referred to by the host by their diminutive names. Thus… one might say in Polish: \textit{Wei jeszcze Sledzika! Koniecznie!} ‘Take some more dear-little-herring (\textsc{dim}). You must!’ The diminutive praises the quality of the food and minimizes the quantity pushed onto the guest’s plate. The speaker insinuates: “Don’t resist! it is a small thing I’m asking you to do — and a good thing!”. The target of the praise is in fact vague: the praise seems to embrace the food, the guest, and the action of the guest desired by the host. The diminutive and the imperative work hand in hand in the cordial, solicitous attempt to get the guest to eat more.
\end{quote}


Markers of expressive meaning often have several possible meanings, which depend heavily on context, and this is true for the Spanish diminutive suffixes as illustrated in \REF{ex:2.21}. Notice that the same diminutive suffix can have nearly opposite meanings (deprecation vs. appreciation; exactness vs. approximation; attenuation vs. intensification) in different contexts (and, in some cases, different dialects). These examples also illustrate the “scalability” of expressive meaning, the fact that it can be intensified through repetition, as in \textit{chiqu-it-it-o}.


\ea \label{ex:2.21}
The expressive uses of Spanish diminutive suffixes \citep{Fortin2011}

\ea Deprecation

\textit{mujer-zuela}  woman\textsc{-dim}  ‘disreputable woman’ + disdain/mockery

 \ex Appreciation

\textit{niñ-ito}  boy-\textsc{dim}  ‘boy’ + endearment/affection

\ex  Hypocorism (nick-name, pet name)

\textit{Carol-ita}  Carol-\textsc{dim}  ‘Carol’ + endearment

\ex Exactness

\textit{igual-ito}  the.same-\textsc{dim}  ‘exactly the same’

\ex Approximation

\textit{floj-illo}  lazy-\textsc{dim}  ‘kind of lazy, lazy-ish’

\ex Attenuation

\textit{ahor-ita}  now-\textsc{dim}  ‘soon, in a little while’ (Caribbean Spanish)

\ex Intensification

\textit{ahor-ita}  now-\textsc{dim}  ‘immediately, right now’ (Latin American Spanish)

\textit{chiqu-it-o}  small-\textsc{dim-masc}  ‘very small’\\
\textit{chiqu-it-it-o}  small-\textsc{dim-dim-masc}  ‘very, very small/teeny-weeny’\\
\textit{chiqu-it-it-…-it-o}  small-\textsc{dim}-\textsc{dim}-\textsc{…}-\textsc{dim}-\textsc{masc} ‘very, very, …, very, small’
\z
\z
\section{Conclusion}\label{sec:} %7. /

In this chapter we started with the observation that speakers use language to talk about the world, for example by referring to things or describing states of affairs. We introduced the distinction between sense and denotation, which is of fundamental importance in all that follows. Knowing the sense of a word is what makes it possible for speakers of a language to identify the denotation of that word in a particular context of use. In a similar way, as we discuss in \chapref{sec:3}, knowing the sense of a sentence is what makes it possible for speakers of a language to judge whether or not that sentence is true in a particular context of use. The issue of ambiguity (a single word, phrase, or sentence with more than one sense) is one that we will return to often in the chapters that follow. Finally, we demonstrated a number of ways in which this kind of descriptive meaning (talking about the world) is different from expressive meaning (expressing the speaker’s emotions or attitudes). In the rest of this book, we will focus primarily on descriptive meaning rather than expressive meaning; but it is important to remember that both “dimensions” of meaning are involved in many (if not most) utterances.



\furtherreading



Birner (2013, \chapref{sec:4}) provides a helpful overview of reference and various related issues.



Abbott (2010, \chapref{sec:2}) provides a good summary of early work by Frege and other philosophers on the distinction between sense and denotation; later chapters provide in-depth discussions of various types of referring expressions. 



For additional discussion of expressive meaning see \citet{Cruse1986,Cruse2000}, \citet{Potts2007b}, and \citet{Kratzer1999}.


\subsubsection{Discussion exercises:}\label{sec:}
\paragraph{A: Sense vs. denotation}

Which of the following pairs of expressions have the same sense? the same denotation? Explain your answer.

\begin{tabularx}{\textwidth}{XXX}
\lsptoprule
a & \textit{cordates} (= ‘animals with hearts’) & \textit{renates} (= ‘animals with kidneys’)\\               
b. & \itshape animals with gills and scales & \itshape fish\\
c. & \itshape your first-born son & \itshape your oldest male offspring\\
d. & \itshape Ronald Reagan & \itshape the Governor of California\\
e. & \itshape my oldest sister & \itshape your Aunt Betty\\
f. & \itshape my pupils & \itshape the students that I teach\\
g. & \textit{the man who invented the phonograph} & \textit{the man who invented the light-bulb}\\
\lspbottomrule
\end{tabularx}

% \ea
\noindent Model answer for (a):\\
\begin{quote} In our world at the present time, all species that have hearts also have kidneys; so these two words have the same denotation in our world at the present time. They do not have the same sense, however, because we can imagine a world in which some species had hearts without kidneys, or kidneys without hearts; so the two words do not have the same denotation in all possible situations. \end{quote}
% \z

\textbf{B: Referring expressions}

% \ea
Which of the following NPs are being used to \textit{refer} to something?
% \z

\begin{enumerate}[label=\alph*.]
\item I never promised you \textit{a rose garden}.
\item St. Benedict, the father of Western monasticism, planted \textit{a rose garden} at his early monastery in Subiaco near Rome\footnote{[http://www.scu.edu/stclaregarden/ethno/medievalgardens.cfm]}
\item My sister wants to marry \textit{a policeman}.
\item My sister married \textit{a policeman}.
\item Leibniz searched for \textit{the solution to the equation}.
\item Leibniz discovered \textit{the solution to the equation}.
\item \textit{No cat} likes being bathed.
\item \textit{All musicians} are temperamental.
\end{enumerate}

\subsubsection{Homework exercises:}\label{sec:}
\paragraph{A: Idiomatic meaning}

Try to find one phrasal idiom (an idiom consisting of two or more words) in a language other than English; give a word-for-word translation and explain its idiomatic meaning.

\paragraph{B: Expressive meaning}

Try to find a word in a language other than English which has purely expressive meaning, like \textit{oops} and \textit{ouch}; and explain how it is used.

\textbf{C: Referring expressions}

For each of the following sentences, state whether or not the underlined nominal expression is being used to refer.

\begin{enumerate}[label=\alph*.]
\item Abraham Lincoln was very close to \textbf{his step-mother}.\\{}
  {}[\textbf{model answer}: the phrase \textit{his step-mother} is used to refer to a specific person,\\
  namely \textstylest{Sarah Bush} Lincoln, so it does refer]
\item  I’m so hungry I could eat \textbf{a horse}.
\item  \textstylest{Senate Majority Leader Curt Bramble, R-Provo, was back in the hospital this weekend after getting} kicked by \textbf{a horse}. [Provo, UT \textit{Daily Herald} Jan. 29, 2007]
\item  Police searched the house for 6 hours but found \textbf{no drugs}.
\item  Edward hopes that his on-line match-making service will help him find   \textbf{the girl of his dreams}.
\item  Susan married \textbf{the first man who proposed to her}.
\item  \textbf{Every city} has pollution problems.
\end{enumerate}


\chapter{{3}: Truth and inference}

\section{Truth as a guide to sentence meaning}\label{sec:} %1. /

Any speaker of English will “understand” the simple sentence in \REF{ex:3.1}, i.e., will know what it “means”. But what kind of knowledge does this involve? Can our hypothetical speaker tell us, for example, whether the sentence is true?


\ea \label{ex:3.1}
\textit{King Henry VIII snores}.
\z


It turns out that a sentence by itself is neither true nor false: its truth value can only be determined relative to a specific situation (or state of affairs, or universe of discourse). In the real world at the time that I am writing this chapter (early in the 21\textsuperscript{st} century), the sentence is clearly false, because Henry VIII died in 1547 AD. The sentence may well have been true in, say, 1525 AD; but most speakers of English probably do not know whether or not it was in fact true, because we do not have total knowledge of the state of the world at that time.



So knowing the meaning of a sentence does not necessarily mean that we know whether or not it is true in a particular situation; but it does mean that we know the kinds of situations in which the sentence would be true. Sentence \REF{ex:3.1} will be true in any universe of discourse in which the individual named \textit{King Henry VIII} has the property of snoring. We will adopt the common view of sentence meanings expressed in \REF{ex:3.2}:


\ea \label{ex:3.2}
“To know the meaning of a [declarative] sentence is to know what the world would have to be like for the sentence to be true.”  [Dowty, \citealt{WallPeters1981}:4]
\z


The meaning of a simple declarative sentence is called a \textsc{proposition}. A proposition is a claim about the world which may (in general) be true in some situations and false in others. Some scholars hold that a sentence, as a grammatical entity, cannot have a truth value. Speakers speak truly when they use a sentence to perform a certain type of speech act, namely a statement (making a claim about the world), provided that the meaning (i.e., the sense) of the sentence corresponds to the situation about which the claim is being made. Under this view, when we speak of sentences as being true or false we are using a common but imprecise manner of speaking. It is the proposition expressed by the sentence, rather than the sentence itself, which can be true or false.



In \sectref{sec:2} we will look at various types of propositions: some which must always be true, some which can never be true, and some (the “normal” case) which may be either true or false depending on the situation. In \sectref{sec:3} we examine some important truth relations that can exist between pairs of propositions, of which perhaps the most important is the \textsc{entailment} relation. Entailment is a type of inference. We say that proposition \textit{p} “entails” proposition \textit{q} if \textit{p} being true makes it certain that \textit{q} is true as well. Finally, in \sectref{sec:4}, we introduce another type of inference known as a \textsc{presupposition}. Presupposition is a complex and controversial topic, but one which will be important in later chapters.


\section{Analytic sentences, synthetic sentences, and contradictions}\label{sec:} %2. /

We have said that knowing the meaning of a sentence allows us to determine the kinds of situations in which the proposition which it expresses would be true. In other words, the meaning of a sentence determines its \textsc{truth conditions}. Some propositions have the interesting property of being true under all circumstances; there are no situations in which such a proposition would be false. We refer to sentences which express such propositions as \textsc{analytic sentences}, or \textsc{tautologies}. Some examples are given in \REF{ex:3.3}:


\ea \label{ex:3.3}
\ea \textit{Today is the first day of the rest of your life}.\footnote{Attributed to Charles Dederich (1913–1997), founder of the Synanon drug rehabilitation program and religious movement.}\\
\ex \textit{Que sera sera}. ‘What will be, will be.’\\
\ex Is this bill all that I want? Far from it. Is it all that it can be? Far from it.\\
  \textit{But when history calls, history calls}.\footnote{Sen. Olympia Snowe explaining her vote in favor of the Baucus health care reform bill, Oct. 2009.}
                       \z
\z


Because analytic sentences are always true, they are not very informative. The speaker who commits himself to the truth of such a sentence is making no claim at all about the state of the world, because the truth of the sentence depends only on the meaning of the words. But in that case, why would anyone bother to say such a thing? It is important to note that the use of tautologies is not restricted to politicians and pop psychology gurus, who may have professional motivations to make risk-free statements which sound profound. In fact, all of us probably say such things more frequently than we realize. We say them because they do in fact have communicative value; but this value cannot come from the semantic (or truth conditional) content of the utterance. The communicative value of these utterances comes entirely from the pragmatic inferences which they trigger. We will talk in more detail in \chapref{sec:8} about how these pragmatic inferences arise.



The opposite situation is also possible, i.e. propositions which are false in every imaginable situation. An example is given in \REF{ex:3.4}. Propositions of this type are said to be \textsc{contradictions}. Once again, a speaker who utters a sentence of this type is not making a truth conditional claim about the state of the world, since there are no conditions under which the sentence can be true. The communicative value of the utterance must be derived by pragmatic inference.


\ea \label{ex:3.4}
And a woman who held a babe against her bosom said, “Speak to us of children.”And he said: “\textbf{\textit{Your children are not your children}}. They are the sons and daughters of Life’s longing for itself…”\footnote{From “On Children”, in \textit{The Prophet} (Kahlil Gibran, 1923).}
\z


Propositions which are neither contradictions nor analytic are said to be \textsc{synthetic}. These propositions may be true in some situations and false in others, so determining their truth value requires not only understanding their meaning but also knowing something about the current state of the world or the situation under discussion. Most of the (declarative) sentences that speakers produce in everyday speech are of this type.



We would expect an adequate analysis of sentence meanings to provide an explanation for why certain sentences are analytic, and why certain others are contradictions. So one criterion for evaluating the relative merits of a possible semantic analysis is to ask how successful it is in this regard.


\section{Meaning relations between propositions}\label{sec:} %3. /

Consider the pair of sentences in \REF{ex:3.5}. The meanings of these two sentences are related in an important way. Specifically, in any situation for which (\ref{ex:3.5}a) is true, (\ref{ex:3.5}b) must be true as well; and in any situation for which (\ref{ex:3.5}b) is false, (\ref{ex:3.5}a) must also be false. Moreover, this relationship follows directly from the meanings of the two sentences, and does not depend on the situation or context in which they are used.


\ea \label{ex:3.5}
\ea Edward VIII has abdicated the throne in order to marry Wallis Simpson.\\
\ex Edward VIII is no longer the King.
                       \z
\z


This kind of relationship is known as \textsc{entailment}; sentence (\ref{ex:3.5}a) \textsc{entails} sentence (\ref{ex:3.5}b), or more precisely, the proposition expressed by (\ref{ex:3.5}a) entails the proposition expressed by (\ref{ex:3.5}b). The defining properties of entailment are those mentioned in the previous paragraph. We can say that proposition p entails proposition q just in case the following three things are true:\footnote{\citet[29]{Cruse2000}.}


\begin{enumerate}[label=(\alph*)]
 \item whenever p is true, it is logically necessary that q must also be true;
 \item whenever q is false, it is logically necessary that p must also be false;
 \item these relations follow directly from the meanings of p and q, and do not depend on the context of the utterance.
\end{enumerate}

This definition gives us some ways to test for entailments. Intuitively it seems clear that the proposition expressed by (\ref{ex:3.6}a) entails the proposition expressed by (\ref{ex:3.6}b). We can confirm this intuition by observing that asserting (\ref{ex:3.6}a) while denying (\ref{ex:3.6}b) leads to a contradiction (\ref{ex:3.6}c). Similarly, it would be highly unnatural to assert (\ref{ex:3.6}a) while expressing doubt about (\ref{ex:3.6}b), as illustrated in (\ref{ex:3.6}d). It would be unnaturally redundant to assert (\ref{ex:3.6}a) and then state (\ref{ex:3.6}b) as a separate assertion; this is illustrated in (\ref{ex:3.6}e).


\ea \label{ex:3.6}
\ea I broke your Ming dynasty jar.\\
\ex Your Ming dynasty jar broke.\\
\ex \#I broke your Ming dynasty jar, but the jar didn’t break.\\
\ex \#I broke your Ming dynasty jar, but I’m not sure whether the jar broke.\\
\ex \#I broke your Ming dynasty jar, and the jar broke.
                       \z
\z


Now consider the pair of sentences in \REF{ex:3.7}. Intuitively it seems that (\ref{ex:3.7}a) entails (\ref{ex:3.7}b); whenever (\ref{ex:3.7}a) is true, (\ref{ex:3.7}b) must also be true, and whenever (\ref{ex:3.7}b) is false, (\ref{ex:3.7}a) must also be false. But notice that (\ref{ex:3.7}b) also entails (\ref{ex:3.7}a). The propositions expressed by these two sentences mutually entail each other, as demonstrated in (\ref{ex:3.7}c--d). Two sentences which mutually entail each other are said to be \textsc{synonymous}, or \textsc{paraphrases} of each other. This means that the propositions expressed by the two sentences have the same truth conditions, and therefore must have the same truth value (either both true or both false) in any imaginable situation.


\ea \label{ex:3.7}
\ea Hong Kong is warmer than Beijing (in December).\\
\ex Beijing is cooler than Hong Kong (in December).\\
\ex \#Hong Kong is warmer than Beijing, but Beijing is not cooler than Hong Kong.\\
\ex \#Beijing is cooler than Hong Kong, but Hong Kong is not warmer than Beijing.
                       \z
\z


A pair of propositions which cannot both be true are said to be \textsc{inconsistent} or \textsc{incompatible}. Two distinct types of incompatibility have traditionally been recognized. Propositions which must have opposite truth values in every circumstance are said to be \textsc{contradictory}. For example, any proposition \textit{p} must have the opposite truth value from its negation (\textit{not p}) in all circumstances. Thus the pair of sentences in \REF{ex:3.8} are contradictory; whenever the first is true, the second must be false, and vice versa.


\ea \label{ex:3.8}
\ea Ringo Starr is my grandfather.\\
\ex Ringo Starr is not my grandfather.
                       \z
\z


On the other hand, it is possible for two propositions to be inconsistent without being contradictory. This would mean that they cannot both be true, but they could both be false in a particular context. We refer to such pairs as \textsc{contrary} propositions. An example is provided in (\ref{ex:3.9}a--b). These two sentences cannot both be true, so (\ref{ex:3.9}c) is contradictory. However, they could both be false in a given situation, so (\ref{ex:3.9}d) is not contradictory.


\ea \label{ex:3.9}
\ea Al is taller than Bill.\\
\ex Bill is taller than Al.\\
\ex \#Al is taller than Bill and Bill is taller than Al.\\
\ex Al is no taller than Bill and Bill is no taller than Al.
                       \z
\z


Finally, two sentences are said to be \textsc{independent} when they are neither incompatible nor synonymous, and when neither of them entails the other. If two sentences are independent, there is no truth value dependency between the two propositions; knowing the truth value of one will not provide enough information to know the truth value of the other.



These meaning relations (incompatibility, synonymy, and entailment) provide additional benchmarks for evaluating a possible semantic analysis: how successful is it in predicting or explaining which pairs of sentences will be synonymous, which pairs will be incompatible, etc.?


\section{Presupposition}\label{sec:} %4. /

In the previous section we discussed how the meaning of one sentence can entail the meaning of another sentence. Entailment is a very strong kind of inference. If we are sure that \textit{p} is true, and we know that \textit{p} entails \textit{q}, then we can be equally sure that \textit{q} is true. In this section we examine another kind of inference, that is, another type of meaning relation in which the utterance of one sentence seems to imply the truth of some other sentence. This type of inference, which is known as a \textsc{presupposition}, is extremely common in daily speech; it has been intensively studied but remains controversial and somewhat mysterious.



As a first approximation, let us define presupposition as information which is linguistically encoded as being part of the common ground at the time of utterance. The term \textsc{common ground} refers to everything that both the speaker and hearer know or believe, and know that they have in common. This would include knowledge about the world, such as the fact that (in our world) there is only one sun and one moon; knowledge that is observable in the speech situation, such as what the speaker is wearing or carrying; or facts that have been mentioned earlier in that same conversation (or discourse).



Speakers can choose to indicate, by the use of certain words or grammatical constructions, that a certain piece of information is part of the common ground. Consider the following example:


\ea
“Take some more tea,” the March Hare said to Alice, very earnestly.\\
“I’ve had nothing yet,” Alice replied in an offended tone, “so I can’t take more.”\\{}
{}[Lewis Carroll, \textit{Alice’s Adventures in Wonderland}, \chapref{sec:7} — “A Mad Tea-Party”]
\z


By using the word \textit{more} (in the sense which seems most likely in this context, i.e. as a synonym for \textit{additional}) the March Hare implies that Alice has already had some tea, and that this knowledge is part of their common ground at that point in the conversation. The word or grammatical construction which indicates the presence of a presupposition is called a \textsc{trigger}; so in this case we can say that \textit{more} “triggers” the presupposition that she has already had some tea. However, in this example the “presupposed” material is not in fact part of the common ground, because Alice has not yet had any tea. This is a case of \textsc{presupposition failure}, which we might define as an inappropriate use of a presupposition trigger to signal a presupposition which is not in fact part of the common ground at the time of utterance. Notice that Alice is offended — not only by the impoliteness of her hosts in not offering her tea in the first place, but also by the inappropriate use of the word \textit{more}.


\subsection{How to identify a presupposition}\label{sec:} %4.1 /

There is an important difference between entailment and presupposition with regard to how the nature of the speech act being performed affects the inference. If \textit{p} entails \textit{q}, then any speaker who states that \textit{p} is true (e.g. \textit{I broke your jar}) is committed to believing that \textit{q} (e.g. \textit{your jar broke}) is also true. However, a speaker who asks whether \textit{p} is true (\textit{Did I break your jar?}) or denies that \textit{p} is true (\textit{I didn’t break your jar}) makes no commitment concerning the truth value of \textit{q}. In contrast, if \textit{p} presupposes \textit{q}, then the inference holds whether the speaker asserts, denies, or asks whether \textit{p} is true. Notice that all of the three sentences in \REF{ex:3.11} imply that the vice president has falsified his dental records. (This presupposition is triggered by the word \textit{regret}.)


\ea \label{ex:3.11}
\ea The vice president regrets that he falsified his dental records.\\
\ex The vice president doesn’t regret that he falsified his dental records.\\
\ex Does the vice president regret that he falsified his dental records?
                       \z
\z


In most cases, if a positive declarative sentence like (\ref{ex:3.12}a) triggers a certain presupposition, that presupposition will also be triggered by a “family” of related sentences (sentences based on the same propositional content) which includes negative assertions, questions, \textit{if}-clauses and certain modalities.\footnote{\citet{CherchiaMcConnell-Ginet1990}.} For example, (\ref{ex:3.12}a) presupposes that Susan has been dating an Albanian monk; this presupposition is triggered by the word \textit{stop}. All of the other sentences in \REF{ex:3.12} trigger this same presupposition, as predicted.


\ea \label{ex:3.12}
\ea Susan has stopped dating that Albanian monk.\\ 
\ex Susan has not stopped dating that Albanian monk.\\
\ex Has Susan stopped dating that Albanian monk?\\
\ex If Susan has stopped dating that Albanian monk, I might introduce her to my cousin.\\
\ex Susan may have stopped dating that Albanian monk.
                       \z
\z


In addition to the presupposition mentioned above, (\ref{ex:3.12}a) also entails that Susan is not currently dating the Albanian monk; but this entailment is not shared by any of the other sentences in \REF{ex:3.12}. This contrast shows us that presuppositions are preserved under negation, questioning, etc. while entailments are not.\footnote{A more technical way of expressing this is to say that presuppositions \textsc{project} through the operators illustrated in \REF{ex:3.12}, while entailments do not.}



The “family of sentences” test is one of the most commonly used methods for distinguishing entailments from presuppositions. To offer another example, the statement \textit{The neighbor’s dog killed my cat} presupposes that the speaker owned a cat, and entails that the cat is dead. If the statement is negated (\textit{The neighbor’s dog didn’t kill my cat}) or questioned (\textit{Did the neighbor’s dog kill my cat?}), the presupposition still holds but entailment does not.



Another test for identifying presuppositions is described by von \citet{FintelMatthewson2008}. They point out that if a presupposition is triggered which is not in fact part of the common ground, the hearer can appropriately object by saying something like, “Wait a minute, I didn’t know that!” This kind of challenge is not appropriate for information that is simply asserted, since speakers do not usually assert something which they believe that the hearer already knows:


\begin{quote}
A presupposition which is not in the common ground at the time of utterance can be challenged by ‘Hey, wait a minute!’ (or other similar responses). In contrast, an assertion which is not in the common ground cannot be challenged in this way. This is shown in []… The ‘Hey, wait a minute!’ test is the best way we know of to test for presuppositions in a fieldwork context.\\
   (\citealt{vonFintelMatthewson2008})
\end{quote}

\ea \label{ex:3.13}
A: The mathematician who proved Goldbach’s Conjecture is a woman.\\
B\textsubscript{1}: Hey, wait a minute. I had no idea that someone proved Goldbach’s Conjecture.\\
B\textsubscript{2}: \#Hey, wait a minute. I had no idea that that was a woman.
\z


A fairly large number of presupposition triggers have been identified in English; a partial listing is presented below. For many of these it seems that translation equivalents in a number of other languages may trigger similar presuppositions, but so far there has been relatively little detailed study of presuppositions in languages other than English.\footnote{Exceptions to this generalization include \citet{LevinsonAnnamalai1992}, \citet{Matthewson2006}, and \citet{TonhauserEtAl2013}.}


\begin{enumerate}[label=\alph*.]
\item Definite descriptions: the use of a definite singular noun phrase, such as Bertrand Russell’s famous example \textit{the King of France}, presupposes that there is a uniquely identifiable individual in the situation under discussion that fits that description. Similarly, the use of a possessive phrase (e.g. \textit{my cat}) presupposes the existence of the possessee (in this case, the existence of a cat belonging to the speaker).
\item Factive predicates (e.g. \textit{regret, aware, realize, know, be sorry that}) are predicates that presuppose the truth of their complement clauses, as illustrated in \REF{ex:3.11} above.\footnote{\citet{KiparskyKiparsky1970}.}
\item Implicative predicates: \textit{manage to} presupposes \textit{try; forget to} presupposes \textit{intend to}; etc.
\item Aspectual predicates: \textit{stop} and \textit{continue} both presuppose that the event under discussion has been going on for some time, as illustrated in \REF{ex:3.12} above; \textit{resume} presupposes that the event was going on but then stopped for some period of time; \textit{begin} presupposes that the event was not occurring before.
\item Temporal clauses (\ref{ex:3.14}a--b) and restrictive relative clauses (\ref{ex:3.14}c) presuppose the truth of their subordinate clauses, while counterfactuals (\ref{ex:3.14}d) presuppose that their antecedent (\textit{if}) clauses are false (see \chapref{sec:19}). Comparisons like (\ref{ex:3.14}e) presuppose that the relevant statement holds true for the object of comparison.
\end{enumerate}

\ea \label{ex:3.14}
\ea Before I moved to Texas, I had never attended a rodeo.\\
  (presupposes that the speaker moved to Texas)

  \ex While his wife was in the hospital, John worked a full 40 hour week.\\
  (presupposes that John’s wife was in the hospital)

  \ex “I’m looking for the man who killed my father.”   [Maddie Ross in the movie \textit{True Grit}]\\
  (presupposes that some man killed the speaker’s father)

  \ex If you had not written that letter, I would not have to fire you.\\
  (presupposes that the hearer did write that letter) 

  \ex Jimmy isn’t as unpredictably gauche as Billy.   (\citealt{Levinson1983}:183)\\
  (presupposes that Billy is unpredictably gauche)
                       \z
                       \z


The tests mentioned above seem to work for all of these types, but in other respects it seems that different kinds of presupposition have slightly different properties. This is one of the major challenges in analyzing presuppositions. We return in \chapref{sec:8} to the issue of how to distinguish between different kinds of inference.


\subsection{Accommodation: a repair strategy}\label{sec:} %4.2 /

Recall that we defined presuppositions as “information which is \textsc{linguistically encoded} as being part of the common ground at the time of utterance.” We crucially did not require that implied information actually \textsc{be} part of the common ground in order to count as a presupposition. We have already seen one outcome that may result from the use of presupposition triggers which do not accurately reflect the common ground at the time of utterance, namely presupposition failure (\ref{ex:3.10} above). Another example of presupposition failure is provided in \REF{ex:3.15}; all of Glinda’s questions in this fragment trigger presuppositions which Dorothy contests. (The fact that the false inferences are triggered by questions is a strong hint that they are presuppositions rather than entailments.)


\ea \label{ex:3.15}
\textbf{Glinda}: Are you a good witch or a bad witch?\\
\textbf{Dorothy}: Who, me?  I’m not a witch at all.  I’m Dorothy Gale, from Kansas.\\
\textbf{Glinda}: Well, is that the witch?\\
\textbf{Dorothy}: Who, Toto?  Toto’s my dog.\\
\textbf{Glinda}: Well, I’m a little muddled. The Munchkins called me because a new witch has just dropped a house on the Wicked Witch of the East. And there’s the house, and here you are and that’s all that’s left of the Wicked Witch of the East. What the Munchkins want to know is, are you a good witch or a bad witch?   [\textit{The Wizard of Oz} (1939 movie)]
\z


Glinda’s first question presupposes that one of the two specified alternatives (\textit{good witch} vs. \textit{bad witch}) is true of Dorothy, and both of these would entail that Dorothy is a witch. Dorothy rejects this presupposition quite vigorously. Glinda’s second question (\textit{Is that the witch?)}, and in particular her use of the definite article, presupposes that there is a uniquely identifiable witch in the context of the conversation.



However, presupposition failure is not the only possible outcome. Another possibility is that the hearer, confronted with a mismatch between a presupposition trigger and the current common ground, may choose to accept the presupposition as if it were part of the common ground; in effect, to add it to the common ground. This is most likely to happen if the presupposed information is uncontroversial and consistent with all information that is already part of the common ground; something that the hearer would immediately accept if the speaker asserted it. For example, suppose I notice that you have not slept well and you explain by saying \textit{My cat got stuck on the roof last night}; and suppose that I did not previously know you had a cat. Technically the presupposition triggered by the possessive phrase \textit{my cat} is not part of the common ground, but I am very unlikely to object or to consider your statement in any way inappropriate. Instead, I will add to my model of the common ground the fact that you own a cat. This process is called \textsc{accommodation}.



It is not uncommon for speakers to encode new information as a presupposition, expecting it to be accommodated by the hearer. For this reason, definitions which state that presuppositions “must be mutually known or assumed by the speaker and addressee for the utterance to be considered appropriate in context” are misleading.\footnote{See for example \url{http://en.wikipedia.org/wiki/Presupposition}.} This fact has long been recognized in discussions of presupposition, as the following quotes illustrate:


\begin{quote}
“I am asked by someone who I have just met, ‘Are you going to lunch?’ I reply, ‘No, I’ve got to pick up my sister.’ Here I seem to presuppose that I have a sister even though I do not assume that the speaker knows this.” \citep[202]{Stalnaker1974}. 
\end{quote}

\begin{quote}
“It is quite natural to say to somebody... ‘My aunt’s cousin went to that concert,’ when one knows perfectly well that the person one is talking to is very likely not even to know that one had an aunt, let alone know that one’s aunt had a cousin. So the supposition must be not that it is common knowledge but rather that it is non-controversial, in the sense that it is something that you would expect the hearer to take from you (if he does not already know).” \citep[190]{Grice1981}
\end{quote}

\subsection{Pragmatic vs. semantic aspects of presupposition}\label{sec:} %4.3 /

Thus far we have treated presupposition primarily as a pragmatic issue. We defined it in terms of the common ground between a specific speaker and hearer at a particular moment, a pragmatic concept since it depends heavily on the context of the utterance and the identity of the speech act participants. Presupposition failure, where accommodation is not possible, causes the utterance to be pragmatically inappropriate or \textsc{infelicitous}.\footnote{We will give a more precise explanation of the term \textsc{infelicitous} in \chapref{sec:10}, as part of our discussion of speech acts.} In contrast, we defined entailment in purely semantic terms: an entailment relation between two propositions must follow directly from the meanings of the propositions, and does not depend on the context of the utterance.



It turns out that presuppositions can have semantic effects as well. We have said that knowing the meaning (i.e. semantic content) of a sentence allows us to determine its truth value in any given situation. Now suppose a speaker utters (\ref{ex:3.16}a) in our modern world, where there is no King of France; or (\ref{ex:3.16}b) in a context where the individual John has no children; or (\ref{ex:3.16}c) in a context where John’s wife had not been in the hospital. Under those circumstances, the sentences would clearly not be true; but would we want to say that they are false? If they were false, then their denials should be true; but the negative statements in \REF{ex:e.17}, if read with normal intonation, would be just as “un-true” as their positive counterparts in the contexts we have just described.


\ea \label{ex:3.16}
\ea The present King of France is bald.  [adapted from \citet{Russell1905}]\\
\ex John’s children are very well-behaved.\\
\ex While his wife was in the hospital, John worked a full 40 hour week.
                       \z
\z

\ea \label{ex:3.17}
\ea The present King of France is not bald.\\
\ex John’s children are not very well-behaved.\\
\ex While his wife was in the hospital, John did not work a full 40 hour week.
                       \z
\z


We have already noted that the presupposition failure triggered by such statements makes them pragmatically inappropriate; but examples like (\ref{ex:3.16}--\ref{ex:3.17}) show that, at least in some cases, presupposition failure can also make it difficult to assign the sentence a truth value. Some of the earliest discussions of presuppositions defined them in purely semantic, truth-conditional terms:\footnote{e.g. \citet{Frege1892}; \citet{Strawson1950,Strawson1952}.} “One sentence \textsc{presupposes} another just in case the latter must be true in order that the former have a truth value at all.”\footnote{\citet[447]{Stalnaker1973}, summarizing the positions of Strawson and Frege. Stalnaker himself argued for a pragmatic analysis.}



Under this definition, presupposition failure results in a truth-value “gap”, or indeterminacy. But there are other cases where presupposition failure does not seem to have this effect. For example, if (\ref{ex:3.18}a) were spoken in a context where the vice president had not falsified his dental records, or (\ref{ex:3.18}b) in a context where Susan had never dated an Albanian monk, these sentences would be pragmatically inappropriate because of the presupposition failure. But it also seems reasonable to say they are false (the vice president can’t regret something he never did; Susan can’t stop doing something she never did), and that their negative counterparts in \REF{ex:3.19} have at least one reading (or sense) which is true. 


\ea \label{ex:3.18}
\ea The vice president regrets that he falsified his dental records.\\
\ex Susan has stopped dating that Albanian monk.
\z
                       \z

\ea \label{ex:3.19}
\ea The vice president doesn’t regret that he falsified his dental records.\\
\ex Susan has not stopped dating that Albanian monk.
                       \z
\z


However, there are various complications concerning the way negation gets interpreted in examples like \REF{ex:3.19}. For example, intonation can affect the interpretation of the sentence. We will return to this issue in \chapref{sec:8}.


\section{Conclusion}\label{sec:} %5. /

The principle that the meaning of a sentence determines its truth conditions (i.e., the kinds of situations in which the proposition it expresses would be true) is the foundation for most of what we talk about in this book, including word meanings. A proposition is judged to be true if it corresponds to the situation about which a claim is made.



A major goal of semantic analysis is to explain how a sentence gets its meaning, that is, why a given form has the particular meaning that it does. In this chapter we have mentioned a few benchmarks for success, things that we would expect an adequate analysis of sentence meanings to provide for us. These benchmarks include explaining why certain sentences are analytic (always true) or contradictions (never true); and predicting which pairs of sentences will be synonymous (always having the same truth value in every possible situation), incompatible (cannot both be true), etc.



In this chapter we have introduced two very important types of inference, entailment and presupposition, which we will refer to in many future chapters. Entailment is strictly a semantic relation, whereas presupposition has to do with pragmatic issues such as managing the common ground and appropriateness of use. However, we have suggested that presupposition failure can sometimes block the assignment of truth values as well.



\furtherreading



Good basic introductions to the study of logic are presented in Allwood et al. (1977, \chapref{sec:3}) and Gamut (1991a, ch. 1). The literature dealing with presupposition is enormous. Helpful overviews of the subject are presented in Levinson (1983, ch. 4), \citet{GeurtsBeaver2011}, Zimmermann \& Sternefeld (2013, ch. 9), and Birner (2013, \chapref{sec:5}). \citet{Potts2015} also provides a good summary, including a comparison of presuppositions with conventional implicatures (which we will discuss in chapters 8 and 11). Von Fintel \& Matthewson (2008, sec. 4.1) discuss cross-linguistic issues.


\subsubsection{Discussion exercises:}\label{sec:}
\paragraph{A: Classifying propositions}

State whether the propositions expressed by the following sentences are analytic, synthetic, or contradictions:

\ea
  a. My sister is a happily married bachelor.\\
\ex Even numbers are divisible by two.\\
\ex All dogs are brown.\\
\ex All dogs are animals.\\
\ex The earth revolves around the sun.\\
\ex The sun does not shine at night.\\
\ex CO\textsubscript{2} becomes a solid when it boils.
\z

\paragraph{B: Relationships between propositions}

Identify the relationship between the following pairs of propositions (\textsc{entailment, paraphrase, contrary,} \textsc{contradictory,} \textsc{independent}):

\begin{stylepoints}
(i)  a. John killed the wasp.\\
\ex The wasp died.
\end{stylepoints}

\begin{stylepoints}
(ii)  a. John killed the wasp.\\
\ex The wasp did not die.
\end{stylepoints}

\begin{stylepoints}
(iii)  a. The wasp is alive.\\
\ex The wasp is dead.
\end{stylepoints}

\begin{stylepoints}
(iv)  a. The wasp is no longer alive.\\
\ex The wasp is dead.
\end{stylepoints}

\begin{stylepoints}
(v)  a. That boy is my son.\\
\ex I am that boy’s parent.
\end{stylepoints}

\begin{stylepoints}
(vi)  a. Fido is a dog.\\
\ex Fido is a cat.
\end{stylepoints}

\begin{stylepoints}
(vii)  a. Fido is a dog.\\
\ex Fido has four legs.
\end{stylepoints}

\paragraph{C: Presuppositions}

Identify the presuppositions and presupposition triggers in the following examples:
\ea
\ea
 John’s children are very well-behaved.\\
\ex Susan has become a vegan.\\
\ex Bill forgot to call his uncle.\\
\ex After he won the lottery, John had to get an unlisted phone number.\\
\ex George is sorry that he broke your Ming dynasty jar.
\z
\z

\paragraph{D: Presuppositions vs. entailments}

Show how you could use the negation and/or question tests to decide whether the (a) sentence \textsc{entails} or \textsc{presupposes} the (b) sentence. Evaluate the two sentences if spoken by the same speaker at the same time and place. [adapted from Saeed (2009: 114, ex. 4.8)]

\begin{stylepoints}
\ea%1
    \label{ex:key:1}

          a. \textit{Dave knows that Jim crashed the car}.\\
\ex \textit{Jim crashed the car}.\\
{}[\textbf{model answer}: The statement \textit{Dave knows that Jim crashed the car}, its negation \textit{Dave doesn’t know that Jim crashed the car}, and the corresponding question \textit{Does} \textit{Dave know that Jim crashed the car?} all lead the hearer to infer that Jim crashed the car. This suggests that the inference is a presupposition.]
\z
\end{stylepoints}

\begin{stylepoints}
\ea%2
    \label{ex:key:2}



          a. \textit{Zaire is bigger than Alaska}.\\
\ex \textit{Alaska is smaller than Zaire}.
\z
\end{stylepoints}

\begin{stylepoints}
\ea%3
    \label{ex:key:3}

          a. \textit{The minister blames her secretary for leaking the memo to the press}.\\
\ex \textit{The memo was leaked to the press}.
\z
\end{stylepoints}

\begin{stylepoints}
\ea%4
    \label{ex:key:4}
  
          a. \textit{Everyone passed the examination}.\\
\ex \textit{No one failed the examination}.
\z
\end{stylepoints}

\begin{stylepoints}
\ea%5
    \label{ex:key:5}
\textit{Mr. Singleton has resumed his habit of drinking stout}.\\
\ex \textit{Mr. Singleton had a habit of drinking stout}.
    \z
\end{stylepoints}

\subsubsection{Homework exercises:}\label{sec:}
\paragraph{A: Classifying propositions}

Classify the following sentences as analytic, synthetic, or contradictions.

\ea
\ea \textit{If it rains, we’ll get wet.}\\
  {}[\textbf{model answer}: synthetic, since we can imagine some contexts in which the sentence\\
  will be true, and other contexts in which it will be false (e.g., if I carry an umbrella).]

\ex \textit{If that snake is not dead then it is alive}.

\ex \textit{Shanghai is the capital of China}.

\ex \textit{My brother is an only child}.

\ex \textit{Abraham Lincoln was the 16\textsuperscript{th}} \textit{president of the United States}.
                       \z
\z
\paragraph{B: Relationships between propositions}

Identify the relationship between the following pairs of propositions (\textsc{entailment, paraphrase, contrary,} \textsc{contradictory,} \textsc{independent}):

\begin{stylepoints}
(i)  a. \textit{Michael is my advisor}.  b. \textit{I am Michael’s advisee}.
\end{stylepoints}

\begin{stylepoints}
(ii)  a. \textit{Stewball was a race horse}.  b. \textit{Stewball was a mammal}.
\end{stylepoints}

\begin{stylepoints}
(iii)  a. \textit{Elvis died of} \textstylest{\textit{cardiac arrhythmia}}.  b. \textit{Elvis is alive}.
\end{stylepoints}

\paragraph{C: Identifying entailments}

For each pair of sentences, decide whether the (\ref{ex:}a) sentence \textsc{entails} the (\ref{ex:}b) sentence. The two sentences should be evaluated as if spoken by the same speaker at the same time and place; so, for example, repeated names and definite NPs refer to the same individuals.

\begin{stylepoints}
\ea%1
    \label{ex:key:1}




          a. \textit{Olivia passed her driving test}.\\
\ex \textit{Olivia didn’t fail her driving test}.\\
{}[\textbf{model answer}: if a is true, b must be true; if b is false, a must be false; this follows from the meanings of the sentences, and does not depend on context. So a entails b.]
    \z
\end{stylepoints}

\begin{stylepoints}
\ea%2
    \label{ex:key:2}




          a. \textit{Fido is a dog}.\\
\ex \textit{Fido has four legs}.
    \z
\end{stylepoints}

\begin{stylepoints}
\ea%3
    \label{ex:key:3}




          a. \textit{That boy is my son}.\\
\ex \textit{I am that boy’s parent}.
    \z
\end{stylepoints}

\begin{stylepoints}
\ea%4
    \label{ex:key:4}




          a. \textit{Not all of our students will graduate}.\\
\ex \textit{Some of our students will graduate}.
    \z
\end{stylepoints}

\paragraph{D: Presuppositions vs. entailments}

Show how you could use the negation test to decide whether the (\ref{ex:}a) sentence \textsc{entails} or \textsc{presupposes} the (\ref{ex:}b) sentence. Again, evaluate the two sentences as being spoken by the same speaker at the same time and place.

\begin{stylepoints}
\ea%1
    \label{ex:key:1}




          a. \textit{The boss realized that Jim was lying}.\\
\ex \textit{Jim was lying}.\\
{}[\textbf{model answer}: Both \textit{The boss realized that Jim was lying} and \textit{The boss didn’t realize that Jim was lying} lead the hearer to infer that Jim was lying. This suggests that the inference is a presupposition.]
    \z
\end{stylepoints}

\begin{stylepoints}
\ea%2
    \label{ex:key:2}




          a. \textit{Singapore is south of Kuala Lumpur}.\\
\ex \textit{Kuala Lumpur is north of Singapore}.
    \z
\end{stylepoints}

\begin{stylepoints}
\ea%3
    \label{ex:key:3}




          a. \textit{I am sorry that Arthur was fired}.\\
\ex \textit{Arthur was fired}.
    \z
\end{stylepoints}

\begin{stylepoints}
\ea%4
    \label{ex:key:4}




          a. \textit{Nobody is perfect}.\\
\ex \textit{Everybody is imperfect}.
    \z
\end{stylepoints}

\begin{stylepoints}
\ea%5
    \label{ex:key:5}




          a. \textit{Lief Ericson returned to Greenland}.\\
\ex \textit{Lief Ericson had previously visited Greenland}.
    \z
\end{stylepoints}

\chapter{{4}: The logic of truth}

\begin{quotation}LOGIC, n.  The art of thinking and reasoning in strict accordance with the limitations and incapacities of the human misunderstanding. The basic of logic is the syllogism, consisting of a major and a minor premise and a conclusion — thus:

\begin{quote}
Major Premise: Sixty men can do a piece of work sixty times as quickly as one man.

Minor Premise: One man can dig a posthole in sixty seconds; therefore —

Conclusion: Sixty men can dig a posthole in one second.
\end{quote}

This may be called the syllogism arithmetical, in which, by combining logic and mathematics, we obtain a double certainty and are twice blessed.\\
{}[entry from \textit{The Devil’s Dictionary} by Ambrose \citet{Bierce1911}]
\end{quotation}

\section{What logic can do for you}\label{sec:} %1. /

In \chapref{sec:1} we mentioned that semanticists often use formal logic as a metalanguage for representing the meanings of sentences and other expressions in human languages. For the most part, this book emphasizes prose description more than formalization; we will use the logical notation a fair bit in Unit 4 but only sporadically in other sections of the book. Nevertheless, it will be helpful for you to become familiar with this notation, not only for the purposes of this book but also to help you read other books and articles about semantics.



In this chapter we will introduce some of the basic symbols and rules of inference for standard logic. Before we begin, it will probably be helpful to address a question which many readers may already be asking themselves, and which others are likely to ask before we get too far into the discussion: why are we doing this? How does translating English (or Samoan or Marathi) sentences into logical formulae help us to understand their meaning?



Representing the complexities of natural language using formal logic is no trivial task, but here are some of the reasons why many scholars have found the effort required in adopting this approach worthwhile. First, every human language is characterized by ambiguity, vagueness, figures of speech, etc. These features can actually be an advantage for communicative purposes, but they make it difficult to provide precise and unambiguous descriptions of word and sentence meanings in English (or Samoan or Marathi). Using formal logic as a metalanguage avoids most of these problems.



Second, we stated in \chapref{sec:3} that one way of measuring the success or adequacy of a semantic analysis is to see whether it can explain or predict various meaning relations between sentences, such as entailment, paraphrase, or incompatibility. Logic is the science of inference. If the meanings of two sentences can be stated as logical formulae, logic provides very precise rules and methods for determining whether one follows as a logical consequence of the other (entailment), whether each follows as a logical consequence of the other (paraphrase), or whether the two are logically inconsistent, i.e. they cannot both be true (incompatibility).



Third, it is often useful to test a hypothesis about the meaning of a sentence by expressing it in logical form, and then using the rules of logical inference to see what the implications would be. For example, suppose our analysis predicts that a certain sentence should mean \textit{p}, and suppose we can show that if a person believes \textit{p}, he is logically committed to believing \textit{q}. Now suppose that native speakers of the language feel that there would be no inconsistency in asserting the sentence in question but denying \textit{q}. This mismatch between logical inference and speaker intuition may give us reason to think that \textit{p} is not the correct meaning of the sentence after all. We will see examples of this kind of reasoning in future chapters.



Fourth, formal logic has proven to be a very powerful tool for modeling compositionality, i.e., for explaining how the meanings of sentences can be predicted from the meanings of the words they contain and the syntactic structure used to combine those words. As we noted in \chapref{sec:1}, this is one of the fundamental goals of semantic analysis. We will get a glimpse of how this can be done in Unit 4.



Finally, formal logic is a recursive system. This means that a relatively small number of symbols and rules can be used to form an unlimited number of different formulae. Any adequate metalanguage for describing the meanings of sentences in a human language must have this property, because (as we noted in \chapref{sec:1}) there is in principle no limit to the number of distinct meaningful sentences that can be produced in any human language.



To illustrate the recursive nature of the system, let us introduce the logical negation operator \textit{¬} ‘not’. The negation operator combines with a single proposition to form a new proposition. So, for example, if we let \textit{p} represent the proposition ‘It is raining,’ then \textit{¬p} (read ‘not p’) would represent the proposition ‘It is not raining.’ This proposition in turn can again combine with the negation operator to form a new proposition \textit{¬(¬p)} ‘It is not the case that it is not raining.’ There is in principle no limit to the number of formulae that can be produced in this way, though in practice sheer boredom would probably be a limiting factor.



We begin in \sectref{sec:2} with a brief discussion of \textsc{inference} and some of the ways in which logic can help us distinguish valid from invalid patterns of inference. \sectref{sec:key:3} deals with \textsc{propositional logic}, which specifies ways of combining simple propositions to form complex propositions. An important fact about this part of the logical system is that the inferences of propositional logic depend only on the truth values of the propositions involved, and not on their meanings. \sectref{sec:key:4} deals with \textsc{predicate logic}, which provides a way to take into account the meanings of individual content words and to state inferences which arise due to the meanings of quantifier words such as \textit{all}, \textit{some}, \textit{none}, etc.


\section{Valid patterns of inference}\label{sec:} %2. /

If someone says to us, \textit{Either Joe is crazy or he is lying, and he is not crazy}, and we believe the speaker to be truthful and well-informed, we will naturally conclude that Joe is lying. This is an example of \textsc{inference}: knowing that one fact or set of facts is true gives us an adequate basis for concluding that some other fact is also true.



Logic is the science of inference. One important goal of logic is to provide a systematic account for the kinds of reasoning or inference that we intuitively know to be correct, like the example mentioned in the previous paragraph. In thinking about such examples it is helpful to lay out each of the \textsc{premises} (the facts which form the basis for the inference) and the \textsc{conclusion} (the fact which is inferred) as shown in \REF{ex:}. For longer and more complex chains of inference, the same format can be used to lay out each step in the reasoning and thereby provide a \textsc{proof} that the conclusion is true.


\ea \label{ex:4.1}
Premise 1: \textit{Either Joe is crazy or he is lying.}\\
Premise 2: \textit{Joe is not crazy}.\\
\FelixHRule
Conclusion: \textit{Therefore,} \textit{Joe is lying.}
\z


As we will see, the kind of inference illustrated in \REF{ex:4.1} does not depend on the meanings of the “content words” (nouns, verbs, adjectives, etc.) but only on the meaning of the logical words, in this case \textit{or} and \textit{not}. Propositional logic, the topic of \sectref{sec:3}, deals with patterns of this type. Some other kinds of reasoning that we intuitively recognize as being correct are illustrated in \REF{ex:4.2}:


\ea \label{ex:4.2}
\ea  Premise 1: \textit{All men are mortal.}\\
Premise 2: \textit{Socrates is a man}.\\
\FelixHRule
Conclusion: \textit{Therefore,} \textit{Socrates is mortal.}
\ex Premise 1: \textit{Arthur is a lawyer.}\\
Premise 2: \textit{Arthur is honest}.\\
\FelixHRule
Conclusion: \textit{Therefore,} \textit{some (= at least one) lawyer is honest.}
                       \z
\z


The kinds of inference illustrated in \REF{ex:4.2} are clearly valid, and have been studied and discussed for over 2000 years. But these patterns cannot be explained using propositional logic alone. Once again, these inferences do not depend on the meanings of the “content words” (\textit{mortal}, \textit{lawyer}, \textit{honest}, etc.). In these examples the inferences follow from the meaning of the \textsc{quantifiers} \textit{all} and \textit{some}. Predicate logic, the topic of \sectref{sec:4}, provides a way of dealing with such cases.



Now consider the inference in \REF{ex:4.3}:


\ea \label{ex:4.3}
Premise: \textit{John killed the wasp.\\
}\FelixHRule
Conclusion: \textit{Therefore,} \textit{the wasp died.}
\z


This inference is not determined by the meanings of logical words or quantifiers, but only by the meanings of the verbs \textit{kill} and \textit{die}. Neither propositional logic nor predicate logic actually addresses this kind of inference. Logic deals with general patterns or forms of reasoning, rather that the meanings of individual words. However, predicate logic provides a notation for representing the meanings of the content words within each proposition, and thus gives us a way of expressing lexical entailments (e.g., \textit{kill} entails \textit{die}; see \chapref{sec:6}).



It is important to remember that a valid form of inference does not (by itself) guarantee a true conclusion. For example, the inferences in \REF{ex:4.4} both make use of a valid pattern discussed in \sectref{sec:3}.2, which is called \textsc{Modus Tollens} ‘method of rejecting/denying’:


\ea \label{ex:4.4}
\ea  Premise 1: \textit{If dolphins are fish, they are cold-blooded.}\\
Premise 2: \textit{Dolphins are not cold-blooded}.\\
\FelixHRule
Conclusion: \textit{Dolphins are not fish}.
\ex Premise 1: \textit{If salmon are fish, they are cold-blooded.}\\
Premise 2: \textit{Salmon are not cold-blooded}.\\
\FelixHRule
Conclusion: \textit{Salmon are not fish}.
                       \z
\z


Even though both of these examples employ the same logic, the results are different: (\ref{ex:4.4}a) leads to a true conclusion while (\ref{ex:4.4}b) leads to a false conclusion. Obviously this difference is closely related to the premises which are used in each case: (\ref{ex:4.4}b) starts from a false premise, namely \textit{Salmon are not cold-blooded}. Valid reasoning guarantees a true conclusion if the premises are true, but if one or more of the premises is false there is no guarantee.



Example (\ref{ex:4.4}b) shows that a false conclusion does not necessarily mean that the reasoning is invalid. Conversely, a true conclusion does not necessarily mean that the reasoning is valid. The examples in \REF{ex:4.5} both make use of an invalid form of reasoning called ‘denying the antecedent.’ This is in fact a common \textsc{fallacy}, i.e., an invalid pattern of inference which people nevertheless often try to use to support an argument. Now, the conclusion in (\ref{ex:4.5}a) is true, but the truth of this statement (\textit{Crocodiles are not warm-blooded}) does not show that the reasoning is valid. It is simply a coincidence that in our world, crocodiles happen to be cold-blooded. It is easy to imagine a slightly different sort of world which is much like our own except that crocodiles and other reptiles are warm-blooded. In that context, the same reasoning would lead to a false conclusion. This shows that the conclusion is not a necessary truth in all contexts for which the premises are true.


\ea \label{ex:4.5}
\ea  Premise 1: \textit{If crocodiles are mammals, they are warm-blooded.\\
}Premise 1: \textit{Crocodiles are not mammals}.\\
\FelixHRule
Conclusion: \textit{Crocodiles are not warm-blooded}.
\ex  Premise 1: \textit{If bats are birds, then they have wings.}\\
Premise 1: \textit{Bats are not birds}.\\
\FelixHRule
Conclusion: \textit{Bats do not have wings}.
\z \z


Another way of showing that this pattern of inference is invalid is to change the content words while preserving the same logical structure, as illustrated in (\ref{ex:4.5}b). In this example the conclusion is false even though both premises are true, showing that the logical structure of the inference is invalid.



We have said that one important goal of logic is to provide a systematic account for the kinds of reasoning or inference that we intuitively know to be correct. In addition, logic can help us move beyond our intuitions in at least two important ways. First, it provides a way of analyzing very complex arguments, for which our intuitions do not give reliable judgements. Second, our intuitive reasoning may sometimes be based on patterns of inference which are not in fact valid. Logic provides an objective method for distinguishing valid from invalid patterns of inference, and a way of proving which patterns belong to each of these types. We now procede to survey the basic notation and concepts used in the two primary branches of logic, beginning with propositional logic.


\section{Propositional logic}\label{sec:} %3. /
\subsection{Propositional operators}\label{sec:} %3.1 /

In \sectref{sec:1} we introduced the logical negation operator “¬”. (An alternate symbol for this is the tilde, “{\textasciitilde}”; so in logical notation, ‘not p’ can be written as either \textit{¬p} or {\textasciitilde}\textit{p}.) Logical negation is referred to as a “one-place” operator, because it combines with a single proposition to form a new proposition. The other basic operators of propositional logic are referred to as “two-place” operators, because they are used to combine two propositions to form a new complex proposition. The basic two-place operators include $\wedge$ ‘and’, $\vee$ ‘or’, and the \textsc{material} \textsc{implication} operator → (generally read as ‘if…then…’). If \textit{p} and \textit{q} are well-formed propositions, then the formulae \textit{p$\wedge$}\textit{q} ‘p and q’, \textit{p$\vee$}\textit{q} ‘p or q’, and \textit{p→}\textit{q} ‘if p, (then) q’ are also well-formed propositions. (The \textit{p} and \textit{q} in these formulae are \textsc{variables} which represent propositions.)



A word of caution is in order here. In reading logical formulae we use English words like \textit{not}, \textit{and}, \textit{or}, and \textit{if} to pronounce the logical operators, for convenience; but we cannot assume that the meanings of these English words are identical to the meanings of the corresponding operators. This turns out to be an interesting and somewhat controversial question, and we will return to it in chapters 9 and 19. For the purposes of this chapter, as a way to introduce the logical notation itself, we will use the English words as simple translation equivalents for the logical operators; but the reader should bear in mind that there is more to be said about this issue, and we will say some of it in later chapters.



These four operators determine the “syntax” of the complex propositions that they are used to create. They specify, for example, that \textit{¬p} and \textit{p$\wedge$}\textit{q} are valid formulae but \textit{p¬} and \textit{pq$\wedge$} are not. These operators also determine certain aspects of the meaning of these complex propositions, specifically their truth values. For example, if we are told that proposition \textit{p} is true in a given situation, we can be very sure that its negation (\textit{¬p}) is false in that situation. Conversely, if \textit{p} is false in a given situation, we know that its negation (\textit{¬p}) must be true in that situation. We do not need to know what \textit{p} actually means in order to make these predictions; all we need to know is its truth value.



The other operators also specify the truth values of the complex propositions that they form based only on the truth values of the individual propositions that they combine with. For this reason, the meanings of these operators (i.e., their contribution to the meaning of a proposition) can be fully specified in terms of truth values. When we have said that \textit{p} and \textit{¬p} must have opposite truth values in any possible situation, we have provided a definition of the negation operator; nothing needs to be known about the specific meaning of \textit{p}. One common way of representing this kind of definition is through the use of a \textsc{truth table}, like that in \REF{ex:4.6}. This table says that whenever \textit{p} is true (T), \textit{not p} must be false (F); and whenever \textit{p} is false, \textit{not p} must be true.


\ea \label{ex:4.6}
\begin{tabular}[t]{>{\sffamily}c>{\sffamily}c}
\lsptoprule
\tablehead{
 p & ¬p\\\midrule
}
  T &  F\\
  F &  T\\
\lspbottomrule
\end{tabular}
\z

In the same way, the operator \textit{$\wedge$} ‘and’ can be defined by the truth table in \REF{ex:4.7}. This table says that \textit{p$\wedge$}\textit{q} (which is also sometimes written \textit{p\&}\textit{q}) is true just in case both \textit{p} and \textit{q} are true, and false in all other situations.


\ea \label{ex:4.7}
\begin{tabular}[t]{>{\sffamily}c>{\sffamily}c>{\sffamily}c}
\lsptoprule
\tablehead{
 p & q & p $\wedge$ q\\\midrule
}
 \sffamily T & \sffamily T & \sffamily T\\
 \sffamily T & \sffamily F & \sffamily F\\
 \sffamily F & \sffamily T & \sffamily F\\
 \sffamily F & \sffamily F & \sffamily F\\
\lspbottomrule
\end{tabular}
\z

Again, the truth value of the complex proposition does not depend on the meaning of the simpler propositions it contains, but only on their truth values and the meaning of \textit{$\wedge$}. Nevertheless, we can assign arbitrary meanings to the variables in order to illustrate the function of the operator. Suppose for example that \textit{p} represents the proposition ‘It is raining,’ and \textit{q} represents the proposition ‘The north wind is blowing.’ The formula \textit{p$\wedge$}\textit{q} would then represent the proposition ‘It is raining and the north wind is blowing.’ The truth table in \REF{ex:4.7} predicts that this proposition will only be true if, at the time of speaking, there is a north wind accompanied by rain; it will be false if the weather is different in either of these respects. This prediction seems to match our intuitions as speakers of English. We can see this by imagining someone saying to us, \textit{It is raining and the north wind is blowing}. We would consider the speaker to have spoken truthfully just in case there was a north wind accompanied by rain, and falsely if the circumstances were otherwise.



The operator $\vee$ ‘or’ is defined by the truth table in \REF{ex:4.8}. This table says that \textit{p}$\vee$\textit{q} is true whenever either \textit{p} is true or \textit{q} is true; it is only false when both \textit{p} and \textit{q} are false. Notice that this \textit{or} of standard logic is the \textsc{inclusive} \textit{or}, corresponding to the English phrase \textit{and/or}, because it includes the case where both \textit{p} and \textit{q} are true. Suppose, for example, that \textit{p} represents the proposition ‘It is raining,’ and \textit{q} represents the proposition ‘It is snowing.’ Imagine a meteorologist looking at a radar display and, based on what he sees there, saying: ‘It is raining or it is snowing.’ This statement would be true if it was raining at the time of speaking, or if it was snowing, or if both things were happening at the same time. (This last possibility is rare but not impossible.)


\ea \label{ex:4.8}
\begin{tabular}[t]{>{\sffamily}c>{\sffamily}c>{\sffamily}c}
\lsptoprule
\tablehead{
 p & q & p $\vee$ q\\\midrule
}
 \sffamily T & \sffamily T & \sffamily T\\
 \sffamily T & \sffamily F & \sffamily T\\
 \sffamily F & \sffamily T & \sffamily T\\
 \sffamily F & \sffamily F & \sffamily F\\
\lspbottomrule
\end{tabular}
\z

In spoken English we often use the word \textit{or} to mean ‘either … or … but not both’. For example, this is normally the usage that we intend when we ask, “Would you like white wine or red?” Table \REF{ex:4.9} shows how we would define this \textsc{exclusive} “sense” of \textit{or}, abbreviated here as \textit{XOR}. The table says that \textit{p XOR q} will be true whenever either p or q is true, but not both; it is false whenever p and q have the same truth value. (We will return in \chapref{sec:9} to the question of whether we should consider the English word \textit{or} to have two distinct senses.)


\ea \label{ex:4.9}
\begin{tabular}[t]{>{\sffamily}c>{\sffamily}c>{\sffamily}c}
\lsptoprule
\tablehead{
 p & q & p \textsf{XOR} q\\\midrule
}
 \sffamily T & \sffamily T & \sffamily F\\
 \sffamily T & \sffamily F & \sffamily T\\
 \sffamily F & \sffamily T & \sffamily T\\
 \sffamily F & \sffamily F & \sffamily F\\
\lspbottomrule
\end{tabular}
\z

The \textsc{material} \textsc{implication} operator (→) is defined by the truth table in \REF{ex:4.10}. (The formula \textit{p}→\textit{q} can be read as \textit{if p (then) q}, \textit{p only if q}, or \textit{q if p}.) The truth table says that \textit{p}→\textit{q} is defined to be false just in case \textit{p} is true but \textit{q} is false; it is true in all other situations.


\ea \label{ex:4.10}
\begin{tabular}[t]{>{\sffamily}c>{\sffamily}c>{\sffamily}c}
\lsptoprule
\tablehead{
 p & q & p → q\\\midrule
}
 \sffamily T & \sffamily T & \sffamily T\\
 \sffamily T & \sffamily F & \sffamily F\\
 \sffamily F & \sffamily T & \sffamily T\\
 \sffamily F & \sffamily F & \sffamily T\\
\lspbottomrule
\end{tabular}
\z

In order to get an intuitive sense of what this definition means, suppose that a mother says to her children, \textit{If it rains this afternoon I will take you to a movie}. Under what circumstances would the mother be considered to have spoken falsely? In applying the truth table we let \textit{p} represent \textit{it rains this afternoon} and \textit{q} represent \textit{I will take you to a movie}. Now suppose that it does not rain. In that case \textit{p} is false, and whether the family goes to a movie or not, no one would accuse the mother of lying or breaking her promise; and this is what the truth table predicts. If it does rain, then \textit{p} is true; and if the mother takes her children to a movie, she has spoken the truth. Only if it rains but she does not take her children to a movie would her statement be considered false. Again, this is just what the truth table predicts. (It turns out that the material implication operator of standard logic does not always correspond to our intuitions about English \textit{if}, and we will have much more to say about this in \chapref{sec:19}.)



For convenience we will introduce one additional operator here, which is referred to as the \textsc{biconditional} operator (\textit{$\leftrightarrow $}). The formula \textit{p$\leftrightarrow $}\textit{q} (read as ‘p if and only if q’) is a short-hand or abbreviation for: (\textit{p}→\textit{q) $\wedge$} \textit{(q}→\textit{p)}. The biconditional operator is defined by the truth table in \REF{ex:4.11}:


\ea \label{ex:4.11}
\begin{tabular}[t]{>{\sffamily}c>{\sffamily}c>{\sffamily}c}
\lsptoprule
\tablehead{
 p & q & p $\leftrightarrow $ q\\\midrule
}
 \sffamily T & \sffamily T & \sffamily T\\
 \sffamily T & \sffamily F & \sffamily F\\
 \sffamily F & \sffamily T & \sffamily F\\
 \sffamily F & \sffamily F & \sffamily T\\
\lspbottomrule
\end{tabular}
\z

This table says that \textit{p}$\leftrightarrow $\textit{q} is true just in case \textit{p} and \textit{q} have the same truth value. Suppose the mother in our previous example had said \textit{I will take you to a movie if and only if it rains this afternoon}. If it did not rain but she took her children to a movie anyway, the truth table says that she would have spoken falsely. This prediction seems linguistically correct, although her children would very likely have forgiven her in this case.



Having introduced the basic operators of propositional logic, let us see how they can be used to identify certain kinds of tautologies and contradictions, and to account for certain kinds of meaning relations between propositions (entailment, paraphrase, and incompatibility), namely those that are the result of logical structure alone.


\subsection{Meaning relations and rules of inference}\label{sec:} %3.2 /

In addition to using truth tables to define logical operators, we can also use them to evaluate more complex logical formulae. To begin with a very simple example, the formula \textit{p$\vee$}\textit{(¬}\textit{p)} represents the logical structure of sentences like \textit{Either you will graduate or you will not graduate}. Sentences of this type are clearly tautologies, and we can show why using a truth table. We start by putting the basic proposition (\textit{p}) at the top of the left column and the formula that we want to prove (\textit{p$\vee$}\textit{(¬}\textit{p)}) at the top of the last (right-most) right column, as shown in (\ref{ex:4.12}a). We can also fill in all the possible truth values for \textit{p} in the left column.


\ea \label{ex:4.12}
\ea \begin{tabular}[t]{>{\sffamily}c>{\sffamily}c>{\sffamily}c}
\lsptoprule
\tablehead{
 p &  & p$\vee$(¬p)\\\midrule
}
 \sffamily T &  & \\
 \sffamily F &  & \\
\lspbottomrule
\end{tabular}
\z \z

The proposition we are trying to prove (\textit{p$\vee$}\textit{(¬}\textit{p)}) is an \textit{or} statement; that is, the highest operator is $\vee$. The two propositions conjoined by $\vee$ are \textit{p} and \textit{¬p}. We already have a column for the truth values of \textit{p}, so the next step is to create a column for the corresponding truth values of \textit{¬p}, as shown in (\ref{ex:4.12}b).

\usecounter{equation}\setcounter{equation}{11}
\ea
\begin{xlista} \exi{b.} \begin{tabular}[t]{>{\sffamily}c>{\sffamily}c>{\sffamily}c}
\lsptoprule
\tablehead{
 p & ¬p & p$\vee$(¬p)\\\midrule
}
 \sffamily T & \sffamily F & \\
 \sffamily F & \sffamily T & \\
\lspbottomrule
\end{tabular}
\end{xlista} \z

The final step in the proof is to calculate the possible truth values of the proposition \textit{p$\vee$}\textit{(¬}\textit{p)}, using the truth table in \REF{ex:} which defines the \textit{$\vee$} operator. The result is shown in (\ref{ex:4.12}c). 


\usecounter{equation}\setcounter{equation}{11}
\ea
\begin{xlista} \exi{c.} \begin{tabular}[t]{>{\sffamily}c>{\sffamily}c>{\sffamily}c}
\lsptoprule
\tablehead{
 p & ¬p & p$\vee$(¬p)\\\midrule
}
 \sffamily T & \sffamily F & \sffamily T\\
 \sffamily F & \sffamily T & \sffamily T\\
\lspbottomrule
\end{tabular}
\end{xlista} \z

Notice that both cells in the right-most column contain T. This means that the formula is always true, under any circumstances; in other words, it is a tautology. The truth of this tautology does not depend in any way on the meaning of \textit{p}, but only on the definitions of the logical operators \textit{$\vee$} and \textit{¬}. Propositions which are necessarily true just because of their logical structure (regardless of the meanings of words they contain) are sometimes said to be “logically true”.



Suppose we change the \textit{or} in the previous example to \textit{and}. This would produce the formula \textit{p$\wedge$}\textit{(¬}\textit{p)}, which corresponds to the logical structure of sentences like Y\textit{ou will graduate and you will not graduate}. It is hard to imagine any context where such a sentence could be true, and using the truth table in \REF{ex:4.13} we can show why this is impossible. Sentences of this type are contradictions; they are never true, under any possible circumstance, as reflected in the fact that both cells in the right-most column contain F.


\ea \label{ex:4.13}
\begin{tabular}[t]{>{\sffamily}c>{\sffamily}c>{\sffamily}c>{\sffamily}c}
\lsptoprule
\tablehead{
 p & ¬p & p$\wedge$(¬p) & \\\midrule
}
 \sffamily T & \sffamily F & \sffamily F\\
 \sffamily F & \sffamily T & \sffamily F\\
\lspbottomrule
\end{tabular}
\z

Now let us consider a slightly more complex example: \textit{((p$\vee$q) $\wedge$ (¬p))} → \textit{q}. To construct a truth table which will allow us to evaluate this formula, we begin by putting the basic propositions \textit{p} and \textit{q} in the left-hand columns (1\& 2). We put the complete formula that we want to prove in the far right column (6). We introduce a new column for each constituent part of the complete formula and calculate truth values for each cell, building from left to right, as seen in \REF{ex:4.14}. First, columns 1 \& 2 are used to construct column 3, based on the truth table for \textit{$\vee$}. Next, column 4 is calculated from column 1. Columns 3 \& 4 are used to construct column 5, based on the truth table for \textit{$\wedge$}. Finally, columns 2 \& 5 are used to construct column 6, based on the truth table for →.


\ea \label{ex:4.14}
\begin{tabular}[t]{*{6}{>{\sffamily}c}}
\lsptoprule
\tablehead{ 1 & 2 & 3 & 4 & 5 & 6 \\
 p & q & p$\vee$q & ¬p & (p$\vee$q)$\wedge$¬p & ((p$\vee$q)$\wedge$¬p) → q\\\midrule
}
 \sffamily T & \sffamily T & \sffamily T & \sffamily F & \sffamily F & \sffamily T\\
 \sffamily T & \sffamily F & \sffamily T & \sffamily F & \sffamily F & \sffamily T\\
 \sffamily F & \sffamily T & \sffamily T & \sffamily T & \sffamily T & \sffamily T\\
 \sffamily F & \sffamily F & \sffamily F & \sffamily T & \sffamily F & \sffamily T\\
\lspbottomrule
\end{tabular}
\z

Notice that every cell in the right-most column contains T. This means that the formula is always true, under any circumstances; in other words, it is a tautology. Furthermore, the truth of this tautology does not depend in any way on the meanings of \textit{p} and \textit{q}, but only on the definitions of the logical operators. This tautology predicts that whenever a proposition of the form \textit{((p$\vee$q) $\wedge$ (¬p))} is true, the proposition \textit{q} must also be true. For example, it explains why the sentence cited at the beginning of \sectref{sec:2} (\textit{Either Joe is crazy or he is lying, and he is not crazy}) must entail \textit{Joe is lying}. A similar entailment relation will hold for any other pair of sentences that have the same logical structure.



Now consider the biconditional formula \textit{(p$\vee$q) $\leftrightarrow $ ¬((¬p) $\wedge$ (¬q))}. Using the procedure outlined above, we can construct the truth table in \REF{ex:4.15}. First, columns 1 \& 2 are used to construct column 3, based on the truth table for \textit{$\vee$}. Next, columns 4 \& 5 are used to construct column 6, based on the truth table for \textit{$\wedge$}. Column 7 is calculated from column 6, and finally columns 3 \& 7 are used to construct column 8, based on the truth table for \textit{$\leftrightarrow $}.


\ea \label{ex:4.15}
\begin{tabular}[t]{>{\sffamily}c>{\sffamily}c>{\sffamily\cellcolor{lsLightGray}}c>{\sffamily}c>{\sffamily}c>{\sffamily}c>{\sffamily\cellcolor{lsLightGray}}c>{\sffamily}c}
\lsptoprule
\tablehead{
 p & q & p$\vee$q & ¬p & ¬q & (¬p)$\wedge$(¬q) & ¬((¬p)$\wedge$(¬q)) & (p$\vee$q) $\leftrightarrow $ ¬((¬p) $\wedge$ (¬q))\\\midrule
}
 \sffamily T & \sffamily T & \sffamily T & \sffamily F & \sffamily F & \sffamily F & \sffamily T & \sffamily T\\
 \sffamily T & \sffamily F & \sffamily T & \sffamily F & \sffamily T & \sffamily F & \sffamily T & \sffamily T\\
 \sffamily F & \sffamily T & \sffamily T & \sffamily T & \sffamily F & \sffamily F & \sffamily T & \sffamily T\\
 \sffamily F & \sffamily F & \sffamily F & \sffamily T & \sffamily T & \sffamily T & \sffamily F & \sffamily T\\
\lspbottomrule
\end{tabular}
\z

Once again we see that every cell in the right-most column contains T, which means that this formula must always be true, purely because of its logical form. The biconditional operator in this formula expresses mutual entailment, that is, a paraphrase relation. This formula explains why the sentence \textit{Either he is crazy or he is lying} must always have the same truth value as \textit{It is not the case that he is both not crazy and not lying}. The first sentence is a paraphrase of the second, simply because of the logical structures involved.



As we noted in an earlier chapter, tautologies are not very informative because they make no claim about the world. But for that very reason, these logical tautologies can be extremely useful because they define logically valid rules of inference. A few tautologies are so famous as rules of inference that they are given Latin names. One of these is called \textsc{Modus Ponens} ‘method of positing/affirming’, also called ‘affirming the antecedent’: \textit{((p→q) $\wedge$ p) → q}. The proof of this tautology is presented in \REF{ex:4.16}.


\ea \label{ex:4.16}
\begin{tabular}[t]{*{5}{>{\sffamily}c}}
\lsptoprule
\tablehead{
 p & q & p→q & (p→q) $\wedge$ p & ((p→q) $\wedge$ p) → q\\\midrule
}
 \sffamily T & \sffamily T & \sffamily T & \sffamily T & \sffamily T\\
 \sffamily T & \sffamily F & \sffamily F & \sffamily F & \sffamily T\\
 \sffamily F & \sffamily T & \sffamily T & \sffamily F & \sffamily T\\
 \sffamily F & \sffamily F & \sffamily T & \sffamily F & \sffamily T\\
\lspbottomrule
\end{tabular}
\z

Modus Ponens defines one of the valid ways of deriving an inference from a conditional statement. It says that if we know that \textit{p→q} is true, and in addition we know or assume that \textit{p} is true, it is valid to infer that \textit{q} is true. An illustration of this pattern of inference is presented as a \textsc{syllogism} in \REF{ex:4.17}.


\ea \label{ex:4.17}
Premise 1: \textit{If John is Estonian he will like this book.}  (p→q)\\
Premise 2: \textit{John is Estonian}.   (p)\\
\FelixHRule
Conclusion: \textit{He will like this book}.   (q)
\z


As we noted in \sectref{sec:2}, Modus Ponens guarantees a valid inference but does not guarantee a true conclusion. The conclusion will only be as reliable as the premises that we begin with. Suppose in this example it turns out that John is Estonian but hates the book. This does not disprove the rule of Modus Ponens; rather, it shows that the first premise is false, by providing a counter-example.



Another valid rule for deriving an inference from a conditional statement is \textsc{Modus Tollens} ‘method of rejecting/denying’, also called ‘denying the consequent’: \textit{((p→q) $\wedge$ ¬q) → ¬p}. This rule was illustrated in example (\ref{ex:4.4}a) above, repeated here as \REF{ex:4.18}. It says that if we know that \textit{p→q} is true, and in addition we know or assume that \textit{q} is false, it is valid to infer that \textit{p} is also false.

\settowidth\jamwidth{(p→q)}
\ea \label{ex:4.18}
Premise 1: \textit{If dolphins are fish, they are cold-blooded.} \jambox{(p→q)}
Premise 2: \textit{Dolphins are not cold-blooded}.   \jambox{(¬q)}
\FelixHRule
Conclusion: \textit{Dolphins are not fish}.   \jambox{(¬p)}
\z


The tautology which we proved in \REF{ex:4.14} is known as the \textsc{Disjunctive Syllogism}: \textit{((p$\vee$q) $\wedge$ (¬p)) → q}. Another example which illustrates this pattern of inference is provided in \REF{ex:4.19}.


\ea \label{ex:4.19}
Premise 1: \textit{Dolphins are either fish or mammals}.  \jambox{(p$\vee$q)}
Premise 2: \textit{Dolphins are not fish}.  \jambox{(¬p)}
\FelixHRule
Conclusion: \textit{Dolphins are mammals}.   \jambox{(q)}
\z


Finally, the tautology known as the \textsc{Hypothetical} \textsc{Syllogism} is illustrated in \REF{ex:4.20}.


\ea \label{ex:4.20}
((p→q) $\wedge$ (q→r)) → (p→r)\\
Premise 1: \textit{If Mickey is a rodent, he is a mammal}.  \jambox{(p→q)}
Premise 2: \textit{If Mickey is a mammal, he is warm-blooded}.  \jambox{(q→r)}
\FelixHRule
Conclusion: \textit{If Mickey is a rodent, he is warm-blooded}.  \jambox{(p→r)}
\z

\ea \label{ex:4.21}
\ea \textit{All men are mortal.}\\
\textit{Socrates is a man}.\\
\FelixHRule
\textit{Therefore,} \textit{Socrates is mortal.}
\ex \textit{Arthur is a lawyer.}\\
\textit{Arthur is honest}.\\
\FelixHRule
\textit{Therefore,} \textit{some (= at least one) lawyer is honest.}
\z \z

The propositional logic outlined in this section is an important part of the logical metalanguage for semantic analysis, but it is not sufficient on its own because it is concerned only with truth values. We need a way to go beyond \textit{p} and \textit{q}, to represent the actual meanings of the basic propositions we are dealing with. \textsc{Predicate logic} gives us a way to include information about word meanings in logical expressions.


\section{Predicate logic}\label{sec:} %4. /

Consider the simple sentences in \REF{ex:4.22}:


\ea \label{ex:4.22}
\ea John is hungry.\\
\ex Mary snores.\\
\ex John loves Mary.\\
\ex Mary slapped John.
                       \z
\z


Each of these sentences describes a property, event or relationship. The element of meaning which determines what kind of property, event or relationship is being described is called the \textsc{predicate}. The words \textit{hungry}, \textit{snores}, \textit{loves}, and \textit{slapped} express the predicates in these examples. The individuals of whom the property or relationship is claimed to be true (\textit{John} and \textit{Mary} in these examples) are referred to as \textsc{arguments}. As we can see from example \REF{ex:4.22}, different predicates require different numbers of arguments: \textit{hungry} and \textit{snore} require just one, \textit{love} and \textit{slap} require two. When a predicate is asserted to be true of the right number of arguments, the result is a well-formed proposition, i.e., a claim about the world which can (in principle) be assigned a truth value, T or F.



In our logical notation we will write predicates in capital letters (to distinguish them from normal English words) and without inflectional morphology. We follow the common practice of using lower case initials to represent proper names. For predicates which require two arguments, the agent or experiencer is normally listed first. So the simple sentence \textit{John is hungry} would be translated into the logical metalanguage as HUNGRY(j), while the sentence \textit{John loves Mary} would be translated LOVE(j,m). Some additional examples are shown in \REF{ex:4.23}.


\ea\label{ex:4.23}
\begin{tabular}[t]{@{}lll}
a. & \textit{Henry VIII snores.} & SNORE(h)\\
b. & \textit{Socrates is a man.} & MAN(s)\\
c. & \textit{Napoleon is near Paris.} & NEAR(n,p)\\
d. & \textit{Abraham Lincoln admired Queen Victoria.} & ADMIRE(a,v)\\
e. & \textit{Jocasta is Oedipus’ mother}. & MOTHER\_OF(j,o)\\
f. & \textit{Abraham Lincoln was tall and homely.} & TALL(a) $\wedge$ HOMELY(a)\\
g. & \textit{Abraham Lincoln was a tall man.}  & TALL(a) $\wedge$ MAN(a)\\
h. & \textit{Joe is neither honest nor competent}. & ¬ (HONEST(j) $\vee$ COMPETENT(j))
\end{tabular}
\z


As these examples illustrate, semantic predicates can be expressed grammatically as verbs, adjectives, common nouns, or even prepositions. They can appear as part of the VP, or as modifiers within NP as in (\ref{ex:4.23}g).\footnote{VP = verb phrase, that is, the verb plus its non-subject arguments. NP = noun phrase.}



We have seen examples of one-place and two-place predicates; there are also predicates which take three arguments, e.g. \textit{give}, \textit{show}, \textit{offer}, \textit{send}, etc. Some predicates, including verbs like \textit{say}, \textit{think}, \textit{believe}, \textit{want}, etc., can take propositions as arguments:


\ea \label{ex:4.24}
\begin{tabular}[t]{lll}
a. & \textit{Henry thinks that Anne is beautiful.}  & THINK(h, BEAUTIFUL(a))\\
b. & \textit{Susan wants to marry Ringo.} & WANT(s, MARRY(s,r))
\end{tabular}
\z

\subsection{Quantifiers (an introduction)}\label{sec:} %3.1 /

All the predicates in examples (\ref{ex:4.22}--\ref{ex:4.24}) have proper names as arguments. Of course we need to be able to represent other kinds of arguments as well. We will discuss this issue in more detail in later chapters, but as a brief introduction let us consider the subject NPs in \REF{ex:4.25}:


\ea \label{ex:4.25}
\ea \textit{All students} are weary.\\
\ex \textit{Some men} snore.\\
\ex \textit{No crocodile} is warm-blooded.
                       \z
\z

The italicized phrases in \REF{ex:4.25} are examples of “quantified” NPs; they contain a special kind of determiner known as a \textsc{quantifier}. Sentence (\ref{ex:4.25}a) makes a universal generalization. It says that if you select anything within the universe of discourse that happens to be a student, that thing will also be weary. Notice that the phrase \textit{all students} does not refer to any specific individual, or set of individuals; that is why we said in \chapref{sec:2} that quantified NPs are generally not referring expressions. Rather, the phrase seems to express a kind of inference: if a given thing is a student, then it will also have the property expressed in the remainder of the sentence.


Sentence (\ref{ex:4.25}b) makes an existential claim. It says that there exists at least one thing within the universe of discourse that is both a man and snores. Actually, this sentence says that there must be at least two such things, but that is not part of the meaning of \textit{some}; it follows from the fact that the noun \textit{men} is plural. (We can show this by comparing (\ref{ex:4.26}a) with (\ref{ex:4.26}b).) \textit{Some} simply means that there exists something within the universe of discourse that has both of the named properties (e.g., being a man and snoring). Sentence (\ref{ex:4.25}c) is a negative existential statement. It says that there does not exist anything within the universe of discourse that is both a crocodile and warm-blooded.


\ea \label{ex:4.26}
\ea \textit{Some guy} in the back row was snoring.  (at least one)\\
\ex \textit{Some guys} in the back row were snoring.  (at least two)
                       \z
\z

Standard predicate logic makes use of two quantifier symbols: the Universal Quantifier ${\forall}$ and the Existential Quantifier ${\exists}$. As the mathematical examples in \REF{ex:4.27} illustrate, these quantifier symbols must introduce a \textsc{variable}, and this variable is said to be \textsc{bound} by the quantifier. The letters \textit{x}, \textit{y} or \textit{z} are normally used as variables that represent individuals. (We can read “${\forall}$x” as ‘for all individuals x’, and “${\exists}$x” as ‘there exists one or more individuals x’.) 

\ea \label{ex:4.27}
\ea  Universal Quantifier:\\
${\forall}$x[x+x = 2x]
\ex  Existential Quantifier:\\
${\exists}$y[y+4 = y/3]
\z \z

Quantifier words must be interpreted relative to the current universe of discourse, that is, the set of individuals currently available for discussion. For example, in order to decide whether sentences like \textit{All students are female} or \textit{No student is wealthy} are true, we need to know what the currently relevant universe of discourse is. If we are discussing a secondary school for economically disadvantaged girls, both statements would be true. In other contexts, either or both of these statements might be false.


In the same way, variables bound by one of the logical quantifier symbols are assumed to be members of the currently relevant \textsc{universal set}, i.e., the set of all elements currently available for consideration.\footnote{The concept of \textsc{universal set} is discussed further in \chapref{sec:13}.} In mathematical contexts, the universal set is often a particular class of numbers, e.g. the integers or the real numbers. In order to evaluate a proposition involving quantifier symbols, like those in \REF{ex:4.27}, the universal set must be specified or assumed from context.



Variables bound by a quantifier do not refer to a specific individual or entity, but rather allow for the arbitrary selection of any individual or entity within the universal set. Once a particular value is assigned to a given variable, the same assignment is understood to hold for all occurrences of that variable within the \textsc{scope} of the quantifier (the material inside the square brackets). So for example, if we assume that the universal set in \REF{ex:4.27} is the set of all real numbers, (\ref{ex:4.27}a) can be interpreted as follows: “Choose any real number. If you add that number to itself, the sum will be equal to that number multiplied by two.” The equation in (\ref{ex:4.27}b) can be interpreted as follows: “There exists some real number which, when added to four, will be equal to the quotient of that same number divided by three.”



The value of an unbound (or “free”) variable, that is, one which is not introduced by a quantifier or which occurs outside the scope of its quantifier, is not defined. The variables in \REF{ex:4.28} are not bound, and as a result the equations in which they occur are neither true nor false; they do not make any claim about the world, until some value is assigned to each variable. (In contrast, both of the equations in \REF{ex:4.27}, where the variables are bound, can be shown to be true.) Of course, it is fairly easy to “solve” the equations in \REF{ex:4.28}, that is, to find the values that must be assigned to each variable in order to make the equations true. But until some value is assigned, no truth value can be determined for the equations.


\ea \label{ex:4.28}
\ea  $x–7 = 4x$\\
\ex  $y+2z = 51$
                       \z
\z


The same applies to variables which occur within logical formulae. A proposition that contains unbound variables is called an \textsc{open proposition}. Such a proposition cannot be assigned a truth value, unless some mechanism is provided for assigning values to the unbound variables.



The universal and existential quantifier symbols allow us to translate the sentences in \REF{ex:4.25} into logical notation, as shown in \REF{ex:4.29}. (We will ignore for the moment the difference in interpretation between singular vs. plural nouns with \textit{some}.)


\ea \label{ex:4.29}
\ea  Universal Quantifier: \textit{All students} \textit{are weary.}\\
${\forall}$x[STUDENT(x) → WEARY(x)]
\ex  Existential Quantifier: \textit{Some men} \textit{snore.}\\
${\exists}$x[MAN(x) $\wedge$ SNORE(x)]
\ex  Negative Existential: \textit{No crocodile is warm-blooded.}\\
¬${\exists}$x[CROCODILE(x) $\wedge$ WARM-BLOODED(x)]
\z \z

Notice that \textit{all} is translated differently from \textit{some} or \textit{no}. The universal quantifier is paired with material implication (→), while the existential quantifier introduces an \textit{and} statement. We will discuss the reason for this difference in more detail in Unit 4, but the fundamental issue is that we want our logical translation to have the same interpretation as the English sentence it is meant to represent. We might interpret the formula in (\ref{ex:4.29}a) roughly as follows: “Choose something within the universe of discourse. We will temporarily call that thing ‘x’. Is x a student? If so, then x will also be weary.” This long-winded paraphrase seems to describe the same state of affairs as the original sentence \textit{All students} \textit{are weary}. However, if we replace → with $\wedge$, we get the formula in \REF{ex:4.30}, which means something very different.

\ea \label{ex:4.30}
${\forall}$x[STUDENT(x) $\wedge$ WEARY(x)]\\
‘Everything in the universe of discourse is a student and is weary.’
\z


So far we have only considered quantifier phrases which occur as subject NPs, but of course they can occur in other syntactic positions as well. When we translate a sentence containing a quantified NP into logical notation, the quantifier always comes at the beginning of the proposition which it takes scope over, even when the quantified NP is functioning as direct object, oblique argument, etc. Some examples are presented in \REF{ex:}. Note that indefinite NPs are often translated as existential quantifiers, as illustrated in (\ref{ex4.31:}b--c).


\ea \label{ex:4.31}
\ea \textit{John loves all girls.}\\
  ${\forall}$x[GIRL(x) → LOVE(j,x)]\\
\ex \textit{Susan has married a cowboy.}\\
  ${\exists}$x[COWBOY(x) $\wedge$ MARRY(s,x)]\\
\ex \textit{Ringo lives in a yellow submarine}.\\
  ${\exists}$x[YELLOW(x) $\wedge$ SUBMARINE(x) ${\wedge}$ LIVE\_IN(r,x)]
                       \z
\z


The patterns of inference observed in example \REF{ex:4.2} above illustrate two basic principles that govern the use of quantifiers. The first principle, which is called \textsc{universal instantiation}, states that anything which is true of all members of a particular class is true of any specific member of that class. This is the principle which licenses the inference shown in (\ref{ex:4.2}a), repeated here as (\ref{ex:4.32}a). The second principle, which is called \textsc{existential generalization}, licenses the inference shown in (\ref{ex:4.2}b), repeated here as (\ref{ex:4.32}b).

\settowidth\jamwidth{${\forall}$x[MAN(x) → MORTAL(x)}
\ea \label{ex:4.32}
\ea  \textit{All men are mortal.}  \jambox{${\forall}$x[MAN(x) → MORTAL(x)]}
\textit{Socrates is a man}.        \jambox{MAN(s)}
\FelixHRule
\textit{Therefore,} \textit{Socrates is mortal.}  \jambox{MORTAL(s)}
\ex   \textit{Arthur is a lawyer.}                \jambox{LAWYER(a)}
\textit{Arthur is honest}.                        \jambox{HONEST(a)}
\FelixHRule
\textit{Therefore,} \textit{some (= at least one) lawyer is honest.}  \jambox{${\exists}$x[LAWYER(x) $\wedge$ HONEST(x)}
\z \z

\subsection{Scope ambiguities}\label{sec:} %3.2. /

When a quantifier combines with another quantifier, with negation, or with various other elements (to be discussed in \chapref{sec:14}), it can give rise to ambiguities of scope. In (\ref{ex:4.33}a) for example, one of the quantifiers must appear within the scope of the other, so there are two possible “readings” for the sentence.


\ea \label{ex:4.33}
\ea \textit{Some man loves every woman.}\\
  \begin{xlisti} 
      \ex ${\exists}$x[MAN(x) $\wedge$ (${\forall}$y[WOMAN(y) → LOVE(x,y)])]\\
      \ex ${\forall}$y[WOMAN(y) → (${\exists}$x[MAN(x) $\wedge$ LOVE(x,y)])]
  \end{xlisti} 
\ex  \textit{All that glitters is not gold.}\\
  \begin{xlisti}
  \ex ${\forall}$x[GLITTER(x) → ¬GOLD(x)]\\
  \ex ¬${\forall}$x[GLITTER(x) → GOLD(x)]
  \end{xlisti}
\z \z


The quantifier that appears farthest to the left in the formula gets a \textsc{wide scope} interpretation, meaning that it takes logical priority; the one which is embedded within the scope of the first quantifier gets a \textsc{narrow scope} interpretation. So the first reading for (\ref{ex:4.33}a) says that there exists some specific man who loves every woman. The second reading for (\ref{ex:4.33}a) says that for any woman you choose within the universe of discourse, there exists some man who loves her. (Try to provide similar paraphrases for the two readings of (\ref{ex:4.33}b). Then try to verify that these sentences involve real ambiguities by finding contexts for each sentence where one reading would be true while the other is false.)


\section{Conclusion}\label{sec:} %4. /

In this chapter we mentioned some of the motivations for using formal logic as a semantic metalanguage. We discussed the notion of valid inference, and showed that valid patterns of reasoning guarantee a true conclusion only when the premises are true. We then showed how propositional logic accounts for certain kinds of inferences, namely those which are determined by the meanings of the logical operators ‘and’, ‘or’, ‘not’, and ‘if’. In this way propositional logic helps to explain certain kinds of tautology and contradiction, as well as certain types of meaning relations between sentences (entailment, paraphrase, etc.), namely those which arise due to the logical structure of the sentences involved. Finally we gave a brief introduction to predicate logic, which allows us to represent the meanings of the propositions, and an even brief introduction to the use of quantifiers, which will be the topic of \chapref{sec:14}.



Our emphasis in this chapter was on translating sentences of English (or some other object language) into logical notation. In Unit 4 we will discuss how we can give an interpretation for these propositions in terms of set theory, and how this helps us understand the compositional nature of sentence meanings.



\furtherreading



Good, brief introductions to propositional and predicate logic are provided in Allwood et al. (1977, chapters 4–5) and Kearns (2000, \chapref{sec:2}). More detailed introductions are provided in \citet{Martin1987} and \citet{Gamut1991a}.\footnote{L. T. F. Gamut is a collective pen-name for the Dutch logicians Johan van Benthem, Jeroen Groenendijk, Dick de Jongh, Martin Stokhof and Henk Verkuyl.}


\subsubsection{Discussion exercises:}\label{sec:}
\ea
\textbf{A}: Create a truth table to prove each of the following tautologies:
\z

\ea
  a. Law of Double Negation:  ¬(¬p) $\leftrightarrow $ p\\
\ex Law of Contradiction:  ¬(p ${\wedge}$ ¬p)\\
\ex Modus Tollens:  [(p → q) ${\wedge}$ ¬q]  →  ¬p
\z

\ea
\textbf{B}: Construct syllogisms, using English sentences, to illustrate each of the following patterns of inference:
\z


\ea
\ea Modus Ponens:  [(p → q) ${\wedge}$ p]  →  q
\ex Modus Tollens:  [(p → q) ${\wedge}$ ¬q]  →  ¬p
\ex Hypothetical Syllogism:  [(p → q) ${\wedge}$ (q → r)]  →  (p → r)
\ex Disjunctive Syllogism:  [(p $\vee$ q) ${\wedge}$ ¬p]  →  q
                       \z
                       \z

\ea
\textbf{C:} Translate the following sentences into logical notation:\\
\ea \textit{All unicorns are herbivores}.\\
\ex \textit{No philosophers admire Nietzsche}.\\
\ex \textit{Some green apples are edible}.\\
\ex \textit{Bill feeds all stray cats}.
                       \z
\z

\subsubsection{Homework exercises:}\label{sec:}
\paragraph{A. Using truth tables}

Arthur has been selected to be a juror in a case which has generated a lot of local publicity. He is asked to promise not to read the newspaper or watch television until the trial is finished. There are two different ways in which he can make this commitment:\\
(a) \textit{I will not read the newspaper or watch television until the trial is finished}.\\
(b) \textit{I will not read the newspaper and I will not watch television until the trial is finished}. 

Construct truth tables for these two sentences to show why they are logically equivalent. You may omit the adverbial clause (\textit{until the trial is finished}) from your table.  (\textbf{Hint}: Let \textbf{p} stand for \textit{I will read the newspaper} and \textbf{q} stand for \textit{I will watch television}. Assume the following translation for sentence (a): ¬(p $\vee$ q). Construct a truth table for this proposition, and a second truth table for sentence (b). If the right-most column of the two tables is identical, that means that the two propositions must have the same truth value under any circumstances.)

sentence (a):

\begin{tabularx}{\textwidth}{XXXX} &  &  & \\
\lsptoprule
\tablehead{
 p & q & p $\vee$ q & ¬(p $\vee$ q)\\
}
&  &  & \\
&  &  & \\
&  &  & \\
\lspbottomrule
\end{tabularx}
sentence (b):

\begin{tabularx}{\textwidth}{XXXXX}
\lsptoprule
&  &  &  & \\
 p & q &  &  & \\
\midrule
&  &  &  & \\
&  &  &  & \\
&  &  &  & \\
\lspbottomrule
\end{tabularx}

\paragraph{B. Translate the following sentences into logical notation}

\ea \ea \textit{All famous linguists quote Chomsky}.
\ex \textit{David tutors some struggling students}.
\ex \textit{No president was Buddhist or Hindu}.
\ex \textit{Alice and Betty married Charlie and David, respectively.}
\z
                       \z

\chapter{{5}: Word senses}

\section{Introduction}\label{sec:} %1. /

In \chapref{sec:2} we introduced the important distinction between sense and denotation. We noted that a single word may have more than one sense, a situation referred to as \textsc{lexical ambiguity}. We also noted that two expressions which have different senses may have the same denotation in some particular context, but two expressions which have the same sense must have the same denotation in every imaginable context. So what if a single word can be used to refer to several different kinds of things? Does that mean it has several different senses? The answer is, sometimes yes and sometimes no. This chapter is designed to help you answer this kind of question for specific cases.



We begin in \sectref{sec:2} with the observation that a speaker often has a variety of ways to refer to a particular thing. The speaker’s choice of words reflect different \textsc{construals}, or ways of thinking about the thing. In \sectref{sec:3} we discuss several diagnostic tests that can be used to distinguish true lexical ambiguity from other similar patterns, such as vagueness and underspecification. We then distinguish two different types of lexical ambiguity, \textsc{polysemy} vs. \textsc{homonymy}, recognizing that making this distinction is not always easy; and we discuss the role of context in enabling hearers to choose the intended sense of ambiguous word forms.



In \sectref{sec:4} we discuss some ways in which new senses of words can be created, including \textsc{coercion} and figures of speech. In \sectref{sec:5} we apply the principles developed in \sectref{sec:3} to a certain pattern of variable denotation, illustrated by words like \textit{book} which can be used to name either a physical object or the text or discourse that it contains.


\section{Word meanings as construals of external reality}\label{sec:} %2. /

Words give us a way to describe the world. However, our linguistic descriptions are never complete. In choosing a word to describe a particular thing or event, we choose to express certain bits of information and leave many others unexpressed. For example, suppose that I am holding a rag in my right hand and moving it back and forth across the surface of a table. If you ask me what I am doing, I might reply with either (\ref{ex:5.1}a) or (\ref{ex:5.1}b).


\ea \label{ex:5.1}
\ea I am wiping the table.\\
\ex I am cleaning the table.\\
\ex I wiped/??cleaned the table but it is no cleaner than before.\\
\ex I cleaned/\#wiped the table without touching it.
                       \z
\z


In this situation, both (\ref{ex:5.1}a) and (\ref{ex:5.1}b) would be true descriptions of the event, but they do not mean the same thing. By choosing the word \textit{clean}, I would be specifying a certain change in the state of the table, but leaving the manner unspecified. By choosing the word \textit{wipe}, I would be specifying a certain manner, but not asserting anything about a change of state. The different entailments associated with these two verbs can be demonstrated using examples like (\ref{ex:5.1}c--d).



To take a second example, suppose that you have a large quartz crystal on your desk, which you use as a paperweight. If I want to look more closely at this object, I could ask for it by saying: \textit{May I look at your paperweight?}; or by saying: \textit{May I look at that quartz crystal?} Clearly the words \textit{paperweight} and \textit{quartz crystal} do not mean the same thing; but in this context, they can have the same referent. The lexical meaning of each word includes features which are true of this referent, but neither word encodes all of the properties of the referent. The choice of which word to use reflects the speaker’s \textsc{construal} of (or way of thinking about) the object, and commits the speaker to certain beliefs but not others concerning the nature of the object.



In analyzing word meanings, we are trying to account for linguistically coded information, rather than all the encyclopedic knowledge (or knowledge about the world) which may be associated with a particular word. For example, the fact that a quartz crystal sinks in water is a fact about the world, but probably not a linguistic property of the word \textit{quartz}. But we need to be aware that this distinction between linguistic knowledge vs. knowledge about the world is often difficult to make.


\section{Lexical ambiguity}\label{sec:} %3. /
\subsection{Ambiguity, vagueness, and indeterminacy}\label{sec:} %3.1 /

In \chapref{sec:2} we discussed cases of lexical ambiguity like those in \REF{ex:5.2}. These sentences are ambiguous because they contain a word-form which has more than one sense, and as a result can be used to refer to very different kinds of things. For example, we can use the word \textit{case} to refer to a kind of container or to a legal proceeding; \textit{lies} can be a noun referring to false statements or a verb specifying the posture or location of something. These words have a variety of referents because they have multiple senses, i.e., they are ambiguous. And as we noted in \chapref{sec:2}, the truth value of each of these sentences in a particular context will depend on which sense of the ambiguous word is chosen.


\ea \label{ex:5.2}
\ea The farmer allows walkers to cross the field for free, but the bull \textit{charges}.\\
\ex Headline: Drunk gets nine months in violin \textit{case}.\\
\ex Headline: Reagan wins on budget, but more \textit{lies} ahead.
                       \z
\z


However, there are other kinds of variable reference as well, ways in which a word can be used to refer to different sorts of things even though it may have only a single sense. For example, I can use the word \textit{cousin} to refer to a child of my parent’s sibling, but the person referred to may be either male or female. Similarly, the word \textit{kick} means to hit something with one’s foot, but does not specify whether the left or right foot is used.\footnote{\citet{Lakoff1970}.} We will say that the word \textit{cousin} is \textsc{indeterminate} with respect to gender, and that the word \textit{kick} is indeterminate with respect to which foot is used.\footnote{We follow \citet{Kennedy2011} in using the term \textsc{indeterminacy}; as he points out, some other authors have used the term \textsc{generality} instead. \citet{Gillon1990} makes a distinction between the two terms, using \textsc{generality} for superordinate terms.} We will argue that such examples are not instances of lexical ambiguity: neither of these cases requires us to posit two distinct senses for a single word form. Our basis for making this claim will be discussed in \sectref{sec:3}.2 below.



Another kind of variable reference is observed with words like \textit{tall} or \textit{bald}. How tall does a person have to be to be called “tall”? How much hair can a person lose without being considered “bald”? Context is a factor; a young man who is considered tall among the members of his gymnastics club might not be considered tall if he tries out for a professional basketball team. But even if we restrict our discussion to professional basketball players, there is no specific height (e.g. two meters) above which a player is considered tall and below which he is not considered tall. We say that such words are \textsc{vague}, meaning that the limits of their possible denotations cannot be precisely defined.\footnote{A number of authors (\citealt{Kempson1977}, \citealt{Lakoff1970}, \citealt{Tuggy1993}) have used the term \textsc{vagueness} as a cover term which includes generality or indeterminacy as a sub-type.}



\citet{Kennedy2011} mentions three distinguishing characteristics of vagueness. First, context-dependent truth conditions: we have already seen that a single individual may be truly said to be tall in one context (a gymnastics club) but not tall in another (a professional basketball team). This is not the case with indeterminacy; if a certain person is my cousin in one context, he or she will normally be my cousin in other contexts as well.



Second, vague predicates have borderline cases. Most people would probably agree that a bottle of wine costing two dollars is cheap, while one that costs five hundred dollars is expensive. But what about a bottle that costs fifty dollars? Most people would probably agree that Einstein was a genius, and that certain other individuals are clearly not. But there are extremely bright people about whom we might disagree when asked whether the term \textit{genius} can be applied to them; or we might simply say “I’m not sure”. Such borderline cases do not typically arise with indeterminacy; we do not usually disagree about whether a certain person is or is not our cousin.



\citet{Gillon1990} provides another example:


\begin{quote}
Vagueness is well exemplified by such words as \textit{city}. Though a definite answer does exist as to whether or not it applies to Montreal [1991 population: 1,016,376 within the city limits] or to Kingsville (Ontario) [1991 population: 5,716]; nonetheless, no definite answer exists as to whether or not it applies to Red Deer (Alberta) [1991 population: 58,145] or Moose Jaw (Saskatchewan) [1991 population: 33,593]. Nor is the lack of an answer here due to ignorance (at least if one is familiar with the geography of Western Canada): no amount of knowledge about Red Deer or Moose Jaw will settle whether or not \textit{city} applies. Any case in which further knowledge will settle whether or not the expression applies is simply not a case evincing the expression’s vagueness; rather it evinces the ignorance of its user… Vagueness is not alleviated by the growth of knowledge, ignorance is.
\end{quote}


Third, vague predicates give rise to “little-by-little” paradoxes.\footnote{The technical term is the \textit{sorites} paradox, also known as the paradox of the heap, the fallacy of the beard, the continuum fallacy, etc.} For example, Ringo Starr was clearly not bald in 1964; in fact, the Beatles’ famous haircut was an important part of their image during that era. Now if in 1964 Ringo had allowed you to pluck out one of his hairs as a souvenir, he would still not have been bald. It seems reasonable to assume that a man who is not bald can always lose one hair without becoming bald. But if Ringo had given permission for every person in Europe to pluck out one of his hairs, he would have become bald long before every fan was satisfied. But it would be impossible to say which specific hair it was whose loss caused him to become bald, because \textit{bald} is a vague predicate.



Another property which may distinguish vagueness from indeterminacy is the degree to which these properties are preserved in translation. Indeterminacy tends to be language-specific. There are many interesting and well-known cases where pairs of translation equivalents differ with respect to their degree of specificity. For example, Malay has no exact equivalent for the English words \textit{brother} and \textit{sister}. The language uses three terms for siblings: \textit{abang} ‘older brother’, \textit{kakak} ‘older sister’, and \textit{adek} ‘younger sibling’. The term \textit{adek} is indeterminate with respect to gender, while the English words \textit{brother} and \textit{sister} are indeterminate with respect to relative age.



Mandarin has several different and more specific words corresponding to the English word \textit{uncle}: [4F2F?][4F2F?] (bóbo) ‘father’s elder brother’; [53D4?][53D4?] (sh\=ushu) ‘father’s younger brother’; [59D1?][4E08?] (g\=uzhàng) ‘father’s sister’s husband’; [8205?][8205?] (jiùjiu) ‘mother’s brother’; [59E8?][4E08?] (yízhàng) ‘mother’s sister’s husband’.\footnote{\url{http://www.omniglot.com/language/kinship/chinese.htm}}  So the English word \textit{uncle} is indeterminate with respect to various factors that are lexically distinguished in Mandarin.



The English word \textit{carry} is indeterminate with respect to manner, but many other languages use different words for specific ways of carrying. Tzeltal, a Mayan language spoken in the State of Chiapas (Mexico), is reported to have twenty-five words for ‘carry’:\footnote{\url{http://www-01.sil.org/mexico/museo/3di-Carry.htm}} 


\ea
1. \textit{cuch} ‘carry on one’s back’\\
2. \textit{q'uech} ‘carry on one’s shoulder’ \\
3. \textit{pach} ‘carry on one’s head’ \\
4. \textit{cajnuc'tay} ‘carry over one’s shoulder’\\
5. \textit{lats'} ‘carry under one’s arm’\\
6. \textit{chup} ‘carry in one’s pocket’\\
7. \textit{tom} ‘carry in a bundle’\\
8. \textit{pet} ‘carry in one’s arms’\\
9. \textit{nol} ‘carry in one’s palm’\\
10. \textit{jelup'in} ‘carry across one’s shoulder’\\
11. \textit{nop'} ‘carry in one’s fist’\\
12. \textit{lat'} ‘carry on a plate’\\
13. \textit{lip'} ‘carry by the corner’\\
14. \textit{chuy} ‘carry in a bag’\\
15. \textit{lup} ‘carry in a spoon’\\
16. \textit{cats'} ‘carry between one’s teeth’\\
17. \textit{tuch} ‘carry upright’\\
18. \textit{toy} ‘carry holding up high’\\
19. \textit{lic} ‘carry dangling from the hand’\\
20. \textit{bal} ‘carry rolled up (like a map)’\\
21. \textit{ch'et} ‘carry coiled up (like a rope)’\\
22. \textit{chech} ‘carry by both sides’\\
23. \textit{lut'} ‘carry with tongs’\\
24. \textit{yom} ‘carry several things together’\\
25. \textit{pich'} ‘carry by the neck’
\z


In contrast, words which are vague in English tend to have translation equivalents in other languages which are also vague. This is because vagueness is associated with certain semantic classes of words, notably with scalar adjectives like \textit{big}, \textit{tall}, \textit{expensive}, etc. Vagueness is a particularly interesting and challenging problem for semantic analysis, and we will discuss it again in later chapters.


\subsection{Distinguishing ambiguity from vagueness and indeterminacy}\label{sec:} %3.2 /

The Spanish word \textit{llave} can be used to refer to things which would be called \textit{key}, \textit{faucet} or \textit{wrench}/\textit{spanner} in English.\footnote{Jonatan Cordova (p.c.) informs me that the word can also be used to mean ‘lock’ in wrestling.} How do we figure out whether \textit{llave} has multiple senses (i.e. is ambiguous), or whether it has a single sense that is vague or indeterminate? A number of linguistic tests have been proposed which can help us to make this decision. 



The most common tests are based on the principle that distinct senses of an ambiguous word are \textsc{antagonistic}.\footnote{\citet[61]{Cruse1986}.} This means that two senses of the word cannot both apply simultaneously. Sentences which seem to require two senses for a single use of a particular word, like those in \REF{ex:5.4}, are called \textsc{puns}. A clash or incompatibility of senses for a single word in sentences containing a co-ordinate structure, like those in \REF{ex:5.5}, is often referred to using the Greek term \textsc{zeugma} (pronounced ['zugm[259?]]). The odd or humorous nature of these kinds of sentences provides evidence that two distinct senses are involved; that is, evidence for a real lexical ambiguity.


\ea \label{ex:5.4}
\ea The hunter went home with five bucks in his pocket.\\
\ex The batteries were given out free of charge.\\
\ex I didn’t like my beard at first. Then it grew on me.\\
\ex When she saw her first strands of gray hair, she thought she’d dye.\\
\ex When the chair in the Philosophy Department became vacant,\\
  the Appointment Committee sat on it for six months.\footnote{\citet[108]{Cruse2000}.}
                       \z
\z

\ea \label{ex:5.5}
\ea Mary and her visa expired on the same day.\footnote{Adapted from \citet[61]{Cruse1986}.}\\
\ex He carried a strobe light and the responsibility for the lives of his men.\footnote{Tim O’Brien, \textit{The Things They Carried}, via grammar.about.com.}\\
\ex On his fishing trip, he caught three trout and a cold.\footnote{\url{http://dictionary.reference.com/browse/zeugma}}
                       \z
\z


Sentence (\ref{ex:5.4}d) illustrates a problem with English spelling, namely that words which are pronounced the same can be spelled differently (\textit{dye} vs. \textit{die}). Because linguistic analysis normally focuses on spoken rather than written language, we consider such word-forms to be ambiguous; we will discuss this issue further in the following section.



Another widely used test for antagonism between two senses is the \textsc{identity test}.\footnote{\citet{Lakoff1970}; \citet{ZwickySadock1975}.} This test makes use of the fact that certain kinds of ellipsis require parallel interpretations for the deleted material and its antecedent. We will illustrate the test first with an instance of structural ambiguity:\footnote{Examples adapted from \citet[512]{Kennedy2011}.}


\ea \label{ex:5.6}
\ea The fish is ready to eat.\\
\ex The fish is ready to eat, and so is the chicken.\\
\ex The fish is ready to eat, but the chicken is not.\\
\ex \#The potatoes are ready to eat, but the children are not.
                       \z
\z


Sentence (\ref{ex:5.6}a) is structurally ambiguous: the fish can be interpreted as either the agent or the patient of \textit{eat}. Both of the clauses in example (\ref{ex:5.6}b) are ambiguous in the same way. This predicts that there should be four logically possible interpretations of this sentence; but in fact only two are acceptable to most English speakers. If the fish is interpreted as an agent, then the chicken must be interpreted as an agent; if the fish is interpreted as a patient, then the chicken must be interpreted as a patient. The parallelism constraint rules out readings where the fish is the eater while the chicken is eaten, or vice versa. The same holds true for example (\ref{ex:5.6}c). Sentence (\ref{ex:5.6}d) is odd because the nouns used strongly favor different interpretations for the two clauses: potatoes must be the patient, while children must be the agent, violating the parallelism constraint.



Example \REF{ex:5.7} illustrates the use of the identity test with an apparent case of lexical ambiguity: \textit{duck} can refer to an action (lowering the head or upper body) or to a water fowl. Sentence (\ref{ex:5.7}a) is ambiguous, because the two senses of \textit{duck} generate two different readings, and one of these readings could be true while the other was false in a particular situation. The same potential ambiguity applies to both of the clauses in (\ref{ex:5.7}b), so again we would predict that four interpretations should be logically possible; but in fact only two are acceptable. Sentence (\ref{ex:5.7}b) can mean either that John and Bill both saw her perform a certain action or that they both saw a water fowl belonging to her. The fact that the parallelism constraint blocks the “crossed” readings provides evidence that these two different interpretations of \textit{duck} are truly distinct senses, i.e. that \textit{duck} is in fact lexically ambiguous.


\ea \label{ex:5.7}
\ea John saw her duck.\\
\ex John saw her duck, and so did Bill.
\z
                       \z


Contrast this with the examples in \REF{ex:5.8}. The word \textit{cousin} in the first clause of (\ref{ex:5.8}a) refers to a male person, while the implicit reference to \textit{cousin} in the second clause of (\ref{ex:5.8}a) refers to a female person. This difference of reference does not violate the parallelism constraint, because the two uses of \textit{cousin} are not distinct senses, even though they would be translated by different words in a language like Italian. The identity test indicates that \textit{cousin} is not lexically ambiguous, but merely unspecified for gender.


\ea \label{ex:5.8}
\ea John is my cousin, and so is Mary.\\
\ex John carried a briefcase, and Bill a backpack.\\
\ex That 3-year old is quite tall, but then so is his father.
                       \z
\z


Similarly, the word \textit{carry} in the first clause of (\ref{ex:5.8}b) probably describes a different action from the implicit reference to \textit{carry} in the second clause. The sentence allows an interpretation under which John carried the briefcase by holding it at his side with one hand, while Bill carried the backpack on his back; in fact, this would be the most likely interpretation in most contexts. The fact that this interpretation is not blocked by the parallelism constraint indicates that \textit{carry} is not lexically ambiguous, but merely unspecified (i.e., indeterminate) for manner. The two uses of \textit{carry} would be translated by different words in a language like Tzeltal, but they are not distinct senses.



The actual height described by the word \textit{tall} in the first clause of (\ref{ex:5.8}c) is presumably much less than the height described by the implicit reference to \textit{tall} in the second clause. The fact that this interpretation is acceptable indicates that \textit{tall} is not lexically ambiguous, but merely vague.



Example \REF{ex:5.9} shows how we might use the identity test to investigate the ambiguity of the Spanish word \textit{llave} mentioned above. These sentences could appropriately be used if both Pedro and Juan bought, broke or found the same kind of thing, whether keys, faucets, or wrenches. But the sentences cannot naturally describe a situation where different objects are involved, e.g. if Pedro bought a key but Juan bought a wrench, etc.\footnote{Jonatan Cordova, Steve and Monica Parker (p.c.).} This fact provides evidence that \textit{llave} is truly ambiguous and not merely indeterminate or vague.


\ea \label{ex:5.9}
\ea  \gll Pedro  compró/rompió  una  llave  y  también  Juan.\\
Pedro  bought/broke  a  key/etc.  and  also  Juan\\
\glt ‘Pedro bought/broke a key/faucet/wrench, and so did Juan.’
\ex \gll  Pedro  encontró  una  llave  al  igual  que  Juan.\\
Pedro  found  a  key/etc.  to.the  same  that  Juan\\
\glt ‘Pedro found a key/faucet/wrench, just like Juan did.’
\z \z


Another test which is sometimes used is the \textsc{sense relations test}: distinct senses will have different sets of synonyms, antonyms, etc. (see discussion of sense relations in \chapref{sec:6}). For example, the word \textit{light} has two distinct senses; one is the opposite of \textit{heavy}, the other is the opposite of \textit{dark}. However, \citet[56-57]{Cruse1986} warns that this test is not always reliable, because contextual features may restrict the range of possible synonyms or antonyms for a particular use of a word which is merely vague or indeterminate.



Another kind of evidence for lexical ambiguity is provided by the \textsc{test of contradiction}.\footnote{\citet{Quine1960}; \citet{ZwickySadock1975}; \citet{Kennedy2011}.} If a sentence of the form \textit{X but not X} can be true (i.e., not a contradiction), then expression X must be ambiguous. For example, the fact that the statement in \REF{ex:5.10} is not felt to be a contradiction provides good evidence for the claim that the two uses of \textit{child} represented here (‘offspring’ vs. ‘pre-adolescent human’) are truly distinct senses.


\ea \label{ex:5.10}
(Aged mother discussing her grown sons and daughters)\\
\textit{They are not children any more, but they are still my children}.
\z


This is an excellent test in some ways, because the essential property of ambiguity is that the two senses must have different truth conditions, and this test involves asserting one reading while simultaneously denying the other. In many cases, however, it can be difficult to find contexts in which such sentences sound truly natural. A few attempts at creating such examples are presented in \REF{ex:5.11}. The fact that such sentences are even possible provides strong evidence for two distinct senses of the relevant word.


\ea \label{ex:5.11}
\ea  Criminal mastermind planning to stage a traffic accident in order to cheat the insurance company: \textit{After the crash, you lie down behind the bus and tell the police you were thrown out of the bus through a window}.\\
Unwilling accomplice: \textit{I’ll lie there, but I won’t lie}.
\ex   Foreman: \textit{I told you to collect a sample of uranium ore from the pit and row it across the river to be tested}.\\
Miner: \textit{I have the ore but I don’t have the oar}.
\ex   Rancher (speaking on the telephone): \textit{I’ve lost my expensive fountain pen; I think I may have dropped it while we were inspecting the sheep. Can you check the sheep pen to see if it is there?}\\
Hired hand: \textit{I am looking at the pen, but I don’t see a pen}.
\z \z


An equivalent way of describing this test is to say that if there exists some state of affairs or context in which a sentence can be both truly affirmed and truly denied, then the sentence must be ambiguous.\footnote{Adapted from \citet[407]{Gillon1990}.} An example showing how this test might be applied to two uses of the word \textit{drink} (alcoholic beverage vs. any beverage) is quoted in \REF{ex:5.12}:


\ea \label{ex:5.12}
\ea \textit{Ferrell has a drink each night before going to bed}.
\ex  “Imagine… this state of affairs: Ferrell has a medical problem which requires that he consume no alcoholic beverages but that he have a glass of water each night before going to bed. One person knows only that he does not consume alcoholic beverages; another knows only that he has a glass of water each night at bedtime. The latter person can truly affirm the sentence in (12a)… But the former person can truly deny it.” (\citealt{Gillon1990}:407)
\z \z


Gillon points out that this is a very useful test because “generality and indeterminacy do not permit a sentence to be both truly affirmed and truly denied” (1990: 410). Sentences like those in \REF{ex:5.13} can only be interpreted as contradictions; they require some kind of pragmatic inference in order to make sense.\footnote{The word \textit{vertebrate} is more “general”, in Gillon’s terms, than words like \textit{fish} or \textit{dog}. We will discuss this kind of sense relation in the following chapter.}


\ea \label{ex:5.13}
\ea[\#]{She is my cousin and she is not my cousin.\\}
\ex[\#]{I am carrying the bag and I am not carrying the bag.\\}
\ex[\#]{This creature is a vertebrate and it is not a vertebrate.}
                       \z
\z

\subsection{Polysemy vs. homonymy}\label{sec:} %3.3 /

It is traditional to distinguish between two types of lexical ambiguity, \textsc{polysemy} (one word with multiple senses) vs. \textsc{homonymy} (different words that happen to sound the same). Both cases involve an ambiguous word form; the difference lies in how the information is organized in the speaker’s mental lexicon.


Of course, it is not easy to determine how information is stored in the mental lexicon. This is not something that native speakers are consciously aware of, so asking them directly whether two senses are “the same word” or not is generally not a reliable procedure. The basic criterion for making this distinction is that in cases of polysemy, the two senses are felt to be “related” in some way; there is “an intelligible connection of some sort” between the two senses.\footnote{\citet[109]{Cruse2000}.} In cases of homonymy, the two senses are unrelated; that is, the semantic relationship between the two senses is similar to that between any two words selected at random.



It is difficult to draw a clear boundary between these two types of ambiguity, and some authors reject the distinction entirely. However, many ambiguous words clearly belong to one type or the other, and the distinction is a useful one. We will adopt a prototype approach, suggesting some properties that are prototypical of polysemy vs. homonymy while recognizing there will be cases which are very difficult to classify.


Some general guidelines for distinguishing polysemy vs. homonymy:

\begin{enumerate}[label=\alph*.]
\item Two senses of a polysemous word generally share at least one salient feature or component of meaning, whereas this need not be true for homonyms.\footnote{\citet{BeekmanCallow1974} suggest that \textit{all} the senses of a polysemous word will share at least one component of meaning, but this claim is certainly too strong.} For example, the sense of \textit{foot} that denotes a unit of length (‘12 inches’) shares with the body-part sense the same approximate size. The sense of \textit{foot} that means ‘base’ (as in \textit{foot of} \textit{a tree/mountain}) shares with the body-part sense the same position or location relative to the object of which it is a part. These common features suggest that \textit{foot} is polysemous. In contrast, the two senses of \textit{row} (pull the oars vs. things arranged in a line) seem to have nothing in common, suggesting that \textit{row} is homonymous.
\item If one sense seems to be a figurative extension of the other (see discussion of figurative senses below), the word is probably polysemous. For example, the sense of \textit{run} in \textit{This road runs from Rangoon to Mandalay} is arguably based on a metonymy between the act of running and the path traversed by the runner, suggesting that this is a case of polysemy.
\item \citet{BeekmanCallow1974} suggest that, for polysemous words, one sense can often be identified as the \textsc{primary sense}, with other senses being classified as secondary or figurative. The primary sense will typically be the one most likely to be chosen if you ask a native speaker to illustrate how the word X is used in a sentence, or if you ask a bilingual speaker what the word X means (i.e., ask for a translation equivalent). For homonymous words, neither sense is likely to be “primary” in this way.\footnote{A similar point is made by Fillmore \& \citet[100]{Atkins2000}.}
\item Etymology (historical source) is used as a criterion in most dictionaries, but for synchronic linguistic analysis it is not a reliable basis for analysis. (Speakers may or may not know where certain words come from historically, and their ideas about such questions are often mistaken.) However, there is often a correlation between etymology and the criteria listed above, because figurative extension is a common factor in semantic change over time, as discussed in \sectref{sec:4}. English spelling may give a clue about etymology, but again is not directly relevant to synchronic linguistic analysis, which normally focuses on spoken language.
\end{enumerate}

Point (d) is a specific application of a more general principle in the study of lexical meaning: word meanings may change over time, and the historical meaning of a word may be quite different from its modern meaning. It is important to base our analysis of the current meanings of words on \textsc{synchronic} (i.e., contemporaneous) evidence, unless we are specifically studying the \textsc{diachronic} (historical) developments. \citet[244]{Lyons1977} expresses this principle as follows:


\begin{quote}
A particular manifestation of the failure to respect the distinction of the diachronic and the synchronic in semantics … is what might be called the \textsc{etymological fallacy}: the common belief that the meaning of words can be determined by investigating their origins. The etymology of a lexeme is, in principle, synchronically irrelevant.
\end{quote}


As an example, Lyons points out that it would be silly to claim that the “real” meaning of the word \textit{curious} in Modern English is ‘careful’, even though that was the meaning of the Latin word from which it is derived.



A number of authors have distinguished between \textsc{regular} or \textsc{systematic} polysemy vs. non-systematic polysemy. Systematic polysemy involves senses which are related in recurring or predictable ways. For example, many verbs naming a change of state (\textit{break, melt, split}, etc.) have two senses, one transitive (V\textsc{\textsubscript{tr}}) and the other intransitive (V\textsc{\textsubscript{intr}}), with V\textsc{\textsubscript{tr}} meaning roughly ‘cause to V\textsc{\textsubscript{intr}}’. Similarly, many nouns that refer to things used as instruments (\textit{hammer, saw, paddle, whip, brush, comb, rake, shovel, plow, sandpaper, anchor, tape, chain, telephone}, etc.) can also be used as verbs meaning roughly to use the instrument to act on an appropriate object. (A single sense can have only a single part of speech, so the verbal and nominal uses of such words must represent distinct senses.)



The kinds of regularities involved in systematic polysemy can be stated in the form of rules. Some authors have suggested that only the base or core meaning needs to be included in the lexicon, because the secondary senses can be derived by rule.\footnote{For example, \citet{Pustejovsky1995}.} But even in the case of systematic polysemy, secondary senses need to be listed because not every extended sense which the rules would license actually occurs in the language. For example, there are no verbal uses for some instrumental nouns, e.g. \textit{scalpel, yardstick, hatchet, pliers, tweezers}, etc. For others, verbal uses are possible only for non-standard uses of the instrument or non-literal senses:


\ea \label{ex:5.14}
\ea Australian Prime Minister Kevin Rudd has \textit{axed} the carbon tax.\\
\ex Alaska Airlines \textit{axed} the flights as a precaution.\\
\ex ?*John \textit{axed} the tree.
\z \z


Traditionally it has been assumed that all the senses of a polysemous word will be listed within a single lexical entry, while homonyms will occur in separate lexical entries. Most dictionaries adopt a format that reflects this organization of the lexicon. The format is illustrated in the partial dictionary listing for the word form \textit{lean} presented in \REF{ex:5.15}.\footnote{Adapted from the Merriam-Webster Online Dictionary (\url{http://www.merriam-webster.com/dictionary/lean} ).} The verbal and adjectival uses of \textit{lean} are treated as homonyms, each with its own lexical entry. Each of the homonyms is analyzed as being polysemous, with the various senses listed inside the appropriate entry.


\ea \label{ex:5.15}
\textit{lean}\textsubscript{1} (V): 1. to incline, deviate, or bend from a vertical position; 2. to cast one’s weight to one side for support ; 3. to rely on for support or inspiration; 4. to incline in opinion, taste, or desire (e.g., \textit{leaning toward a career in chemistry}).\\[1.25\baselineskip]

\textit{lean}\textsubscript{2} (Adj): 1. lacking or deficient in flesh; 2. containing little or no fat (\textit{lean meat}); 3. lacking richness, sufficiency, or productiveness (\textit{lean profits}, \textit{the lean years}); 4. deficient in an essential or important quality or ingredient, e.g. (a) of ore: containing little valuable mineral; (b) of fuel mixtures: low in combustible component.
\z


This is not the only way in which a lexicon could be organized, but we will not explore the various alternatives here. The crucial point is that polysemous senses are “related” while homonymous senses are not.


\subsection{One sense at a time}\label{sec:} %3.4 /

When a lexically ambiguous word is used, the context normally makes it clear which of the senses is intended. As \citet[53]{Cruse1986} points out, a speaker generally intends the hearer to be able to identify the single intended sense based on context:

\begin{quote}
“[A] context normally also acts in such a way as to cause a single sense, from among those associated with any ambiguous word form, to become operative. When a sentence is uttered, it is rarely the utterer’s intention that it should be interpreted in two (or more) different ways simultaneously… This means that, for the vast majority of utterances, hearers are expected to identify specific intended senses for every ambiguous word form that they contain.”
\end{quote}


\citet[54]{Cruse1986} cites the sentence in \REF{ex:5.16}, which contains five lexically ambiguous words. (Note that the intended sense of \textit{burn} in this sentence, ‘a small stream’, is characteristic of Scottish English.)


\ea \label{ex:5.16}
Several rare ferns grow on the steep banks of the burn where it runs into the lake.
\z

Cruse writes,

\begin{quote}
In such cases, there will occur a kind of mutual negotiation between the various options [so as to determine which sense for each word produces a coherent meaning for the sentence as a whole]… It is highly unlikely that any reader of this sentence will interpret \textit{rare} in the sense of ‘undercooked’ (as in \textit{rare steak}), or \textit{steep} in the sense of ‘unjustifiably high’ (as in \textit{steep charges})… or \textit{run} in the sense of ‘progress by advancing each foot alternately never having both feet on the ground simultaneously’, etc.
\end{quote}


A very interesting use of this principle occurs in the short story “Xingu”, by Edith \citet{Wharton1916}. In the following passage, Mrs. Roby is describing something to the members of her ladies’ club, which they believe (and which she allows them to believe) to be a deep, philosophical book. After the discussion is over, however, the other members discover that she was actually describing a river in Brazil. The words which are underlined below are ambiguous; all of them must be interpreted with one sense in a discussion of a philosophical work, but another sense in a discussion of a river.

\todo[inline]{underlining emphasis gone missing here}
\ea \label{ex:5.17}
“Of course,” Mrs. Roby admitted, “the difficulty is that one must give up so much time to it. It’s very long.”\\
“I can’t imagine,” said Miss Van Vluyck tartly, “grudging the time given to such a subject.”\\
“And deep in places,” Mrs. Roby pursued; (so then it was a book!) “And it isn’t easy to skip.”\\
“I never skip,” said Mrs. Plinth dogmatically.\\
“Ah, it’s dangerous to, in Xingu. Even at the start there are places where one can’t. One must just wade through.”\\
“I should hardly call it wading ,” said Mrs. Ballinger sarcastically.\\
Mrs. Roby sent her a look of interest. “Ah — you always found it went swimmingly?”\\
Mrs. Ballinger hesitated. “Of course there are difficult passages,” she conceded modestly.\\
“Yes; some are not at all clear — even,” Mrs. Roby added, “if one is familiar with the original.\footnote{Apparently a play upon an archaic sense of \textit{original} meaning ‘source’ or ‘origin’.}”\\
“As I suppose you are?” Osric Dane interposed, suddenly fixing her with a look of challenge.\\
Mrs. Roby met it by a deprecating smile. “Oh, it’s really not difficult up to a certain point; though some of the branches are very little known, and it’s almost impossible to get at the source.”
\z

Mrs. Roby’s motives seem to be noble — she is rescuing the ladies of the club from further humiliation by an arrogant visiting celebrity, Mrs. Osric Dane (a popular author). But when the other members discover the deception, they are so provoked that they demand Mrs. Roby’s resignation.

Cotterell \& \citet[175]{Turner1989} point out the implications of the “one sense at a time” principle for exegetical work:

\begin{quote}
The context of the utterance usually singles out … the \textit{one} sense, which is intended, from amongst the various senses of which the word is potentially capable…  When an interpreter tells us his author could be using such-and-such a word with sense \textit{a}, or he could be using it with sense \textit{b}, and then sits on the fence claiming perhaps the author means \textit{both}, we should not too easily be discouraged from the suspicion that the interpreter is simply fudging the exegesis.
\end{quote}


Sometimes, of course, the speaker does intend both senses to be available to the hearer; but this is normally intended as some kind of play on words, e.g. a pun. The humor in a pun (for those people who enjoy them) lies precisely in the fact that this is not the way language is normally used.


\subsection{Disambiguation in context}\label{sec:} %3.5 /

Word meanings are clarified or restricted by their context of use in several different ways. If a word is indeterminate with respect to a certain feature, the feature can be specified by linguistic or pragmatic context. For example, the word \textit{nurse} is indeterminate with respect to gender; but if I say \textit{The nurse who checked my blood pressure was pregnant}, the context makes it clear that the nurse I am referring to is female.



We noted in the preceding section that the context of use generally makes it clear which sense of a lexically ambiguous word is intended. This is not to say that misunderstandings never arise, but in a large majority of cases hearers filter out unintended senses automatically and unconsciously. It is important to recognize that knowledge about the world plays an important role in making this disambiguation possible. For example, a slogan on the package of Wasa crispbread proudly announces, \textit{Baked since 1919}. There is a potential ambiguity in the aspect of the past participle here. It is our knowledge about the world (and specifically about how long breads and crackers can safely be left in the oven), rather than any feature of the linguistic context, which enables us to correctly select the habitual, rather than the durative, reading. The process is automatic; most people who see the slogan are probably not even aware of the ambiguity.



Because knowledge about the world plays such an important role, disambiguation will be more difficult with translated material, or in other situations where the content is culturally unfamiliar to the reader/hearer. But in most monocultural settings, Ravin \& Leacock’s (2000) assessment seems fair:


\begin{quote}
Polysemy is rarely a problem for communication among people. We are so adept at using contextual cues that we select the appropriate senses of words effortlessly and unconsciously… Although rarely a problem in language use, except as a source of humour and puns, polysemy poses a problem for semantic theory and in semantic applications, such as translation or lexicography.
\end{quote}


If lexical ambiguity is not (usually) a problem for human speakers, it is a significant problem for computers. Much of the recent work on polysemy has been carried out within the field of computational linguistics. Because computational work typically deals with written language, more attention has been paid to \textsc{homographs} (words which are spelled the same) than to \textsc{homophones} (words which are pronounced the same), in contrast to traditional linguistics which has been more concerned with spoken language. Because of English spelling inconsistencies, the two cases do not always coincide; Ravin \& Leacock cite the example of \textit{bass} [bæs] ‘fish species’ vs. \textit{bass} [be\textsuperscript{j}s] ‘voice or instrument with lowest range’, homographs which are not homophones.



As Ravin \& Leacock note, lexical ambiguity poses a problem for translation. The problem arises because distinct senses of a given word-form are unlikely to have the same translation equivalent in another language. Lexical ambiguity can cause problems for translation in at least two ways: either the wrong sense may be chosen for a word which is ambiguous in the source language, or the nearest translation equivalent for some word in the source language may be ambiguous in the target language. In the latter case, the translated version may be ambiguous in a way that the original version was not.



A striking example of the former type occurred on the menu of a Chinese restaurant which offered ‘fried enema’ rather than ‘fried sausage’. The Chinese name of the dish is \textit{zhá guànchang} (\texttt{[70B8?][704C?][8178?]}). The last two characters in the name refer to a kind of sausage made of wheat flour stuffed into hog casings; but they also have another sense, namely ‘enema’.\footnote{\url{http://languagelog.ldc.upenn.edu/nll/?p=2236}} 



Much medieval and renaissance art, most famously the sculptural masterpiece by Michelangelo, depicts Moses with horns coming out of his forehead. This practice was based on the Latin Vulgate translation of a passage in Exodus which describes Moses’ appearance when he came down from Mt. Sinai.\footnote{Exodus 34:29-35.} The Hebrew text uses the verb \textit{qaran} to describe his face. This verb is derived from the noun \textit{qeren} meaning ‘horn’, and in some contexts it can mean ‘having horns’;\footnote{Psalm 69:31.} but most translators, both ancient and modern, have agreed that in this context it has another sense, namely ‘shining, radiant’ or ‘emitting rays’. St. Jerome, however, translated \textit{qaran} with the Latin adjective \textit{cornuta} ‘horned’.\footnote{There is some disagreement as to whether St. Jerome simply made a mistake, or whether he viewed the reference to horns as a live metaphor and chose to preserve the image in his translation. The first artistic depiction of a horned Moses appeared roughly 700 years after Jerome’s translation, which might be taken as an indication that the metaphorical sense was in fact understood by readers of the Vulgate at first, but was lost over time. (see Ruth Mellinkoff. 1970. \textit{The Horned Moses in Medieval Art and Thought} (California Studies in the History of Art, 14). University of California Press.)}



As noted above, a translation equivalent which is ambiguous in the target language can create ambiguity in the translated version that is not present in the original. For example, the French word \textit{apprivoiser} ‘to tame’ plays a major role in the book \textit{Le Petit Prince} ‘The Little Prince’ by Antoine de Saint-Exupéry. In most (if not all) Portuguese versions this word is translated as \textit{cativar}, which can mean ‘tame’ but can also mean ‘catch’, ‘capture’, ‘enslave’, ‘captivate’, ‘enthrall’, ‘charm’, etc. This means that the translation is potentially ambiguous in a way that the original is not. The first occurrence of the word is spoken by a fox, who explains to the little prince what the word means; so in that context the intended sense is clear. However, the word occurs frequently in the book, and many of the later occurrences might be difficult for readers to disambiguate on the basis of the immediate context alone.



It is not surprising that homonymy should pose a problem for translation, because homonymy is an accidental similarity of form; there is no reason to expect the two senses to be associated with a single form in another language. If we do happen to find a pair of homonyms in some other language which are good translation equivalents for a pair of English homonyms, we regard it as a remarkable coincidence. But even with polysemy, where the senses are related in some way, we cannot in general expect that the different senses can be translated using the same word in the target language. Beekman \& \citet[103]{Callow1974} state:


\begin{quote}
Whether multiple senses of a word arise from a shared [component] of meaning or from relations which associate the senses [i.e. figurative extensions—PK], the cluster of senses symbolized by a single word is always specific to the language under study.
\end{quote}


Perhaps Beekman \& Callow overstate the unlikelihood that a single word in the target language can carry some or all of the senses of a polysemous word in the source language. Since there is an intelligible relationship between polysemous senses, it is certainly possible for the same relationship to be found in more than one language; but often this turns out not to be the case, and that is why polysemy is often a source of problems.


\section{Context-dependent extensions of meaning}\label{sec:} %4. /

\citet{Cruse1986,Cruse2000} distinguishes between \textsc{established} vs. \textsc{non-established} senses. An established sense is one that is permanently stored in the speaker’s mental lexicon, one which is always available; these are the senses that would normally be listed in a dictionary. A lexically ambiguous word is one that has two or more established senses.



We have seen how context determines a choice between existing (i.e., established) senses of lexically ambiguous words. But context can also force the hearer to “invent” a new, non-established sense for a word. When Mark Twain described a certain person as “a good man in the worst sense of the word,” his hearers were forced to interpret the word \textit{good} with something close to the opposite of its normal meaning (e.g., puritanical, self-righteous, or judgmental). Clearly this “sense” of the word \textit{good} is not permanently stored in the hearer’s mental lexicon, and we would not expect to see it listed in a dictionary entry for \textit{good}. It exists only on the occasion of its use in this specific context.



A general term for the process by which context creates non-established senses is \textsc{coercion}.\footnote{This term was coined by \citet{MoensSteedman1988}.} Coercion provides a mechanism for extending the range of meanings of a given word. It is motivated by the assumption that the speaker intends to communicate something intelligible, relevant to current purposes, etc. If none of the established senses of a word allow for a coherent or intelligible sentence meaning, the hearer tries to create an extended meaning for one or more words that makes sense in the current speech context.



Coerced meanings are not stored in the lexicon, but are calculated as needed from the established or default meaning of the word plus contextual factors; so there is generally some identifiable relationship between the basic and extended senses. Several common patterns of extended meaning were identified and named by ancient Greek philosophers; these are often referred to as \textsc{tropes}, or “figures of speech”.


\subsection{Figurative senses}\label{sec:} %4.1 /

Some of the best-known figures of speech are listed in \REF{ex:5.18}:

\ea \label{ex:5.18}
\textbf{Some well-known tropes}\\
\begin{description}
\item[Metaphor:] traditionally defined as a figure of speech in which an implied comparison is made between two unlike things; but see comments below.
\item[Hyperbole:] A figure of speech in which exaggeration is used for emphasis or effect; an extravagant statement. (e.g., \textit{I have eaten more salt than you have eaten rice}. — Chinese saying implying seniority in age and wisdom)
\item[Euphemism:] Substitution of an inoffensive term (such as \textit{passed away}) for one considered offensively explicit (\textit{died}).
\item[Metonymy:] A figure of speech in which one word or phrase is substituted for another with which it is closely associated (such as \textit{crown} for \textit{monarch}).
\item[Synecdoche (/sɪˈnɛk də ki/):] A figure of speech in which a part is used to represent the whole, the whole for a part, the specific for the general, the general for the specific, or the material for the thing made from it. Considered by some to be a form of metonymy.
\item[Litotes:] A figure of speech consisting of an understatement in which an affirmative is expressed by negating its opposite (e.g. \textit{not bad} to mean ‘good’).
\item[Irony:] “statements that imply a meaning in opposition to their literal meaning”.\footnote{Wikipedia.}
\end{description}
\z


The question of how metaphors work has generated an enormous body of literature, and remains a topic of controversy. For our present purposes, it is enough to recognize all of these figures of speech as patterns of reasoning that will allow a hearer to provide an extended sense when all available established senses fail to produce an acceptable interpretation of the speaker’s utterance.


\subsection{How figurative senses become established}\label{sec:} %4.2 /

As mentioned above, figurative senses are not stored in the speaker/hearer’s mental lexicon; rather, they are calculated as needed, when required by the context of use. However, some figurative senses become popular, and after frequent repetition they lose the sense of freshness or novelty associated with their original use; we call such expressions “clichés”. At this stage they are remembered, rather than calculated, but are perhaps not stored in the lexicon in the same way as “normal” lexical items; they are still felt to be figurative rather than established senses. Probable examples of this type include: \textit{fishing for compliments}, \textit{sowing seeds of doubt}, \textit{at the end of the day}, \textit{burning the candle at both ends, boots on the ground, lash out,} …


At some point, these frequently used figurative senses may become lexicalized, and begin to function as established senses. For example, the original sense of \textit{grasp} is ‘to hold in the hand’; but a new sense has developed from a metaphorical use of the word to mean ‘understand’. Similar examples include \textit{freeze} ‘become ice’ > ‘remain motionless’; \textit{broadcast} ‘plant (seeds) by scattering widely’ > ‘transmit via radio or television’; and, more recently, the use of \textit{hawk} and \textit{dove} to refer to advocates of war and advocates of peace, respectively. Once this stage is reached, the hearer does not have to calculate the speaker’s intended meaning based on specific contextual or cultural factors; the intended meaning is simply selected from among the established senses already available, as with normal cases of lexical ambiguity.



When established senses develop out of metaphors, they are referred to as \textsc{conventional metaphors}, in contrast to “novel” or “creative” metaphors which are newly created. Conventional metaphors are sometimes referred to as “dead” or “frozen” metaphors, phrases which are themselves conventional metaphors expressing the intuition that the meaning of such expressions is static rather than dynamic.



Finally, in some cases the original “literal” sense of a word may fall out of use, leaving what was originally a figurative sense as the only sense of that word. This seems to be happening with the compound noun \textit{night owl}, which originally referred to a type of bird. Many current dictionaries (including the massive \textit{Random House Unabridged}) now list only the conventional metaphor sense, i.e., a person who habitually stays out late at night. (** more examples?? **)



This discussion shows how figurative senses may lead to polysemy. Earlier we noted that translation equivalents in different languages are unlikely to share the same range of polysemous senses. For example, the closest translation equivalent for \textit{grasp} in Malay is \textit{pĕgang}; but this verb never carries the sense of ‘understand’. Novel (i.e., creative) metaphors can sometimes survive and be interpretable when translated into a different language, because the general patterns of meaning extension listed in \REF{ex:5.18}, if they are not universal, are at least used across a wide range of languages. Conventional (i.e., “frozen”) metaphors, however, are much less likely to work in translation, because the specific contextual features which motivated the creative use of the metaphor need no longer be present. 


\section{“Facets” of meaning}\label{sec:} %5. /

The sentences in (\ref{ex:5.19}--\ref{ex:5.22}) show examples of different uses which are possible for certain classes of words. These different uses are often cited as cases of systematic polysemy, i.e., distinct senses related by a productive rule of some kind.\footnote{See for example \citet{Pustejovsky1995}, \citet{NunbergZaenen1992}.} However, \citet{Cruse2000,Cruse2004} argues that they are best analyzed as “facets” of a single sense, by which he means “fully discrete but non-antagonistic readings of a word”.\footnote{\citet[116]{Cruse2000}.}

\settowidth\jamwidth{[\textsc{information content}]}
\ea  \label{ex:5.19}
\textit{book} \citep{Cruse2004}:\\
\ea My chemistry book makes a great doorstop.            \jambox{[\textsc{physical object}]}
\ex My chemistry book is well-organized but a bit dull.  \jambox{[\textsc{information content}]}
                       \z
\z

\ea \label{ex:5.20}
 \textit{bank} (\citealt{Cruse2000}:116; similar examples include \textit{school, university}, etc.):\\
\ea The bank in the High Street was blown up last night.  \jambox{[\textsc{premises}]}
\ex That used to be the friendliest bank in town.         \jambox{[\textsc{personnel}]}
\ex This bank was founded in 1575.                        \jambox{[\textsc{institution}]}
                       \z
\z

\ea \label{ex:5.21} \textit{Britain} (\citealt{Cruse2000}:117; \citealt{CroftCruse2004}:117):\\
\ea Britain lies under one metre of snow.                    \jambox{[\textsc{land mass}]}
\ex Britain today is mourning the death of the Royal corgi.  \jambox{[\textsc{populace}]}
\ex Britain has declared war on San Marino.                  \jambox{[\textsc{political entity}]}
                       \z
\z

\ea \label{ex:5.22}
\textit{chicken, duck}, etc. (\citealt{CroftCruse2004}:117):\\
\ea My neighbor’s chickens are noisy and smelly.  \jambox{[\textsc{animal}]}
\ex This chicken is tender and delicious.         \jambox{[\textsc{meat}]}
                       \z
\z


Cruse describes facets as “distinguishable components of a global whole”.\footnote{Croft \& \citet[116]{Cruse2004}.} The word \textit{book} for example names a complex concept which includes both the physical object (the tome) and the information which it contains (the text). In the most typical uses of the word, it is used to refer to both the object and its information content simultaneously. In contexts like those seen in \REF{ex:5.19}, however, the word can be used to refer to just one facet or the other (text or tome).



Cruse’s strongest argument against the systematic polysemy analysis is the fact that these facets are non-antagonistic; they do not give rise to zeugma effects, as illustrated in \REF{ex:5.23}. In this they are unlike normal polysemous senses, which are antagonistic. Under the systematic polysemy analysis we might derive the senses illustrated in (\ref{ex:5.19}--\ref{ex:5.22}) by a kind of metonymy, similar to that illustrated in \REF{ex:5.24}.\footnote{\citet{Nunberg1979,Nunberg1995}.} However, as the examples in \REF{ex:5.25} demonstrate, figurative senses are antagonistic with their literal counterparts. This suggests that facets are not figurative senses.


\ea \label{ex:5.23}
\ea This is a very interesting book, but it is awfully heavy to carry around. (\citealt{Cruse2004})\\
\ex My religion forbids me to eat or wear rabbit.  [\citealt{NunbergZaenen1992}]
                       \z
\z

\ea \label{ex:5.24}
\ea I’m parked out back.\\
\ex The ham sandwich at table seven left without paying.\\
\ex Yeats is widely read although he has been dead for over 50 years.\\
\ex Yeats is widely read, even though most of it is now out of print.
                       \z
\z

\ea \label{ex:5.25}
\ea[\#]{The ham sandwich at table seven was stale and left without paying.\\}
\ex[\#]{The White House needs a coat of paint but refuses to ask Congress for the money.}
                       \z
\z


We cannot pursue a detailed discussion of these issues here. It may be that some of the examples in question are best treated in one way, and some in the other. The different uses of animal names illustrated in \REF{ex:5.22}, for example, creature vs. meat, seem like good candidates for systematic polysemy, because they differ in grammatical properties (mass vs. count nouns). But the non-antagonism of the other cases seems to be a problem for the systematic polysemy analysis.


\section{6. Conclusion}\label{sec:}

In this chapter we described several ways of identifying lexical ambiguity, based on two basic facts. First, distinct senses of a single word are “antagonistic”, and as a result only one sense is available at a time in normal usage. The incompatibility of distinct senses can be observed in puns, in zeugma effects, and in the identity requirements under ellipsis. Second, true ambiguity involves a difference in truth conditions; so sentences which contain an ambiguous word can sometimes be truly asserted under one sense of that word and denied under the other sense, in the same context. Neither of these facts applies to vagueness or indeterminacy.



Lexical ambiguity is actually quite common, but only rarely causes confusion between speaker and hearer. The hearer is normally able to identify the intended sense for an ambiguous word based on the context in which it is used. Where none of the established senses lead to a sensible interpretation in a given context, new senses can be triggered by coercion. In \chapref{sec:8} we will discuss some of the pragmatic principles which guide the hearer in working out the intended sense.



\furtherreading



\citet{Kennedy2011} provides an excellent overview of lexical ambiguity, indeterminacy, and vagueness. These issues are also addressed in \citet{Gillon1990}. Cruse (1986, ch. 3) and (2000, ch. 6) discusses many of the issues covered in this chapter, including tests for lexical ambiguity, “antagonistic” senses, polysemy vs. homonymy, and contextual modification of meaning.


\subsubsection{Discussion exercises:}\label{sec:}
\ea
\textbf{A:} Do the italicized words illustrate ambiguity, vagueness, or indeterminacy?\\
\ea She spends her afternoons \textit{filing} correspondence and her fingernails.\\
\ex He spends his afternoons \textit{washing} clothes and dishes.\\
\ex He was a \textit{big} baby, even though both of his parents are \textit{small}.\\
\ex The weather wasn’t very \textit{bright}, but then neither was our tour guide.\\
\ex Donald Trump smokes \textit{expensive} cigars but drives a \textit{cheap} car.\\
\ex That boy couldn’t \textit{carry} a tune in a bucket.
                       \z
\z

\begin{stylepoints}
\textbf{B:} In each of the following examples, state which word is ambiguous as demonstrated by the antagonism or zeugma effect. Is it an instance of polysemy or homonymy?
\end{stylepoints}

\begin{stylepoints}
  a. “You are free to execute your laws, and your citizens, as you see fit.”\footnote{\textit{Star Trek: The Next Generation}, via grammar.about.com}
\end{stylepoints}

\begin{stylepoints}
  b. “… and covered themselves with dust and glory.”\footnote{Mark Twain, \textit{The Adventures of Tom Sawyer}}
\end{stylepoints}

\begin{stylepoints}
  c. Arthur declined my invitation, and Susan a Latin pronoun.
\end{stylepoints}

\begin{stylepoints}
  d. Susan can’t bear children.
\end{stylepoints}

\begin{stylepoints}
  e. The batteries were given out free of charge.
\end{stylepoints}

\begin{stylepoints}
  f. My astrologer wants to marry a star.
\end{stylepoints}

\paragraph{C: Figurative senses}

Identify the type of figure illustrated by the italicized words in the following passages:

\begin{enumerate}
\item Fear is the \textit{lock} and laughter the \textit{key} to your heart.\footnote{Crosby, Stills \& Nash – “Suite: Judy Blue Eyes”}
\item The \textit{White House} is concerned about terrorism.
\item She has six hungry \textit{mouths} to feed.
\item That joke is \textit{as old as the hills}.
\item It’s \textit{not the prettiest} quarter I’ve ever seen, Mr. Liddell.\footnote{Sam Mussabini in \textit{Chariots of Fire}.}
\item as \textit{pleasant and relaxed} as a coiled rattlesnake\footnote{Kurt Vonnegut in \textit{Breakfast of Champions}}
\item Headline: Korean “\textit{comfort women}” get controversial apology, compensation from Japanese government\footnote{news.com.au, December 30, 2015}
\end{enumerate}
\paragraph{D: Semantic shift}

Identify the figures of speech that provided the source for the following historical shifts in word meaning:

\begin{stylepoints}
a) \textit{bead} (< ‘prayer’)\\
b) \textit{pastor}\\
c) \textit{drumstick} (for ‘turkey leg’)\\
d) \textit{glossa} (Greek) ‘tongue; language’\\
e) \textit{pioneer} (< Old French \textit{peon(ier)} ‘foot-soldier’; cognate: \textit{pawn})
\end{stylepoints}

\subsubsection{Homework exercises:}\label{sec:}
\paragraph{A: Lexical ambiguity}

Do the uses of \textit{strike} in the following two sentences represent distinct senses (lexical ambiguity), or just indeterminacy? Provide linguistic evidence to support your answer.

1. The California Gold Rush began when James Marshall \textit{struck} gold at Sutter’s Mill.

2. Balaam \textit{struck} his donkey three times before it turned and spoke to him.

\paragraph{B: Dictionary entries}

Without looking at any published dictionary, draft a dictionary entry for \textit{mean}. Include the use of \textit{mean} as a noun, as an adjective, and at least three senses of \textit{mean} as a verb.

\paragraph{C: Polysemy}etc.\footnotemark{}
\todo{not footnotes in titles}
\footnotetext{Adapted from \citet{Cruse2000}.}

How would you describe the relationship between the readings of the italicized words in the following pairs of examples? You may choose from among the following options: \textsc{polysemy, homonymy, vagueness, indeterminacy, figurative use.} If none of these terms seem appropriate, describe the sense relation in prose.

\begin{enumerate}
\item a. Mary ordered an \textit{omelette}.\\
\ex The \textit{omelette} at table 6 wants his coffee now.
\item a. They \textit{led} the prisoner away.\\
\ex They \textit{led} him to believe that he would be freed.
\item I will wear a \textit{cheap} wristwatch, but I don’t want to drive a \textit{cheap} car.
\item a. My \textit{cousin} married an actress.\\
\ex My \textit{cousin} married a policeman.
\item a. Could you loan me your \textit{pen}? Mine is out of ink.\\
\ex The goats escaped from their \textit{pen} and ate up my artichokes.
\item a. Wittgenstein’s \href{http://en.wikipedia.org/wiki/Philosophical_Investigations}{Philosophical Investigations} is too \textit{deep} for me.\\
\ex This river is too \textit{deep} for my Land Rover to ford.
\end{enumerate}

\chapter{{6}: Lexical sense relations}

\section{Meaning relations between words}\label{sec:} %1. /

A traditional way of investigating the meaning of a word is to study the relationships between its meaning and the meanings of other words: which words have the same meaning, opposite meanings, etc. Strictly speaking these relations hold between specific senses, rather than between words; that is why we refer to them as sense relations. For example, one sense of \textit{mad} is a synonym of \textit{angry}, while another sense is a synonym of \textit{crazy}.



In \sectref{sec:2} we discuss the most familiar classes of sense relations: synonymy, several types of antonymy, hyponymy, and meronymy. We will try to define each of these relations in terms of relations between sentence meanings, since it is easier for speakers to make reliable judgments about sentences than about words in isolation. Where possible we will mention some types of linguistic evidence that can be used as diagnostics to help identify each relation. In \sectref{sec:3} we mention some of the standard ways of defining words in terms of their sense relations. This is the approach most commonly used in traditional dictionaries.


\section{Identifying sense relations}\label{sec:} %2. /

Let’s begin by thinking about what kinds of meaning relations are likely to be worth studying. If we are interested in the meaning of the word \textit{big}, it seems natural to look at its meaning relations with words like \textit{large}, \textit{small}, \textit{enormous}, etc. But comparing \textit{big} with words like \textit{multilingual} or \textit{extradite} seems unlikely to be very enlightening. The range of useful comparisons seems to be limited by some concept of semantic similarity or comparability.



Syntactic relationships are also relevant. The kinds of meaning relations mentioned above (same meaning, opposite meaning, etc.) hold between words which are mutually substitutable, i.e., which can occur in the same syntactic environments, as illustrated in (\ref{ex:6.1}a). These relations are referred to as \textsc{paradigmatic} sense relations. We might also want to investigate relations which hold between words which can occur in construction with each other, as illustrated in (\ref{ex:6.1}b). (In this example we see that \textit{big} can modify some head nouns but not others.) These relations are referred to as \textsc{syntagmatic} relations.


\ea \label{ex:6.1}
\ea Look at that \textit{big/large/small/enormous/?\#discontinuous/*snore} mosquito!\\
\ex Look at that big \textit{mosquito/elephant/?\#surname/\#color/*discontinuous/*snore}!
                       \z
\z


We will consider some syntagmatic relations in \chapref{sec:7}, when we discuss selectional restrictions. In this chapter we will be primarily concerned with paradigmatic relations.


\subsection{Synonyms}\label{sec:} %2.1 /

We often speak of synonyms as being words that “mean the same thing”. As a more rigorous definition, we will say that two words are synonymous (for a specific sense of each word) if substituting one word for the other does not change the meaning of a sentence. For example, we can change sentence (\ref{ex:6.2}a) into sentence (\ref{ex:6.2}b) by replacing \textit{frightened} with \textit{scared}. The two sentences are semantically equivalent (each entails the other). This shows that \textit{frightened} is a synonym of \textit{scared}.


\ea \label{ex:6.2}
\ea John \textit{frightened} the children.\\
\ex John \textit{scared} the children.
                       \z
\z


“Perfect” synonymy is extremely rare, and some linguists would say that it never occurs. Even for senses that are truly equivalent in meaning, there are often collocational differences as illustrated in (\ref{ex:6.3}--\ref{ex:6.4}). Replacing \textit{bucket} with \textit{pail} in (\ref{ex:6.3}a) does not change meaning; but in (\ref{ex:6.3}b), the idiomatic meaning that is possible with \textit{bucket} is not available with \textit{pail}. Replacing \textit{big} with \textit{large} does not change meaning in most contexts, as illustrated in (\ref{ex:6.4}a); but when used as a modifier for certain kinship terms, the two words are no longer equivalent (\textit{big} becomes a synonym of \textit{elder}), as illustrated in (\ref{ex:6.4}b).


\ea \label{ex:6.3}
\ea John filled the \textit{bucket}/\textit{pail}.\\
\ex John kicked the \textit{bucket}/??\textit{pail}.
                       \z
\z

\ea \label{ex:6.4}
\ea Susan lives in a \textit{big}/\textit{large} house.\\
\ex Susan lives with her \textit{big}/\textit{large} sister.\footnote{Adapted from \citet[66]{Saeed2009}.}
                       \z
\z

\subsection{Antonyms}\label{sec:} %2.2 /

Antonyms are commonly defined as words with “opposite” meaning; but what do we mean by “opposite”? We clearly do not mean ‘as different as possible’. As noted above, the meaning of \textit{big} is totally different from the meanings of \textit{multilingual} or \textit{extradite}, but neither of these words is an antonym of \textit{big}. When we say that \textit{big} is the opposite of \textit{small}, or that \textit{dead} is the opposite of \textit{alive}, we mean first that the two terms can have similar collocations. It is odd to call an inanimate object \textit{dead}, in the primary, literal sense of the word, because it is not the kind of thing that could ever be \textit{alive}. Second, we mean that the two terms express a value of the same property or attribute. \textit{Big} and \textit{small} both express degrees of size, while \textit{dead} and \textit{alive} both express degrees of vitality. So two words which are antonyms actually share most of their components of meaning, and differ only with respect to the value of one particular feature.



The term \textsc{antonym} actually covers several different sense relations. Some pairs of antonyms express opposite ends of a particular scale, like \textit{big} and \textit{small}. We refer to such pairs as \textsc{scalar} or \textsc{gradable} antonyms. Other pairs, like \textit{dead} and \textit{alive}, express discrete values rather than points on a scale, and name the only possible values for the relevant attribute. We refer to such pairs as \textsc{simple} or \textsc{complementary} antonyms. Several other types of antonyms are commonly recognized as well. We begin with simple antonyms.


\subsubsection{Complementary pairs (simple antonyms)}\label{sec:} %2.2.1 /
\begin{quote}
“All men are created equal. Some, it appears, are created a little more equal than others.”\\
—Ambrose Bierce, In \textit{The San Francisco Wasp} magazine, September 16, 1882
\end{quote}


Complementary pairs such as \textit{open/shut}, \textit{alive/dead}, \textit{male/female}, \textit{on/off}, etc. exhaust the range of possibilities, for things that they can collocate with. There is (normally) no middle ground; a person is either alive or dead, a switch is either on or off, etc. The defining property of simple antonyms is that replacing one member of the pair with the other, as in \REF{ex:6.5}, produces sentences which are \textsc{contradictory.} As discussed in \chapref{sec:3}, this means that the two sentences must have opposite truth values in every circumstance; one of them must be true and the other false in all possible situations where these words can be used appropriately.


\ea \label{ex:6.5}
\ea The switch is on.\\
\ex The switch is off.\\
\ex ??The switch is neither on nor off.
                       \z
\z


If two sentences are contradictory, then one or the other must always be true. This means that simple antonyms allow for no middle ground, as indicated in (\ref{ex:6.5}c). The negation of one entails the truth of the other, as illustrated in \REF{ex:}.


\ea \label{ex:6.6}
\ea[??]{The post office is not open today, but it is not closed either.\\}
\ex[??]{Your headlights are not off, but they are not on either.}
                       \z
\z


A significant challenge in identifying simple antonyms is the fact that they are easily coerced into acting like gradable antonyms.\footnote{\citet[463]{Cann2011}.} For example, \textit{equal} and \textit{unequal} are simple antonyms; the humor in the quote by Ambrose Bierce at the beginning of this section arises from the way he uses \textit{equal} as if it were gradable. In a similar vein, zombies are often described as being \textit{undead}, implying that they are not dead but not really alive either. However, the gradable use of simple antonyms is typically possible only in certain figurative or semi-idiomatic expressions. The gradable uses in \REF{ex:6.7} seem natural, but those in \REF{ex:6.8} are not. The sentences in \REF{ex:6.9} illustrate further contrasts. For true gradable antonyms, like those discussed in the following section, all of these patterns would generally be fully acceptable, not odd or humorous.


\ea \label{ex:6.7}
\ea half-dead, half-closed, half-open\\
\ex more dead than alive\\
\ex deader than a door nail
                       \z
\z

\ea \label{ex:6.8}
\ea ?half-alive\\
\ex \#a little too dead\\
\ex \#not dead enough\\
\ex \#How dead is that mosquito?\\
\ex \#This mosquito is deader than that one.
                       \z
\z

\ea \label{ex:6.9}
\ea I feel fully/very/??slightly alive.\\
\ex This town/\#mosquito seems very/slightly dead.
                       \z
\z

\subsubsection{2.2.2 Gradable (scalar) antonyms}\label{sec:}

A defining property of gradable (or scalar) antonyms is that replacing one member of such a pair with the other produces sentences which are \textsc{contrary}, as illustrated in (\ref{ex:6.10}a--b). As discussed in \chapref{sec:3}, contrary sentences are sentences which cannot both be true, though they may both be false (\ref{ex:6.10}c).


\ea \label{ex:6.10}
\ea My youngest son-in-law is extremely diligent.\\
\ex My youngest son-in-law is extremely lazy.\\
\ex My youngest son-in-law is neither extremely diligent nor extremely lazy.
                       \z
\z


Note, however, that not all pairs of words which satisfy this criterion would normally be called “antonyms”. The two sentences in \REF{ex:6.11} cannot both be true (when referring to the same thing), which shows that \textit{turnip} and \textit{platypus} are \textsc{incompatibles}; but they are not antonyms. So our definition of gradable antonyms needs to include the fact that, as mentioned above, they name opposite ends of a single scale and therefore belong to the same semantic domain.


\ea \label{ex:6.11}
\ea This thing is a turnip.\\
\ex This thing is a platypus.
                       \z
\z


The following diagnostic properties can help us to identify scalar antonyms, and in particular to distinguish them from simple antonyms:\footnote{Adapted from \citet[67]{Saeed2009}; Cruse (1986: 204 ff).}


\begin{enumerate}[label=\alph*.]
\item Scalar antonyms typically have corresponding intermediate terms, e.g. \textit{warm, tepid, cool} which name points somewhere between \textit{hot} and \textit{cold} on the temperature scale.
\item Scalar antonyms name values which are relative rather than absolute. For example, a small elephant will probably be much bigger than a big mosquito, and the temperature range we would call hot for a bath or a cup of coffee would be very cold for a blast furnace.
\item As discussed in \chapref{sec:5}, scalar antonyms are often vague.
\item Comparative forms of scalar antonyms are completely natural (\textit{hotter}, \textit{colder}, etc.), whereas they are normally much less natural with complementary antonyms, as illustrated in (\ref{ex:6.8}e) above.
\item The comparative forms of scalar antonyms form a converse pair (see below).\footnote{\citet[232]{Cruse1986}.} For example, \textit{A is longer than B}  $\leftrightarrow $  \textit{B is shorter than A}.
\item One member of a pair of scalar antonyms often has privileged status, or is felt to be more basic, as illustrated in \REF{ex:6.12}.
\end{enumerate}

\ea \label{ex:6.12}
\ea How old/??young are you?\\
\ex How tall/??short are you?\\
\ex How deep/??shallow is the water?
                       \z
\z

\subsubsection{Converse pairs}\label{sec:} %2.2.3 /

Converse pairs involve words that name an asymmetric relation between two entities, e.g. \textit{parent-child, above}-\textit{below}, \textit{employer-employee}.\footnote{\citet[231]{Cruse1986} refers to such pairs as \textsc{relational} \textsc{opposites}.} The relation must be asymmetric or there would be no pair; symmetric relations like \textit{equal} or \textit{resemble} are (in a sense) their own converses. The two members of a converse pair express the same basic relation, with the positions of the two arguments reversed. If we replace one member of a converse pair with the other, and also reverse the order of the arguments, as in (\ref{ex:6.13}--\ref{ex:6.14}), we produce sentences which are semantically equivalent (paraphrases).


\ea \label{ex:6.13}
\ea Michael is my advisor.\\
\ex I am Michael’s advisee.
                       \z
\z

\ea \label{ex:6.14}
OWN(x,y) $\leftrightarrow $ BELONG\_TO(y,x)\\
ABOVE(x,y) $\leftrightarrow $ BELOW(y,x)\\
PARENT\_OF(x,y) $\leftrightarrow $ CHILD\_OF(y,x)
\z

\subsubsection{2.2.4 Reverse pairs}\label{sec:}

Two words (normally verbs) are called \textsc{reverses} if they “denote motion or change in opposite directions… [I]n addition… they should differ only in respect of directionality” \citep[226]{Cruse1986}. Examples include \textit{push/pull, come/go}, \textit{fill/empty}, \textit{heat/cool}, \textit{strengthen/weaken}, etc. Cruse notes that some pairs of this type (but not all) allow an interesting use of \textit{again}, as illustrated in \REF{ex:6.15}. In these sentences, \textit{again} does not mean that the action named by the second verb is repeated (\textsc{repetitive} reading), but rather that the situation is restored to its original state (\textsc{restitutive} reading).


\ea \label{ex:6.15}
\ea The nurse heated the instruments to sterilize them, and then cooled them \textit{again}.\\
\ex George filled the tank with water, and then emptied it \textit{again}.
                       \z
\z

\subsection{Hyponymy and taxonomy}\label{sec:} %2.3 /

When two words stand in a generic-specific relationship, we refer to the more specific term (e.g. \textit{moose}) as the \textsc{hyponym} and to the more generic term (e.g. \textit{mammal}) as the \textsc{superordinate} or \textsc{hyperonym}. A generic-specific relationship can be defined by saying that a simple positive non-quantified statement involving the hyponym will entail the same statement involving the superordinate, as illustrated in \REF{ex:6.16}. (In each example, the hyponym and superordinate term are underlined.) We need to specify that the statement is positive, because negation reverses the direction of the entailments \REF{ex:6.17}.


\ea \label{ex:6.16}
\ea \textit{Seabiscuit was a stallion}  entails:  \textit{Seabiscuit was a horse}.\\
\ex \textit{Fred stole} \textit{my bicycle}  entails:  \textit{Fred took} \textit{my bicycle}.\\
\ex \textit{John assassinated} \textit{the Mayor}  entails:  \textit{John killed} \textit{the Mayor}.\\
\ex \textit{Arthur looks like a squirrel}  entails:  \textit{Arthur looks like a rodent}.\\
\ex \textit{This pot is made of copper}  entails:  \textit{This pot is made of metal}.
                       \z
\z

\ea \label{ex:6.17}
\ea \textit{Seabiscuit was not a horse}  entails:  \textit{Seabiscuit was not a stallion}.\\
\ex \textit{John did not kill} \textit{the Mayor}  entails:  \textit{John did not assassinate} \textit{the Mayor}.\\
\ex \textit{This pot is not made of metal}  entails:  \textit{This pot is not made of copper}.
                       \z
\z


\textsc{Taxonomy} is a special type of hyponymy, a classifying relation. \citet[137]{Cruse1986} suggests the following diagnostic: X is a \textsc{taxonym} of Y if it is natural to say \textit{An X is a kind/type of Y}. Examples of taxonomy are presented in (\ref{ex:6.18}a--b), while the examples in (\ref{ex:6.18}c--d) show that other hyponyms are not fully natural in this pattern. (The word \textsc{taxonymy} is also used to refer to a generic-specific hierarchy, or system of classification.)


\ea \label{ex:6.18}
\ea \textit{A beagle} \textit{is a kind of dog}.\\
\ex \textit{Gold} \textit{is a type of metal}.\\
\ex \textit{?A} \textit{stallion} \textit{is a kind of horse.}\\
\ex ??\textit{Sunday is a} \textit{kind of day of the week}.
                       \z
\z


\textsc{Taxonomic sisters} are taxonyms which share the same superordinate term, such as \textit{squirrel} and \textit{mouse} which are both hyponyms of \textit{rodent}. Taxonomic sisters must be incompatible, in the sense defined above; for example, a single animal cannot be both a squirrel and a mouse. But that property alone does not distinguish taxonomy from other types of hyponymy. Taxonomic sisters occur naturally in sentences like the following:

\ea \label{ex:6.19} \ea \textit{A beagle} \textit{is a kind of dog, and so is a Great Dane}.\\
\ex \textit{Gold} \textit{is a type of metal, and copper is another type of metal}.
\z \z


Cruse notes that taxonomy often involves terms that name \textsc{natural kinds} (e.g., names of species, substances, etc.). Natural kind terms cannot easily be paraphrased by a superordinate term plus modifier, as many other words can (see sec. 3 below):


\ea 
\label{ex:6.20}
\ea “\textit{Stallion” means a male} \textit{horse.}\\
\ex \textit{“Sunday” mean}\textit{s the first day of the week}.\\
\ex ??\textit{“Beagle”} \textit{means a \_\_\_} \textit{dog}.\\
\ex ??\textit{“Gold”} \textit{means a \_\_\_ metal}.\\
\ex ??\textit{“Dog”} \textit{means a \_\_\_} \textit{animal}.
\z \z


We must remember that semantic analysis is concerned with properties of the object language, rather than scientific knowledge. The taxonomies revealed by linguistic evidence may not always match standard scientific classifications. For example, the authoritative \textit{Kamus Dewan} (a Malay dictionary published by the national language bureau in Kuala Lumpur) gives the following definition for \textit{labah-labah} ‘spider’:


\ea \label{ex:6.21
}\textit{labah-labah: sejenis serangga} \textit{yang berkaki lapan}\\
‘spider: a kind of insect that has eight legs’
\z


This definition provides evidence that in Malay, \textit{labah-labah} ‘spider’ is a taxonym of \textit{serangga} ‘insect’, even though standard zoological classifications do not classify spiders as insects. (Thought question: does this mean that \textit{serangga} is not an accurate translation equivalent for the English word \textit{insect}?)



Similar examples can be found in many different languages. For example, in Tuvaluan (a Polynesian language), the words for ‘turtle’ and ‘dolphin/whale’ are taxonyms of \textit{ika} ‘fish’.\footnote{\citet[192]{Finegan1999}.} The fact that turtles, dolphins and whales are not zoologically classified as fish is irrelevant to our analysis of the lexical structure of Tuvaluan.


\subsection{Meronymy}\label{sec:} %2.4 /

A \textsc{meronymy} is a pair of words expressing a part-whole relationship. The word naming the part is called the meronym. For example, \textit{hand}, \textit{brain} and \textit{eye} are all meronyms of \textit{body}; \textit{door}, \textit{roof} and \textit{kitchen} are all meronyms of \textit{house}; etc.



Once again, it is important to remember that when we study patterns of meronymy, we are studying the structure of the lexicon, i.e., relations between words and not between the things named by the words. One linguistic test for identifying meronymy is the naturalness of sentences like the following: \textit{The parts of an X include the Y, the Z, ...} \citep[161]{Cruse1986}.



A meronym is a name for a part, and not merely a piece, of a larger whole. Human languages have many words that name parts of things, but few words that name pieces. Cruse (1986:158–59) lists three differences between parts and pieces. First, a part has autonomous identity: many shops sell automobile parts which have never been structurally integrated into an actual car. A piece of a car, on the other hand, must have come from a complete car. (Few shops sell pieces of automobile.) Second, the boundaries of a part are motivated by some kind of natural boundary or discontinuity — potential for separation or motion relative to neighboring parts, joints (e.g. in the body), difference in material, narrowing of connection to the whole, etc. The boundaries of a piece are arbitrary. Third, a part typically has a definite function relative to the whole, whereas this is not true for pieces.


\section{Defining words in terms of sense relations}\label{sec:} %3. /

Traditional ways of defining words depend heavily on the use of sense relations; hyponymy has played an especially important role. The classical form of a definition, going back at least to Aristotle (384–322 BC), is a kind of phrasal synonym; that is, a phrase which is mutually substitutable with the word being defined (same syntactic distribution) and equivalent or nearly equivalent in meaning.



The standard way of creating a definition is to start with the nearest superordinate term for the word being defined (traditionally called the \textit{genus proximum}), and then add one or more modifiers (traditionally called the \textit{differentia specifica}) which will unambiguously distinguish this word from its hyponymic sisters. So, for example, we might define \textit{ewe} as ‘an adult female sheep’; \textit{sheep} is the superordinate term, while \textit{adult} and \textit{female} are modifiers which distinguish ewes from other kinds of sheep.



This structure can be further illustrated with the following well-known definition by Samuel Johnson (1709-1784), himself a famous lexicographer. It actually consists of two parallel definitions; the superordinate term in the first is \textit{writer}, and in the second \textit{drudge}. The remainder of each definition provides the modifiers which distinguish lexicographers from other kinds of writers or drudges.


\ea \label{ex:6.22}
\textit{Lexicographer}: A writer of dictionaries; a harmless drudge that busies himself in tracing the [origin], and detailing the signification of words.
\z


Some additional examples are presented in \REF{ex:6.23}. In each definition the superordinate term is underlined while the distinguishing modifiers are placed in square brackets.


\ea \label{ex:6.23}
\ea \textit{fir} (N): a kind of tree [with evergreen needles].\footnote{Hartmann \& \citet[62]{James1998}.}\\
\ex \textit{rectangle} (N): a [right-angled] quadrilateral.\footnote{Svensén (2009: 219).}\\
\ex \textit{clean} (Adj): free [from dirt].\footnote{Svensén (2009: 219).}
                       \z
\z


However, as a number of authors have pointed out, many words cannot easily be defined in this way. In such cases, one common alternative is to define a word by using synonyms (\ref{ex:6.24}a--b) or antonyms (\ref{ex:6.24}c--d).


\ea \label{ex:6.24}
\ea \textit{grumpy}: moodily cross; surly\footnote{\url{http://www.merriam-webster.com/dictionary/}}\\
\ex \textit{sad}: affected with or expressive of grief or unhappiness\footnote{\url{http://www.merriam-webster.com/dictionary/}}\\
\ex \textit{free}: not controlled by obligation or the will of another;\\
  not bound, fastened, or attached.\footnote{\url{http://www.thefreedictionary.com/free}} \\
\ex \textit{pure}: not mixed or adulterated with any other substance or material.\footnote{\url{http://oxforddictionaries.com/us/definition/american_english/pure}} 
                       \z
\z


Another common type of definition is the \textsc{extensional} definition, which spells out the denotation of the word rather than its sense as in a normal definition. This type is illustrated in \REF{ex:6.25}.


\ea \label{ex:6.25}
Definitions from Merriam-Webster on-line dictionary:\\
\ea   \textit{New England}: the NE United States comprising the states of Maine, New Hampshire, Vermont, Massachusetts, Rhode Island, \& Connecticut
\ex  \textit{cat}: any of a family (Felidae) of carnivorous, usually solitary and nocturnal, mammals (as the domestic cat, lion, tiger, leopard, jaguar, cougar, wildcat, lynx, and cheetah)
\z \z

Some newer dictionaries, notably the COBUILD dictionary, make use of full sentence definitions rather than phrasal synonyms, as illustrated in \REF{ex:6.26}.

\ea \label{ex:6.26}
confidential: Information that is \textbf{confidential} is meant to be kept secret or private.\footnote{COBUILD dictionary, 3\textsuperscript{rd} edition (2001); cited in \citet{Rundell2006}.}
\z

\section{Conclusion}\label{sec:} %4. /

In this chapter we have mentioned only the most commonly used sense relations (some authors have found it helpful to refer to dozens of others). We have illustrated various diagnostic tests for identifying sense relations, many of them involving entailment or other meaning relations between sentences. Studying these sense relations provides a useful tool for probing the meaning of a word, and for constructing dictionary definitions of words.



\furtherreading



Cruse (1986, chapters 4–12) offers a detailed discussion of each of the sense relations mentioned in this chapter. \citet{Cann2011} provides a helpful overview of the subject.


\subsubsection{Discussion exercise:}\label{sec:}

Identify the meaning relations for the following pairs of words, and provide linguistic evidence that supports your identification:

\begin{tabular}{@{\hspace{2em}}>{\itshape}l@{\hspace{1em}}>{\itshape}l}
sharp & dull\\
finite & infinite\\
two & too\\
arm & leg\\
hyponym &  hyperonym\\
silver  & metal\\
insert  & extract  \\
\end{tabular}


\subsubsection{Homework exercises:}\label{sec:}
\paragraph{A: Antonyms}\footnotemark{}
\todo{footnote}
\footnotetext{Adapted from \citet[82]{Saeed2009}, ex. 3.4.}

Below is a list of incompatible pairs. (a) Classify each pair into one of the following types of relation: \textsc{simple antonyms, gradable antonyms, reverses, converses,} or \textsc{taxonomic sisters}. (b) For each pair, provide at least one type of linguistic evidence (e.g. example sentences) that supports your decision, and where possible mention other types of evidence that would lend additional support.\\
\textit{legal/illegal  fat/thin  raise/lower  wine/beer\\
lend to/borrow from  lucky/unlucky  married/unmarried}

\chapter{{7}: Components of lexical meaning}

\section{Introduction}\label{sec:} %1. /

The traditional model of writing definitions for words, which we discussed in \chapref{sec:6}, seems to assume that word meanings can (in many cases) be broken down into smaller elements of meaning.\footnote{\citet[126]{Engelberg2011}.} For example, we defined \textit{ewe} as ‘an adult female sheep’, which seems to suggest that the meanings of the words \textit{sheep}, \textit{adult}, and \textit{female} are included in the meaning of \textit{ewe}.\footnote{Svensén (2009: 218), in his \textit{Handbook of Lexicography}, identifies such intensional definitions as “the classic type of definition”. He explicitly defines intension (i.e. sense) in terms of components of meaning: “The term \textsc{intension} denotes the content of the concept, which can be defined as the combination of the distinctive features comprised by the concept.” Svensén seems to have in mind the representation of components of word meaning as binary distinctive features, the approach discussed in \sectref{sec:4} below.} In fact, if the phrase ‘adult female sheep’ is really a synonym for \textit{ewe}, one might say that the meaning of \textit{ewe} is simply the combination of the meanings of \textit{sheep}, \textit{adult}, and \textit{female}. Another way to express this intuition is to say that the meanings of \textit{sheep}, \textit{adult}, and \textit{female} are \textsc{components} of the meaning of \textit{ewe}.



In this chapter we introduce some basic ideas about how to identify and represent a word’s components of meaning. Most components of meaning can be viewed as entailments or presuppositions which the word contributes to the meaning of a sentence in which it occurs. We discuss lexical entailments in \sectref{sec:2} and \textsc{selectional restrictions} in \sectref{sec:3}. Selectional restrictions are constraints on word combinations which rule out collocations such as \#\textit{Assassinate that cockroach!} or \#\textit{This cabbage is nervous}, and we will treat them as a type of presupposition.



In \sectref{sec:4} we summarize one influential approach to word meanings, in which components of meaning were represented as binary distinctive features. We will briefly discuss the advantages and limitations of this approach, which is no longer widely used. In \sectref{sec:5} we introduce some of the foundational work on the meanings of verbs.



Our study of the components of word meanings will primarily be based on evidence from sentence meanings, for reasons discussed in earlier chapters. We focus here on descriptive meaning. Of course, words can also convey various kinds of expressive (or \textsc{affective}) meaning, signaling varying degrees of politeness, intimacy, formality, vulgarity, speaker’s attitudes, etc., but we will not attempt to deal with these issues in the current chapter.


\section{Lexical entailments}\label{sec:} %2. /

When people talk about the meaning of one word (e.g. \textit{sheep}) being “part of”, or “contained in”, the meaning of some other word (e.g. \textit{ewe}), they are generally describing a lexical entailment. Strictly speaking, of course, entailment is a meaning relation between propositions or sentences, not words. When we speak of “lexical entailments”, we mean that the meaning relation between two words creates an entailment relation between sentences that contain those words. This is illustrated in (\ref{ex:7.1}--\ref{ex:7.4}). In each pair of sentences, the (a) sentence entails the (b) sentence because the meaning of the italicized word in the (b) sentence is part of, or is contained in, the meaning of the italicized word in the (a) sentence. We can say that \textit{ewe} lexically entails \textit{sheep}, \textit{assassinate} lexically entails \textit{kill}, etc.


\ea \label{ex:7.1}
\ea John \textit{assassinated} the Mayor.\\
\ex John \textit{killed} the Mayor.
                       \z
\z

\ea
\label{ex:7.2}
\ea John is a \textit{bachelor}.\\
\ex John is \textit{unmarried}.
\z \z

\ea \label{ex:7.3}
\ea  John \textit{stole} my bicycle.\\
\ex John \textit{took} my bicycle.
\z \z

\ea \label{ex:7.4}
\ea Fido is a \textit{dog}.\\
\ex Fido is an \textit{animal}.
                       \z
\z


These intuitive judgments about lexical entailments can be supported by additional linguistic evidence. Speakers of English feel sentences like \REF{ex:7.5}, which explicitly describe the entailment relation, to be natural. Sentences like \REF{ex:7.6}, however, which seem to cast doubt on the entailment relation, are unnatural or incoherent:\footnote{examples from \citet[14]{Cruse1986}.}


\ea \label{ex:7.5}
\ea It can’t possibly be a dog and not an animal.\\
\ex It’s a dog and therefore it’s an animal.\\
\ex If it’s not an animal, then it follows that it’s not a dog.
                       \z
\z

\ea \label{ex:7.6}
\ea \#It’s not an animal, but it’s just possible that it’s a dog.\\
\ex \#It’s a dog, so it might be an animal.
                       \z
\z


\citet[12]{Cruse1986} mentions several additional tests for entailments which can be applied here, including the following:


\ea
Denying the entailed component leads to contradiction:\\
\ea \#John killed the Mayor but the Mayor did not die.\\
\ex \#It’s a dog but it’s not an animal.\\
\ex \#John is a bachelor but he is happily married.\\
\ex \#The child fell upwards.
\z
                       \z

\ea
Asserting the entailed component leads to unnatural redundancy (or \textsc{pleonasm}):\\
\ea \#It’s a dog and it’s an animal.\\
\ex ??Kick it with one of your feet.  (\citealt{Cruse1986}:12)\\
\ex ??He was murdered illegally.  (\citealt{Cruse1986}:12)
                       \z
\z

\section{Selectional restrictions}\label{sec:} %3. /

In addition to lexical entailments, another important aspect of word meanings has to do with constraints on specific word combinations. These constraints are referred to as \textsc{selectional restrictions}. The sentences in \REF{ex:7.9} all seem quite odd, not really acceptable except as a kind of joke, because they violate selectional restrictions.


\ea \label{ex:7.9}
\ea \#This sausage doesn’t appreciate Mozart.\\
\ex \#John drank his sandwich and took a big bite out of his coffee.\\
\ex \#Susan folded/perforated/caramelized her reputation.\\
\ex \#Your exam results are sleeping.\\
\ex \#The square root of oatmeal is Houston.\\
\ex \textit{My Feet Are Smiling} (title of guitarist Leo Kottke’s sixth album)\\
\ex “They’ve a temper, some of them — particularly verbs: they’re the proudest…”\\
  {}[Humpty Dumpty, in \textit{Through the Looking Glass}]
                       \z
\z


As we noted in \REF{ex:7.7}, denying an entailment leads to a contradiction. In contrast, violations of selectional restrictions like those in \REF{ex:7.9} lead to dissonance rather than contradiction.\footnote{Such violations are sometimes called “category mistakes”, or “sortal errors”, especially in philosophical literature.} \citet[95]{Chomsky1965} proposed that selectional restrictions were triggered by syntactic properties of words, but McCawley, Lakoff and other authors have argued that they derive from word meanings. If they were purely syntactic, they should hold even in contexts like those in \REF{ex:7.10}. The fact that these sentences are acceptable suggests that the constraints are semantic rather than syntactic in nature.

\ea \label{ex:7.10}
\ea He’s become irrational – he thinks his exam results are sleeping.\\
\ex You can’t say that John drank his sandwich.
                       \z
\z

The lexical entailments of words which occur in questions or negated statements can often be denied without contradiction, as illustrated in \REF{ex:7.11}. Selectional restrictions, in contrast, hold even in questions, negative statements, and other non-assertive environments \REF{ex:7.12}. This suggests that they are a special type of presupposition, and we will assume that this is the case.\footnote{The idea that selectional restrictions can be treated as lexical presuppositions was apparently first proposed by Fillmore, but was first published by \citet{McCawley1968}.}


\ea \label{ex:7.11}
\ea John didn’t kill the Mayor; the Mayor is not even dead.\\
\ex Is that a dog, or even an animal?\\
\ex John is not a bachelor, he is happily married.\\
\ex The snowflake did not fall, it floated upwards.
                       \z
\z

\ea \label{ex:7.12}
\ea \#Did John drink his sandwich?\\
\ex \#John didn’t drink his sandwich; maybe he doesn’t like liverwurst.\\
\ex \#Are your exam results sleeping?\\
\ex \#My feet aren’t smiling.
                       \z
\z

Selectional restrictions are part of the meanings of specific words; that is, they are linguistic in nature, rather than simply facts about the world. \citet[21]{Cruse1986} points out that hearers typically express astonishment or disbelief on hearing a statement that is improbable, given what we know about the world (\ref{ex:7.13}--\ref{ex:7.14}). This is quite different from hearers’ reactions to violations of selectional restrictions like those in \REF{ex:7.9}. Those sentences are linguistically unacceptable, and hearers are more likely to respond, “You can’t say that.”

\ea \label{ex:7.13}
A: Our kitten drank a bottle of claret.\\
B: No! Really?  (\citealt{Cruse1986}:21)
\z

\ea \label{ex:7.14}
A: I know an old woman who swallowed a goat/cow/bulldozer/\#participle.\\
B: That’s impossible!
\z


It is fairly common for words with the same basic entailments to differ with respect to their selectional restrictions. German has two words corresponding to the English word \textit{eat}: \textit{essen} for people and \textit{fressen} for animals. (One might use \textit{fressen} to insult or tease someone — basically saying they eat like an animal.) In a Kimaragang\footnote{An Austronesian language of northern Borneo.} version of the Christmas story, the translator used the word \textit{paalansayad} to render the phrase which is expressed in the King James Bible as \textit{great with child}. This word correctly expresses the idea that Mary was in a very advanced stage of pregnancy when she arrived in Bethlehem; but another term had to be found when someone pointed out that \textit{paalansayad} is normally used only for water buffalo and certain other kinds of livestock.



It is sometimes helpful to distinguish selectional restrictions (a type of presupposition triggered by specific words, as discussed above) from \textsc{collocational restrictions}.\footnote{We follow the terminology of Cruse (1986: 107, 279–80) here. Not everyone makes this distinction. In some work on translation principles, e.g. \citet{BeekmanCallow1974}, a violation of either type is referred to as a \textsc{collocational clash}.} Collocational restrictions are conventionalized patterns of combining two or more words. They reflect common ways of speaking, or “normal” usage, within the speech community. Some examples of collocational restrictions are presented in \REF{ex:7.15}.


\ea \label{ex:7.15}
\ea John died/passed away/kicked the bucket.\\
\ex My prize rose bush died/\#passed away/\#kicked the bucket.\\
\ex When we’re feeling under the weather, most of us welcome a big/?\#large hug.\\
\ex He is (stark) raving mad/\#crazy.\footnote{Jim Roberts, p.c.}\\
\ex dirty/\#unclean joke\\
\ex unclean/\#dirty spirit
                       \z
\z


Violations of a collocational restriction are felt to be odd or unnatural, but they can typically be repaired by replacing one of the words with a synonym, suggesting that collocational restrictions are not, strictly speaking, due to lexical meaning \textit{per se}.


\section{Componential analysis}\label{sec:} %4. /

Many different theories have been proposed for representing components of lexical meaning. All of them aim to develop a formal representation of meaning components which will allow us to account for semantic properties of words, such as their sense relations, and perhaps some syntactic properties as well.



One very influential approach during the middle of the 20\textsuperscript{th} century was to treat word meanings as bundles of distinctive semantic features, in much the same way that phonemes are defined in terms of distinctive phonetic/phonological features.\footnote{One early example of this approach is found in \citet{Nida1951}.} This approach is sometimes referred to as \textsc{componential analysis} of meaning. Some of the motivation for this approach can be seen in the following famous example from Hjemslev (1943/1953). The example makes it clear that the feature of gender is an aspect of meaning that distinguishes many pairs of lexical items within certain semantic domains. If we were to ignore this fact and just treat each word’s meaning as an \textsc{atom} (i.e., an unanalyzable unit), we would be missing a significant generalization.


\ea
\begin{tabular}[t]{*{5}{l}}
\lsptoprule
 & horse & human & child & sheep\\
\midrule 
“he” & stallion & man & boy & ram\\
“she” & mare & woman & girl & ewe\\
\lspbottomrule
\end{tabular}
\z

Features like gender and adulthood are binary, and so lend themselves to representation in either tree or matrix format, as illustrated in \REF{ex:7.17}. Notice that in addition to the values + and –, features may be unspecified (represented by ⌀ in the matrix). For example, the word \textit{foal} is unspecified for gender, and the word \textit{horse} is unspecified for both age and gender.


\ea \label{ex:7.17}
Binary feature analysis for horse terms:\\
\begin{multicols}{2}
\begin{tabular}[t]{lll}
\lsptoprule
& [adult] & [male]\\\midrule
\itshape horse & ⌀ & ⌀\\
\itshape stallion & + & +\\
\itshape mare & + & –\\
\itshape foal & – & ⌀\\
\itshape colt & – & +\\
\itshape filly & – & –\\
\lspbottomrule
\end{tabular}\\
\begin{forest}
[\scshape horse
  [??, edge label={node[midway,above left,font=\scriptsize]{[+A]}}
    [\itshape stallion,edge label={node[midway,above left,font=\scriptsize]{[+M]}}]
    [\itshape mare, edge label={node[midway,above right,font=\scriptsize]{[−A]}}]
  ] [foal, edge label={node[midway,above right,font=\scriptsize]{[+M]}}
    [\itshape colt,edge label={node[midway,above left,font=\scriptsize]{[+M]}}]
    [\itshape filly, edge label={node[midway,above right,font=\scriptsize]{[−M]}}]
  ]
]
\end{forest}
\end{multicols}
\z


\ea  \label{ex:7.18} Binary feature analysis for human terms:\\
\begin{tabular}[t]{lll} 
\lsptoprule
& [adult] & [male]\\ \midrule
\textit{man\textsubscript{1}}\textit{/human} & ⌀ & ⌀\\
\textit{man\textsubscript{2}} & + & +\\
\itshape woman & + & –\\
\itshape child & – & ⌀\\
\itshape boy & – & +\\
\itshape girl & – & –\\
\lspbottomrule
\end{tabular}
\z

Componential analysis provides neat explanations for some sense relations. Synonymous senses can be represented as pairs that share all the same components of meaning. Complementary pairs are perfectly modeled by binary features: the two elements differ only in the polarity for one feature, e.g. [+/– alive], [+/– awake], [+/– possible], [+/– legal], etc. The semantic components of a hyperonym (e.g. \textit{child} [+human, –adult]) are a proper subset of the semantic components of its hyponyms (e.g. \textit{boy} [+human, –adult, +male]); \textit{girl} [+human, –adult, –male])). In other words, each hyponym contains all the semantic components of the hyperonym plus at least one more; and these “extra” components are the ones that distinguish the meanings of taxonomic sisters. Reverse pairs might be treated in a way somewhat similar to complementary pairs; they differ in precisely one component of meaning, typically a direction, with the dimension and manner of motion and the reference point held steady.



On the other hand, it is not so easy to define gradable antonyms, converse pairs, or meronyms in this way. Moreover, while many of the benefits of this kind of componential analysis are shared by other approaches, a number of problems have been pointed out which are specific to the binary feature approach.\footnote{The following discussion is based on Engelberg (2011: 129–30); Lyons (1977: 317 ff).}



First, there are many lexical distinctions which do not seem to be easily expressible in terms of binary features, at least not in any plausible way. Species names, for example, are a well-known challenge to this approach. What features distinguish members of the cat family (\textit{lion, tiger, leopard, jaguar, cougar, wildcat, lynx, cheetah}, etc.) from each other? Similar issues arise with color terms, types of metal, etc. In order to deal with such cases, it seems that the number of features would need to be almost as great as the number of lexical items.



Second, it is not clear how to use simple binary features to represent the meanings of two-place predicates, such as \textit{recognize}, \textit{offend}, \textit{mother (of)}, etc. The word \textit{recognize} entails a change of state in the first argument, while the word \textit{offend} entails a change of state in the second argument. A simple feature matrix like those above cannot specify which argument a particular feature applies to.



Third, some word meanings cannot be adequately represented as an unordered bundle of features, whether binary or not. For example, many studies have been done concerning the semantic components of kinship terms in various languages. This is one domain in which the components need to be ordered or structured in some way; ‘mother’s brother’s spouse’ (one sense of \textit{aunt} in English) would probably not, in most languages, be called by the same term as ‘spouse’s mother’s brother’ (no English term available). Verb meanings also seem to require structured components. For example, ‘want to cause to die’ (part of the meaning of \textit{murderous}) is quite different from ‘cause to want to die’ (similar to one sense of \textit{mortify}).



Fourth, we need to ask how many features would be needed to describe the entire lexicon of a single language? Binary feature analysis can be very efficient within certain restricted semantic domains, but when we try to compare a wider range of words, it is not clear that the inventory of features could be much smaller than the lexicon itself.


\section{Verb meanings}\label{sec:} %5. /

Much of the recent research on lexical semantics has focused on verb meanings. One reason for this special interest in verbs is the fact that verb meanings have a direct influence on syntactic structure, and so syntactic evidence can be used to supplement traditional semantic methods.



A classic paper by Charles \citet{Fillmore1970} distinguishes two classes of transitive verbs in English: “surface contact” verbs (e.g., \textit{hit, slap, strike, bump, stroke}) vs. “change of state” verbs (e.g., \textit{break, bend, fold, shatter, crack}). Fillmore shows that the members of each class share certain syntactic and semantic properties which distinguish them from members of the other class. He further argues that the correlation between these syntactic and semantic properties supports a view of lexical semantics under which the meaning of a verb is made up of two kinds of elements: (a) systematic components of meaning that are shared by an entire class; and (b) idiosyncratic components that are specific to the individual root. Only the former are assumed to have syntactic effects. This basic insight has been foundational for a large body of subsequent work in the area of verbal semantics.



Fillmore begins by using syntactic criteria to distinguish the two classes, which we will refer to for convenience as the \textit{hit} class vs. the \textit{break} class. Subsequent research has identified additional criteria for making this distinction. One of the best-known tests is the \textsc{causative-inchoative} alternation.\footnote{Fillmore (1970: 122–23).} \textit{Break} verbs generally exhibit systematic polysemy between a transitive and an intransitive sense. The intransitive sense has an \textsc{inchoative} (change of state) meaning while the transitive sense has a causative meaning \REF{ex:7.19}. As illustrated in \REF{ex:7.20}, \textit{hit} verbs do not permit this alternation, and often lack intransitive senses altogether.


\ea \label{ex:7.19}
\ea  John broke the window (with a rock).\\
\ex  The window broke.
                       \z
\z

\ea \label{ex:7.20}
\ea  John hit the tree (with a stick).\\
\ex *The tree hit.
                       \z
\z


Additional tests include “body-part possessor ascension” (\ref{ex:7.21}--\ref{ex:7.22}),\footnote{\citet[126]{Fillmore1970}.} the \textsc{conative} alternation (\ref{ex:7.23}--\ref{ex:7.24}),\footnote{\citet{GuersselEtAl1985}; \citet{Levin1993}.} and the \textsc{middle} alternation \REF{ex:7.25}.\footnote{\citet{Fillmore1977}; Hale and \citet{Keyser1987}; \citet{Levin1993}.} Each of these tests demonstrates a difference between the two classes in terms of the potential syntactic functions (subject, direct object, oblique argument, or unexpressed) of the agent and patient.


\ea \label{ex:7.21}
\ea  I \{hit/slapped/struck\} his leg.\\
\ex  I \{hit/slapped/struck\} him on the leg.
                       \z
\z

\ea \label{ex:7.22}
\ea  I \{broke/bent/shattered\} his leg.\\
\ex *I \{broke/bent/shattered\} him on the leg.
                       \z
\z

\ea \label{ex:7.23}
\ea  Mary hit the piñata.\\
\ex  Mary hit at the piñata.\\
\ex  I slapped the mosquito.\\
\ex  I slapped at the mosquito.
                       \z
\z

\ea \label{ex:7.24}
\ea  Mary broke the piñata.\\
\ex  *Mary broke at the piñata.\\
\ex  I cracked the mirror.\\
\ex  *I cracked at the mirror.
                       \z
\z

\ea \label{ex:7.25}
\ea  This glass breaks easily.\\
\ex *This fence hits easily.
                       \z
\z


These various syntactic tests (and others not mentioned here) show a high degree of \textsc{convergence}; that is, the class of \textit{break} verbs identified by any one test matches very closely the class of \textit{break} verbs identified by the other tests. This convergence strongly supports the claim that the members of each class share certain properties in common. \citet[125]{Fillmore1970} suggests that these shared properties are semantic components: “change of state” in the case of the \textit{break} verbs and “surface contact” in the case of the \textit{hit} verbs. Crucially, he provides independent semantic evidence for this claim, specifically evidence that \textit{break} verbs do but \textit{hit} verbs do not entail a change of state \REF{ex:7.26}.\footnote{\citet[125]{Fillmore1970}.} Sentence (\ref{ex:7.26}a) is linguistically acceptable, although surprising based on our knowledge of the world, while (\ref{ex:7.26}b) is a contradiction. Example \REF{ex:7.27} presents similar evidence for the entailment of “surface contact” in the case of the \textit{hit} verbs.
 
\ea \label{ex:7.26}
\ea  I \textit{hit} the window with a hammer; it didn’t faze the window,\\
  but the hammer shattered.\\
\ex *I \textit{broke} the window with a hammer; it didn’t faze the window,\\
  but the hammer shattered.
                       \z
\z

\ea \label{ex:7.27}
\ea *I \textit{hit} the window without touching it.\\
\ex  I \textit{broke} the window without touching it.
                       \z
\z


Without this kind of direct semantic evidence, there is a great danger of falling into circular reasoning, e.g.: \textit{break} verbs permit the causative-inchoative alternation because they contain the component “change of state”, and we know they contain the component “change of state” because they permit the causative-inchoative alternation. As many linguists have learned to our sorrow, it is all too easy to fall into this kind of trap.



While \textit{break} verbs (e.g., \textit{break, bend, fold, shatter, crack}) all share the “change of state” component, they do not all mean the same thing. Each of these verbs has aspects of meaning which distinguish it from all the other members of the class, such as the specific nature of the change and selectional restrictions on the object/patient. \citet[131]{Fillmore1970} suggests that only the shared component of meaning has syntactic consequences; the idiosyncratic aspects of meaning that distinguish one \textit{break} verb from another do not affect the grammatical realization of arguments.



\citet{Levin1993} builds on and extends Fillmore’s study of verb classes in English. In her introduction she compares the \textit{break} and \textit{hit} verbs with two additional classes, \textit{touch} verbs (\textit{touch, pat, stroke, tickle}, etc.) and \textit{cut} verbs (\textit{cut, hack, saw, scratch, slash}, etc.). Using just three of the diagnostic tests discussed above, she shows that each of these classes has a distinctive pattern of syntactic behavior, as summarized in \REF{ex:7.28}. The examples in (\ref{ex:7.29}--\ref{ex:7.31}) illustrate the behavior of \textit{touch} verbs and \textit{cut} verbs.\footnote{Examples adapted from Levin (1993: 6–7).}


\ea English transitive verb classes\footnote{\citet[8]{Levin1993}}\\
\begin{tabularx}{\textwidth}{XXXXX} 
\lsptoprule
& \textit{touch} verbs & \textit{hit} verbs & \textit{cut} verbs & \textit{break} verbs\\\midrule
body-part possessor ascension & \scshape yes & \scshape yes & \scshape yes & \scshape no\\
conative alternation & \scshape no & \scshape yes & \scshape yes & \scshape no\\
middle & \scshape no & \scshape no & \scshape yes & \scshape yes\\
\lspbottomrule
\end{tabularx}
\z

\ea \label{ex:7.29}
\textsc{body-part possessor ascension}:\\
\ea  I touched Bill’s shoulder.\\
\ex  I touched Bill on the shoulder.\\
\ex  I cut Bill’s arm.\\
\ex  I cut Bill on the arm.
                       \z
\z

\ea \label{ex:7.39}  \textsc{conative alternation}:\\
\ea  Terry touched the cat.\\
\ex *Terry touched at the cat.\\
\ex  Margaret cut the rope.\\
\ex  Margaret cut at the rope.
                       \z
\z

\ea \label{ex:7.31}
\textsc{middle}:\\
\ea  The bread cuts easily.\\
\ex *Cats touch easily.
                       \z
\z


Levin proposes the following explanation for these observations. Body-part possessor ascension is possible only for verb classes which share the surface contact component of meaning. The conative alternation is possible only for verb classes whose meanings include both contact and motion. The middle construction is possible only for transitive verb classes whose meanings include a caused change of state. The four classes pattern differently with respect to these tests because each of the four has a distinctive set of meaning components, as summarized in \REF{ex:7.32}.


\ea \label{ex:7.32}
\textbf{shared} \textbf{components of meaning}\footnote{adapted from \citet[268]{Saeed2009}.}\\
\textit{touch} verbs  \textsc{contact}\textit{\\
hit} verbs  \textsc{motion, contact}\textit{\\
cut} verbs  \textsc{motion, contact, change}\textit{\\
break} verbs  \textsc{change}
\z


These verb classes have been found to be grammatically relevant in other languages as well. \citet{Levin2015} cites the following examples: \citet{DeLancey1995,DeLancey2000} on Lhasa Tibetan; \citet{GuersselEtAl1985} on Berber, Warlpiri, and Winnebago; \citet{Kroeger2010} on Kimaragang Dusun; \citet{Vogel2005} on Jarawara.

In the remainder of her book, \citet{Levin1993} identifies 192 classes of English verbs, using 79 diagnostic patterns of \textsc{diathesis} alternations (changes in the way that arguments are expressed syntactically). She shows that these verb classes are supported by a very impressive body of evidence. However, she states that establishing these classes is only a means to an end; the real goal is to understand meaning components:


\begin{quote}
{}[T]here is a sense in which the notion of verb class is an artificial construct. Verb classes arise because a set of verbs with one or more shared meaning components show similar behavior… The important theoretical construct is the notion of meaning component, not the notion of verb class. [\citealt{Levin1993}:9–10]
\end{quote}


Like Fillmore, Levin argues that not all meaning components are grammatically relevant, but only those which define class membership. The aspects of meaning that distinguish one verb from another within the same class (e.g. \textit{punch} vs. \textit{slap}) are idiosyncratic, and do not affect syntactic behavior. Evidence from diathesis alternations can help us determine the systematic, class-defining meaning components, but will not provide an analysis for the idiosyncratic aspects of the meaning of a particular verb.

As noted above, verb meanings cannot be represented as an unordered bundle of components, but must be structured in some way. One popular method, referred to as \textsc{lexical decomposition}, is illustrated in \REF{ex:7.33}. This formula was proposed by Rappaport-Hovav \& \citet[109]{Levin1998} as a partial representation of the systematic components of meaning for verbs like \textit{break}. In this formula, x represents the agent and y the patient. The idiosyncratic aspects of meaning for a particular verb root would be associated with the \textsc{state} predicate (e.g. \textit{broken}, \textit{split}, etc.).


\ea \label{ex:7.33}
{}[[x ACT] CAUSE [BECOME [y <\textsc{state}> ]]]
\z

\section{Conclusion}\label{sec:} %6. /

The idea that verb meanings may consist of two distinct parts, a systematic, class-defining part vs. an idiosyncratic, verb-specific part, is similar to proposals that have been made for content words in general. \citet[131]{Fillmore1970} notes that a very similar idea is found in the general theory of word meaning proposed by \citet{KatzFodor1963}. These authors suggest that word meanings are made up of systematic components of meaning, which they refer to as \textsc{semantic markers}, plus an idiosyncratic residue which they refer to as the \textsc{distinguisher}.



This proposal is controversial, but there do seem to be some good reasons to distinguish systematic vs. idiosyncratic aspects of meaning. As we have seen, Fillmore and Levin demonstrate that certain rules of syntax are sensitive to some components of meaning but not others, and that the grammatically relevant components are shared by whole classes of verbs. Additional motivation for making this distinction comes from the existence of systematic polysemy. It seems logical to expect that rules of systematic polysemy must be stated in terms of systematic aspects of meaning.



However, there is no general consensus as to what the systematic aspects of meaning are, or how they should be represented.\footnote{For one influential proposal, see \citet{Pustejovsky1995}.} Some scholars even deny that components of meaning exist, arguing that word meanings are \textsc{atoms}, in the sense defined in \sectref{sec:4}.\footnote{E.g. \citet{Fodor1975} and subsequent work.} Under this “atomic” view of word meanings, lexical entailments might be expressed in the form of \textsc{meaning postulates} like the following:


\ea
${\forall}$x[STALLION(x) → MALE(x)]\\
${\forall}$x[BACHELOR(x) → ¬MARRIED(x)]
\z


Many scholars do believe that word meanings are built up in some way from smaller elements of meaning. However, a great deal of work remains to be done in determining what those smaller elements are, and how they are combined.



\furtherreading



\citet{Engelberg2011} provides a good overview of the various approaches to and controversies about lexical decomposition and componential analysis. Lyons (1977:317 ff) discusses some of the problems with the binary feature approach to componential analysis. The first chapter of \citet{Levin1993} gives a very good introduction to the Fillmore-type analysis of verb classes and what they can tell us about verb meanings, and \citet{Levin2015} presents an updated cross-linguistic survey of the topic. 


\subsubsection{Discussion exercises:}\label{sec:}
\paragraph{A.  Componential analysis of meaning}

Construct a table of semantic components, represented as binary features, for each of the following sets of words:

\begin{enumerate}
\item \itshape
bachelor, spinster, widow, widower, husband, wife, boy, girl
\item \itshape
walk, run, march, limp, stroll
\item \itshape
cup, glass, mug, tumbler, chalice, goblet, stein
\end{enumerate}

\paragraph{B.  Locative-alternation (“spray-load”) verbs}\footnotemark{}
\footnotetext{Adapted from \citet{Saeed2009}, \chapref{sec:9}.}

Based on the following examples, fill in the table below to show which verbs allow the goal or location argument to be expressed as direct object and which verbs allow the displaced theme argument to be expressed as direct object. Try to formulate an analysis in terms of meaning components to account for the patterns you find in the data.

\ea
\label{ex:key:1}
\ea%1
Jack sprayed paint on the wall.\\
\ex Jack sprayed the wall with paint.
    \z
\z

\ea
\label{ex:key:2}
\ea%2
Bill loaded the cart with apples.\\
\ex Bill loaded the apples onto the cart.
    \z
\z

\ea
\label{ex:key:3}
\ea%3
William filled his mug with guava juice.\\
\ex *William filled guava juice into his mug.
    \z
\z

\ea
    \label{ex:key:4}
\ea%4
 *William poured his mug with guava juice.\\
\ex  William poured guava juice into his mug.
    \z
\z

\ea
\ea%5
    \label{ex:key:5}




          a. Ailbhe pushed the bicycle into the shed.\\
\ex \#Ailbhe pushed the shed with the bicycle.  [different meaning]
    \z
\z

\ea
    \label{ex:key:6}
\ea%6
 Harvey pulled me onto the stage.\\
\ex \#Harvey pulled the stage with me.   [different meaning]
    \z
\z

\ea
    \label{ex:key:7}
\ea%7
 Libby coated the chicken with oil.\\
\ex ?*Libby coated the oil onto the chicken.
    \z
\z

\ea
    \label{ex:key:8}
\ea%8
Mike covered the ceiling with paint.\\
\ex * Mike covered the paint onto the ceiling.
    \z
\z

\begin{tabularx}{\textwidth}{XXX}
\lsptoprule
\bfseries\scshape Verb & \bfseries\scshape Theme = object & \bfseries\scshape Location = object\\\midrule
\itshape load &  & \\
\itshape spray &  & \\
\itshape fill &  & \\
\itshape cover &  & \\
\itshape coat &  & \\
\itshape pour &  & \\
\itshape push &  & \\
\itshape pull &  & \\
\lspbottomrule
\end{tabularx}
\subsubsection{Homework exercises:}\label{sec:}
\paragraph{A. Causative/inchoative alternation}\footnotemark{}
\footnotetext{Adapted from \citet[298]{Saeed2009}, ex. 9.3.}

Levin \& Rappaport Hovav (1995:102–5) propose a semantic explanation for why some change of state verbs participate in the \textsc{causative/inchoative} alternation (\textit{John broke the window} vs. \textit{the window broke}), while others do not. They suggest that verbs which name events that must involve an animate, intentional and volitional agent never appear in the intransitive form. This hypothesis predicts that only (but not necessarily all) verbs which allow an inanimate force as subject should participate in the alternation, as illustrated in (a--b). Your tasks: (i) construct examples like those in (a--b) to test this prediction for the following verbs, and explain what your examples show us about the hypothesis: \textit{melt, write, shrink, destroy}; (ii) Use Levin \& Rappaport Hovav’s hypothesis to explain the contrasts in sentences (c--d).

\begin{enumerate}[label=\alph*.]
\item A terrorist/*tornado assassinated the governor.\\
*The governor assassinated.
\item The storm broke all the windows in my office.\\
All the windows in my office broke.
\item The sky/*table cleared.
\item Paul’s window/*contract/*promise broke.
\end{enumerate}

\chapter{{8}: Grice’s theory of Implicature}

\section{Sometimes we mean more than we say}\label{sec:} %1. /

The story in \REF{ex:8.1} concerns a ship’s captain and his first mate (second in command):

\ea \label{ex:8.1}
\textbf{The Story of the Mate and the Captain} (\citealt{Meibauer2005}, adapted from \citealt{Posner1980})\\
\begin{quote}
A captain and his mate have a long-term quarrel. The mate drinks more rum than is good for him, and the captain is determined not to tolerate this behaviour any longer. When the mate is drunk again, the captain writes in the logbook: “Today, 11th October, the mate is drunk.” When the mate reads this entry during his next watch, he gets angry. Then, after a short moment of reflection, he writes in the logbook: “Today, 14th October, the captain is not drunk.”
\end{quote}
\z


The mate’s log entry communicates something bad and false (namely that the captain is frequently or habitually drunk) by saying something good and true (the captain is not drunk today). It provides a striking example of how widely \textsc{sentence meaning} (the semantic content of the sentence) may differ from \textsc{utterance meaning}. Recall that we defined utterance meaning as “the totality of what the speaker intends to convey by making an utterance;”\footnote{\citet[27]{Cruse2000}.} so utterance meaning includes the semantic content plus any pragmatic meaning created by the use of the sentence in a specific context.



In this chapter and the next we will explore the question of how this kind of context-dependent meaning arises. Our discussion in this chapter will focus primarily on the ground-breaking work on this topic by the philosopher H. Paul Grice. Grice referred to the kind of inference illustrated in \REF{ex:8.1} as a \textsc{conversational implicature}, and suggested that such inferences arise when there is a real or apparent violation of our shared default expectations about how conversations work.



In \sectref{sec:2} we introduce the concept of conversational implicature, and in \sectref{sec:3} we summarize the default expectations about conversation which Grice proposed as a way of explaining these implicatures. In \sectref{sec:4} we distinguish two different types of conversational implicature, and mention briefly a different kind of inference which Grice referred to as \textsc{conventional implicature}. In sections 5–6 we discuss various diagnostic properties of conversational implicatures, and talk about how to distinguish conversational implicatures from entailments and presuppositions.


\section{Conversational implicatures}\label{sec:} %2. /

Let us begin by considering the simple conversation in \REF{ex:8.2}: 


\ea \label{ex:8.2}
\textbf{Arthur}: Can you tell me where the post office is?\\
\textbf{Bill}: I’m a stranger here myself.
\z


As a reply to Arthur’s request for directions, Bill’s statement is clearly intended to mean ‘No, I cannot.’ But the sentence meaning, or semantic content, of Bill’s statement does not contain or entail this intended meaning. The statement conveys the intended meaning only in response to that specific question. In a different kind of context, such as the one in \REF{ex:8.3}, it could be intended to convey a very different meaning: willingness to engage in conversation on a wider range of topics, or at least sympathy for Arthur’s situation.


\ea \label{ex:8.3}
\textbf{Arthur}: I’ve just moved to this town, and so far I’m finding it pretty tedious; I haven’t met a single person who is willing to talk about anything except next week’s local elections.\\
\textbf{Bill}: I’m a stranger here myself.
\z


When the same sentence is used in two different contexts, these are two distinct utterances which may have different utterance meanings. But since the sentence meaning is identical, the difference in utterance meaning must be due to pragmatic inferences induced by the different contexts. As mentioned above, Grice referred to the kind of pragmatic inference illustrated in these examples as \textsc{conversational implicature}. Examples (\ref{ex:8.2}--\ref{ex:8.3}) illustrate the following characteristics of conversational implicatures:


\begin{enumerate}
\item The implicature is different from the literal sentence meaning; in Grice’s terms, what is implicated is different from “what is said”.
\item Nevertheless, the speaker intends for the hearer to understand both the sentence meaning and the implicature; and for the hearer to be aware that the speaker intends this.
\item Conversational implicatures are context-dependent, as discussed above.
\item Conversational implicatures are often unmistakable, but they are not “inevitable”, i.e. they are not logically necessary. In the context of \REF{ex:8.2}, for example, Bill’s statement is clearly intended as a negative reply; but it would not be logically inconsistent for Bill to continue as in \REF{ex:8.4}. In Grice’s terms we say that conversational implicatures are \textsc{defeasible}, meaning that they can be cancelled or blocked when additional information is provided.
\end{enumerate}

\ea \label{ex:8.4}
\textbf{Arthur}: Can you tell me where the post office is?\\
\textbf{Bill}: I’m a stranger here myself; but it happens that I have just come from the post office, so I think I can help you.
\z


Conversational implicatures are not something strange and exotic; they turn out to be extremely common in everyday language use. Once we become aware of them, we begin to find them everywhere. They are an indispensable part of the system we use to communicate with each other.


\section{Grice’s Maxims of Conversation}\label{sec:} %3. /

The connection between what is said and what is implicated, taking context into account, cannot be arbitrary. It must be rule-governed to a significant degree, otherwise the speaker could not expect the hearer to reliably understand the intended meaning.



Grice was not only the first scholar to describe the characteristic features of implicatures, but also the first to propose a systematic explanation for how they work. Grice’s lecture series at Harvard University in 1967, where he laid out his analysis of implicatures, triggered an explosion of interest in and research about this topic. It is sometimes cited as the birth date of Pragmatics as a separate field of study. Of course a number of authors have proposed revisions and expansions to Grice’s model, and we look briefly at some of these in the next chapter; but his model remains the starting point for much current work and is the model that we will focus on in this chapter.



Grice’s fundamental insight was that conversation is a cooperative activity. In order to carry on an intelligible conversation, each party must assume that the other is trying to participate in a meaningful way. This is true even if the speakers involved are debating or quarreling; they are still trying to carry on a conversation. Grice proposed that there are certain default assumptions about how conversation works. He stated these in the form of a general \textsc{Cooperative Principle} \REF{ex:8.5} and several specific sub-principles which he labeled “maxims” \REF{ex:8.6}:


\begin{stylepoints} \label{ex:8.5}
\textbf{The Cooperative Principle} \citep[45]{Grice1975}\\
Make your conversational contribution such as is required, at the stage at which it occurs, by the accepted purpose or direction of the talk exchange in which you are engaged.
\end{stylepoints}

\begin{stylepoints} \label{ex:8.6}
\textbf{The Maxims of Conversation} \citep[45-46]{Grice1975}

QUALITY: Try to make your contribution one that is true.\\
1. Do not say what you believe to be false.\\
2. Do not say that for which you lack adequate evidence.

QUANTITY:\\
1. Make your contribution as informative as is required (for the current purposes of the exchange).\\
2. Do not make your contribution more informative than is required.

RELATION (or RELEVANCE): Be relevant.
 
MANNER: Be perspicuous.\\
1. Avoid obscurity of expression.\\
2. Avoid ambiguity.\\
3. Be brief (avoid unnecessary prolixity).\\
4. Be orderly.
\end{stylepoints}


It is important to remember that Grice did not propose the Cooperative Principle as a code of conduct, which speakers have a moral obligation to obey. A speaker may communicate either by obeying the maxims or by breaking them, as long as the hearer is able to recognize which strategy is being employed. The Cooperative Principle is a kind of background assumption: what is necessary in order to make rational conversation possible is not for the speaker to follow the principle slavishly, but for speaker and hearer to share a common awareness that it exists.



We might draw an analogy with radio waves. Radio signals start with a “carrier wave” having a specific, constant frequency and amplitude. The informative part of the signal, e.g. the audio frequency wave that represents the music, news report, or football match being broadcast, is superimposed as variation in the frequency (for FM) or amplitude (for AM) of the carrier wave. The complex wave form which results is transmitted to receivers, where the intended signal is recovered by “subtracting” the carrier wave. In order for the correct signal to be recovered, the receiver must know the frequency and amplitude of the carrier wave. Furthermore, the receiver must assume that variations from this base frequency and amplitude are intended to be meaningful, and are not merely interference due to lightning, sunspots, or the neighbors’ electrical gadgets.



The analogue of the wave form for pragmatic inferences is the sentence meaning, i.e. the literal semantic content of the utterance. The Cooperative Principle and maxims specify the default frequency and amplitude of the carrier wave. When a speaker appears to violate one of the maxims, a pragmatic inference is created; but this is only possible if the hearer assumes that the speaker is actually being cooperative, and thus the apparent violations are intended to be meaningful.



For example, Bill’s reply to Arthur’s request for directions to the post office in \REF{ex:8.2} appears to violate the maxim of relevance. Arthur might interpret the reply as follows: “Bill’s statement that he is a stranger here has nothing to do with the location of the post office. Bill seems to be violating the maxim of relevance, but I assume that he is trying to participate in a rational conversation; so he must actually be observing the conversational maxims, or at least the Cooperative Principle. I know that strangers in a town typically do not know where most things are located. I believe that Bill knows this as well, and would expect me to understand that his being a stranger makes it unlikely that he can provide the information I am requesting. If his reply is intended to mean ‘No, I cannot,’ then it is actually relevant and there is no violation. So in order to maintain the assumption that Bill is observing the Cooperative Principle, I must assume that this is what he intends to communicate.”



Of course, the sentence meaning is not just a means to trigger implicatures; it is itself part of the meaning which is being communicated. Utterance meaning is composed of the sentence meaning plus any pragmatic inference created by the specific context of use. Grice’s model is intended to explain the pragmatic part of the meaning. In example \REF{ex:8.2}, the answer to Arthur’s literal yes-no question is conveyed by pragmatic inference, while the sentence meaning explains the reason for this answer, and so is felt to be more polite than a blunt “No” would be.



Grice described several specific patterns of reasoning which commonly give rise to conversational implicatures. First, there are cases in which there is an apparent violation, but no maxim is actually violated. Our analysis of example \REF{ex:8.2} was of this type. Bill’s statement \textit{I am a stranger here myself} was an apparent violation of the maxim of relevance, but the implicature that it triggered actually was relevant; so there was no real violation. Two of Grice’s classic examples of this type are shown in (\ref{ex:8.7}--\ref{ex:8.8}).\footnote{\citet[51]{Grice1975}.} In both cases the second speaker’s reply is an apparent violation of the maxim of relevance, but it triggers an implicature that is relevant (\textit{You can buy petrol there} in \REF{ex:8.7}, \textit{Maybe he has a girlfriend in New York} in \REF{ex:8.8}).


\ea \label{ex:8.7}
A: I am out of petrol [=gasoline].\\
B: There is a garage [=service station] around the corner.
\z

\ea \label{ex:8.8}
A: Smith doesn’t seem to have a girlfriend these days.\\
B: He has been paying a lot of visits to New York lately.
\z


Second, Grice noted cases in which an apparent violation of one maxim is the result of conflict with another maxim. He illustrates this type with the example in \REF{ex:8.9}. 


\ea \label{ex:8.9}
A: Where does C live?\\
B: Somewhere in the South of France.
\z


B’s reply here seems to violate the maxim of quantity, specifically the first sub-maxim, since it is not as informative as would be appropriate in this context. A is expected to be able to infer that B cannot be more informative without violating the maxim of quality (second sub-maxim) by saying something for which he lacks adequate evidence. So the intended implicature is, “I do not know exactly where C lives.”



Third, Grice described cases in which one of the maxims is “flouted”, by which he meant a deliberate and obvious violation, intended to be recognized as such. Two of his examples of this type are presented in (\ref{ex:8.10}--\ref{ex:8.11}).


\ea \label{ex:8.10}
A professor is writing a letter of reference for a student who is applying for a job as a philosophy teacher:\\
“Dear Sir, Mr. X’s command of English is excellent, and his attendance at tutorials has been regular. Yours, etc.”
\z

\ea \label{ex:8.11}
Review of a vocal recital:\\
“Miss X produced a series of sounds that corresponded closely with the score of \textit{Home sweet home}.”
\z


The professor’s letter in \REF{ex:8.10} flouts the maxims of quantity and relevance, since it contains none of the information that would be expected in an academic letter of reference. The review in \REF{ex:8.11} flouts the maxim of manner, since there would have been a shorter and clearer way of describing the event, namely “Miss X sang \textit{Home sweet home}.”



As we noted in an earlier chapter, speakers sometimes utter sentences which are tautologies or contradictions. In such cases, the communicative value of the utterance comes primarily from the pragmatic inferences which are triggered; the semantic (i.e. truth conditional) content of the sentence contributes little or nothing. Grice observes that tautologies like those in \REF{ex:8.12} can be seen as flouting the maxim of quantity, since their semantic content is uninformative. Metaphors, irony, and other figures of speech like those in \REF{ex:8.13} can be seen as flouting the maxim of quality, since their literal semantic content is clearly untrue and intended to be recognized as such.


\ea \label{ex:8.12}
\ea War is war.\\
\ex Boys will be boys.
                       \z
\z

\ea \label{ex:8.13}
\ea You are the cream in my coffee.\\
\ex Queen Victoria was made of iron. (\citealt{Levinson1983}:110)\\
\ex A fine friend he turned out to be!
                       \z
\z


Von \citet{FintelMatthewson2008} consider the question of whether Grice’s Cooperative Principle and maxims hold for all languages. Of course, differences in culture, lexical distinctions, etc. will lead to differences in the specific implicatures which arise, since these are calculated in light of everything in the common ground between speaker and hearer.\footnote{See for example \citet{Matsumoto1995}.} They note a single proposed counter example to Grice’s model, from Malagasy \citep{Keenan1974}; but they endorse the response of \citet{Prince1982}, who points out that the speakers in Keenan’s examples actually do obey Grice’s principles, given their cultural values and assumptions. Their conclusion echoes that of \citet[419]{Green1990}:


\begin{quote}
“[I]t would astonish me to find a culture in which Grice’s maxims were not routinely observed, and required for the interpretation of communicative intentions, and all other things being equal, routinely exploited to create implicature.”
\end{quote}

\section{Types of implicatures}\label{sec:} %4. /
\subsection{4.1 Generalized\textmd{ }Conversational Implicature}\label{sec:}

Grice distinguished two different types of conversational implicatures. He referred to examples like those we have considered up to this point as \textbf{\textsc{particularized conversational implicatures}}, meaning that the intended inference depends on particular features of the specific context of the utterance. The second type he referred to as \textbf{\textsc{generalized conversational implicatures}}. This type of inference does not depend on particular features of the context, but is instead typically associated with the kind of proposition being expressed. Some examples are shown in \REF{ex:8.14}.


\ea \label{ex:8.14}
\ea  She gave him the key and he opened the door.\\
\textsc{Implicature}: She gave him the key \textit{and then} he opened the door.\\
\ex  The water is warm.\\
\textsc{Implicature}: The water is not hot.\\
\ex  It is possible that we are related.\\
\textsc{Implicature}: It is not necessarily true that we are related.\\
\ex  Some of the boys went to the rugby match.\\
\textsc{Implicature}: Not all of the boys went to the rugby match.\\
\ex   John has most of the documents.\\
\textsc{Implicature}: John does not have all of the documents.\\
\ex That man is either Martha’s brother or her boyfriend.\\
\textsc{Implicature}: The speaker does not know whether the man is Martha’s brother or boyfriend.\\
\z \z


Generalized conversational implicatures are motivated by the same set of maxims discussed above, but they typically do not involve a violation of the maxims. Rather, the implicature arises precisely because the hearer assumes that the speaker is obeying the maxims; if the implicated meaning were not true, then there would be a violation. In (\ref{ex:8.14}a) for example, assuming that the semantic content of English \textit{and} is simply logical \textit{and} ($\wedge$), the implicated sequential meaning (‘and then’) is motivated by the maxim of manner (sub-maxim: Be orderly). If the actual order of events was not the one indicated by the sequential order of the conjoined clauses, the speaker would have violated this maxim; therefore, unless there is evidence to the contrary, the hearer will assume that the sequential meaning is intended. (We will return in the next chapter to the question of whether this is an adequate analysis of the meaning of English \textit{and}.)



A widely discussed type of generalized conversational implicature involves non-maximal degree modifiers, that is, words which refer to intermediate points on a scale. (Implicatures of this type are often referred to as \textsc{scalar implicatures}.) The word \textit{warm} in (\ref{ex:8.14}b), for example, belongs to a set of words which identify various points on a scale of temperature: \textit{frigid, cold, cool, lukewarm, warm, hot, burning/sizzling/scalding}, etc. The choice of the word \textit{warm} implicates ‘not hot’ by the maxim of quantity. If the speaker knew that the water was hot but only said that it was warm, he would not have been as informative as would be appropriate in most contexts; a hearer stepping into a full bath tub, for example, would be justified in complaining if the water turned out to be painfully hot and not just warm. This inference does not depend on particular features of the context, but is normally triggered by any use of the word \textit{warm} unless something in the context prevents it from arising. The same reasoning applies to \textit{possible} in (\ref{ex:8.14}c) \textit{some} in (\ref{ex:8.14}d), and \textit{three} in (\ref{ex:8.14}e).



The maxim of quantity also motivates the implicature in (\ref{ex:8.14}f), since if the speaker knew which alternative was correct but only made an \textit{or} statement, he would not have been as informative as would be appropriate in most contexts. Again, this inference would normally be triggered by any similar use of the word \textit{or} unless something in the context prevents it from arising.



The indefinite article can trigger generalized conversational implicatures concerning the possessor of the indefinite NP, with different implicatures depending on whether the head noun is alienable as in (\ref{ex:8.15}a--b) or inalienable as in (\ref{ex:8.15}c--d).\footnote{Exx. (\ref{ex:8.15}a--b) are adapted from \citet[56]{Grice1975}.} How to account for this difference is somewhat puzzling.


\ea \label{ex:8.15}
\ea  I walked into a house.\\
\textsc{Implicature}: The house was not my house.
\ex Arthur is meeting a woman tonight.\\
\textsc{Implicature}: The woman is not Arthur’s wife or close relative.
 \ex   I broke a finger yesterday.\\
\textsc{Implicature}: The finger was my finger.
\ex  \textbf{Lady Glossop}: How would you ever support a wife, Mr. Wooster?\\
\textbf{Bertie}: Well, it depends on whose wife it was. I would’ve said a gentle pressure beneath the left elbow when crossing a busy street normally fills the bill.\\
{}[\textit{Jeeves and Wooster}, Season 1, Episode 1; ITV1]
\z
\z

\subsection{Conventional Implicature}\label{sec:} %4.2 /

Grice identified another type of inference which he called \textsc{conventional implicatures}; but he said very little about them, and never developed a full-blown analysis. In contrast to conversational implicatures, which are context-sensitive and motivated by the conversational maxims, conventional implicatures are part of the conventional meaning of a word or construction. This means that they are not context-dependent or pragmatically explainable, and must be learned on a word-by-word basis. However, unlike the kinds of lexical entailments that we discussed in \chapref{sec:6}, conventional implicatures do not contribute to the truth conditions of a sentence, and for this reason have sometimes been regarded as involving pragmatic rather than semantic content.



Grice illustrated the concept of conventional implicature using the conjunction \textit{therefore}. He suggested that this word does not affect the truth value of a sentence; the claim of a causal relationship is only conventionally implicated and not entailed:


\begin{quote}
If I say (smugly), \textit{He is an Englishman; he is, therefore, brave}, I have certainly committed myself, by virtue of the meaning of my words, to its being the case that his being brave is a consequence of (follows from) his being an Englishman. But while I have said that he is an Englishman, and said that he is brave, I do not want to say that I have said (in the favored sense [i.e. as part of the truth-conditional semantic content—PK]) that it follows from his being an Englishman that he is brave, though I have certainly indicated, and so implicated, that this is so. I do not want to say that my utterance of this sentence would be, strictly speaking, false should the consequence in question fail to hold. (\citealt{Grice1975}:44)
\end{quote}


Frege had earlier expressed very similar views concerning words like \textit{still} and \textit{but}, though he never used the term “conventional implicature”. He pointed out that the truth-conditional meaning of \textit{but} is identical to that of \textit{and}. The difference between the two is that \textit{but} indicates a contrast or counter-expectation. But this is only conventionally implicated, in Grice’s terms; if there is in fact no contrast between the two conjuncts, that does not make the sentence false.


\begin{quote}
With the sentence \textit{Alfred has still not come} one really says ‘Alfred has not come’ and, at the same time, hints that his arrival is expected, but it is only hinted. It cannot be said that, since Alfred’s arrival is not expected, the sense of the sentence is therefore false. … The word \textit{but} differs from \textit{and} in that with it one intimates that what follows is in contrast with what would be expected from what preceded it. Such suggestions in speech make no difference to the thought [i.e. the propositional content—PK]. [\citealt{Frege1918}-19/1956]
\end{quote}


A few more examples of conventional implicatures (CI) are given in \REF{ex:8.16}:


\ea \label{ex:8.16}
\ea  I was in Paris last spring \textit{too}.\footnote{\url{http://people.umass.edu/partee/MGU_2009/materials/MGU098_2up.pdf}} \\
CI: some other specific/contextually salient person was in Paris last spring\\
\ex   \textit{Even} Bart passed the test.\footnote{\citet{Potts2007a}.}\\
CI: Bart was among the least likely to pass the test
\z
\z


Conventional implicatures turn out to have very similar properties to certain kinds of presuppositions, and there has been extensive debate over the question of whether it is possible or desirable to distinguish conventional implicatures from presuppositions. We will have more to say about conventional implicatures in \chapref{sec:11}.


\section{Distinguishing features of conversational implicatures}\label{sec:} %5. /

Grice’s analysis of conversational implicatures implies that they will have certain properties which allow us to distinguish them from other kinds of inference. We have already mentioned the most important of these, namely the fact that they are \textsc{defeasible}. This term means that the inference can be cancelled by adding an additional premise. For example, conversational implicatures can be explicitly negated or denied without giving rise to anomaly or contradiction, as illustrated in \REF{ex:8.17}. This makes them quite different from entailments, as seen in \REF{ex:8.18}.


\ea \label{ex:8.17}
\ea  Dear Sir, Mr. X’s command of English is excellent, and his attendance at tutorials has been regular. \textit{And, needless to say, he is highly competent in philosophy}. Yours, etc.
\ex He has been paying a lot of visits to New York lately, \textit{but I don’t think he has a girlfriend there, either}.
\ex John has most of the documents; \textit{in fact, he has all of them}.
\z \z
\ea \label{ex:8.18}
John killed the wasp (\#but the wasp did not die).
\z 


A closely related property is that conversational implicatures are \textsc{suspendable}:\footnote{\citet{Horn1972}; \citet{Sadock1978}.} the speaker may explicitly choose not to commit to the truth or falsehood of the inference, without giving rise to anomaly or contradiction. This is illustrated in (\ref{ex:8.19}a). Again, the opposite is true for entailments, as seen in (\ref{ex:8.19}b).


\ea \label{ex:8.19}
\ea The water must be warm by now, \textit{if not boiling}.\\
\ex \#The water must be warm by now, \textit{if not cold}.
                       \z
\z


Conversational implicatures are \textsc{calculable}, that is, capable of being worked out on the basis of (i) the literal meaning of the utterance, (ii) the Cooperative Principle and its maxims, (iii) the context of the utterance, (iv) background knowledge, and (v) the assumption that (i)--(iv) are available to both participants of the exchange and that they are both aware of this. However, conversational implicatures are also \textsc{indeterminate}: sometimes multiple interpretations are possible for a given utterance in a particular context.



Because conversational implicatures are not part of the conventional meaning of the linguistic expression, and because they are triggered by the semantic content of what is said rather than its linguistic form, replacing words with synonyms, or a sentence with its paraphrase as in \REF{ex:8.20}, will generally not change the conversational implicatures that are generated, assuming the context is identical. Grice used the somewhat obscure term \textsc{nondetachable} to identify this property. He explicitly notes that implicatures involving the maxim of Manner are exceptions to this generalization, since in those cases it is precisely the speaker’s choice of linguistic form which triggers the implicature.\footnote{\citet[58]{Grice1975}.}


\ea \label{ex:8.20}
A: Smith doesn’t seem to have a girlfriend these days.\\
B1: He has been paying a lot of visits to New York lately.\\
B2: He travels to New York quite frequently, I have noticed.
\z


\citet[294]{Sadock1978} noted another useful diagnostic property, namely that conversational implicatures are \textsc{reinforceable}. He used this term to mean that the implicature can be overtly stated without creating a sense of anomalous redundancy (\ref{ex:8.21}a--b). This is another respect in which conversational implicatures differ from entailments (\ref{ex:8.21}c).


\ea \label{ex:8.21}
\ea John is a capable fellow, \textit{but I wouldn’t call him a genius}.\\
\ex Some of the boys went to the soccer match, \textit{but not all}.\\
\ex ?*Some of the boys went to the soccer match, \textit{but not none}.
                       \z
\z

\section{How to tell one kind of inference from another}\label{sec:} %6. /

The table in \REF{ex:8.22} summarizes some of the characteristic properties of entailments, conversational implicatures, and presuppositions.\footnote{Thanks to Seth Johnston for suggesting this type of summary table.} In this section we will work through some examples showing how we can use these properties as diagnostic tools to help us determine which kind of inference we are dealing with in any particular example.



Two general comments need to be kept in mind. First, before we begin applying these tests, it is important to ask whether there is in fact a linguistic inference to be tested. The question is this: if a speaker whom we believe to be truthful and well-informed says \textit{p}, would this utterance in and of itself give us reason to believe \textit{q}? If so, we can apply the tests to determine the nature of the inference from \textit{p} to \textit{q}. But if not, applying the tests will only cause confusion. For example, if our truthful and well-informed speaker says \textit{My bank manager has just been murdered}, it seems reasonable to assume that the bank will soon be hiring a new manager.\footnote{This example comes from \citet[54]{Saeed2009}.} However, this expectation is based on our knowledge of how the world works, and not the meaning of the sentence itself; there is no linguistic inference involved. If the bank owners decided to leave the position unfilled, or even to close that branch office entirely, it would not render the speaker’s statement false or misleading.



Second, any one test may give unreliable results in a particular example, because so many complex factors contribute to the meaning of an utterance. For this reason it is important to use several tests whenever possible, and choose the analysis that best explains the full range of available data. Presuppositions are especially tricky, partly because they are not a uniform class; different sorts seem to behave differently in certain respects. Some specific issues regarding presuppositions are discussed below.


\ea\label{ex:8.22}
\begin{tabularx}{\textwidth}{lXXXX}
\lsptoprule
&  & Entailment & Conversational Implicature & Presupposition\\\midrule
a. & Cancellable\slash defeasible & \scshape no & \scshape yes & sometimes\footnotemark{}\\
b. & Suspendable & \scshape no & \scshape yes & sometimes\\
c. & Reinforceable & \scshape no & \scshape yes & \scshape no\\
d. & Preserved under negation and questioning & \scshape no & \scshape no & \scshape yes\\
\lspbottomrule
\end{tabularx}
\z
\footnotetext{Some presuppositions seem to be cancellable, but only if the clause containing the trigger is negated. Presuppositions triggered by positive statements are generally not cancellable.}

Let us begin with some simple examples. If our truthful and well-informed speaker makes the statement in \REF{ex:8.23}, we would certainly infer that the wasp is dead. We can test to see whether this inference is cancellable/defeasible, as in (\ref{ex:8.23}a); the result is a contradiction. We can test to see whether the inference can be suspended, as in (\ref{ex:8.23}b); the result is quite unnatural. We can test to see whether the inference is reinforceable, as in (\ref{ex:8.23}c); the result is unnaturally redundant.


\begin{stylepoints} \label{ex:8.23}
\textsc{stated}: \textit{John killed the wasp}.\\
\textsc{inferred}: The wasp died.\\
\ea \#John killed the wasp, but the wasp did not die.\\
\ex \#John killed the wasp, but I’m not sure whether the wasp died.\\
\ex ?\#John killed the wasp, and the wasp died.\\
\ex Did John kill the wasp?\\
\ex John did not kill the wasp (and the wasp did not die).
                       \z
\end{stylepoints}


In applying the final test, we are asking whether the same inference is created by a family of related sentences, which includes negation and questioning of the original statement. Clearly if someone asks the question in (\ref{ex:8.23}d), that would not give us any reason to believe that the wasp died. Similarly, the negative statement in (\ref{ex:8.23}e) gives us no reason to believe that the wasp died. We can demonstrate this by showing that it would not be a contradiction to assert, in the same sentence, that the wasp did not die; note the contrast with (\ref{ex:8.23}a), which is a contradiction. We have seen that all four tests in this example produce negative results. This pattern matches the profile of entailment; so we conclude that \textit{John killed the wasp} entails \textit{The wasp died}.



Now let us apply the tests to Grice’s example \REF{ex:8.24}; specifically we will be testing the inference that arises from B’s reply, \textit{There is a garage around the corner}. The sentences in (\ref{ex:8.24}a--c) show that this inference is defeasible (additional information can block the inference from arising), suspendable, and reinforceable. Neither the question in (\ref{ex:8.24}d) nor the negative statement in (\ref{ex:8.24}e) would give A any reason to believe that he could buy petrol around the corner. (The phrase \textit{any more} could be added in (\ref{ex:8.24}e) to make the negative statement sound a bit more natural. In applying these tests, it is important to give the test every opportunity to succeed. Since naturalness is an important criterion for success, it is often helpful to adjust the test sentences as needed to make them more natural, provided the key elements of meaning are not lost or distorted.)


\begin{stylepoints} \label{ex:8.24}
A: \textit{I am out of petrol}.\\
B: \textit{There is a garage around the corner}.\\
\textsc{inferred}: You can buy petrol there.\\
\ea There is a garage around the corner, but they aren’t selling petrol today.\\
\ex There is a garage around the corner, but I’m not sure whether they sell petrol.\\
\ex There is a garage around the corner, and you can buy petrol there.\\
\ex Is there a garage around the corner?\\
\ex There is no garage around the corner (any more).
                       \z
\end{stylepoints}


In this example the first three tests produce positive results, while the last one (the “family of sentences” test) is negative. This pattern matches the profile of conversational implicature; so we conclude that \textit{There is a garage around the corner} (when spoken in the context of A’s statement) conversationally implicates \textit{You can buy petrol there}. Of course, we already knew this, based on our previous discussion. What we are doing here is illustrating and validating the tests by showing how they work with relatively simple cases where we think we know the answer. This gives us a basis for expecting that the tests will work for more complex cases as well.



Finally consider the inference shown in \REF{ex:8.25}. The sentences in (\ref{ex:8.25}a--c) show that this inference is not defeasible (\ref{ex:8.25}a) or reinforceable (\ref{ex:8.25}c), but it is suspendable (\ref{ex:8.25}b). Both the question in (\ref{ex:8.25}d) and the negative statement in (\ref{ex:8.25}e) seem to imply that John used to chew betel nut. These results match the profile of a presupposition, as expected (\textit{stopped chewing} presupposes \textit{used to chew}).


\begin{stylepoints} \label{ex:8.25}
\textsc{stated}: \textit{John has stopped chewing betel nut}.\\
\textsc{inferred}: John used to chew betel nut.\\
\ea \#John has stopped chewing betel nut, and in fact he has never chewed it.\\
\ex John has stopped chewing betel nut, if he (ever/really) did chew it.\\
\ex ?\#John has stopped chewing betel nut, and he used to chew it.\\
\ex Has John stopped chewing betel nut?\\
\ex John has not stopped chewing betel nut.
                       \z
\end{stylepoints}


Recall that we mentioned in \chapref{sec:3} another test which is useful for identifying presuppositions, the “Hey, wait a minute” test.\footnote{Von \citet{Fintel2004}.} If a speaker’s utterance presupposes something that is not in fact part of the common ground, it is quite appropriate for the hearer to object in the way shown in (\ref{ex:8.26}a). However, it is not appropriate for the hearer to object in this way just because the main point of the assertion is not in fact part of the common ground (\ref{ex:8.26}b). In fact, it would be unnatural for the speaker to assert something that is already part of the common ground.


\begin{stylepoints} \label{ex:8.26}
\textsc{statement}: \textit{John has stopped chewing betel nut}.\\
\ea \textsc{response 1}: \textit{Hey, wait a minute, I didn’t know that John used to chew betel nut}!\\
\ex \textsc{response 2}: \#\textit{Hey, wait a minute, I didn’t know that John has stopped\\
  chewing chew betel nut}!
                       \z
\end{stylepoints}


We mentioned above that it is important to use several tests whenever possible, because any one test may run into unexpected complications in a particular context. For example, our discussion in \sectref{sec:4}.1 would lead us to believe that the word \textit{most} should trigger the generalized conversational implicature \textit{not all}. The examples in \REF{ex:8.27} are largely consistent with this prediction. They indicate that the inference is defeasible (\ref{ex:8.27}a), suspendable (\ref{ex:8.27}b), and reinforceable (\ref{ex:8.27}c). However, the “family of sentences” tests produce inconsistent results. The question in (\ref{ex:8.27}d) fails to trigger the inference, as expected, but the negative statement in (\ref{ex:8.27}e) seems to \textit{entail} (not just implicate) that not all of the boys went to the soccer match.


\begin{stylepoints} \label{ex:8.27}
\textsc{stated}: \textit{Most of the boys went to the soccer match}.\\
\textsc{inferred}: Not all of the boys went to the soccer match.\\
\ea Most of the boys went to the soccer match; in fact, I think all of them went.\\
\ex Most of the boys went to the soccer match, if not all of them.\\
\ex Most of the boys went to the soccer match, but not all of them.\\
\ex Did most of the boys go to the soccer match?\\
\ex Most of the boys didn’t go to the soccer match.\\
\ex If most of the boys went to the soccer match, dinner will probably be late this evening.
                       \z
\end{stylepoints}


As mentioned in \chapref{sec:4}, combining clausal negation with quantified noun phrases often creates ambiguity; we see here that it can introduce other complexities as well. This is a situation where preservation under negation is not a reliable indicator. However, other members of the “family of sentences”, including the question (\ref{ex:8.27}d) and conditional clause (\ref{ex:8.27}f), can be used, and show that the inference is not preserved. So the overall pattern of results confirms that this is a conversational implicature.



The table in \REF{ex:8.22} indicates that presuppositions are normally preserved under negation, and this is the first (and often the only) test that many people use for identifying presuppositions. But as we have seen, negating a sentence can introduce new complications. In discussing the presupposition in \REF{ex:8.25} we noted that the negative statement (\ref{ex:8.25}e), repeated here as (\ref{ex:8.28}a), seems to imply that John used to chew betel nut. This is true if the sentence is read with neutral intonation; but if it is read with what \citet{Jespersen1933} calls “the peculiar intonation indicative of contradiction”, indicated in (\ref{ex:8.28}b), it becomes possible to explicitly deny the presupposition without contradiction or anomaly. This is an instance of \textsc{presupposition-cancelling negation}.


\begin{stylepoints} \label{ex:8.28}
\ea John hasn’t stopped chewing betel nut.\\
\ex John hasn’t \textsc{stopped} chewing betel nut, he never \textsc{did} chew it.
                       \z
\end{stylepoints}


\citet{Horn1985,Horn1989} argues that cases of presupposition-cancelling negation like (\ref{ex:8.28}b) involve a special kind of negation which he refers to as \textsc{metalinguistic} \textsc{negation}. Metalinguistic negation is typically used to contradict something that the addressee has just said, implied, or implicitly accepted.\footnote{Karttunen \& Peters (1979: 46–47).} The negated clause is generally spoken with the special intonation pattern mentioned above, and is typically followed by a correction or “rectification” as in (\ref{ex:8.28}b).



Some additional examples of metalinguistic negation are presented in \REF{ex:8.29}. These examples show clearly that metalinguistic negation is different from normal, logical negation which is used to deny the truth of a proposition. If the negation used in these examples was simply negating the propositional content, the sentences would be contradictions, because \textit{horrible} entails \textit{bad}, \textit{all} entails \textit{most}, etc. Horn claims that what is negated in such examples is not the propositional content but the conversational implicature: asserting \textit{bad} implicates \textit{not horrible}; asserting \textit{most} implicates \textit{not all}. Metalinguistic negation is used to reject the statements in the first clause as being inappropriate or “infelicitous”, because they are not strong enough.


\ea \label{ex:8.29}
\ea That [1983] wasn’t a \textsc{bad} year, it was \textsc{horrible}.\footnote{A quote from the famous baseball player Reggie Jackson, cited in \citet[382]{Horn1989}.}\\
\ex I’m not \textsc{hungry}, I’m \textsc{starving}.\\
\ex \textsc{Most} of the boys didn’t go to the soccer match, \textsc{all} of them went.
                       \z
\z


For our present purposes what we need to remember is that, in testing to see whether an inference is preserved under negation (one of the “family of sentences” tests), we must be careful to use normal, logical negation rather than metalinguistic negation.


\section{Conclusion}\label{sec:} %7. /

Conversational implicatures are the paradigm example of a pragmatic inference: meaning derived not from the words themselves but from the way those words are used in a particular context. They are an indispensable part of our everyday communication. In order for a hearer to correctly interpret the part of the speaker’s intended meaning which is not encoded by the words themselves, these implicatures must be derived in a systematic way, based on principles which are known to both speaker and hearer. Grice proposed a fairly simple account of these principles, starting with some basic assumptions about the nature of conversation as a cooperative activity. Some later modifications to Grice’s theory will be mentioned in \chapref{sec:9}.



\furtherreading



\citet[ch. 3]{Levinson1983} and \citet[ch. 2]{Birner2013} present good introductions to Grice’s treatment of conversational implicature. Grice’s most famous papers (e.g. \citeyear{Grice1975,Grice1978,Grice1981}) are also quite readable. (References to more recent work on conversational implicature will provided in the next chapter.)


\subsubsection{Discussion exercises:}\label{sec:}
\paragraph{A. Identifying types of inference}

For each of the examples in (1–4), determine whether the inference triggered by the statement is \textsc{(A)} a \textsc{particularized conversational implicature}, \textsc{(B)} a \textsc{generalized conversational implicature}, \textsc{(C)} a \textsc{presupposition}, (D) an \textsc{entailment}, or (E) none of these.

\begin{stylepoints}
\ea%1
    \label{ex:key:1}




          \textsc{stated}: My mother is the mayor of Waxahachie.\\
\textsc{inferred}: The mayor of Waxahachie is a woman.
    \z
\end{stylepoints}

\begin{stylepoints}
\ea%2
    \label{ex:key:2}




          \textsc{stated}: That man is either Martha’s brother or her boyfriend.\\
\textsc{inferred}: The speaker does not know whether the man is Martha’s brother or boyfriend.
    \z
\end{stylepoints}

\begin{stylepoints}
\ea%3
    \label{ex:key:3}




          \textsc{stated}: My great-grandfather was arrested this morning for drag racing.\\
\textsc{inferred}: I have a great-grandfather.
    \z
\end{stylepoints}

\begin{stylepoints}
\ea%4
    \label{ex:key:4}




          \textsc{stated}: That’s a great joke – Ham, Shem and Japheth couldn’t stop laughing when\\
  they heard it from Noah.\\
\textsc{inferred}: The joke has lost some of its freshness.
    \z
\end{stylepoints}

For each of the sentences in (5), determine what inference is most likely to be triggered by the statement, and what kind of inference it is, using the same five options as above.

\ea
    \label{ex:key:5}
\ea%5
 I didn’t realize that they are husband and wife.
\ex Charles continues to wear a cabbage on his head.
\ex It is possible that we are related.
\ex  Who stole my durian smoothie?
\ex  Q: Who is that guy over there?\\
A: That is the male offspring of my parents.  [Kearns]
\ex  Arthur is almost as unscrupulous as Susan.
\z
\z

What kind of inference is involved in the following joke?


\ea%6
    \label{ex:key:6}
Q: How many months have 28 days? \\
A: All of them.    
\z

\subsubsection{Homework exercises:}\label{sec:}
\paragraph{A. Conversational implicature}

For each pair of sentences, (i) identify the likely implicature carried by B’s reply; (ii) state which maxim is most important in triggering the implicature, and (iii) explain how the implicature is derived.\footnote{adapted from \citet[226]{Saeed2009}, ex. 7.6.}

\ea
\ea  A: Are you coming out for a pint tonight?\\
  B: My in-laws are coming over for dinner.\\
\textsf{\textbf{\textsc{model answer}}}\textsf{: the most likely implicature here is that B is unable to go out with A. It is triggered by the maxims of quantity and relevance: the literal meaning of B’s reply does not provide the information requested (yes or no), and does not seem to be relevant. By assuming that B intends to communicate that he is obligated to eat with his in-laws, A can interpret B’s statement as being both appropriately informative and relevant.}

\ex  A: Who is that couple?\\
  B: That is my mother and her husband.

\ex  A: Did you enjoy having your sister and her family come to visit?\\
  B: The children were perfect angels. We didn’t really want that antique table anyway,\\
    and I’m sure the cat likes to have its tail pulled.

\ex  A: Jones has just taken a second mortgage on his house.\\
  B: I think I saw him at the casino last weekend.

\ex  A: Did you make us a reservation for dinner tonight?\\
  B: I meant to.
\z
\z

\paragraph{B. Presupposition, Entailment, Implicature}\footnotemark{}
\footnotetext{adapted from MIT course notes.}

What is the relation (if any) between each statement and the bracketed statements which follow? Pick one of the following four answers:\\
  Presupposition; Entailment; Conversational Implicature; no inference.

\ea%1
    \label{ex:key:1}

          John is allegedly a good player.

{}[John is a good player.]
    \z

\ea%2
    \label{ex:key:2}
          Oscar and Jenny are middle-aged.

{}[Jenny is middle-aged.]
    \z

\ea%3
    \label{ex:key:3}
          Maria is an Italian radiologist.

\ea {}[a. Some Italian is a radiologist.]

\ex {}[b. Maria is Italian.]
    \z \z

\ea%4
    \label{ex:key:4}
    Not everyone will get the correct answer.

{}[Someone will get the correct answer.]
    \z

\ea%5
    \label{ex:key:5}
    Pete installed new cabinets after Hans painted the walls.

{}[Hans painted the walls.]
    \z

\ea%6
    \label{ex:key:6}
          Dempsey and Tunney fought in Philadelphia in \citealt{September1926}.

{}[Dempsey and Tunney fought each other.]
    \z

\ea%7
    \label{ex:key:7}
          John believes that pigs do not have wings.

{}[Pigs do not have wings.]
    \z

\ea%8
    \label{ex:key:8}




          John realizes that pigs do not have wings.

{}[Pigs do not have wings.]
    \z

\ea%9
    \label{ex:key:9}
    Don is at home or at work.

{}[a. Don is at home.]

{}[b. I don't know whether Don is at home or at work.]
    \z

\ea%10
    \label{ex:key:10}
          My older brother called.

{}[I have an older brother.]
    \z

\ea%11
    \label{ex:key:11}




          Max has quit jogging, at least until his ankle heals.

{}[a. Max does not jog now.]

{}[b. Max used to jog.]
    \z

\chapter{{9}: Pragmatic inference after Grice}

\section{Introduction}\label{sec:} %1. /

Grice’s work on implicatures triggered an explosion of interest in pragmatics. In the subsequent decades, a wide variety of applications, extensions, and modifications of Grice’s theory have been proposed.



One focus of the theoretical discussion has been the apparent redundancy in the set of maxims and sub-maxims proposed by Grice. Many pragmaticists have argued that the same work can be done with fewer maxims.\footnote{See Birner (2013, ch. 3) for a good summary of the competing positions on this issue.} In the extreme case, proponents of Relevance Theory have argued that only the Principle of Relevance is needed.



Rather than focusing on such theoretical issues directly, in this chapter we will discuss some of the analytical questions that have been of central importance in the development of pragmatics after Grice. In \sectref{sec:2} we return to the question raised in \chapref{sec:4} concerning the degree to which the English words \textit{and}, \textit{or}, and \textit{if} have the same meanings as the corresponding logical operators. Grice himself suggested that some apparently distinct “senses” of these words could be analyzed as generalized conversational implicatures. \sectref{sec:key:3} discusses a type of pragmatic “enrichment” that seems to be required in order to determine the truth-conditional meaning of a sentence. \sectref{sec:key:4} discusses how the relatively clean and simple distinction between semantics vs. pragmatics which we have been assuming up to now is challenged by recent work on implicatures.


\section{Meanings of English words vs. logical operators}\label{sec:} %2. /

As we hinted in \chapref{sec:4}, the logical operators $\wedge$ ‘and’, $\vee$ ‘or’, and → ‘if…then’ seem to have a different and often narrower range of meaning than the corresponding English words. A number of authors have claimed that the English words are ambiguous, with the logical operators corresponding to just one of the possible senses. Grice argued that the English words actually have only a single sense, which is more or less the same as the meaning of the corresponding logical operator, and that the different interpretations arise through pragmatic inferences. Before we examine these claims in more detail, we will first illustrate the variable interpretations of the English words, in order to show why such questions arise in the first place.



Let us begin with \textit{and}.\footnote{We focus here on the use of \textit{and} to conjoin two clauses (or VPs), since this is closest to the function of logical $\wedge$. We will not be concerned with coordination of other categories in this chapter.} The truth table in \chapref{sec:4} makes it clear that logical $\wedge$ is commutative; that is, \textit{p$\wedge$}\textit{q} is equivalent to \textit{q$\wedge$}\textit{p}. This is also true for some uses of English \textit{and}, such as \REF{ex:9.1}. In other cases, however, such as (\ref{ex:9.2}--\ref{ex:9.4}), reversing the order of the clauses produces a very different interpretation.


\ea \label{ex:9.1}
\ea The Chinese invented the folding umbrella and the Egyptians invented the sailboat.\\
\ex The Egyptians invented the sailboat and the Chinese invented the folding umbrella
                       \z
\z

\ea \label{ex:9.2}
\ea She gave him the key and he opened the door.\\
\ex He opened the door and she gave him the key.
                       \z
\z

\ea \label{ex:9.3}
\ea The Lone Ranger jumped onto his horse and rode into the sunset.\footnote{\citet[56]{Kempson1975}, cited in \citet{Gazdar1979}.}\\
\ex ?The Lone Ranger rode into the sunset and jumped onto his horse.
                       \z
\z

\ea \label{ex:9.4}
\ea The janitor left the door open and the prisoner escaped.\\
\ex ?The prisoner escaped and the janitor left the door open.
                       \z
\z


It has often been noted that when \textit{and} conjoins clauses which describe specific events, as (\ref{ex:9.2}--\ref{ex:9.3}), there is a very strong tendency to interpret it as meaning ‘and then’, i.e., to assume a sequential interpretation. When the second event seems to depend on or follow from the first, as in (\ref{ex:9.4}a), there is a tendency to assume a causal interpretation, ‘and therefore’. The question to be addressed is, do such examples prove that English \textit{and} is ambiguous, having two or three (or more) distinct senses?



We stated in \chapref{sec:4} that the $\vee$ of standard logic is the “inclusive or”, corresponding to the English \textit{and/or}. We also noted that the English word \textit{or} is often used in the “exclusive” sense (XOR), meaning ‘either … or … but not both’. Actually either interpretation is possible, depending on the context, as illustrated in \REF{ex:9.5}. (The reader should determine which of these examples contains an \textit{or} that would most naturally be interpreted with the exclusive reading, and which with the inclusive reading.) Does this variable interpretation mean that English \textit{or} is ambiguous? 


\ea \label{ex:9.5}
\ea Every year the Foundation awards a scholarship to a student of Swedish\\
  or Norwegian ancestry.\\
\ex You can take the bus or the train and still arrive by 5 o’clock.\\
\ex If the site is in a particularly sensitive area, or there are safety considerations,\\
  we can refuse planning permission.\footnote{\citet[113]{Saeed2009}.}\\
\ex Stop or I’ll shoot!\footnote{\citet[113]{Saeed2009}.}
                       \z
\z


Finally let us briefly consider the meaning of material implication (→) compared with English \textit{if}. If these two meant the same thing, then according to the truth table for material implication in \chapref{sec:4}, all but one of the sentences in \REF{ex:9.6} should be true. (The reader can refer to the truth table to determine which of these sentences is predicted to be false.) However, most English speakers find all of these sentences very odd; many speakers are unwilling to call any of them true.


\ea \label{ex:9.6}
\ea If Socrates was a woman then $1+1=3$.\footnote{\url{http://en.wikipedia.org/wiki/Material_conditional}} \\
\ex If 2 is odd then 2 is even.\footnote{\url{http://en.wikipedia.org/wiki/Material_conditional}}\\
\ex If a triangle has three sides then the moon is made of green cheese.\\
\ex If the Chinese invented gunpowder then Martin Luther was German.
                       \z
\z


Similarly, analyzing English \textit{if} as material implication in \REF{ex:9.7} would predict some unlikely inferences, based on the rule of \textit{modus tollens}.


\ea \label{ex:9.7}
\ea If you’re hungry, there’s some pizza in the fridge.\\
  (predicted inference: \#If there’s no pizza in the fridge, then you’re not hungry.)\\
\ex If you really want to know, I think that dress is incredibly ugly.\\
  (predicted inference: \#If I don’t think that dress is ugly,\\
    then you don’t really want to know.)
                       \z
\z


Part of the oddness of the “true” sentences in \REF{ex:9.6} relates to the fact that material implication is defined strictly in terms of truth values; there does not have to be any connection between the meanings of the two propositions. English \textit{if}, on the other hand, is normally used only where the two propositions do have some sensible connection. Whether this preference can be explained purely in pragmatic terms is an interesting issue, as is the question of how many senses we need to recognize for English \textit{if} and whether any of these senses are equivalent to →. We will return to these questions in \chapref{sec:19}. In the present chapter we focus on the meanings of \textit{and} and \textit{or}.


\subsection{2.1 On the ambiguity of \textit{and}}\label{sec:} 

In \chapref{sec:8} we mentioned that the sequential (‘and then’) use of English \textit{and} can be analyzed as a generalized conversational implicature motivated by the maxim of manner, under the assumption that its semantic content is simply logical \textit{and} ($\wedge$). An alternative analysis, as mentioned above, involves the claim that English \textit{and} is polysemous, with logical \textit{and} ($\wedge$) and sequential ‘and then’ as two distinct senses. Clearly both uses of \textit{and} are possible, given the appropriate context; example (\ref{ex:9.8}a) (like (\ref{ex:9.1}a) above) is an instance of the logical \textit{and} use, while (\ref{ex:}b) (like (\ref{ex:}b-c) above) is most naturally interpreted as involving the sequential ‘and then’ use. The question is whether we are dealing with semantic ambiguity (two distinct senses) or pragmatic inference (one sense plus a potential conversational implicature). How can we decide between these two analyses?


\ea \label{ex:9.8}
\ea Hitler was Austrian and Stalin was Georgian.\\
\ex They got married and had a baby.
                       \z
\z


\citet{Horn2004} mentions several arguments against the lexical ambiguity analysis for \textit{and}:


\begin{enumerate}[label=\roman*.]
\item The same two uses of \textit{and} are found in most if not all languages. Under the semantic ambiguity analysis, the corresponding conjunction in (almost?) every language would just happen to be ambiguous in the same way as in English.
\item No natural language contains a conjunction \textit{shmand} that would be ambiguous between “and also” and “and earlier” readings so that \textit{They had a baby shmand they got married} would be interpreted either atemporally (logical \textit{and}) or as “They had a baby and, before that, they got married.”
\item Not only temporal but causal asymmetry (‘and therefore’, illustrated in (\ref{ex:9.1}d)) would need to be treated as a distinct sense. And a variety of other uses (involving “stronger” or more specific uses of the conjunction) arise in different contexts of utterance. How many senses are we prepared to recognize?
\item The same “ambiguity” exhibited by \textit{and} arises when two clauses describing related events are simply juxtaposed (\textit{They had a baby. They got married.}). This suggests that the sequential interpretation is not in fact contributed by the conjunction \textit{and}.
\item The sequential ‘and then’ interpretation is defeasible, as illustrated in \REF{ex:9.9}. This strongly suggests that we are dealing with conversational implicature rather than semantic ambiguity.
\end{enumerate}

\ea \label{ex:9.9}
They got married and had a baby, but not necessarily in that order.
\z


Taken together, these arguments seem quite persuasive. They demonstrate that English \textit{and} is not polysemous; its semantic content is logical \textit{and} ($\wedge$). The sequential ‘and then’ use can be analyzed as a generalized conversational implicature.


\subsection{2.2 On the ambiguity of \textit{or}}\label{sec:}

As noted in \chapref{sec:4}, similar questions arise with respect to the meaning(s) of \textit{or}. The English word \textit{or} can be used in either the inclusive sense ($\vee$) or the exclusive sense (XOR). The inclusive reading is most likely in (\ref{ex:9.10}a--b), while the exclusive reading is most likely in (\ref{ex:9.10}c--d).


\ea \label{ex:9.10}
\ea Mary has a son or daughter.\footnote{Barbara Partee, 2004 lecture notes. \url{http://people.umass.edu/partee/RGGU_2004/RGGU047.pdf}} \\
\ex We would like to hire a sales manager who speaks Chinese or Korean.\\
\ex I can’t decide whether to order fried noodles or pizza.\\
\ex Stop or I’ll shoot!\footnote{\citet[113]{Saeed2009}.}
                       \z
\z


Barbara Partee points out that examples like \REF{ex:9.11} are sometimes cited as sentences where only the exclusive reading of \textit{or} is possible; but in fact, such examples do not distinguish the two senses. These are cases where our knowledge of the world makes it clear that both alternatives cannot possibly be true. She says that such cases involve “intrinsically mutually exclusive alternatives”. Because we know that \textit{p$\wedge$q} cannot be true in such examples, \textit{p$\vee$q} and \textit{pXORq} are indistinguishable; if one is true, the other must be true as well.


\ea \label{ex:9.11}
\ea Mary is in Prague or she is in Stuttgart.\footnote{Barbara Partee, 2004 lecture notes. \url{http://people.umass.edu/partee/RGGU_2004/RGGU047.pdf}} \\
\ex Christmas falls on a Friday or Saturday this year.
                       \z
\z


\citet{Grice1978} argues that English \textit{or}, like \textit{and}, is not polysemous. Rather, its semantic content is inclusive \textit{or} ($\vee$), and the exclusive reading arises through a conversational implicature motivated by the maxim of quantity.



In fact, using \textit{or} can trigger more than one implicature. If a speaker says \textit{p or q} but actually knows that p is true, or that q is true, he is not being as informative as required or expected. So the statement \textit{p or q} triggers the implicature that the speaker does not know p to be true or q to be true. By the same reasoning, it triggers the implicature that the speaker does not know either p or q individually to be false. Now if p and q are both true, and the speaker knows it, it would be more informative (and thus expected) for the speaker to say \textit{p and q}. If he instead says \textit{p or q}, he is violating the maxim of quantity. Thus the statement \textit{p or q} also triggers the implicature that the speaker is not in a position to assert \textit{p and q}.



So in contexts where the speaker might reasonably be expected to know if \textit{p and q} were true, the statement \textit{p or q} will trigger the implicature that \textit{p and q} is not true, which produces the exclusive reading. When nothing can be assumed about the speaker’s knowledge, it is harder to see how to derive the exclusive reading from Gricean principles; several different explanations have been proposed. But another reason for thinking that the exclusive reading arises through a conversational implicature is that it is defeasible, e.g. \textit{I will order either fried noodles or pizza; in fact I might get both}.



Gazdar (1979:81–82) presents another argument against analyzing English \textit{or} as being polysemous. If \textit{or} is ambiguous between an inclusive and an exclusive sense, then when sentences containing \textit{or} are negated, the result should also be ambiguous, with senses corresponding to \textit{¬(p$\vee$q)} vs. \textit{¬(pXORq)}. The crucial difference is that \textit{¬(pXORq)} will be true and \textit{¬(p$\vee$q)} false if \textit{p$\wedge$q} is true. (The reader should consult the truth tables in \chapref{sec:4} to see why this is the case.) For example, if \textit{or} were ambiguous, sentence (\ref{ex:9.12}a) should allow a reading which is true if Mary has both a son and a daughter, and (\ref{ex:9.12}b) should allow a reading under which I would allow my daughter to marry a man who both smokes and drinks. However, for most English speakers these readings of (\ref{ex:9.12}a--b) are not possible, at least when read the sentences are with normal intonation.

 
\ea \label{ex:9.12}
\ea Mary doesn’t have a son or daughter.\footnote{Barbara Partee, 2004 lecture notes. \url{http://people.umass.edu/partee/RGGU_2004/RGGU047.pdf}} \\
\ex The man who marries my daughter must not smoke or drink.
                       \z
\z


\citet[47]{Grice1978}, in the context of discussing the meaning of \textit{or}, proposed a principle which he called \textbf{Modified Occam’s Razor}: “Senses are not to be multiplied beyond necessity.” This principle would lead us to favor an analysis of words like \textit{and} and \textit{or} as having only a single sense, with additional uses being derived by pragmatic inference, unless there is clear evidence in favor of polysemy.


\section{Explicatures: bridging the gap between what is said vs. what is implicated}\label{sec:} %3. /

Grice’s model seems to assume that the speaker meaning (total meaning that the speaker intends to communicate) is the sum of the sentence meaning (“what is said”, i.e. the meaning linguistically encoded by the words themselves) plus implicatures. Moreover, implicatures were assumed not to affect the truth value of the proposition expressed by the sentence; truth values were assumed to depend only on sentence meaning.\footnote{Of course, the implicatures themselves also have propositional content, which may be true or false/misleading even if the literal sentence meaning is true.}



In many cases, however, the meaning linguistically encoded by the words themselves does not amount to a complete proposition, and so cannot be evaluated as being either true or false. Grice recognized that the proposition expressed by a sentence like (\ref{ex:9.13}a) is not complete, and its truth value cannot be determined, until the referents of pronouns and deictic elements are specified. Most authors also assume that any potential ambiguities in the linguistic form (like the syntactic and lexical ambiguities in b) must be resolved before the propositional content and truth conditions of the sentence can be determined.


\ea \label{ex:9.13}
\ea She visited me here yesterday.\\
\ex Old men and women gathered at the bank.
                       \z
\z


Determining reference and disambiguation both depend on context, and so involve a limited kind of pragmatic reasoning. However, it turns out that there are many cases in which more significant pragmatic inferences are required in order to determine the propositional content of the sentence. Kent \citet{Bach1994} identifies two sorts of cases where this is needed: “Filling in is needed if the sentence is semantically \textsc{under-determinate}, and fleshing out will be needed if the speaker cannot plausibly be supposed to mean just what the sentence means.”



The first type, which Bach refers to as \textsc{semantic under-determination}, involves sentences which fail to express a complete proposition (something capable of being true or false), even after the referents of pronouns and deictic elements have been determined and ambiguities resolved; some examples are presented in \REF{ex:9.14}.\footnote{Examples \ref{ex:}–\ref{ex:}) are adapted from \citet{Bach1994}.}


\ea \label{ex:9.14}
\ea Steel isn’t strong enough.\\
\ex Strom is too old.\\
\ex The princess is late.\\
\ex Tipper is ready.
                       \z
\z


In these cases a process of \textsc{completion} (or “filling in” the missing information) is required to produce a complete proposition. This involves adding information to the propositional meaning which is unexpressed but implicit in the original sentence, as indicated in \REF{ex:9.15}. The hearer must be able to provide this information from context and/or knowledge of the world. The truth values of these sentences can only be determined after the implicit constituent is added to the overtly expressed meaning.


\ea \label{ex:9.15}
\ea Steel isn’t strong enough [to stop this kind of anti-tank missile].\\
\ex Strom is too old [to be an effective senator].\\
\ex The princess is late [for the party].\\
\ex Tipper is ready [to dance].
                       \z
\z


The under-determination of the sentences in \REF{ex:9.14} is not due to syntactic deletion or ellipsis; they are semantically incomplete, but not syntactically incomplete. The examples in (\ref{ex:9.16}--\ref{ex:9.17}) show that the potential for occurring in such constructions may be lexically specific, and that close synonyms may differ in this respect.


\ea \label{ex:9.16}
\ea The king has arrived. [at the palace]\\
\ex *The king has reached.
                       \z
\z

\ea \label{ex:9.17}
\ea Al has finished. [speaking]\\
\ex *Al has completed.
                       \z
\z


The second type of sentence that Bach discusses involves those in which “there is already a complete proposition, something capable of being true or false (assuming linguistically unspecified references have been assigned and any ambiguities have been resolved), albeit not the one that is being communicated by the speaker.” For example, imagine that a mother says (\ref{ex:9.18}a) to her young son who is crying loudly because he cut his finger.


\ea \label{ex:9.18}
\ea You’re not going to die.\\
\ex You’re not going to die. [from this cut]
                       \z
\z


Clearly she does not intend to promise immortality, although that is what the literal meaning of her words seems to say. In order to determine the intended propositional content of the sentence, the meaning has to be \textsc{expanded} (or “fleshed out”) as shown in (\ref{ex:9.18}b). Once again, the hearer must be able to provide this additional information from context and/or knowledge of the world. A more complex kind of pragmatic reasoning is required here than would be involved in assigning referents to deictic elements or resolving lexical ambiguities. Further examples are provided in \REF{ex:}, illustrating how identical sentence structures can be expanded differently on the basis of knowledge about the world.


\ea \label{ex:9.18}
\ea I have eaten breakfast. [today]\\
\ex I have eaten caviar. [before]\\
\ex I have nothing to wear. [nothing appropriate for a specific event]\\
\ex I have nothing to repair. [nothing at all]
                       \z
\z


Bach uses the term \textsc{impliciture} to refer to the kinds of inference illustrated in this section. The choice of this label is not ideal, because the words \textit{impliciture} and \textit{implicature} look so much alike. A very similar concept is discussed within Relevance Theory under the label \textsc{explicature},\footnote{\citet{SperberWilson1986}; \citet{Carston1988}.} expressing the idea that the overtly expressed content of the sentence needs to be explicated in order to arrive at the full sentence meaning intended by the speaker. In the discussion that follows we will adopt the term \textsc{explicature}.\footnote{We are ignoring for now the relatively minor differences between Bach’s notion of impliciture and the Relevance Theory notion of explicature; see \citet{Bach2010} for discussion.}



\citet[11]{Bach1994} describes the difference between “impliciture” (=explicature) and implicature as follows:


\begin{quote}
Although both impliciture and implicature go beyond what is explicit in the utterance, they do so in different ways. An implicatum is completely separate from what is said and is inferred from it (more precisely, from the saying of it). What is said is one proposition and what is communicated in addition to that is a conceptually independent proposition, a proposition with perhaps no constituents in common with what is said... 
\end{quote}

\begin{quote}
In contrast, implicitures are built up from the explicit content of the utterance by conceptual strengthening … which yields what would have been made fully explicit if the appropriate lexical material had been included in the utterance. Implicitures are, as the name suggests, implicit in what is said, whereas implicatures are implied by (the saying of) what is said. 
\end{quote}


In other words, implicatures are distinct from sentence meaning. They are communicated in addition to the sentence meaning and have independent truth values. A true statement could trigger a false implicature, or vice versa. Explicatures are quite different. The truth value of the sentence cannot be determined until the explicatures are added to the literal meanings of the words.



Since explicatures involve pragmatic reasoning, we must recognize the fact that pragmatic inferences can affect truth-conditional content. Further evidence that supports this same conclusion is discussed in the following section.


\section{Implicatures and the semantics/pragmatics boundary}\label{sec:} %4. /

In \chapref{sec:1} we defined the semantic content of an expression as the meaning that is associated with the words themselves, independent of context. We defined pragmatic meaning as the meaning which arises from the context of the utterance. We have implicitly assumed that the truth conditions of a sentence depend only on the “semantic content” or sentence meaning, and not on pragmatic meaning. Many authors have made the same assumption, using the term “truth conditional meaning” as a synonym for “sentence meaning”. However, our discussion of explicatures has demonstrated that this view is too simplistic. Additional challenges to this simplistic view arise from research on implicatures.



As discussed in \chapref{sec:8}, the conventional implicatures associated with words like \textit{but} or \textit{therefore} are part of the conventional meaning of these words, and not context-dependent; they would be part of the relevant dictionary definitions and must be learned on a word-by-word basis. Nevertheless, both Frege and Grice argued that these conventional implicatures do not contribute to the truth conditions of a sentence. So conventional meaning is not always truth-conditional. We will discuss this issue in more detail in \chapref{sec:11}.



The opposite situation has been argued to hold in the case of generalized conversational implicatures. In \sectref{sec:2} above we presented compelling evidence which shows that the sequential ‘and then’ use of \textit{and} is not due to lexical ambiguity (polysemy), but must be a pragmatic inference. It is often cited as a paradigm example of generalized conversational implicature. However, as noted by \citet{Levinson1995,Levinson2000} among others, this inference does affect the truth conditions of the sentence in examples like (\ref{ex:9.20}--\ref{ex:9.21}). Sentence (\ref{ex:9.20}a) could be judged to be true in the same context where (\ref{ex:9.20}b) is judged to be false. This difference can only be due to the sequential interpretation of \textit{and}; if \textit{and} means only $\wedge$, then the two sentences are logically equivalent. Similarly, if \textit{and} means only $\wedge$, then \REF{ex:9.21} should be a contradiction; the fact that it is not can only be due to the sequential interpretation of \textit{and}.


\ea \label{ex:9.20}
\ea  If the old king has died of a heart attack and a republic has been declared, then Tom will be quite content.\footnote{\citet[58]{Cohen1971}.}
\ex  If a republic has been declared and the old king has died of a heart attack, then Tom will be quite content.\footnote{\citet[69]{Gazdar1979}.}
\z \z

\ea \label{ex:9.21}
If he had three beers and drove home, he broke the law; but\\
if he drove home and had three beers, he did not break the law.
\z


Such examples have been extensively debated, and a variety of analyses have been proposed. For example, proponents of Relevance Theory argue that the sequential ‘and then’ use of \textit{and} is an explicature: a pragmatic inference that contributes to truth conditions.\footnote{\citet{Carston1988,Carston2004}} A similar analysis is proposed for most if not all of the inferences that Grice and the “neo-Griceans” have identified as generalized conversational implicatures: within Relevance Theory they are generally treated as explicatures.



This controversy is too complex to address in any detail here, but we might make one observation in passing. At the beginning of \chapref{sec:8} we provided an example (the story of the captain and his mate) of how we can use a true statement to implicate something false. That example involved a particularized conversational implicature, but it is possible to do the same thing with generalized conversational implicatures as well. The following example involves a scalar implicature. It is taken from a news story about how Picasso’s famous mural “Guernica” was returned to Spain after Franco’s death. The phrase \textit{Not all of them} in this context implicates \textit{not none} (that is, ‘I have some of them’) by the maxim of Quantity, because \textit{none} is a stronger (more informative) term than \textit{not all}.


\ea \label{ex:9.22}
  To demonstrate that the Spanish Government had in fact paid Picasso to paint the mural in 1937 for the Paris International Exhibition, Mr. Fernandez Quintanilla had to secure documents in the archives of the late Luis Araquistain, Spain’s Ambassador to France at the time. But Araquistain’s son, poor and opportunistic, demanded \$2 million for the archives, which Mr. Fernandez Quintanilla rejected as outrageous. He managed, however, to obtain from the son photocopies of the pertinent documents, which in 1979 he presented to Roland Dumas [Picasso’s lawyer]… “This changes everything,” a startled Mr. Dumas told the Spanish envoy when he showed him the photocopies of the Araquistain documents. “You of course have the originals?” the lawyer asked casually. “\textbf{\textit{Not all of them}},” replied Mr. Fernandez Quintanilla, not lying but not telling the truth, either.\\
   {}[\textit{The New York Times}, November 2, 1981; cited in \citealt{Horn1992}]
\z


Mr. Fernandez Quintanilla was not lying, because the literal sentence meaning of his statement was true. But he was not exactly telling the truth either, because his statement triggered (and was clearly intended to trigger) an implicature that was false; in fact he had none of the originals.



Such examples show that generalized conversational implicatures can be used to communicate false information, even when the literal meaning of the sentence is true. It would be hard to account for this fact if these generalized conversational implicatures are considered to be explicatures, because explicatures do not have a truth value that is independent of the truth value of the literal sentence meaning. Rather, explicatures represent inferences that are needed in order to determine the truth value of the sentence.


\subsection{Why numeral words are special}\label{sec:} %4.1 /

Scalar implicatures have received an enormous amount of attention in the recent pragmatics literature. Many early discussions of scalar implicatures relied heavily on examples involving cardinal numbers, which seem to form a natural scale (1, 2, 3, …). However, various authors have pointed out that numbers behave differently from other scalar terms.



\citet{Horn2004} uses examples (\ref{ex:9.23}--\ref{ex:9.25}) to bring out this difference. Because \textit{all} is a stronger term than \textit{many} within the scale <\textit{none, some, many, all}>, A’s use of \textit{many} in \REF{ex:9.23} entails ‘(at least) many’ and implicates ‘not all’.\footnote{\textit{Many} is used here in its proportional sense; see \chapref{sec:14} for discussion.} B’s reply states that the implicature does not in fact hold in the current situation; but this does not render the propositional content of the sentence false. That is why it would be unnatural for B to begin the reply with \textit{No}, as in B1. The acceptability of reply B2 follows from the fact that implicatures are defeasible.


\ea \label{ex:9.23}
A: Did many of the guests leave?\\
B1: ?No, all of them.\\
B2: Yes, (in fact) all of them.
\z


If numerals behaved in the same way as other scalars, we would expect A’s use of \textit{two} in \REF{ex:9.24} to entail ‘at least two’ and implicate ‘not more than two’. However, if B actually does have more than two children, it seems to be more natural here for B to reply with \textit{No} rather than \textit{Yes}. This indicates that B is rejecting the literal propositional content of the question, not an implicature.


\ea \label{ex:9.24}
A: Do you have two children?\\
B1: No, three.\\
B2: ?Yes, (in fact) three.
\z


Such examples suggest that numerals like \textit{two} allow two distinct readings: an ‘at least 2’ reading vs. an ‘exactly 2’ reading, and that neither of these is derived as an implicature from the other. A’s question in \REF{ex:9.24} is most naturally interpreted as involving the ‘exactly’ reading. However, there are certain contexts (such as discussing a government subsidy that is available for families with two or more children) in which the ‘at least’ reading would be preferred, and in such contexts reply B2 would be more natural.



Example (\ref{ex:9.25}a) is acceptable under the ‘exactly 3’ reading of the numeral, under which \textit{not three} is judged to be true whether the actual number is more than three or less than three. The fact that (\ref{ex:}b) is unacceptable shows that the word \textit{like} does not have an ‘exactly (or merely) like’ reading. Based on the scale <\textit{hate, dislike, neutral, like, love/adore}>, using the word \textit{like} entails ‘at least like (= have positive feelings)’ and implicates ‘not more than like (not love/adore)’. Sentence (\ref{ex:}b) attempts to negate the both the entailment and the implicature at the same time, and the result is unacceptable.\footnote{Of course, as pointed out at the end of \chapref{sec:8}, given the right context and using a special marked intonation it is sometimes possible to negate the implicature alone, as in: “She didn’t \textsc{líke} the movie — she \textsc{adóred} it.”}


\ea \label{ex:9.25}
\ea Neither of us has three kids — she has two and I have four.\\
\ex \#Neither of us liked the movie — she adored it and I hated it.
                       \z
\z


\citet{Horn1992} notes several other properties which set numerals apart from other scalar terms, and which demonstrate the two distinct readings for numerals:


\begin{enumerate}
\item Mathematical statements do not allow “at least” readings (\ref{ex:}a). Also, round numbers are more likely to allow “at least” readings than very precise numbers (\ref{ex:9.26}b--c).

\ea \label{ex:9.26}
\ea * $2+2=3$ (should be true under “at least 3” reading)\\
\ex I have \$200 in my bank account, if not more.\\
\ex I have \$201.37 in my bank account, \#if not more.
                       \z
\z
 
\item numerical scales are potentially reversible depending on the context (\ref{ex:9.27}--\ref{ex:9.28}); this kind of reversal is not possible with other scalar terms \REF{ex:9.29}.
\ea \label{ex:9.27}
\ea That bowler is capable of breaking 100 (he might even score 150).\\
\ex That golfer is capable of breaking 100 (he might even score 90).
                       \z
\z

\ea \label{ex:9.28}
\ea You can survive on 2000 calories per day (or more).\\
\ex You can lose weight on 2000 calories per day (or less).
                       \z
\z

\ea \label{ex:9.29}
\ea He ate some of your mangoes, if not all/*none of them.\\
\ex This class room is always warm, if not hot/*cool.
                       \z
\z

\item the “at least” interpretation is only possible with the distributive reading of numerals, not the collective reading \REF{ex:9.30}; this is not the case with other scalar quantifiers \REF{ex:9.31}.

\ea \label{ex:9.30}
\ea Four salesmen have called me today, if not more.\\
\ex Four students carried this sofa upstairs for me, \#if not more.
                       \z
\z

\ea \label{ex:9.31}
\ea Most of the students have long hair, perhaps all of them.\\
\ex Most of the students surrounded the stadium, perhaps all of them.
                       \z
\z

\item the “at least” interpretation is disfavored when a numeral is the focus of a question \REF{ex:9.32}, but this is not the case with other scalar quantifiers \REF{ex:9.33}:

\ea
Q: Do you have two children?\\
A1: No, three.\\
A2: ?Yes, in fact three.
\z

\ea
Q: Are many of your friends linguists?\\
A1: ??No, all of them.\\
A2: Yes, in fact all of them.
\z


It is important to bear in mind that sentences like \REF{ex:9.34} can have different truth values depending on which reading of the numeral is chosen:


\ea \label{ex:9.34}
If Mrs. Smith has three children, there will be enough seatbelts for the whole family to ride together.
\z


One possible analysis might be to treat the alternation between the ‘at least n’ vs. ‘exactly n’ readings as a kind of systematic polysemy. However, it seems that most pragmaticists prefer to treat numeral words as being underspecified or indeterminate between the two, with the intended reading in a given context being supplied by explicature.\footnote{See for example \citet{Horn1992} and \citet{Carston1998}.}


\section{5. Conclusion}\label{sec:}

The large body of work exploring the implications of Grice’s theory of implicature has forced us to recognize that Grice’s relatively simple view of the boundary between semantics and pragmatics is not tenable. Early work in pragmatics often assumed that pragmatic inferences did not affect the truth-conditional content of an utterance, apart from the limited amount of contextual information needed for disambiguation of ambiguous forms, assignment of referents to pronouns, etc. Under this view, truth-conditional content is almost the same thing as conventional meaning.



In this chapter we have discussed various ways in which pragmatic inferences do contribute to truth-conditional content. We have seen that some (at least) generalized conversational implicatures affect truth-conditions, and we have seen that other types of pragmatic inferences, which we refer to as explicatures, are needed in order to determine the truth value of a sentence. In \chapref{sec:11} we discuss the opposite kind of challenge, namely cases where conventional meaning (semantic content) does not contribute to the truth-conditional meaning of a sentence. But first, in \chapref{sec:10}, we discuss a special type of conversational implicature known as an \textsc{indirect speech act}.



\furtherreading



Birner (2012, ch. 3) presents a good overview of the issues discussed here, including a very helpful comparison of Relevance Theory with the “Neo-Gricean” approaches of Levinson and Horn. \citet{Horn2004} and \citet{Carston2004} provide helpful surveys of recent work on implicature, Horn from a Neo-Gricean perspective and Carston from a Relevance Theory perspective. \citet{Bach2010} discusses the differences between his notion of “impliciture” and the Relevance Theory notion of explicature. \citet{Geurts2011} provides a good introduction to, and a detailed analysis of, scalar and quantity implicatures.


\subsubsection{Discussion exercises:}\label{sec:}
\paragraph{A. Explicature}

Identify the explicatures which would be necessary in order to evaluate the truth value for each of the following examples:\footnote{Examples (\ref{ex:}c-e) are taken from \citet{CarstonHall2012}.}

\begin{enumerate}
\item \textit{He arrived at the bank too early}.
\item \textit{All students must pass phonetics}.
\item \textit{No-one goes there anymore}.
\item \textit{To buy a house in London you need money}.
\item {}[Max: How was the party? Did it go well?]\\
  Amy: \textit{There wasn’t enough drink and everyone left early}.
\end{enumerate}
\paragraph{B. Pragmatics in the lexicon}

\citet{Horn1972} observes that many languages have lexical items which express positive universal quantification (\textit{all, every, everyone, everything, always, both}, etc.) and the corresponding negative concepts (\textit{no, none, nothing, no one, never, neither}, etc.). In each case, the positive term can be paraphrased in terms of the corresponding negative, and vice versa. For example, \textit{Everything is negotiable} can be paraphrased as \textit{Nothing} \textit{is non-negotiable}. However, most languages seem to lack negative counterparts to the existential quantifiers (\textit{some, someone, sometimes}, etc.). In order to paraphrase an existential statement like \textit{Something is negotiable}, we have to use a quantifying phrase, rather than a single word, as in \textit{Not everything is non-negotiable}.

Try to formulate a pragmatic explanation for this lexical asymmetry, i.e., the fact that few if any languages have lexical items that mean \textit{not everything, not everyone, not always, not both,} etc. (\textbf{Hint}: think about the kinds of implicatures that might be triggered by the various classes of quantifying words.)

\chapter{{10}: Indirect Speech Acts}

\section{Introduction}\label{sec:} %1. /

Deborah \citet{Tannen1981} recounts the following experience as a visitor to Greece:


\begin{quote}
While I was staying with a family on the island of Crete, no matter how early I awoke, my hostess managed to have a plate of scrambled eggs waiting on the table for me by the time I was up and dressed; and at dinner every evening, dessert included a pile of purple seeded grapes. Now I don’t happen to like seeded grapes or eggs scrambled, but I had to eat them both because they had been set out—at great inconvenience to my hosts—especially for me. It turned out that I was getting eggs scrambled because I had asked, while watching my hostess in the kitchen, whether she ever prepared eggs by beating them, and I was getting grapes out of season because I had asked at dinner one evening how come I hadn’t seen grapes since I had arrived in Greece. My hosts had taken these careless questions as hints—that is, indirect expressions of my desires. In fact, I had not intended to hint anything, but had merely been trying to be friendly, to make conversation.
\end{quote}


Tannen’s hosts believed that she was trying to communicate more than the literal meaning of her words, that is, that she was trying to implicate something without saying it directly. Moreover, the implicature which they (mistakenly) understood had the effect of doing more than the literal meaning of her words would do. Her utterances, taken literally, were simply questions, i.e., requests for information. Her hosts interpreted these utterances as implicated requests to provide her with scrambled eggs and grapes. In other words, Tannen’s hosts interpreted these utterances as \textsc{indirect speech acts}.



A speech act is an action that speakers perform by speaking: offering thanks, greetings, invitations, making requests, giving orders, etc. A \textsc{direct speech act} is one that is accomplished by the literal meaning of the words that are spoken. An \textsc{indirect speech act} is one that is accomplished by implicature.



\citet{Tannen1981} states that “misunderstandings like these are commonplace between members of what appear to (but may not necessarily) be the same culture. However, such mix ups are especially characteristic of cross-cultural communication.”\footnote{See also \citet{Tannen1975,Tannen1986}.} For this reason, indirect speech acts are a major focus of research in the areas of applied linguistics and second language acquisition. They also constitute a potential challenge for translation.



We begin this chapter in \sectref{sec:2} with a summary of J.L. Austin’s theory of speech acts, another foundational contribution to the field of pragmatics. Austin begins by identifying and analyzing a previously unrecognized class of utterances which he calls \textsc{performatives}. He then generalizes his account of performatives to apply to all speech acts.



In \sectref{sec:3} we summarize Searle’s theory of indirect speech acts. Searle builds on Austin’s theory, with certain modifications, and goes on to propose answers to two fundamental questions: How do hearers recognize indirect speech acts (i.e., how do they know that the intended speech act is not the one expressed by the literal meaning of the words spoken), and having done so, how do they correctly identify the intended speech act? (Both of these issues tend to be difficult for even advanced language learners.) An important part of Searle’s answer to these questions is the recognition that indirect speech acts are a special type of conversational implicature.



In \sectref{sec:4} we touch briefly on some cross-linguistic issues, including the question of whether Searle’s theory provides an adequate account for indirect speech acts in all languages.


\section{2. Performatives}\footnotemark{}\label{sec:}
\footnotetext{Much of the discussion in this section is based on \citet{Austin1961}, which is the transcript of an unscripted radio address he delivered on the BBC in 1956.}

In \chapref{sec:3} we cited the definition of sentence meaning repeated here in \REF{ex:}:


\ea
“To know the meaning of a [declarative] sentence is to know what the world would have to be like for the sentence to be true.”  [Dowty, \citealt{WallPeters1981}:4]
\z


Perhaps you wondered, gentle reader, how we might define the meaning of a non-declarative sentence, such as a question or a command? It must be possible for someone to know the meaning of a question without knowing what the world would have to be like for the question to be true —a question is not the sort of thing which \textsc{can} be true, but clearly this does not mean that questions are meaningless.



The semantic analysis of questions and commands is an interesting and challenging area of research, but one that we will not attempt to address in the present book. Even if we restrict our attention to declarative sentences, however, we find some for which the definition in \REF{ex:} does not seem to be directly applicable. J.L. Austin, in a 1955 series of lectures at Harvard University (published as \citealt{Austin1962}), called attention to a class of declarative sentences which cannot be assigned a truth value, because they do not make any claim about the state of the world. Some examples are presented in (\ref{ex:}--\ref{ex:}).


Austin’s examples:

\ea
  a.  ‘I do’ (sc. take this woman to be my lawful wedded wife) — as uttered in the course of the marriage ceremony.
\z

\ea
  b.  ‘I name this ship the Queen Elizabeth’ — as uttered when smashing the bottle against the stem.
\z

\ea
  c.  ‘I bet you sixpence it will rain tomorrow.’
\z

\ea
further examples:\\
\ea I hereby sentence you to 10 years in prison.\\
\ex I now pronounce you man and wife.\\
\ex I declare this meeting adjourned.\\
\ex By virtue of the authority vested in me by the State of XX, and through the Board of Governors of the University of XX, I do hereby confer upon each of you the degree for which you have qualified, with all the rights, privileges and responsibilities appertaining.
                       \z
\z


Austin pointed out that when someone says \textit{I now pronounce you man and wife} or \textit{I hereby declare this meeting adjourned}, the speaker is not describing something, but doing something. The speaker is not making a claim about the world, but rather changing the world. For this reason, it doesn’t make sense to ask whether these statements are true or false. It does, however, make sense to ask whether the person’s action was successful or appropriate. Was the speaker licensed to perform a marriage ceremony at that time and place, or empowered to pass sentence in a court of law? Were all the necessary procedures followed completely and correctly? etc.



Austin called this special class of declarative sentences \textsc{performatives}. He argued that we need to recognize performatives as a new class of \textsc{speech acts} (things that people can do by speaking), in addition to the commonly recognized speech acts such as statements, questions, and commands. Austin refers to the act which the speaker intends to perform by speaking as the \textsc{illocutionary force} of the utterance.\footnote{Austin distinguished \textsc{illocutionary act,} the act which the speaker intends to perform “in speaking”, from \textsc{locutionary act} (the act of speaking) and \textsc{perlocutionary act} (the actual result achieved “by speaking” the utterance).}



As noted above, it does not make sense to try to describe truth conditions for performatives. Instead, Austin says, we need to identify the conditions under which the performative speech act will be \textsc{felicitous}, i.e. successful, valid, and appropriate. He identifies the following kinds of \textsc{Felicity Conditions}:


\ea






  \textbf{Felicity Conditions} (\citealt{Austin1962}:14–15):
\z

(A.1) There must exist an accepted conventional procedure having a certain conventional effect, that procedure to include the uttering of certain words by certain persons in certain circumstances, and further,

(A.2) the particular persons and circumstances in a given case must be appropriate for the invocations of the particular procedure invoked.

(B.1) The procedure must be executed by all participants both correctly and

(B.2) completely.

(C.1) Where, as often, the procedure is designed for use by persons having certain thoughts or feelings, or for the inauguration of certain consequential conduct on the part of any participant, then a person participating in and so invoking the procedure must in fact have those thoughts or feelings, and the participants must intend so to conduct themselves, and further

(C.2) must actually so conduct themselves subsequently.\footnote{I have replaced Austin’s “gamma” ($\Gamma $) with “C”, for convenience.}

Austin referred to violations of conditions A–B as \textsc{misfires}; if these conditions are not fulfilled, then the intended acts are not successfully performed or are invalid. For example, if a person who is not licensed to perform a marriage ceremony says \textit{I now pronounce you man and wife}, the couple being addressed does not become legally married as a result of this utterance. Violations of C Austin called \textsc{abuses}. If this condition is violated, the speech act is still performed and would be considered valid, but it is done insincerely or inappropriately. For example, if someone says \textit{I promise to return this book by Sunday}, but has no intention of doing so, the utterance still counts as a promise; but it is an insincere promise, a promise which the speaker intends to break.


Performatives can be distinguished from normal declarative sentences by the following special features:


\ea
Properties of explicit performatives:
\z

\begin{enumerate}
\item They always occur in indicative mood and simple present tense, with a non-habitual interpretation. As we will see in \chapref{sec:20}, the simple present form of an event-type verb in English typically requires a habitual interpretation; but this is not the case for the examples in (\ref{ex:}--\ref{ex:}).
\item They frequently contain a \textsc{performative verb}, i.e. a verb which can be used either to describe or to perform the intended speech act (e.g. \textit{sentence}, \textit{declare}, \textit{confer}, \textit{invite}, \textit{request}, \textit{order}, \textit{accuse}, etc.).
\item Performative clauses normally occur in active voice with a first person subject, as in (\ref{ex:}--\ref{ex:}), but passive voice with second or third person subject is possible with certain verbs; see examples in \REF{ex:}.
\item Performatives can optionally be modified by the performative adverb \textit{hereby}; this adverb cannot be used with non-performative statements.
\end{enumerate}
\ea
\ea Passengers are requested not to talk to the driver while the bus is moving.\\
\ex You are hereby sentenced to 10 years in prison.\\
\ex Permission is hereby granted to use this software for non-commercial purposes.\\
\ex Richard Smith is hereby promoted to the rank of Lieutenant Colonel.
                       \z
\z


Austin refers to performative sentences which exhibit the features listed in \REF{ex:} as \textsc{explicit performatives}. He notes that explicit performatives can often be paraphrased using sentences which lack some or all of these features. For example, the performative \textit{I hereby order you to shut the door} is more commonly expressed using a simple imperative, \textit{Shut the door!} Similarly, the performative \textit{I hereby invite you to join me for dinner} would be more politely and naturally expressed using a question, \textit{Would you like to join me for dinner?} Since the same speech act can be performed with either expression, it would seem odd to classify one as a performative but not the other. We will refer to utterances which function as paraphrases of explicit performatives but lack the features listed in \REF{ex:} as \textsc{implicit performatives}.



Conversely, it turns out that most speech acts can be paraphrased using an explicit performative. For example, the question \textit{Is it raining?} can be paraphrased as a performative: \textit{I hereby ask you whether it is raining}. In the same way, simple statements can be paraphrased \textit{I hereby inform you that…}, and commands can be paraphrased \textit{I hereby order/command you to…}. Once again, if the same speech act can be performed with either expression, it seems odd to classify one as a performative but not the other. These observations lead us to the conclusion that virtually all utterances should be analyzed as performatives, whether explicit or not.



But if all utterances are to be analyzed as performatives, then the label \textsc{performative} doesn’t seem to be very useful; what have we gained? In fact we have gained several important insights into the meaning of sentential utterances. First, in addition to their propositional content, all such utterances have an \textsc{illocutionary force}, which is an important aspect of their meaning. In the case of explicit performatives, we can identify the illocutionary force by simply looking at the performative verb; but with implicit performatives, as discussed below, the illocutionary force depends partly on the context of the utterance.



Second, all utterances have Felicity Conditions. Certain speech acts (namely statements) also have truth conditions; but Felicity Conditions are something that needs to be analyzed for all speech acts, including statements. As discussed in the following section, in order to explain how indirect speech acts work, we need to identify the Felicity Conditions for the intended act.



The concept of Felicity Conditions is useful in other contexts as well. For example, it would be very odd for someone to say \textit{The cat is on the mat, but I do not believe that it is}.\footnote{This is an example of Moore’s paradox.} Austin suggests that this statement is not a logical contradiction but rather a violation of the Felicity Conditions for statements. One of the Felicity Conditions would be that a person should not make a statement which he knows or believes to be false (essentially equivalent to Grice’s maxim of Quality). It is just as outrageous to make a statement and then explicitly deny that you believe it, as it is to make a promise and then explicitly deny that you intend to carry it out (\textit{I promise that I shall be there, but I haven’t the least intention of being there}). We might refer to such an utterance as a pragmatic contradiction.



A similar situation would arise if someone were to say \textit{All of John’s children are bald}, when in fact he knew perfectly well that John had no children. Austin says that the problem with this statement is the same as with a man who offers to sell a piece of land that does not belong to him. If a transaction were made under these circumstances, it would not be legally valid; the sale would be null and void. Austin says that the statement \textit{All of John’s children are bald} would similarly be “void for lack of reference” if John has no children. So Austin may have been the first to suggest that presupposition failure is a pragmatic issue (an infelicity), and not purely semantic.


\section{Indirect speech acts}\label{sec:} %3. /

The Nigerian professor Ozidi Bariki describes a conversation in which he said to a friend:


\begin{quote}
“I love your left hand.” (The friend had a cup of tea in his hand). The friend, in reaction to my utterance, transferred the cup to his right hand. That prompted me to say: “I love your right hand”. My friend smiled, recognized my desire for tea and told his sister, “My friend wants tea”… My friend’s utterance addressed to his sister in reaction to mine was a representative, i.e. a simple statement: “my friend wants a tea”. The girl rightly interpreted the context of the representative to mean a directive. In other words, her brother (my friend) was ordering her to prepare some tea.  (\citealt{Bariki2008})
\end{quote}


This brief dialogue contains two examples of indirect speech acts. In both cases, the utterance has the form of a simple statement, but is actually intended to perform a different kind of act: request in the first case and command in the second. The second statement, “My friend wants tea,” was immediately and automatically interpreted correctly by the addressee. (In African culture, when an older brother makes such a statement to his younger sister, there is only one possible interpretation.) The first statement, however, failed to communicate. Only after the second attempt was the addressee able to work out the intended meaning, not automatically at all, but as if he was trying to solve a riddle.



Bariki uses this example to illustrate the role that context plays in enabling the hearer to identify the intended speech act. But it also shows us that context alone is not enough. In the context of the first utterance, there was a natural association between what was said (\textit{your left hand}) and what was intended (a cup of tea); the addressee was holding a cup of tea in his left hand. In spite of this, the addressee was unable to figure out what the speaker meant. The contrast between this failed attempt at communication and the immediately understood statement \textit{My friend wants tea}, suggests that there are certain principles and conventions which need to be followed in order to make the illocutionary force of an utterance clear to the hearer.



We might define an \textsc{indirect speech act} (following \citealt{Searle1975}) as an utterance in which one illocutionary act (the \textsc{primary act}) is intentionally performed by means of the performance of another act (the \textsc{literal act}). In other words, it is an utterance whose form does not reflect the intended illocutionary force. \textit{My friend wants tea} is a simple declarative sentence, the form which is normally used for making statements. In the context above, however, it was correctly interpreted as a command. So the literal act was a statement, but the primary act was a command.



Most if not all languages have grammatical and/or phonological means of distinguishing at least three basic types of sentences: statements, questions, and commands. The default expectation is that declarative sentences will express statements, interrogative sentences will express questions, and imperative sentences will express commands. When these expectations are met, we have a \textsc{direct speech act} because the grammatical form matches the intended illocutionary force. Explicit performatives are also direct speech acts.



An indirect speech act will normally be expressed as a declarative, interrogative, or imperative sentence; so the literal act will normally be a statement, question, or command. One of the best-known types of indirect speech act is the Rhetorical Question, which involves an interrogative sentence but is not intended to be a genuine request for information.



Why is the statement \textit{I love your left hand} not likely to work as an indirect request for tea? \citet{Searle1969,Searle1975} proposes that in order for an indirect speech act to be successful, the literal act should normally be related to the Felicity Conditions of the intended or primary act in certain specific ways. Searle re-stated Austin’s Felicity Conditions under four headings: \textsc{preparatory conditions} (background circumstances and knowledge about the speaker, hearer, and/or situation which must be true in order for the speech act to be felicitous); \textsc{sincerity condition}s (necessary psychological states of speaker and/or hearer); \textsc{propositional content} (the kind of situation or event described by the underlying proposition); \textsc{essential condition} (the essence of the speech act; what the act “counts as”). These four categories are illustrated in \REF{ex:} using the speech acts of promising and requesting:


\ea
\textbf{Felicity Conditions for promises and requests} (adapted from \citealt{Searle1969,Searle1975})\\
(S = speaker; H = hearer; A = action)
\z

\begin{tabularx}{\textwidth}{XXX} & \scshape promise & \scshape request\\
\lsptoprule
\scshape preparatory conditions & (i) S is able to perform A\\
(ii) H wants S to perform A, and S believes that H wants S to perform A\\
(iii) it is not obvious that S will perform A & H is able to perform A\\
\scshape sincerity condition & S intends to perform A & S wants H to perform A\\
\scshape propositional content & predicates a future act by S & predicates a future act by H\\
\scshape essential condition & counts as an undertaking by S to do A & counts as an attempt by S to get H to do A\\
\lspbottomrule
\end{tabularx}

Generally speaking, speakers perform an indirect speech act by stating or asking about one of the Felicity Conditions (apart from the essential condition). The examples in \REF{ex:} show some sentences that could be used as indirect requests for tea. Sentences (\ref{ex:}a--b) ask about the preparatory condition for a request, namely the hearer’s ability to perform the action. Sentences (\ref{ex:}c--d) state the sincerity condition for a request, namely that the speaker wants the hearer to perform the action. Sentences (\ref{ex:}e-f) ask about the propositional content of the request, namely the future act by the hearer.


\ea
\ea Do you have any tea?\\
\ex Could you possibly give me some tea?\\
\ex I would like you to give me some tea.\\
\ex I would really appreciate a cup of tea.\\
\ex Will you give me some tea?\\
\ex Are you going to give me some tea?
                       \z
\z


All of these sentences could be understood as requests for tea, if spoken in the right context, but they are clearly not all equivalent: (\ref{ex:}b) is a more polite way of asking than (\ref{ex:}a); (\ref{ex:}d) is a polite request, whereas (\ref{ex:}c) sounds more demanding; (\ref{ex:}e) is a polite request, whereas (\ref{ex:}f) sounds impatient and even rude.



Not every possible strategy is actually available for a given speech act. For example, asking about the sincerity condition for a request is generally quite unnatural: \#\textit{Do I want you to give me some tea?} This is because speakers do not normally ask other people about their own mental or emotional states. So that specific strategy cannot be used to form an indirect request.



We almost automatically interpret examples like (\ref{ex:}b) and (\ref{ex:}e) as requests. This tendency is so strong that it may be hard to recognize them as indirect speech acts. The crucial point is that their grammatical form is that of a question, not a request. However, some very close paraphrases of these sentences, such as those in \REF{ex:}, would probably not be understood as requests in most contexts.


\ea
\ea Do you currently have the ability to provide me with tea?\\
\ex Do you anticipate giving me a cup of tea in the near future?
                       \z
\z


We can see the difference quite clearly if we try to add the word \textit{please} to each sentence. As we noted in \chapref{sec:1}, \textit{please} is a marker of politeness which is restricted to occurring only in requests; it does not occur naturally in other kinds of speech acts. It is possible, and in most cases fairly natural, to add \textit{please} to any of the sentences in \REF{ex:}, even to those which do not sound very polite on their own. However, this is not possible for the sentences in \REF{ex:}. This difference provides good evidence for saying that the sentences in \REF{ex:} are not naturally interpretable as indirect requests.


\ea






  a. Could you possibly give me some tea, please?\\
\ex Will you give me some tea, please?\\
\ex I would like you to give me some tea, please.\\
\ex Are you going to give me some tea (?please)?\\
\ex Do you currently have the ability to provide me with tea (\#please)?\\
\ex Do you anticipate giving me a cup of tea in the near future (\#please)?
\z


The contrast between the acceptability of (\ref{ex:}b) and (\ref{ex:}e) as requests vs. the unacceptability of their close paraphrases in \REF{ex:} suggests that the form of the sentence, as well as its semantic content, helps to determine whether an indirect speech act will be successful or not. We will return to this issue below, but first we need to think about a more fundamental question: How does the hearer recognize an indirect speech act? In other words, how does he know that the primary (intended) illocutionary force of the utterance is not the same as the literal force suggested by the form of the sentence?



Searle suggests that the key to solving this problem comes from Grice’s Co-operative Principle. If someone asks the person sitting next to him at a dinner \textit{Can you pass me the salt?}, we might expect the addressee to be puzzled. Only under the most unusual circumstances would this question be relevant to the current topic of conversation. Only under the most unusual circumstances would the answer to this question be informative, since few people who can sit up at a dinner table are physically unable to lift a salt shaker. In most contexts, the addressee could only believe the speaker to be obeying the Co-operative Principle if the question is not meant as a simple request for information, i.e., if the intended illocutionary force is something other than a question.



Having recognized this question as an indirect speech act, how does the addressee figure out what the intended illocutionary force is? Searle’s solution is essentially the Gricean method of calculating implicatures, enriched by an understanding of the Felicity Conditions for the intended speech act. \citet{Searle1975} suggests that the addressee might reason as follows: “This question is not relevant to the current topic of conversation, and the speaker cannot be in doubt about my ability to pass the salt. I believe him to be cooperating in the conversation, so there must be another point to the question. I know that a preparatory condition for making a request is the belief that the addressee is able to perform the requested action. I know that people often use salt at dinner, sharing a common salt shaker which they pass back and forth as requested. Since he has mentioned a preparatory condition for requesting me to perform this action, I conclude that this request is what he means to communicate.”



So it is important that we understand indirect speech acts as a kind of conversational implicature. However, they are different in certain respects from the implicatures that Grice discussed. For example, Grice stated that implicatures are “non-detachable”, meaning that semantically equivalent sentences should trigger the same implicatures in the same context. However, as we noted above, this is not always true with indirect speech acts. In the current example, Searle points out that the question \textit{Are you able to pass me the salt?}, although a close paraphrase of \textit{Can you pass me the salt?}, is much less likely to be interpreted as a request (\#\textit{Are you able to please pass me the salt?}). How can we account for this?



Searle argues that, while the meaning of the indirect speech act is calculable or explainable in Gricean terms, the forms of indirect speech acts are partly conventionalized. Searle refers to these as “conventions of usage”, in contrast to normal idioms like \textit{kick the bucket} (for ‘die’) which we might call conventions of meaning or sense.



Conventionalized speech acts are different from normal idioms in several important ways. First, the meanings of normal idioms are not calculable or predictable from their literal meanings. The phrase \textit{kick the bucket} contains no words which have any component of meaning relating to death.



Second, when an indirect speech act is performed, both the literal and primary acts are understood to be part of what is meant. In Searle’s terms, the primary act is performed “by way of” performing the literal act. We can see this because, as illustrated in \REF{ex:}, the hearer could appropriately reply to the primary act alone (A1), the literal act alone (A2), or to both acts together (A3). Moreover, in reporting indirect speech acts, it is possible (and in fact quite common) to use matrix verbs which refer to the literal act rather than the primary act, as illustrated in (\ref{ex:}--\ref{ex:}).


\ea
Q: Can you (please) tell me the time?\\
A1: It’s almost 5:30.\\
A2: No, I’m sorry, I can’t; my watch has stopped.\\
A3: Yes, it’s 5:30.
\z

\ea
\ea Will you (please) pass me the salt?\\
\ex He asked me whether I would pass him the salt.
                       \z
\z

\ea
\ea I want you to leave now (please).\\
\ex He told me that he wanted me to leave.
                       \z
\z


In this way indirect speech acts are quite similar to other conversational implicatures, in that both the sentence meaning and the pragmatic inference are part of what is communicated. They are very different from normal idioms, which allow either the idiomatic meaning (the normal interpretation), or the literal meaning (under unusual circumstances), but never both together. The two senses of a normal idiom are antagonistic, as we can see by the fact that some people use them to form (admittedly bad) puns:


\ea






  Old milkmaids never die — they just kick the bucket.\footnote{Richard \citet{Lederer1988} \textit{Get Thee to a Punnery}. Wyrick \& Company.}
\z


\citet[196]{Birner2012} points out that under Searle’s view, indirect speech acts are similar to generalized conversational implicatures. In both cases the implicature is part of the default interpretation of the utterance; it will arise unless it is blocked by specific features in the context, or is explicitly negated, etc. We have to work pretty hard to create a context in which the question \textit{Can you pass the salt?} would not be interpreted as a request, but it can be done.\footnote{\citet[69]{Searle1975} suggests that a doctor might ask such a question to check on the progress of a patient with an injured arm.}



Searle states that politeness is one of the primary reasons for using an indirect speech act. Notice that all of the sentences in \REF{ex:}, except perhaps (\ref{ex:}f), sound more polite than the simple imperative: \textit{Give me some tea!} He suggests that this motivation may help to explain why certain forms tend to be conventionalized for particular purposes.



\section{Indirect speech acts across languages}\label{sec:} %4. /

Searle states that his analysis of indirect speech acts as conventions of usage helps to explain why the intended illocutionary force is sometimes preserved in translation, and sometimes not. (This again is very different from the idiomatic meanings of normal idioms, which generally do not survive in translation.) He points out that literal translations of a question like \textit{Can you help me?} would be understood as requests in French and German, but not in Czech. The reason that the intended force is sometimes preserved in translation is that indirect speech acts are calculable. They are motivated by Gricean principles which are widely believed to apply to all languages, subject to a certain amount of cultural variation. The reason that the intended force is not always preserved in translation is that indirect speech acts are partly conventionalized, and different languages may choose to conventionalize different specific forms.



It is often difficult for non-native speakers to recognize and correctly interpret indirect speech acts in a second language. \citet[175]{Wierzbicka1985}, for example, states: “Poles learning English must be taught the potential ambiguity of \textit{would you–} sentences, or \textit{why don’t you–} sentences, just as they must be taught the polysemy of the word \textit{bank}.” This has been a major area of research in second language acquisition studies, and most scholars agree that this is a significant challenge even for advanced learners of another language.



There is less agreement concerning whether the same basic principles govern the formation of indirect speech acts in all languages. Numerous studies have pointed out cross-linguistic differences in the use of specific linguistic features, preferred or conventionalized patterns for specific speech acts, cultural variation in ways of showing politeness, contexts where direct vs. indirect speech acts are preferred, etc.



\citet{Wierzbicka1985} argues that Searle’s analysis of indirect speech acts is not universally applicable, but reflects an Anglo-centric bias. She points out for example that English seems to be unusual in its strong tendency to avoid the use of the imperative verb form. The strategy of expressing indirect commands via questions is so strongly preferred that it is no longer a marker of politeness; it is frequently used (at least in Australian English) in impolite speech laced with profanity, obscenity, or other expressives indicating anger, contempt, etc. \citet{Kalisz1992} agrees with many of Wierzbicka’s specific observations concerning differences between English and Polish, but argues that Searle’s basic claims about the nature of indirect speech acts are not disproven by these differences.



It is certainly true that there is a wide range of variation across languages in terms of what counts as an apology, promise, etc., and in the specific features that distinguish appropriate from inappropriate ways for performing a particular speech act. For example, \citet{OlshtainCohen1989} recount the following incidents to illustrate differences in acceptable apologies between English and Israeli Hebrew:


\begin{quote}
One morning, Mrs G., a native speaker of English now living in Israel, was doing her daily shopping at the local supermarket. As she was pushing her shopping cart she unintentionally bumped into Mr Y., a native Israeli. Her natural reaction was to say “I’m sorry” (in Hebrew). Mr Y. turned to her and said, “Lady, you could at least apologize”. On another occasion the very same Mr Y. arrived late for a meeting conducted by Mr W. (a native speaker of English) in English. As he walked into the room he said “The bus was late”, and sat down. Mr W. obviously annoyed, muttered to himself “These Israelis, why don’t they ever apologize!” [\citealt{OlshtainCohen1989}: 53]
\end{quote}


In a similar vein, \citet{Egner2002} shows that in many African cultures, a promise only counts as a binding commitment when it is repeated. Clearly there are many significant differences across languages in the conventional features of speech acts; but this does not necessarily mean that the underlying system which makes it possible to recognize and interpret indirect speech acts is fundamentally different.



Searle’s key insights are that indirect speech acts are a type of conversational implicature, and that the felicity conditions for the intended act play a crucial role in the interpretation of these implicatures. Given our current state of knowledge, it seems likely that these basic principles do in fact hold across languages. But like most cross-linguistic generalizations in semantics and pragmatics, this hypothesis needs to be tested across a wider range of languages.


\section{5. Conclusion}\label{sec:}

A speech act is an action that speakers perform by speaking. Languages typically have grammatical ways of distinguishing sentence types (moods) corresponding to at least three basic speech acts: statements, commands, and questions. When the speaker’s intended speech act (or \textsc{illocutionary force}) corresponds to the sentence type that is chosen, a direct speech act is performed. In addition, the declarative sentence type is generally used for a special class of direct speech acts which we call \textsc{explicit performatives}. When the speaker’s intended speech act does not correspond to the sentence type that is chosen, an indirect speech act is performed. Indirect speech acts are conversational implicatures, and their interpretation can be explained in Gricean terms; but in addition, they are often partly conventionalized.



All speech acts are subject to felicity conditions, that is, conditions that must be fulfilled in order for the speech act to be \textsc{felicitous} (i.e., valid and appropriate). Successful indirect speech acts typically involve literal sentence meanings which state or query the felicity conditions for the primary (i.e., intended) speech act.



\furtherreading



Birner (2012, ch. 6) presents a useful overview of the issues addressed in this chapter. \citet{Austin1961}, based on a radio address he delivered on the BBC, provides a readable, non-technical introduction to his theory of performatives. \citet{Searle1975} provides a concise summary of his theory of indirect speech acts. \citet{BrownLevinson1978} is the foundational study of sociolinguistic and pragmatic aspects of politeness across languages. The volumes edited by Blum-Kulka, House, \& \citet{Kasper1989} and \citet{GassNeu2006} contain studies on indirect speech acts in cross-cultural and second language communication.


\subsubsection{Discussion exercises:}\label{sec:}
\paragraph{A. Identifying indirect speech acts}
Identify both the literal and primary act in each of the following indirect speech acts (square brackets are used to provide [context]):

\begin{enumerate}
\item{} [S1: My motorcycle is out of the shop; let’s go for another ride.]\\
S2: \textit{Do you think I’m crazy?}
\item{} [senior citizen dialing the police:]\\
\textit{I’m alone in the house and someone is trying to break down my door.}
\item{} [S1: I’m really sorry for bumping into your car.]\\
S2: \textit{Don’t give it another thought.}
\end{enumerate}
\paragraph{B. Indirect speech act strategies}

Assume that the felicity conditions for offers are essentially the same as for promises. Try to make up one example of a sentence that would work as an indirect offer for each of the following strategies:

\begin{enumerate}
\item by querying the preparatory conditions of the direct offer;
\item by stating the preparatory conditions of the direct offer;
\item by stating the propositional content of the direct offer;
\item by stating the sincerity condition of the direct offer.
\end{enumerate}
\subsubsection{Homework exercises:}\footnotemark{}\label{sec:}
\footnotetext{Modeled after Saeed (2009: 251–53).}
\paragraph{A. Performatives}

State whether the following utterances would be naturally interpreted as explicit performatives, and explain the evidence which supports your conclusion.

\begin{enumerate}
\item \rmfamily
I acknowledge you as my legal heir.
\end{enumerate}

\textsc{model answer}: \textit{I hereby} \textit{acknowledge you as my legal heir} is quite natural. The verb is simple present tense, referring to a single event with no habitual meaning. It is active indicative with first person singular subject. Therefore this utterance is an explicit performative.


\begin{enumerate}
\item \rmfamily
Smith acknowledges you as his legal heir.
\item \rmfamily
I request the court to reconsider my petition.
\item \rmfamily
I’m promising Mabel to take her to a movie next week.
\item \rmfamily
I promised Mabel to take her to a movie next week.
\item \rmfamily
I expect that you will arrive on time from now on.
\item \rmfamily
You are advised that anything you say may be used as evidence against you.
\end{enumerate}
\paragraph{B. Indirect speech acts (1)}

For each of the following indirect speech acts, identify both the literal and primary act.

\begin{enumerate}
\item \rmfamily
{}[young woman to man who has just prosed to her]
\end{enumerate}

\textrm{I hope that we can always remain friends.\\
}model answer: literal act = statement; primary act = refusal.


\begin{enumerate}
\item \rmfamily
{}[housewife to next-door neighbor]
\end{enumerate}
\rmfamily
Can you spare a cup of sugar?


\begin{enumerate}
\item 
\textrm{[flight attendant to passenger who is standing in the aisle]}
\end{enumerate}
\rmfamily
The captain has turned on the “fasten seatbelt” sign.


\begin{enumerate}
\item 
\textrm{[host to friend who has just arrived for a visit]}
\end{enumerate}
\rmfamily
How would you like a cup of coffee?


\begin{enumerate}
\item 
\textrm{[office manager to colleague who has invited him to go out for lunch]}
\end{enumerate}
\rmfamily
Look at that pile of papers in my inbox!


\begin{enumerate}
\item 
\textrm{[addressing neighbor who has broken his arm]}
\end{enumerate}
\rmfamily
I will mow your lawn for you this month.


\paragraph{C. Indirect speech acts (2)}

Based on felicity conditions for requests, and using your own examples, try to form one indirect request for each of the following strategies.

\ea
\ea by querying the preparatory condition of the direct request\\
\textsf{model answer: preparatory condition = Hearer is able to perform action.\\
Possible ISAs using this strategy:\\
Can you give me a ride to church tomorrow?\\
Would you be able to give me a ride to church tomorrow?}

\ex by stating the preparatory condition of the direct request;

\ex by querying the propositional content of the direct request.

\ex by stating the sincerity condition of the direct request
\z
\z

\chapter{{11}: Conventional implicature and “use-conditional” meaning}

\section{Introduction}\label{sec:} %1. /

In \chapref{sec:8} we mentioned the somewhat mysterious concept of \textsc{conventional implicature}. This term was coined by Grice, but he commented only briefly on what he meant by it. The most widely cited example of an expression that carries a conventional implicature is the word \textit{but}. Grice used the example in (\ref{ex:}a), based on a cliché of the Victorian era:


\ea
\ea She is poor but she is honest.\\
\ex She is poor and she is honest.  [\citealt{Grice1961}: 127]
                       \z
\z


Grice argued that a speaker who says (\ref{ex:}a) only \textsc{asserts} (\ref{ex:}b). The word \textit{but} provides an additional element of meaning, indicating that the speaker believes there to be a contrast between poverty and honesty. This extra element of meaning (implied contrast or counter-expectation) is the conventional implicature. It is said to be conventional because it is an inherent part of the meaning of \textit{but}, and is not derived from the context of use. Grice called it an “implicature” because he, like Frege before him, felt that if this additional element of meaning is false but (\ref{ex:}b) is true, we would not say that the person who says (\ref{ex:}a) is making a false statement. In other words, the conventional implicature does not contribute to the truth conditions of the statement.\footnote{Recall similar comments by Frege regarding \textit{but}, which were quoted in \chapref{sec:8}.}



Nevertheless, someone might object to (\ref{ex:}a) as in \REF{ex:}, claiming that the word \textit{but} has been misused. The core of this objection would not be the truth of the statement in (\ref{ex:}a) but the appropriateness of the conjunction which was chosen.


\ea
What do you mean “but”? There is no conflict between poverty and honesty!
\z


Recent work by Christopher Potts and others has tried to clarify the nature of conventional implicature, and has greatly extended the range of expressions which are included under this label. In this chapter we will look at some of these expression types.



A core property of conventional implicatures is that they do not change the conditions under which the sentence will be true, but rather the conditions under which the sentence can be appropriately used. For this reason, some authors have made a distinction between \textsc{truth-conditional meaning} vs. \textsc{use-conditional meaning}.\footnote{\citet{Gutzmann2015}, \citet{Recanati2004}.} The truth-conditional meaning that is asserted in (\ref{ex:}a) would be equivalent to the meaning of (\ref{ex:}b), while the implied contrast between \textit{poor} vs. \textit{honest} comes from the use-conditional meaning of \textit{but}. The term “use-conditional meaning” seems to cover essentially the same range of phenomena as “conventional implicature”, and we will treat these terms as synonyms.\footnote{In this we follow the usage of \citet{Gutzmann2015}.}



We begin in \sectref{sec:2} with a discussion of the definition and diagnostic properties of conventional implicatures, as described by Potts. We illustrate this discussion using certain types of adverbs in English which seem to contribute use-conditional meaning rather than truth-conditional meaning. In the rest of the chapter we look at some use-conditional expressions in other languages: honorifics in Japanese (sec. 3), politeness markers in Korean (sec. 4), honorific pronouns and other polite register lexical choices (sec. 5), and discourse particles in German (sec. 6).


\section{Distinguishing truth-conditional vs. use-conditional meaning}\label{sec:} %2. /
\subsection{Diagnostic properties of conventional implicatures}\label{sec:} %2.1 /

A passage from Grice’s comments on conventional implicatures was quoted in \chapref{sec:8}, which included the following discussion of the meaning of \textit{therefore}:


\begin{quote}
If I say (smugly), \textit{He is an Englishman; he is, therefore, brave}, I have certainly committed myself, by virtue of the meaning of my words, to its being the case that his being brave is a consequence of (follows from) his being an Englishman…  I do not want to say that my utterance of this sentence would be, strictly speaking, false should the consequence in question fail to hold.  (\citealt{Grice1975}:44)
\end{quote}


Based on Grice’s comments, Potts formulates a definition of conventional implicatures that includes the following points: (i) conventional implicatures are (normally) beliefs of the speaker (“I have certainly committed \textsc{myself}”), and so in a sense “speaker-oriented”; (ii) they are part of the intrinsic, conventional meaning of a given expression or construction (“by virtue of the meaning of my words”), and so are not cancellable; (iii) they do not contribute to the truth-conditional content which is the main point of the assertion.\footnote{\citet{Potts2005,Potts2012}; see also \citet[39]{Horn1997}.}



Potts uses the term “at issue content” to refer to the main point of an utterance: the core information which is asserted in a statement or queried in a question.\footnote{Another way of thinking about this is to say that the “at issue” content represents the change which the speaker intends to make in the common ground.} So in Grice’s example, the at issue content of the assertion is \textit{He is English and brave}. The conventional implicature contributed by \textit{therefore} is that a causal relationship exists between two situations (in this case, between being an Englishman and being brave).



The definition outlined above leads us to expect that conventional implicatures will have certain properties that allow us to distinguish them from other kinds of meaning. Potts suggests that conventional implicatures are:\footnote{\citet{Potts2015}; a similar list is presented for expressives in \citet{Potts2007c}.}



\textsc{conventional}, i.e., semantic in nature rather than pragmatic (as we defined those terms in \chapref{sec:9}). They must be learned as part of the meaning of a given word or construction, and cannot be calculated from context.



\textsc{secondary}: not part of the “at-issue” content, but rather used to provide supporting content, contextual information, editorial comments, evaluation, etc.



\textsc{independent}: separate from and logically independent of the at-issue content.



\textsc{“scopeless”:} since conventional implicatures are not part of the “at-issue” content, they are typically not altered by negation, interrogative mood, etc. Often they take scope over the whole sentence even when embedded in subordinate clauses.



\textsc{not presupposed:}\footnote{Potts uses the term “Backgrounded” for this concept.} not assumed to be shared by the addressee, in contrast to presuppositions. So, for example, while the addressee might challenge a conventional implicature, as illustrated in \REF{ex:} above, the “Hey, wait a minute” response seems less natural (\ref{ex:}d).



Many of these properties are similar to the properties of expressive meaning that we listed in \chapref{sec:2}. This is no accident, since expressives provide a clear example of use-conditional meaning. The expressive term \textit{jerk} in example (\ref{ex:}a) reflects a negative attitude toward Peterson, and this negative attitude is a belief of the speaker. The negative attitude is not calculated from the context, but comes directly from the conventional meaning of the word \textit{jerk}. It is not part of the “at-issue” content of the sentence, so a hearer who does not share this negative attitude would not judge (\ref{ex:}a) to be a false statement. The negative attitude is still expressed if the sentence is negated or questioned (\ref{ex:}b-c).


\ea
\ea That jerk Peterson is the only economist on this committee.\\
\ex That jerk Peterson isn’t the only economist on this committee.\\
\ex Is that jerk Peterson the only economist on this committee?\\
\ex \#Hey, wait a minute! I didn’t know that Peterson was a jerk!
                       \z
\z


Potts lists a wide variety of other expression types that illustrate these properties, including non-restrictive relative clauses and other kinds of parenthetical comments. In the remainder of this section we will focus on certain types of adverbs which seem to express use-conditional meanings.


\subsection{2.2  Speaker-oriented adverbs}\label{sec:}

In this section we will discuss two classes of English adverbs. The \textsc{evaluative adverbs (}e.g. \textit{(un)fortunately, oddly, sadly, surprisingly, inexplicably}) provide information about the speaker’s attitude toward the proposition being expressed. The \textsc{speech act adverbials (}e.g. \textit{frankly}, \textit{honestly}, \textit{seriously}, \textit{confidentially}) provide information about the manner in which the speaker is making the current statement. We will use the term \textsc{speaker-oriented adverbs} as a generic term that includes both of these classes.\footnote{The label \textsc{evaluative adverbs} comes from \citet{Ernst2009}. Ernst uses the term \textsc{speaker-oriented adverbs} as to include not only evaluative adverbs and speech act adverbials, but also modal adverbs like \textit{probably}. \citet{Potts2005} uses the term \textsc{speaker-oriented adverbs} to refer to the class that I call \textsc{evaluative adverbs}.}



There are several reasons for thinking that speaker-oriented adverbs occurring in statements do not contribute to the truth-conditional content that is being asserted. The adverbs in \REF{ex:}, for example, seem to contradict the asserted proposition: one cannot tell a lie \textit{frankly}; the faculty are unlikely to make their demand \textit{confidentially}; and the mayor, it seems, was not curious enough. Yet these sentences are not contradictions, precisely because these adverbs are not understood as contributing to the “at issue” propositional content of the sentence. Rather, they provide information about the manner in which the speech act is being performed (\ref{ex:}a--b) or the speaker’s attitude toward the proposition expressed (\ref{ex:}c).


\ea
\ea \textit{Frankly}, your cousin is a habitual liar.\\
\ex \textit{Confidentially}, the faculty are planning to demand that the provost resign.\\
\ex \textit{Curiously} the mayor never asked where all the money came from.
                       \z
\z


Because they do not contribute to the proposition that is being asserted, it would be inappropriate to challenge the truth of a statement based on the content expressed by these adverbs (\ref{ex:}--\ref{ex:}). The hearer may express disagreement with the adverbial content by saying something like: \textit{I agree that p, but I do not consider that curious/fortunate/etc}. But this would not be grounds for calling the original statement false.


\ea
A: \textit{Curiously}/\textit{fortunately} the mayor never asked where all the money came from.\\
B: That’s not true; he asked me just last week.\\
B[2B9?]: \#That’s not true; he never asked, but there is nothing curious/fortunate about that.
\z

\ea
A: \textit{Frankly/confidentially}, Jones is not the best-qualified candidate for this job.\\
B: That’s not true; he is the only candidate who holds a relevant degree.\\
B[2B9?]: \#That’s not true; he is not qualified, but you are not speaking frankly/confidentially.
\z


Further evidence for the claim that these speaker-oriented adverbs are not part of the propositional content being asserted comes from their behavior under negation and questioning. When a sentence containing an evaluative or speech act adverbial is negated or questioned, the adverb itself cannot be interpreted as part of what is being negated or questioned. For example, (\ref{ex:}a) cannot mean ‘It is not fortunate that the best team won’ but only ‘It is fortunate that the best team did not win.’ Example (\ref{ex:}b) cannot mean ‘Was it unfortunate that he lost the vision in that eye?’ but only ‘Did he lose the vision in that eye? If so, it was unfortunate.’ Speech act adverbials in questions like (\ref{ex:}c) are not part of what is being questioned, but generally describe the manner in which the speaker wants the addressee to answer the question. As such examples show, evaluative and speech act adverbials are not interpreted as being under the scope of sentence negation or interrogative mood.


\ea
\ea … the best team \textit{fortunately} didn’t win on this occasion.\footnote{\url{http://sportwitness.ning.com/forum/topics/nextgen}} \\
\ex Was it ok or did he \textit{unfortunately} lose the vision in that eye?\footnote{\url{https://www.inspire.com/groups/preemie/discussion/rop-after-2-ops-scarring-is-pulling-the-retina-away/}} \\
\ex Is he, \textit{frankly}, combative enough? (referring to a potential presidential candidate)\footnote{\href{http://www.wbur.org/2011/12/21/romney-nh-6}{{www.wbur.org/2011/12/21/romney-nh-6}}} 
                       \z
\z


These claims about speaker-oriented adverbs apply only to their use as sentence adverbs, where the speaker uses them to describe his own manner of speaking or attitude toward the current speech act. Sentence adverbs occur most freely in sentence initial position, as in (\ref{ex:}a) and (\ref{ex:}a); but other positions are also possible (normally with the adverb set off from the rest of the sentence by pauses) as illustrated in (\ref{ex:}b-d) and (\ref{ex:}b-d).


\ea
\ea \textit{Curiously}, the mayor never asked where all the money came from.\\
\ex The mayor, c\textit{uriously}, never asked where all the money came from.\\
\ex The mayor never asked, c\textit{uriously}, where all the money came from.\\
\ex The mayor never asked where all the money came from, c\textit{uriously}.
                       \z
\z

\ea
\ea \textit{Frankly/confidentially}, Jones is not the best-qualified candidate for this job.\\
\ex Jones, \textit{confidentially}, is not the best-qualified candidate for this job.\\
\ex Jones is not, \textit{frankly}, the best-qualified candidate for this job.\\
\ex Jones is not the best-qualified candidate for this job, \textit{frankly}.
                       \z
\z


A number of speech act adverbials also have a second use as manner adverbs, typically occurring within the VP as in (A). In this use they describe the manner of the agent of a reported speech act. When these forms are used as manner adverbs, they do contribute to the “at issue” content of the sentence. We can see that this is so because the truth of an assertion can be challenged if such an adverb is misused, as in (B).


\ea
A: Jones told the committee \textit{frankly/confidentially} about his criminal record.\\
B: That’s not true; he told them, but he did not speak frankly/confidentially.
\z


Moreover, these manner adverbs are part of the propositional content which can be negated (\ref{ex:}b) and questioned (\ref{ex:}b). This contrasts with the behavior of the same forms used as sentence adverbs, which are not interpreted as being included under negation (\ref{ex:}a) or questioning (\ref{ex:}a).


\ea
\ea Jones did not, \textit{confidentially}, inform the committee about his criminal record.\\
\ex Jones did not inform the committee \textit{confidentially} about his criminal record;\\
  he told them in a public hearing.
                       \z
\z

\ea
\ea \textit{Confidentially}, did Jones tell the committee about this?\\
\ex Did Jones tell you this \textit{confidentially}, or can we inform the other members\\
  of the committee?
                       \z
\z


A number of the evaluative adverbs are morphologically related to an adjective that takes a propositional argument. In simple sentences, the adverb and adjective can be used to paraphrase each other, as seen in (\ref{ex:}--\ref{ex:}).


\ea
\ea \textit{Fortunately}, Jones doesn’t realize how valuable this parchment is.\\
\ex It is \textit{fortunate} that Jones doesn’t realize how valuable this parchment is.
                       \z
\z

\ea






  a. \textit{Curiously} the mayor never asked where all the money came from.\\
\ex It is \textit{curious} that the mayor never asked where all the money came from.
\z

\ea
\ea \textit{Oddly}, Jones never got that parchment appraised before he put it up for auction.\\
\ex It is \textit{odd} that Jones never got that parchment appraised before he put it up for auction.
                       \z
\z


However, evaluative adjectives, in contrast to the corresponding evaluative adverbs, do contribute to the “at issue” content of the utterance. They can provide grounds for challenging the truth of a statement, as in \REF{ex:}, and they are part of the propositional content which can be negated \REF{ex:} or questioned \REF{ex:}.


\ea
A: It is \textit{curious/fortunate} that the mayor never asked where all the money came from.\\
B: That’s not true; the fact that he never asked is \{not curious at all/most unfortunate\}.
\z

\ea
It is not \textit{odd} that Jones asked for an appraisal before he bought that parchment;\\
  it seems natural under the circumstances.
\z

\ea
A: Was it \textit{odd} that Jones did not ask for an appraisal?\\
B. No, I think it was fairly natural under the circumstances.
\z


To summarize, we have argued that evaluative adverbs and speech act adverbials in English contribute use-conditional rather than truth-conditional meaning to the utterances in which they occur. We argued this on the grounds that they are independent of and secondary to the “at issue” propositional content of the utterance, they cannot be negated or questioned, and they do not affect the truth value of a statement. But clearly the meaning which these adverbs contribute is conventional: it has to be learned, rather than being calculated from the context of use. Moreover, they are not presupposed, that is, they are not treated as if they were already part of the common ground.


\section{Japanese honorifics}\label{sec:} %3. /

Honorifics are grammatical markers which speakers use to show respect or deference to someone whom they consider to be higher in social status than themselves. Japanese has two major types of honorifics. One type is used to show respect toward someone referred to in the sentence, with different forms used for subjects vs. non-subjects. We will refer to this type as “argument honorifics”.\footnote{This term is based on the term used by \citet{Potts2005}, “argument-oriented honorifics”. \citet{Harada1976}, one of the first detailed discussions of this topic in English, refers to these as “propositional honorifics”. Harada was the original source of the term “performative honorifics” for those which show respect to the addressee, a terminology which is now widely adopted.} The other type is used to show respect to the addressee, and so are considered to be a mark of polite speech. This type is often referred to as “performative honorifics”, because they indicate something about the context of the current speech event, specifically the relationship between speaker and addressee. We will instead refer to this second type as “addressee honorifics”.



The use of an argument honorific to indicate the speaker’s respect for a person referred to in the sentence is illustrated in (\ref{ex:}a), which shows respect for the referent of the subject NP (Prof. Sasaki). The use of an addressee honorific to indicate the speaker’s respect for the addressee is illustrated in (\ref{ex:}b).


\ea
\ea  \gll Sasaki  sensei=wa  watasi=ni  koo  \textbf{o}-hanasi.\textbf{ni.nat}-ta.\\
Sasaki  teacher=\textsc{top}  1sg=\textsc{dat}  this.way  speak.\textsc{hon-past}\\
\glt ‘Prof. Sasaki told me this way.’  [\citealt{Harada1976}: 501]
\ex \gll
    Watasi=wa  sono  hito=ni  koo  hanasi-\textbf{masi}-ta.\\
\textsc{1sg}=\textsc{top}  that  man=\textsc{dat}  this.way  speak-\textsc{hon-past}\\
\glt ‘I told him (=that man) this way.’  (polite speech)   [\citealt{Harada1976}: 502]
\z \z


Argument honorifics are only allowed in sentences which refer to someone socially superior to the speaker; sentence (\ref{ex:}a) is unacceptable, because no such person is referred to. But addressee honorifics are not subject to this constraint (\ref{ex:}b).


\ea \ea \gll *Ame=ga  \textbf{o}-huri.\textbf{ni.nat}-ta.\\
  rain=\textsc{nom}  fall.\textsc{hon-past}\\
\glt (intended: ‘It rained.’)  [\citealt{Harada1976}: 502]
\ex \gll
Ame=ga  huri-\textbf{masi}-ta.\\
rain=\textsc{nom}  fall-\textsc{hon-past}\\
\glt ‘It rained.’  (polite speech)   [\citealt{Harada1976}: 502]
\z \z


In the remainder of this section we will focus primarily on addressee honorifics. \citet{Potts2005} analyzes addressee honorifics as conventional implicature triggers, specifically as a kind of expressive. This means that addressee honorifics do not contribute to the truth-conditional “at-issue” content of the sentence. The truth conditions of (\ref{ex:}b) would not be changed if the honorific marker were deleted. Misuse of the honorific (e.g. for referring to someone socially inferior), or dropping the honorific when it is expected, would not make the statement false, only rude and/or inappropriate.\footnote{Thanks to Eric Shin Doi for very helpful discussion of these issues, and for providing the examples in \REF{ex:}.}



As we would predict under Pott’s proposal, the honorific meaning cannot be part of the propositional content that is negated or questioned. (\ref{ex:}a--b) are felt to be just as polite as (\ref{ex:}b); the element of respect is neither negated in (\ref{ex:}a) nor questioned in (\ref{ex:}b).


\ea
\ea \gll Ame=ga  huri-\textbf{mas-en}  desi-ta.\\
rain=\textsc{nom}  fall-\textsc{hon-neg  cop-past}\\
\glt ‘It didn’t rain.’  (polite speech)

\ex
 \gll  Ame=wa  huri-\textbf{masi}-ta-ka?\\
rain=\textsc{top}  fall-\textsc{hon-past-ques}\\
\glt ‘Did it rain?’  (polite speech)
\z \z


We have seen that addressee honorifics express beliefs or attitudes of the speaker. They are independent of and secondary to the “at issue” propositional content of the utterance. They cannot be negated or questioned, and do not affect the truth value of a statement. Thus they clearly fit Potts’ definition of conventional implicatures.


\section{Korean “speech style” markers}\label{sec:}  %4. /

Korean also has the same two types of honorifics as Japanese, argument honorifics vs. addressee honorifics.\footnote{\citet{KimSells2007}} As part of the addressee honorific system, Korean distinguishes grammatically six levels of politeness, often referred to as “speech styles”: formal, semiformal, polite, familiar, intimate, and plain.\footnote{\citet{Martin1992}, \citet{Pak2008}, \citet{Sohn1999}} A seventh level, “super-polite”, was used for addressing kings and queens; it is now considered archaic, and is used mostly in prayers. The choice of speech style marking depends on “(i) the \textit{relationship} between speaker and addressee (e.g., intimacy, politeness), and (ii) the \textit{formality} of the situation”.\footnote{Pak, \citet{PortnerZanuttini2013}} The uses of these styles, as described by \citet[120]{Pak2008}, is summarized in \REF{ex:}:

\begin{tabularx}{\textwidth}{XX}
\lsptoprule

 \bfseries Speech styles & \bfseries Contexts of use\\
 Formal & used for speaking to someone to whom deference is due (e.g., ones superior or employer, a professor, a high official, etc.); or on formal occasions such as oral news reports and public lectures\\
 Semiformal & could be used by a husband speaking to his wife, or by a younger superior speaking to an older subordinate; gradually disappearing from daily usage\\
 Polite & used by adults for speaking to adults who are not close friends or family members; to address a socially equal or superior person; or by children speaking to adults in a polite way\\
 Familiar & mostly used by male adults, for speaking to male adult friends, an adolescent, or a son-in-law\\
 Intimate (also\\
called “half-talk”) & used for talking to family members or close friends\\
 Plain & used by adults for speaking to children or younger siblings, and by children among themselves; also used in written texts and newspapers\\
\lspbottomrule
\end{tabularx}

Speech style is marked grammatically by a verbal suffix referred to as the “sentence ender”. Since Korean is an SOV language, the main clause verb typically occurs at the end of the sentence and hosts the sentence ender. The sentence ender is actually a portmanteau suffix which encodes three distinct grammatical features: (a) “speech style” (i.e. politeness); (b) “special mood” (not discussed here); and (c) “sentence type” (i.e. speech act; this corresponds to the major mood category in other languages).\footnote{\citet{Sohn1999}, \citet{Pak2008}.} Korean has an unusually rich inventory of speech act markers. The exact number is a topic of controversy; \citet{Sohn1999} lists four major sentence types (declarative, interrogative, imperative, and “propositive” or hortative); plus several minor types including admonitive (warning), promissive, exclamatory, and apperceptive (new or currently perceived information?). Combinations of four of the speech styles with two sentence types (declarative and imperative) are illustrated in \REF{ex:}; the “sentence enders” are italicized.\footnote{These examples are taken from Pak, \citet{PortnerZanuttini2013}.}




\begin{tabularx}{\textwidth}{XXXXX}
\lsptoprule
\hhline{~----} & \bfseries Declarative &  & \bfseries Imperative & \\
\hhline{~----}
\bfseries Formal: & chayk=ul & ilk-ess-\textit{supnita}. & chayk=ul & ilk-\textit{usipsio.}\\
& book=\textsc{acc} & read-\textsc{past-decl.form} & book=\textsc{acc} & read-\textsc{imper.form}\\
& \multicolumn{2}{c}{‘I read the book.’} & \multicolumn{2}{c}{‘Please read the book!’}\\
\hhline{~----}
\bfseries Polite: & chayk=ul & ilk-ess-\textit{eyo}. & chayk=ul & ilk-\textit{useyyo.}\\
& book=\textsc{acc} & read-\textsc{past-decl.pol} & book=\textsc{acc} & read-\textsc{imper.pol}\\
& \multicolumn{2}{c}{‘I read the book.’} & \multicolumn{2}{c}{‘Please read the book.’}\\
\hhline{~----}
\bfseries Intimate: & chayk=ul & ilk-ess-\textit{e}. & chayk=ul & ilk-\textit{e}.\\
& book=\textsc{acc} & read-\textsc{past-decl.int} & book=\textsc{acc} & read-\textsc{imper.int}\\
& \multicolumn{2}{c}{‘I read the book.’} & \multicolumn{2}{c}{‘Read the book!’}\\
\hhline{~----}
\bfseries Plain: & chayk=ul & ilk-ess-\textit{ta}. & chayk=ul & ilk-\textit{ela}.\\
& book=\textsc{acc} & read-\textsc{past-decl} & book=\textsc{acc} & read-\textsc{imper}\\
& \multicolumn{2}{c}{‘I read the book.’} & \multicolumn{2}{c}{‘Read the book’}\\
\hhline{~----}
\lspbottomrule
\end{tabularx}

Like Japanese honorifics, the Korean speech style markers contribute information about the current speech act, specifically the relationship between speaker and hearer, rather than contributing to the “at-issue” propositional content of the utterance. Use of the wrong speech style marker in a particular situation would not cause a statement to be considered false, but would be felt to be inappropriate. A speaker who committed such an error would probably be corrected quickly and emphatically. Moreover, the information contributed by the speech style markers cannot be negated or questioned. The negative statement in (\ref{ex:}b) and the question in (\ref{ex:}c) are felt to be just as polite as the corresponding positive statement in (\ref{ex:}a), and would be appropriate in the same range of situations.\footnote{Thanks to Shin-Ja Hwang for very helpful discussion of these issues.}


\ea \ea \gll Pi=ka  w-ayo.\\
rain=\textsc{nom}  come-\textsc{decl.pol}\\
\glt ‘It is raining.’ (polite)

\ex \gll Pi=ka  an-w-ayo.\\
rain=\textsc{nom}  \textsc{neg}-come-\textsc{decl.pol}\\
\glt ‘It is not raining.’ (polite)

\ex \gll  Pi=ka  w-ayo?\\
rain=\textsc{nom}  come-\textsc{decl.pol}\\
\glt ‘Is it raining?’ (polite)  [\citealt{Sohn1999}:269–270]
\z
\z

\section{Other ways of marking politeness}\label{sec:} %5. /

Honorific markers and speech style markers like those discussed in the previous two sections have no descriptive content, but only a use-conditional, utterance modifying function. However, there are words in many languages which express both normal descriptive content plus a use-conditional function as a marker of politeness.



One of the most common ways across languages of showing respect or politeness to the addressee is by distinguishing polite vs. familiar forms of the second person pronoun, e.g. \textit{vous} vs. \textit{tu} in French, \textit{Sie} vs. \textit{du} in German, etc. Malay has a very complex system of first and second person pronouns. The neutral first person singular form is \textit{saya}; \textit{aku} is considered more intimate, for use with friends and family members. \textit{Beta} is the first person singular form used by royalty, and \textit{patik} is the first person singular form used by commoners when addressing royalty. There is no native Malay second person singular pronoun which is truly neutral; \textit{kamu}, \textit{awak}, and \textit{engkau} are all felt to be informal or intimate to varying degrees. The term \textit{anda} was invented as part of the standardization of Malaysian as a national language to fill this gap, but is rarely used in conversational speech. Second person pronouns tend to be avoided when addressing royalty or other highly respected people, by using titles, kin terms, etc. instead.



Lexical substitution as a means of honorification is not limited to pronouns. Balinese and Javanese are famous for their speech levels, or registers. In these languages, two or more forms are available for thousands of lexical items, e.g. Balinese \textit{makita} (high) vs. \textit{edot} (low) ‘want’; \textit{sanganan} (high) vs. \textit{jaja} (low) ‘cake’.\footnote{\citet{Arka2005}} The choice of which form to use is determined by the relative social status, caste, etc. of the speaker and addressee. Korean and Japanese also have suppletive forms for some words, e.g. Korean \textit{pap} (plain) vs. \textit{cinci} (polite) ‘cooked rice, meal’. The primary meaning contributed by words of this sort is to the truth-conditional content of the sentence; their use-conditional politeness function is in a sense secondary.


\section{Discourse particles in German}\label{sec:} %6. /

German and Dutch are well-known for their large inventories of discourse particles. These particles have been intensively studied, but their meanings are difficult to define or paraphrase. Those that occur in the “middlefield” (i.e., between the V2/Aux position and the position of clause-final verbs) have traditionally been referred to as \textit{modalpartikeln} ‘modal particles’ in German, although they do not express modality in the standard sense of that term.\footnote{Palmer (1986: 45–46).} Some examples and a description from \citeauthor{Zimmermann2011} (2011, p. 2013) are presented in \REF{ex:}.


\ea
\ea Max ist \textit{ja} auf See.\\
\ex Max ist \textit{doch} auf See.\\
\ex Max ist \textit{wohl} auf See.\\
‘Max is \textsc{prtcl} at sea.’
                       \z

\begin{quote}
“The sentences in (a–c) do not differ in propositional content: they all have the same truth-conditions… A difference in the choice of the particle (\textit{ja}, \textit{doch}, \textit{wohl}) leads to a difference in felicity conditions, however, such that each sentence will be appropriate in a different context. As a first approximation, (\ref{ex:}a) indicates that the speaker takes the hearer to be aware of the fact that Max is at sea. In contrast, (\ref{ex:}b) signals that the speaker takes the hearer not to be aware of this fact at the time of utterance. (\ref{ex:}c), finally, indicates a degree of speaker uncertainty concerning the truth of the proposition expressed. In each case, the discourse particle does not contribute to the descriptive, or propositional, content of the utterance, but to its expressive content.”
\end{quote}
\z


Most of the German “modal particles” are homophonous with a stressed variant belonging to one of the standard parts of speech. For example, stressed \textit{ja} means ‘yes’ and stressed \textit{wohl} means ‘probably’. However, when used as particles these words are unstressed and take on a variety of meanings, many of which are difficult to paraphrase or translate. Some of the variant meanings of \textit{ja} and \textit{doch} are illustrated in (\ref{ex:}--\ref{ex:}).


\ea
\ea  Die Malerei war \textit{ja} schon immer sein Hobby.\\
\glt ‘<\textit{As you know}>, painting has always been his hobby.’

\ex  Dein Mantel ist \textit{ja} ganz schmutzig.\\
\glt ‘<\textit{Hey}> your coat is all dirty.’ (not previously known to hearer)

\ex Fritz hat \textit{ja} noch gar nicht bezahlt.\\
\glt ‘<\textit{Hey}> Fred has not paid yet.’ (newly discovered by speaker)\\
{}[\citealt{König1991,KönigEtAl1990,Waltereit2001}]
\z \z

\ea \ea   A: Maria kommt mit. ‘Maria is coming with me.’\\
    B: Sie ist \textit{doch} verreist. ‘She has left, <\textit{hasn’t she}>?’
\ex  Du bist also \textit{doch} gekommen! ‘So you came <\textit{after all}>!’
\ex Ich war \textit{doch} letztes Jahr schon dort. ‘<\textit{Did you forget?}> I was here last year.’  [\citet{Karagjosova2000}; \url{http://en.wikipedia.org/wiki/German\_modal\_particle}]
\z \z 


In the passage quoted above, \citet{Zimmermann2011} states that these particles contribute to the expressive content of the utterance rather than its descriptive, or at-issue, content; they affect the felicity conditions of the utterance, but not its truth-conditions. So, for example, all of the sentences in \REF{ex:} would be true if Max is in fact at sea at the time of speaking. Using the wrong particle would make the utterance infelicitous, but not false. Other authors have reached similar conclusions. \citet{Waltereit2001} states:


\begin{quote}
{}[“Modal particles”] modify the preparatory conditions, as they evoke a speech situation in which the desired preparatory conditions are fulfilled… Preparatory conditions describe the way the speech act fits into the social relation of speaker and addressee, and they describe how their respective interests are concerned by the act.\footnote{cf. \citet{Searle1969}}
\end{quote}


\citet{Karagjosova2000} states that “[modal particles] indicate if and how incoming information in dialogue is processed by the interlocutors in terms of its consistency with the information or beliefs the interlocutors already have.” For example, modal particles may indicate whether a proposition has succeeded in becoming \textsc{grounded}, i.e., part of the shared assumptions (\textsc{common ground}) of the speaker and hearer. She continues:


\begin{quote}
{}[T]he meaning of [modal particles] seems not to be part of the proposition indeed and thus not part of the truth conditions of the sentence they occur in. …  [W]e conclude that \textit{doch} does not contribute to the sentence meaning but to the utterance meaning and represents thus semantically an utterance modifier rather than a sentence modifier.
\end{quote}


The hypothesis that German modal particles function as utterance modifiers, and do not contribute to truth-conditional content, is supported by the fact that they cannot be negated, as seen in \REF{ex:}. Moreover, they cannot be questioned and cannot function as the answer to a question.\footnote{This point is mentioned in most descriptions of the German modal particles, including \citet{Bross2012} and \citet{Gutzmann2015}.}


\ea
Hein ist \textit{ja} nicht zuhause.\\
‘\textit{As you know}, Hein is not at home.’  [\citealt{Gutzmann2015}, sec. 7.2.2.2]\\
(cannot mean: ‘You do not know that Hein is not at home.’)
\z

\section{Conclusion}\label{sec:} %7. /

In this chapter we have looked at several types of expressions in various languages that seem to contribute “use-conditional” rather than truth-conditional meanings. The characteristic properties of such expressions are those identified by Potts in his work on conventional implicatures. They tend to be speaker-oriented; independent of and secondary to the “at-issue”, truth-conditional content of the utterance; excluded from negation and questioning; and not assumed to be part of common ground.



We noted that speech act adverbials in English (e.g. \textit{frankly}, \textit{confidentially}) can function either as sentence adverbs with use-conditional meanings, or as manner adverbs with truth-conditional meanings. In future chapters we will see that similar ambiguities arise with certain conjunctions, notably \textit{because} (\chapref{sec:18}) and \textit{if} (\chapref{sec:19}). We will argue that, at least for \textit{because}, such ambiguities need not be treated as polysemy (distinct senses), but can be seen as a kind of pragmatic ambiguity: a single sense that can function on two levels, modifying the sentence meaning or the utterance meaning. In the first case, it contributes truth-conditional meaning, while in the second case it contributes use-conditional meaning.



\furtherreading



\citet{Potts2007a,Potts2007b} and (\citeyear{Potts2012}) provide concise introductions to his analysis of conventional implicatures. \citet{Potts2007c} focuses more specifically on expressives. \citet{Scheffler2013} applies this analysis to sentence adverbs in English and German. \citet{Gutzman2015} presents an introduction to the idea of use-conditional meaning in \chapref{sec:2}, and an analysis of the German “modal particles” in \chapref{sec:7}.


\subsubsection{Discussion exercise:}\label{sec:}

\textbf{A.} Use the kinds of evidence discussed in this chapter to determine whether the italicized expressions in the following examples contribute truth-conditional or use-conditional meaning:

\ea
\ea Sir Richard Whittington, \textit{a medieval cloth merchant}, served four terms as Lord Mayor of London.

\ex Wilma \textit{probably} loves sauerkraut.

\ex Fred loves sauerkraut \textit{too}.

\ex Mrs. Natasha Griggs, \textit{who served six years as MP for Darwin}, is a cancer survivor.

\ex Baxter \textit{reportedly} supported Suharto.
\z
\z

\chapter{{12}: How meanings are composed}

\section{Introduction}\label{sec:} %1. /

One of the central goals of semantics is to explain how meanings of sentences are related to the meanings of their parts. In \chapref{sec:3} we discussed the simple sentence in \REF{ex:}, and how the meaning of the sentence determines the conditions under which it would be true.


\ea
\textit{King Henry VIII snores}.
\z


Let us now consider the question of how the meaning of this sentence is composed from the meanings of its parts. What are the parts, and what kinds of meanings do they express? Any syntactic description of the sentence will recognize two immediate constituents: the subject NP \textit{King Henry VIII} and the intransitive verb (or VP) \textit{snores}. These two phrases express different kinds of meaning. The subject NP is a referring expression, specifically a proper name, which refers to an individual in the world. The intransitive VP expresses a property which may be true of some individuals but not of others in a given situation. The result of combining them, i.e. the meaning of the sentence as a whole, is a \textsc{proposition} (or claim about the world) which may be true in some situations and false in others. Sentence \REF{ex:} expresses an assertion that the individual named by the subject NP (King Henry VIII) has the property named by the VP (he snores). This pattern for combining NP meanings with VP meanings is seen in many, perhaps most, simple declarative sentences.



The same basic principle holds not just for sentences but for any expression (apart from idioms) consisting of more than one word: the meaning of the whole is composed, or built up, in a predictable way from the meanings of the parts. This is what makes it possible for us to understand newly-created sentences. One way of expressing this principle is the following:


\begin{stylepoints}
\textsc{Principle of Compositionality:\\
}the meaning of a complex expression is determined by the meanings of its constituent expressions and the way in which they are combined.
\end{stylepoints}


Many semanticists adopt as a working hypothesis a stronger version of this principle, which says (roughly speaking) that there must be a one-to-one correspondence between the syntactic rules that build constituents and the semantic rules that provide interpretations for those constituents. Adopting this stronger version of the principle places significant constraints on the way these rules get written.\footnote{\citet[322]{Partee1995}.} In \chapref{sec:13} we will see a few very simple examples of how syntactic and semantic rules can be correlated.



In this chapter we lay a foundation for discussing compositionality in the more general sense expressed in \REF{ex:}. We are trying to understand what is involved in the claim that the meanings of phrases and sentences are predictable based on the meanings of their constituents and the manner in which those constituents get combined.



We begin in \sectref{sec:2} by describing two very simple examples of compositional meaning: first, the combination of a subject NP with a VP to form a simple clause (\textit{Henry snores}); and second, the combination of a modifying adjective with a common noun (\textit{yellow} \textit{submarine}). In \chapref{sec:13} we will formulate rules to account for these patterns, among others.



In \sectref{sec:3} we provide some historical context for the study of compositionality by sketching out some ideas from the German logician Gottlob Frege (1848–1925). We will summarize Frege’s arguments for the claim that denotations, as well as senses, must be compositional. But Frege also pointed out that there are some contexts where the denotation of a complex expression is not fully predictable from the denotations of its constituents. We discuss one such context in \sectref{sec:4}, namely complement clauses of verbs like \textit{think}, \textit{believe}, \textit{want}, etc. In \sectref{sec:5} we discuss a particular type of ambiguity which can arise in such contexts.


\section{Two simple examples}\label{sec:} %2. /

Let us return now to the question of how the meaning of the simple sentence in \REF{ex:} is composed from the meanings of its parts. As we noted, the sentence contains two immediate constituents: the subject NP \textit{King Henry VIII} and the intransitive verb (or VP) \textit{snores}. The NP \textit{King Henry VIII} is a proper name, a “rigid designator”, and so always refers to the same individual; its denotation does not depend on the situation. The intransitive VP \textit{snores} expresses a property which may be true of a particular individual at one time or in one situation, but not in other times or situations; so its denotation does depend on the situation in which it is used. We will refer to the set of all things which snore in the current universe of discourse as the \textsc{denotation set} of the predicate \textit{snores}. The result of combining the subject NP with the intransitive VP is a sentence whose meaning is a proposition, and this proposition will be true just in case the individual named \textit{King Henry VIII} is a member of the denotation set of \textit{snores}; i.e., if the king has the property of snoring in the time and situation being described.



This same basic rule of interpretation works for a great many simple declarative sentences: the proposition expressed by the sentence as a whole will be true just in case the referent of the subject NP is a member of the denotation set of the VP. Of course there are many other cases for which this simple rule is not adequate; but in the present book we will touch on these only briefly.



The Principle of Compositionality also applies to complex expressions which are smaller than a sentence, including noun phrases. Even though these phrasal expressions do not have truth values, they do have denotations which are determined compositionally. In \chapref{sec:1} we briefly discussed the compositionality of the phrase \textit{yellow} \textit{submarine}. Suppose we refer to the denotation set of the word \textit{yellow} (i.e., the set of all yellow things in our universe of discourse) as Y, and the denotation set of the word \textit{submarine} (i.e., the set of all submarines in our universe of discourse) as S. The meaning of the phrase \textit{yellow} \textit{submarine} is predictable from the meaning of its individual words and the way they are combined. Knowing the rules of English allows speakers to predict that the denotation set of the phrase will be the set of all things which belong both to Y and to S; in other words, the set of all things in our universe of discourse which are both yellow and submarines.



As these simple examples illustrate, our analysis of denotations and truth values will be stated in terms of set membership and relations between sets. For this reason we will introduce some basic terms and concepts from set theory at the beginning of \chapref{sec:13}. Set theory will also be crucial for analyzing the meanings of quantifiers (words and phrases such as \textit{everyone}, \textit{some people}, \textit{most countries}, etc.). Quantifiers (the focus of \chapref{sec:14}) are an interesting and important topic of study in their own right, but they are also important because certain other kinds of expressions can actually be analyzed as quantifiers (see \chapref{sec:16}, for example).



But before we proceed with a more detailed discussion of these issues, it will be helpful to review some of Frege’s insights.


\section{3. Frege on compositionality and substitutivity}\label{sec:}

Many of the foundational concepts in truth-conditional semantics come from the work of Gottlob Frege, whose distinction between Sense and Denotation we discussed in \chapref{sec:2}. The Principle of Compositionality in \REF{ex:} is often referred to as “\textsc{Frege’s principle}”. Frege himself never expressed the principle in these words, and there is some disagreement as to whether he actually believed it.\footnote{Specifically, there is debate as to whether Frege believed that compositionality holds for senses, as well as denotations \citep[12]{Gamut1991b}. \citet{Pelletier2001}, for example, argues that he did not. A number of modern scholars have argued against the Principle of Compositionality; see \citet{Goldberg2015} for a summary.} But there are passages in several of his works that seem to imply or assume that sentence meanings are compositional in this sense, including the following:


It is astonishing what language accomplishes. With a few syllables it expresses a countless number of thoughts [=propositions], and even for a thought grasped for the first time by a human it provides a clothing in which it can be recognized by another to whom it is entirely new. This would not be possible if we could not distinguish parts in the thought that correspond to parts of the sentence, so that the construction of the sentence can be taken to mirror the construction of the thought. … The question now arises how the construction of the thought proceeds, and by what means the parts are put together so that the whole is something more than the isolated parts.   [Gottlob Frege (1923–6), “Logische Untersuchungen. Dritter Teil: Gedankengefüge”, quoted in \citealt{HeimKratzer1998}:2]


In this passage Frege argues for the compositionality of “thoughts”, i.e. propositions; but the same kind of reasoning requires that the meaning of smaller expressions (e.g. noun phrases) be compositional as well. And in many cases, not only senses but also denotations are compositional. One way of seeing this involves substituting one expression for another which is co-referential, i.e., has the same denotation in that particular context.



In our world, the expressions \textit{Abraham Lincoln} and \textit{the 16\textsuperscript{th}} \textit{president of the United States} refer to the same individual. For this reason, if we replace one of these expressions with the other as illustrated in (\ref{ex:}--\ref{ex:}), the denotation of the larger phrase is not affected.


\ea
\ea the wife of Abraham Lincoln\\
\ex the wife of the 16\textsuperscript{th} president of the United States
                       \z
\z

\ea
\ea the man who killed Abraham Lincoln\\
\ex the man who killed the 16\textsuperscript{th} president of the United States
                       \z
\z


Both of the NPs in \REF{ex:} refer to Mary Todd Lincoln; both of the NPs in \REF{ex:} refer to John Wilkes Booth. This is what we expect if the denotation of the larger phrase is compositional, i.e., predictable from the denotations of its constituent parts: replacing one of those parts with another part having the same denotation does not affect the denotation of the whole. (This principle is referred to as the principle of \textsc{substitutivity}.)



A second way of observing the compositionality of denotations arises when non-referring expressions occur as constituents of a larger expression. In a world where there is no such person as Superman, i.e., a world in which this name lacks a denotation, phrases which contain the name \textit{Superman} (like those in ) will also lack a denotation, i.e. will fail to refer.


\ea
\ea the mother of Superman\\
\ex the man who Superman rescued
                       \z
\z


These observations support the claim that the denotation of a complex expression is (often) predictable from the denotations of its constituent parts. Since sentences are formed from constituent parts (words and phrases) which have denotations, this suggests that the denotations of sentences might also be compositional. In his classic paper \textit{Über Sinn und Bedeutung} ‘On sense and denotation’, \citet{Frege1892} argued that this is true; but he recognized that it may seem odd (at least at first) to suggest that sentences have denotations as well as senses. Sentences are not “referring expressions” in the normal sense of that term, so what could their denotation be?



Frege considered the possibility that the denotation of a sentence is the proposition which it expresses. But this hypothesis leads to unexpected results when we substitute one co-referential expression for another. Samuel Clemens was an American author who wrote under the pen name Mark Twain; so these two names both refer to the same individual. Since the two names have the same denotation, we expect that replacing one name with the other, as illustrated in \REF{ex:}, will not affect the denotation of the sentence as a whole.


\ea
\ea \textit{The Prince and the Pauper} was written by Mark Twain.\\
\ex \textit{The Prince and the Pauper} was written by Samuel Clemens.
                       \z
\z


Of course, the resulting sentences must have the same truth value; it happens that both are true. However, a person who speaks English but does not know very much about American literature could, without inconsistency, believe (\ref{ex:}a) without believing (\ref{ex:}b). For Frege, if a rational speaker can simultaneously believe one sentence to be true while believing another to be false, the two sentences cannot express the same proposition.



Examples like \REF{ex:} lead to the same conclusion. Abraham Lincoln was the 16\textsuperscript{th} president of the United States, so replacing the phrase \textit{Abraham Lincoln} with the phrase \textit{the 16\textsuperscript{th}} \textit{president of the United States} should not change the denotation of the sentence as a whole. But the facts of history could have been different: Abraham Lincoln might have died in infancy, or lost the election in 1860, etc. Under those conditions, sentence (\ref{ex:}b) might well be true while sentence (\ref{ex:}a) is false. This again is evidence that the two sentences do not express the same proposition, since a single proposition cannot be simultaneously true and false in any single situation.


\ea
\ea Abraham Lincoln ended slavery in America.\\
\ex The 16\textsuperscript{th} president of the United States ended slavery in America.
                       \z
\z


Frege concludes that the denotation of a (declarative) sentence is not the proposition which it expresses, but rather its truth value. Frege identifies the proposition expressed by a sentence as its sense.



There are clear parallels between the truth value of a sentence and the denotation of a noun phrase. First, neither can be determined in isolation, but only in relation to a specific situation or universe of discourse. Second, both may have different values in different situations. Third, both are preserved under substitution of co-referring expressions. This was illustrated for noun phrases in (\ref{ex:}--\ref{ex:}), and for sentences in (\ref{ex:}--\ref{ex:}). Finally, we noted that NPs which contain non-referring expressions as constituents, like those in \REF{ex:}, will also fail to refer, i.e. will lack a denotation. In the same way, Frege argued that sentences which contain non-referring expressions will lack a truth value. He states that sentences like those in \REF{ex:} are neither true nor false; they cannot be evaluated, because their subject NPs fail to refer. These parallels provide strong motivation for considering the denotation of a sentence to be its truth value.


\ea
\ea Superman rescued the Governor’s daughter.\\
\ex The largest even number is divisible by 7.
                       \z
\z

However, certain types of sentences, such as those in \REF{ex:}, contain a non-referring expression but never-the-less do seem to have a truth value. Even in a world where there is no Santa Claus and no fountain of youth, it would be possible to determine whether these sentences are true or false. Sentences of this type are said to be \textsc{referentially opaque}, meaning that their denotation is not predictable from the denotations of their constituent parts. In these specific examples, the opacity is due to special properties of verbs like \textit{believe} and \textit{hope}. (We will discuss other types of opacity in \chapref{sec:15}.)

\ea
\ea The Governor still believes in Santa Claus.\\
\ex Ponce de León hoped to find the fountain of youth.
                       \z
\z

\section{Propositional attitudes}\label{sec:} %4. /

\textit{Believe} and \textit{hope} belong to a broad class of verbs which are often referred to as \textsc{propositional attitude verbs}, because they take a propositional argument (expressed as a complement clause) and denote the mental state or attitude of an experiencer toward this proposition. Other verbs in this class include \textit{think, expect, want, know,} etc. As we have just mentioned, the complement clauses of these verbs are referentially opaque. Some further examples of sentences involving such verbs are presented in \REF{ex:}.


\ea
\ea John believes [that the airplane was invented by an Irishman].\\
\ex Henry wants [to marry a Catholic].\\
\ex Mary knows [that Abraham Lincoln ended slavery in America].
                       \z
\z


Frege pointed out that when we substitute one co-referential expression for another in the complement clause of a propositional attitude verb, the truth value of the sentence as a whole can be affected. For example, since \textit{Mark Twain} and \textit{Samuel Clemens} refer to the same individual, the principle of substitutivity predicts that the positive statement in (\ref{ex:}a) and its corresponding negative statement in (\ref{ex:}b) should have opposite truth values. However, it is clearly possible for both sentences to be true at the same time (and for the same person named \textit{Mary}). By the same token, the principle of substitutivity predicts that (\ref{ex:}c) and (\ref{ex:}d) should have the same truth value. However, it is hard to imagine a person of normal intelligence of whom (\ref{ex:}d) could be true.


\ea
\ea Mary knows [that \textit{The Prince and the Pauper} was written by Mark Twain].\\
\ex Mary does not know [that \textit{The Prince and the Pauper} was written by\\
  Samuel Clemens].\\
\ex Mary does not know [that Samuel Clemens is Mark Twain].\\
\ex ?\#Mary does not know [that Samuel Clemens is Samuel Clemens].
                       \z
\z


As mentioned above, this property of propositional attitude verbs is called \textsc{referential opacity}; the complements of propositional attitude verbs are an example of an \textsc{opaque context}, that is, a context where denotation does not appear to be compositional, because the principle of substitutivity fails. Frege used the following pair of examples to further illustrate referential opacity. Both of the complement clauses in \REF{ex:} are true statements, but only the first is something that Copernicus actually believed (he believed that the planetary orbits were circles). Since the denotation of a declarative clause is its truth value, and since the two complement clauses have the same truth value if considered on their own, the principle of substitutivity would predict that sentences (\ref{ex:}a) and (\ref{ex:}b) as a whole should have the same denotation, i.e., the same truth value. But in fact (\ref{ex:}a) is true while (\ref{ex:}b) is false.


\ea
\ea Copernicus believed [that the earth revolves around the sun].\\
\ex Copernicus believed [that the planetary orbits are ellipses].
                       \z
\z


Propositional attitude verbs pose a significant problem for the principle of Compositionality. Frege’s solution was to propose that the denotation of a clause or NP “shifts” in opaque contexts, so that in these contexts they refer to their customary sense, rather than to their normal denotation. For example, the denotation of the complement clauses in \REF{ex:}, because they occur in an opaque context, is not their truth value but the proposition they express (their customary sense). This shift explains why NPs or clauses with different senses are not freely substitutable in these contexts, even though they may seem to have the same denotation.



Frege’s proposal is analogous in some ways to the referential “shift” which occurs in contexts where a word or phrase is \textsc{mentioned}, as in (\ref{ex:}b), rather than \textsc{used}, as in (\ref{ex:}a). In such contexts, the quoted word or phrase refers only to itself. Substitutivity fails when referring expressions are mentioned, as illustrated in (\ref{ex:}c--d). Even though both names refer to the same individual when used in the normal way, these two sentences are not equivalent: (\ref{ex:}c) is true, but (\ref{ex:}d) is false.


\ea
\ea Maria is a pretty girl.\\
\ex \textit{Maria} is a pretty name.\\
\ex Samuel Clemens adopted the pen name \textit{Mark Twain}.\\
\ex Mark Twain adopted the pen name \textit{Samuel Clemens}.
                       \z
\z


We can now understand why sentences like those in \REF{ex:}, which contain a non-referring expression, never-the-less can have a truth value. \textit{Hope} and \textit{want} are propositional attitude verbs. Thus the denotation of their complement clauses is not their truth value but the propositions they express. The denotation (i.e., truth value) of the sentence as a whole can be derived compositionally, because all the constituents have well-defined denotations.


\ea
\ea Ponce de León hoped to find the fountain of youth.\\
\ex James Thurber wanted to see a unicorn.
                       \z
\z

\section{5. \textit{De dicto} vs. \textit{de re} ambiguity}\label{sec:}

Another interesting property of opaque contexts, including the complements of propositional attitude verbs, is that definite NPs occurring in such contexts can sometimes receive two different interpretations. They can either be used to refer to a specific individual, as in (\ref{ex:}a), or they can be used to identify a type of individual, or property of individuals, as in (\ref{ex:}b).


\ea
\ea I hope to meet with \textit{the Prime Minister} next year, (after he retires from office).\\
\ex I hope to meet with \textit{the Prime Minister} next year; (we’ll have to wait for\\
  the October election before we know who that will be).
                       \z
\z


The former reading, which refers to a specific individual, is known as the \textit{de re} (‘about the thing’) interpretation. The latter reading, in which the NP identifies a property of individuals, is known as the \textit{de dicto} (‘about the word’ or ‘about what is said’) interpretation. The same kind of ambiguity is illustrated in \REF{ex:}.


\ea
\ea I wanted \textit{my husband} to be a Catholic, (but he said he was too old to convert).\\
\ex I wanted \textit{my husband} to be a Catholic, (but I ended up marrying a Sikh).
                       \z
\z


Under the \textit{de re} interpretation, the definite NP denotes a particular individual: the person who is serving as Prime Minister at the time of speaking in (\ref{ex:}a), and the individual who is married to the speaker at the time of speaking in (\ref{ex:}a). Under the \textit{de dicto} interpretation, the semantic contribution of the definite NP is not what it refers to but its sense: a property (e.g. the property of being Prime Minister, or the property of being married to the speaker) rather than a specific individual. This “shift” from denotation to sense in opaque contexts is similar to the facts about complement clauses discussed in the previous section. A similar type of ambiguity is observed with indefinite NPs, as illustrated in \REF{ex:}.


\ea
\ea The opposition party wants to nominate \textit{a retired movie} star for President.\\
\ex The Dean believes that I am collaborating with \textit{a famous linguist}.
                       \z
\z


With indefinites, the two readings are often referred to as \textsc{specific} vs. \textsc{non-specific}; but we can apply the terms \textit{de dicto} vs. \textit{de re} to these cases as well.\footnote{We follow von \citet{Heusinger2011} in using the terms this way.} Under the specific (\textit{de re}) reading, the phrase \textit{a retired movie star} in (\ref{ex:}a) refers to a particular individual, e.g. Ronald Reagan or Joseph Estrada (former president of the Philippines); so under this reading sentence (\ref{ex:}a) means that the opposition party has a specific candidate in mind, who happens to be a retired actor (whether the party leaders realize this or not). Under the non-specific (\textit{de dicto}) reading, the phrase refers to a property or type, rather than a specific individual. Under this reading sentence (\ref{ex:}a) means that the opposition party does not have a specific candidate in mind, but knows what kind of person they want; and being a retired actor is one of the qualifications they are looking for.



These \textit{de dicto}–\textit{de re} ambiguities involve true semantic ambiguity, as seen by the fact that the two readings have different truth conditions. For example, suppose I am collaborating with Noam Chomsky on a book of political essays. The Dean knows about this collaboration, but knows Chomsky only through his political writings, and does not realize that he is also a famous linguist. In this situation, sentence (\ref{ex:}b) will be true under the \textit{de re} reading but false under the \textit{de dicto} reading.



As we will see in our discussion of quantifiers (\chapref{sec:14}), \textit{de dicto}–\textit{de re} ambiguities can often be explained or analyzed as instances of \textsc{scope ambiguity}. However, the specific vs. non-specific ambiguity of indefinite NPs is found even in contexts where no scope effects are involved.\footnote{\citet{FodorSag1982}.}


\section{Conclusion}\label{sec:} %6. /

The passage from Frege quoted at the beginning of \sectref{sec:3} describes the astonishing power of human language: “[E]ven for a thought grasped for the first time by a human it provides a clothing in which it can be recognized by another to whom it is entirely new.” It is this productivity, the ability to communicate novel ideas, that we seek to understand when we try to account for the compositionality of sentence meanings.



In the next two chapters we offer a very brief introduction to a widely-used method for modeling how meanings of complex expressions are composed from the meanings of their constituent parts. Building on Frege’s intuition (discussed in \sectref{sec:3} above) that the denotation of a sentence is its truth value, we describe a method for composing denotations of words and phrases to derive the truth conditions of the proposition expressed by a sentence. Then in \chapref{sec:15} we discuss additional contexts where, as with the propositional attitude verbs discussed in \sectref{sec:4} above, a purely denotational treatment is inadequate.



\furtherreading



Abbott (2010, \sectref{sec:2}.1) provides a good summary of Frege’s famous paper on sense and denotation. \citet{Goldberg2015} and Pagin \& Westerståhl (2010) discuss some of the challenges to the Principle of Compositionality. \citet{Zalta2017} provides an overview of Frege’s life and work.


\subsubsection{Discussion exercise:}\label{sec:}

\textbf{A.} Discuss the validity of the following inference (assuming that (\ref{ex:}a) and (\ref{ex:}b) are true):

\ea
\ea
Oedipus wants to marry Jocasta

\ex Jocasta is Oedipus’ mother

-----------------

\ex Therefore, Oedipus wants to marry his mother
\z
\z
\chapter{{13}: Modeling compositionality}

\section{Introduction}\label{sec:} %1. /

We have said that one of the most important goals of semantic theory is to understand the compositional nature of meaning, i.e., the knowledge which allows speakers to correctly predict how word meanings will combine in complex expressions. One way of exploring this topic is to construct formal rule systems which model the abilities of speakers in this respect.



Just as syntacticians try to construct rule systems which replicate the judgments of native speakers about the grammaticality of sentences, semanticists try to construct rule systems which replicate the ability of speakers to identify the denotation of an expression in a particular context of use, and in particular, to determine the truth values of sentences in a given context. A crucial step in this kind of analysis is to describe the situation under discussion in very explicit terms, so that predictions about denotations can be easily checked. The explicit description of a situation is called a \textsc{model}, so this general approach to semantics is often referred to as \textsc{Model Theory}.\footnote{A \textsc{model} can also be defined an interpretation under which a given sentence or set of sentences is true \citep{Hodge2013}. But by spelling out the denotations of the basic expressions used in the sentence(s) under discussion, the model also specifies the relevant facts about a particular situation.}



This chapter provides a very brief introduction to the Model Theory approach to the study of compositionality. This approach, which has proven to be remarkably productive, involves stating rules of semantic interpretation for the constituents that are formed by productive syntactic processes. We mentioned two such processes in \chapref{sec:12}: the combination of subject NP with VP, and the combination of modifying adjective with head noun. In this chapter we will provide a bit more detail about how we might formulate the rules of semantic interpretation for these and other constituents.



Our goal in this chapter is not to provide detailed explanation of the Model Theory approach, but merely to give a glimpse of how it works and some sense of what the goals are. This will provide helpful context for our discussion in future chapters of topics such as quantifiers, modality, tense, etc.



\sectref{sec:key:2} provides a brief description of the rationale behind this approach. In \sectref{sec:3} we introduce some basic terms and concepts for describing sets and relations between sets, because our rules of interpretation will be stated in terms of set relations. \sectref{sec:key:4} introduces the formal notation that is used for specifying a \textsc{model}, in the sense defined above, and \sectref{sec:5} gives some examples of how rules of semantic interpretation might be stated for several types of syntactic constituents. The over-arching goal of all these steps is to account for the ability of native speakers to determine whether the proposition expressed by a given sentence is true or false in some particular context. This, you will recall, has been our benchmark for the analysis of sentence meanings.


\section{Why a “model” might be useful}\label{sec:} %2. /

Language is a very complex system. In earlier chapters we have studied a variety of factors that affect how hearers will interpret the meanings of sentences: lexical ambiguity, vagueness, figurative and other coerced senses, implicatures and other pragmatic inferences, knowledge about the world, etc. In order to make progress in understanding how compositionality works, the Model Theory approach attempts to isolate the rules for combining word meanings from these other complicating factors. This same basic strategy is adopted in many other fields of research as well. For example, if medical researchers are investigating genetic factors which may contribute to heart disease or diabetes, they will do everything possible to control for other contributing factors such as diet, age, exercise, life-style, environmental factors, etc. The specification of a test situation in terms of an explicit “model”, as illustrated below, within which the rule system can be tested, is a way of controlling for lexical ambiguity, vagueness, incomplete knowledge about the world, etc.



A model must specify two things: first, the set of all individual entities in the situation; and second, the denotations of the basic vocabulary items of the language, at least those that occur in the expressions being analyzed. This would include words which function as predicates (verbs, adjectives, and common nouns), and proper names, but not the non-denoting words like \textit{not}, \textit{and}, \textit{if}, etc. Our semantic analysis can then be stated in terms of rules of interpretation, which will specify the denotation of complex expressions formed by combining these vocabulary items according to the syntactic rules of the language.



As a preliminary example, imagine a very simple situation which contains just three individuals: King Henry VIII, Anne Boleyn, and Thomas More. Our model of this situation would include the listing of these individuals, plus the denotation sets for the content words available for use. Let us begin with a limited vocabulary consisting of just three proper names (\textit{Henry}, \textit{Anne}, and \textit{Thomas}) plus three predicate words: \textit{snore}, \textit{man}, and \textit{woman}. The denotation set for \textit{man} would include Henry VIII and Thomas More. The denotation set for \textit{woman} would include just Anne Boleyn. Let’s assume that King Henry VIII is the only person in this situation who snores; then he would be the only member of the denotation set for \textit{snore}. The denotation of the proper name \textit{Thomas} would be the individual Thomas More, etc.



In \chapref{sec:12} we stated a rule of interpretation for simple sentences: the proposition expressed by a (declarative) sentence will be true just in case the referent of the subject NP is a member of the denotation set of the VP. We can use this rule to evaluate sentence (\ref{ex:}a) relative to the situation described by the model we have just constructed. The rule says that the sentence will be true just in case the individual named \textit{Henry} (i.e., King Henry VIII) is a member of the denotation set of \textit{snore}. Since this is true in our model, the sentence is true relative to this model. The same rule of interpretation allows us to determine that sentence (\ref{ex:}b) is false relative to this model. In \chapref{sec:14} we will discuss additional rules that will allow us to evaluate (\ref{ex:}c), which is false relative to this model, and (\ref{ex:}d), which is true relative to this model.


\ea
\ea \textit{Henry snores}.\\
\ex \textit{Anne snores}.\\
\ex \textit{All men snore}.\\
\ex \textit{No women snore}.
                       \z
\z


Notice that this approach seeks to provide an account for compositional meaning, but not for the meanings (i.e., senses) of individual content words. In other words, Model Theory does not try to represent the process by which speakers of English determine that King Henry VIII would be referred to as a \textit{man} and Anne Boleyn would be referred to as a \textit{woman}, etc. We simply start with a model which specifies the denotation sets for content words. In adopting this approach, we are not denying the important role that word senses play in our use of language, or treating word meanings as a trivial issue that can be taken for granted. In fact, accounting for word meanings is a very complex and difficult undertaking, as our earlier discussions of the issue have demonstrated. Rather, the Model Theory approach assumes that it is possible to make progress in understanding compositionality without solving all of the difficult questions surrounding word meanings; and this strategy has proven to be extremely successful and productive.



As we have already hinted, the rules of interpretation which we formulate will be stated in terms of set membership and relations between sets. For that reason, before we proceed with our discussion of compositionality, we need to introduce some of the basic terminology and notation used for speaking about sets.


\section{Basic concepts in set theory}\label{sec:} %3. /

A \textsc{set} (in the mathematical sense) is a clearly-defined collection of things. We use braces, or “curly brackets”, to represent sets. So, for example, the denotation set of the word \textit{man} in the simple model described above could be written as shown in (\ref{ex:}a). This is a set which contains two elements, or \textsc{members}, both of which are men. If we focus on denotation sets of content words, the members of a set will normally all be the same kind of thing, as in (\ref{ex:}a). For sets in general, however, this does not have to be the case. The set defined in (\ref{ex:}b) contains four members which are very different from each other; but this is still a well-defined set.


\ea
\ea  \{ King Henry VIII, Thomas More \}
\ex  \{Orwell’s novel \textit{1984}, Noam Chomsky, $\sqrt{2}$, Sally McConnell-Ginet’s breakfast muffin on 4-Sept-1988\}\footnote{This example is taken from Chierchia \& McConnell-\citet[431]{Ginet1990}.}
\z
\z


The identity of a set is defined by its membership. If two sets have the same members, they are in fact the same set. When we list the members of a set, the order in which the members are listed is irrelevant; so all of the orderings shown in \REF{ex:} describe the same set:


\ea
\{a,b,c\} = \{b,a,c\} = \{c,a,b\} = \{a,b,c,b,a\} etc. 
\z


We use the Greek letter epsilon to indicate that a certain element belongs to a given set. The formula “x ${\in}$ B” can be read as: “x is a member (or element) of set B”. This would be true, for example, if B = \{x,y,z\}; but false if B = \{w,y,z\}. The formula “x ${\notin}$ B” means that x is not a member of set B.



It is possible for a set to have an infinite number of members. Examples of such sets include the set of all integers; the set of all rational numbers (i.e., quotients of integers); the set of all finite strings of letters of the Roman alphabet; the set of all finite strings of words found in the Oxford English Dictionary; and the set of all real numbers. (The membership of this last set turns out to be a higher order of infinity than that of the other sets just mentioned; but that topic will not concern us here.)



It is possible for a set to have no members. In fact, there is exactly one set of this kind, and it is called the \textsc{empty set} (often symbolized as “⌀”). The fact that there can be only one empty set follows from the principle that a set is defined by its membership. (If there were two sets, A and B, both of which have no members, then they would contain exactly the same members; and so by the principle stated above, they would be the same set.)



A set is distinct from any of its members. A set containing just one element is a different thing from the element itself. For example, the set consisting of a single individual, e.g. \{Paul~Kroeger\}, is not the same thing as the individual himself. \{Paul~Kroeger\} is an abstract concept, but Paul Kroeger is (at the time of writing) a living, breathing human being. To take another example, the empty set is not the same as nothing; it is a set that contains nothing. And the set containing the empty set is not itself empty; it has exactly one member, namely the empty set:


\ea
  \{⌀\} ≠ ⌀
\z


The \textsc{cardinality} of a set is the number of members or elements which belong to that set. For example, the cardinality of the set \{a,b,c\} is 3, because it has three members. We use the symbol {\textbar}B{\textbar} to refer to the cardinality of set B; so {\textbar}\{a,b,c\}{\textbar} = 3. Some further examples are given below:


\ea
{\textbar}\{a,b,c,d,f\}{\textbar} = 5\\
{\textbar}⌀{\textbar} = 0\\
{\textbar}\{⌀\}{\textbar} = 1
\z


In order for a given collection of things to be a well-defined set, it must be possible to determine precisely what is and is not a member of the set. For example, the phrase \textit{the set of all sets that do not contain themselves} does not identify a well-defined set. This is because its membership cannot be precisely determined. In fact, the proposed definition of the set gives rise to a paradox. Suppose that such a set exists. Does this set contain itself? If so, then it is not a “set that does not contain itself” and so should not be a member of the set. But if it is not a member of the set, then it does not contain itself, and so it must belong to the set.\footnote{This puzzle is a version of “Russell’s paradox”, which Bertrand Russell discovered in 1901 and described in a letter to Frege on June 16, 1902. Apparently it had also been noticed by Ernst Zermelo a few years earlier. It posed a major challenge to Frege’s work on the foundations of mathematics.}



The membership of a set can be specified either by listing its members, as in (-), or by stating a rule of membership (e.g., \textit{the set of all female British monarchs}, \textit{the set of all months whose name includes the letter “r”}, \textit{the set of all integers}, etc.). A general notation for defining the membership of a set is illustrated in \REF{ex:}, which is one way of describing the set of all even numbers (we will call this set E): ‘the set of all numbers which are divisible by 2’. In this notation, the variable is assumed to be an element of the currently relevant \textsc{universal set}, or universe of discourse.\footnote{See \chapref{sec:4}.} The colon in this notation stands for ‘such that’. (Some authors use a vertical bar {\textbar} instead of the colon.) If we assume that the currently relevant universal set is the set of all real numbers, then the set description in \REF{ex:} can be read as: ‘the set of all real numbers x such that x/2 is an integer.’


\ea
E = \{x: $\frac{x}{2}$ is an integer\}
\z

\subsection{3.1  Relations and functions}\label{sec:}

Up to this point all of our examples have involved sets of individuals: numbers, letters, people, etc. But we can also define sets of couples (or triples, quadruples, etc.) of individuals. For example, the set of all married couples who crossed the Atlantic ocean on the \textit{Mayflower} in the autumn of 1620 is a well defined set. This set contained 18 members, and each member of the set was a pair of people: \{Isaac \& Mary Allerton, William \& Dorothy Bradford, William \& Mary Brewster, Myles \& Rose Standish, Edward \& Elizabeth Winslow, …\}. Since the set is defined as a set of pairs, William Bradford (the first governor of the Plymouth Bay colony) was not himself a member of this set; but he was a member of a pair that did belong to the set.



In this example, the members of each pair can be distinguished by the title “Mr.” vs. “Mrs.”, no matter which one is mentioned first; but this is not always the case. As we will see, it is often useful to define sets of pairs of things in which the members of each pair are distinguished by specifying the order in which they occur. We refer to such pairs as \textsc{ordered pairs}, using the notation $\langle$x,y$\rangle$ to represent the pair which consists of x followed by y. Unlike sets, two ordered pairs may have the same members but still be distinct, if those members occur in different orders. So $\langle$x,y$\rangle$ and $\langle$y,x$\rangle$ are two distinct ordered pairs, but \{x,y\} and \{y,x\} are two different ways of representing the same set.



A set of ordered pairs is called a \textsc{relation}. The \textsc{domain} of the relation is the set of all the first elements of each pair and its \textsc{range} is the set of all the second elements. So, referring to the two sets defined in \REF{ex:}, the domain of A is the set \{a,c,f\}, while the range of A is the set \{3,4,6,7\}. The domain of B is the set \{2,3,4,5,6,7\}, while the range of B is the set \{2,3,4,7\}. 


\ea
A = \{$\langle$a,3$\rangle$, $\langle$f,4$\rangle$, $\langle$c,6$\rangle$, $\langle$a,7$\rangle$\}\\
B = \{$\langle$2,3$\rangle$, $\langle$3,2$\rangle$, $\langle$4,7$\rangle$, $\langle$5,2$\rangle$, $\langle$6,7$\rangle$, $\langle$7,4$\rangle$\}
\z


A set of ordered pairs defines a \textsc{mapping}, or correspondence, from the domain onto the range. The mappings defined by sets A and B are shown in \REF{ex:}:

\todo{draw object}
\ea
\ea  Set A\\
\begin{tikzpicture}
\matrix[matrix of nodes] (matrix1) {
a &[3em] 3\\
  & 7     \\
c & 6     \\
f & 4     \\
};
\draw[-{Triangle[]}] (matrix1-1-1) -- (matrix1-1-2); 
\draw[-{Triangle[]}] (matrix1-1-1) -- (matrix1-2-2);
\draw[-{Triangle[]}] (matrix1-3-1) -- (matrix1-3-2);
\draw[-{Triangle[]}] (matrix1-4-1) -- (matrix1-4-2);
\end{tikzpicture}
\ex  Set B\\
\begin{tikzpicture}
\matrix[matrix of nodes] (matrix2) {
2 &[3em] 3\\
3 & 2\\
4 & 7\\
5 & \\
6 & \\
7 & 4\\
};
\draw[-{Triangle[]}] (matrix2-1-1) -- (matrix2-1-2); 
\draw[-{Triangle[]}] (matrix2-2-1) -- (matrix2-2-2);
\draw[-{Triangle[]}] (matrix2-3-1) -- (matrix2-3-2);
\draw[-{Triangle[]}] (matrix2-4-1) -- (matrix2-2-2);
\draw[-{Triangle[]}] (matrix2-5-1) -- (matrix2-3-2);
\draw[-{Triangle[]}] (matrix2-6-1) -- (matrix2-6-2);
\end{tikzpicture}

\z \z


A \textsc{function} is a relation (= a set of ordered pairs) in which each element of the domain is mapped to a single, unique value in the range. The relation which corresponds to set A above is not a function, because A contains two distinct ordered pairs which have the same first element ($\langle$a,3$\rangle$ and $\langle$a,7$\rangle$). The relation which corresponds to set B is a function, even though B contains distinct ordered pairs which have the same second element ($\langle$3,2$\rangle$ and $\langle$5,2$\rangle$; $\langle$4,7$\rangle$ and $\langle$6,7$\rangle$). What matters is that each member of the domain occurs in just one ordered pair.



The function B is defined in \REF{ex:} by listing all the ordered pairs which belong to it. Another way of defining this same function is shown in \REF{ex:}. The first member of each ordered pair is called an \textsc{argument} of the function, while the second member of each ordered pair is called a \textsc{value}. The information in \REF{ex:} is equivalent to that in (\ref{ex:}b), showing how the function maps each argument onto a unique value. The format in \REF{ex:} is more convenient for stating the value which corresponds to a single argument, when we do not need to list the entire set.


\ea
B(2) = 3\\
B(3) = 2\\
B(4) = 7\\
B(5) = 2\\
B(6) = 7\\
B(7) = 4
\z


The membership of any set can be expressed as a function which maps the elements of that function onto the set \{1,0\}. In this context, 1 represents “True” and 0 represents “False”. Functions of this kind are called the \textsc{characteristic functions} (or, sometimes, “membership functions”). For example, the characteristic function of set C (members of the Beatles, as specified in a), is the function f\textsubscript{1} as defined in (\ref{ex:}a). The characteristic function of set D (numbers between 10 and 20, as specified in b) is the function f\textsubscript{2} as defined in (\ref{ex:}b). (The abbreviation “iff” stands for “if and only if”.)


\ea
\ea C = \{John, Paul, George, Ringo\}\\
\ex D = \{x: 10 < x < 20\}
                       \z
\z

\ea
\ea  f\textsubscript{1}(John) = 1\\
f\textsubscript{1}(Paul) = 1\\
f\textsubscript{1}(George) = 1\\
f\textsubscript{1}(Ringo) = 1\\
in all other cases, f\textsubscript{1}(x) = 0
\ex f\textsubscript{2}(x) = 1 iff 10 < x < 20\\
in all other cases, f\textsubscript{2}(x) = 0
\z \z

\subsection{3.2  Operations and relations on sets}\label{sec:}

When we use set concepts and terminology as a tool for interpreting sentences, we will often want to say something about the relationship between two sets, or to combine two or more sets in certain ways to define a new set. In order for this to be possible, we must assume that the elements of each of the sets under discussion are drawn from the same universal set. This universal set is referred to as U.



A very important relation which may hold between two sets is the \textsc{subset} relation, also referred to as \textsc{set} \textsc{inclusion}. We say that set A is a \textsc{subset} of set B (written “A${\subseteq}$B”) if A is included in B; that is, if all the elements which are members of A are also members of B. We can illustrate this situation using the sets defined in \REF{ex:}. The universal set U is assumed to be the set of all integers between 1 and 10. By comparing the elements in set A with those in set B, we see that all the elements which are members of A are also members of B; so in this context, “A${\subseteq}$B” is a true proposition. However, “B${\subseteq}$A” would be false in this context, because there are some members of B which are not members of A, namely 2, 5, and 7.


\ea
U = \{1,2,3,4,5,6,7,8,9,10\}\\
A = \{3,4,6\}\\
B = \{2,3,4,5,6,7\}
\z


\figref{fig:key:1} illustrates the subset relation in the form of a diagram, where each oval represents one of the sets.\footnote{This way of representing sets is called a Venn diagram.} Additional examples in standard set notation are provided in \REF{ex:}.


\begin{figure}
% \includegraphics[width=\textwidth]{figures/KroegerIntroSemPragprepubv2g-img1.png}
\begin{tikzpicture}
 \node at (0,0) (A) {A}; 
 \node[left=3em of A] (B) {B};
 \node[fit =(A),draw,ellipse] (fit1) {};
 \node[fit =(B) (fit1),draw,ellipse] (fit2) {};
\end{tikzpicture}

  \textsf{A}\textbf{${\subseteq}$}\textsf{B}\\

\caption{\label{fig:key:1} Set inclusion (the subset \textbf{relation)}}
\end{figure}

\ea
\ea \{a,b,c\} ${\subseteq}$ \{a,b,c,d,f\}\\
\ex \{a,b,c\} ${\not\subset}$ \{c,d,f\}  ** need correct symbol **\\
\ex \{a,b,c\} ${\subseteq}$ \{a,b,c\}\\
\ex ${\forall}$S (where S is a set), ⌀ ${\subseteq}$ S
                       \z
\z


Every set is a subset of itself, because all the elements which are members of set A are by definition members of set A. For this reason, the proposition “A${\subseteq}$A” will be true whenever A is a well-defined set, as illustrated in (\ref{ex:}c). If we want to specify that set A is a subset of set B, but that the two sets are not equal, we can write “A${\subset}$B”. This symbol means that set A is a \textsc{proper subset} of set B. The proposition “A${\subset}$A” will be false for any set A.



Since the elements of every set must be members of the current universal set U, “A${\subseteq}$U” must always be true. If “U${\subseteq}$A” is true, than it must be the case that A=U.



The \textsc{intersection} of two sets, written “A${\cap}$B”, is defined as the set consisting of all elements which are both members of A and members of B. We can illustrate this situation using the sets defined in \REF{ex:}. By comparing the elements in set A with those in set B, we see that the two sets share only the following elements in common: 3, 4, and 6; so A${\cap}$B = \{3,4,6\}.


\ea
U = \{1,2,3,4,5,6,7,8,9,10\}\\
A = \{2,3,4,6\}\\
B = \{3,4,5,6,7,8\}
\z


\figref{fig:key:2} illustrates set intersection in the form of a diagram: the ovals represent two sets, labeled A and B, while the shaded portion which is included in both ovals represents the intersection of the two sets (A${\cap}$B). Another example in standard set notation is provided in \REF{ex:}.

\todo{drawobject}
\begin{figure}
\begin{tikzpicture}
\scope % A \cap B
\clip (0,0) circle (1);
\fill[fill=gray!40] (1,0) circle (1);
\endscope
\draw (0,0) circle (1)
      (1,0) circle (1);
\node at (-.5,0) {A};
\node at (1.5,0) {B};
\node at (0.5,-1.5) (text) {\textsf{A${\cap}$B}};
\draw[-{Triangle[]}] (text.north) -- (0.5,0);
\end{tikzpicture}
 
% % \includegraphics[width=\textwidth]{figures/KroegerIntroSemPragprepubv2g-img2.png}


\caption{\label{fig:key:2}: Set intersection}
\end{figure}

\begin{stylepoints}
\{a,b,c\} ${\cap}$ \{c,d,f\} = \{c\}
\end{stylepoints}


The \textsc{union} of two sets, written “A${\cup}$B”, is the set consisting of all elements which are either members of A or members of B. Returning to the sets defined in \REF{ex:}, the union of the two sets is formed by combining all the elements from both, which yields the following result: A${\cup}$B = \{2,3,4,5,6,7,8\}. \figref{fig:key:3} illustrates this in the form of a diagram, and another example in standard set notation is provided in \REF{ex:}.


\begin{figure}
\begin{tikzpicture}
\scope % A \cap B
\clip (0,0) circle (1);
\fill[fill=gray!40] (1,0) circle (1);
\endscope
\draw[fill=gray!40] (0,0) circle (1)
      (1,0) circle (1);
\node at (-.5,0) {A};
\node at (1.5,0) {B};
\node at (0.5,-1.5) (text) {\textsf{A${\cup}$B}};
\end{tikzpicture}

\caption{\label{fig:key:3}Set union}
\end{figure}

\begin{stylepoints}
\{a,b,c\} ${\cup}$ \{c,d,f\} = \{a,b,c,d,f\}
\end{stylepoints}


The \textsc{complement} of set A, written as \=A or A[2B9?], is defined as the set which contains all the elements of U that are not elements of A. Some simple examples are shown in \REF{ex:}. Here, the only elements of U which are not in A are 1 and 5, so \=A = \{1,5\}. Similarly, the elements of U which are not in B are 1, 2, 5, and 6; so \textsuperscript{ [35E?]}B = \{1,2,5,6\}.


\ea
U = \{1,2,3,4,5,6\}\\
A = \{2,3,4,6\}\\
\=A = \{1,5\}\\
B = \{3,4\}\\
\=B = \{1,2,5,6\}
\z


This basic notion of complement set involves complements relative to the universal set U. It is often useful to refer to the complement of one set relative to some other set. The complement of A relative to B, written “B–A”, is the set consisting of all elements which are members of B but not members of A.\footnote{This operation is sometimes referred to as “set subtraction.”} Another way of expressing this definition is the following: B–A = B${\cap}$\textsuperscript{ [35E?]}A. \figref{fig:key:4} illustrates this in the form of a diagram, and several examples in standard set notation are provided in \REF{ex:}.

\todo{drawobject}
\begin{figure}
% [Warning: Draw object ignored]\\
  
\begin{tikzpicture}
\scope
\clip (-2,-2) rectangle (2,2)
      (1,0) circle (1);
\fill[white] (0,0) circle (1);
\endscope
\scope
\clip (-2,-2) rectangle (2,2)
      (0,0) circle (1);
\fill[gray!40] (1,0) circle (1);
\endscope
\draw (0,0) circle (1)
      (1,0) circle (1);
\node at (-.5,0) {A};
\node at (1.5,0) {B};
\node at (1.25,-1.5) (text) {\textsf{B–A}};
\draw[-{Triangle[]}] (text.north) -- (1.25,-.5);
\end{tikzpicture}
\caption{\label{fig:key:4} Set complementation}
\end{figure}

\begin{stylepoints}
\{a,b,c\} – \{b,c\} = \{a\}\\
\{a,b,c,d,f\} – \{a,b,c,j,k,p\} = \{d,f\}\\
A – ⌀ = A\\
⌀ – A = ⌀\\
U – A = \=A
\end{stylepoints}


To summarize, we have defined three basic operations on sets (\textsc{intersection}, \textsc{union}, and \textsc{complement} or “difference”), and one relation between sets, namely \textsc{inclusion} (the \textsc{subset} relation). The three operations provide ways of combining two existing sets to define a new set. It is important to note that “A${\cap}$B”, “A${\cup}$B”, and “B–A” are names of sets; but “A${\subseteq}$B” is a proposition, a claim about the membership of the two sets, which could be true or false.



More precise definitions of set intersection, union, complementation, and inclusion (the subset relation) are provided in \REF{ex:}. These definitions will help us to understand, for example, why the interpretation of an “and” statement frequently involves the intersection of two sets while the interpretation of an “or” statement frequently involves the union of two sets.


\ea
${\forall}$x [x ${\in}$ (A${\cap}$B)  $\leftrightarrow $  ((x${\in}$A) $\wedge$ (x${\in}$B))]  [\textsc{intersection}]\\
${\forall}$x [x ${\in}$ (A${\cup}$B)  $\leftrightarrow $  ((x${\in}$A) $\vee$ (x${\in}$B))]  [\textsc{union}]\\
${\forall}$x [x ${\in}$ (A–B)  $\leftrightarrow $  ((x${\in}$A) $\wedge$ (x${\notin}$B))]  [\textsc{complement}]\\
(A ${\subseteq}$ B)  $\leftrightarrow $  ${\forall}$x [(x${\in}$A) → (x${\in}$B)]  [\textsc{subset}]
\z

\section{Truth relative to a “model”}\label{sec:} %4. /

We have noted several times that denotations, including the denotations of referring expressions and truth values of sentences, can only be evaluated relative to a particular situation of use. In order to develop and test a set of interpretive rules, which can correctly predict the denotation of a particular expression in any given situation, it is important to provide very explicit descriptions for the test situations. As stated above, this kind of description of a situation is called a \textsc{model}, and must include two types of information: (i) the \textsc{domain}, i.e., the set of all individual entities in the situation; and (ii) the denotation sets for the basic vocabulary items in the expressions being analyzed.



As a first illustration of how the system works, let us return to our simple situation containing just three individuals: King Henry VIII, Anne Boleyn, and Thomas More. Our model of this situation, which we might call Model 1, would provide the information listed in \REF{ex:}. We often use the name “U” as a convenient way to refer to the domain (the “universal set” of individuals). The notation \textsc{$\llbracket$ x}$\rrbracket$  represents the denotation (or “semantic value”) of x within the current model. This notation can be used either for object language expressions or for logical formulae; so, for example, $\llbracket$ SNORE$\rrbracket$  names the same set as $\llbracket$ \textit{snores}$\rrbracket$ . By convention we use small letters for logical “constants”, e.g. proper names, and capital letters for predicates.


\textbf{Model 1}

\begin{enumerate}
\item the set of individuals U = \{ King Henry VIII, Anne Boleyn, Thomas More \}
\item denotations\textsc{:\\
{}$\llbracket$ }\textsc{MAN}$\rrbracket$  = \{ King Henry VIII, Thomas More \}\\
\textsc{$\llbracket$}WOMAN$\rrbracket$  = \{ Anne Boleyn \}\\
\textsc{$\llbracket$}SNORE$\rrbracket$  = \{ King Henry VIII \}\\
\textsc{$\llbracket$}a$\rrbracket$  = Anne Boleyn\\
\textsc{$\llbracket$}h$\rrbracket$  = King Henry VIII\\
\textsc{$\llbracket$}t$\rrbracket$  = Thomas More
\end{enumerate}

The denotation sets encode information about the current state of the world. For example, this model indicates that King Henry VIII is the only person in the current situation who snores. We can use the defined vocabulary items to build simple declarative sentences about the individuals in this situation, and then try to provide interpretations for each sentence in terms of set membership, as illustrated in \REF{ex:}. These interpretations express the truth conditions for each sentence. We can use them to evaluate the truth of each sentence relative to Model 1. For example, the sentence in (\ref{ex:}a), \textit{Thomas More is a man}, will be true in any situation where the individual Thomas More is a member of the denotation set of the word \textit{man}. Since this is the case in Model 1, the sentence is true relative to this model.




\begin{tabularx}{\textwidth}{XXXXXX}
\lsptoprule

\bfseries\scshape English sentence & \multicolumn{2}{c}{\bfseries\scshape logical form} & \multicolumn{2}{c}{\bfseries\scshape interpretation} & \bfseries\scshape truth value\\

a. \textit{Thomas More is a man}. & \multicolumn{2}{c}{MAN(t)
\newline
Thomas More ${\in}$ \textsc{$\llbracket$}MAN$\rrbracket$ } & \multicolumn{2}{c}{ T} & \\

ex \textit{Anne Boleyn is} \textit{a man or} \textit{a woman}. & \multicolumn{2}{c}{MAN(a) $\vee$ WOMAN(a)} & \multicolumn{2}{c}{Anne Boleyn ${\in}$
\newline
(\textsc{$\llbracket$}MAN$\rrbracket$  ${\cup}$ \textsc{$\llbracket$}WOMAN$\rrbracket$ )} & T\\

ex \textit{Henry VIII is a man who snores}. & \multicolumn{2}{c}{MAN(h) $\wedge$ \textsc{SNORE}(h)} & \multicolumn{2}{c}{Henry VIII ${\in}$ (\textsc{$\llbracket$}MAN$\rrbracket$  ${\cap}$ \textsc{$\llbracket$ SNORE}$\rrbracket$ )} & T\\

 \textsc{$\llbracket$ WO}MAN$\rrbracket$  ${\cap}$ \textsc{$\llbracket$ SNORE}$\rrbracket$  = ⌀Td. \textit{All men snore}.& \multicolumn{2}{c}{${\forall}$x[MAN(x) → \textsc{SNORE}(x)]} & \multicolumn{2}{c}{\textsc{$\llbracket$}MAN$\rrbracket$  ${\subseteq}$ \textsc{$\llbracket$ SNORE}$\rrbracket$ } & F\\
 
\multicolumn{1}{c}{¬${\exists}$x[WOMAN(x) $\wedge$ \textsc{SNORE}(x)]
\newline
ex \textit{No women snore.}} &  & \multicolumn{2}{c}{} & \multicolumn{2}{c}{}\\
\lspbottomrule
\end{tabularx}
\todo{check alphabetical labels}

The interpretations in (\ref{ex:}b-e) can be derived from the corresponding logical forms, based on the definitions of intersection, union, and subset provided in \REF{ex:}. For example, the \textit{or} statement in (\ref{ex:}b) constitutes a claim that a certain individual (Anne Boleyn) is a member of the union of two sets, because the definition of A${\cup}$B involves an \textit{or} statement. Once the truth conditions are stated in terms of set relations, we can determine the truth values for each sentence by inspecting the membership of the denotation sets specified in the model. The statement in (\ref{ex:}b) is true relative to Model 1 because the individual Anne Boleyn is a member of the set $\llbracket$ WOMAN$\rrbracket$ , and thus a member of $\llbracket$ MAN$\rrbracket$  ${\cup}$ $\llbracket$ WOMAN$\rrbracket$ .


\section{Rules of interpretation}\label{sec:} %5. /

Stating the truth conditions for individual sentences like those in \REF{ex:} is a useful first step, but does not yet replicate what speakers can do in their productive use of the language. Ultimately our goal is to provide general rules of interpretation which will predict the correct truth conditions for sentences based on their syntactic structure. As a further step toward this goal, let us return to the sentence in (\ref{ex:}a), which we have already discussed several times.


\ea
\ea \textit{King Henry VIII snores}.\\
\ex \textit{Anne Boleyn} \textit{snores}.
                       \z
\z


We have already stated an informal rule of interpretation for simple sentences: the proposition expressed by a (declarative) sentence will be true just in case the referent of the subject NP is a member of the denotation set of the VP. We can now restate this rule in a slightly more formal manner. We will assume that the basic syntactic structure of the clause is [NP VP]. The semantic rule we wish to state operates in parallel with the syntactic rule which licenses this structure, as suggested in \REF{ex:}. (Recall that the semantic value, i.e. the denotation, of a sentence is its truth value.)


\ea
\textbf{syntax}: S  →  NP\textsubscript{subj}  VP\\
\textbf{semantics}: The semantic value of a sentence is “true” if the semantic value of the subject is a member of the set which is the semantic value of the VP, and “false” otherwise;\\
{}$\llbracket$ S$\rrbracket$  = ‘true’  iff  $\llbracket$ NP\textsubscript{subj}$\rrbracket$  ${\in}$ $\llbracket$ VP$\rrbracket$ 
\z


Applying this rule to the sentence in (\ref{ex:}a), we get the formula in \REF{ex:}.This formula says that the sentence will be true just in case King Henry VIII is a member of the denotation set of \textit{snores}. Since this is true in our model, the sentence is true relative to this model. The same rule of interpretation allows us to determine that sentence (\ref{ex:}b) is false relative to this model.


\ea
{}$\llbracket$ \textit{King Henry VIII snores}$\rrbracket$  = ‘true’  iff  $\llbracket$ \textit{King Henry VIII}$\rrbracket$  ${\in}$ $\llbracket$ \textit{snores}$\rrbracket$ 
\z


The statement in \REF{ex:} can be expressed in logical notation as in (\ref{ex:}a). This formula is a specific instance of the general rule for evaluating the truth of propositions involving a one-place predicate. This general rule, shown in (\ref{ex:}b), states that the proposition \textit{P($\alpha $)} is true if and only if the entity denoted by \textit{$\alpha $} is an element of the denotation set of \textit{P}.


\ea
\ea  $\llbracket$ SNORE(h)$\rrbracket$  = ‘true’  iff  $\llbracket$ h$\rrbracket$  ${\in}$ $\llbracket$ SNORE$\rrbracket$ 
\ex  if $\alpha $ refers to an entity and P is a one-place predicate,\\
  then  $\llbracket$ P($\alpha $)$\rrbracket$  = ‘true’  iff  $\llbracket$ $\alpha $ $\rrbracket$  ${\in}$ $\llbracket$ P$\rrbracket$ 
\z \z


Let us now add a few more vocabulary items to our simple model, calling the new version Model 1[2B9?]. This revised model presumably reflects the early period of the marriage, ca. 1532–1533 AD, when Henry and Anne were happy and in love. Note also that Thomas More had fallen out of favor with the king around this time.


\textbf{Model 1[2B9?]}

\begin{enumerate}
\item the set of individuals U = \{ King Henry VIII, Anne Boleyn, Thomas More \}
\item denotations\textsc{:\\
{}$\llbracket$ }\textsc{MAN}$\rrbracket$  = \{ King Henry VIII, Thomas More \}\\
\textsc{$\llbracket$}WOMAN$\rrbracket$  = \{ Anne Boleyn \}\\
\textsc{$\llbracket$}SNORE$\rrbracket$  = \{ King Henry VIII \}\\
\textsc{$\llbracket$}HAPPY$\rrbracket$  = \{ King Henry VIII, Anne Boleyn \}\\
\textsc{$\llbracket$}LOVE$\rrbracket$  = \{ $\langle$King Henry VIII, Anne Boleyn$\rangle$, $\langle$ Anne Boleyn, King Henry VIII$\rangle$ \}\\
\textsc{$\llbracket$}ANGRY\_AT$\rrbracket$  = \{ $\langle$King Henry VIII, Thomas More$\rangle$ \}\\
\textsc{$\llbracket$}a$\rrbracket$  = Anne Boleyn\\
\textsc{$\llbracket$}h$\rrbracket$  = King Henry VIII\\
\textsc{$\llbracket$}t$\rrbracket$  = Thomas More
\end{enumerate}

Model 1[2B9?] includes some two-place (i.e, transitive) predicates, and should allow us to evaluate simple transitive sentences like those in \REF{ex:}. The denotation set of a transitive predicate like LOVE or ANGRY\_AT is not a set of individuals, but a set of ordered pairs. Sentence (\ref{ex:}a) expresses the proposition stated by the logical formula in (\ref{ex:}a). The truth conditions for this proposition are stated in terms of set membership in (\ref{ex:}b): the proposition will be true just in case the ordered pair $\langle$King Henry VIII, Anne Boleyn$\rangle$ is a member of the denotation set of LOVE. Since this is true in Model 1[2B9?], sentence (\ref{ex:}a) is true with respect to this model. The formula in (\ref{ex:}b) is an instance of the general pattern stated in (\ref{ex:}c).


\ea
\ea \textit{King Henry VIII loves Anne Boleyn}.\\
\ex \textit{King Henry VIII is angry at} \textit{Thomas More}.
                       \z
\z

\ea
\ea  LOVE(h,a)\\
\ex  $\llbracket$ LOVE(h,a)$\rrbracket$  = ‘true’  iff  $\langle$$\llbracket$ h$\rrbracket$ ,$\llbracket$ a$\rrbracket$ $\rangle$ ${\in}$ $\llbracket$ LOVE$\rrbracket$ 
\ex  if $\alpha $, $\beta $ refer to entities and P is a two-place predicate,\\
  then  $\llbracket$ P($\alpha $,$\beta $)$\rrbracket$  = ‘true’  iff  $\langle$$\llbracket$ $\alpha $$\rrbracket$ ,$\llbracket$ $\beta $$\rrbracket$ $\rangle$ ${\in}$ $\llbracket$ P$\rrbracket$ 
\z \z


So far we have been dealing with the meanings of complete sentences all at once. This is possible only for the very simple kinds of sentences discussed thus far, but more importantly, it misses the point of the exercise. If we hope to account for the compositional nature of sentence meaning, modeling speakers’ and hearers’ ability to interpret novel sentences, we need to pay attention to syntactic structure. The sentences in \REF{ex:} share the same basic syntactic structure as those in \REF{ex:}, namely [NP VP]. This suggests that the rule of interpretation stated in \REF{ex:} should apply to the sentences in \REF{ex:} as well.



The main syntactic difference between the sentences in \REF{ex:} and those in \REF{ex:} is the structure of VP: transitive in \REF{ex:}, intransitive in \REF{ex:}. In order to apply rule \REF{ex:} to the sentences in \REF{ex:}, we need another rule which will provide the semantic value of a transitive VP. Intuitively, rule \REF{ex:} says that the proposition expressed by a (declarative) sentence will be true just in case the referent of the subject NP is a member of the denotation set of the VP. So we need to say that sentence (\ref{ex:}a) will be true just in case King Henry VIII belongs to a certain set. What is the relevant set? It would be the set of all individuals that love Anne Boleyn. This set will be the denotation set of the VP \textit{loves Anne Boleyn}. The standard notation for defining such a set is shown in (\ref{ex:}a), which says that the denotation set of this VP will be the set of all individuals x such that the ordered pair $\langle$x, Anne Boleyn$\rangle$ is an element of the denotation set of the transitive verb \textit{love}.


\ea
\ea{}  $\llbracket$ \textit{loves Anne Boleyn}$\rrbracket$  = \{x: $\langle$x, Anne Boleyn$\rangle$ ${\in}$ $\llbracket$ LOVE$\rrbracket$ \}
\ex \textbf{syntax}: VP  →  V\textsubscript{trans}  NP\textsubscript{obj}\\
\textbf{semantics}: The semantic value of a VP containing a transitive verb meaning P together with an object NP meaning $\alpha $ is the set of all individuals x for which P(x,$\alpha $) is true;\\
{}$\llbracket$ VP$\rrbracket$  =  \{x: $\langle$x, $\llbracket$ NP\textsubscript{obj}$\rrbracket$ $\rangle$ ${\in}$ $\llbracket$ V\textsubscript{trans}$\rrbracket$ \}
\z \z


The general rule for deriving denotation sets of transitive VPs is stated in (\ref{ex:}b). The denotation sets formed by this rule are sets of individuals, so it makes sense to ask whether the referent of a subject NP is a member of one of these denotation sets. In other words, the denotation sets formed by rule (\ref{ex:}b) are the right kind of sets to function as VP denotations in rule \REF{ex:}. So this approach allows us to model the step-wise derivation of sentence denotations. The rule of interpretation stated in \REF{ex:} applies to both transitive and intransitive sentences. In the case of transitive sentences, rule (\ref{ex:}b) “feeds”, or provides the input to, rule \REF{ex:}.



Rule \REF{ex:} can also be applied to intransitive sentences with non-verbal predicates like those in \REF{ex:}, provided we can determine the denotation set of the VP.


\ea
\ea \textit{King Henry VIII is happy}.\\
\ex \textit{King Henry VIII is a man}.\\
\ex \textit{King Henry VIII is a happy man}.
                       \z
\z


We can assume that the semantic contribution of the copular verb \textit{is} is essentially nil (apart from tense, which we are ignoring for the moment). That means that the denotation set of the VP \textit{is happy} will be identical to $\llbracket$ HAPPY$\rrbracket$ , which is a set of individuals. For now we will also assume that the semantic contribution of the indefinite article in a predicate NP is nil.\footnote{This assumption applies only to predicate NPs, and not to indefinite NPs in argument positions.} So the denotation set of the VP \textit{is a man} will be identical to $\llbracket$ MAN$\rrbracket$ , which is also a set of individuals. In general, the denotation sets of common nouns and many adjectives are of the same type as the denotation sets of intransitive verbs; this is observable in the denotations assigned in \REF{ex:}. So no extra work is needed to interpret sentences (\ref{ex:}a--b), using rule \REF{ex:}.



Sentence (\ref{ex:}c) is more complex, because the predicate NP contains a modifying adjective as well as the head noun. As with transitive verbs, we can determine the denotation set of the VP (in this case, \textit{is a happy man}) by asking what set the sentence asserts that Henry VIII belongs to? Here the relevant set is the set of happy men, i.e., the set of all individuals who are both happy and men.



The combination of word meanings in \textit{happy man} follows the same pattern we have already discussed in connection with the phrase \textit{yellow submarine}. The proposition asserted in (\ref{ex:}c) might be represented by the formula in (\ref{ex:}a). The truth conditions for this proposition are stated in terms of set membership in (\ref{ex:}b). (Recall the definition of intersection given in \REF{ex:}.) The general rule for interpreting modifying adjectives is stated in (\ref{ex:}c); we use the category label N[2B9?] for the constituent formed by A+N. Ignoring once again any possible semantic contribution of the copula and the indefinite article, the denotation set of the VP \textit{is a happy man} is simply $\llbracket$ HAPPY$\rrbracket$  ${\cap}$ $\llbracket$ MAN$\rrbracket$ . This is a set of individuals, and so rule \REF{ex:} will apply correctly to sentence (\ref{ex:}c) as well.


\ea
\ea  HAPPY(h) $\wedge$ MAN(h)\\
\ex  $\llbracket$ HAPPY(h) $\wedge$ MAN(h)$\rrbracket$  = ‘true’  iff  $\llbracket$ h$\rrbracket$  ${\in}$ ($\llbracket$ HAPPY$\rrbracket$  ${\cap}$ $\llbracket$ MAN$\rrbracket$ )
                       \z
\z

\ea
  c.  \textbf{syntax}: N[2B9?]  →  A N\\
\textbf{semantics}: The semantic value of an N[2B9?] constituent containing a modifying adjective and a head noun is the intersection of the semantic values of the adjective and noun;\\
{}$\llbracket$ A N$\rrbracket$  =  $\llbracket$ A$\rrbracket$  ${\cap}$ $\llbracket$ N$\rrbracket$ 
\z

\section{Conclusion}\label{sec:} %6. /

In this chapter we have worked through a compositional analysis for the meanings of simple sentences like those in \REF{ex:}, \REF{ex:}, and \REF{ex:}. We have developed a rule of semantic interpretation for simple clauses of the form [NP VP] (see rule ), a similar rule for transitive VPs (rule b), and a rule for adjective modifiers (\ref{ex:}c). We have shown how these rules can be applied in a step-wise fashion to derive the truth-conditions of a simple sentence from the denotations of the words that it contains and the manner in which those words are combined syntactically.



In discussing the meanings of quantifiers, conditionals, tense markers etc. in later chapters we will focus more on understanding the phenomena than on formalizing the rule system, but we will still draw heavily on the concepts introduced in this chapter. Moreover, an important assumption in everything that follows is that our description of the meanings of these elements must be compatible with the kind of compositional analysis illustrated in this chapter.



\furtherreading



Good brief introductions to set theory are provided in Allwood, Andersson \& Dahl (1977, ch. 2), Martin (1987, ch. 2), Coppock (2016,ch. 2); and McCawley (1981a, ch. 5). Readable introductory textbooks include \citet{Halmos1960} and \citet{Enderton1977}. Formal introductions to truth-conditional semantics are provided in Dowty, \citet{WallPeters1981} and \citet{HeimKratzer1998}. An informal discussion of this approach is presented in \citet{Bach1989}. A brief introduction to Model Theory is provided by \citet{Hodges2013}. Standard textbooks for this topic include \citet{ChangKeisler1990} and \citet{Hodges1997}.


\subsubsection{Discussion exercises:}\label{sec:}
\paragraph{A. Set theory}

Fill in the following tables:

\ea%1
    \label{ex:key:1}




        

\begin{tabularx}{\textwidth}{XXX}
\lsptoprule

 Set A & Set B\par

 A${\cap}$B & \\
a. the set of all mammals & the set of all animals that lay eggs & the set of all \textsc{monotremes} (Platypus plus four species of echidna)\\
b. \{p,q,s,t\} & \{q,t,w,x\} & \\
c. the set of all odd numbers & the set of all even numbers & \\
d. the set of all Hollywood starspeople with 2-syllable first names

e. the set containing all members of the Beatles & the set of all governors of California (past and present) & \\
&  & \\
\lspbottomrule
\end{tabularx}
 \z
 
\ea%2
    \label{ex:key:2}




        

\begin{tabularx}{\textwidth}{XXX}
\lsptoprule

 Set A & Set B & A${\cup}$B\\
a. the set of all the books of the Old Testament
   
the set of all the books of the New Testament & the set of the canonical books of the Bible & \\
b. \{p,q,s,t\} & \{q,t,w,x\} & \\
c. the set of all odd numbers & the set of all even numbers & \\
d. the set of all members of the \\
    British House of Lords & the set of all members of the British House of Commons & \\
e. the set of all female\\
    British monarchs & the set of all female French monarchs\footnotemark{} & \\
\lspbottomrule
\end{tabularx}
\footnotetext{Note: there were no female French monarchs.}

    \z
\ea%3
    \label{ex:key:3}




        

\begin{tabularx}{\textwidth}{XXX}
\lsptoprule

 Set A & Set B & A–B\\
a. nations that have won at least one FIFA World Cup title & nations that have won a FIFA World Cup title playing in their own homeland
                       

\{Brazil, Spain\}\\
(as of Dec. 2016) & \\
b. \{p,q,s,t\} & \{q,t,w,x\} & \\
c. the set of all integers & the set of all even numbers & \\
d. the set of all cordates & the set of all renates & \\
e. the set of all French monarchs & the set of all female French monarchs & \\
\lspbottomrule
\end{tabularx}
    \z
\ea%4
    \label{ex:key:4}




        

\begin{tabularx}{\textwidth}{XXX}
\lsptoprule

YesSet A & Set B & A${\subseteq}$B?\\
ex \{p,q,s,t\}the set of all mammals

a. the set of all \textsc{monotremes} & \{q,t,w,x\} & \\
b. the set of all odd numbers & the set of all integers & \\
c. the set of all cordates & the set of all renates & \\
d. the set of all Indo-European\\
    languages & the set of all SVO languages & \\
&  & \\
\lspbottomrule
\end{tabularx}
    \z
\paragraph{B. Model theory}

\ea%1
    \label{ex:key:1}




         Sketch a picture of the situation defined by the following model:

    \z
\begin{enumerate}
\item the set of individuals U = \{Able, Baker, Charlie, Doug, Echo, Fred, Geronimo\}
\item denotation assignments\textsc{:\\
{}$\llbracket$ }\textsc{FISH}$\rrbracket$  = \{Able, Baker, Charlie, Doug\}\\
\textsc{$\llbracket$}SUBMARINE$\rrbracket$  = \{Echo\}\\
\textsc{$\llbracket$}SEAHORSE$\rrbracket$  = \{Fred, Geronimo\}\\
\textsc{$\llbracket$}RED$\rrbracket$  = \{Able, Baker, Fred\}\\
\textsc{$\llbracket$}GREEN$\rrbracket$  = \{Charlie, Geronimo\}\\
\textsc{$\llbracket$}BLUE$\rrbracket$  = \{Doug, Echo\}\\
\textsc{$\llbracket$}SWIM$\rrbracket$  = \{Able, Baker, Charlie, Doug, Fred, Geronimo\}\\
\textsc{$\llbracket$}OCTOPUS$\rrbracket$  = ⌀\\
{}$\llbracket$ FOLLOW$\rrbracket$  = \{$\langle$Able, Echo$\rangle$, $\langle$Doug, Able$\rangle$, $\langle$Doug, Echo$\rangle$, $\langle$Charlie, Fred$\rangle$\}\\
\textsc{$\llbracket$}a$\rrbracket$  = Able  \textsc{$\llbracket$}e$\rrbracket$  = Echo\\
\textsc{$\llbracket$}b$\rrbracket$  = Baker  \textsc{$\llbracket$}f$\rrbracket$  = Fred\\
\textsc{$\llbracket$}c$\rrbracket$  = Charlie  \textsc{$\llbracket$}g$\rrbracket$  = Geronimo\\
\textsc{$\llbracket$}d$\rrbracket$  = Doug
\end{enumerate}
\begin{stylepoints}
\ea%2
    \label{ex:key:2}




          Complete the following table by providing logical formulae and set-theoretic interpretations for sentences (e–i), and evaluate the truth value of each sentence relative to the model provided above.
    \z
\end{stylepoints}

\begin{tabularx}{\textwidth}{XXXX}
\lsptoprule

\bfseries\scshape English sentence & \bfseries\scshape logical form & \bfseries\scshape set interpretation & \textbf{\textsc{current truth value}}\\
a. \textit{Geronimo is a seahorse}. & SEAHORSE(g) & Geronimo ${\in}$ \textsc{$\llbracket$}SEAHORSE$\rrbracket$  & T\\
b. \textit{Doug is a blue fish}.

BLUE(d) $\wedge$ \textsc{FISH}(d) & Doug ${\in}$ (\textsc{$\llbracket$}BLUE$\rrbracket$  ${\cap}$ \textsc{$\llbracket$ FISH}$\rrbracket$ ) & T & \\
c. \textit{Charlie is red or green}. & RED(c) $\vee$ GREEN(c) & Charlie ${\in}$ (\textsc{$\llbracket$}RED$\rrbracket$  ${\cup}$ \textsc{$\llbracket$}GREEN$\rrbracket$ ) & T\\
d. \textit{All fish are red}. & ${\forall}$x[\textsc{FISH}(x) → RED(x)] & \textsc{$\llbracket$ FISH}$\rrbracket$  ${\subseteq}$ \textsc{$\llbracket$}RED$\rrbracket$  & F\\
e. \textit{Echo swims}. &  &  & \\
f. \textit{All fish swim}. &  &  & \\
g. \textit{No submarine is red}. &  &  & \\
h. \textit{Two fish are red}.

i. \textit{Some seahorse is green}. &  &  & \\
&  &  & \\
\lspbottomrule
\end{tabularx}
\todo{check alphabetical labels}

\ea%3
    \label{ex:key:3}
    Draw annotated tree diagrams for the following sentences showing how their truth conditions would be derived compositionally from our rules of interpretation:\\
\ea \textit{Henry snores}.\\
\ex \textit{Henry loves Jane}.\\
\ex \textit{Henry is a happy man}.
                       \z
    \z

\subsubsection{Homework exercises:}\label{sec:}
\begin{stylepoints}
\textbf{A:} Assume that the following individuals are included in our universe of discourse:\\
{}$\llbracket$ b$\rrbracket$  = Mrs. Bennet\\
{}$\llbracket$ c$\rrbracket$  = Mr. Collins\\
{}$\llbracket$ d$\rrbracket$  = Mr. Darcy\\
{}$\llbracket$ e$\rrbracket$  = Elizabeth (Bennet)\\
{}$\llbracket$ l$\rrbracket$  = Lydia (Bennet)\\
{}$\llbracket$ w$\rrbracket$  = Mr. Wickham
\end{stylepoints}

\begin{stylepoints}
(i) For each of the following logical formulae, provide an English translation and an interpretation stated in terms of set notation.\\
(ii) Create a model under which sentences (\ref{ex:}b--c) will be false, and the rest (including (\ref{ex:}a)) will be true.\footnote{Patterned loosely after \citet[350]{Saeed2009}.}
\end{stylepoints}

\paragraph{Example:}

\begin{enumerate}
\item LOVE(d,e)
\end{enumerate}

\textsf{[Model answer]:\\
English translation: ‘Mr. Darcy loves/loved Elizabeth.’\\
truth conditions:} $\langle$\textsf{Darcy, Elizabeth}$\rangle$\textsf{} ${\in}$\textsf{} $\llbracket$ \textsf{LOVE}$\rrbracket$ 

\begin{enumerate}
\item REJECT(e,c)
\item ${\forall}$x [(MAN(x) $\wedge$ WEALTHY(x)) → ADMIRE(b,x)]
\item ${\exists}$x [MAN(x) $\wedge$ WEALTHY(x) $\wedge$ ADMIRE(b,x)]
\item ¬${\exists}$x [WOMAN(x) $\wedge$ LOVE(x,c)]
\item DECEIVE(w,l) $\wedge$ RESCUE(d,l)
\item ${\forall}$x [WOMAN(x) → CHARM(w,x)] $\wedge$ ${\forall}$y [MAN(y) → ANGER(w,y)]
\end{enumerate}

\chapter{{14}: Quantifiers}

\section{Introduction}\label{sec:} %1. /

As we noted in \chapref{sec:13}, sentences like those in (\ref{ex:}a-c) seem to require some modifications to the simple rules of interpretation we have developed thus far:


\ea
\ea \textit{All men snore}.\\
\ex \textit{No women snore}.\\
\ex \textit{Some man snores}.
                       \z
\z


Most of the sentences that we discussed in that chapter had proper names for arguments. We analyzed those sentences as asserting that a specific individual (the referent of the subject NP) is a member of a particular set (the denotation set of the VP). The sentences in (\ref{ex:}a-c) present a new challenge because the subject NPs are quantified noun phrases, and do not refer to specific individuals.



Quantifier words like \textit{all}, \textit{some}, and \textit{no} have been intensively studied by semanticists, and the present chapter summarizes some of this research. In \sectref{sec:2} we present evidence for the somewhat surprising claim that quantifier words express a relationship between two sets. This insight, which we will argue follows from the general principle of compositionality, provides the critical foundation for all that follows. In \sectref{sec:3} we show why the standard predicate logic notation that we introduced in \chapref{sec:4} cannot express the meanings of certain kinds of quantifiers. We then introduce a different format, called the \textsc{restricted quantifier} notation, which overcomes this problem. In \sectref{sec:4} we discuss two classes of quantifier words, \textsc{cardinal quantifiers} vs. \textsc{proportional quantifiers}, which differ in both semantic properties and syntactic distribution. \sectref{sec:key:5} discusses an important property of quantifiers which was mentioned briefly in \chapref{sec:4}, namely their potential for ambiguous scope relations with other quantifiers (or various other types of expressions) occurring within the same sentence.


\section{2. Quantifiers as relations between sets}\label{sec:}

Let us begin by asking what claim sentence (\ref{ex:}a) makes about the world. Under what circumstances will it be true? Intuitively, it will be true in any situation in which all of the individuals that are men have the property of snoring; that is, when every member of the denotation set $\llbracket$ MAN$\rrbracket$  is also a member of the denotation set $\llbracket$ SNORE$\rrbracket$ . But this is equivalent to saying that $\llbracket$ MAN$\rrbracket$  is a subset of $\llbracket$ SNORE$\rrbracket$ , as indicated in table \REF{ex:} of \chapref{sec:13}.



Now let us think about how this meaning is composed. We have said that the sentence \textit{All men snore} expresses an assertion that the set of all men is a subset of the set of entities that snore. This interpretation is expressed in the formula in \REF{ex:}. Clearly the semantic contribution of \textit{men} is $\llbracket$ MAN$\rrbracket$ , and the semantic contribution of \textit{snore} is $\llbracket$ SNORE$\rrbracket$ . That means that the semantic contribution of \textit{all} can only be the subset relation itself.


\ea
{}$\llbracket$ \textit{All men snore}$\rrbracket$  = true $\leftrightarrow $ $\llbracket$ MAN$\rrbracket$  ${\subseteq}$ $\llbracket$ SNORE$\rrbracket$ 
\z


Now it may seem odd to suggest that \textit{all} really means ‘subset’, but that is what the principle of compositionality seems to lead us to. The subset relation is a relation between two sets. More abstractly, we can think of the determiner \textit{all} as naming a relation between two sets, in this case the set of all men and the set of all individuals that snore.



Now let us consider sentence (\ref{ex:}b), \textit{No women snore}. Under what circumstances will this sentence be true? Intuitively, it will be true in any situation in which no individual who is a woman has the property of snoring; that is, when no individual is a member both of the denotation set $\llbracket$ WOMAN$\rrbracket$  and of the denotation set $\llbracket$ SNORE$\rrbracket$ . But this is equivalent to saying that the intersection of $\llbracket$ WOMAN$\rrbracket$  with $\llbracket$ SNORE$\rrbracket$  is empty, as indicated in table \REF{ex:} of \chapref{sec:13}. This interpretation is expressed in the formula in \REF{ex:}. By the same reasoning that we used above, the principle of compositionality leads us to the conclusion that the determiner \textit{no} means ‘empty intersection’. Once again, this is a relation between two sets.


\ea
{}$\llbracket$ \textit{No woman snores}$\rrbracket$  = true $\leftrightarrow $ ($\llbracket$ WOMAN$\rrbracket$  ${\cap}$ $\llbracket$ SNORE$\rrbracket$  = ⌀)
\z


Sentence (\ref{ex:}c), \textit{Some man snores}, will be true in any situation in which at least one individual who is a man has the property of snoring. This is equivalent to saying that the intersection of $\llbracket$ MAN$\rrbracket$  with $\llbracket$ SNORE$\rrbracket$  is non-empty, as indicated in \REF{ex:}. The principle of compositionality leads us to the conclusion that the determiner \textit{some} means ‘non-empty intersection’.


\ea
{}$\llbracket$ \textit{Some man snores}$\rrbracket$  = true $\leftrightarrow $ ($\llbracket$ MAN$\rrbracket$  ${\cap}$ $\llbracket$ SNORE$\rrbracket$  ≠ ⌀)
\z


The key insight which has helped semanticists understand the meaning contributions of quantifier words like \textit{all}, \textit{some}, and \textit{no}, is that these words name relations between two sets. The table in \REF{ex:} lists these and several other quantifying determiners, showing their interpretations stated as a relation between two sets. In these examples the two sets are \textsc{$\llbracket$}STUDENT$\rrbracket$  (the set of all students), which for convenience we will refer to as S, and \textsc{$\llbracket$}BRILLIANT$\rrbracket$  (the set of all brilliant individuals) which for convenience we will refer to as B.


\begin{tabularx}{\textwidth}{XXX}
\lsptoprule
& a. \textit{All students are brilliant}. & S ${\subseteq}$ B\\
& b. \textit{No students are brilliant}. & \textsc{S} ${\cap}$ \textsc{B} = ⌀\\
& c. \textit{Some students are brilliant}. & \textsc{{\textbar}S} ${\cap}$ \textsc{B}{\textbar} ${\geq}$ 2\\
& d. \textit{A/Some student is brilliant}. & \textsc{S} ${\cap}$ \textsc{B} ≠ ⌀; or:  \textsc{{\textbar}S} ${\cap}$ \textsc{B}{\textbar} ${\geq}$ 1\\
& e. \textit{Four students are brilliant}. & \textsc{{\textbar}S} ${\cap}$ \textsc{B}{\textbar} = 4\footnotemark{}\\
& f. \textit{Most students are brilliant}. & \textsc{{\textbar}S} ${\cap}$ \textsc{B}{\textbar} > \textsc{{\textbar}S} – \textsc{B}{\textbar}; or: \textsc{{\textbar}S} ${\cap}$ \textsc{B}{\textbar} > ½\textsc{{\textbar}S}{\textbar}\\
& g. \textit{Few students are brilliant}. & \textsc{{\textbar}S} ${\cap}$ \textsc{B}{\textbar} $\langle$ some contextually defined number\\
& h. \textit{Both students are brilliant}. & S ${\subseteq}$ B  \& \textsc{{\textbar}S}{\textbar} = 2\\
\lspbottomrule
\end{tabularx}
\footnotetext{Recall from \chapref{sec:9} that numerals seem to allow two different interpretations. In light of that discussion, this sentence could mean either \textsc{{\textbar}S} ${\cap}$ \textsc{B}{\textbar} = 4 or \textsc{{\textbar}S} ${\cap}$ \textsc{B}{\textbar} ${\geq}$ 4 depending on context. For the purposes of this chapter we will ignore the ‘at least’ reading.}

Notice that we have distinguished plural vs. singular uses of \textit{some} by stating that plural \textit{some} (ex. 5c) indicates an intersection with cardinality of two or more. The interpretation suggested in (h) indicates that the meaning of \textit{both} includes the subset relation and the assertion that the cardinality of the first set equals two. This amounts to saying that \textit{both} means ‘all two of them’. Strictly speaking, it might be more accurate to treat the information about cardinality as a presupposition, because that part of the meaning is preserved in questions (\textit{Are} \textit{both students brilliant?}), conditionals (\textit{If} \textit{both students are brilliant, then …}), etc. However, we will not pursue that issue here.



All of the examples in \REF{ex:} involve relations between two sets. We might refer to quantifiers of this type as two-place quantifiers. Three-place quantifiers are also possible, i.e., quantifiers that express relations among three sets. Some examples are provided in \REF{ex:}.


\ea
\ea  \textit{Half as many} guests attended \textit{as} were invited.\\
\textsc{{\textbar}} \textsc{$\llbracket$}GUEST$\rrbracket$  ${\cap}$ \textsc{$\llbracket$}ATTEND$\rrbracket$  {\textbar}  =  ½\textsc{{\textbar}} \textsc{$\llbracket$}GUEST$\rrbracket$  ${\cap}$ \textsc{$\llbracket$}INVITE$\rrbracket$  {\textbar}
\ex In every Australian election from 1967 to 1998, \textit{more} men \textit{than} women voted for the Labor party.\\
{\textbar} \textsc{$\llbracket$}MAN$\rrbracket$  ${\cap}$ \{x: <x,l> ${\in}$ $\llbracket$ VOTE\_FOR$\rrbracket$ \}{\textbar}  >  {\textbar} \textsc{$\llbracket$ WO}MAN$\rrbracket$  ${\cap}$ \{x: <x,l> ${\in}$ $\llbracket$ VOTE\_FOR$\rrbracket$ \}{\textbar}
\z \z


The kinds of meanings expressed by quantifying determiners can also be expressed by adverbs. \citet{Lewis1975} refers to adverbs like \textit{always}, \textit{sometimes}, \textit{never}, etc. as “unselective quantifiers”, because they can quantify over various kinds of things. The examples in \REF{ex:} show these adverbs quantifying over times: \textit{always} means ‘at all times’, \textit{never} means ‘at no time’, etc. The examples in \REF{ex:} show these same adverbs quantifying over individual entities. If \textit{usually} in (\ref{ex:}b) were interpreted as quantifying over times, it would imply that the color of a dog’s eyes might change from one moment to the next. If \textit{sometimes} in (\ref{ex:}c) were interpreted as quantifying over times, it would imply that the sulfur content of a lump of coal might change from one moment to the next.


\ea
Quantifying over times:\\
\ea In his campaigns Napoleon \textit{always} relied upon surprise and speed.\footnote{\url{http://www.usafa.edu/df/dfh/docs/Harmon28.pdf}} \\
\ex Churchill \textit{usually} took a nap after lunch.\\
\ex De Gaulle \textit{sometimes} scolded his aide-de-camp (= Chief of Staff).\\
\ex George Washington \textit{never} told a lie.
                       \z
\z

\ea
Quantifying over individual entities:\\
\ea A triangle \textit{always} has three sides. (= ‘\textit{All} triangles have three sides.’)\\
\ex Dogs \textit{usually} have brown eyes. (= ‘\textit{Most} dogs have brown eyes.’)\\
\ex Bituminous coal \textit{sometimes} contains more than one percent sulfur by weight.\\
  (= ‘\textit{Some} bituminous coal contains more than one percent sulfur by weight.’)\\
\ex A rectangle \textit{never} has five corners. (= ‘\textit{No} rectangles have five corners.’)
                       \z
\z


In a number of languages, including English, quantifying determiners like \textit{all} can optionally occur in adverbial positions, as illustrated in \REF{ex:}. This alternation is often referred to as \textsc{quantifier float}:


\ea
\ea \textit{All} the children will go to the party.\\
\ex The children will \textit{all} go to the party.
                       \z
\z


Not all languages make use of quantifying determiners; adverbial quantifiers seem to be more common cross-linguistically. Other strategies for expressing quantifier meanings are attested as well: quantificational verb roots, verbal affixes, particles, etc. For some languages it has been claimed that the syntactic means available for expressing quantification limits the range of quantifier meanings which can be expressed.\footnote{\citet{Baker1995}; \citet{Bittner1995}; \citet{KoenigMichelson2010}.} Most of the examples in our discussion below involve English quantifying determiners, and these have been the focus of a vast amount of study. However, we should not forget that other quantification strategies are also common.


\section{3. Quantifiers in logical form}\label{sec:}

Our analysis of \textit{all} as denoting a subset relation, \textit{no} as meaning ‘empty intersection’, and \textit{some} as meaning ‘non-empty intersection’, is reflected in the logical forms we proposed in \chapref{sec:4} for sentences involving these words. These logical forms are repeated here in \REF{ex:}.


\ea
\ea \textit{All men snore}.  ${\forall}$x[MAN(x) → SNORE(x)]\\
\ex \textit{No women snore}.  ${\lnot}$${\exists}$x[WOMAN(x) $\wedge$ SNORE(x)]\\
\ex \textit{Some man snores}.  ${\exists}$x[MAN(x) $\wedge$ SNORE(x)]
                       \z
\z


Now we are in a position to understand why these forms work as translations of the English quantifier words. The use of material implication (→) in (\ref{ex:}a) follows from the definition of the subset relation which we presented in \chapref{sec:13}, repeated here in (\ref{ex:}a). The use of logical $\wedge$ ‘and’ in (\ref{ex:}b-c) follows from the definition of set intersection presented in \chapref{sec:13}, repeated here in (\ref{ex:}b).


\ea
\ea  (A ${\subseteq}$ B)  $\leftrightarrow $  ${\forall}$x[(x${\in}$A) → (x${\in}$B)]  [\textsc{subset}]\\
  ($\llbracket$ MAN $\rrbracket$  ${\subseteq} \llbracket$ SNORE $\rrbracket$ )  $\leftrightarrow $  ${\forall}$x[(x${\in}\llbracket$ MAN$\rrbracket$ ) → (x${\in}\llbracket$ SNORE $\rrbracket$ )]
\ex  ${\forall}$x[x ${\in}$ (A${\cap}$B)  $\leftrightarrow $  ((x${\in}$A) $\wedge$ (x${\in}$B))]  [\textsc{intersection}]\\
  ($\llbracket$ MAN$\rrbracket$ ${\cap}$ $\llbracket$ SNORE$\rrbracket$  ≠ ⌀) $\leftrightarrow $  ${\exists}$x[(x${\in}\llbracket$ MAN $\rrbracket$ ) $\wedge$ (x${\in} \llbracket$  SNORE $\rrbracket$ )]
\z \z


Many other quantifier meanings can also be expressed using the basic predicate logic notation. For example, the NP \textit{four men} could be translated as shown in \REF{ex:}:


\ea
\textit{Four men snore}.\\
${\exists}$w${\exists}$x${\exists}$y${\exists}$z[w${\neq}$x${\neq}$y${\neq}$z $\wedge$ MAN(w) $\wedge$ MAN(x) $\wedge$ MAN(y) $\wedge$ MAN(z) $\wedge$ SNORE(w) $\wedge$ SNORE(x) $\wedge$ SNORE(y) $\wedge$ SNORE(z)]
\z


As we can see even in this simple example, the standard predicate logic notation is a somewhat clumsy tool for this task. Moreover, it turns out that there are some quantifier meanings which cannot be expressed at all using the predicate logic we have introduced thus far. For example, the interpretation for \textit{most} suggested in (\ref{ex:}f) is that the cardinality of the intersection of the two sets is greater than half of the cardinality of the first set. The basic problem here is that the logical predicates we have been using thus far represent properties of individual entities. This type of logic is called \textsc{first-order logic}. However, the cardinality of a set is not a property of any individual, but rather a property of the set as a whole. What we would need in order to express quantifier meanings like \textit{most} is some version of \textsc{second-order logic}, which deals with properties of sets of individuals.



For example, we could define the denotation set of a NP like \textit{most men} to be the set of all properties which are true of most men. The sentence \textit{Most men snore} would be true just in case the property of snoring is a member of $\llbracket$ \textit{most men}$\rrbracket$ .\footnote{This analysis, under which quantified NPs denote sets of sets, is called the Generalized Quantifier approach. The meanings of the quantified NPs themselves are referred to as Generalized Quantifiers, which leads to a certain amount of ambiguity in the use of the word \textit{quantifier}. Sometimes it is used to refer to the whole NP, and sometimes just to the quantifying determiner.} However, the mathematical formalism of this approach is more complex than we can handle in the present book. Rather than trying to work out all the technical details, we will proceed from here on with a more descriptive approach.



One convenient way of expressing propositions which contain quantifier meanings like \textit{most} is called the \textsc{restricted quantifier} notation. This notation consists of three parts: the quantifier operator, the restriction, and the nuclear scope. In example (\ref{ex:}a), the operator is \textit{most}; the restriction is the open proposition “STUDENT(x)”; and the nuclear scope is the open proposition “BRILLIANT(x)”. This same format can be used for other quantifiers as well, as illustrated in (\ref{ex:}b-c).


\ea
\ea \textit{Most students are brilliant}.  [\textit{most} x: STUDENT(x)] BRILLIANT(x)\\
\textsc{(operator} = “\textit{most}”; \textsc{restriction} = “STUDENT(x)”; \textsc{scope} = “BRILLIANT(x)”)
\ex  \textit{No women snore}.  [\textit{no} x: WOMAN(x)] SNORE(x)\\
\ex \textit{All brave men are lonely}.  [\textit{all} x: MAN(x) $\wedge$ BRAVE(x)] LONELY(x)
\z \z


In contrast to the standard logical notation, using this restricted quantifier notation allows us to adopt a uniform procedure for interpreting sentences which contain quantifying determiners:


\begin{enumerate}
\item the quantifying determiner itself specifies the operator;
\item the remainder of the NP which contains the quantifying determiner specifies the material in the restriction;
\item the rest of the sentence specifies the material in the nuclear scope.
\end{enumerate}

For example, the quantifying determiner in (\ref{ex:}c) is \textit{all}; this determines the operator. The remainder of the NP which contains the quantifying determiner is \textit{brave men}; this specifies the material in the restriction (MAN(x) $\wedge$ BRAVE(x)). The rest of the sentence (\textit{are lonely}) specifies the material in the nuclear scope (LONELY(x)). Some additional examples are provided in \REF{ex:}.


\ea
\ea \textit{Most men who snore are libertarians}.\\
  {}[\textit{most} x: MAN(x) $\wedge$ SNORE(x)] LIBERTARIAN(x)\\
\ex \textit{Few strict Baptists drink or smoke}.\\
  {}[\textit{few} x: BAPTIST(x) $\wedge$ STRICT(x)] DRINK(x) $\vee$ SMOKE(x)
                       \z
\z


Of course, translations in this format do not tell us what the quantifying determiners actually mean; the meaning of each quantifier needs to be defined separately, as illustrated in \REF{ex:}:


\ea
\ea{} [\textit{all} x: P(x)] Q(x)  $\leftrightarrow $  \textsc{$\llbracket$}P$\rrbracket$  ${\subseteq}$ \textsc{$\llbracket$}Q$\rrbracket$ \\
\ex{} [\textit{no} x: P(x)] Q(x)  $\leftrightarrow $  \textsc{$\llbracket$}P$\rrbracket$  ${\cap}$ \textsc{$\llbracket$}Q$\rrbracket$  = ⌀\\
\ex{} [\textit{four} x: P(x)] Q(x)  $\leftrightarrow $  {\textbar} \textsc{$\llbracket$}P$\rrbracket$  ${\cap}$ \textsc{$\llbracket$}Q$\rrbracket$  {\textbar}  = 4\\
\ex{} [\textit{most} x: P(x)] Q(x)  $\leftrightarrow $  {\textbar} \textsc{$\llbracket$}P$\rrbracket$  ${\cap}$ \textsc{$\llbracket$}Q$\rrbracket$  {\textbar}  >  ½\textsc{{\textbar}}\textsc{$\llbracket$}P$\rrbracket$ {\textbar}
                       \z
\z


As these definitions show, a quantifying determiner names a relation between two sets: one defined by the predicate(s) in the restriction (represented by P in the formulae in ), and the other defined by the predicate(s) in the scope (represented by Q). Interpretations for the examples in \REF{ex:} are shown in \REF{ex:}. Use these examples to study how the content of the restriction and scope of the logical form in restricted quantifier notation get inserted into the set theoretic interpretation.


\ea
\ea  \textit{Most students are brilliant}.\\
{}[\textit{most} x: STUDENT(x)] BRILLIANT(x)\\
{\textbar} \textsc{$\llbracket$}STUDENT$\rrbracket$  ${\cap}$ \textsc{$\llbracket$}BRILLIANT$\rrbracket$  {\textbar}  >  ½\textsc{{\textbar}}\textsc{$\llbracket$}STUDENT$\rrbracket$ {\textbar}
\ex \textit{No women snore}.\\
{}[\textit{no} x: WOMAN(x)] SNORE(x)\\
\textsc{$\llbracket$}WOMAN$\rrbracket$  ${\cap}$ \textsc{$\llbracket$}SNORE$\rrbracket$  = ⌀
\ex   \textit{All brave men are lonely}.\\
{}[\textit{all} x: MAN(x) $\wedge$ BRAVE(x)] LONELY(x)\\
\textsc{($\llbracket$ }MAN$\rrbracket$  ${\cap}$ \textsc{$\llbracket$}BRAVE$\rrbracket$ )  ${\subseteq}$ \textsc{$\llbracket$}LONELY$\rrbracket$ 
\z \z


This same procedure applies whether the quantified NP is a subject, object, or oblique argument. Some examples of quantified object NPs are given in \REF{ex:}.


\ea
\ea \textit{John loves all pretty girls.}\\
{}[\textit{all} x: GIRL(x) $\wedge$ PRETTY(x)] LOVE(j,x)\\
($\llbracket$ GIRL$\rrbracket$  ${\cap}$ $\llbracket$ PRETTY$\rrbracket$ ) ${\subseteq}$ \{x: <j,x> ${\in}$ $\llbracket$ LOVE$\rrbracket$ \}
\ex \textit{Susan has married a cowboy who teases her.}\\
{}[\textit{an} x: COWBOY(x) $\wedge$ TEASE(x,s)] MARRY(s,x)\\
($\llbracket$ COWBOY$\rrbracket$  ${\cap}$ \{x: <x,s> ${\in}$ $\llbracket$ TEASE$\rrbracket$ ) ${\cap}$ \{y: <s,y> ${\in}$ $\llbracket$ MARRY$\rrbracket$ \} ≠ ⌀
\z \z


At least for the moment, we will provisionally treat the articles \textit{the} and \textit{a(n)} as quantifying determiners. We will discuss the definite article below in \sectref{sec:4}. For now we will treat the indefinite article as an existential quantifier, as illustrated in (\ref{ex:}b). (Note that this applies to indefinite articles occurring in argument NPs, not predicate NPs. We suggested in \chapref{sec:13} that indefinite articles occurring in predicate NPs typically do not contribute any independent meaning.)



Compound words such as \textit{someone}, \textit{everyone}, \textit{no one}, \textit{something}, \textit{nothing}, \textit{anything}, \textit{everywhere}, etc. include a quantifier root plus another root that restricts the quantification to a general class (people, things, places, etc.). It is often helpful to include this “classifier” meaning as a predicate within the restriction of the quantifier, as illustrated in \REF{ex:}.


\ea
\ea \textit{Everyone loves Snoopy}.  [\textit{all} x: PERSON(x)] LOVE(x,s)\\
\ex \textit{Columbus discovered something}.  [\textit{some} x: THING(x)] DISCOVER(c,x)\\
\ex \textit{Nowhere on Earth is safe}.  [\textit{no} x: PLACE(x) $\wedge$ ON(x,e)] SAFE(x)
                       \z
\z

\section{4. Two types of quantifiers}\label{sec:}

Quantifier determiners like \textit{all}, \textit{every}, and \textit{most}, are referred to as \textsc{proportional quantifiers} because they express the idea that a certain proportion of one class is included in some other class. Certain complex determiners like \textit{four out of (every) five} are also proportional quantifiers. Quantifier determiners like \textit{no}, \textit{some}, \textit{four}, and \textit{several}, in contrast, are referred to as \textsc{cardinal quantifiers} because they provide information about the cardinality of the intersection of two sets.\footnote{Proportional quantifiers are sometimes referred to as \textsc{strong} \textsc{quantifiers}, and cardinal quantifiers are sometimes referred to as \textsc{weak} \textsc{quantifiers}.} \textit{Several} is vague; for most speakers it probably indicates a set containing more than two members, but not too much more (less than ten? less than seven?). Nevertheless, it clearly expresses cardinality rather than proportion.



The determiners \textit{many} and \textit{few} are ambiguous between a cardinal sense and a proportional sense. Sentence (\ref{ex:}a) can be interpreted in a way which is not a contradiction, even though the student body at Cal Tech is a tiny fraction of the total population of America. However, this interpretation must involve the proportional senses of \textit{many} and \textit{few}; the cardinal senses would give rise to a contradiction. Sentence (\ref{ex:}b) can only be interpreted as involving the cardinal senses of \textit{many} and \textit{few}, since the sentence does not invoke any specific set of problems or solutions from which a certain proportion could be specified.


\ea
\ea Few people in America have an IQ over 145, but many students at Cal Tech are in\\
  that range.\\
\ex Today we are facing many problems, but we have few solutions.
                       \z
\z


Both the cardinal and proportional senses of \textit{many} and \textit{few} are vague, and this can make it tricky to distinguish the two senses in some contexts. Cardinal \textit{many} probably means more than several, but how much more? Generally speaking, proportional \textit{many} should probably be more than half, and proportional \textit{few} should probably be less than half; but how much more, or how much less? And in certain contexts, even this tendency need not hold. In a country where 80\% of the citizens normally come out to vote, we might say \textit{Few people bothered to vote this year} if the turnout dropped below 60\%. In a city where less than 20\% of the citizens normally bother to vote in local elections, we might say \textit{Many people came to vote this year} if the turnout reached 40\%. So, like other vague expressions, the meanings of \textit{many} and \textit{few} are partly dependent on context.



Relationships expressed by cardinal quantifiers are generally symmetric, as illustrated in the examples in (\ref{ex:}--\ref{ex:}):\footnote{This symmetry follows from the fact that cardinal quantifiers generally have meanings of the form {\textbar}A${\cap}$B{\textbar}=n; and the intersection function is commutative (A ${\cap}$ B = B ${\cap}$ A).}


\ea
\ea No honest men are lawyers.   (a entails b)\\
\ex No lawyers are honest men.
                       \z
\z

\ea
\ea Three senators are Vietnam War veterans.   (a entails b)\\
\ex Three Vietnam War veterans are senators.
\z \z

\ea
\ea Some drug dealers are federal employees.   (a entails b)\\
\ex Some federal employees are drug dealers.
\z \z

\ea
\ea Several Indo-European languages are verb-initial.   (a entails b)\\
\ex Several verb-initial languages are Indo-European.
                       \z
\z


Relationships expressed by proportional quantifiers, in contrast, are not symmetric, as illustrated in the examples in (\ref{ex:}--\ref{ex:}):


\ea
\ea All brave men are lonely.   (a does not entail b)\\
\ex All lonely men are brave.
                       \z
\z

\ea
\ea  Most Popes are Italian.   (a does not entail b)\\
\ex Most Italians are Popes.
\z \z

\ea
\ea Few people are Zoroastrians.   (a does not entail b, in proportional sense of \textit{few})\\
\ex Few Zoroastrians are people.
                       \z
\z


There are several distributional differences which distinguish these two classes of determiners. The best known of these has to do with existential constructions. Only cardinal quantifiers can occur as the “pivot” in the existential \textit{there} construction; proportional quantifiers are ungrammatical in this environment.\footnote{\citet{Milsark1977}.} (It is important to distinguish the existential \textit{there} from several other constructions involving \textit{there}. Sentences like (\ref{ex:}b-c) might be grammatical with the locative \textit{there}, or with the list \textit{there} as in \textit{There’s John, there’s Bill, there’s all our cousins, …}; but these other uses are irrelevant to the present discussion.)


\ea
\ea[]{There are several/some/no/many/six unicorns in the garden.\\}
\ex[*]{There are all/most unicorns in the garden.\\}
\ex[*]{There is every unicorn in the garden.\\}
                       \z
\z


This contrast may be related to the fact that proportional quantifiers seem to presuppose the existence of a contextually relevant and identifiable set.\footnote{\citet{BarwiseCooper1981} suggest that asserting existence is a tautology for most proportional quantifier phrases, vacuously true if the reference set is empty and necessarily true if it is not empty. It is a contradiction for proportional quantifiers like \textit{neither}.} In order for sentence (\ref{ex:}a) to be a sensible statement, a special context is required which specifies the relevant set of people. For example, we might be discussing a town where most people are Baptist. Similarly, if sentence (\ref{ex:}b) is intended to be a sensible statement, a special context is required to specify the relevant set of students. For example, we might be discussing graduation requirements for a particular linguistics program. This “discourse familiarity” of the restriction set is required by proportional quantifiers, but not by cardinal quantifiers. The sentences in \REF{ex:} do not require any specific context in order to be acceptable. (Of course context could be relevant in determining what the vague quantifier \textit{many} means.)


\ea
\ea Most people attend the Baptist church.\\
\ex All students are required to pass phonetics.
                       \z
\z

\ea
\ea Many people attend the Baptist church.\\
\ex Six hundred students got grants from the National Science Foundation this year.\\
\ex No aircraft are allowed to fly over the White House.
                       \z
\z


Discourse familiarity is of course one type of definiteness. We suggested above that the indefinite article \textit{a(n)} could be analyzed as an existential quantifier, roughly synonymous with singular \textit{some}. Under this analysis, \textit{a(n)} would be a cardinal quantifier, because it specifies a non-empty intersection. Similarly, one way of analyzing the definite article \textit{the} is to treat it as a special universal quantifier, meaning something like ‘all of them’ with plural nouns and ‘all one of them’ with singular nouns. Since \textit{all} is a proportional quantifier, this analysis predicts that \textit{the} should also function as a proportional quantifier. \textit{The} seems to trigger a presupposition that the individual or group named by the NP in which it occurs is uniquely identifiable in the context of the utterance. This presupposition might be seen as following from the general requirement of discourse familiarity for the restriction set of a proportional quantifier.\footnote{\citet{Kearn2000}.}



This analysis of the articles gets some support from the observation that \textit{a(n)} can, but \textit{the} cannot, occur with existential \textit{there}. This is exactly what we would expect if \textit{a(n)} is a cardinal quantifier while \textit{the} is a proportional quantifier.


\ea






  There is a/*the unicorn in the garden.  (under existential reading)
\z

\section{Scope ambiguities}\label{sec:} %5. /

As noted in \chapref{sec:4}, when a quantifier combines with another quantifier, negation, or certain other kinds of elements, it can give rise to ambiguities of scope. For example, the sentence \textit{I did not find many valuable books} allows for two readings, as shown in \REF{ex:}. The first reading could be paraphrased as ‘there were many valuable books which I did not find’. The second reading could be paraphrased as ‘there were not many valuable books which I found.’ The difference in the two readings depends on the scope of negation: it takes scope over the quantified NP in reading (\ref{ex:}b), but not in reading (\ref{ex:}a).


\ea
\textit{I did not find many valuable books}.\\
\ea  [\textit{many} x: BOOK(x) $\wedge$ VALUABLE(x)] ¬FIND(speaker,x)\\
\ex  ¬[\textit{many} x: BOOK(x) $\wedge$ VALUABLE(x)] FIND(speaker,x)
                       \z
\z


This is a real semantic ambiguity because the two readings have different truth conditions. For example, suppose that a library contains 10,000 books, of which 600 are considered valuable. One day the library catches fire. The next day the librarian goes in to search for the surviving books, and finds 300 which are considered valuable. In this context, 300 books could plausibly be described as “many”, in which case the first reading would be true while the second reading would be false.



In \chapref{sec:4} we noted that the proverb \textit{All that glitters is not gold} actually has two possible readings. Once again the ambiguity arises from the interaction between the quantifier and clausal negation: either may occur within the scope of the other, as shown in \REF{ex:}. However, many English speakers are not aware of any ambiguity in this proverb. The mock syllogism in \REF{ex:} has been proposed as an example of fallacious reasoning. In fact, the reasoning is sound under one possible reading of the proverb (the (\ref{ex:}a) reading), but not under the intended reading of the proverb (the (\ref{ex:}b) reading).


\ea
\textit{All that glitters is not gold.}\\
\ea  [\textit{all} x: GLITTER(x)] ¬GOLD(x)\\
\ex  ¬[\textit{all} x: GLITTER(x)] GOLD(x)
                       \z
\z

\ea
All that glitters is not gold.\\
This rock glitters.\\
Therefore, this rock is not gold.\footnote{http://www.fallacyfiles.org/scopefal.html}
\z


Part of the reason that speakers do not feel the proverb to be ambiguous is that only one reading is consistent with what we know about the world. However, it also seems to be the case that the (\ref{ex:}b) reading is generally preferred in sentences of this type. On the other hand, naturally occurring examples of the (\ref{ex:}a) reading can be found as well, such as those listed in \REF{ex:}. (In each case the context makes it clear that the intended reading gives widest scope to the quantifier; so (\ref{ex:}c) for example is intended to mean that no person is perfect.)


\ea
\ea All social features are not working.\\
\ex All external storage devices are not being detected as drives.\\
\ex Every person is not perfect.
                       \z
\z


Example \REF{ex:} illustrates how ambiguity can (and frequently does) arise from the interaction between the two quantifiers: either may occur within the scope of the other. The (\ref{ex:}a) reading says that there are many individual linguists who have read every paper by Chomsky. The (\ref{ex:}b) reading says that for any given paper by Chomsky there are many individual linguists who have read it. It would be possible for the (\ref{ex:}b) reading to be true while the (\ref{ex:}a) reading is false under the same circumstances.


\ea
\textit{Many linguists have read every paper by Chomsky}.\\
\ea  [many x: LINGUIST(x)] ([every y: PAPER(y) $\wedge$ BY(y,c)] READ(x,y))\\
\ex  [every y: PAPER(y) $\wedge$ BY(y,c)] ([many x: LINGUIST(x)] READ(x,y))
                       \z
\z


A similar example is presented in \REF{ex:}. The (\ref{ex:}a) reading says that every student in some contextually-determined set, e.g. all those enrolled in a certain course, knows two languages; but each student could know a different pair of languages. The (\ref{ex:}b) reading says that there is some specific pair of languages, e.g. Urdu and Swahili, which every student in the relevant set knows. (Another example of this type was mentioned in \chapref{sec:4}, ex. 29a.)


\ea
\textit{Every student knows two languages}.\\
\ea  [every x: STUDENT(x)] ([two y: LANGUAGE(y)] KNOW(x, y))\\
\ex  [two y: LANGUAGE(y)] ([every x: STUDENT(x)] KNOW(x, y))
                       \z
\z


Scope ambiguities can also arise when a quantifier combines with a modal auxiliary, as illustrated in (\ref{ex:}--\ref{ex:}). (The symbol ${\lozenge}$ stands for ‘possibly true’ and the symbol ${\square}$ stands for ‘necessarily true’.) As we will see in \chapref{sec:16}, many modals appear to be lexically ambiguous; but that is not the source of the ambiguity in these examples. As with negation, the modal operator can either be interpreted within the scope of the quantifier (the (a) readings), or it can take scope over the quantifier (the (b) readings). Try to paraphrase the two readings for each of these sentences.


\ea
\textit{Every student might fail the course}.\footnote{\citet[48]{Abbott2010}.}\\
\ex ${\forall}$x[STUDENT(x) → ${\lozenge}$ FAIL(x)]\\
\ex ${\lozenge}$ ${\forall}$x[STUDENT(x) → FAIL(x)]
\z

\ea
\textit{Some sanctions must be imposed}.\\
\ex ${\exists}$x[SANCTION(x) $\wedge$ ${\square}$ BE-IMPOSED(x)]\\
\ex ${\square}$ ${\exists}$x[SANCTION(x) $\wedge$ BE-IMPOSED(x)]
\z


We will mention just one more possible source of scope ambiguity, namely the interaction between a quantifier and a propositional attitude verb. Consider the example in \REF{ex:}:


\ea
\textit{John thinks that he has visited every state.}\\
\ea  [\textit{all} x: STATE(x)] (THINK(j, VISIT(j,x)))\\
\ex  THINK(j, [\textit{all} x: STATE(x)] VISIT(j,x))
                       \z
\z


The (a) reading could be true and the (b) reading false if John has no idea how many states there are in the United States; but for each of the 50 states, when you ask him whether he has visited that specific state, he answers “I think so.” The (b) reading could be true and the (a) reading false if John believes that there are only 48 states, and knows that he has visited all of them; he knows that he has not visited Alaska or Hawaii, but doesn’t believe that they are states.



It is possible to analyze many cases of \textit{de dicto-de re} ambiguity (\chapref{sec:12}) as scope ambiguities involving propositional attitude verbs, if we treat the indefinite article as an existential quantifier. An example is presented in \REF{ex:}. The (a) reading says that there is some specific individual who is a cowboy, and Susan wants to marry this individual. This is the \textit{de re} reading. It could be true even if Susan does not realize that her prospective husband is a cowboy. The (b) reading says that whoever Susan marries, she wants him to be a cowboy. This is the \textit{de dicto} reading. It could be true even if Susan does not yet have a specific individual in mind.


\ea
\textit{Susan wants to marry a cowboy.}\\
\ea  ${\exists}$x[COWBOY(x) $\wedge$ WANT(s, MARRY(s,x))]\\
\ex  WANT(s, ${\exists}$x[COWBOY(x) $\wedge$ MARRY(s,x)])
                       \z
\z


Based on this analysis, the \textit{de re} reading is often referred to as the “wide scope” reading, meaning that the existential quantifier takes scope over the propositional attitude verb. The \textit{de dicto} reading is often referred to as the “narrow scope” reading, meaning that the quantifier occurs within the scope of the propositional attitude verb.\footnote{Some scholars argue that \textit{de dicto-de re} ambiguity cannot always be reduced to scope relations; see for example \citet{FodorSag1982}.}


\section{Conclusion}\label{sec:} %6. /

We have argued that the meaning contribution of a quantifier, whether expressed by a determiner, adverb, or some other category, is best understood as a relationship between two sets. We introduced a new format for logical formulae involving quantification, the restricted quantifier notation, which is flexible enough to handle all sorts of quantifiers. This notation also makes it possible to state rules of semantic interpretation which treat quantifiers in a more uniform way, although we did not spell out the technical details of how we might do this. A very important step in the interpretation of a quantifier is determining its scope, and we discussed several contexts in which scope interactions can create ambiguous sentences.



These concepts will be important in later chapters, especially in \chapref{sec:16} where we discuss modality. As discussed in that chapter, a very influential analysis of modality is based on the claim that modal expressions like \textit{may}, \textit{must}, \textit{could}, etc. are really a special type of quantifier.



\furtherreading



Kearns (2000, \chapref{sec:4}) provides a clear and helpful introduction to quantification. A brief overview of this very large topic is provided in Gutierrez-\citet{Rexach2013}, a longer overview in \citet{Szabolcsi2015}. \citet{Lewis1975} is the classic work on quantifying adverbs. Barwise\& \citet{Cooper1981} is one of the foundational works on Generalized Quantifiers, and a detailed discussion is presented in Peters \& Westerståhl (2006).


\subsubsection{Discussion exercises:}\label{sec:}
\paragraph{A. Restricted quantifier notation} 

Express the following sentences in restricted quantifier notation, and provide an interpretation in terms of set relations:

\ea
  a. \textit{Every Roman is patriotic}.\\
\textsf{  model answer: [}\textsf{\textit{every}}\textsf{ x: ROMAN(x)] PATRIOTIC(x)\\} $\llbracket$ \textsf{ROMAN} $\rrbracket$ \textsf{} ${\subseteq}$\textsf{} $\llbracket$ \textsf{PATRIOTIC}$\rrbracket$ 
\z

\ea
  b. \textit{Some wealthy Romans are patriotic}.\\
\ex \textit{Both Romans are patriotic}.\\
\ex \textit{Caesar loves all Romans who obey him.}\\
\ex \textit{Most loyal Romans love Caesar.}
\z

\paragraph{B. Scope Ambiguities}

Use logical notation to express the two readings for the following sentences, and state which reading seems most likely to be intended, if you can tell.

\ea
  a. Some man loves every woman.\\
\ex Many theologians do not understand this doctrine.\\
\ex This doctrine is not understood by many theologians.\\
\ex Two-thirds of the members did not vote for the amendment.\\
\ex You can fool some of the people all of the time.\\
\ex A woman gives birth in the United States every five minutes.\\
\ex He tries to read Plato’s \textit{Republic} every year.\footnote{Marilyn Quayle, on the reading habits of her husband; \textit{Wall Street Journal}, January 20, 1993.}
\z

\subsubsection{Homework exercises:}\label{sec:}
\begin{stylepoints}
\textbf{Exercise A:} Translate the following sentences into predicate logic, using the \textbf{\textsc{standard}} [not restricted] format for the existential and universal quantifiers, ${\exists}$ and ${\forall}$. If any sentence allows two interpretations, provide the logical formulae for both readings.
\end{stylepoints}

\begin{enumerate}
\item Solomon answered every riddle.\\
\textsf{Model answer:} ${\forall}$\textsf{x[RIDDLE(x) → ANSWER(s,x)]}
\item All ambitious politicians visit Paris.
\item Someone betrayed Caesar.
\item All critical systems are not working.
\item No German general supported Stalin.
\item Not every German general supported Hitler.
\item Some people believe every wild rumor.
\item Socrates inspires all sincere scholars who read Plato.
\end{enumerate}

\textbf{Exercise B:} Translate the sentences below into logical formulae, using restricted quantifier notation.\footnote{Ex. B-C are patterned after Kearns (2000: 89–90).}

\textsf{Example:}  Arthur eats everything that Susan cooks.\\
\textsf{[Every x: THING(x) \& COOK(s,x)] EAT(a,x)}

\begin{enumerate}
\item Donald mistrusts most reports from Brussels.\\
{}[hint: treat \textit{from} as a two-place predicate]
\item Few who know him like Arthur.
\item William sold Betsy every arrowhead that he found.
\item Twenty-one movies were directed and produced by Alfred Hitchcock.
\item Most travelers entering or leaving Australia visit Sydney.
\item No one\textsubscript{i} remembers every promise he\textsubscript{i} makes.
\item Some officials who boycotted both meetings were sacked by Reagan.
\item Jane Austen and E. M. Forster wrote six novels each.
\item Rachel met and interviewed several famous musicians.
\item Most children will not play if they are sad.
\end{enumerate}
\begin{stylepoints}
\textbf{Exercise C:} The underlined phrases in the sentences below can be analyzed as quantifiers. State the truth conditions for these sentences in terms of set relations.
\end{stylepoints}

\begin{stylepoints}
\textsf{Example:} \textsf{More than twenty}\textsf{ senators are guilty.   {\textbar}} $\llbracket$ \textsf{SENATOR}$\rrbracket$ \textsf{ ${\cap}$} $\llbracket$ \textsf{GUILTY}$\rrbracket$ \textsf{ {\textbar} > 20}
\end{stylepoints}

\begin{enumerate}
\item Between six and twelve generals are loyal.
\item Both sisters are champions.
\item The twelve apostles were Jewish.
\item Just two of the seven guides are bilingual.
\item Neither candidate is honest.
\item Fewer than five crewmen are sober.
\end{enumerate}

The discontinuous determiners in the next examples express three-place quantifier meanings:

\begin{enumerate}
\item More men than women snore.
\item Exactly as many Americans are lawyers as are prisoners.\footnote{Actually the figures are only approximately equal, but there are clearly too many of both.}
\item Fewer wrestlers than boxers are famous.
\end{enumerate}

\chapter{{15}: Intensional contexts}

\section{Introduction}\label{sec:} %1. /

In \chapref{sec:12} we discussed the apparent failure of compositionality in the complement clauses of propositional attitude verbs (\textit{believe, expect, want,} etc.). This apparent failure is observable in several ways. First, the principle of substitutivity does not seem to hold in these complement clauses: replacing one NP with another that has the same referent can change the truth value of the proposition expressed by the sentence as a whole. For example, even if sentences (\ref{ex:}a--b) are assumed to be true, we cannot apply the principle of substitutivity to conclude that (\ref{ex:}c) must be true as well.


\ea
\ea Charles Dickens was the author of \textit{Oliver Twist}.\\
\ex George Cruikshank claimed to be the author of \textit{Oliver Twist}.\footnote{Actually, George Cruikshank only claimed that “the original ideas and characters… emanated from me.” (\url{http://www.bl.uk/collection-items/george-cruikshanks-claims-of-plagiarism-against-charles-Dickens} )}\\
\ex George Cruikshank claimed to be Charles Dickens.
                       \z
\z


Second, the presence within one of these complement clauses of a NP which lacks a denotation does not prevent the proposition expressed by the sentence as a whole from having a truth value. A third special property of these complement clauses is that NPs occurring within them may exhibit the \textit{de re} vs. \textit{de dicto} ambiguity.



These three properties are characteristic of \textsc{opaque} contexts, i.e., contexts in which the denotation of a complex expression cannot be composed or predicted just by looking at the denotations of its constituents; we must look at senses as well. In recent work these contexts are often referred to as \textsc{intensional} contexts, for reasons that will be explained in \sectref{sec:2}.



In this chapter we discuss several types of intensional contexts. \sectref{sec:key:2} reviews our earlier discussion of propositional attitude verbs, and explains the term \textsc{intension}. \sectref{sec:key:3} discusses certain types of adjectives whose composition with the noun they modify cannot be modeled as simple set intersection. These adjectives are often referred to as \textsc{intensional adjectives}. \sectref{sec:key:4} briefly discusses some other intensional contexts involving tense, modality, counterfactuals, and “intensional verbs” such as \textit{want} and \textit{seek}. \sectref{sec:key:5} provides some examples of languages in which the subjunctive mood is used as a grammatical marker of intensionality. \sectref{sec:key:6} briefly discusses the lambda operator, which is used to define functions, and how it can be used to represent intensions as functions.


\section{When substitutivity fails}\label{sec:} %2. /

In \chapref{sec:12} we used the following examples to illustrate the apparent failure of the law of substitutivity in the complement clauses of propositional attitude verbs:


\ea
\ea Mary believes [that \textit{The Prince and the Pauper} was written by Mark Twain].\\
\ex Mary does not believe [that \textit{The Prince and the Pauper} was written by\\
  Samuel Clemens].
                       \z
\z


Normally we can replace one word or phrase in a sentence by another word or phrase that has the same denotation, without affecting the truth value of the sentence as a whole. So, since the names \textit{Mark Twain} and \textit{Samuel Clemens} refer to the same individual, we would expect the two sentences in \REF{ex:} to be contradictory. But this is not the case; it would be possible for both sentences to be true at the same time and for the same person named \textit{Mary}, without any logical inconsistency. Since the denotation of the sentence is its truth value, such examples seem to challenge the Principle of Compositionality, at least as it applies to denotations.



As you will recall, Frege’s solution to this apparent failure of compositionality was to suggest that the denotation of the complement clauses of these verbs is not their truth value when evaluated as independent clauses, but rather the propositions which they express. Essentially, Frege was pointing out that the speaker in \REF{ex:} is not making a claim about the authorship of the book, but about Mary’s current beliefs. The truth value of the sentence as a whole depends not on who the actual author was, but only on what propositions Mary believes.



The denotation of a sentence is its truth value, while the proposition which it expresses is its sense. A technical synonym for ‘sense’ is the term \textsc{intension}. Frege showed that sentences which contain propositional attitude verbs are in fact compositional, but we can only calculate their denotation based on the intension (sense) of the complement clause. Thus these sentences are an example of an \textsc{intensional context}, that is, a context where the denotation of a complex expression depends on the sense (intension) of one or more of its constituents.



Another special property of propositional attitude verbs discussed in \chapref{sec:12} is the potential for \textit{de dicto} vs. \textit{de re} ambiguity, illustrated in \REF{ex:}. The speaker in (\ref{ex:}a), for example, may be expressing either a desire to meet the individual who is the Prime Minister at the moment of speaking (\textit{de re}), or a desire to meet the individual who will be serving in that role at the specified time (\textit{de dicto}).


\ea
\ea I hope to meet with \textit{the Prime Minister} next year.\\
\ex I think that \textit{your husband} is a lucky man.\\
  (\textit{de re}: because I saw him winning at the casino last night.)\\
  (\textit{de dicto}: any man who is married to you would be considered fortunate.)
                       \z
\z


Under the \textit{de re} reading, the noun phrase gets its normal denotation in the relevant context, referring to the specific individual who is the Prime Minister at the moment of speaking (\ref{ex:}a), or who is married to the addressee at the moment of speaking (\ref{ex:}b). Under the \textit{de dicto} reading, the denotation of the noun phrase is the property which corresponds to its sense: the property of being Prime Minister in (\ref{ex:}a), the property of being married to the addressee in (\ref{ex:}b). So under the \textit{de dicto} reading, the truth value of the whole proposition depends on the sense, rather than the denotation, of a particular constituent.



In the next section we look at certain kinds of adjectives which pose a similar challenge to compositionality.


\section{Non-intersective adjectives}\label{sec:} %3. /

Our paradigm example of an adjective modifier has been the word \textit{yellow}. As we have discussed a number of times, the phrase \textit{yellow} \textit{submarine} is compositional in a very straightforward way: its denotation set will be the intersection of the denotation sets $\llbracket$ \textit{yellow}$\rrbracket$  and $\llbracket$ \textit{submarine}$\rrbracket$ . This intersection corresponds to the set of all things in our universe of discourse which are both yellow and submarines.



Adjectives that behave like \textit{yellow} are referred to as \textsc{intersective} adjectives, because they obey the rule of interpretation formulated in \chapref{sec:13}: $\llbracket$ Adj N$\rrbracket$  = $\llbracket$ Adj$\rrbracket$  ${\cap}$ $\llbracket$ N$\rrbracket$ . Some examples of noun phrases involving other intersective adjectives are presented in \REF{ex:}.


\ea
\ea Otacilio is a \textit{Brazilian} poet.\\
\ex Marilyn was a \textit{blonde} actress.\\
\ex Arnold is a \textit{carnivorous} biped.
                       \z
\z


Now the definition of intersection guarantees that if one of the sentences in \REF{ex:} is true, then the individual named by the subject NP must be a member of both the denotation set of the head noun and the denotation set of the adjective modifier. This means that the inference in \REF{ex:} will be valid.


\ea
\textit{Arnold is a carnivorous biped.}\\
\textit{Arnold is a mammal.\\
———\FelixHRule
Therefore, Arnold is a carnivorous mammal.}
\z


However, there are other adjectives for which this pattern of inference will not be valid. Consider for example the syllogism in \REF{ex:}. It would be possible for a rational speaker of English to believe the two premises but not believe the conclusion, without being logically inconsistent. A similar example from \textit{The} \textit{Wizard of Oz} is presented in \REF{ex:}. Such examples force us to conclude that adjectives like \textit{typical} are not intersective.\footnote{This is also true for \textit{bad} in the sense Oz intended in the phrase \textit{bad Wizard}; but \textit{bad} is a tricky word, and the various senses probably do not all belong to the same semantic type. Of course the polysemy is also part of the problem with the invalid inference in \REF{ex:}.}


\ea
\textit{Bill Clinton is a typical politician.}\\
\textit{Bill Clinton is a Baptist.\\
———\FelixHRule
}??\textit{Therefore, Bill Clinton is a typical Baptist.}   [\textbf{\textsc{not valid}}]
\z

\ea
\ea  Dorothy: \textit{Oh — you’re a very bad man!}\\
Wizard: \textit{Oh, no, my dear. I — I’m a very good man. I’m just a very bad Wizard.}\\
{}[\textit{Wizard of Oz} (movie)]
                       \z
\z

\ea
  b.  \textit{Oz is a bad Wizard.}\\
\textit{Oz is a man.\\
———\FelixHRule
??Therefore, Oz is a bad man.}   [\textbf{\textsc{not valid}}]
\z


Barbara \citet{Partee1995} suggested the following illustration: imagine a situation in which all surgeons are also violinists. For example, suppose that a certain hospital wanted to put on a benefit concert, and all the staff members were assigned to play instruments according to their specialties: all the surgeons would play the violin, anesthesiologists the cello, nurses would play woodwinds, administrative staff the brass instruments, etc. Within this universe of discourse, the words \textit{surgeon} and \textit{violinist} have the same denotation sets; in other words, $\llbracket$ \textit{surgeon}$\rrbracket$  = $\llbracket$ \textit{violinist}$\rrbracket$ . However, the phrases \textit{skillful surgeon} and \textit{skillful violinist} do not necessarily have the same denotation sets, as seen by the failure of the following inference:


\ea






  \textit{Francis is a skillful surgeon.}\\
\textit{Francis is a violinist.\\
———\FelixHRule
}??\textit{Therefore, Francis is a skillful violinist.}   [\textbf{\textsc{not valid}}]
\z


This example provides another instance in which two expressions having the same denotation (\textit{surgeon} and \textit{violinist}) are not mutually substitutable, keeping the truth conditions constant. Yet the meanings of phrases like \textit{typical politician} and \textit{skillful surgeon} are still compositional, because if we know what each word means we will be able to predict the meanings of the phrases. The trick is that with adjectives like these, as with propositional attitude verbs, we need to combine senses rather than denotations.



We have seen that the meanings of adjectives like \textit{typical} and \textit{skillful} do not combine with meanings of the nouns they modify as the simple intersection of the two denotation sets. In other words, the rule of interpretation $\llbracket$ Adj~N$\rrbracket$  = $\llbracket$ Adj$\rrbracket$  ${\cap}$ $\llbracket$ N$\rrbracket$  does not hold for these adjectives. However, the following constraint on the denotation of the phrases does hold: $\llbracket$ Adj~N$\rrbracket$  ${\subseteq}$ $\llbracket$ N$\rrbracket$ . In other words, the denotation set of the phrase will be a subset of the denotation set of the head noun. This means that anyone who is a typical politician must be a politician; and anyone who is a skillful surgeon must be a surgeon. Adjectives that satisfy this constraint are referred to as \textsc{subsective} adjectives. (Of course, all intersective adjectives are subsective as well; but since the term “intersective” makes a stronger claim, saying that a certain adjective is subsective will trigger an implicature that it is not intersective, by the maxim of Quantity.)



Subsective adjectives are intensional in the sense defined in \sectref{sec:2}: they combine with the senses, rather than the denotations, of the nouns they modify. One way of representing this is suggested in the following informal definition of \textit{skillful}:


\ea






  \textit{skillful} combines with common nouns (N) to form a phrase which denotes a set of individuals. Any given individual within the universe of discourse will belong to the set of all “skillful Ns” just in case that individual belongs to the set of all Ns and is extremely good at the activity named by N.\\
{}[\textsc{selectional restriction}: \textit{skillful} combines with nouns that name the actor of a volitional activity.]
\z


Certain types of adjectives turn out to be neither intersective nor subsective. Some examples are presented in (Error: Reference source not found).


\ea






  a. \textit{former} Member of Parliament\\
\ex \textit{alleged} terrorist
\z


The adjective \textit{former} is not subsective because a former Member of Parliament is no longer a Member of Parliament; so any person who can be referred to as a “former Member of Parliament” will not belong to the denotation set of \textit{Member of Parliament}. This also proves that \textit{former} is not intersective. Moreover, it is not clear that the adjective \textit{former} even has a denotation set; how could we identify the set of all “former” things? Similarly, an alleged terrorist may or may not actually be a terrorist; we can’t be sure whether or not such a person will belong to the denotation set of \textit{terrorist}. This means that \textit{alleged} is not subsective. And once again, the adjective by itself doesn’t seem to have a denotation set; it would have to be the set of all “alleged” things, whatever that might mean. So \textit{alleged} cannot be intersective either.



How do we calculate the denotation of phrases like those in (Error: Reference source not found)? Although they cannot be defined as a simple intersection, the phrases are still compositional; knowing what each word means allows us to predict the meanings of the phrases. The trick is that with adjectives like these, as with propositional attitude verbs, we need to combine senses rather than denotations. In other words, these adjectives are intensional: they combine with the senses of the nouns they modify. Informal definitions of \textit{former} and \textit{alleged} are suggested in \REF{ex:}:


\ea
\ea  \textit{former} combines with common nouns (N) to form a phrase which denotes a set of individuals. Any given individual within the universe of discourse will belong to the set of all “former Ns” just in case that individual has belonged to the set of all Ns at some time in the past, but no longer does.
\ex  \textit{alleged} combines with common nouns (N) to form a phrase which denotes a set of individuals. Any given individual (x) within the universe of discourse will belong to the set of all “alleged Ns” just in case there is some other individual who claims that x belongs to the set of all Ns.
\z \z


The adjective \textit{former} has the interesting property that a “former N” cannot be a member of the denotation set $\llbracket$ N$\rrbracket$ . In other words, denotation sets of phrases containing the word \textit{former} are subject to the following constraint: $\llbracket$ Adj~N$\rrbracket$  ${\cap}$ $\llbracket$ N$\rrbracket$  = ⌀. Adjectives that satisfy this constraint are referred to as \textsc{privative} adjectives. Other privative adjectives include: \textit{counterfeit, spurious, imaginary, fictitious, fake, would-be, wannabe, past, fabricated} (in one sense). Some prefixes have similar semantics, e.g. \textit{ex-, pseudo-, non-}.



As we have seen, the adjective \textit{alleged} is not subsective; but it is not privative either, because an alleged terrorist may or may not belong to the denotation set of \textit{terrorist}. We can refer to this type of adjectives as \textsc{non-subsective}. Other non-subsective adjectives include: \textit{potential, possible, arguable, likely, predicted, putative, questionable}.



At first glance, many common adjectives like \textit{big}, \textit{old}, etc. seem to be intensional as well. \citet{Partee1995} discusses the invalid inference in \REF{ex:}, which seems to indicate that adjectives like \textit{tall} are non-intersective. The crucial point is that a height which is considered tall for a 14-year-old boy would probably not be considered tall for an adult who plays on a basketball team. This variability in the standard of tallness could lead us to conclude that \textit{tall} does not define a denotation set on its own but combines with the sense of the head noun that it modifies, in much the same way as \textit{typical} and \textit{skillful}.


\ea
\textit{Win is a tall 14-year-old.}\\
\textit{Win is a basketball player.\\
———\FelixHRule
??Therefore, Win is a tall basketball player.}   [\textbf{\textsc{not valid}}]
\z


However, \citet{Siegel1976} argues that words like \textit{tall}, \textit{old}, etc. are in fact intersective; but they are also context-dependent and vague. The boundaries of their denotation sets are determined by context, including (but not limited to) the specific head noun which they modify. Once the boundary is determined, then the denotation set of the adjective can be identified, and the denotation set of the NP can be defined by simple intersection.



One piece of evidence supporting this analysis is the fact that a variety of contextual factors may contribute to determining the boundaries, and not just the meaning of the head noun. Partee notes that the standard of tallness which would apply in (\ref{ex:}a) is probably much shorter than the standard which would apply in (\ref{ex:}b), even though the same head noun is being modified in both examples.


\ea
\ea My two-year-old son built a really tall snowman yesterday.\\
\ex The fraternity brothers built a really tall snowman last weekend.
                       \z
\z


She adds:


“Further evidence that there is a difference between truly non-intersective subsective adjectives like \textit{skillful} and intersective but vague and context-dependent adjectives like \textit{tall} was noted by \citet{Siegel1976}: the former occur with \textit{as}-phrases, as in \textit{skillful as a surgeon}, whereas the latter take \textit{for}-phrases to indicate comparison class: \textit{tall for an East coast mountain}.” (\citealt{Partee2007})


\citet{Bolinger1967} noted that some adjectives are ambiguous between an intersective and a (non-intersective) subsective sense; examples are presented in (\ref{ex:}--\ref{ex:}).\footnote{Adapted from \citealt{Morzycki2013} (\chapref{sec:2}). The adjective \textit{bad} mentioned above is probably also ambiguous in this way.} The fact that the (b) sentences can have a non-contradictory interpretation shows that this is a true lexical ambiguity; contrast \#\textit{Arnold is a carnivorous biped, but he is not carnivorous}.


\ea
\ea  \textit{Marya is a beautiful dancer}.   (\citealt{Siegel1976})\\
intersective: Marya is beautiful and a dancer.\\
subsective: Marya dances beautifully.
\ex  \textit{Marya is not beautiful, but she is a beautiful dancer}.
\z \z

\ea
\ea \textit{Floyd is an old friend}.   [\citealt{Morzycki2013} ms, ch. 2]\\
intersective: Floyd is old and a friend.\\
subsective: Floyd has been a friend for a long time.
\ex  \textit{Floyd is an old friend, but he is not old}.
\z \z

\ea
\ea  \textit{He is a poor liar}.   [cf. \citealt{Bolinger1967}]\\
intersective: Floyd is poor and a liar.\\
subsective: Floyd is not skillful in telling lies.
\ex  \textit{He is a poor liar, but he is not poor}.
\z \z


Thus far we have only considered adjectives which occur as modifiers within a noun phrase; but many adjectives can also function as clausal predicates, as illustrated in \REF{ex:}. In order to be used as a predicate in this way, the adjective must have a denotation set. Since all intersective adjectives must have a denotation set, they can generally (with a few idiosyncratic exceptions) be used as predicates, as seen in \REF{ex:}.


\ea
John is happy/sick/rich/Australian.
\z

\ea
\ea Otacilio is a Brazilian poet; therefore he is Brazilian.\\
\ex Marilyn was a blonde actress; therefore she was blonde.\\
\ex Arnold is a carnivorous biped; therefore he is carnivorous.
                       \z
\z


When an adjective which is ambiguous between an intersective and a subsective sense is used as a predicate, generally speaking only the intersective sense is available \REF{ex:}. So, for example, (\ref{ex:}c) is most naturally interpreted as a pun which makes a somewhat cynical commentary on the way of the world.


\ea
\ea[\#]{Marya is a beautiful dancer; therefore she is beautiful.\\}
\ex[\#]{Floyd is an old friend; therefore he is old.\\}
\ex[]{He is a poor liar; therefore he is poor.}
                       \z
\z


We have already noted that the adjectives \textit{former} and \textit{alleged} don’t seem to have a denotation set. As predicted, these adjectives cannot be used as predicates, and the same is true for many other non-subsective adjectives as well (\ref{ex:}a). However, given the right context, some non-subsective adjectives can be used as predicates (\ref{ex:}b,c). In such cases it appears that information from the context must be used in order to construct the relevant denotation set. In addition, cases like (\ref{ex:}c) may require a kind of coercion to create a new sense of the word \textit{money}, one which refers to things that look like money. As Partee points out, similar issues arise with phrases like \textit{stone lion} and \textit{chocolate bunny}.


\ea
\ea[*]{That terrorist is former/alleged/potential/…\\}
\ex[]{His illness is imaginary.\\}
\ex[]{This money is counterfeit.\\}
                       \z
\z


The main conclusion to be drawn from this brief introduction to the semantics of adjectives is that compositionality cannot always be demonstrated by looking only at denotations. All of the adjectives that we have discussed turned out to be fully compositional in their semantic contributions; but we have seen several classes of adjectives whose semantic contributions cannot be defined in terms of simple set intersection. These adjectives are said to be \textsc{intensional}, because their meanings must combine with the sense (intension) of the head nouns being modified.


\section{Other intensional contexts}\label{sec:} %4. /

As discussed above, intensional contexts are contexts where the denotation of an expression (e.g., the truth value of a sentence) cannot be determined from the denotations of its constituent parts. In addition to those we have already mentioned, namely propositional attitude verbs and non-intersective adjectives, a number of other linguistic features are known to create such contexts as well. These include tense, modality, and counterfactuals. We will discuss these topics in more detail in later chapters; here we focus only on issues of compositionality.



To begin with, let us contrast the intensional behavior of modality (markers of possibility and necessity) with the behavior of a non-intensional operator, negation. Modals are similar to negation in certain ways: both combine with a single proposition to create a new proposition. The crucial difference is this: in order to determine the truth value of a negated proposition, we only need to know the truth value of the original proposition. For example, both of the sentences in \REF{ex:}, if spoken in 2006, would have been false. For that reason, we can be sure that both of the negated sentences in \REF{ex:}, if spoken in 2006, would have been true.


\ea
(spoken in 2006)\\
\ea Barack Obama is the first black President of the United States.  [F]\\
\ex Nelson Mandela is the first black President of the United States.  [F]
                       \z
\z

\ea
(spoken in 2006)\\
\ea Barack Obama is not the first black President of the United States.  [T]\\
\ex Nelson Mandela is not the first black President of the United States.  [T]
                       \z
\z


But with modal operators like \textit{might}, \textit{could}, \textit{must}, etc., it is not enough to know the truth value of the original proposition; we need to evaluate its meaning, in combination with that of the modal operator. Even though both of the sentences in \REF{ex:} had the same truth value in 2006, the addition of the modal in \REF{ex:} creates sentences which would have had different truth values at that time.


\ea
(spoken in 2006)\\
\ea Barack Obama could be the first black President of the United States.  [T]\\
\ex Nelson Mandela could be the first black President of the United States.  [F]
                       \z
\z


Tense is another operator which combines with a single proposition to create a new proposition. As with modality, knowing the truth value of the original proposition does not allow us to determine the truth value of the tensed proposition. Both of the present tense sentences in (\ref{ex:}a--b), spoken in 2014, are false; but the corresponding past tense sentences in (\ref{ex:}c--d) have different truth values.


\ea
(spoken in 2014)\\
\ea Hillary Clinton is the Secretary of State.  [F]\\
\ex Lady Gaga is the Secretary of State.  [F]\\
\ex Hillary Clinton was/has been the Secretary of State.  [T]\\
\ex Lady Gaga was/has been the Secretary of State.  [F]
                       \z
\z


Similarly, knowing that the present tense sentence in (\ref{ex:}a) is true does not allow us to determine the truth value of the corresponding future tense sentence (\ref{ex:}b).


\ea
\ea Henry is Anne’s husband.  [assume T]\\
\ex In five years, Henry will (still) be Anne’s husband.  [?]
                       \z
\z


As we have seen, one of the standard diagnostics for intensional contexts is the failure of substitutivity: in intensional contexts, substituting one expression with another that has the same denotation may affect the truth value of the sentence as a whole. The examples in \REF{ex:} illustrate again the failure of substitutivity in the complement clause of a propositional attitude verb. They refer to an Englishman named James Brooke who, through a combination of military success and diplomacy, made himself the king (or \textit{Rajah}) of Sarawak, comprising most of northwestern Borneo. During the years 1842 to 1868, the phrases \textit{James Brooke} and \textit{the White Rajah of Borneo} referred to the same individual. Suppose that sentence (\ref{ex:}a) was spoken in 1850, perhaps by one of Brooke’s old mates from the Bengal Army. Even if (\ref{ex:}a) was true at the time of speaking, sentence (\ref{ex:}b) spoken at that same time by the same speaker would certainly have been false.


\ea
(spoken in 1850)\\
\ea I do not believe that James Brooke is the White Rajah of Borneo.\\
\ex I do not believe that James Brooke is James Brooke.
                       \z
\z


The examples in \REF{ex:} illustrate the failure of substitutivity in a counterfactual statement. Sentence (\ref{ex:}a) is something that a rational person might believe; at least it is a claim which could be debated. Sentence (\ref{ex:}b) is derived from (\ref{ex:}a) by substituting one NP (\textit{the first black President of the United States}) with another (\textit{Barack Obama}) that has the same denotation. Clearly sentence (\ref{ex:}b) is not something that a rational person could believe.


\ea
\ea Martin Luther King might have become the first black President of the United States.\\
\ex Martin Luther King might have become Barack Obama.
                       \z
\z


The examples in \REF{ex:} also illustrate the failure of substitutivity in a counterfactual; but instead of replacing one NP with another, this time we replace one clause with another. The two consequent clauses are based on propositions which have the same truth value in our world: both would be false if expressed as independent assertions. But replacing one clause with the other changes the truth value of the sentence as a whole: (\ref{ex:}a) is clearly true, while (\ref{ex:}b) is almost certainly false.


\ea
\ea If Beethoven had died in childhood, we would never have heard his\\
  magnificent symphonies.\\
\ex If Beethoven had died in childhood, Columbus would never have\\
  discovered America.
                       \z
\z


Another class of verbs which create intensional contexts are the so-called \textsc{intensional verbs}. Prototypical examples of this type are the verbs of searching and desiring. These verbs license \textit{de dicto} vs. \textit{de re} ambiguities in their direct objects, as illustrated in \REF{ex:}. Sentence (\ref{ex:}a) could mean that the speaker is looking for a specific dog (\textit{de re}), perhaps because it got lost or ran away; or it could mean that the speaker wants to acquire a dog that fits that description but does not have a specific dog in mind (\textit{de dicto}). Sentence (\ref{ex:}b) could mean that John happens to be interested in the same type of work as the addressee (\textit{de re}); or that John wants to be doing whatever the addressee is doing (\textit{de dicto}).


\ea
\ea I’m looking for \textit{a black cocker spaniel}.\\
\ex John wants \textit{the same job as you}.
                       \z
\z


The direct objects of such verbs are referentially opaque, meaning that substitution of a coreferential NP can affect the truth value of a sentence. Suppose that your friend John is bitten by a dog; and suppose that the dog belongs to the notorious gangster Al Capone, but John does not know this. If John goes around looking for the owner in order to demand compensation, then sentence (\ref{ex:}a) would be true, but (\ref{ex:}b) would be false.


\ea
\ea John is looking for \textit{the owner of the dog that bit him}.\\
\ex John is looking for \textit{Al Capone}.
                       \z
\z


Furthermore, if the direct objects of intensional verbs fail to refer in a particular situation, it may still be possible to assign a truth value to the sentence. Both sentences in \REF{ex:} could be true even though in each case the denotation set of the direct object is empty. All of these properties are characteristic of intensional contexts.


\ea
\ea Arthur is looking for \textit{the fountain of youth}.\\
\ex John wants \textit{a unicorn} for Christmas.
                       \z
\z

\section{Subjunctive mood as a marker of intensionality}\label{sec:} %5. /

In some languages, intensional contexts may require special grammatical marking. A number of European languages (among others) use subjunctive mood for this purpose. Let us note from the very beginning that the distribution of the subjunctive is a very complex topic, and that there can be significant differences in this regard even between closely related dialects.\footnote{See for example \citet{Marques2004}.} It is very unlikely that all uses of subjunctive mood in any particular language can be explained on the basis of intensionality alone. But it is clear that intensionality is one of the factors which determine the use of the subjunctive.



Consider the Spanish sentences in \REF{ex:}, which are discussed by \citet{Partee2008}.\footnote{This contrast is also discussed by \citet{Quine1956} and a number of subsequent authors.} Partee states that neither sentence is ambiguous in the way that the English translations are. The relative clause in indicative mood (\ref{ex:}a) can only refer to a specific individual, whereas the relative clause in subjunctive mood (\ref{ex:}b) can only have a non-specific interpretation.


\ea
\ea \gll María  busca  a  un  profesor  que  enseña  griego.\\
Maria  looks.for  to  a  professor  who  teaches-\textsc{indic}  Greek.\\
\glt ‘Maria is looking for a professor who teaches Greek.’  (\textit{de re})
\ex \gll María  busca  (a)  un  profesor  que  enseñe  griego.\\
Maria  looks.for  to  a  professor  who  teaches-\textsc{subjunc}  Greek.\\
\glt ‘Maria is looking for a professor who teaches Greek.’  (\textit{de dicto})
\z \z


A similar pattern is found in relative clauses in modern Greek. The marker for subjunctive mood in modern Greek is the particle \textit{na}. \citet{Giannakidou2011} says that the indicative relative clause in (\ref{ex:}a) can only refer to a specific individual, whereas the subjunctive relative clause in (\ref{ex:}b) can only have a non-specific interpretation.


\ea
\ea \gll Theloume  na  proslavoume  mia  gramatea  [pu  gnorizi  kala  japonezika.]\\
want.1pl  \textsc{subjunc}  hire.1pl  a  secretary  \textsc{rel}  know.3sg  good  Japanese\\
\glt ‘We want to hire a secretary that has good knowledge of Japanese.’  (\textit{de re})\\
(Her name is Jane Smith.)\\
\ex \gll  Theloume  na  proslavoume  mia  gramatea [pu  na  gnorizi  kala  japonezika.]\\
want.1pl  \textsc{subjunc}  hire.1pl  a  secretary \textsc{rel}  \textsc{subjunc}  know.3sg  good  Japanese\\
\glt ‘We want to hire a secretary that has good knowledge of Japanese.’  (\textit{de dicto})\\
(But it is hard to find one, and we are not sure if we will be successful).\\
\z \z


Giannakidou states that because of this restriction, a definite NP cannot contain a subjunctive relative clause \REF{ex:}. Also, the object of a verb of creation with future time reference cannot contain an indicative relative clause, because it refers to something that does not exist at the time of speaking \REF{ex:}.


\ea 
\ea \gll I  Roxani  theli  na  pandrefti  \{enan/*ton\}  andra\\
the  R.  want.3sg  \textsc{subjunc}  marry.3sg  \{a/*the\}  man\\
\ex \gll\relax    [pu  na  exi  pola  lefta].\\
 \textsc{rel}  \textsc{subjunc}  have  much  money\\
\glt ‘Roxanne wants to marry a/*the man who has a lot of money.’\\
\z \z

\ea
\gll Prepi  na  grapso  mia  ergasia  [pu  *(na)\footnotemark  ine  pano  apo  15  selidhes.]\\
must.3sg  \textsc{subjunc}  write.1sg  an  essay  \textsc{rel}  \textsc{subjunc}  is  more  than  15  pages\\
\glt ‘I have to write an essay longer than 15 pages.’\\
\z
\footnotetext{This notation indicates that the subjunctive marker is obligatory; that is, the sentence is ungrammatical without the subjunctive marker.}

The pattern that emerges from these and other examples is that subjunctive mood is used when the noun phrase containing the relative clause refers to a property rather than to a specific individual.


\section{Defining functions via lambda abstraction}\label{sec:} %6. /

In our brief discussion of compositionality in chapters 13–14 we focused primarily on denotations, and expressed the truth conditions of sentences in terms of set membership. So, for example, the denotation of a predicate like \textit{yellow} or \textit{snore}, in a particular context or universe of discourse, is the set of individuals within that context which are yellow, or which snore. The sentence \textit{Henry snores} will be true in any model in which the individual named Henry belongs to the denotation set of \textit{snore}.



We noted in \chapref{sec:13} that the membership of a set can always be expressed as a function, namely its characteristic function. So it is possible to restate the truth conditions of sentences, and to show how these truth conditions are derived compositionally, in terms of functions rather than set membership. The two approaches (sets vs. functions) are essentially equivalent, but for a number of constructions the functional representation provides a simpler, more general, and more convenient way of stating the rules of interpretation.



We will not explore this approach in any detail in the present book, but it will be useful for the reader to be aware of a notation for defining functions that is very widely used in formal semantics. In the standard function-argument format that we learn in secondary school, functions generally have names. For example, the two functions defined in (\ref{ex:}a) are named “f\textsubscript{1}” and “f\textsubscript{2}”. In this kind of definition, the function takes a bound variable (x) as argument and expresses the value as a formula which contains the bound variable. When the function is applied to a real argument, we calculate the value by substituting that argument for the bound variable in the formula. So for example, f\textsubscript{1}(13) = 13 – 4 = 9.


\ea
\ea  named functions:\\
f\textsubscript{1}(x) = x – 4  f\textsubscript{1}(13) = 9\\
f\textsubscript{2}(x) = 3x\textsuperscript{2} + 1  f\textsubscript{2}(2) = 13
\ex  anonymous functions:\\
{}[$\lambda $x. x – 4]  [$\lambda $x. x – 4](13) = 9\\
{}[$\lambda $x. 3x\textsuperscript{2} + 1]  [$\lambda $x. 3x\textsuperscript{2} + 1](2) = 13
\z \z


Another way of defining functions, using the Greek letter lambda ($\lambda $), is illustrated in (\ref{ex:}b). These two functions are identical to f\textsubscript{1} and f\textsubscript{2}, but written in a different format. Once again, when the function is applied to an argument, we calculate the value by substituting that argument for the bound variable which is introduced by the $\lambda $. However, in this format the functions have no names. Functions defined using $\lambda $ are sometimes described as “anonymous functions”.



We can also think of lambda ($\lambda $) as an operator which changes propositions into predicates by replacing some element of the proposition with an appropriate bound variable. For example, from the proposition \textit{Caesar loves Brutus} we can derive “[$\lambda $y.~Caesar~loves~y]” by replacing the object NP with the variable y. This formula represents a predicate which corresponds to the property of being loved by Caesar. Alternatively, we can derive “[$\lambda $x.~x~loves~Brutus]” by replacing the subject NP with the variable x. This formula represents a predicate which corresponds to the property of being someone who loves Brutus.



This process is referred to as \textsc{lambda abstraction}. Once again, when we apply these derived predicates to an argument, as illustrated in (\ref{ex:}a-c), the result is calculated by replacing the bound variable with the argument. (The argument in the first example is b, representing Brutus; in the second example the argument is c, representing Caesar; and in the third example the argument is a, representing Marc Antony.)


\ea
\ex{} [$\lambda $y. LOVE(c,y)](b) = LOVE(c,b) ‘Caesar loves Brutus’\\
ii. [$\lambda $x. LOVE(x,b)](c) = LOVE(c,b) ‘Caesar loves Brutus’\\
iii. [$\lambda $x. LOVE(x,c) $\wedge$ HATE(x,b)](a) = LOVE(a,c) $\wedge$ HATE(a,b)\\
  ‘Antony loves Caesar and hates Brutus’
\z


Predicates derived by lambda abstraction can be interpreted as characteristic functions of the corresponding denotation set, as described in \chapref{sec:13}:


\ea \ea
 [$\lambda $y. LOVE(c,y)](n) =  1 iff Caesar loves n\\
  0 otherwise
\ex   [$\lambda $x. LOVE(x,b)](n) =  1 iff n loves Brutus\\
  0 otherwise
\ex  [$\lambda $x. LOVE(x,c) $\wedge$ HATE(x,b)](n) =  1 iff n loves Caesar and hates Brutus\\
  0 otherwise
\z \z


The semantic value of an intransitive predicate like \textit{snore} can be represented as a function which takes a single argument: [$\lambda $x.~SNORE(x)]. The semantic value of the sentence \textit{Henry snores} can be derived by applying this function to the semantic value of the subject NP, as shown in \REF{ex:}:


\ea
{}[$\lambda $x. SNORE(x)](h)  =  SNORE(h)\\
  =  1 iff Henry snores\\
    0 otherwise
\z


The semantic value of a transitive predicate like \textit{love} can be represented as a function which takes two arguments: [$\lambda $y.~[$\lambda $x.~LOVE(x,y)]]. In calculating the truth conditions for a sentence like \textit{Caesar loves Brutus}, the function named by the verb is applied first to the semantic value of the object NP, as shown in (\ref{ex:}a), to derive the semantic value of the VP. The function named by the VP is then applied to the semantic value of the subject NP, as shown in (\ref{ex:}b), to derive the semantic value of the sentence as a whole.


\ea
\ea  [$\lambda $y. [$\lambda $x. LOVE(x,y)]](b) = [$\lambda $x. LOVE(x,b)] ‘is someone who loves Brutus’
\ex  [$\lambda $x. LOVE(x,b)](c) =  LOVE(c,b) ‘Caesar loves Brutus’
\z \z


In formal semantics, intensions (senses) are often defined as functions from possible worlds to denotations. (Roughly speaking, a “possible world” is any way the universe might conceivably be without changing the structure of the language being investigated.) The intuition behind this analysis is that, as discussed in \chapref{sec:2}, it is knowing the meaning (sense) of a word like \textit{yellow} or \textit{speak} that allows us to identify the set of all yellow things or speaking things in any particular context. So we can think of the senses of these words as a mapping, or function, from each possible world to the expression’s denotation in that world.



Using the lambda abstraction operator, and using w as a variable over the domain of possible worlds, we might represent the intension of \textit{speak} as: “[$\lambda $w.~[$\lambda $x.~SPEAK(x)~in~w]]”; and the intension of \textit{yellow} as: “[$\lambda $w.~[$\lambda $x.~YELLOW(x)~in~w]]”.


\section{Conclusion}\label{sec:} %7. /

In \chapref{sec:13} we worked through some simple examples showing how the truth value of a sentence uttered at a particular time and situation can be calculated based on the denotations of the constituent parts of the sentence at that same time and situation. In this chapter we discussed a variety of linguistic features which make this calculation more complex. For many of these opaque (or intensional) contexts, we can only calculate the truth value of a sentence in a given situation if we know what the denotation of a constituent would be in some other situation.\footnote{Cf. Chierchia \& McConnell-Ginet (1990: 204–208).} For example, statements in the past or future tense, like examples (\ref{ex:}--\ref{ex:}), require knowledge about denotations at some time other than the time of speaking. Statements of possibility (ex. ) and counterfactuals (\ref{ex:}--\ref{ex:}) require judgments about ways that the world might have been, i.e., other possible situations or “possible worlds”. Some of the non-intersective adjectives, such as \textit{former} and \textit{potential}, have similar effects.



As we stated in \chapref{sec:2}, it is knowing the sense of an expression that allows speakers to identify the denotation of that expression in various situations. What all the phenomena discussed in this chapter have in common is that the denotation of some complex expression (e.g., the truth value of a sentence) cannot be compositionally determined from the denotations of its parts alone; we have to refer to senses as well.



\furtherreading



Kearns (2011, \chapref{sec:7}) presents a good overview of referential opacity, and Zimmermann \& Sternefeld (2013, ch. 8) provide a good introduction to the analysis of intensions as functions on possible worlds. Van \citet{Benthem1988} and \citet{Gamut1991b} provide more detailed discussions of intensional logic and its applications. \citet{Partee1995} discusses non-intersective adjectives (among other issues) in relation to compositionality. For an introduction to lambda abstraction, see Coppock (2016: 93 ff.); Kearns (2011: 62–75); Heim \& Kratzer (1998: 34 ff.).


\chapter{{16}: Modality}

\section{Possibility and necessity}\label{sec:} %1. /

Von \citet{Fintel2006} defines \textsc{Modality} as “a category of linguistic meaning having to do with the expression of possibility and necessity.” Most if not all languages have lexical means for expressing these concepts, e.g. \textit{It is possible that…} or \textit{It is necessary} \textit{that…}; but in this chapter we will focus our attention on the kinds of modality which can be expressed grammatically, e.g. by verbal affixation, particles, or auxiliary verbs. In English, modality is expressed primarily by \textsc{modal auxiliaries}: \textit{may, might, must, should, could, ought to,} etc. (The phrase \textit{have to} is often included in discussions of the English modals because it is a close synonym of \textit{must}; but it does not have the unique syntactic distribution of a true auxiliary verb in English, and the syntactic differences sometimes have semantic consequences.)



In \sectref{sec:2} we outline the range of modal meanings along two basic dimensions. The first of these is strength, or degree of certainty (e.g., \textit{must} is said to be “stronger” than \textit{might}). The second dimension is the type of certainty or lack of certainty which is being expressed, e.g. certainty of knowledge, requirement by an authority, etc. We will see that in many languages the same modal forms can be used for two or more different types of modality. We will see some evidence suggesting that such forms are polysemous, but also some reasons for challenging this assumption.



In \sectref{sec:3} we outline a very influential analysis of modal operators as quantifiers, and show how this accounts for some of the puzzling observations discussed in \sectref{sec:2}. In \sectref{sec:4} we discuss some of the variation across languages in terms of how modal meanings are packaged, and show how the quantifier analysis can account for these differences. In \sectref{sec:5} we focus on one important type of modality, referred to as \textsc{epistemic} modality, which expresses degree of certainty in light of what the speaker knows. Some authors have claimed that epistemic modality is not part of the propositional content of the utterance; we review several kinds of evidence that support the opposite conclusion.


\section{The range of modal meanings: strength vs. type of modality}\label{sec:} %2. /

As we noted in \chapref{sec:14}, modality can be thought of as an operator that combines with a basic proposition (\textit{p}) to form a new proposition (\textit{It is possible that p} or \textit{It is necessarily the case that p}). The range of meanings expressible by grammatical markers of modality varies along two basic semantic dimensions.\footnote{\citet{Hacquard2011}.} First, some markers are “stronger” than others. For example, the statement in (\ref{ex:}a) expresses a stronger commitment on the part of the speaker to the truth of the base proposition (\textit{Arthur is home by now}) than (\ref{ex:}b), and (\ref{ex:}b) expresses a stronger commitment than (\ref{ex:}c).


\ea
\ea Arthur \textit{must/has to} be home by now.\\
\ex Arthur \textit{should} be home by now.\\
\ex Arthur \textit{might} be home by now.
                       \z
\z


Second, it turns out that the concepts of “possibility” and “necessity”, which are used to define modality, each include a variety of sub-types. In other words, there are several different ways in which a proposition may be possibly true or necessarily true. The two which have been discussed most extensively, \textsc{epistemic} vs. \textsc{deontic} modality, are illustrated in (\ref{ex:}--\ref{ex:}).


\ea
\ea John didn’t show up for work. He \textit{must} be sick.  [spoken by co-worker; Epistemic]\\
\ex John didn’t show up for work. He \textit{must} be fired.  [spoken by boss; Deontic]
                       \z
\z

\ea
\ea The older students \textit{may} leave school early (unless the teachers watch them carefully).\\
\ex The older students \textit{may} leave school early (if they inform the headmaster first).
                       \z
\z


Epistemic modality defines possibility and necessity on the basis of the speaker’s knowledge of the relevant situation, i.e. whether the proposition is possibly or necessarily true in light of available evidence. Deontic modality defines possibility and necessity on the basis of some authoritative person or code of conduct which is relevant to the current situation, i.e. whether the truth of the proposition is required or permitted by the relevant authority. Examples (\ref{ex:}a) and (\ref{ex:}a) illustrate the epistemic sub-type, under which \textit{He must be sick} means ‘Based on the available evidence, I am forced to conclude that he is sick;’ and \textit{The older students may leave school early} means ‘Based on my knowledge of the current situation, I do not know of anything which would prevent the older students from leaving school early.’ Examples (\ref{ex:}b) and (\ref{ex:}b) illustrate the deontic sub-type, under which \textit{He must be fired} means ‘Someone in authority requires that he be fired;’ and \textit{The older students may leave school early} means ‘The older students have permission from an appropriate authority to leave school early.’



The strength of modality (possibility vs. necessity) is often referred to as the modal “force”, and the type of modality (e.g. epistemic vs. deontic) is often referred to as the modal “flavor”.


\subsection{Are modals polysemous?}\label{sec:} %2.1 /

Examples (\ref{ex:}--\ref{ex:}) also illustrate another important fact about modals: in English, as in many other languages, a single form may be used to express more than one type of modality. As these examples show, both \textit{must} and \textit{may} have two distinct uses, which are often referred to as distinct senses: epistemic vs. deontic. In fact, speakers can create puns which play on these distinct senses. One such example is found in the following passage from “The Schartz-Metterklume Method” (1911), a short story by the British author H. H. Munro (writing under the pen-name “Saki”). In this story, a young Englishwoman, Lady Carlotta, is accidentally left behind on a country railway platform when she gets out to stretch her legs. She is mistaken for a new governess who is due to arrive that day to teach the children of a local family:


\begin{quote}
Before she [Lady Carlotta] had time to think what her next move might be she was confronted by an imposingly attired lady, who seemed to be taking a prolonged mental inventory of her clothes and looks. “\textit{You must be Miss Hope}, the governess I’ve come to meet,” said the apparition, in a tone that admitted of very little argument. “Very well, \textit{if I must I must},” said Lady Carlotta to herself with dangerous meekness.
\end{quote}


“Dangerous meekness” sounds like a contradiction in terms, but in this case it is the literal truth; Lady Carlotta’s novel teaching methods turn the whole household upside down.



As discussed in \chapref{sec:5}, this kind of antagonism between the epistemic vs. deontic senses of \textit{must} strongly suggests that the word is polysemous. Similar arguments could be made for \textit{may}, \textit{should}, etc. This apparent polysemy of the grammatical markers of modality is one of the central issues that a semantic analysis needs to address. But in spite of the strong evidence for distinct senses (lexical ambiguity), there is other evidence which might lead us to question whether these variant readings really involve polysemy or not.



First, as we noted in \chapref{sec:5}, distinct senses of a given word-form are unlikely to have the same translation equivalent in another language. However, this is just what we find with the English modals: the various uses of words like \textit{must} and \textit{may} do have the same translation equivalent in a number of other languages. This fact is especially striking because these words are not restricted to just two readings, epistemic vs. deontic; several other types of modality are commonly identified, which can be expressed using the same modal auxiliaries. Example \REF{ex:} illustrates some of the uses of the modal \textit{have to}; a similar range of uses can be demonstrated for \textit{must}, \textit{may}, etc. (We return to the differences among these specific types in \sectref{sec:3} below. As discussed below, the term \textsc{root} modality is often used as a cover term for the non-epistemic types.)


\ea
{}[adapted from \citealt{vonFintel2006}]\\
\ea It \textit{has to} be raining. [after observing people coming inside with wet umbrellas;\\
  \textsc{epistemic} modality]\\
\ex Visitors \textit{have to} leave by six pm. [hospital regulations; \textsc{deontic}]\\
\ex John \textit{has to} work hard if he wants to retire at age 50. [to attain desires; \textsc{bouletic}]\footnote{Example (\ref{ex:}c) is adapted from \citet{Hacquard2011}. Von \citet{Fintel2006} offers the following definition: “\textsc{Bouletic} modality, sometimes \textsc{boulomaic} modality, concerns what is possible or necessary, given a person’s desires.”}\\
\ex I \textit{have to} sneeze. [given the current state of one’s nose; \textsc{dynamic}]\footnote{Von Fintel uses the term \textsc{circumstantial} modality for what I have called \textsc{dynamic} modality. Huddleston and \citet[178]{Pullum2002} define dynamic modality as being “concerned with properties and dispositions of persons, etc., referred to in the clause, especially by the subject NP.” The most common examples of dynamic modality are expressions of ability with the modal \textit{can}. The term \textsc{circumstantial} modality has a more general usage, as discussed below.}\\
\ex To get home in time, you \textit{have to} take a taxi. [in order to achieve the stated purpose;   \textsc{teleological}]
\z \z


\citet{Hacquard2007} points out that the same range of uses occurs with modal auxiliaries in French as well:


\begin{quote}
It is a robust cross-linguistic generalization that the same modal words are used to express various types of modality. The following French examples illustrate. The modal in (\ref{ex:}a) receives an epistemic interpretation (having to do with what is known, what the available evidence is), while those in (\ref{ex:}b-d) receive a ‘root’ or ‘circumstantial’ interpretation (having to do with particular circumstances of the base world): (\ref{ex:}b) is a case of deontic modality (having to do with permissions/obligations), (\ref{ex:}c) an ability and (\ref{ex:}d) a goal-oriented modality (having to do with possibilities/necessities given a particular goal of the subject).
\end{quote}

\ea
\ea  Il est 18 heures. Anne n’est pas au bureau. Elle \textit{peut/doit} être chez elle.\\
\glt ‘It’s 6:00pm. Anne is not in the office. She \textit{may/must} be at home.’
\ex   Le père de Anne lui impose un régime très strict. Elle \textit{peut/doit} manger du brocoli.\\
\glt ‘Anne’s father imposes on her a strict diet. She \textit{can/must} eat broccoli.’
\ex Anne est très forte. Elle \textit{peut} soulever cette table.\\
\glt ‘Anne is very strong. She \textit{can} lift this table.’
\ex  Anne doit être à Paris à 17 heures. Elle \textit{peut/doit} prendre le train pour aller à P.\\
\glt ‘Anne must be in Paris at 5pm. She \textit{can/must} take the train to go to P.’
\z \z


It is somewhat unusual for the same pattern of polysemy to exist for a particular word in two languages. What we see in the case of modals is something far more surprising: multiple word forms from the same semantic domain, each of which having multiple readings translatable by a single form in not just one but many other languages. Normal polysemy does not work this way.



A second striking fact about the modal auxiliaries in English is that the ranking discussed above in terms of “strength” seems to hold across the various readings or uses of these modals. Linguistic evidence for this ranking comes from examples like those in (\ref{ex:}--\ref{ex:}).\footnote{Examples from von \citet{Fintel2006}.} These examples involve the deontic readings; similar evidence can be given for the epistemic readings, as illustrated in (\ref{ex:}--\ref{ex:}).


\ea
\ea You \textit{should}/\textit{ought} \textit{to} call your mother, but of course you don’t \textit{have to}.\\
\ex \#You \textit{have to} call your mother, but of course you \textit{shouldn’t}.
\z \z

\ea
\ea I \textit{should} go to confession, but I’m not going to.\\
\ex \#I \textit{must} go to confession, but I’m not going to.
                       \z
\z

\ea
\ea Arthur \textit{should} be home by now, but he doesn’t \textit{have to} be.\\
\ex \#Arthur \textit{must/has to} be home by now, but he \textit{shouldn’t} be.\\
  (bad on epistemic reading)\\
\ex Arthur \textit{might} be home by now, but he doesn’t \textit{have to} be.\\
\ex \#Arthur \textit{must/has to} be home by now, but he \textit{might} not be.\\
  (bad on epistemic reading)
                       \z
\z

\ea
\ea \#Arthur \textit{must/has to} be home by now, but I consider it unlikely.\\
  (bad on epistemic reading)\\
\ex \#Arthur \textit{should} be home by now, but I consider it unlikely.\\
  (bad on epistemic reading)\\
\ex Arthur \textit{might} be home by now, but I consider it unlikely.
                       \z
\z


Evidence of this kind would lead us to define the following hierarchies for epistemic and deontic modality. What is striking, of course, is that the two hierarchies are identical. Again, this is not the type of pattern we expect to find with “normal” polysemy.


\ea






 a.  Epistemic modal strength hierarchy:\\
{}[\textsc{necessity}]  \textit{must/have to} > \textit{should/ought to} > \textit{may/might/could}  [\textsc{possibility}]
\z

\ea
  b.  Deontic modal strength hierarchy:\\
{}[\textsc{obligation}]  \textit{must/have to} > \textit{should/ought to} > \textit{may/might/could}  [\textsc{permission}]
\z


The challenge for a semantic analysis is to define the meanings of the modal auxiliaries in a way that can explain these unique and surprising properties. In the next section we will describe a very influential analysis which goes a long way toward achieving this goal.


\section{Modality as quantification over possible worlds}\label{sec:} %3. /

Angelika Kratzer proposed that the English modals are not in fact polysemous.\footnote{\citet{Kratzer1981,Kratzer1991}} On the contrary, she suggested that English (like a number of other languages) has only one set of modal operators, which are underspecified (indeterminate) regarding the type of modality (epistemic, deontic, etc.). The strength of the modal is lexically determined, with the individual modals functioning semantically as a kind of quantifier that quantifies over situations. The specific type of modality depends on the range of situations which is permitted by the context. This section offers a brief and informal introduction to her approach.


\subsection{A simple quantificational analysis}\label{sec:} %3.1 /

Kratzer’s analysis builds on a long tradition of earlier work that treats a modal auxiliary as a kind of quantifier which quantifies over “possible worlds”. (We can think of possible worlds as possible situations or states of affairs; in other words, “ways that things might be”.) A marker of necessity functions as a universal quantifier: it indicates that the basic proposition is true in all possible states of affairs. A marker of possibility functions as an existential quantifier: it indicates that there is at least one state of affairs in which the basic proposition is true.



In \chapref{sec:14} we introduced two symbols from modal logic: ${\lozenge}$ = ‘it is possible that’; and ${\square}$ = ‘it is necessarily the case that’. The use of these symbols is illustrated in the logical forms for two simple modal statements in \REF{ex:}.


\ea
\ea \textit{Arthur must be at home}.  logical form: ${\square}$ AT\_HOME(a)\\
\ex \textit{Arthur may be at home}.  logical form: ${\lozenge}$ AT\_HOME(a)
                       \z
\z


The possible worlds analysis claims that the logical forms in \REF{ex:}, which make use of the modal operators, express the same meaning as those in \REF{ex:}, which are stated in terms of the standard logical quantifiers. The “w” in \REF{ex:} is a variable which stands for a possible world or state of affairs. So under this analysis, \textit{Arthur must be home} means that the proposition \textit{Arthur is home} is true in all possible worlds, while \textit{Arthur might be home} means that the proposition \textit{Arthur is home} is true in at least one possible world.


\ea
\ea \textit{Arthur must be at home}.  meaning: ${\forall}$w[AT\_HOME(a) in w]\\
\ex \textit{Arthur may be at home}.  meaning: ${\exists}$w[AT\_HOME(a) in w]
                       \z
\z


As we noted in \sectref{sec:2}, words like \textit{must} and \textit{may} allow both epistemic and deontic readings (among others). These different types (or “flavors”) of modality can be represented by different restrictions on the quantification, i.e., different limits on the kinds of possible worlds that the quantified variable (\textit{w}) can refer to. Epistemic readings arise when \textit{w} can range over all “epistemically accessible” worlds, i.e., situations which are consistent with what the speaker knows about the actual situation. Deontic readings arise when \textit{w} can range over all “perfect obedience” worlds, i.e., situations in which the requirements of the relevant authority are obeyed. This analysis is illustrated in (\ref{ex:}--\ref{ex:}), using the restricted quantifier notation.


\ea
\textit{Arthur must be at home}.\\
\textbf{a. Epistemic}:\\
{}[all w: w is consistent with what I know about the actual world] AT\_HOME(a) in w\\
\textbf{b. Deontic}:\\
{}[all w: w is consistent with what the relevant authority requires] AT\_HOME(a) in w
\z

\ea
\textit{Arthur may be at home}.\\
\textbf{a. Epistemic}:\\
{}[some w: w is consistent with what I know about the actual world] AT\_HOME(a) in w\\
\textbf{b. Deontic}:\\
{}[some w: w is consistent with what the relevant authority requires] AT\_HOME(a) in w
\z


The unrestricted quantifications in \REF{ex:} express logical possibility or necessity: a claim that proposition p is true in at least one imaginable situation, or in every imaginable situation. Such statements are said to involve \textsc{alethic} modality. As von \citet{Fintel2006} points out, “It is in fact hard to find convincing examples of alethic modality in natural language.” An example of logical (or alethic) possibility might be the statement, “I might never have been born.” It is possible for me to imagine states of affairs in which I would not exist (my father might have been killed in the war, my mother might have chosen to attend a different school, etc.); but none of these states of affairs are epistemically possible, because they are inconsistent with what I know about the real world. Examples of logical (alethic) necessity are probably limited to tautologies, analytically true statements, etc.; it is hard to find any other type of statement which must be true in every imaginable situation.



Analyzing modals as quantifiers accounts for a number of interesting facts. For example, the simple tautologies of modal logic stated in \REF{ex:} show how either of the two modal operators can be defined in terms of the other. (\ref{ex:}a) states that saying \textit{p is possibly true} is equivalent to saying \textit{it is not necessarily the case that p is false}. (\ref{ex:}b) states that saying \textit{p is necessarily true} is equivalent to saying \textit{it is not possible that p is false}. It turns out that the two basic quantifiers of standard logic can be defined in terms of each other in exactly the same way, as shown by the tautologies in \REF{ex:}. This remarkable parallelism is predicted immediately if we analyze necessity in terms of universal quantification and possibility in terms of existential quantification.


\ea
\ea   (${\lozenge}$ p)  $\leftrightarrow $  ¬(${\square}$ ¬p)\\
\ex   (${\square}$ p)  $\leftrightarrow $  ¬(${\lozenge}$ ¬p)
                       \z
\z

\ea
\ea   (${\exists}$x[P(x)])  $\leftrightarrow $  ¬(${\forall}$x[¬P(x)])\\
\ex   (${\forall}$x[P(x)])  $\leftrightarrow $  ¬(${\exists}$x[¬P(x)])
                       \z
\z


We noted in \chapref{sec:14} that combining quantifiers and modals in the same sentence often leads to scope ambiguities. The examples in (\ref{ex:}--\ref{ex:}) are repeated from \chapref{sec:14}. The quantificational analysis again predicts this fact: if modals are really quantifiers, then the ambiguities in (\ref{ex:}--\ref{ex:}) arise as expected from the interaction of two quantifiers.


\ea
\textit{Every student might fail the course}.\footnote{\citet[48]{Abbott2010}.}\\
\ea ${\forall}$x[STUDENT(x) → ${\lozenge}$ FAIL(x)]\\
\ex ${\lozenge}$ ${\forall}$x[STUDENT(x) → FAIL(x)]
                       \z
\z

\ea
\textit{Some sanctions must be imposed}.\\
\ea ${\exists}$x[SANCTION(x) $\wedge$ ${\square}$ BE-IMPOSED(x)]\\
\ex ${\square}$ ${\exists}$x[SANCTION(x) $\wedge$ BE-IMPOSED(x)]
                       \z
\z


While this analysis works well in many respects, Kratzer points out that it makes the wrong predictions in certain cases. For example, suppose that Arthur has robbed a bank, and that robbing banks is against the law. Intuitively, we would say that sentence (\ref{ex:}a) is true in this situation. However, the analysis shown in (\ref{ex:}b) actually predicts the opposite, because in all possible worlds consistent with what the law requires, no one robs banks. In particular, Arthur does not rob a bank (or commit any other crime) in those worlds, and so would not go to prison. Similarly, the analysis predicts that both (\ref{ex:}a) and (\ref{ex:}b) should be true, because the antecedent will be false in all possible worlds consistent with what the law requires. (Recall from \chapref{sec:4} that \textit{p→q} is always considered to be true when p is false.)


\ea
\ea \textit{Arthur must go to prison}.  [Deontic]\\
\ex{} [all w: w is consistent with what the law requires] GO\_TO\_PRISON(a) in w
                       \z
\z

\ea
\ea \textit{If Arthur has robbed a bank, he} \textit{must go to prison}.\\
\ex \textit{If Arthur has robbed a bank, he} \textit{must not go to prison}.
                       \z
\z


To take another example, suppose that when a serious crime is committed, the law allows the government to confiscate the house, car, and other assets of the guilty party to compensate the victim; but that the government is not allowed to confiscate the assets of anyone who does not commit a crime. If Arthur is convicted of a serious crime, the judge may truthfully say the sentence in (\ref{ex:}a). But once again, the analysis in (\ref{ex:}b) predicts that this statement should be false, since there is no possible world consistent with what the law requires in which Arthur commits a crime, so no such world in which his assets may be confiscated.


\ea
\ea \textit{The state may confiscate Arthur’s assets}.  [Deontic]\\
\ex{} [some w: w is consistent with what the law requires]\\
  the state confiscates Arthur’s assets in w
\z
\z


The problem with examples of this type is that we begin with an actual situation that is not consistent with what the law requires. The correct interpretation of the modal reflects the assumption that what happens next, in response to this non-ideal situation, should be as close to the ideal required by law as possible.


\subsection{Kratzer’s analysis}\label{sec:} %3.2 /

Kratzer addresses this problem by arguing that restrictions on the sets of possible worlds available for modal quantifiers must be stated in two components. The first, which she calls the \textsc{modal base}, specifies the class of worlds which are eligible for consideration, i.e., worlds that are \textsc{accessible}. The second component, which she calls the \textsc{ordering source}, specifies a ranking among the accessible worlds. It identifies the “best”, or highest-ranking, world or worlds among those that are accessible. The modal’s domain of quantification contains just these optimal (highest-ranking) accessible worlds.



Let us see how this approach would apply to example (\ref{ex:}a). Deontic modality involves a \textsc{circumstantial} modal base, i.e., one that picks out worlds in which certain relevant circumstances of the actual world hold true. In this case, one of the relevant circumstances of the actual world is the fact that Arthur has robbed a bank. The relevant ordering source in this example is what the law requires: the optimal worlds will be those in which the law is obeyed as completely as possible, given the circumstances. An informal rendering of the interpretation of this sentence is presented in (\ref{ex:}b). The first clause in the restriction represents the modal base, and the second clause in the restriction represents the ordering source.


\ea
\ea \textit{Arthur must go to prison}.  [Deontic]\\
\ex{} [all w: (the relevant circumstances of the actual world are also true in w) and (the \\
  law is obeyed as completely as possible in w)] GO\_TO\_PRISON(a) in w
\z
\z


Epistemic modals require a different kind of modal base and ordering source. The fundamental difference between the two types of modality is summarized by \citet[1494]{Hacquard2011} as follows:


\begin{quote}
Circumstantial [= root; PRK] modality looks at the material conditions which cause or allow an event to happen; epistemic modality looks at the knowledge state of the speaker to see if an event is compatible with various sources of information available.
\end{quote}


The \textsc{epistemic} modal base, which would be relevant for epistemic modals like that in (\ref{ex:}a), picks out worlds consistent with what is known about the actual world, i.e., consistent with the available evidence. Epistemic modals frequently invoke a \textsc{stereotypical} ordering source: the optimal worlds are those in which the normal, expected course of events is followed as closely as possible, given the known facts. An informal rendering of the interpretation of (\ref{ex:}a) is presented in (\ref{ex:}b).


\ea
\ea \textit{Arthur must be at home}.  [Epistemic]  (=a)\\
\ex{} [all w: (w is consistent with the available evidence) and (the normal course of events \\
  is followed as closely as possible)] AT\_HOME(a) in w
\z
\z


This rendering of the meaning of epistemic \textit{must} is more accurate than the analysis suggested in (\ref{ex:}a) for the same example. That earlier analysis would lead us to predict that \textit{Arthur must be at home} entails \textit{Arthur is at home}, since the actual world is one of the worlds that are consistent with what the speaker knows about the actual world. But this prediction is clearly wrong; saying \textit{Arthur is at home} makes a more definite claim than \textit{Arthur must be at home}. By using \textit{must} in this context, the speaker is implying: “I do not have direct knowledge, but based on the evidence I can’t imagine a realistic situation in which Arthur is not at home.” The use of the stereotypical ordering source in (\ref{ex:}b) helps to account for this inferential character of epistemic \textit{must}. It helps us understand why statements of epistemic necessity are usually better paraphrased with the adverb \textit{evidently} than with \textit{necessarily}.\footnote{Kratzer states that another advantage of her theory is that it provides a better way to deal with “graded modality” i.e. intermediate-strength modals of “weak necessity” like \textit{ought} or \textit{should}, as well as phrases such as \textit{very likely} or \textit{barely possible}. We will not discuss graded modality in this chapter.}



Another important part of Kratzer’s proposal is the claim that the modal auxiliaries in languages like English and French are not in fact polysemous. Kratzer suggests that the lexical entry for words like \textit{must} and \textit{may} specifies only the strength of modality (i.e., the choice of quantifier operator), and that they are indeterminate as to the type or “flavor” of modality (epistemic vs. deontic, etc.). The type of modality depends on the choice of modal base and ordering source, which are determined by context (linguistic or general).



Part of the evidence for this claim is the observation that type of modality can be overtly specified by adverbial phrases or other elements in the sentence, as seen in \REF{ex:}.\footnote{From \citet{Hacquard2011}.} Notice that these adverbial phrases do not feel redundant, as they probably would if the modal auxiliary specified a particular type of modality as a lexical entailment. For sentences where there is no explicit indication of type of modality, the intended type will be inferred based on the context of the utterance.


\ea
\ea \textsc{Epistemic:}\\
(In view of the available evidence,) John \textit{must/may} be the murderer.\\
\ex \textsc{Deontic:}\\
(In view of his parents’ orders,) John \textit{may} watch TV, but he \textit{must} go to bed at 8pm.\\
\ex \textsc{Ability:}\\
(In view of his physical abilities,) John \textit{can} lift 200 lbs.\\
\ex \textsc{Teleological:}\\
(In view of his goal to get a PhD,) John \textit{must} write a dissertation.\\
\ex \textsc{Bouletic:}\\
(In view of his desire to retire at age 50,) John \textit{should} work hard now.
                       \z
\z


While Kratzer’s analysis provides an elegant explanation for the unusual pattern of polysemy which we discussed in \sectref{sec:2}, this explanation cannot be applied to all grammatical markers of modality. In the next section we discuss examples of modals for which type of modality seems to be lexically specified.


\section{Cross-linguistic variation}\label{sec:} %4. /

In \sectref{sec:2} we noted that it is common for a single modal form to be used for several different types of modality; but there are also many languages where this does not occur. Even in English, not all modals allow both epistemic and deontic uses. \textit{Might} is used almost exclusively for epistemic possibility, at least in main clauses.\footnote{In indirect speech-type complements, \textit{might} can function as the past tense form of \textit{may}, e.g. \textit{Mary said that I might visit her}. In such contexts the deontic reading is possible. (See \chapref{sec:20} for a discussion of the “sequence of tenses” in indirect speech complements.)} \textit{Can} is used almost exclusively for root modalities, although the negated forms \textit{cannot} and \textit{can’t} do allow epistemic uses. What these examples show is that it is possible, even in English, for both strength and type of modality to be lexically specified.



\citet{Matthewson2010} shows that in St’át’imcets (Lillooet Salish), clitic modality markers are lexically specified for the type of modality, with strength of modality determined by context; see examples in \REF{ex:}. In this regard, St’át’imcets is the mirror image of English.


\ea
\ea  \gll wá7=\textbf{k’a}  s-t’al  l=ti=tsítcw-s=a  s=Philomena\\
be=\textbf{\textsc{epis}}  \textsc{stat}-stop  in=\textsc{det}=house-3sg.\textsc{poss}=\textsc{exis}  \textsc{nom}=Philomena\\
\glt ‘Philomena must/might be in her house.’   [only epistemic]
\ex \gll lán=lhkacw=\textbf{ka}  áts’x-en  ti=kwtámts-sw=a\\
already=2sg.\textsc{subj}=\textbf{\textsc{deon}}  see-\textsc{dir}  \textsc{det}=husband-2sg.\textsc{poss}=\textsc{exis}\\
\glt ‘You must/can/may see your husband now.’   [only deontic]
\z \z


The St’át’imcets data might be analyzed roughly along the lines suggested in \REF{ex:}: the modal markers \textit{=k’a} and \textit{=ka} are both defined in terms of a quantifier which is underspecified for strength, but they lexically specify different types (or flavors) of modality:


\ea
\ea
\textbf{Epistemic} \textit{=}\textbf{\textit{k’a}}:\\
‘Philomena must/might be in her house.’ (a)\\
{}[\textsc{all/some} w: (w is consistent with the available evidence) and (the normal course of events is followed as closely as possible)] AT\_HOME(p) in w
\ex 
 \textbf{Deontic} \textit{=}\textbf{\textit{ka}}:\\
‘You must/can/may see your husband now.’ (b)\\
{}[\textsc{all/some} w: (the relevant circumstances of the actual world are also true in w) and (the requirements of the relevant authority are satisfied as completely as possible in w)] hearer sees husband in w
\z \z


This contrast between St’át’imcets and English provides additional support for the conclusion that either strength or type of modality, or both, may be lexically specified. It is possible for both patterns to be found within a single language. The Malay modal \textit{mesti} ‘must’ has both epistemic and deontic uses, like its English equivalent. The Malay modal \textit{mungkin} ‘probably, possibly’ has only epistemic uses, but the strength of commitment is context-dependent, much like the clitic modality markers in St’át’imcets.



Van der \citet{AuweraAmmann2013} report on a study of modal marking in 207 languages, focusing on the question of whether a single modal form can be used to express both epistemic and deontic modality. They report that this is possible in just under half (102) of the languages in their sample: in 105 of the languages, all of the modal markers are lexically specified as either epistemic or deontic/root, with no ambiguity possible. Only 36 of the languages in the sample are like English and French, with markers of both possibility (\textit{may}) and necessity (\textit{must}) which are ambiguous between epistemic and deontic readings. In the remaining 66 languages there is a modal marker for one degree of strength, either possibility ‘may’ or necessity ‘must’, which is ambiguous between epistemic and deontic readings; but not for the other degree of strength.



The 36 languages which have ambiguous markers for both possibility and necessity are mostly spoken in Europe, and most of them express modality using auxiliary verbs; but neither of these tendencies is absolute. West Greenlandic (Eskimo) is a non-European member of this group which expresses modality with verbal suffixes. The suffix \textit{-ssa} ‘must’ has a deontic/root necessity reading in (\ref{ex:}a) and an epistemic necessity reading in (\ref{ex:}b). The suffix \textit{-sinnaa} ‘can’ has a root possibility reading in (\ref{ex:}a) and an epistemic possibility reading in (\ref{ex:}b).


\textbf{West Greenlandic} (\citealt{Fortescue1984}: 292–94, p.c.; cited in van der \citealt{AuweraAmmann2013}) 

\ea
\ea \gll Inna-jaa-\textit{ssa}-atit.\\
go.to.bed-early-\textsc{nec-indic}.2sg\\
\glt ‘You must go to bed early.’  [\textsc{deontic}]
\ex \gll Københavni-mii-\textit{ssa}-aq.\\
Copenhagen-be.in-\textsc{nec-indic}.3sg\\
\glt ‘She must be in Copenhagen.’  [\textsc{epistemic}]
\z \z

\ea \gll Timmi-\textit{sinnaa}-vuq.\\
fly-can-\textsc{indic}.3sg\\
\glt ‘It can fly.’  [\textsc{root}]
\ex \gll  Nuum-mut  aalla-reer-\textit{sinnaa}-galuar-poq ...\\
Nuuk-\textsc{allative}  leave-already-can-however-3sg.\textsc{indic}\\
\glt ‘He may well have left for Nuuk already, but...’  [\textsc{epistemic}]
\z


Most of the research on modality to this point has focused on languages of the European type. There is no obvious reason why modal markers in other types of language should not also be analyzed as quantifiers over possible worlds, since (as we have seen) lexical entries for modal markers can specify strength, type of modality, or both. However, this is a hypothesis which should probably be held lightly, pending more detailed investigation of the less-studied languages.


\section{On the nature of epistemic modality}\label{sec:} %5. /

As mentioned in our discussion of types of modality in \sectref{sec:1}, the most basic distinction is between epistemic modality and all the other types. \citet[1486]{Hacquard2011} observes that “epistemics deal with possibilities that follow from the speaker’s knowledge, whereas roots deal with possibilities that follow from the circumstances surrounding the main event and its participants.”



Epistemic modality is often said to be “speaker-oriented”,\footnote{Bybee, \citet{PerkinsPagliuca1994}.} because it encodes possibility or necessity in light of the speaker’s knowledge. Non-epistemic modal marking may reflect various facets of the circumstances surrounding the described situation or event. These include the requirements of some authoritative person or code (\textsc{deontic}); and the agent’s ability (\textsc{dynamic}), goals (\textsc{teleological}), or desires (\textsc{bouletic}).\footnote{These examples illustrate the most commonly recognized types of modality; but as von \citet{Fintel2006} observes, “In the descriptive literature on modality, there is taxonomic exuberance far beyond these basic distinctions.”} Van der \citet{AuweraAmmann2013} use the term \textsc{situational} as a cover term for the non-epistemic types, which seems like a very appropriate choice; but the term \textsc{root} is firmly established in linguistic usage.



Epistemic modality also differs from root modality in its interaction with time reference. Epistemic modality in the present time tends to be restricted (at least in English) to states (\ref{ex:}a) and imperfective events, either progressive (\ref{ex:}c) or habitual (\ref{ex:}a). Deontic modality occurs freely with both states and events, but tends to be future oriented; deontic readings are often impossible with past events \biberror{(c, c)}. Epistemic necessity (\textit{must}) is typically impossible with future events (\ref{ex:}b), which is not surprising because speakers generally do not have certain knowledge of the future. Epistemic possibility (\textit{may}), however, is fine with future events (\ref{ex:}b). 


\ea
\ea Henry must be in Brussels this week.  [epistemic or deontic]\\
\ex Henry must write a book this year.  [future; only deontic]\\
\ex Henry must be writing a book this year.  [present; only epistemic]
                       \z
\z

\ea
\ea Mary must attend Prof. Lewis’s lecture every week.  [epistemic or deontic]\\
\ex Mary must attend Prof. Lewis’s lecture tomorrow.  [only deontic]\\
\ex Mary must have attended Prof. Lewis’s lecture yesterday.  [only epistemic]
                       \z
\z

\ea
\ea Mary may attend Prof. Lewis’s lecture every week.  [epistemic or deontic]\\
\ex Mary may attend Prof. Lewis’s lecture tomorrow.  [epistemic or deontic]\\
\ex Mary may have attended Prof. Lewis’s lecture yesterday.  [only epistemic]
                       \z
\z


When the modal itself is inflected for past tense, e.g. \textit{had to} in \REF{ex:}, either reading is possible; but the scope of the tense feature is different in the two readings.\footnote{\textit{Have to} is used here because true modal auxiliaries in English cannot be inflected for tense.}


\ea
Jones had to be in the office when his manager arrived.  [epistemic or deontic]
\z


Under the deontic reading, tense takes scope over the modality: the obligation for the agent to behave in a certain way is part of the situation being described as holding true at some time in the past, prior to the time of speaking. Under the epistemic reading, the modality is outside the scope of the past tense: the speaker’s knowledge now (at the time of speaking) leads him to conclude that a certain situation held true at some time in the past. As von \citet{Fintel2006} points out, the interactions between modality and tense-aspect are complex and poorly understood, and we will not pursue these issues further here.



\citet[1688]{Papafragou2006} describes another kind of difference which has been claimed to exist between epistemic vs. “root” modality:


\begin{quote}
It is often claimed in the linguistics literature that epistemic modality, unlike other kinds of modality, does not contribute to the truth conditions of the utterance. Relatedly, several commentators argue that epistemic modality expresses a comment on the proposition expressed by the rest of the utterance…  The intuition underlying this view is that epistemic modality in natural language marks the degree and/or source of the speaker’s commitment to the embedded proposition.
\end{quote}


However, some of the standard tests for propositional content indicate that this is not the case: both types of modality can be part of the proposition and contribute to its truth conditions. We will mention three tests which provide evidence that epistemic modality does not just express a comment on or attitude toward the proposition, but is actually a part of the proposition itself. First, epistemic modality is part of what can be felicitously challenged, as illustrated in \REF{ex:}.\footnote{Cf. \citet[1698]{Papafragou2006}.}


\ea
A: Jones is the only person who stood to gain from the old man’s death;\\
  he must be the murderer.\\
B: That’s not true; he could be the murderer, but he doesn’t have to be.
\z


In this mini-conversation, speaker B explicitly denies the truth of A’s statement, but only challenges its modality. In other words, B denies \textit{${\square}$p} without denying \textit{p}. In this respect epistemic modals are quite different from the speaker-oriented adverbs which we discussed in \chapref{sec:11}. Those adverbs cannot felicitously be challenged in the same way, because (as we argued) they are not a part of the proposition being asserted.



Second, epistemic modality can be the focus of a yes-no question, as illustrated in (\ref{ex:}--\ref{ex:}). In these questions the information requested concerns the addressee’s degree of certainty, not just the identity of the murderer. The wrong choice of modal can trigger the answer “No”, as in \REF{ex:}, showing that modality contributes to the truth conditions of the sentence. In contrast, when an inappropriate speaker-oriented adverb is added to a yes-no question, it will not cause the answer to change from “Yes” to “No” \REF{ex:}.


\ea
A: Must Jones be the murderer?\\
B: Yes, he must/\#is.  or: No, but I think it is very likely.
\z

\ea
A: Might Jones be the murderer?\\
B: Yes, he might/\#is.  or: No, that is impossible.
\z

\ea
A: Was Jones unfortunately arrested for embezzling?\\
B: Yes/\#No; he was arrested for embezzling, but that is not unfortunate.
\z


Third, epistemic modality can be negated by normal clausal negation, although this point is frequently denied. It is true that some English modals exhibit differences in this regard between their epistemic vs. deontic uses. With \textit{may}, for example, negation takes scope over the modal in the deontic reading, but not in the epistemic reading \REF{ex:}. The modal \textit{must}, on the other hand, takes scope over negation in both of these readings \REF{ex:}.


\ea
Smith may not be the candidate.  [epistemic: possible that not p]\\
  {}[deontic: not permitted that p]
\z

\ea
Smith must not be the candidate.  [epistemic: evident that not p]\\
  {}[deontic: required that not p]
\z


However, while most English modals (including \textit{must} and \textit{may}, as we have just seen) take scope over negation in the epistemic reading, there are a few counter-examples, as illustrated in (\ref{ex:}--\ref{ex:}).\footnote{The same scope holds for the “root” readings of these examples as well.}


\ea
Smith cannot be the candidate.  [epistemic: not possible that p]
\z

\ea
Jones doesn’t have to be the murderer.  [epistemic: not necessary that p]
\z


Examples like these show that even in English, epistemic modality can sometimes be negated by normal clausal negation. Moreover, German \textit{müssen} ‘must’ takes opposite scope from English \textit{must} in both epistemic and deontic readings \REF{ex:}.


\ea
\ea  \gll Er  \textit{muss  nicht}  zu  hause  bleiben.\\
he  must  not  at  home  remain\\
\glt ‘He doesn’t have to stay home.’   [deontic: not required that p; von \citet{Fintel2006}]
\ex \gll Er  \textit{muss  nicht}  zu  Hause  geblieben  sein.  Er  kann  auch  weggegangen  sein.\\
he  must  not  at  home  remained  be  he  can  also  away.gone  be\\
\glt ‘It doesn’t have to be the case that he stayed home (or: He didn’t necessarily stay home). He may also have gone away.’\\
{}[epistemic: not necessary that p; Susi Wurmbrand, p.c.]
\z \z


\citet{Idris1980} states that the Malay modal \textit{mesti} ‘must’ interacts with negation much like its English equivalent, in particular, that negation cannot take scope over the epistemic use of the modal. Now auxiliary scope in Malay correlates closely with word order. When the modal precedes and takes scope over the clausal negator \textit{tidak} ‘not’, as in (\ref{ex:}a), both the epistemic and the deontic readings are possible. When the order is reversed, as in (\ref{ex:}b), Idris states that only the deontic reading is possible.


\ea
\ea \gll Dia  \textit{mesti  tidak}  belajar\\
3sg  must  \textsc{neg}  study\\
\glt ‘He must not study.’ (i.e., ‘I am certain that he does not study.’)\\
        \textbf{[epistemic: evident that not p]}\\
‘He is obliged not to study.’  \textbf{[deontic: required that not p]}
\ex \gll Dia  \textit{tidak  mesti}  belajar\\
3sg  \textsc{neg}  must  study\\
\glt ‘He is not obliged to study.’  \textbf{[deontic: not required that p]}
\z \z


A number of authors have cited these examples in support of the claim that epistemic modality always takes scope over clausal negation.\footnote{See for example de \citet{Haan1997}; Drubig (2001 ms).} However, corpus examples like those in \REF{ex:} show that the epistemic use of \textit{mesti} is in fact possible within the scope of clausal negation.


\ea
\ea \gll Inflasi  \textit{tidak  mesti}  ber-punca  dari  pemerintah…\\
inflation  \textsc{neg}  must  \textsc{mid}-source  from  government\\
\glt ‘Inflation does not have to have the government as its source…’\\
(… it can arise due to other reasons as well)      \textbf{[epistemic: not necessary that p]}\\
{}[\url{http://wargamarhaen.blogspot.com/2011/09/jangan-dok-rasa-pilihanraya-lambat-lagi.html}]
\ex \gll  Hiburan  itu  \textit{tidak  mesti}  mem-bahagia-kan,  tapi\\
entertainment  that  \textsc{neg}  must  bless/make.happy  but\\
\z \z

\ea
  \gll kebahagiaan  itu  sudah  pasti  meng-hibur-kan.\\
happiness  that  already  certain  comfort/entertain\\
\glt ‘Entertainment does not necessarily bring happiness, but happiness will definitely bring comfort.’          \textbf{[epistemic: not necessary that p]}\\
{}[\url{http://skbbs-tfauzi.zoom-a.com/katahikmat.html}]
\z


So we have seen evidence that epistemic modality can be negated by normal clausal negation in Malay, in German, and even in English. Once again, this is not true of evaluative or speech act adverbials: they are never interpreted within the scope of clausal negation, as we demonstrated in \chapref{sec:11}. Taken together, the three types of evidence we have reviewed here provide strong support for the conclusion that epistemic modality is a part of the propositional content of the utterance and contributes to the truth conditions.


\section{Conclusion}\label{sec:} %6. /

In this chapter we have sketched out an analysis which treats modals as quantifiers over possible worlds. This analysis helps to explain why modals are similar to quantifiers in certain ways, for example, in the scope ambiguities that arise when they are combined with other quantifiers.



The analysis also helps to explain the unusually systematic pattern of “polysemy” observed in the English modals, as well as the fact that this same pattern shows up in many other languages as well. This is not how polysemy usually works. Under Kratzer’s analysis, the English modals are not in fact polysemous, but rather indeterminate for type of modality. The strength of the modal (necessary vs. possible) is lexically entailed, but the type of modality (epistemic vs. deontic etc.) is determined by context.



Modals in French and many other languages work in much the same way as the English modals; but this is certainly not the case for all languages, perhaps not even for a majority of them. However, the quantificational analysis can account for these other languages as well. Strength of modality is represented in the quantifier operator, while type of modality is represented in the restriction on the class of possible worlds. Either or both of these can be lexically specified in particular languages, or for specific forms in any language.



Epistemic modality is different in certain ways from all the other types (known collectively as \textsc{root} modality). Some authors have claimed that epistemic modality is not part of the propositional content of the utterance. We argued that this is wrong, based on the fact that epistemic modality can be questioned and challenged, and (at least in some languages) can be negated as well. We return to these issues in the next chapter, where we discuss the difference between markers of epistemic modality vs. markers of \textsc{evidentiality} (source of information).



\furtherreading



Von \citet{Fintel2006} and \citet{Hacquard2011} provide very useful overviews of the semantic analysis of modality, as well as references to much recent work on this subject. Hacquard in particular provides a good introduction to Kratzer’s treatment of modals. \citet{Matthewson2016} presents an introduction and overview with frequent references to Salish and other languages whose modals are quite different from those of English. De \citet{Haan2006} presents a helpful typological study of modality. A brief introduction to modal logic can be found in \citet{Garson2016}; recent textbooks on the subject include Blackburn, de Rijke, \& \citet{Venema2008} and van \citet{Benthem2010}.


\subsubsection{Discussion exercises:}\label{sec:}
\paragraph{A: Deontic vs. epistemic modality}
\ea
Identify the type of modality in the following statements:\\
\ea \textit{You must leave tomorrow}.\\
\ex \textit{You must have offended the Prime Minister very seriously}.\\
\ex \textit{You must be very patient}.\\
\ex \textit{You must use a Mac}.\\
\ex \textit{You must be using a Mac}.
                       \z
\z

\paragraph{B: Ambiguous type of modality}

Use the restricted quantifier notation to express two possible types of modality (deontic vs. epistemic) for the following sentences:

\ea
 \ea \textit{Arnold must trust you}.  (assume “h” = hearer)\\
\textsf{Model answer:\\
}\textsf{\textbf{Epistemic}}\textsf{: [all w: (w is consistent with the available evidence) $\wedge$ (the normal course \\
  of events is followed as closely as possible in w)] TRUST(a,h) in w\\
}\textsf{\textbf{Deontic}}\textsf{: [all w: (the relevant circumstances of the actual world are also true in w) $\wedge$\\
  (the relevant authority’s requirements are satisfied as completely as possible in w)]\\
  TRUST(a,h) in w}
\ex \textit{You may annoy Mr. Roosevelt}.\\
\ex \textit{You must be very patient}.
\z \z

\paragraph{C: Scope ambiguities}

Use the restricted quantifier notation to express the two possible scope relations for the indicated reading of the following sentences:

\ea
\ea \textit{No terrorist must enter the White House}.  [deontic]\\
Model answer:\\
\begin{xlisti} \ex\relax [all w: (the relevant circumstances of the actual world are also true in w) $\wedge$\\
  (the relevant authority’s requirements are satisfied as completely as possible in w)]   ([no x: TERRORIST(x)] ENTER(x,wh) in w)\\
\ex\relax [no x: TERRORIST(x)] ([all w: (the relevant circumstances of the actual world are also \\
  true in w) $\wedge$ (the relevant authority’s requirements are satisfied as completely as \\
  possible in w)] ENTER(x,wh) in w)
\end{xlisti}
\ex Many prisoners must be released.  [deontic]\\
\ex Every candidate could be disqualified.  [epistemic]
\z
\z
\todo{check indentation}

\subsubsection{Homework exercises:}\label{sec:}
\paragraph{A: Epistemic vs. deontic modality}

For each of the sentences below, describe two contexts: one where the modal would most likely have an epistemic reading, the other where the modal would most likely have a deontic reading:

\begin{enumerate}
\item \itshape Arnold must not recognize me.
\item \itshape Henry ought to be in his office by now.
\item \itshape Baxter may support Suharto.
\item \itshape George should be working late tonight.
\item \itshape You have to know how to drive.
\end{enumerate}
\paragraph{B: Restricted quantifier representation}

Use the restricted quantifier notation to express two types of modality (epistemic vs. deontic) for the following sentences. For convenience, you may use the abbreviation “sp” to refer to the speaker and “h” to refer to the hearer.

\ea
\ea
\textit{You must exercise regularly}.\\
\ex \textit{I should be on time this evening}.\\
\ex \textit{Rick may not remain in Casablanca}.
\z
\z

\paragraph{C: Scope ambiguities}
\todo{check labels}
\begin{exe}
 \ex \begin{xlisti}
  \ex Use the restricted quantifier notation to express the deontic reading of the two indicated interpretations for the following sentence: 
  \textit{No professors must be fired}.\\
  \begin{xlisti}
    \ex ¬${\exists}$x[PROFESSOR(x) $\wedge$ ${\square}$ FIRED(x)]\\
    \ex ${\square}$ ¬${\exists}$x[PROFESSOR(x) $\wedge$ FIRED(x)]
  \end{xlisti}
\ex  Use the restricted quantifier notation to express the two possible scope interpretations for the epistemic reading of the following sentences:\\
\begin{xlista}
 \ex \textit{Every student could graduate}.\\
 \ex \textit{Some of the suspects must be guilty}.
\end{xlista}
\end{xlisti}
\end{exe}

\chapter{{17}: Evidentiality}

\section{Markers that indicate the speaker’s source of information}\label{sec:} %1. /

The Tagalog particle \textit{daw} {\textasciitilde} \textit{raw} is used to indicate that the speaker heard the information being communicated from someone else, as illustrated in example \REF{ex:}. ‘Hearsay’ markers like this are one of the most common types of \textsc{evidential} marker among the world’s languages.


\ea
\gll Mabuti  \textit{raw}  ang=ani.\\
good  \textsc{hearsay}  \textsc{nom}=harvest\\
\glt ‘\textit{They say that} the harvest is good.’   [\citealt{SchachterOtanes1972}:423]
\z


The term \textit{evidential} refers to a grammatical marker which indicates the speaker’s source of information. Evidentials have often been treated as a type of epistemic modality, but in this chapter we will argue that the two categories are distinct. We begin in \sectref{sec:2} with a brief survey of some common types of evidential systems found across languages. In \sectref{sec:3} we present a more careful definition of the term \textsc{evidential} and discuss the distinction between evidentiality and epistemic modality. In \sectref{sec:4} we discuss some of the ways in which we can distinguish evidentiality from other categories, such as tense or modality, which may tend to correlate with evidentiality. \sectref{sec:key:5} reviews a proposed distinction between two types of evidential marking. In some languages evidential markers seem to function as illocutionary (speech act) modifiers, while in other languages evidential markers seem to contribute to the propositional content of the utterance. In terms of the distinction we made in \chapref{sec:11}, the former type can be identified as contributing use-conditional meaning, while the latter can be identified as contributing truth-conditional meaning.


\section{Some common types of evidential systems}\label{sec:}  %2. /

As mentioned in the previous section, hearsay markers are one of the most common types of evidential marker cross-linguistically. Another common type of evidential marking is seen in languages like Cherokee, which distinguish \textsc{direct} from \textsc{indirect} knowledge. Evidentiality in Cherokee is signaled by a contrast between two different past tense forms.\footnote{\citet{Pulte1985}; Pulte uses the terms “experienced past” vs. “nonexperienced past”.} Cherokee speakers use the direct form \textit{-[28C?]ʔi} to express what they have experienced personally, e.g. something they have seen, heard, smelled, felt, etc. They use the indirect form \textit{-eʔi} to express what they have heard from someone else; or what they have inferred based on observable evidence (e.g., seeing puddles one might say ‘It rained-\textsc{indirect}’); or what they have assumed based on prior knowledge.



Many languages which have evidential systems make only a two-way distinction, e.g. between direct vs. indirect knowledge, or between hearsay/reported information vs. other sources. However, more complex systems are not uncommon. Huallaga Quechua has three contrastive evidential categories, marked by clitic particles which (in the default pattern) attach to the verb:\footnote{If any single constituent in the sentence gets narrow focus, the evidential clitic follows the focused constituent. If not, the clitic occupies its default position after the verb.} \textit{=mi} marks “direct” knowledge (e.g. eye-witness or personal experience); \textit{=shi} marks hearsay; and \textit{=chi} marks conjecture and/or inference.\footnote{\citet{Weber1989}.} The following sentences provide a minimal contrast illustrating the use of these particles. Each of the sentences contains the same basic propositional content (\textit{You also hit me}); the choice of particle indicates how the speaker came to believe this proposition.


\ea

  \textbf{Huallaga Quechua evidentials} \citep[421]{Weber1989}
\ea
\gll Qam-pis  maqa-ma-shka-nki  \textit{=mi}\\
you-also  hit-1\textsc{obj}-\textsc{perf-2subj  =direct}\\
\glt ‘You also hit me (I saw and/or felt it).’

\ex \gll Qam-pis  maqa-ma-shka-nki  \textit{=shi}\\
you-also  hit-1\textsc{obj}-\textsc{perf-2subj  =hearsay}\\
\glt ‘(Someone told me that) you also hit me (I was drunk and can’t remember).’

 \ex \gll Qam-pis  maqa-ma-shka-nki  \textit{=chi}\\
you-also  hit-1\textsc{obj}-\textsc{perf-2subj  =conject}\\
\glt ‘(I infer that) you also hit me.’\\
(I was attacked by a group of people, and I believe you were one of them).
\z \z


A few languages are reported to have five or even six grammatically distinguished evidential categories. A widely cited example of a five-category system is Tuyuca, a Tucanoan language of Colombia. Evidentiality in Tuyuca is marked by portmanteau suffixes which indicate tense and subject agreement, as well as evidential category; and these suffixes are obligatory in every finite clause in the language.\footnote{\citet{Barnes1984}.} The use of these five evidential categories is illustrated by the minimal contrasts in \REF{ex:}.


\textbf{Tuyuca evidential system} \citep{Barnes1984}

\ea
\ea \gll  díiga  apé  -wi  \\
soccer  play  -\textsc{visual}\\
\glt ‘He played soccer.’ (I saw him play.)
\ex \gll díiga  apé  -ti\\
soccer  play  -\textsc{nonvisual}\\
\glt ‘He played soccer.’ (I heard the game and him, but I didn’t see it or him.)
\ex \gll  díiga  apé  -yi\\
soccer  play  -\textsc{inference}\\
\glt ‘He played soccer.’ (I have seen evidence that he played: his distinctive shoe print on the playing field. But I did not see him play.)
\ex \gll  díiga  apé  -yigi\\
soccer  play  -\textsc{hearsay}\\
\glt ‘He played soccer.’ (I obtained the information from someone else.)
\ex \gll  díiga  apé  -hĩyi\\
soccer  play  -\textsc{assumed}\\
\glt ‘He played soccer.’ (It is reasonable to assume that he did.)
\z \z


The \textsc{visual} category (\ref{ex:}a) is used for states or events which the speaker actually sees, for actions performed by the speaker, and for “timeless” knowledge which is shared by the community. The \textsc{nonvisual} category (\ref{ex:}b) is used for information which the speaker perceived directly by some sense other than seeing; that is, by hearing, smell, touch, or taste. The \textsc{inference} category (\ref{ex:}c), which Barnes labels “apparent”, is used for conclusions which the speaker draws based on direct evidence. The \textsc{hearsay} category (\ref{ex:}d), which Barnes labels “secondhand”, is used for information which the speaker has heard from someone else. The \textsc{assumed} category (\ref{ex:}e) is used for information which the speaker assumes based on background knowledge about the situation.


\section{Evidentiality and epistemic modality}\label{sec:} %3. /

Having examined some examples of the kinds of distinctions that are typically found in evidential systems, let us think about what kind of meaning these grammatical markers express. \citet{Aikhenvald2004}, in her very important book on this topic, defines evidentiality as follows:


\begin{quote}
Evidentiality is a linguistic category whose primary meaning is source of information… [T]his covers the way in which information was acquired, without necessarily relating to the degree of speaker’s certainty concerning the statement or whether it is true or not… To be considered as an evidential, a morpheme has to have ‘source of information’ as its core meaning; that is, the unmarked, or default interpretation.  (\citealt{Aikhenvald2004}:3)
\end{quote}


There are several important points to be noted in this definition. First, evidentiality is a grammatical category.\footnote{cf. \citet[1]{Aikhenvald2004}.} All languages have lexical means for expressing source of information (\textit{I was told that p}; \textit{I infer that p}; \textit{apparently}; \textit{it is said}; etc.), but the term \textsc{evidential} is normally restricted to grammatical morphemes (affixes, particles, etc.). Second, an evidential marker must have source of information as its core meaning. This is significant because evidentiality often correlates with other semantic features, such as degree of certainty. Such a correlation is not surprising, since a speaker will naturally feel more certain of things he has seen with his own eyes than things he learned by hearsay. (We return below to the question of how we can know which factor represents the marker’s “core meaning”.)



It is not unusual for evidential meanings to arise as secondary functions of markers of modality, tense, etc. For example, the German modal verb \textit{sollen} ‘should’ has a secondary usage as a hearsay marker, as illustrated in \REF{ex:}. This form is often cited in discussions of evidentiality; but under Aikhenvald’s strict definition of the term, it would not be classified as an evidential, because its primary function is to mark modality.\footnote{\citet[1]{Aikhenvald2004} estimates that about a quarter of the world’s languages have grammatical markers of evidentiality. In contrast, the WALS on-line database indicates that evidentiality markers are present in 57\% of the sample (237 out of 418 languages). But this is based on a broader definition of the term: WALS includes cases like German \textit{sollen}, where a modal or some other grammatical marker has a secondary evidential function.}


\ea
\gll Kim  \textit{soll}  einen  neuen  Job  angeboten  bekommen  haben.\\
Kim  should a  new  job  offered  get  have\\
\glt ‘Kim has supposedly been offered a new job.’  [\citealt{vonFintel2006}]
\z


A third claim implicit in Aikhenvald’s definition is that evidentiality is distinct from epistemic modality. She states this explicitly a bit later:


\begin{quote}
Evidentials may acquire secondary meanings — of reliability, probability and possibility (known as epistemic extensions), but they do not have to… Evidentiality is a category in its own right, and not a subcategory of any modality… That evidentials may have semantic extensions related to probability and speaker’s evaluation of trustworthiness of information does not make evidentiality a kind of modality. [\citealt{Aikhenvald2004}:7–8]
\end{quote}


Epistemic modality of course is the linguistic category whose primary function is to indicate the speaker’s degree of certainty concerning the proposition that is being expressed. As we have just noted, there is a close correlation between source of information and degree of certainty, and a number of authors have classified evidentiality as a kind of modality.\footnote{\citet{Palmer1986}, \citet{Frawley1992}, \citet{MatthewsonEtAl2007}, \citet{Izvorski1997}.} But Aikhenvald maintains that the two categories need to be distinguished. 



Of course, the question of whether evidentiality is a type of epistemic modality depends in part on how one defines \textit{modality}; but this is not just a terminological issue. We argued in \chapref{sec:16} that modal markers, including epistemic modals, contribute to the propositional content of an utterance. There is good evidence that evidential markers in a number of languages do not contribute to propositional content but function as illocutionary modifiers, and so must be distinct from epistemic modality. But before we review some of this evidence, it will be helpful to think about how we go about identifying a morpheme’s “primary function”.


\section{Distinguishing evidentiality from tense and modality}\label{sec:} %4. /

It is not always easy to distinguish empirically between evidential markers and epistemic modals. Tense and aspect markers can also be a problem, because they too can have secondary evidential functions or associations. Perfect aspect in particular often carries an indirect evidential connotation, and indirect evidence markers frequently develop out of perfect aspect markers.\footnote{\citet{Izvorski1997}; \citet{BybeeEtAl1994}.} For example, in Iranian Azerbaijani (closely related to Turkish) the suffix \textit{-miş} is polysemous between an older perfect sense and a more recent evidential sense.\footnote{\citet{Lee2008}.} We can see that the two senses are distinct in the modern language, because they can co-occur in the same word as seen in \REF{ex:}.


\ea
\gll zefer  qazan-miş-miş-am\\
victory  win-\textsc{perf}-\textsc{indirect-1sg}\\
\glt ‘reportedly I have won’  [Noah Lee, p.c.]
\z


So then, when we encounter a grammatical morpheme which seems to indicate source of information in at least some contexts, but has other functions as well, how can we decide what to call it? In other words, how do we determine its “primary function”? The key is to search for contexts where the expected correlation does not hold, so that the two possible analyses would make different predictions.



David Weber (1989:421 ff.) compares his analysis of the Huallaga Quechua evidential clitics with an alternative analysis which treats them as \textsc{validational} markers, that is, indicators of the speaker’s degree of commitment to the truth of the proposition being expressed. The choice between these two analyses is not immediately obvious, because there is a correlation between source of information and speaker’s degree of commitment. As we have noted, a speaker is likely to be more certain of knowledge gained through direct experience than of knowledge gained through hearsay or inference. In many contexts the direct evidential \textit{=mi} (which is optional) can be used to indicate certainty; and hearers may sometimes interpret the hearsay evidential \textit{=shi} as indicating uncertainty on the part of the speaker.



However, when there is a conflict between source of information and degree of commitment, it is source of information that determines the choice of clitic. For example, if someone were to say ‘My mother’s grandfather’s name was John,’ the direct evidential \textit{=mi} would be extremely unnatural, no matter how firmly the speaker believes what he is saying. The hearsay evidential \textit{=shi} must be used instead, because it is very unlikely in that culture for the speaker to have actually met his great-grandfather. Similarly, in describing cultural practices which the speaker firmly believes but has not personally experienced (e.g., ‘Having chewed coca, their strength comes to them’), the hearsay evidential is strongly preferred.



The general principle is that when we are trying to identify the meaning of a certain form, and there are two or more semantic factors that seem to correlate with the presence of that form, we need to find or create situations in which only one of those factors is possible and test whether the form would appear in such situations.


\section{.Two types of evidentials}\label{sec:} %5 /

In \sectref{sec:3} we mentioned that evidential markers in some languages do not contribute to propositional content but function as illocutionary modifiers. One of the best documented examples of this type is Cuzco Quechua as described by Martina Faller.\footnote{\citet{Faller2002,Faller2003,Faller2006}, inter alia.} Faller analyzes the evidential enclitics in Cuzco Quechua as “illocutionary modifiers which add to or modify the sincerity conditions of the act they apply to.” She notes that “they do not contribute to the main proposition expressed, can never occur in the scope of propositional operators such as negation, and can only occur in illocutionary force bearing environments.”\footnote{Faller (2002: v).}



We present here some of her evidence for saying that the evidential enclitics do not contribute to the propositional content of the utterance, focusing on the Reportative clitic \textit{=si}. First, the evidential is always interpreted as being outside the scope of negation. In example \REF{ex:}, the contribution of the Reportative evidential (‘speaker was told that p’) cannot be interpreted as part of what is being negated; so (ii) is not a possible interpretation for this sentence.


\ea
\gll Ines-qa  mana=s  qaynunchaw  ñaña-n-ta-chu  watuku-rqa-n.\\
Inés-\textsc{top}  not=\textsc{report}  yesterday  sister-3-\textsc{acc-neg}  visit\textsc{-past}\textsc{\textsubscript{1}}-3\\
\glt \textit{propositional content} = ‘Inés didn’t visit her sister yesterday.’\\
\textit{evidential meaning}: (i) speaker was told that Inés did not visit her sister yesterday\\
  \textit{not}:  (ii) \# speaker was not told that Inés visited her sister yesterday\\
{}[\citealt{Faller2002}, sec. 6.3.1]
\z


Second, the contribution of the Reportative evidential is not part of what can be challenged. If a speaker makes the statement in (\ref{ex:}a), a hearer might challenge the truth of the statement based on the facts being reported, as in (\ref{ex:}b); but it would be infelicitous to challenge the truth of the statement based on source of information, as in (\ref{ex:}c). (This test is sometimes called the \textsc{assent/dissent diagnostic}.\footnote{\citet{Papafragou2006}.}) In other words, the contribution of the evidential does not seem to be part of what makes the statement true or false.


\ea
\ea
\gll Ines-qa  qaynunchay  ñaña-n-ta=s  watuku-sqa.\\
Inés-\textsc{top}  yesterday  sister-\textsc{acc}=\textsc{report}  visit\textsc{-past}\textsc{\textsubscript{2}}\\
\glt \textit{propositional content} = ‘Inés visited her sister yesterday.’\\
\textit{evidential meaning}: speaker was told that Inés visited her sister yesterday
\ex \gll  Mana=n  chiqaq-chu.  Manta-n-ta-lla=n  watuku-rqa-n.\\
not=\textsc{direct}  true-\textsc{neg}  mother-3-\textsc{acc}-\textsc{limit}=\textsc{direct}  visit-\textsc{past\textsubscript{1}}-3\\
\glt ‘That’s not true. She only visited her mother.’
\ex \gll  Mana=n  chiqaq-chu.  \#Mana=n  chay-ta  willa-rqa-sunki-chu.\\
not=\textsc{direct}  true-\textsc{neg}  not=\textsc{direct}  this-\textsc{acc}  tell\textsc{-past}\textsc{\textsubscript{1}}-3S.2O-\textsc{neg}\\
\glt ‘That’s not true. \#You were not told this.’  [\citealt{Faller2002}, sec. 5.3.3]
\z \z


Third, Faller’s statement that the evidential enclitics “can only occur in illocutionary force bearing environments” means that they are restricted to main clauses or clauses which express an independent speech act. This is a characteristic feature of many illocutionary modifiers. In particular, conditional clauses are typically not the kind of environment where illocutionary modifiers can occur.\footnote{\citet{Ernst2009}, \citet{Haegeman2010a}.} Faller states that evidential enclitics cannot occur within conditional clauses, as illustrated in \REF{ex:}.


\ea
\gll Mana(*=si)  para-sha-n-chu  chayqa  ri-sun-chis.\\
not=\textsc{report}  rain-\textsc{prog}-3-\textsc{neg}  then  go-1\textsc{fut}-\textsc{pl}\\
\glt ‘If it is not raining we will go.’  [\citealt{Faller2003}, ex. 8]
\z


The German auxiliary \textit{sollen} ‘should’, when used as a reportative or hearsay marker, behaves quite differently. For example, it is possible for \textit{sollen} to occur within a conditional clause, as illustrated in \REF{ex:}.


\ea
F.C.B.F.A.N.: Bei uns \textit{soll} es heute schneien!!\\
‘It is \textit{said} (=predicted) to snow near us today.’\\
FAHRBACH: Also \textit{wenn es bei dir schneien soll}, dann schneit es bei mir auch.\\
‘\textit{If it said to snow near you}, then it will snow near me as well.’\\
   {}[\url{http://www.kc-forum.com/archive/index.php/t-45696}, cited in \citealt{Faller2006}]
\z


The assent/dissent diagnostic reveals another difference. German Reportative \textit{sollen}, like the Quechua Reportative evidential, allows the hearer to challenge the basic propositional content of the sentence. But in addition, it is possible to challenge the truth of a statement with \textit{sollen} based on the source of information, as illustrated in \REF{ex:}.\footnote{\citet{Faller2006}.} This is impossible with the Quechua Reportative. Both of these differences are consistent with the hypothesis that German Reportative \textit{sollen} is part of the propositional content of the utterance.


\ea
A: Laut Polizei \textit{soll} die Gärtnerin die Juwelen gestohlen haben.\\
   ‘According to the police, the gardener \textit{is said} to have stolen the jewels.’\\
B: Nein, das stimmt nicht. Das ist die Presse, die das behauptet.\\
   ‘No, that’s not true. It is the press who is claiming this.’  (\citealt{Faller2006})
\z


A number of languages have evidentials which behave much like those of Cuzco Quechua. However, there are other languages in which evidentials seem to contribute to the propositional content of the utterance, like German Reportative \textit{sollen}. \citet{Murray2010} suggests that we need to recognize two different types of evidential, which we will refer to as \textsc{illocutionary evidentials} and \textsc{propositional evidentials}.\footnote{Murray uses the terms \textsc{illocutionary evidentials} vs. \textsc{epistemic evidentials}.} Illocutionary evidentials function as illocutionary operators; examples are found in Quechua, Kalaallisut, and Cheyenne. Propositional evidentials are part of the propositional content of the utterance; examples are found in German, Turkish, Bulgarian, St’át’imcets (Lillooet Salish), and Japanese.



These two types of evidentials share a number of properties in common, but Murray identifies several tests that distinguish the two classes. For example, illocutionary evidentials cannot be embedded within a conditional clause \REF{ex:}, while this is possible for propositional evidentials \REF{ex:}. Second, a speaker who makes a statement using a hearsay or reportative evidential of the illocutionary type is not committed to believing that the propositional content of the utterance is possibly true. So it is not a contradiction, nor is it infelicitous, for a speaker to assert something as hearsay and then deny that he believes it, as illustrated in \REF{ex:}.


\ea
\ea \gll Para-sha-n=si,  ichaqa  mana  crei-ni-chu.\\
rain-\textsc{prog}-3=\textsc{report}  but  not  believe-1-\textsc{neg}\\
\glt ‘It is raining (someone says), but I don’t believe it.’ [Cuzco Quechua; \citealt{Faller2002}: 194]
\ex \gll  É-hoo'k[22F?]hó-n\.ese  naa  oha  ná-sáa-oné'séomátséstó-he-⌀.\\
3-rain-\textsc{report}.\textsc{inan.sg}  and  \textsc{contr}  1-\textsc{neg}-believe\textsubscript{INAN}-\textsc{mod\textsubscript{A}}\textsubscript{NIM}-\textsc{dir}\\
\glt ‘It’s raining, they say, but I don’t believe it.’  [Cheyenne; \citealt{Murray2010}: 58]
\z \z


A hearsay or reportative evidential of the propositional type, however, commits the speaker to believing that it is at least possible for the expressed proposition to be true. For this reason, the St’át’imcets example in \REF{ex:} is infelicitous.


\ea
(Context: You had done some work for a company and they said they put your pay, \$200, in your bank account; but actually, they didn’t pay you at all.)\\
\gll *Um’-en-tsal-itás  ku7  i  án’was-a  xetspqíqen’kst  táola, t’u7  aoz  kw  s-7um’-en-tsál-itas  ku  stam’.\\
 give-\textsc{dir}-1sg.\textsc{obj}-3pl.\textsc{erg}  \textsc{report}  \textsc{det.pl}  two-\textsc{det}  hundred  dollar but  \textsc{neg}  \textsc{det}  \textsc{nom}-give-\textsc{dir}-1sg.\textsc{obj}-3pl.\textsc{erg}  \textsc{det}  what\\
\glt ‘They gave me \$200 [I was told], but they didn’t give me anything.’\\
   {}[\citealt{Matthewsonetal2007}]
\z


Third, illocutionary evidentials are always speaker-oriented. This means that they indicate the source of information of the speaker, and cannot be used to indicate the source of information of some other participant. This is illustrated in the Quechua example in \REF{ex:}.


\ea
\gll Pilar-qa  yacha-sha-n  Marya-q  hamu-sqa-n-ta\{-n/-s/-chá\}.\\
Pilar-\textsc{top}  know-\textsc{prog}-3  Marya-\textsc{gen}  come-\textsc{past.prtcpl}-3-\textsc{acc\{-dir}/-\textsc{report}/-\textsc{conject\}}\\
\glt \textit{propositional content} = ‘Pilar knows that Marya came.’\\
\textit{evidential meaning}: (i) speaker has direct/reportative/conjectural evidence that\\
    Pilar knows that Marya came\\
\textit{not}: (ii) \#Pilar has direct/reportative/conjectural evidence that Marya came\\
     {}[\citealt{Faller2002}, ex. 184]
\z


Propositional evidentials, in contrast, can be used to indicate the source of information of some participant other than the speaker. In the St’át’imcets example in \REF{ex:}, for example, the reportative evidential is interpreted as marking Lémya7’s source of information. It indicates that Lémya7’s statement was based on hearsay evidence. The speaker in \REF{ex:} already had direct evidence for this information before hearing it from Lémya7.


\ea
\gll tsut  s-Lémya7  kw  sqwemémn’ek  ku7  s-Mary,  t’u7  plán-lhkan ti7  zwát-en  —  áts’x-en-lhkan  s-Mary  áta7  tecwp-álhcw-a  inátcwas\\
say  \textsc{nom}-name  \textsc{det}  pregnant  \textsc{report  nom}-name  but  already-1sg.\textsc{subj} \textsc{dem}  know-\textsc{dir}    see-\textsc{dir}-1sg.\textsc{subj}  \textsc{nom}-name  \textsc{deic}  buy-place-\textsc{det}  yesterday\\
\glt ‘Lémya7 said that [she was told that] Mary is pregnant, but I already knew that — I had seen Mary at the store.’   [\citealt{Matthewson2007}]
\z

A fourth difference that Murray demonstrates is that markers of tense or modality never take semantic scope over illocutionary evidential markers, whereas this is possible with propositional evidentials.\footnote{See Murray (2010, sec. 3.4.2) for examples.}



There seems to be a strong tendency for illocutionary evidential markers to be “true evidentials” in Aikhenvald’s sense, i.e., grammatical morphemes whose primary function is to mark source of information; and for propositional evidentials to be evidential uses/senses of morphemes whose primary function is something else: perfect aspect in Turkish and Bulgarian; modality in German and St’át’imcets. In terms of the distinction we made in \chapref{sec:11}, illocutionary evidentials seem to contribute use-conditional meaning, while propositional evidentials seem to contribute truth-conditional meaning.


\section{Conclusion}\label{sec:} %6. /

We have suggested that a single type of meaning (source of information) can be contributed on two different levels or dimensions: truth-conditional vs. use-conditional. In \chapref{sec:18} we will argue that a similar pattern is observable with adverbial reason clauses. The conjunction \textit{because} expresses a causal relationship, but this causal relationship may either be asserted as part of the truth-conditional propositional content of the sentence, or may function as a kind of illocutionary modifier.



There is much more to be said about evidentials, but we cannot pursue the topic further here. In addition to the semantic issues introduced (all too briefly) above, the use of grammatical evidential markers interacts in interesting ways with discourse genre, world-view, first and second language acquisition, language contact, and translation, to name just a few.



\furtherreading



\citet{Aikhenvald2004} is the primary source for typological and descriptive details about the meanings and functions of evidential markers, and for discussion of the other issues mentioned in the last sentence of this chapter. De \citet{Haan2012} provides a useful overview of the subject, while De \citet{Haan1999,Haan2005} discusses the relationship between evidentiality and epistemic modality.


\chapter{{18}: \textit{Because}}

\section{Introduction}\label{sec:} %1. /

In this chapter we explore the meaning of the conjunction \textit{because} by asking what contribution it makes to the meaning of a sentence. \textit{Because} is used to connect two propositions, so its contribution to the meaning of the sentence will be found in the semantic relationship between those two propositions.



We begin in \sectref{sec:2} by comparing reason clauses introduced by \textit{because} with time clauses introduced by \textit{when}. Time clauses function as adverbial modifiers, but we will argue that \textit{because} has a different function: it combines two propositions into a new proposition which asserts that a causal relationship exists. An important piece of evidence for this analysis comes from certain scope ambiguities which arise in \textit{because} clauses but not in time clauses.



Conjunctions are often polysemous,\footnote{\citet{Aikhenvald2009}.} and various authors have noted that \textit{because} can be used in more than one way. We examine the various uses of \textit{because} in \sectref{sec:3}, but we will argue that \textit{because} is not polysemous. Rather, it has just one sense which can be used in different domains, or dimensions, of meaning: truth-conditional vs. use-conditional. The term \textsc{pragmatic ambiguity} has been proposed to describe such cases, and this term seems appropriate based on the evidence presented below.



In \sectref{sec:4} we will see that the various uses of \textit{because} correlate with different syntactic structures. We will propose diagnostic tests for distinguishing co-ordinate from subordinate \textit{because} clauses. We argue that all of the semantic functions of \textit{because} are possible in the co-ordinate structure, but only one function is possible in the subordinate structure. In \sectref{sec:5} we show that a similar situation holds in German, where the difference between co-ordinate vs. subordinate structures is clearly marked.


\section{2. \textit{Because} as a two-place operator}\label{sec:}

Adverbial clauses occur in complex sentences, in which two (or more) propositions are combined to produce a single complex proposition. However, not all adverbial clauses have the same semantic properties. The examples in (\ref{ex:}--\ref{ex:}) illustrate some of the differences between time clauses and reason clauses:


\ea
\ea Prince Harry wore his medals when he visited the Pope.\\
\ex Prince Harry didn’t wear his medals when he visited the Pope.\\
\ex Did Prince Harry wear his medals when he visited the Pope?
                       \z
\z

\ea
\ea Arthur married Susan because she is rich.\\
\ex Arthur didn’t marry Susan because she is rich.\\
\ex Did Arthur marry Susan because she is rich?
                       \z
\z


All three sentences in \REF{ex:} imply that Harry visited the Pope. As we noted in \chapref{sec:3}, time clauses trigger a presupposition that the proposition they contain is true. Reason clauses do not trigger this kind of presupposition. While sentence (\ref{ex:}a) implies that Susan is rich, sentences (\ref{ex:}b-c) do not carry this inference. Sentence (\ref{ex:}b) could be spoken appropriately by a person who does not believe that Susan is rich, and sentence (\ref{ex:}c) could be spoken appropriately by a person who does not know whether Susan is rich.



So \textit{q because p} does not presuppose that \textit{p} is true; but it entails that both \textit{p} and \textit{q} are true. This entailment is demonstrated in \REF{ex:}.


\ea
\ea George VI became King of England because Edward VIII abdicated;\\
  \#but George did not become king.\\
\ex George VI became King of England because Edward VIII abdicated;\\
  \#but Edward did not abdicate.
                       \z
\z


A second difference between time clauses and reason clauses involves the effect of negation. The negative statement in (\ref{ex:}b) is ambiguous. It can either mean ‘Arthur didn’t marry Susan, and his reason for not marrying her was because she is rich;’ or ‘Arthur did marry Susan, but his reason for marrying her was not because she is rich.’ No such ambiguity arises in sentence (\ref{ex:}b).



The time clause in (\ref{ex:}a) functions as a modifier; it makes the proposition expressed in the main clause more specific or precise, by restricting its time reference. \textit{Because} clauses seem to have a different kind of semantic function. \citet{Johnston1994} argues that \textit{because} is best analyzed as an operator CAUSE, which combines two propositions into a single proposition by asserting a causal relationship between the two.\footnote{This operator is probably different from the causal operator involved in morphological causatives, which is often thought of as a relation between an individual and an event/situation.} We might define this operator as shown in \REF{ex:}:


\begin{stylepoints}
\textit{CAUSE(p,q)} is true iff \textit{p} is true, \textit{q} is true, and \textit{p} being true causes \textit{q} to be true.
\end{stylepoints}


For example, if \textit{p} and \textit{q} are descriptions of events in the past, \textit{CAUSE(p,q)} would mean that \textit{p} happening caused \textit{q} to happen. A truth table for \textit{CAUSE} would look very much like the truth table for \textit{and}; but there is a crucial additional element of meaning that would not show up in the truth table, namely the causal relationship between the two propositions.\footnote{The definition of causality is a long-standing problem in philosophy, which we will not address here. One way to think about it makes use of a counter-factual (see \chapref{sec:19}): \textit{CAUSE(p,q)} means that if \textit{p} had not happened, \textit{q} would not have happened either.}



This analysis provides an immediate explanation for the ambiguity of sentence (\ref{ex:}b) in terms of the scope of negation:


\ea
  \textit{Arthur didn’t marry Susan because she is rich.}\\
\ea ¬CAUSE(RICH(s), MARRY(a,s))\\
\ex CAUSE(RICH(s), ¬MARRY(a,s))
                       \z
\z


If this approach is on the right track, we would expect to find other kinds of scope ambiguities involving \textit{because} clauses as well. This prediction turns out to be correct: in sentences of the form \textit{p because} \textit{q}, if the first clause contains a scope-bearing expression such as a quantifier, modal, or propositional attitude verb, that expression may be interpreted as taking scope either over the entire sentence or just over its immediate clause. Some examples are provided in (\ref{ex:}--\ref{ex:}).


\ea
\textit{Few people admired Churchill because he joined the Amalgamated Union of Building Trade Workers.}\\
\ea CAUSE(JOIN(c,aubtw), [few x: person(x)] ADMIRE(x,c))\\
\ex{} [few x: person(x)] CAUSE(JOIN(c,aubtw), ADMIRE(x,c))
                       \z
\z

\ea
\textit{I believed that you love me because I am gullible.}\\
\ea BELIEVE(s, CAUSE(GULLIBLE(s), LOVE(h,s))\\
\ex CAUSE(GULLIBLE(s), BELIEVE(s, LOVE(h,s)))\\
    {}[s = speaker; h = hearer]
                       \z
\z


One reading for sentence \REF{ex:}, which is clearly false in our world, is that only a few people admired Churchill, and the reason for this was that he joined the AUBTW. The other reading for sentence \REF{ex:}, very likely true in our world, is that only a few people’s admiration of Churchill was motivated by his joining of the AUBTW; but many others may have admired him for other reasons. (The reader should work out the two readings for sentence \REF{ex:}.)


\section{3. Use-conditional \textit{because}}\label{sec:}

Now let us consider the apparent polysemy of \textit{because}. Sweetser (1990:76–78) suggests that \textit{because} (and a number of other conjunctions) can be used in three different ways: 


\begin{quote}
Conjunction may be interpreted as applying in one of (at least) three domains [where] the choice of a “correct” interpretation depends not on form, but on a pragmatically motivated choice between viewing the conjoined clauses as representing content units, logical entities, or speech acts. [1990:78]
\end{quote}

\ea
\ea John came back because he loved her.   [\textsc{content} domain]\\
\ex John loved her, because he came back.   [\textsc{epistemic} domain]\\
\ex What are you doing tonight, because there’s a good movie on. [\textsc{speech act} domain]
                       \z
\z


The content domain has to do with “real-world causality”; in (\ref{ex:}a), John’s love causes him to return. The epistemic domain (\ref{ex:}b) has to do with the speaker’s grounds for making the assertion expressed in the main clause: the content of the \textit{because} clause (\textit{he came back}) provides evidence for believing the assertion (\textit{John loved her}) to be true. Sweetser explains the speech act domain (\ref{ex:}c) as follows:


\begin{quote}
{}[T]he \textit{because} clause gives the cause of the \textit{speech act} embodied by the main clause. The reading is something like ‘I \textit{ask} what you are doing tonight because I want to suggest that we go see this good movie.’ [1990:77]
\end{quote}


Sweetser denies that these three uses involve different senses of \textit{because}. Rather, she argues that \textit{because} has a single sense which can operate on three different levels, or domains, of meaning. She describes this situation, taking a term from \citet{Horn1985}, as a case of \textsc{pragmatic ambiguity}; in other words, an ambiguity of usage rather than an ambiguity of sense.



This seems like a very plausible suggestion; but any such proposal needs to account for the fact that the various uses of \textit{because} are distinguished by a number of real differences, both semantic and structural. The most obvious of these is the presence of pause, or “comma intonation”, between the two clauses. The pause is optional with “content domain” uses of \textit{because}, as in (\ref{ex:}a), but obligatory with other uses. If the pause is omitted in (\ref{ex:}b-c), the sentences can only be interpreted as expressing real-world causality, even though this interpretation is somewhat bizarre. (With the pause, (\ref{ex:}b) illustrates an “epistemic” use while (\ref{ex:}c) illustrates a “speech act” use.)


\ea
\ea Mary scolded her husband (,) because he forgot their anniversary.\\
\ex Arnold must have sold his Jaguar \#(,) because I saw him driving a 1995 minivan.\\
\ex Are you hungry \#(,) because there is some pizza in the fridge?
                       \z
\z


Several of the tests that we used in previous chapters to distinguish truth-conditional propositional content from use-conditional meaning also distinguish the “content domain” use from the other uses of \textit{because}: questionability, capacity for being negated, and capacity for being embedded within conditional clauses. Let us look first at the interpretation of yes-no questions. When “content domain” uses of \textit{because} occur as part of a yes-no question, the causal relationship itself is part of what is being questioned, as in (\ref{ex:}a). With other uses, however, the causal relationship is not questioned; the scope of the interrogative force is restricted to the main clause, as in (b, epistemic) and (c, speech act). If we try to interpret (\ref{ex:}b-c) as questioning the causal relationship (the reading which is required if we omit the pause), we get rather bizarre “content domain” interpretations.


\ea
\ea Did Mary scold her husband because he forgot their anniversary?\\
\ex Did Arnold sell his Jaguar, because I just saw him driving a 1995 minivan?\\
\ex Are you going out tonight, because I would like to come and visit you?
                       \z
\z


We find a similar difference regarding the scope of negation. As noted in \sectref{sec:2}, when a sentence containing a \textit{because} clause is negated, the negation can be interpreted as taking scope over the whole sentence including the causal relationship. But this is only possible with “content domain” uses of \textit{because}, like (\ref{ex:}a). With “epistemic” (\ref{ex:}b) or “speech act” (\ref{ex:}c) uses, negation only takes scope over the main clause. Once again, attempting to interpret negation with widest scope in (\ref{ex:}b-c) results in bizarre readings involving real-world causality.


\ea
\ea Arthur didn’t marry Susan because she is rich.\\
\ex You couldn’t have failed phonetics, because you graduated.\\
\ex Mary is not home, because I assume that you really came to see her.
                       \z
\z


Similarly, “content domain” uses of \textit{because} can be embedded within conditional clauses, as seen in (\ref{ex:}a); but this is impossible with “epistemic” (\ref{ex:}b) or “speech act” (\ref{ex:}c) uses:


\ea
\ea[]{If Mary scolded her husband because he forgot their anniversary, they will be back on speaking terms in a few days.\\}
\ex[??]{If Arnold sold his Jaguar because I just saw him driving a 1995 minivan, he is  likely to regret it.\\}
\ex[??]{If you are hungry because there is some pizza in the fridge, please help yourself.}
\z \z


Looking back at the differences we have listed so far, we see that in each case the “content domain” use of \textit{because} behaves differently from the other two uses, while the “epistemic” and “speech act” uses always seem to behave in the same way. In other words, the evidence we have considered up to this point provides solid grounds for distinguishing two uses of \textit{because}, but not for distinguishing the “epistemic” and “speech act” uses.



The evidence we have considered thus far suggests that “content domain” uses of \textit{because} contribute to truth-conditional propositional content, while “epistemic” and “speech act” uses of \textit{because} contribute use-conditional meaning. In light of this evidence, we will adopt Sweetser’s suggestion that \textit{because} has a single sense, treating the different uses as a case of pragmatic ambiguity. However, we will posit just two (rather than three) relevant domains (or dimensions) of meaning: truth-conditional vs. use-conditional.\footnote{A number of other authors have made a similar two-way distinction for \textit{because} clauses, with use-conditional \textit{because} clauses treated as a type of speech act adverbial; see for example \citet{Scheffler2008,Scheffler2013} and \citet{ThompsonLongacreHwang2007}.} 



In use-conditional functions of \textit{because}, the conjunction expresses a causal relationship between the proposition expressed by the \textit{because} clause and the speech act expressed in the main clause,\footnote{As argued by Sweetser.} as illustrated in (\ref{ex:}b-c).


\ea
\ea \textit{John came back because he loved her}. (\textit{her} =Mary)  [\textsc{truth-conditional}]\\
CAUSE(LOVE(j,m), COME\_BACK(j))
\ex   \textit{John loved her, because he came back}.   [\textsc{use-conditional}]\\
CAUSE(COME\_BACK(j), I assert that LOVE(j,m))
\ex   \textit{What are you doing tonight, because there’s a good movie on}.  [\textsc{use-conditional}]\\
CAUSE(there’s a good movie on, I ask you what you are doing tonight)
\z \z


The nature of the causal relationship in use-conditional functions is often closely related to the felicity conditions for the particular speech act involved. One of the felicity conditions for making an assertion is that the speaker should have adequate grounds for believing that the assertion is true. Sweetser’s “epistemic” \textit{because} clauses, like the one in (\ref{ex:}b), provide evidence which forms all or part of the grounds for the assertion expressed in the main clause.



Sweetser’s “speech act” \textit{because} clauses often explain the speaker’s reason for performing the speech act or why it is felicitous in that specific context. The \textit{because} clause in (\ref{ex:}c) explains why the speaker is asking the question, and so provides guidance for the hearer as to what kind of answer will be relevant to the speaker’s purpose.



Two clauses which are joined by use-conditional \textit{because} behave in some ways like separate speech acts. As illustrated in examples (\ref{ex:}c), (\ref{ex:}c), and (\ref{ex:}b-c) above, a main clause that is followed by a use-conditional \textit{because} clause can contain a question, even when the \textit{because} clause itself is an assertion. It is also possible for the main clause to contain a command in this context, as illustrated in \REF{ex:}.\footnote{The fact that the \textit{because} clauses in these examples start with \textit{I know that …} blocks any potential interpretion as “content domain” \textit{because} clauses.}


\ea
\ea Give me the tickets, because I know that you will forget them somewhere.\\
\ex Take my sandwich, because I know that you have not eaten anything today.
                       \z
\z


Such examples show that a use-conditional \textit{because} clause and its main clause can have separate illocutionary forces, and so can constitute distinct speech acts.


\section{Structural issues: co-ordination vs. subordination}\label{sec:} %4. /

Another difference between truth-conditional vs. use-conditional \textit{because} clauses is that only the truth-conditional type can be fronted, as illustrated in \REF{ex:}. Sentences (\ref{ex:}b-c) would most naturally be interpreted as use-conditional examples if the \textit{because} clause followed the main clause. But when the \textit{because} clause is fronted they can only be interpreted as expressing real-world causality, even though this interpretation is somewhat bizarre.


\ea
\ea  Because it’s raining, we can’t go to the beach.  [\textsc{truth-conditional}]\\
\ex ??Because I saw Arnold driving a 1995 minivan, he sold his Jaguar. \\
  {}[*\textsc{use-conditional}]\\
\ex ??Because I assume that you came to see her, Mary hasn’t come home yet.\\
  {}[*\textsc{use-conditional}]
                       \z
\z


\citet{Haspelmath1995} points out that subordinate clauses can often be fronted, but this is typically impossible for co-ordinate clauses. The examples in (\ref{ex:}--\ref{ex:}) show that a variety of subordinate clauses in English can be fronted. The examples in (\ref{ex:}--\ref{ex:}) show that this same pattern of fronting is not possible with co-ordinate clauses (though of course it would be possible to reverse the order of the clauses leaving the conjunction in place between them). In light of this observation, the fact that use-conditional \textit{because} clauses cannot be fronted suggests that they may actually be co-ordinate clauses rather than subordinate clauses.


\ea
\ea George will give you a ride when you are ready.\\
\ex When you are ready, George will give you a ride.
                       \z
\z

\ea
\ea Paul will sing you a song if you ask him nicely.\\
\ex If you ask him nicely, Paul will sing you a song.
\z \z

\ea
\ea Ringo draped towels over his snare drum in order to deaden the sound.\\
\ex In order to deaden the sound, Ringo draped towels over his snare drum.
                       \z
\z

\ea
\ea George played the sitar and John sang a solo.\\
\ex *And John sang a solo, George played the sitar.
                       \z
\z

\ea
\ea Paul asked for tea but the waiter brought coffee.\\
\ex *But the waiter brought coffee, Paul asked for tea.
                       \z
\z


As we noted above, a pause (comma intonation) is optional before truth-conditional \textit{because} clauses but obligatory before use-conditional \textit{because} clauses. (We focus here on the situation where the \textit{because} clause follows the main clause, since pause is always obligatory when the \textit{because} clause is fronted.) We can explain this observation if we assume that a pause in this context is an indicator of co-ordinate structure, and that use-conditional functions of \textit{because} are only possible in co-ordinate structures. Truth-conditional interpretations of \textit{because} are possible in either co-ordinate or subordinate structures, i.e., with or without a pause. Only the truth-conditional interpretation is possible in subordinate structures (where there is no pause), even when this interpretation is pragmatically unlikely or bizarre (see b-c).



Additional support for the hypothesis that a pause is a marker of co-ordination comes from the fact that the scope ambiguities discussed in \sectref{sec:2} disappear when a pause is inserted between the two clauses. The examples in \REF{ex:} are not ambiguous, whereas the corresponding examples with no pause are (see b, , and ). It is not surprising that an operator in a matrix clause can take scope over a subordinate clause; it would be much less common for an operator in one half of a co-ordinate structure to take scope over the other half.


\ea
\ea Arthur didn’t marry Susan, because she is rich.\\
\ex Few people admired Churchill, because he joined the trade union.\\
\ex I believed that you love me, because I am gullible.
                       \z
\z


Interrogative force exhibits similar scope effects: example \REF{ex:} shows that when a pause is present, the causal relationship cannot be part of what is being questioned. And example \REF{ex:} shows that a co-ordinate \textit{because} clause cannot be embedded within a conditional clause.


\ea
Did Mary scold her husband, because he forgot their anniversary?\\
  (can only be understood as reason for asking, not as reason for scolding)
\z

\ea
\#If Mary scolded her husband, because he forgot their anniversary, they will be back\\
  on speaking terms in a few days.
\z


In the previous section we used negation, questioning, and embedding within \textit{if} clauses to argue that Sweetser’s “epistemic” and “speech act” \textit{because} clauses contribute use-conditional rather than truth-conditional meaning. But if those uses of \textit{because} are only possible in co-ordinate structures, one might wonder whether perhaps the different behavior of negation, questioning and embedding is due to purely structural factors, and is therefore not semantically relevant?



However, there is at least one test that can be applied to co-ordinate structures, and this test confirms the semantic distinction we argued for in the previous section. This is the challengeability test: the truth of a statement can typically only be challenged on the basis of truth-conditional propositional content. As the following examples show, the truth of a statement which contains a “content” \textit{because} clause can be appropriately challenged based on the causal relationship itself, even when the co-ordinate structure is used as in \REF{ex:}. With “epistemic” and “speech act” \textit{because} clauses, however, the truth of the statement can be challenged based on the content of the main clause, but not based on the causal relationship or the content of the \textit{because} clause (\ref{ex:}--\ref{ex:}).


\ea
A: Mary is leaving her husband, because he refuses to look for a job.\\
B: That is not true; Mary is leaving her husband because he drinks too much.
\z

\ea
A: Mary is at home, because her car is in the driveway.\\
B1: That is not true. She is not home; she went out on her bicycle.\\
B2: \#That is not true; you know that Mary is home because you just talked with her.
\z

\ea
A: There is some pizza in the fridge, because you must be starving.\\
B1: That is not true; we ate the pizza last night.\\
B2: \#That is not true; you told me about the pizza because want to get rid of it.
\z


To summarize, we have proposed that adverbial clauses introduced by \textit{because} can occur in two different structural configurations, co-ordinate or subordinate. Co-ordinate \textit{because} clauses must be separated from the main clause by a pause (comma intonation), but this pause is not allowed before subordinate \textit{because} clauses (when they follow the main clause). The co-ordinate structure allows either truth-conditional or use-conditional interpretations of \textit{because}, but only the truth-conditional use is possible in the subordinate structure. Subordinate \textit{because} clauses can occur within the scope of clausal negation and interrogative force, and can be embedded within conditional clauses; but none of these things is possible with co-ordinate \textit{because} clauses.


\section{5. Two words for ‘because’ in German}\footnotemark{}\label{sec:}
\footnotetext{The material in this section is based almost entirely on the work of Tatjana \citet{Scheffler2005,Scheffler2008}, and all examples that are not otherwise attributed come from these works.}

The situation in German is very similar, but the distinction between co-ordinate and subordinate structures is much easier to recognize in German than in English. German has two different words which are translated as ‘because’. Both of these words can be used to describe real-world causality, as illustrated in (\ref{ex:}--\ref{ex:}). In each case, the a and b sentences have the same English translation.


\ea
\ea \gll Ich  habe  den  Bus  verpasst,  \textit{weil}  ich  spät  dran  war.\\
1sg  \textsc{aux}  the.\textsc{acc}  bus  missed  because  1sg  late  there  was\\
\ex  Ich habe den Bus verpasst, \textit{denn} ich war spät dran.\\
\glt ‘I missed the bus because I got there late.’ [http://answers.yahoo.com]
\z \z

\ea
\ea \gll  Die  Straße  ist  ganz  naß,  \textit{weil}  es  geregnet  hat.\\
the.\textsc{nom}  street  is  all  wet  because  it  rained  \textsc{aux}\\
\ex \gll Die Straße ist ganz naß, \textit{denn} es hat geregnet.\\
‘The street is wet because it rained.’  [\citealt{Scheffler2008}, sec. 3.1]\\
\z \z

However, in other contexts the two words are not interchangeable. Only \textit{denn} can be used to translate use-conditional functions of \textit{because}. This includes both Sweetser’s “epistemic” use, as in \REF{ex:}, and her “speech act” use, as in \REF{ex:}. \textit{Weil} cannot be used in such sentences.


\ea
\ea Es hat geregnet, \textit{denn} die Straße ist ganz naß.\\
\ex *Es hat geregnet, \textit{weil} die Straße ganz naß ist.\\
‘It was raining, because the street is wet.’
                       \z
\z

\ea
\ea \gll Ist  vom  Mittag  noch  etwas  übrig?\\
is  from  midday  still  anything  left.over\\
\gll \textit{Denn}  ich  habe  schon  wieder  Hunger.\\
because  1sg  have  already  again  hunger\\
\ex ?? Ist vom Mittag noch etwas übrig? \textit{Weil} ich schon wieder Hunger habe.\\
\glt ‘Is there anything left over from lunch? Because I’m already hungry again.’
\z \z


There are structural differences between the two conjunctions as well: \textit{weil} is a subordinating conjunction, whereas \textit{denn} is a co-ordinating conjunction. The difference between subordination and co-ordination in German is clearly visible due to differences in word order. In German main clauses, the auxiliary verb (or tensed main verb if there is no auxiliary) occupies the second position in the clause, as illustrated in (\ref{ex:}a). In subordinate clauses, however, the auxiliary or tensed main verb occupies the final position in the clause, as illustrated in (\ref{ex:}b).\footnote{This is true for subordinate clauses which are introduced by a conjunction or complementizer. Where there is no conjunction or complementizer at the beginning of the subordinate clause, the auxiliary or tensed main verb occupies the second position.}


\ea
\ea  \gll Ich  \textit{habe}  zwei  Hunde  gekauft.\\
1sg.\textsc{nom}  have  two  dogs  bought.\textsc{prtcpl}\\
\glt ‘I have bought two dogs.’
\ex \gll Sie  sagt,  daß  er  dieses  Buch  gelesen  \textit{hätte}.\\
3sg.\textsc{f}.\textsc{nom}  says  that  3sg.\textsc{m}.\textsc{nom}  this.\textsc{acc}  book  read  have.\textsc{sbjnt}\\
\glt ‘She says that he has read this book.’
\z \z


Looking back at examples (\ref{ex:}--\ref{ex:}), we can see that the tensed verbs \textit{war} ‘was’ and \textit{hat} ‘has’ occur in second position following \textit{denn} but in final position following \textit{weil}. This contrast provides a clear indication that \textit{weil} clauses are subordinate while \textit{denn} clauses are co-ordinate. Further evidence that \textit{weil} clauses are subordinate while \textit{denn} clauses are co-ordinate comes from their syntactic behavior. First, \textit{weil} clauses can be fronted but \textit{denn} clauses cannot, as shown in \REF{ex:}. Second, \textit{weil} clauses can stand alone as the answer to a \textit{why}-question like that in \REF{ex:}, whereas \textit{denn} clauses cannot. This is one of the classic tests for syntactic constituency. The contrast in \REF{ex:} suggests that \textit{weil} combines with the clause that it introduces to form a complete syntactic constituent, whereas \textit{denn} does not. This is what we would expect if \textit{weil} is a subordinating conjunction and \textit{denn} is a co-ordinating conjunction.\footnote{Notice that the tensed verb \textit{sah} ‘saw’ occupies the final position in (\ref{ex:}a).}


\ea
\ea \textit{Weil} es geregnet hat, ist die Straße naß.\\
\ex *\textit{Denn} es hat geregnet, ist die Straße naß.\\
\glt ‘Because it rained, the street is wet.’
\z \z

\ea
\ea \gll Warum  ist  die  Katze  gesprungen?  \textit{Weil}  sie  eine  Maus  sah.\\
why  \textsc{aux}  the.\textsc{nom}  cat  jumped  because  she  a  mouse  saw\\
\ex  Warum ist die Katze gesprungen? —*\textit{Denn} sie sah eine Maus.\\
\glt ‘Why did the cat jump? — Because it saw a mouse.’
\z \z


In our earlier discussion we demonstrated that subordinate \textit{because} clauses in English can be negated, questioned, or embedded within conditional clauses; whereas none of these things is possible with co-ordinate \textit{because} clauses. Interestingly, a very similar pattern emerges in German. As illustrated in \REF{ex:}, \textit{weil} clauses can be interpreted within the scope of main clause negation, whereas \textit{denn} clauses cannot.


\ea
\ea \gll  Paul  ist  nicht  zu  spät  gekommen,  \textit{weil}  er  den  Bus  verpaßt  hat.\\
Paul  \textsc{aux}  \textsc{neg}  too  late  come  because  he  the.\textsc{acc}  bus  missed  \textsc{aux}\\
\gll   {}[Sondern  er  hatte  noch  zu  tun.]\\
  rather  he  had  still  to  do\\
\ex  \#Paul ist nicht zu spät gekommen, \textit{denn} er hat den Bus verpaßt.\\
  {}[Sondern er hatte noch zu tun.]\\
\glt ‘Paul wasn’t late because he missed the bus.\\
{}[But rather, because he still had work to do.]’
\z \z


Similarly, \textit{weil} clauses in questions can be interpreted as part of what is being questioned, that is, within the scope of the interrogative force (\ref{ex:}a). \textit{Denn} clauses cannot be interpreted in this way, as shown in (\ref{ex:}b).


\ea
\ea Wer kam zu spät, \textit{weil} er den Bus verpaßt hat?\\
\ex ?? Wer kam zu spät, \textit{denn} er hat den Bus verpaßt?\\
\glt ‘Who was late because he missed the bus?’
                       \z
\z


\textit{Denn} clauses cannot be embedded within a subordinate clause, whereas this is possible with \textit{weil} clauses. Example \REF{ex:} illustrates this contrast in a complement clause, and \REF{ex:} in a conditional clause.


\ea
\ea  \gll Ich  glaube  nicht,  daß  Peter  nach  Hause  geht,  \textit{weil}  er  Kopfschmerzen  hat.\\
1sg  believe  \textsc{neg}  \textsc{comp}  Peter  to  home  goes  because  he  headache  has\\
\ex  \#Ich glaube nicht, daß Peter nach Hause geht, \textit{denn} er hat Kopfschmerzen.\\
\glt ‘I don’t believe that Peter is going home because he has a headache.’
\z \z

\ea
\ea Wenn Peter zu spät kam, \textit{weil} er den Bus verpaßt hat, war es seine eigene Schuld.\\
\ex \#Wenn Peter zu spät kam, \textit{denn} er hat den Bus verpaßt, war es seine eigene Schuld.\\
\glt ‘If Peter was late because he missed the bus, it was his own fault.’
\z \z


\citet{Scheffler2008} points out that \textit{denn} clauses are normally unacceptable if the content of the \textit{because}-clause is evident or has been previously mentioned. This explains why only \textit{weil} is possible in the mini-conversation in \REF{ex:}. This interesting observation suggests that \textit{denn} clauses, because of their coordinate structure, count as independent assertions. As we noted in \chapref{sec:3}, in our discussion of entailments, asserting a fact which is already part of the common ground typically creates an unnatural redundancy.


\ea
\ea Es hat heute sehr geregnet.\\
— Ja, die ganze Straße steht unter Wasser, \textit{weil} es geregnet hat.\\
\ex Es hat heute sehr geregnet.\\
— \#Ja, die ganze Straße steht unter Wasser, \textit{denn} es hat geregnet.\\
\glt ‘It rained a lot today.\\
— Yes, the whole street is submerged under water because it rained.’
                       \z
\z


A number of other languages also have two words for ‘because’, including Modern Greek, Dutch, and French.\footnote{\citet{Pit2003}; \citet{Kitis2006}.}


\section{Conclusion}\label{sec:} %6. /

We have identified two basic uses of \textit{because} in English: truth-conditional vs. use-conditional. These two uses can be distinguished using familiar tests for truth-conditional propositional content. First, truth-conditional \textit{because} clauses can be part of what is negated or questioned when the sentence as a whole is negated or questioned, but this is not the case with use-conditional \textit{because}. Second, truth-conditional \textit{because} clauses can be embedded within \textit{if} clauses, but use-conditional \textit{because} clauses cannot. Third, the truth of a statement can be appropriately challenged based on the causal relationship expressed in a truth-conditional \textit{because} clause, but not on that expressed in a use-conditional \textit{because} clause.



We have also identified two different structural configurations in which \textit{because} may occur: co-ordinate vs. subordinate. Diagnostics for distinguishing these two structures include the following: (i) Subordinate \textit{because} clauses can be fronted, but co-ordinate \textit{because} clauses cannot. (ii) Co-ordinate \textit{because} clauses must be separated from the main clause by a pause (comma intonation), but this pause is not allowed before subordinate \textit{because} clauses. (iii) Scope ambiguities involving negation, quantifiers, modals, or propositional attitude verbs are possible with subordinate \textit{because} clauses, but not with co-ordinate \textit{because} clauses.



We proposed the following structural constraint on the interpretation of \textit{because}: the truth-conditional use of \textit{because} may occur in either a subordinate or a co-ordinate clause, but the use-conditional interpretation is possible only in the co-ordinate structure. This same constraint holds in German as well, but in German the two structures are introduced by different conjunctions: \textit{weil} for subordinate reason clauses, and \textit{denn} for co-ordinate reason clauses.



\furtherreading



Sæbø (1991) and (2011, sec. 3.3) provide a good overview of the semantics of causal connectives like \textit{because}, and a comparison with other types of adverbial connectives. \citet{Lewis1973b} and (2000) lay out two different versions of his counterfactual analysis of causation. Scheffler (2013, ch. 4) provides a detailed discussion of the syntax and semantics of the two German conjunctions meaning ‘because’.


\subsubsection{Discussion exercises:}\label{sec:}

\textbf{A:} Explain the scopal ambiguity of the following sentences, and state the two readings in logical notation:

\ea
\textsf{Model answer:}\\
\textsf{\textit{Arthur didn’t marry Susan because she is rich}}\textsf{.}\\
\ea ¬CAUSE(RICH(s), MARRY(a,s))\\
\ex CAUSE(RICH(s), ¬MARRY(a,s))\\
\z \z

\begin{enumerate}
\item 
\textit{Mrs. Thatcher will not win because she is a woman}. (spoken in 1979)
\item \itshape
Tourists rarely visit Delhi because the food is so spicy.
\item 
\textit{I doubt that Peter is happy because he was fired.}
\end{enumerate}

\textbf{B:} Show how you could use some of the tests discussed in \chapref{sec:18} to determine whether the \textit{because} clauses in the following examples contribute truth-conditional or use-conditional meaning:

\begin{enumerate}
\item 
\textit{Arthur works for the State Department, because he has a STATE.GOV e-mail address}.
\item 
\textit{Oil prices are rising, because OPEC has agreed to cut production}.
\end{enumerate}
\subsubsection{Homework exercises:}\label{sec:}

\chapref{sec:18} provides this analysis for the scopal ambiguity of the following sentence:

\ea
  \textit{Arthur didn’t marry Susan because she is rich.}\\
\ea ¬CAUSE(RICH(s), MARRY(a,s))\\
\ex CAUSE(RICH(s), ¬MARRY(a,s))
\z \z

Provide a similar analysis showing the two possible readings for each of the following sentences. (If you wish, you may write out the clauses in prose rather than using formal logic notation, e.g.: ¬CAUSE(\textit{Susan is rich}, \textit{Arthur marry Susan})).

\begin{enumerate}
\item \textit{Steve Jobs didn’t start Apple because he loved technology}.\footnote{https://www.fastcompany.com/3001441/do-steve-jobs-did-dont-follow-your-passion}
\item \textit{Arnold must have sold his Jaguar because I saw him driving a minivan}.
\item \textit{Few Texans voted for Romney because he is a Mormon}.
\item \textit{Susan believes that A.G. Bell was rich because he invented the telephone}.
\end{enumerate}

\chapter{{19}: Conditionals}

\begin{quote}
“Exactly what conditionals mean and how they come to mean what they mean is one of the oldest problems in natural language semantics. According to Sextus Empiricus, the Alexandrian poet Callimachus reported that the Greek philosophers’ debate about the semantics of the little word \textit{if} had gotten out of hand: ‘Even the crows on the roof-tops are cawing about which conditionals are true’.”\footnote{von \citet{Fintel2011}}
\end{quote}

\section{Conditionals and modals}\label{sec:} %1. /

A \textsc{conditional} sentence is a bi-clausal structure of the form \textit{if p} (\textit{then) q}. The conjunction \textit{if} seems to indicate that a certain kind of relationship holds between the meanings of the two clauses. However, as the passage quoted above demonstrates, the exact nature of this relationship has been a topic of controversy for thousands of years.



An intuitive description of the construction, suggested by the term \textsc{conditional}, is that the \textit{if} clause describes some condition under which the \textit{then} clause is claimed to be true. For example, the conditional sentence in \REF{ex:} claims that the proposition \textit{You are my second cousin} is true under a certain condition, namely that Atatürk was your great-grandfather.


\ea
If Atatürk was your great-grandfather, then you are my second cousin.
\z


Much recent work on the meaning of conditional constructions builds on the similarities between conditionals and modals. The analysis of modality that we sketched in \chapref{sec:16} treats modal operators as quantifiers over possible worlds: modals of necessity are universal quantifiers, while modals of possibility are existential quantifiers. The difference between epistemic vs. deontic or other types of modality is the result of restricting this quantification to specific kinds of worlds. For example, we analyzed epistemic \textit{must} as meaning something like, “In all worlds which are consistent with what I know about the actual world, and in which the normal course of events is followed…”



Conditionals can also be analyzed in terms of possible worlds. One way of evaluating the truth of a conditional statement like \REF{ex:} is to adopt the following procedure:\footnote{This is a version of the “Ramsey Test” from \citet{Stalnaker1968}.} Add the content of the \textit{if} clause to what is currently known about the actual world. Under those circumstances, would the \textit{then} clause be true? We might suggest the following paraphrase for sentence \REF{ex:}: “In all possible worlds which are consistent with what I know about the actual world, and in which the normal course of events is followed, and in which Atatürk was your great-grandfather, you are my second cousin.”



An adequate analysis needs to provide not only a reasonable paraphrase but also an explanation for how this meaning is derived compositionally, addressing questions like the following: What do the individual meanings of the two clauses contribute to the meaning of the sentence as a whole? What does \textit{if} mean? These questions lead to some very complex issues, to which this chapter can provide only a brief introduction.



It will be easier to talk about conditional sentences if we introduce some standard terminology for referring to the parts of such sentences. We refer to the \textit{if} clause as the \textsc{antecedent} (also known as the “protasis”); and to the \textit{then} clause as the \textsc{consequent} (or “apodosis”). The names “antecedent” and “consequent” reflect the most basic ordering of these clauses (\textit{if p, q}), not only in English but (apparently) in all languages.\footnote{Greenberg (1963: 84–85); \citet[83]{Comrie1986}} But in many languages the opposite order (\textit{q if p}) is possible as well. Regardless of which comes first in any particular sentence, the antecedent names the condition under which the consequent is claimed to be true.



One factor that makes the analysis of conditional sentences so challenging is that the conditional structure can be used for a variety of different functions, not only in English but in many other languages as well. We introduce the most common of these in \sectref{sec:2}. In \sectref{sec:3} we focus on “standard” conditionals, i.e. those in which neither the antecedent nor the consequent is asserted or presupposed to be true. In many languages these conditionals may be marked by tense, mood, or other grammatical indicators to show the speaker’s degree of confidence as to how likely the antecedent is to be true.



In \sectref{sec:4} we will return to the question raised in \chapref{sec:9} as to whether the meaning of English \textit{if} can be adequately represented or defined in terms of the material implication operator (→) of propositional logic. We will see that, for a number of reasons, this does not seem to be possible. (Of course, that does not mean that the material implication operator is useless for doing natural language semantics; it is an indispensible part of the logical metalanguage. It just means that material implication does not provide a simple translation equivalent for English \textit{if}.)



We go on in \sectref{sec:5} to discuss one very influential approach to defining the meaning of \textit{if}, which takes it to be a marker of restriction for modals or other types of quantifiers. \sectref{sec:key:6} discusses some of the special challenges posed by \textsc{counterfactual} conditionals, in which the antecedent is presupposed to be false. In \sectref{sec:7} we argue for a distinction between truth-conditional vs. speech act conditionals, and provide some evidence for the claim that speech act conditionals are not part of the propositional content that is being asserted, questioned, etc.


\section{2. Four uses of \textit{if}}\label{sec:}

In this section we introduce the most commonly discussed functions of the conditional construction. As noted above, the \textsc{standard} conditional, illustrated in \REF{ex:}, does not commit the speaker to believing either the antecedent or the consequent to be true, but does seem to commit the speaker to believing that some type of relation exists (often a causal relationship) between the two propositions. Most authors take this to be the most basic usage of \textit{if}.


\ea
\textsc{standard conditionals}\\
\ea If it does not rain, we will eat outside.\\
\ex If the TV Guide is correct, there is a good documentary on PBS tonight.\\
\ex There are biscuits on the sideboard if Bill has not moved them.\\
\ex If you take another step, I’ll knock you down.\\
\ex If Mary’s husband forgets their anniversary (again!), she will never forgive him.\\
\ex If you see George, you should invite him to the party.
                       \z
\z


The sentences in \REF{ex:} provide examples of \textsc{relevance} \textsc{conditionals}, also known as “biscuit conditionals” because of the famous example listed here as (\ref{ex:}a). When the consequent is a statement, as in (\ref{ex:}a-d), the relevance conditional seems to commit the speaker to believing the consequent to be true, regardless of whether the antecedent is true or not.\footnote{This claim has been challenged by some authors.}


\ea
\textsc{relevance} \textsc{conditionals} (a.k.a. “biscuit conditionals”):\\
\ea There are biscuits on the sideboard if you want them.  (\citealt{Austin1956})\\
\ex PBS will broadcast \textit{Die Walküre} tonight, if you like Wagner.  (\citealt{Bennet2003})\\
\ex If I may say so, you do not look well.\\
\ex He’s not the sharpest knife in the drawer, if you know what I mean.\\
\ex If you went to the office party, how did Susan look?
                       \z
\z


The replies in (\ref{ex:}--\ref{ex:}) illustrate \textsc{factual} \textsc{conditionals}.\footnote{These examples are adapted from   \citet[671]{BhattPancheva2006}.} Factual conditionals carry the presupposition that someone other than the speaker (often the addressee) believes or has said that the proposition expressed by the antecedent is true.


\ea
A. This book that I was assigned to read is really stupid.\\
B. I haven’t read it, but if it is that stupid you shouldn’t bother with it.
\z

\ea
A. My boyfriend Joe is really smart.\\
B. Oh yeah? If he’s so smart why isn’t he rich?
\z


The final type that we will mention is the \textsc{concessive conditional}, illustrated in \REF{ex:}. (Accent marks are used here to indicate intonation peak.) A speaker who uses a concessive conditional asserts that the consequent is true no matter what, regardless of whether the antecedent is true or false. This is made explicit when, as is often the case, the antecedent is preceded by \textit{even if}. Notice that the most basic order for concessive conditionals seems to be the opposite of that for standard conditionals, i.e., the consequent comes first. In order for the antecedent to be stated first, it must be marked by \textit{even}, focal stress, or some other special marker.


\ea
\textsc{Concessive Conditionals}\\
\ea I wouldn’t marry you if you were the last man on éarth.\\
\ex (Even) if the bridge were stánding I wouldn’t cross. (\citealt{Bennett1982})\\
\ex I’m going to finish this project (even) if it kílls me.
                       \z
\z


We need to distinguish concessive conditionals from concessive adverbial clauses,\footnote{Thompson, Longacre, \& \citet{Hwang2007}.} which can be marked with various conjunctions including \textit{if}. Some examples of concessive adverbial clauses are presented in \REF{ex:}, and examples of concessive adverbial clauses with \textit{if} in \REF{ex:}.\footnote{The examples in \REF{ex:} come from LanguageLog: \url{http://itre.cis.upenn.edu/~myl/languagelog/archives/000408.html}}  This kind of concessive construction commits the speaker to believing that both the antecedent and the consequent are true.


\ea
\ea Even though the bridge is still standing, I won’t cross it.\\
\ex Although she loves him, she does not plan to marry him.\\
\ex While no one has seen Bigfoot, few people here doubt its existence.
                       \z
\z

\ea
\ea It’s all perfectly normal — if troublesome to varying degrees.\\
\ex Virtual colon dissection is promising, if flawed.\\
\ex It was fair and balanced if perhaps a little old.\\
\ex Today hashing is a global, if little known, pursuit.\\
\ex If Eskimos have dozens of words for snow, Germans have as many for\\
  bureaucracy. [\textit{The Economist}, October 11th, 2003, p. 56, col. 2]\\
\ex If Mozart was a life-long admirer of J. C. Bach, his views on Clementi were\\
  disparaging, to put it mildly.  [OED, citing 1969 \textit{Listener} 24 Apr. 585/1]
                       \z
\z


The contrast in truth commitments mentioned above is illustrated in \REF{ex:}. The standard conditional in (\ref{ex:}a) does not imply that the speaker believes either the antecedent or the consequent to be true, so denying the consequent does not lead to contradiction or anomaly. The concessive conditional in (\ref{ex:}b) and the relevance conditional in (\ref{ex:}c) both imply that the speaker believes the consequent to be true, regardless of the truth of the antecedent; so denying the consequent is a contradiction.


\ea
\ea I wouldn’t marry Bill if he were a starving linguist; but as things stand I might end up\\
  marrying him (since he is a dentist).  [\textsc{standard} \textsc{conditional}]\\
\ex I wouldn’t marry Bill if he were the last man on éarth; \#but I suppose I might end up\\
  marrying him.  [\textsc{concessive conditional}]\\
\ex If you really want to know, I would never marry Bill; \#but I suppose I might end up\\
  marrying him.  [\textsc{relevance} \textsc{conditional}]
                       \z
\z


In the long history of the study of conditionals and their meanings, a variety of additional functions and gradations have been identified and named (often with multiple competing names for the same function, as we have already seen in the case of “relevance” or “biscuit” conditionals). \sectref{sec:key:7} below provides some evidence for making a distinction between truth-conditional vs. speech act uses of the conditional form. This is of course the same distinction that we were led to in the previous chapter in our discussion of causation. We will argue that the standard conditionals in \REF{ex:} involve a truth-conditional usage, whereas the relevance conditionals in \REF{ex:} involve a speech act usage. The factual and concessive conditionals in (\ref{ex:}--\ref{ex:}) are harder to classify.


\section{Degrees of hypotheticality}\label{sec:} %3. /

One widely discussed property of standard conditionals is that they can be used to express varying degrees of hypotheticality,\footnote{See for example \citet{Comrie1986}; Thompson, Longacre, \& \citet{Hwang2007}.} reflecting the speaker’s judgment as to how likely it is that the antecedent is actually true. In languages where verbs are inflected for tense and/or mood, verbal morphology is often used to signal these distinctions. However, other kinds of marking are also found, as illustrated below; and in some languages this distinction is not grammatically marked at all, but is determined entirely by contextual clues.



As a number of authors have noted, there is a cross-linguistic tendency for the antecedent to be interpreted as more hypothetical (less certain) when it is stated in the past tense than in present tense. However, tense marking also serves to indicate the actual time frame of the described event. (See \chapref{sec:21} for a detailed discussion of tense marking.) For this reason, there is generally no one-to-one correlation between tense and degree of hypotheticality. Some English examples are presented in (\ref{ex:}--\ref{ex:}).


\ea
\ea If Bill \textit{is} your uncle, then you must know his daughter Margaret.\\
\ex If David \textit{was} your thesis advisor, then he knows your work pretty well.\\
\ex If Susan \textit{wins} the election, she will become the mayor of Des Moines.\\
\ex Results have not yet been announced, but if Susan \textit{won} the election,\\
  the current mayor will have to find a new job.\\
\ex ‘‘It would make it more important if that \textit{be} the case,’’ he [Ralph Nader] said\\
  yesterday.\footnote{New York Daily News, 5 \citealt{February2007}; cited in \citet{Gomes2008}.}
                       \z
\z


In the indicative mood, either present or past tense can be used when the speaker has reason to believe that the antecedent is true, as illustrated in (\ref{ex:}a--b). Such examples are sometimes referred to as \textsc{reality} conditionals.\footnote{Thompson, Longacre, \& \citet{Hwang2007}.} These same two verb forms can also be used in \textsc{hypothetical} conditionals, those in which the speaker simply doesn’t know whether the antecedent is true or not, as illustrated in (\ref{ex:}c--d). In these examples, the tense marking of the verb in the antecedent functions in the normal way, to indicate the location in time of the situation described by that clause. The subjunctive mood can be used for hypothetical conditionals as well, as illustrated in (\ref{ex:}e). However, it is not always easy to recognize the subjunctive in English. The past indicative and past subjunctive are distinguished in Modern English only for the verb \textit{to be}, as illustrated in (\ref{ex:}a).\footnote{The present subjunctive is identical to the bare infinitive form. It is archaic in conditionals, though still used occasionally in formal registers as in (\ref{ex:}e), but preserved in other uses, including optatives (\textit{God bless you}; \textit{long live the King}).}



\textsc{Counterfactual} conditionals, which normally presuppose that the speaker believes the antecedent to be false, tend to be expressed in the subjunctive as seen in (\ref{ex:}--\ref{ex:}). Example (\ref{ex:}a) demonstrates the preference for the subjunctive over the past indicative in counterfactual conditionals, although many speakers will use or at least accept the past indicative in casual speech.


\ea
\ea If I \textit{were/?was} you, I would apply for a different job.\\
\ex If I \textit{had been} your thesis advisor, you would have been lucky to finish at all.
                       \z
\z

\ea
“Sir, if you \textit{were} my husband, I would poison your drink.”\\
“Madam, if you \textit{were} my wife, I would drink it.”\\
  (Exchange between Lady Astor and Winston Churchill)
\z


\citet{Comrie1986} argues that the degrees of hypotheticality associated with conditionals are not limited to three discrete categories, but rather form a continuum from most certain (reality conditionals) to most doubtful (counterfactuals). The examples in \REF{ex:} lend some support to this claim, at least for English. All three of these examples can be interpreted as hypothetical conditionals referring to a present situation, i.e., the state of the world at the time of speaking; none of them requires that the speaker know whether the antecedent is true or not. However, the past indicative in (\ref{ex:}b) seems more doubtful than the present indicative in (\ref{ex:}a), and the subjunctive mood in (\ref{ex:}c) seems more doubtful than the indicative mood in (\ref{ex:}b). (Without any additional context, the subjunctive conditional in (\ref{ex:}c) would most likely be interpreted as a counterfactual; but given the right context, the hypothetical reading is certainly possible as well.) In the same way, both (\ref{ex:}a) and (\ref{ex:}b) can be interpreted as hypothetical conditionals, but (\ref{ex:}b) expresses more doubt than (\ref{ex:}a). Notice that in (\ref{ex:}b), the tense marking of the antecedent does not reflect the time of the described situation, but is used to mark a high degree of hypotheticality.


\ea
\ea If Alice \textit{is} a spy, she probably carries a gun.\\
\ex If Alice \textit{was} a spy, she would probably carry a gun.\\
\ex If Alice \textit{were} a spy, she would probably carry a gun.
                       \z
\z

\ea
\ea If Authur still \textit{loves} her, he will catch the first train home.\\
\ex If Authur still \textit{loved} her, he would catch the first train home.
                       \z
\z


These examples show that, in English conditional clauses, tense and mood morphology have partly overlapping functions. Both past tense and subjunctive mood can serve to make the antecedent seem less likely. Similar patterns are found in other languages as well.



The use of tense and mood in Portuguese conditionals is illustrated in \REF{ex:}.\footnote{Examples from \citet{Gomes2008}.} Example (\ref{ex:}a) is what we have called a reality conditional, (\ref{ex:}b) is a hypothetical conditional, and (\ref{ex:}c) is a counterfactual conditional. Notice that the difference between the hypothetical and counterfactual conditionals is formally a difference in tense inflection, rather than mood, on the antecedent verb. Notice too the “conditional mood” form of the verb in the consequent of (\ref{ex:}c). A number of Romance languages have such forms, which occur in the consequent of counterfactual conditionals and typically have several other uses as well (e.g. “future in the past” tense; see \chapref{sec:21}).


\ea
\ea \gll Se  ela  é  italiana,  ela  é  européia.\\
if  she  is  Italian,  she  is  European\\
\glt ‘If/since she is Italian, she is European.’ (I know that she is Italian.)
\ex \gll Se  ela  for  italiana,  ela  é  européia.\\
if  she  be.3sg.\textsc{fut.sbjnct}  Italian,  she  is  European\\
\glt ‘If she be Italian, she is European.’ (I do not know whether she is Italian or not.)
\ex \gll  Se  ela  fosse  italiana,  ela  seria  européia.\\
if  she  be.3sg.\textsc{imperf.sbjnct}  Italian,  she  would.be.\textsc{cond}  European\\
\glt ‘If she were Italian, she would be European.’ (I know that she is not Italian.)
\z \z


In Russian counterfactual conditionals, both the antecedent and consequent appear in the subjunctive-conditional mood (\ref{ex:}b), in contrast to the indicative mood used in hypothetical conditionals (\ref{ex:}a):\footnote{These examples are from Chung and \citet[251]{Timberlake1985}, who use the term \textsc{irrealis} mood for what I have called the subjunctive-conditional mood.}


\ea
\ea \gll Esli  ja  pribudu  na  vokzal,  menja  posadjat  v  tjur’mu.\\
if  I  arrive.\textsc{indic}  at  station  me  put.\textsc{indic}  in  prison\\
\glt ‘If I arrive at the station, they will throw me in prison.’ 
\ex \gll Esli  by  ja  pribyl  na  vokzal,  menja  by  posadili  v  tjur’mu.\\
if  \textsc{cond}  I  arrive.\textsc{cond}  at  station  me  \textsc{cond}  put.\textsc{cond}  in  prison\\
\glt ‘If I had shown up at the station, they would have thrown me in prison.’
\z \z


The contrast between hypothetical vs. counterfactual conditionals can be marked in other ways as well. Irish has two distinct words for ‘if’: \textit{dá} is used in counterfactual conditionals (\ref{ex:}a), while \textit{má} is used in hypothetical conditionals (\ref{ex:}b).\footnote{\citet{McCloskey2001}.} A similar situation is reported in Welsh and some varieties of Arabic.


\ea
\ea  \gll Dá  leanfadh  sé  dá  chúrsa,  bheadh  deireadh  leis.\\
if  follow.\textsc{cond}  he  of.his  course  be.\textsc{cond}  end  with.him\\
\glt ‘If he had persisted in his course, he’d have been finished.’
\ex \gll  Má  leanann  tú  de  do  chúrsa,  beidh  aithreachas  ort.\\
if  follow.\textsc{pres}  you  of  your  course  be.\textsc{fut}  regret  on.you\\
\glt ‘If you persist in your (present) course, you’ll be sorry.’  [\citealt{McCloskey2001}]
\z \z


In Tolkapaya (also known as Western Yavapai), a Yuman language of North America, counterfactuals are distinguished from other kinds of conditionals by the suffix \textit{–th} attaching to the auxiliary of the consequent clause.\footnote{\citet{HardyGordon1980}.} In other (non-conditional) contexts, this suffix is used to mark “non-factual” propositions, including “failed attempts, unfulfilled desires, descriptions of a state that formerly obtained but which no longer does, and situations where the realization of one event precludes that of another” (\citealt{HardyGordon1980}:191).



A very similar case is found in Kimaragang Dusun, spoken in northeastern Borneo.\footnote{\citet{Kroeger2017}.} The frustrative particle \textit{dara} appears in main clauses which express failed attempts, unfulfilled desires or intentions, former states that no longer obtain, and things done in vain. This same particle appears in the consequent clause of counterfactual conditionals, as seen in \REF{ex:}, distinguishing counterfactuals from other types of conditionals like those in \REF{ex:}. Notice that non-past tense is used in the consequent of a counterfactual even if the situation which failed to materialize would have been prior to the time of speaking, as in (\ref{ex:}b).


\ea
\ea \gll  Ong  noguring  no  koniab  ino,  atanaman  no   do  paray  benoy  \textit{dara}.\\
if  plowed.\textsc{potent.pst}  already  yesterday  that  planted.\textsc{potent.npst}  already \textsc{acc}  rice  today  \textsc{frus}\\
\glt ‘If that (field) had been plowed yesterday, it could have been planted with rice today.’
\ex \gll  Amu  \textit{dara}  agamit  i  kambing  ong  konoko  ginipit  sid  susut.\\
\textsc{neg}  \textsc{frus}  caught.\textsc{potent.npst}  \textsc{nom}  goat  if  not  trapped\textsc{.pst  loc}  below\\
\glt ‘The goat could not have been caught if we hadn’t trapped it under the house.’
\z \z

\ea
\ea \gll  Ong  amu  nu  ibaray  ino  siin  dino,  mangan  tekaw  posutay.\\
if  \textsc{neg}  you  pay  that  money  that  \textsc{aux}  I.you  cane\\
\glt ‘If you don’t pay that money I’ll cane you.’
\ex \gll  Kaanak=i’  dati  yalo  dilo’  ong  sumambat  do=duktur.\\
able.to.bear.child=\textsc{emph}  \textsc{prob}  3sg  that  if  meet  \textsc{acc}=doctor\\
\glt ‘She could probably have children if she goes to the doctor.’
\z \z


Some languages do not mark the degree of hypotheticality at all, at least not in their most common conditional sentence patterns. In these languages, a single sentence can be ambiguous between the reality, hypothetical, and counterfactual conditional readings; the intended meaning must be determined from context. For example, the Japanese sentence in \REF{ex:} could be interpreted either as a hypothetical conditional (expressing the hope of a father whose son is missing in action), or as a counterfactual conditional (expressing the sorrow of a father whose son has been killed).\footnote{\citet[627]{Akatsuka1985}.} \citet{Comrie1986} mentions Mandarin and Indonesian as examples of other languages where a similar ambiguity is normal.


\ea
\gll Musuko=ga  ikite  i-tara,  ii  noni  naa!\\
son=\textsc{nom}  alive  be-if  good  though  \textsc{exclam}\\
\glt ‘If my son is alive, I’ll be so happy.’\\
or: ‘If my son were alive, I would be so happy.’
\z


To sum up, counterfactual conditionals get distinctive marking in many languages, but not in all languages. Now let us return to the fundamental question raised in \sectref{sec:1}: what does \textit{if} mean?


\section{4. English \textit{if} vs. material implication}\label{sec:}

In \chapref{sec:9} we presented evidence in support of Grice’s analysis of the English words \textit{and} and \textit{or}. Grice suggested that the lexical semantic content of these words is actually equivalent to their logical counterparts ($\wedge$ and $\vee$), and that apparent differences in meaning are best understood as conversational implicatures. This approach seems to work fairly well for those two words; could a similar approach work for English \textit{if}? Grice argued that it could, specifically proposing that the lexical semantic content of English \textit{if} is equivalent to the material implication operator (→). However, there are a number of reasons to believe that this approach will not work for \textit{if}.



First, if \textit{if} really means material implication, then the truth table for material implication predicts that the sentences in \REF{ex:} should all be true. (Recall that \textit{p→q} is only false when \textit{p} is true and \textit{q} is false.) However, this does not match our intuitions about these sentences; most English speakers are very reluctant to call any of them true.


\ea
\ea If Socrates was a woman then 1+1=3.\footnote{\url{http://en.wikipedia.org/wiki/Material_conditional}} \\
\ex If the Amazon flows through Paris then triangles have three sides.\\
\ex If the Chinese invented gunpowder then Martin Luther was German.
                       \z
\z


What makes these sentences seem so odd is that there is no relationship between the antecedent and consequent. Whatever \textit{if} means, it seems to require that some such relationship be present. Grice argued that this inference of relationship between antecedent and consequent is only a conversational implicature. Several other authors have also proposed that the semantic content of \textit{if} is simply material implication, and that the apparent differences between the two are pragmatic rather than semantic in nature. Other authors have tried to account for the requirement of relationship between antecedent and consequent by suggesting that \textit{if} \textit{p then q} expresses the claim that \textit{p→q} is true in all possible worlds, i.e., under any imaginable circumstances.\footnote{C.I. \citet{Lewis1918}, cited in Von \citet{Fintel2011}.} But any attempt to derive the meaning of \textit{if} from material implication must deal with a number of problems.



As discussed in \chapref{sec:4}, the meaning of the material implication operator is entirely defined by its truth table. We need to know the truth values for both \textit{p} and \textit{q} (but nothing else) before we can determine the truth value for \textit{p→q}. But this does not match our judgments about the truth of English conditionals. It would be entirely possible for a competent native speaker to believe that sentence \REF{ex:} is true without knowing whether either of the two clauses alone expresses a true proposition. What is being asserted in \REF{ex:} is not a specific combination of truth values, but a relationship between the meanings of the clauses.\footnote{The material in this paragraph and the next are based on observations made by \citet{Podlesskaya2001}.}


\ea
If this test result is accurate, your son has TB.
\z


This point is further demonstrated by the fact that, in addition to statements, questions and commands may also appear as the consequent clause of a conditional, as illustrated in \REF{ex:}. This is significant because questions and commands cannot be assigned a truth value.


\ea
\ea If you are offered a fellowship, will you accept it?\\
\ex If you want to pass phonetics, memorize the IPA chart!
                       \z
\z


Finally, as we will argue in more detail below, the antecedent in a speech act conditional like \REF{ex:} does not specify conditions under which the consequent is true, but rather conditions under which the speech act performed by the consequent may be felicitous.\footnote{In order to account for such examples under the assumption that \textit{if} is equivalent to the material implication operator, we could interpret them as “conditional speech acts”; so (\ref{ex:}c) would have an interpretation something like: “If I am permitted to say so, then I hereby assert that you do not look well.” But in fact someone who says (\ref{ex:}c) seems to be asserting the consequent unconditionally; it is only the felicity of the assertion that is conditional.}


\ea
\ea If you have a pen, may I please borrow it?\\
\ex If you want my advice, don’t invite George to the party!\\
\ex If I may say so, you do not look well.
                       \z
\z


Even if we focus only on truth values, the logical properties of → make predictions which do not seem to hold true for English \textit{if}. For example, it is easy to show (from the truth table for →) that \textit{¬(p→q)} logically entails \textit{p}. So if the semantic value of \textit{if} is material implication, anyone who believes that (\ref{ex:}a) is false is committed to believing that (\ref{ex:}b) is true. However, it does not seem to be logically inconsistent for a speaker to believe both statements to be false.


\ea
\ea If I win the National Lottery, I will be happy for the rest of my life.\\
\ex I will win the National Lottery.
                       \z
\z


Counterfactuals raise a number of special problems for the material implication analysis. We will mention here just one famous example, shown in \REF{ex:}.\footnote{This example is due to David \citet{Lewis1973a}.} It is easy to show that \textit{p→q} logically implies \textit{(p}$\wedge$\textit{r) → q}. So if the semantic value of \textit{if} is material implication, anyone who believes that (\ref{ex:}a) is true should be committed to believing that (\ref{ex:}b) is true. However, it does not seem to be logically inconsistent for a speaker to believe the first statement to be true while believing the second to be false.


\ea
\ea If kangaroos had no tails, they would topple over.\\
\ex If kangaroos had no tails and they used crutches, they would topple over.
                       \z
\z


Many other similar examples have been pointed out, and various solutions have been proposed.\footnote{See Von \citet{Fintel2011} for a good summary; see also \citet[83-87]{Gazdar1979}; Bennett (2003: ch. 2-3).} As we noted in \sectref{sec:1} above, even if material implication is not logically equivalent to English \textit{if}, that does not mean that it is irrelevant to natural language semantics. It will always be an important part of the logical metalanguage that semanticists use. But in view of the many significant differences between material implication and English \textit{if}, it seems reasonable to look for some other way of capturing the meaning of \textit{if}.


\section{\textit{5. If} as a restrictor}\label{sec:}

A radically different approach to defining the meaning of \textit{if} was proposed by \citet{Kratzer1986}, based on a suggestion by David \citet{Lewis1975}. As we mentioned in \chapref{sec:14}, Lewis analyzes adverbs like \textit{always}, \textit{sometimes}, \textit{usually}, \textit{never}, etc. as “unselective quantifiers”, because they can quantify over various kinds of things. He points out that conditional clauses can be used to specify the situations, entities, or units of time which are being quantified over, as illustrated in \REF{ex:}. However, it is difficult to say exactly what the \textit{if} means in such examples.


\ea
\ea  If it is sunny, we \textit{always/usually/rarely/sometimes/never} play soccer. [D. \citealt{Lewis1975}]\\
\textit{always}: ${\forall}$\textsubscript{d} [SUNNY(d) → (we play soccer on d)]\\
\textit{sometimes}: ${\exists}$\textsubscript{d} [SUNNY(d) $\wedge$ (we play soccer on d)]\\
\textit{usually}: ???
\ex  If a man wins the lottery, he \textit{always/usually/rarely/sometimes/never} dies happy.\\
\textit{always}: ${\forall}$\textsubscript{x} [MAN(x) $\wedge$ WIN(x,lottery) → DIE\_HAPPY(x)]\\
\textit{sometimes}: ${\exists}$\textsubscript{x} [MAN(x) $\wedge$ WIN(x,lottery) $\wedge$ DIE\_HAPPY(x)]\\
\textit{usually}: ???
\z \z


Example (\ref{ex:}a) is a standard conditional whose antecedent expresses the proposition SUNNY(d), using \textit{d} as a variable for days. The adverbs \textit{always}, \textit{sometimes}, etc, specify the quantifier part of the meaning. The word \textit{if} seems to name the relation between the antecedent and the consequent; but with \textit{always} this relation is expressed by →, with \textit{sometimes} the relation is expressed by $\wedge$, and with adverbs like \textit{usually} and \textit{rarely} there is no way to express the relation in standard logical form. A similar problem arises in (\ref{ex:}b). What these examples show is that we cannot identify any consistent contribution of the word \textit{if} to the meaning of the sentence in this construction.



Using the restricted quantifier notation allows us to give a uniform compositional analysis for such sentences, regardless of which adverb is used. As shown in \REF{ex:}, the antecedent of the conditional clause contributes material to the restriction on the quantifier, and the consequent specifies the material in the scope of the quantifier. But notice that there is no element of meaning in these expressions corresponding to the word \textit{if}. Lewis concludes that in this construction, \textit{if} “has no meaning apart from the adverb it restricts.”


\ea
If a man wins the lottery, he \textit{always/usually/rarely/sometimes/never} dies happy.\\
\textit{always}:  [\textit{all} x: MAN(x) $\wedge$ WIN(x,lottery)] DIE\_HAPPY(x)\\
\textit{sometimes}:  [\textit{some} x: MAN(x) $\wedge$ WIN(x,lottery)] DIE\_HAPPY(x)\\
\textit{usually}:  [\textit{most} x: MAN(x) $\wedge$ WIN(x,lottery)] DIE\_HAPPY(x)\\
\textit{rarely}:  [\textit{few} x: MAN(x) $\wedge$ WIN(x,lottery)] DIE\_HAPPY(x)\\
\textit{never}:  [\textit{no} x: MAN(x) $\wedge$ WIN(x,lottery)] DIE\_HAPPY(x)
\z


\citet{Kratzer1986} proposed that Lewis’s analysis could be extended to all indicative (i.e., non-counterfactual) standard conditionals. If the conditional sentence contains a quantifier-like element in the consequent, the word \textit{if} serves only as a grammatical marker introducing material that contributes to the restriction on the quantifier. This is illustrated in \REF{ex:} for normal quantifier phrases, and in \REF{ex:} for epistemic and deontic modality.


\ea
\ea  Every student will succeed if he works hard.\\
{}[\textit{all} x: STUDENT(x) $\wedge$ WORK\_HARD(x)] SUCCEED(x)
\ex  No student will succeed if he goofs off.\\
{}[\textit{no} x: STUDENT(x) $\wedge$ GOOF\_OFF(x)] SUCCEED(x)
\z \z

\ea
\ea  If John did not come to work, he must be sick.  [epistemic necessity]\\
  {}[\textit{all} w: (w is consistent with what I know about the actual world) $\wedge$\\
  (the normal course of events is followed as closely as possible in w) $\wedge$\\
  (John did not come to work in w)] SICK(j) in w
\ex  If John did not come to work, he must be fired.  [deontic necessity]\\
  {}[\textit{all} w: (the relevant circumstances of the actual world are also true in w) $\wedge$\\
  (the relevant authority’s requirements are satisfied as completely as possible in w) $\wedge$\\
  (John did not come to work in w)] FIRED(j) in w
\z \z


Kratzer suggests that when a conditional sentence does not contain an overt quantifier-like element, the presence of \textit{if} leads the hearer to assume a default quantifier. In some contexts, this default element would be epistemic necessity, as in (\ref{ex:}a). In other contexts, the default element could be generic frequency, as in (\ref{ex:}b).\footnote{Examples from Von \citet{Fintel2011}. As we will see in \chapref{sec:21}, the English simple present tense has special properties which explain the generic frequency interpretation of examples like (\ref{ex:}b).}


\ea
\ea \textit{If John left at noon, he’s home by now}.   [implied: epistemic necessity]\\
  {}[\textit{all} w: (w is consistent with what I know about the actual world) $\wedge$\\
  (the normal course of events is followed as closely as possible in w) $\wedge$\\
  (John left at noon in w)] HOME(j) in w (by time of speaking)
\ex \textit{If John leaves work on time, he has dinner with his family}. [implied: generic frequency]\\
  {}[\textit{all} d: (d is a day) $\wedge$ (John leaves work on time in d)]\\
    John has dinner with his family in d
\z \z


\citet[11]{Kratzer1986} summarizes her proposal as follows:


\begin{quote}
“The history of the conditional is the story of a syntactic mistake. There is no two-place \textit{if} … \textit{then} connective in the logical forms for natural languages. \textit{If}-clauses are devices for restricting the domains of various operators. Whenever there is no explicit operator, we have to posit one.”
\end{quote}


Her point is that the “conditional” meaning, the sense of relationship between antecedent and consequent, is not encoded by the word \textit{if}. Rather, it comes from the structure of the quantification itself. The function of \textit{if} is to mark certain material (the antecedent) as belonging to the restriction rather than the scope of the quantifier.



The proposal that \textit{if} “does not carry any distinctive conditional meaning”\footnote{Von \citet{Fintel2011}.} may get some support from the observation that conditional readings can arise in sentences where two clauses are simply juxtaposed without any marker at all, as seen in (\ref{ex:}--\ref{ex:}).


\ea
Examples of juxtaposed conditionals from LanguageLog\footnote{\url{http://itre.cis.upenn.edu/~myl/languagelog/archives/004521.html}} \\
\ea “Listen,” Renda said, “\textit{we get to a phone we’re out of the country before morning}.”
\ex “He could have been a little rusty early on, and then the inning he gave up four runs I think he kind of lost his composure a little bit,” Orioles manager Sam Perlozzo said. “\textit{He just did a little damage control in that situation, we’re OK}.” [AP Recap of Toronto-Baltimore game of May 22, 2007; David Ginsburg, AP Sports Writer]
\z \z

\ea
INIGO: We’re really in a terrible rush.\\
MIRACLE MAX: Don’t rush me, sonny.\\
  \textit{You rush a miracle man, you get rotten miracles}.  [\textit{The Princess Bride}]
\z

\section{6. Counterfactual conditionals}\footnotemark{}\label{sec:}
\footnotetext{This section draws heavily on von \citet{Fintel2012}.}

The Lewis-Kratzer proposal provides a great deal of help in understanding how the meaning of a conditional sentence is compositionally derived. However, determining the right meanings for certain types of conditionals is still a significant challenge. Counterfactuals are an especially challenging case. Consider the contrast between the hypothetical conditional in (\ref{ex:}a) and the counterfactual conditional in (\ref{ex:}b).\footnote{Counterfactual and hypothetical conditionals are often referred to as “subjunctive” and “indicative” conditionals, respectively; but as we noted in \sectref{sec:3}, there is not always a perfect correlation between verb morphology and the degree of hypotheticality.}


\ea
\ea If Shakespeare did not write \textit{Hamlet}, someone else did.\\
\ex If Shakespeare had not written \textit{Hamlet}, someone else would have.\footnote{\citet{Morton2004}.}
                       \z
\z


Most English speakers would probably agree that the hypothetical conditional in (\ref{ex:}a) is true, but would probably judge the counterfactual conditional in (\ref{ex:}b) to be false. This contrast suggests that some different rule of interpretation must apply to counterfactual conditionals. We have said that a counterfactual conditional presupposes that the antecedent is false; but this by itself is not sufficient to cause sentence (\ref{ex:}b) as a whole to be regarded as false. Notice that even a speaker who believes the antecedent in (\ref{ex:}a) to be false, i.e., who believes that Shakespeare did write \textit{Hamlet}, would probably judge the sentence as a whole to be true.



Ideally we would like to apply the same analysis of \textit{if} to both types of conditionals, but this would make it hard to explain why the two sentences in \REF{ex:}, which are structurally very similar have different truth conditions. What makes the counterfactual conditional in (\ref{ex:}b) so odd is that it seems to imply that there is (or was) something about our world which made the writing of \textit{Hamlet} inevitable. The hypothetical conditional in (\ref{ex:}a) carries no such inference. How can we account for this difference?



In the preceding section we sketched out a procedure for interpreting conditionals that do not contain an overt quantifier. In many contexts, an epistemic necessity modal has to be assumed in order to arrive at the intended interpretation. The truth conditions of the sentence are calculated by adding the content of the antecedent to what is known about the actual world in order to derive the appropriate restriction on the set of possible worlds. This procedure yields an interpretation something like \REF{ex:} for the hypothetical conditional in (\ref{ex:}a). Intuitively, this feels like a reasonable interpretation. Part of what we know about the world is that plays do not grow on trees, so if a play such as \textit{Hamlet} exists (another part of what we know about the actual world), then someone must have written it.


\ea
{}[\textit{all} w: (w is consistent with the available evidence) $\wedge$ (the normal course of events is followed as closely as possible in w) $\wedge$ (Shakespeare did not write \textit{Hamlet} in w)] someone else wrote \textit{Hamlet} in w
\z


With the counterfactual conditional in (\ref{ex:}b), the process is more complex. We cannot simply add the content of the antecedent to what is known about the actual world, because the antecedent is assumed to be false in the actual world. One approach is to quantify over those possible worlds in which the antecedent is true, but which are otherwise as similar as possible to the actual world. Roughly speaking, (\ref{ex:}b) could be paraphrased as follows: “For all worlds w in which Shakespeare did not write \textit{Hamlet}, but which are otherwise as similar as possible to the actual world \textit{in the relevant ways}: someone else wrote \textit{Hamlet} in w.” Of course, the success of such an analysis depends on how one determines the relevant points of similarity that need to be considered.



This general approach can help explain why the counterfactual conditionals in (\ref{ex:}a--b), repeated here as (\ref{ex:}a--b), have different truth conditions. Sentence (\ref{ex:}a) restricts the domain of quantification to worlds which are as similar as possible to the actual world, aside from the stipulation that kangaroos have no tails. In these worlds presumably kangaroos do not use crutches, since that would constitute an extra unforced difference as compared to the actual world. Sentence (\ref{ex:}b) however adds the additional stipulation that kangaroos do use crutches in all the relevant worlds. For this reason, kangaroos would be more likely to topple over in the worlds relevant to evaluating (\ref{ex:}a) than in those relevant to evaluating (\ref{ex:}b).


\ea
\ea If kangaroos had no tails, they would topple over.\\
\ex If kangaroos had no tails and they used crutches, they would topple over.
                       \z
\z


Now the phrase “as similar as possible” is admittedly vague, and it is reasonable to wonder whether using this criterion to restrict the domain of quantification will be very helpful in determining the meaning of a sentence. However, some authors have argued that the vagueness and context-dependence of the term are in fact good things, because counterfactuals themselves are somewhat vague, and the correct interpretation depends heavily on context.\footnote{Lewis (1973a: 91 ff); von \citet{Fintel2012}.} Consider the following examples from \citet[221]{Quine1960}:


\ea
\ea If Caesar were in command, he would use the atom bomb.\\
\ex If Caesar were in command, he would use catapults.
                       \z
\z


A given feature of the real world may be given more or less priority in determining relative closeness between two worlds depending on various contextual factors, including the purposes of the speaker. In (\ref{ex:}a), for example, Caesar’s ruthless nature may outrank his historical setting, but in (\ref{ex:}b) the technological resources of his era are given higher priority. The speaker’s purpose plays an important role in determining which ordering source should be applied in each case. \citet[221]{Quine1960} expresses this principle in the following words:


\begin{quote}
The subjunctive [= counterfactual; PK] conditional depends, like indirect quotation and more so, on a dramatic projection: we feign belief in the antecedent and see how convincing we then find the consequent. What traits of the real world to suppose preserved in the feigned world of the contrary-to-fact antecedent can only be guessed from a sympathetic sense of the fabulist’s likely purpose in spinning his fable.
\end{quote}


The pair of sentences in \REF{ex:} above is quite similar to the famous pair in \REF{ex:}. Once again, the hypothetical conditional in (\ref{ex:}a) seems to be true, while most people would probably judge the counterfactual conditional in (\ref{ex:}b) to be false. However, the historical facts in this case are still somewhat controversial and poorly understood, which makes it difficult to decide which points of comparison would be relevant for determining the “most similar” possible worlds.


\ea
\ea If Oswald didn’t kill Kennedy, someone else did. \\
\ex If Oswald hadn’t killed Kennedy, someone else would have.\footnote{\citet{Adams1970}.}
                       \z
\z


Consider instead the counterfactual conditional in \REF{ex:}. While not everyone would consider this sentence to be true, it at least makes a claim that a historian could consider as a serious hypothesis:


\ea
If John Wilkes Booth hadn’t killed Abraham Lincoln, someone else would have.
\z


What claim does \REF{ex:} make? Based on our discussion above, this sentence could be paraphrased roughly as follows: “For all worlds w in which Booth did not kill Lincoln, but which are otherwise as similar as possible to the actual world in the relevant ways: someone else killed Lincoln in w.” In this context, relevant points of similarity to the real world on April 14, 1865 (the night when Lincoln was shot) might include the following:


\begin{enumerate}
\item The on-going civil war: Gen. Lee’s army had surrendered in Virginia on April 9, 1865 but fighting continued for a few more months to the south and west;
\item The location of the capital city, Washington DC, on the border between a Confederate state (Virginia) and a Union state (Maryland) where many residents (including Booth) were pro-slavery and sympathetic to the Confederacy;
\item The lax provisions in place for protecting the President during that era;
\item The anger aroused among supporters of slavery by Abraham Lincoln’s speech of April 11, 1865, in which he announced his intention to extend voting rights to at least some African-Americans, including those who had fought for the Union.
\end{enumerate}

By asserting that Lincoln’s assassination would take place in \textit{any} world which shares these properties (and perhaps others) with the real world, sentence \REF{ex:} seems to imply that the assassination was inevitable.



There is much more to be said about counterfactuals, but further discussion would be beyond the scope of the present book. We turn now to another use of the conditional sentence pattern, which we will argue contributes use-conditional rather than truth-conditional meaning.


\section{7. Speech Act conditionals}\label{sec:}

Relevance conditionals are often referred to as \textsc{speech act conditionals}, and in this section we try to understand why this label is appropriate. Let us begin by considering how a relevance conditional is used. As we noted in \sectref{sec:2}, relevance conditionals like those in \REF{ex:} commit the speaker to believing that the consequent is true; and this raises the question of why a speaker who believes \textit{q} would choose to say \textit{if p then q} rather than just \textit{q}?


\ea
\ea If you are hungry, there’s some pizza in the fridge.\\
\ex If you need anything, my name is Arnold.\\
\ex I am planning to watch Brazil vs. Argentina tonight, if you are interested.\\
\ex You look like you need to sit down, if you don’t mind my saying so.
                       \z
\z


One important function of the \textit{if} clause in such cases is to prevent unintended implicatures from arising and/or guide the hearer toward the intended implicature.\footnote{\citet{DeRoseGrandy1999}, \citet{Franke2008}.} If the speaker in (\ref{ex:}a) simply announces \textit{There’s some pizza in the fridge}, in a context where the topic of conversation is something other than left-over food, the comment will seem irrelevant. This could lead the hearer, who assumes that the speaker is observing the Maxim of Relevance (see \chapref{sec:8}), to seek an implicature which renders the statement relevant. But the context may not be adequate for the hearer to succeed in this attempt. (Was I supposed to clean the fridge? Is this fridge only supposed to be used for bio-medical supplies?) The conditional clause functions first as a relevance hedge, warning the hearer that the statement which follows may not be relevant if certain conditions do not hold. The conditional clause also serves to guide the hearer toward the intended implicature: in this example, the statement \textit{There’s some pizza in the fridge} is intended as an indirect speech act, specifically an offer or invitation to have something to eat.



Similarly, the \textit{if} clause in (\ref{ex:}b) helps the hearer to correctly interpret the assertion in the consequent as an offer to be of service, rather than (for example) an initiation of mutual introductions. The \textit{if} clause in (\ref{ex:}c) helps the hearer to correctly interpret the consequent as an invitation to watch a soccer match.



The term \textsc{relevance conditional} reflects what is perhaps the most common function of the \textit{if} clause in this construction, namely to specify the conditions under which the assertion in the consequent will be relevant. Now relevance is one of the felicity conditions for making an assertion; so the conditional clause is used by the speaker to avoid making an infelicitous assertion. The \textit{if} clause in (\ref{ex:}d) (\textit{if you don’t mind my saying so}) functions as a politeness hedge, rather than a relevance hedge; but the basic function is again to avoid making an infelicitous assertion.



An important feature of relevance conditionals is that the consequent need not be an assertion at all; other speech acts are possible as well. The examples below show that the consequent of a relevance conditional may be a command (\ref{ex:}a) or a question (\ref{ex:}b-c). 


\ea
\ea If you want my advice, ask her to marry you right away.\\
\ex If you have heard from Michael recently, how is he doing?\\
\ex What did you do with that left-over pizza, if you don’t mind my asking?
                       \z
\z


Once again, the \textit{if} clause in such examples refers to the felicity conditions for performing the speech act expressed by the consequent. One of the felicity conditions for asking a question is that the speaker believes that the hearer has access to the information being requested. The \textit{if} clause in (\ref{ex:}b) specifies a condition under which it is reasonable to expect that the addressee will know something about Michael’s current situation. The \textit{if} clauses in (a,c) seem to address the preparatory conditions for commands and questions, respectively, which include the relationship between speaker and hearer, and the degree to which the speaker feels free to advise or ask the hearer on a particular topic.



In view of the fact that this construction can be used to hedge a variety of felicity conditions, and not just relevance, the more general term \textsc{speech act conditionals} seems quite appropriate. This label also suggests that these conditional clauses may function as speech act modifiers, similar to the speech act adverbials we discussed in \chapref{sec:11}. This hypothesis is supported by the fact that the conditional relation between the two clauses can be questioned with standard conditionals, but not with speech act conditionals.



There is an important difference between relevance conditionals that contain questions, like that in (\ref{ex:}b), vs. “questions about conditionals”, illustrated in (\ref{ex:}a).\footnote{This point is made by Van der \citet{Auwera1986}, which is also the source of the examples in \REF{ex:}.}


\ea
\ea  Q: If you inherit, will you invest?\\
A: Yes, if I inherit, I will invest.
\ex  Q: If you saw John, did you talk to him?\\
A: Yes, I talked to him.\\
A: \#Yes, if I saw John, I talked to him.
\z \z


In questions about conditionals (i.e., a standard conditional within an interrogative sentence), the conditional meaning is part of what is being questioned. Therefore it is natural and appropriate to include the conditional clause in the answer, as seen in (\ref{ex:}a). In a speech act conditional that contains a question, however, the conditional meaning is not part of what is being questioned. Rather, the \textit{if} clause specifies a condition under which it would be appropriate or felicitous to ask the question. Therefore it is not appropriate to include the conditional clause in the answer, as in (\ref{ex:}b), except perhaps as a somewhat annoying joke. This contrast suggests that speech act conditionals function as illocutionary modifiers, rather than as part of the “at issue” propositional content of the sentence.



Several syntactic differences have been noted between speech act conditionals and standard conditionals.\footnote{\citet{BhattPancheva2006}.} First, speech act conditionals can only be embedded in the complements of indirect speech verbs, and not under propositional attitude verbs \REF{ex:}. Both kinds of embedding are possible for standard conditionals \REF{ex:}.


\ea
\ea John said that if you are thirsty there is beer in the fridge.\\
\ex *John believes that if you are thirsty there is beer in the fridge.
                       \z
\z

\ea
\ea John said that if he drinks too much wine that he gets dizzy.\\
\ex John believes that if he drinks too much wine that he gets dizzy.
                       \z
\z


Second, standard conditionals allow the consequent to be introduced with the pro-form \textit{then} \REF{ex:}, but speech act conditionals do not \REF{ex:}.


\ea
\ea If it does not rain, then we will eat outside.\\
\ex If I see him again, then I will invite him.
                       \z
\z

\ea
\ea \#If I may be honest, then you are not looking good.\\
\ex \#If you want to know, then 4 isn’t a prime number.\\
\ex \#If you are thirsty, then there is beer in the fridge.
                       \z
\z


Third, the word order in Dutch and German seems to indicate that standard conditionals occupy a different structural position from speech act conditionals. As we mentioned in \chapref{sec:18}, Dutch and German are “verb-second” (V2) languages. This means that in main clauses (or, more generally, clauses not introduced by a complementizer), the inflected verb or auxiliary must immediately follow the first constituent of the clause. As the Dutch examples in (\ref{ex:}--\ref{ex:}) show,\footnote{Examples (\ref{ex:}--\ref{ex:}) are originally from Iatridou (1991: ch. 2)} standard conditionals occupy the clause-initial position, causing the inflected verb to immediately follow the conditional clause. However, this is not the case with speech act conditionals. The fact that the main clause subject in \REF{ex:} must precede the verb indicates that the conditional clause is not a constituent of the main clause at all; it attaches to some higher node in the sentence.


\ea
\ea \gll [Als  Jan  weg-gaat]  ga  ik  ook  weg.\\
 if  John  away-goes  go  I  also  away\\
\glt ‘If John goes away, I will go away too.’    [\textsc{standard conditional}]
\ex  *[Als Jan weggaat] ik ga ook weg.
\z \z

\ea
\ea \gll [Als  je  het  wil  weten]  4  is  geen  priem  getal.\\
 if  you  it  want  know  4  is  no  prime  number\\
\glt ‘If you want to know, 4 is not a prime number.’  [\textsc{speech act} \textsc{conditional}]
\ex  *[Als je het wil weten] is 4 geen priem getal.
\z \z


The minimal pair in \REF{ex:} shows how word order can disambiguate standard conditionals vs. speech act conditionals in German.\footnote{The examples in \REF{ex:} are from \citet[102]{Scheffler2013}.} The main clause verb in (\ref{ex:}a) immediately follows the conditional clause, forcing it to be interpreted as a standard conditional: \textit{I will stay home only if you need me}. In contrast, the main clause verb in (\ref{ex:}b) follows its subject NP, forcing it to be interpreted as a speech act conditional: \textit{I’ll be at home all day and you can reach me there if you need me}. Again, the word order facts indicate that the standard conditional is embedded within the main clause, whereas the speech act conditional is not.


\ea
\ea  \gll\relax [Wenn  Du  mich  brauchst],  bleibe  ich  den  ganzen  Tag  zu  Hause.\\
 if  you  me  need  stay  I  the  whole  day  at  house\\
\glt ‘[If you need me], (only then) I will stay at home all day.’  [\textsc{standard conditional}]

\ex
  \gll\relax [Wenn  Du  mich  brauchst],  ich  bleibe  den  ganzen  Tag  zu  Hause.\\
 if  you  me  need  I  stay  the  whole  day  at  house\\
\glt ‘[If you need me], I’ll be at home all day (anyway).’   [\textsc{speech act} \textsc{conditional}]
\z
\z

A final difference that we will mention here concerns the potential for pronouns to function as “bound variables”. A pronoun which occurs in the antecedent clause of a standard conditional can be interpreted as being “bound” by a quantifier phrase that occurs in the consequent clause. This was seen in example \REF{ex:} above, repeated here as \REF{ex:}. However, this interpretation is not available in speech act conditionals, as illustrated in \REF{ex:}. This contrast provides additional evidence that the antecedent clause of a standard conditional is more tightly integrated into the syntax of the main clause than the antecedent clause of a speech act conditional.\footnote{See \citet{EbertEtAl2008} for similar examples in German.}


\ea
\ea{} [Every student]\textsubscript{i} will succeed if he\textsubscript{i} works hard.\\
\ex{} [No student]\textsubscript{i} will succeed if he\textsubscript{i} goofs off.
\z
\z

\ea
\ea \#[Every student]\textsubscript{i} should study trigonometry, if he\textsubscript{i} wants my opinion.\\
\ex \#[No student]\textsubscript{i} gave a very impressive speech, if he\textsubscript{i} doesn’t mind my saying so.
\z
\z

Concessive conditionals share some of these properties with relevance conditionals. For example, the concessive meaning is lost when the consequent contains \textit{then} (\ref{ex:}a), or when the conditional is embedded in the complement of a propositional attitude verb (\ref{ex:}b). But the semantic function of concessive conditionals seems quite different from that of relevance conditionals.


\ea
\ea \#If you were the last man on earth, then I would not marry you.\\
\ex \#Mary believes that if John were the last man on earth, she would not marry him.
\z
\z


Some of the similarities between concessive conditionals and relevance conditionals seem to be related to the fact that in both types, the speaker asserts that the consequent is true, without condition. This limits the kinds of inferences that can be triggered. For example, standard conditionals of the form \textit{if} \textit{p then q} typically create a generalized conversational implicature: \textit{p if and only if q}. This implicature can be explained in terms of the maxim of Quantity. If the speaker was in a position to assert that q was true, whether or not p was true, then the most informative way to communicate this fact would be to simply say \textit{q}. Saying \textit{if} \textit{p then q} is less informative, and so gives the hearer reason to infer that the speaker is not in a position to assert that q is true (\ref{ex:}a). However, this implicature is not triggered by relevance or concessive conditionals (\ref{ex:}b-c).


\ea
\ea If you take another step, I’ll knock you down.\\
  (implicature: If you do not take another step, I will not knock you down.)\\
\ex If you are hungry, there is some pizza in the fridge.\\
  (does not implicate: If you are not hungry, there is no pizza in the fridge.)\\
\ex I wouldn’t marry you if you were the last man on éarth.\\
  (does not implicate: I would marry you if you were not the last man on earth.)
\z
\z


We mentioned a related fact in \chapref{sec:9}, namely that the rule of \textit{modus tollens} (denying the consequent) does not hold for all uses of the English word \textit{if}. We can now see that the rule works for standard conditionals (\ref{ex:}a), but not for relevance or concessive conditionals (\ref{ex:}b-c).


\ea
\ea Mother said that if her meeting was cancelled, she would come home; but she’s\\
  not home, so I guess her meeting was not cancelled.\\
\ex Mother says that if we are hungry, there’s some pizza in the fridge; but there’s\\
  no pizza in the fridge, \#so I guess we are not hungry.\\
\ex I wouldn’t marry that man (even) if he became a millionaire; \#so if I end up marrying\\
  him, you will know that he did not become a millionaire.
                       \z
\z


It seems natural to ask whether the analysis we outlined in \sectref{sec:5} for standard conditionals can be extended to account for speech act conditionals as well? In \chapref{sec:18} we analyzed the contrast between truth-conditional vs. speech act uses of \textit{because} as a case of “pragmatic ambiguity”: a single sense used in two different ways. In the truth-conditional use (\ref{ex:}a), \textit{because} indicates a causal relation between two propositions. In the speech act use, \textit{because} indicates a causal relation between the truth of a proposition and the performance of a speech act. We might paraphrase (\ref{ex:}b) as meaning something like: ‘Because I would like to come and visit you, I hereby ask you whether you are going out tonight.’


\ea
\ea Mary scolded her husband because he forgot their anniversary again.\\
\ex Are you going out tonight, because I would like to come and visit you.
                       \z
\z


A somewhat parallel approach to speech act conditionals is possible. Our discussion at the beginning of this section suggests that the antecedent of a speech act conditional specifies a condition under which the speech act performed in the consequent will be felicitous, whereas the antecedent in standard conditionals specifies a condition under which the proposition expressed in the consequent will be true.


\section{Conclusion}\label{sec:} %8. /

We began with the intuition that in a conditional sentence \textit{if p} (\textit{then) q}, the \textit{if} clause describes some condition under which the \textit{then} clause will be true. We noted that modals have a somewhat similar function, in that modal operators (in particular, modal markers of necessity) specify sets of possible worlds in which the basic proposition will be true. In \chapref{sec:16} we analyzed modals as quantifiers over possible worlds, and it seems plausible that a similar approach might work for conditionals as well.



A quantificational analysis of conditionals is further supported by the observation that, when the consequent clause in a conditional sentence contains a quantifier-type expression (e.g. \textit{all}, \textit{usually}, \textit{should}, etc.), the word \textit{if} seems to have no independent meaning. Rather, the antecedent of the conditional is added to the restriction of the quantifier, as illustrated in (\ref{ex:}--\ref{ex:}) above. When there is no overt quantifier in the consequent, the meaning of the conditional sentence can generally be well paraphrased in terms of epistemic necessity or (given the appropriate tense marking on the consequent’s verb) generic frequency.



This kind of quantificational analysis for conditionals seems to work well for hypothetical conditionals, but other uses of the conditional form present additional challenges. In the case of counterfactuals, some more elaborate means seems to be required to restrict the set of relevant possible worlds. In the case of speech act conditionals, the issue does not seem to be the truth of the consequent but the felicity or appropriateness of the associated speech act. Whether all the various uses of \textit{if} can be unified under a single sense remains an open and much-discussed question.



\furtherreading



Von \citet{Fintel2011} provides a good introduction to the study of conditionals, including a summary of much recent work on the topic. \citet{Comrie1986} offers a useful typological study of the construction. \citet{Kratzer1986} provides a very clear and readable argument for her restrictor analysis. Kearns (2000:61–64) provides a brief and helpful introduction to the analysis of counterfactual conditionals, and von \citet{Fintel2012} provides an excellent overview of the topic. \citet{BhattPancheva2006} discuss the syntactic structure of conditionals and how the structure relates to the meaning. They also present a good discussion of the various uses of \textit{if}.


\subsubsection{Discussion exercises:}\label{sec:}
\paragraph{A: Types of conditional}

Identify the type of conditional expressed in each of the following sentences. Use one of the following labels: \textsc{standard, relevance, concessive}, or \textsc{factual}; and for standard conditionals, add one of the following: \textsc{reality, hypothetical, counterfactual}.

\begin{enumerate}
\item \itshape
I wouldn’t eat that stew if you paid me.
\item \itshape
If you place your order now, I will include the batteries for free.
\item \itshape
If you have no money, where did you get all this electronic equipment?
\item \itshape
If wishes were horses, beggars would ride.
\item \itshape
I just told you that I have a meeting with a client this evening. And if I have a meeting with a client, there is no way I can go to the game with you.
\item \itshape
If you like seafood, there is a great restaurant down by the harbor.
\item \itshape
If you had waited for me, I would have married you.
\item \itshape
I’ll show you the agenda if you promise not to tell anyone.
\end{enumerate}
\paragraph{B: Restrictor analysis}

Use the restricted quantifier notation to express the interpretation of the following sentences, omitting the words in parentheses:

\begin{enumerate}
\item 
\textit{Few boxers are famous if they lose.}
\item \itshape
Subtitles are often funny if they are mistranslated.
\item 
\textit{John must pass Greek if he drops Hebrew}.
\item 
\textit{If the Bishop was preaching, we used to be late (for Sunday dinner).}
\end{enumerate}
\subsubsection{Homework exercises:}\label{sec:}
\paragraph{A: Types of conditional}

Show how you could use some of the tests discussed in \chapref{sec:19} to determine whether the conditional clauses in the following examples conditional are \textsc{standard} conditionals or \textsc{speech act} conditionals.

\ea \ea \itshape If you want my advice, I will do some research and send you an e-mail.\\
\ex If you want my advice, Arnold is not the right man for you.
\z \z

\paragraph{B: Restrictor analysis}

(i) Use the restricted quantifier notation to express the interpretation of the following sentences:

\begin{enumerate}
\item \itshape
Most students are happy if they pass.
\item \itshape
If the light is on, Arthur must be at home.
\item \itshape
If it rains, I drive to work.
\end{enumerate}
\ea
(ii) Use the restricted quantifier notation to express the two possible interpretations for the following sentence:\\
\textit{Arthur may not visit Betty if she insults him.}
\z

\chapter{{20}: Aspect and \textit{Aktionsart}}

\section{Introduction}\label{sec:} %1. /

In this final unit of the book we look at the meanings of grammatical morphemes that mark tense and aspect. Tense and aspect markers both contribute information about the time of the event or situation being described. Broadly speaking, tense markers tell us something about the situation’s location in time, as illustrated in \REF{ex:}, while aspect markers tell us something about the situation’s distribution over time, as illustrated in \REF{ex:}. 


\textbf{Lithuanian tense marking} (Chung and \citealt{Timberlake1985}:204)

\begin{tabularx}{\textwidth}{XXX}
\lsptoprule
a. & dirb-\textit{au}\\
work-1sg\textsc{.past} & ‘I worked/ was working’\\
b. & dirb-\textit{u}\\
work-1sg\textsc{.present} & ‘I work/ am working’\\
c. & dirb-\textit{s}-iu\\
work-\textsc{future-}1sg & ‘I will work/ will be working’\\
\lspbottomrule
\end{tabularx}
\ea
\textbf{Aspect marking in English}:\\
\ea When I got home from the hospital, my wife \textit{wrote} a letter to my doctor.\\
                       \z
\ex When I got home from the hospital, my wife \textit{was writing} a letter to my doctor.
\z


As we will see, many of the same issues that we encountered in our study of word meanings are also relevant to the study of tense and aspect markers: distinguishing entailments from selectional restrictions and other presuppositions; implicature and coercion as sources of new meanings; potential for polysemy and idiomatic senses; etc.



This chapter focuses on aspect, while the next chapter looks at tense. We begin in \sectref{sec:2} with a discussion of \textsc{situation type}, sometimes referred to as \textsc{situation aspect} or \textit{Aktionsart} (German for ‘action type’). It turns out that situation type, e.g. the difference between events vs. states, can have a significant effect on the interpretation of both tense and aspect markers.



In \sectref{sec:3} we introduce the notion of \textsc{Topic Time}, the time under discussion, which will play an important role in our approach to both tense and aspect. \sectref{sec:key:4} discusses grammatical aspect, exploring the kinds of aspectual meaning that are most commonly distinguished by grammatical markers across languages. Sections 5 and 6 explore some of the ways that situation type (\textit{Aktionsart}) and grammatical aspect interact with each other.


\section{2. Situation type (\textit{Aktionsart})}\label{sec:}

Before we think about the kinds of meanings that tense and aspect markers can express, we need to think first about the kinds of situations that speakers may want to describe. We can divide all situations into two basic classes, \textsc{states} vs. \textsc{events}. (This is why we speak of “situation type” rather than “event type”; we need a term that includes states as well as events.)\footnote{The word \textit{eventuality} is sometimes used as an alternative to \textit{situation}.} Informally we might define events as situations in which something “happens”, and states as situations in which nothing happens.



Roughly speaking, if you take a video of a state it will look like a snapshot, because nothing changes; but if you take a video of an event, it will not look like a snapshot, because something will change. In more precise terms we might define a state as a situation which is homogeneous over time: it is construed as being the same at every instant within the time span being described. Examples of sentences which describe stative situations include: \textit{this tea is cold}; \textit{my puppy is playful}; \textit{George is my brother}. Of course, to say that a state is a situation in which nothing changes does not mean that these situations will never change. Tea can be re-heated, puppies grow up, etc. It simply means that such changes are not part of the situation currently being described.



Conversely, we can define an \textsc{event} as a situation which is not homogeneous over time, i.e., a situation which involves some kind of change. In more technical terminology, events are said to be \textsc{dynamic}, or internally complex. Examples of sentences which describe eventive situations include: \textit{my tea got cold}; \textit{my puppy is playing}; \textit{George hit my brother; Susan will write a letter}.



In classifying situations into various types, we are interested in those distinctions which are linguistically relevant, so it is important to have linguistic evidence to support the distinctions that we make.\footnote{It turns out that situation type plays an important role in syntax as well as semantics.} A number of tests have been identified which distinguish states from events. For example, only sentences which describe eventive situations can be used appropriately to answer the question \textit{What happened?}\footnote{Jackendoff (\citeyear{Jackendoff1976}: 100, \citeyear{Jackendoff1983}: 179).} Applying this test leads us to conclude that sentences (\ref{ex:}a-d) describe eventive situations while sentences (e-h) describe stative situations.


\ea
What happened was that…\\
\ea Mary kissed the bishop.\\
\ex the sun set.\\
\ex Peter sang Cantonese folk songs.\\
\ex the grapes rotted on the vine.\\
\ex *Sally was Irish.\\
\ex *the grapes were rotten.\\
\ex *William had three older brothers.\\
\ex *George loved sauerkraut.
                       \z
\z


A second test is that only eventive situations can be naturally described using the progressive (\textit{be V-ing}) form of the verb, although with some states the progressive can be used to coerce a marked interpretation. This test indicates that sentences (\ref{ex:}a-c) describe eventive situations while sentences (\ref{ex:}d-g) describe stative situations. Sentences (h-i) involve situations which, based on other evidence, we would classify as stative. Here the progressive is acceptable only with a special, coerced interpretation: (h) is interpreted to mean that this situation is temporary and not likely to last long, while (i) is interpreted to mean that Arthur is behaving in a certain way (an eventive interpretation). In some contexts (\ref{ex:}e) might be acceptable with a coerced interpretation like that of (i).


\ea
\ea Mary is kissing the bishop.\\
\ex The sun is setting.\\
\ex Peter is singing Cantonese folk songs.\\
\ex *This room is being too warm.\\
\ex *Sally is being Irish.\\
\ex *William is having a headache.\\
\ex *George is loving sauerkraut.\\
\ex George is loving all the attention he is getting this week.\\
\ex Arthur is being himself.
                       \z
\z


A third test is that in English, eventive situations described in the simple present tense take on a \textsc{habitual} interpretation, whereas no such interpretation arises with states in the simple present tense. For example, (\ref{ex:}c) means that Peter is in the habit of singing Cantonese folk songs; he does it on a regular basis. In contrast, (\ref{ex:}e) does not mean that William gets headaches frequently or on a regular basis; it is simply a statement about the present time (=time of speaking). This test indicates that sentences (\ref{ex:}a-c) describe eventive situations while sentences (\ref{ex:}d-e) describe stative situations.


\ea
\ea Mary kisses the bishop (every Saturday).\\
\ex The sun sets in the west.\\
\ex Peter sings Cantonese folk songs.\\
\ex This room is too warm.\\
\ex William has a headache.
                       \z
\z


Some authors have cited certain tests as evidence for distinguishing state vs. event, which in fact are tests for agentive/volitional vs. non-agentive/non-volitional situations. For example, only agentive/volitional situations can normally be expressed in the imperative; be modified by agent-oriented adverbials (e.g. \textit{deliberately}); or appear as complements of Control predicates (\textit{try, persuade, forbid}, etc.). It turns out that most states are non-agentive, but not all non-agentive predicates are states (e.g. \textit{die, melt, fall}, \textit{bleed}, etc.). Moreover, some stative predicates can occur in imperatives or control complements (\textit{Be careful! He is trying to be good. I persuaded her to be less formal.}), indicating that these states are at least potentially volitional. It is important to use the right tests for the right question.



A second important distinction is between \textsc{telic} vs. \textsc{atelic} events. A telic event is one that has a natural endpoint. Examples include dying, arriving, eating a sandwich, crossing a river, and building a house. In each case, it is easy to know when the event is over: the patient is dead, the sandwich is gone, the house is built, etc.



Many telic events (e.g. \textit{build}, \textit{destroy}, \textit{die}, etc.) involve some kind of change of state in a particular argument, generally the patient or theme. This argument “measures out” the event, in the sense that once the result state is achieved, the event is over.\footnote{The term “measures out” comes from \citet{Tenny1987}. \citet{Dowty1991} uses the term “incremental theme” for arguments that “measure out” the event in gradual/incremental stages, so that the state of the incremental theme directly reflects the progress of the event.} Some telic events are measured out by an argument that does not undergo any change of state, e.g. \textit{read a novel}: when the novel is half read, the event is half over, but the novel does not necessarily change in any way. Other telic events are measured out or delimited by something which is not normally expressed as an argument at all, e.g. \textit{run five miles}, \textit{fly to Paris}, \textit{drive from Calgary to Vancouver}, etc. Motion events like these are measured out by the path which is traversed; the progress of the theme along the path reflects the progress of the event. As \citet{Dowty1991} points out, with many such predicates the path can optionally be expressed as a syntactic argument: \textit{swim the English channel, ford the river, hike the Annapurna Circuit, drive the Trans-Amazonian Highway}, etc.



Atelic events are those which do not have a natural endpoint. Examples include singing, walking, bleeding, shivering, looking at a picture, carrying a suitcase, etc. There is no natural part of these events which constitutes their end point. They can continue indefinitely, until the actor decides to stop or something else intervenes to end the event. Atelic events do not involve a specified change of state, and no argument “measures them out”.



\citet{Dowty1979} identifies several tests which distinguish telic vs. atelic events. The two most widely used are illustrated in (\ref{ex:}--\ref{ex:}). A description of an atelic event can naturally be modified by time phrases expressing duration, as in \REF{ex:}; this is unnatural with telic events. In contrast, a description of a telic event can naturally be modified by time phrases expressing a temporal boundary, as in \REF{ex:}; this is unnatural with atelic events.


\ea
For ten minutes Peter…\\
\ea sang in Cantonese.\\
\ex chased his pet iguana.\\
\ex stared at the man sitting next to him.\\
\ex *broke three teeth.\\
\ex *recognized the man sitting next to him.\\
\ex *found his pet iguana.
                       \z
\z

\ea
In ten minutes Peter…\\
\ea ??sang in Cantonese.  (could only mean, ‘In ten minutes Peter began to sing…’)\\
\ex *chased his pet iguana.\\
\ex *stared at the man sitting next to him.\\
\ex broke three teeth.\\
\ex recognized the man sitting next to him.\\
\ex found his pet iguana.
                       \z
\z


Situation Aspect is sometimes referred to as “lexical aspect”, because certain verbs tend to be associated with particular situation types. For example, \textit{die} and \textit{break} are inherently telic, whereas \textit{chase} and \textit{stare} are fundamentally atelic. However, in many sentences the whole VP (and sometimes the whole clause) helps to determine the situation type which is being described. For example, with many transitive verbs the telicity of the event depends on whether or not the object NP is quantified or specified in some way: \textit{eat ice cream} is atelic, but \textit{eat a pint of ice cream} is telic; \textit{sing folk songs} is atelic, but \textit{sing “The Skye boat song”} is telic. Similarly, as noted above, the telicity of motion events may depend on whether or not the path is delimited in some way: \textit{walk} is atelic, but \textit{walk to the beach} is telic.



Based on the two distinctions we have discussed thus far, we can make the following classification of situation types:


\ea \begin{forest}
[Types of situations/eventualities
 [Event
  [Telic (bounded)] [Atelic (unbounded)]
 ] [State]
]     
    \end{forest}
\z


A third distinction which will be important is that between \textsc{durative} vs. \textsc{punctiliar} (=instantaneous) situations. Durative situations are those which extend over a time interval (singing, dancing, reading poetry, climbing a mountain), while punctiliar situations are those which are construed as happening in an instant (recognizing someone, reaching the finish line, snapping your fingers, a window breaking). One test that can help in making this distinction is that punctiliar situations described in the progressive (\textit{He is tapping on the door/blinking his eyes}/etc.) normally require an iterative interpretation (something that happens repeatedly, over and over). This is not the case with durative situations (\textit{He is reading your poem/climbing the mountain}/etc.).



Five major situation types are commonly recognized, and these can be distinguished using the three features discussed above as shown in \REF{ex:}.\footnote{The first four of these types are well known from the work of \citet{Dowty1979} and \citet{Vendler1957}. The Semelfactive class was added by Smith (1991/1997), based on \citet[42]{Comrie1976}.} Activities are atelic events such as \textit{dance, sing, carry a sword, hold a sign}, etc. Achievements are telic events (normally involving a change of state) which are construed as being instantaneous: \textit{break, die, recognize, arrive, find}, etc. Accomplishments are durative telic events, meaning that they require some period of time in order to reach their end-point. Accomplishments often involve a process of some kind which results in a change of state. Examples include \textit{eat a pint of ice cream, build a house, run to the beach, clear a table}, etc. Semelfactives are instantaneous events which do not involve any change of state: \textit{blink, wink, tap, snap, clap, click}, etc. Although they are punctiliar, they are considered to be atelic because they do not involve a change of state and nothing measures them out.


\textbf{Aktionsart} (situation types) (\citealt{Smith1997}:3)

\begin{tabularx}{\textwidth}{XXXX}
\lsptoprule

\bfseries\scshape Situations & \bfseries\scshape Static & \bfseries\scshape Durative & \bfseries\scshape Telic\\
State & + & + & –\footnotemark{}\\
Activity & – & + & –\\
Accomplishment & – & + & +\\
Achievement & – & – & +\\
Semelfactive & – & – & –\\
\lspbottomrule
\end{tabularx}
\footnotetext{Smith leaves the telicity of states unspecified, because it is not contrastive; here I follow Van Valin \& La\citet[93]{Polla1997} in specifying states as atelic.}

For some purposes it is helpful to make a further distinction between two kinds of states: stage-level (temporary) vs. individual-level (permanent).\footnote{\citet{Carlson1977}, \citet{Kratzer1995}.} We will refer to these situation types often in our discussion of the meanings of tense and aspect markers. But first we begin that discussion by identifying three “cardinal points” for time reference: the time of speaking, the time of situation, and “topic time”.


\section{Time of speaking, time of situation, and “topic time”}\label{sec:} %3. /

Tense markers are often described as “locating” a situation in time, as seen in the following widely-cited definitions of tense \REF{ex:}:


\ea
\ea  “\textbf{\textsc{Tense}} is grammaticalised expression of location in time… [T]enses locate situations either at the same time as the present moment…, or prior to the present moment, or subsequent to the present moment.” (\citealt{Comrie1985}:9, 14)
\ex  “\textbf{\textsc{Tense}} refers to the grammatical expression of the time of the situation described in the proposition, relative to some other time.” \citep{Bybee1985}
\z \z


These definitions state that tense markers specify the time of a situation relative to some other time, generally the “present moment” (= the time of speaking). However, as \citet{Klein1994} points out, examples like the following seem to pose a problem for the claim that tense “locates situations in time”:


\ea
\ea  I took a cab back to the hotel. \textit{The cab driver was Latvian}. (\citealt{Michaelis2006})
\ex They found John in the bathtub. \textit{He was dead}.  (\citealt{Klein1994}:22)
\ex  Tuesday morning we ate leftovers from Chili’s for breakfast and checked out of the Little America Hotel… \textit{The Grand Canyon was enormous}. We walked along the rim taking pictures amazed at how beautiful and massive the canyon is. [http://scottnmegan.blogspot.com/2009/04/arizona-part-2.html]
\z \z


If the past tense in the italicized portions of these examples indicates that the described situation is located prior to the time of speaking, does that mean that the cab driver was no longer Latvian at the time of speaking, or that John was no longer dead at the time of speaking, or that the Grand Canyon was no longer enormous at the time of speaking? In light of examples like these, Klein suggests that tense actually locates or restricts the speaker’s \textsc{assertion}, rather than locating the situation itself. That is, tense indicates the location of the time period about which the speaker is making a claim.



Klein uses the term \textsc{Topic Time} to refer to the time period about which the speaker is making a claim, or in his words, “the time span to which the speaker’s claim on this occasion is confined” (1994:4). This choice of terminology builds on the widely used definition of “Topic” as “what we are talking about.” So Topic Time is the time span that we are talking about. Klein distinguishes Topic Time (TT) from the two other significant times mentioned above: TSit, the time of the event or situation which is being described; and TU, the Time of Utterance (=time of speaking).\footnote{As we will discuss in \chapref{sec:21}, Klein’s framework is based on a proposal by Reichenbach (1947, § 51).}



The Topic Time can be specified by time adverbs like \textit{yesterday} or \textit{next year}, or by temporal adverbial clauses as seen in example \REF{ex:} above (\textit{When I got home from the hospital}). It can also be determined by the context. For example, in a narrative sequence like that in (\ref{ex:}c), the Topic Time is partly determined by the clause’s position in the sequence. Event-type verbs in the simple past tense move the Topic Time forward, whereas stative predicates in the simple past tense inherit the Topic Time from the previous main-line event. The italicized portion of that example makes an assertion only about the Topic Time at that stage of the narrative; no assertion is made about the Time of Utterance.



\citet[4]{Klein1994} describes an imaginary mini-dialogue between a judge and a witness in a courtroom. He points out that the second sentence of the witness’s reply cannot be felicitously expressed in the present tense, even though the book in question is presumably still in Russian at the time of speaking. That is because the judge’s question establishes a specific topic time (\textit{when you looked into the room}) prior to the time of the current speech event, and any felicitous reply must be relevant to the same topic time.


\ea
Judge: What did you notice when you looked into the room?\\
Witness: There was a book on the table. \textit{It was/\#is in Russian}.   (\citealt{Klein1994}:4)
\z


Klein assumes that the values of TSit and TT are time intervals, rather than simple points in time, whereas TU can be treated as a point. Using these three concepts, Klein defines tense and aspect as follows:


\ea
\ea \textsc{Tense} indicates a temporal relation between TT and TU;\\
\ex \textsc{Aspect} indicates a temporal relation between TT and TSit.
\z \z


We can illustrate Klein’s definition of aspect using the examples in \REF{ex:}, repeated here as \REF{ex:}. As noted above, the temporal adverbial clause in these examples (\textit{When I got home from the hospital}) specifies the location of Topic Time. The duration of Topic Time in this case seems to be somewhat vague and context-dependent, influenced partly by our knowledge of how long it takes to write a letter. The use of \textsc{perfective} aspect in (\ref{ex:}a) indicates that the writing of the letter occurred completely within Topic Time. Under the most natural interpretation, the writing began after the speaker arrived home, and was completed shortly thereafter. The use of \textsc{imperfective} aspect in (\ref{ex:}b) indicates that the writing of the letter extended beyond the limits of Topic Time. Under the most natural interpretation, the writing began before the speaker arrived home, and may not even be completed at the time of speaking.\footnote{The terms \textsc{perfective} and \textsc{imperfective} will be defined more carefully in \sectref{sec:4} below.}


\ea
\ea When I got home from the hospital, my wife \textit{wrote} a letter to my doctor.\\
\ex When I got home from the hospital, my wife \textit{was writing} a letter to my doctor.
                       \z
\z


We will discuss Klein’s definition of tense in \chapref{sec:21}. In the remainder of this chapter we focus on aspect.


\section{4. Grammatical Aspect (= “viewpoint aspect”)}\label{sec:} 

Situation type (\textit{Aktionsart}) is an inherent property of the situation itself. Grammatical aspect is a feature of the speaker’s description of the situation, i.e., a part of the claim that is being made about the situation under discussion. Grammatical aspect is sometimes referred to as \textsc{viewpoint aspect}, reflecting the intuition that grammatical aspect markers indicate something about the way the speaker chooses to view or describe the situation, rather than some property of the situation itself.



This intuition is reflected in some widely cited definitions of aspect. \citet[3]{Comrie1976}, for example, says: “Aspects are different ways of viewing the internal temporal constituency of a situation.” Smith (1991/1997:2–3) states: “Aspectual viewpoints present situations with a particular perspective or focus, rather like the focus of a camera lens. Viewpoint gives a full or partial view of the situation talked about.” Using Smith’s metaphor of the camera lens, we could describe \textsc{perfective} aspect as a wide angle view: the situation fits inside the time frame of the speaker’s perspective. The \textsc{imperfective} is like a zoom or close-up view, focusing on just a part of the situation being described, with the situation as a whole extending beyond the boundaries of the speaker’s perspective.



Both of these definitions are helpful, but they may tend to obscure a very important point about the nature of grammatical aspect, namely that grammatical aspect markers contribute to the truth conditions of the sentence. For example, sentences (\ref{ex:}a--b) differ only in their aspect. Both are marked for past tense, but (\ref{ex:}b) is marked for \textsc{imperfective} aspect while (\ref{ex:}a) involves \textsc{perfective} aspect. If spoken in the year 2010, (\ref{ex:}b) would (reportedly) be true while (\ref{ex:}a) would be false, due to the intervention of a neighboring country. So different aspect markers represent different claims about the world.


\ea
\ea The Syrians \textit{built} a nuclear weapon with North Korean technology.\\
\ex The Syrians \textit{were building} a nuclear weapon with North Korean technology.
                       \z
\z


Klein’s definition of aspect, which was mentioned in the previous section, reflects this insight by relating the time structure of the situation not to the speaker’s perspective, but to the time about which a claim is being asserted (Topic Time): aspect indicates a temporal relation between TT and TSit. As a first approximation, we can define \textsc{perfective} aspect as indicating that the situation time fits inside Topic Time (TSit ${\subseteq}$ TT); and \textsc{imperfective} aspect as indicating that Topic Time fits completely inside situation time (TT ${\subset}$ TSit). These are objective claims about the relationship between two time intervals, which can be evaluated as being true or false in a particular situation.



To take another example, the temporal adverbial clause in (\ref{ex:}a--b) establishes the topic time for the main clause in each sentence. The imperfective form of the main clause in (\ref{ex:}a) indicates that the topic time is completely contained within the situation time. In other words, the boundaries (and in particular the end point) of TSit, the “digging a tunnel” event, extend beyond the boundaries of TT, the time during which the guards were at the Christmas party. For this reason, the imperfective description of the event in (\ref{ex:}a) may be true even if the tunnel was never completed. In contrast, the perfective form of the main clause in (\ref{ex:}b) indicates that the situation time is contained within the topic time. This means that the entire “digging a tunnel” event took place within the time span of the guards attending the party.


\ea
\ea While the guards were at the Christmas party, the prisoners \textit{were digging} a tunnel \\
  under the fence (but they never finished it).\\
\ex While the guards were at the Christmas party, the prisoners \textit{dug} a tunnel \\
  under the fence (\#but they never finished it).
                       \z
\z


Digging a tunnel is a telic situation, specifically an accomplishment; so its endpoint, or culmination, is an integral part of the event. For this reason, the perfective description of the event in (\ref{ex:}b) would not be true if the tunnel was not completed. This example illustrates how an imperfective description of an event may be true in a situation in which a perfective description of that same event would be false. The diagrams in \REF{ex:} represent the relative locations of the Time of Utterance, Topic Time (time during which the guards were at the party), and Situation Time (prisoners digging a tunnel) for examples (\ref{ex:}a--b).


\ea
\ea{}         [  TT  ]  \textbf{{\textbar}}    [a; \textsc{imperfective aspect}]\\                       
  …===TSit===…  TU
  \ex{}       [  TT  ]  \textbf{{\textbar}}    [b; \textsc{perfective aspect}]\\
    \textbf{{\textbar}}=TSit=\textbf{{\textbar}}    TU
\z
\z

\subsection{Typology of grammatical aspect}\label{sec:} %4.1 /

\citet{Comrie1976} classifies the most commonly marked aspectual categories in the following hierarchy, which starts with the contrast between perfective vs. imperfective.





\ea \begin{forest}
[\textsc{aspects}
[\textsc{imperfective}
  [continuous
    [non-progressive] [progressive]
  ] [habitual]
] [\textsc{perfective}]
]     
\end{forest}
\z 


In many languages, including English, perfective is the default or unmarked way of describing an event in the past. It simply asserts that the event happened. Notice that we illustrated the perfective in examples (\ref{ex:}--\ref{ex:}) using the simple past tense; the lack of overt aspect marking indicates perfective aspect. However, aspect is distinct from tense. Many languages distinguish perfective vs. imperfective in the future (e.g. \textit{will eat} vs. \textit{will be eating}) as well as the past.



Different kinds of imperfective meaning are grammatically distinguished in some languages. \textsc{Habitual} aspect describes a recurring event or on-going state which is a characteristic property of a certain period of time.\footnote{Comrie (1976: 27–28).} Imperfective aspect which is not habitual is typically called either \textsc{continuous} or \textsc{progressive}. The difference between these two categories lies in their selectional restrictions, rather than in their entailments. The term \textsc{progressive} is generally applied to non-habitual imperfective markers that are used only for describing events, and not for states. Comrie uses the term \textsc{continuous} for non-habitual imperfective aspect markers that are not restricted in this way, but can be used for both states and events. In some languages, however, the term \textsc{continuous} is applied to aspect markers that are used primarily for states.



English does not have a general imperfective aspect marker. The \textit{be + V-ing} form illustrated in (\ref{ex:}a) is specifically progressive in meaning. Habitual meaning can be expressed using the simple present tense as in (\ref{ex:}b), or (for habituals in the past) with the auxiliary \textit{used to} as in (\ref{ex:}c).


\ea
\ea Mary is playing tennis.\\
\ex Mary plays tennis.\\
\ex Mary used to play tennis.
                       \z
\z


Spanish does have a general imperfective form as well as a more specific progressive. The imperfective form is ambiguous between habitual vs. continuous meaning, as illustrated in (\ref{ex:}b).\footnote{\citealt{Comrie1976}: 25.}


\ea
\ea \textit{Juan llegó}.  ‘Juan arrived.’  [\textsc{perfective}]\\
\ex \textit{Juan llegaba}.  ‘Juan was arriving/used to arrive.’  [\textsc{imperfective}]\\
\ex \textit{Juan estaba llegando}.  ‘Juan was arriving.’  [\textsc{progressive}]
                       \z
\z

\subsection{Imperfective aspect in Mandarin Chinese}\label{sec:} %4.2 /

The Mandarin imperfective aspect markers \textit{zài} ‘progressive’ and \textit{–zhe} ‘continuous’ are often cited as a paradigm example of Comrie’s distinction between continuous and progressive aspects. The most important difference between the two morphemes lies in the types of situations that each one can modify. \textit{Zài} occurs only with events (\ref{ex:}a); it cannot be used to mark states (\ref{ex:}b). In main clauses, -\textit{zhe} is used primarily for states (\ref{ex:}a--b), and is generally unacceptable with events (\ref{ex:}c), though there appears to be some dialect variation in this regard.\footnote{See for example \citet[738]{KleinEtAl2000}, ex. 10. Also, \citet{LiThompson1981} state that the combination of \textit{–zhe} plus final particle \textit{ne} has a distinct sense and can be used with events.}

\ea
\ea \gll  Zh\=angs\=an  zài  tiào.\\
Zhangsan  \textsc{prog}  jump\\
\glt ‘Zhangsan is jumping.’   [\citealt{LiThompson1981}:222]
\ex \gll  *Wo  zài  x[1D0?]hu\=an  Měiguó.\\
  1sg  \textsc{prog}  like  America\\
\glt (intended: ‘I am liking America.’)  (\citealt{Sun2011}:90)
\z \z

\ea
\ea \gll  Ch\=ezi  zài  wàimian  tíng-zhe.\\
car  at  outside  remain-\textsc{cont} \\
\glt ‘The car is parked outside.’  [\citealt{LiThompson1981}:220]
\ex \gll  T\=a  zài  chuáng-shàng  t[1CE?]ng-zhe.\\
3sg  at  bed-on  lie-\textsc{cont}\\
\glt ‘He is lying on the bed.’  [\citealt{LiThompson1981}:220]
\ex \gll  *Zh\=angs\=an  tiào-zhe.\\
  Zhangsan  jump-\textsc{cont}\\
\glt (intended: ‘Zhangsan is jumping.’)   [\citealt{LiThompson1981}:222]
\z \z


Some verbs allow both a stative and an eventive sense. For example, \textit{chu\=an} can mean either ‘wear’ or ‘put on’; \textit{ná} can mean either ‘hold’ or ‘pick up’. In such cases, \textit{zài} selects the eventive reading and \textit{–zhe} the stative.


\ea
\ea \gll  T\=a  zài  chu\=an  pí-xié.\\
3sg  \textsc{prog}  wear  leather-shoe\\
\glt ‘He is putting on leather shoes.’  [\citealt{LiThompson1981}:221]
\ex \gll T\=a  chu\=an-zhe  pí-xié.\\
3sg  wear-\textsc{cont}  leather-shoe\\
\glt ‘He is wearing leather shoes.’  [\citealt{LiThompson1981}:221]
\z \z

\ea
\ea \gll  T\=a  zài  ná  bàozh[1D0?].\\
3sg  \textsc{prog}  hold  newspaper\\
\glt ‘He is picking up a/the newspaper.’  [\citealt{LiThompson1981}:220]
\ex \gll  T\=a  ná-zhe  bàozh[1D0?].\\
3sg  hold-\textsc{cont}  newspaper\\
\glt ‘He is holding a/the newspaper.’  [\citealt{LiThompson1981}:220]
\z \z


\citet{Yeh1993} and a number of subsequent authors have noted that only individual-level (temporary) states can be marked with \textit{–zhe}; it is generally incompatible with stage-level (permanent) states.\footnote{There also seem to be a number of idiosyncratic lexical restrictions as to which stative predicates can combine with \textit{–zhe}.}


\ea
\gll   *T\=a  c\=onghuì-zhe.\\
  3sg  intelligent-\textsc{cont}  \\
\glt (for: ‘He is intelligent.’)  (\citealt{Smith1997}:274)
\z


Although these examples have all been translated in the present tense, present time reference is not part of the meaning of either marker, as illustrated by the past time reference in \REF{ex:}.


\ea
\gll N[1D0?]  d\=angshí  mí-zhe  M[1CE?]kès\={\i},  \=Engés\={\i}  Lièníng.\\
2sg  then  fascinate-\textsc{cont}  Marx  Engels  Lenin\\
\glt ‘At that time you were fascinated by Marx, Engels and Lenin.’  (\citealt{Smith1997}:274)
\z


So far we have considered only main clause uses of these markers. In adverbial clauses like those in \REF{ex:}, \textit{–zhe} occurs freely with both stative and eventive predicates. As \citet[275]{Smith1997} notes, \textit{-zhe} is grammatically obligatory in this context; it cannot be replaced by \textit{zai}. This illustrates an important general point: the function of a tense or aspect marker in subordinate clauses may be quite different from its function in main clauses. When we are trying to determine the semantic properties of a morpheme, it may be necessary to treat these two uses separately.


\ea
\ea \gll T\=a  k\=u-zhe  p[1CE?]o  huí  ji\=a  qù  le.\\
3sg  cry-\textsc{cont}  run  return  house  go  \textsc{cos}\\
\glt ‘He ran home crying.’  [\citealt{LiThompson1981}:223]
\ex \gll  Xi[1CE?]o  g[1D2?]u  yáo-zhe  wěiba  p[1CE?]o  le.\\
small  dog  shake-\textsc{cont}  tail  run  \textsc{cos}\\
\glt ‘The little dog ran away wagging its tail.’  [\citealt{LiThompson1981}:223]
\z \z

\subsection{Perfect and prospective aspects}\label{sec:} %4.3 /

Using Klein’s terminology, we can define \textsc{perfect} (or \textsc{retrospective}) aspect as indicating that the situation time is prior to Topic Time (TSit < TT); and \textsc{prospective} aspect as indicating that the situation time is later than Topic Time (TT < TSit). The perfect in English is marked by the auxiliary \textit{have} + past participle, e.g. \textit{has eaten}, \textit{has arrived}, etc. \citet[64]{Comrie1976} suggests that the \textit{going to V} construction (e.g., \textit{the ship is going to sail}) is a way of expressing the prospective aspect in English. Other ways to express this meaning include \textit{the ship is about to sail} and \textit{the ship is on the point of sailing}.



The terms \textsc{perfect} and \textsc{perfective} are often confused, even by some linguists, but it is important to be clear about the distinction. We will discuss the perfect in some detail in \chapref{sec:22}.


\subsection{Minor aspect categories}\label{sec:} %4.4 /

A number of languages have aspect markers which refer to the “phase” of the situation being described. For example, some languages have an \textsc{inceptive} aspect, which indicates that the beginning of the situation falls within the topic time. Such markers often get translated as \textit{begin to X}. (The term \textsc{inchoative} is sometimes used for this meaning, but more commonly this term is restricted to changes of state or entering a state, e.g. \textit{to become fat, get old, get rich}, etc.) Some languages have a \textsc{terminative} or \textsc{completive} aspect, which indicates that the end of the situation falls within the topic time. \textsc{continuative} aspect would mean \textit{continue to X}, or \textit{keep on X-ing}.



\textsc{iterative} (or \textsc{repetitive}) aspect is used to refer to events which occur repeatedly. Such forms are often translated into English using phrases like \textit{over and over, more and more, here and there}, etc. \textsc{Distributive} aspect might be considered a sub-type of iterative; it indicates that an action is done by or to members of a group, one after another.\footnote{\url{http://www-01.sil.org/linguistics/GlossaryOflinguisticTerms/WhatIsDistributiveAspect.htm}} 


\section{5. Interactions between situation type (\textit{Aktionsart}) and grammatical aspect}\label{sec:}

The definitions we have adopted predict that certain grammatical aspects will not be available for certain situation types. For example, the definition of imperfective aspect as indicating that TT ${\subset}$ TSit implies that a situation expressed in the imperfective cannot be strictly punctiliar; the situation time must have some duration. A semelfactive event is construed as being instantaneous; it has no duration. For this reason, when a semelfactive event is described in the imperfective (e.g., \textit{he was tapping on the window}), it cannot be interpreted as referring to a single instance, but must receive an iterative (= repetitive) interpretation. Similarly, an instantaneous change of state cannot be described in the imperfective (e.g. ??\textit{He was recognizing his old classmate}) without some very unusual context. With other changes of state, the use of the imperfective (e.g., \textit{he was dying}) may shift the reference from the change itself to the process leading up to the change. This kind of shift can be seen as a type of coercion.



The same constraint applies to semelfactives in Mandarin. In Chinese as in English, a semelfactive event described in the imperfective cannot be interpreted as referring to a single instance, but must receive an iterative interpretation \REF{ex:}. As we would predict, this is possible only with the progressive \textit{zai}, and not with the continuous \textit{–zhe}.


\ea
\gll Zh\=angs\=an  zài  qi\=ao  mén.\\
Zhangsan  \textsc{prog}  knock  door\\
\glt ‘Zhangsan is knocking on the door.’  (\citealt{Smith1997}:272)
\z


Similarly, the definition of perfective aspect as indicating that TSit ${\subseteq}$ TT makes predictions about the kinds of situations that can appropriately be expressed in the perfective. When a state is described in the perfective aspect, what is asserted is that the state was true during the topic time, as discussed above. When an event is described in perfective aspect, what is being asserted is that the whole event took place within the topic time. For activities, which do not have an inherent endpoint, perfective descriptions in the past can be interpreted as bounded events, as in (\ref{ex:}a): ‘I played tennis \textit{for a while}.’ Alternatively, as illustrated in (\ref{ex:}b), they can get a habitual interpretation, which has properties similar to a state.


\ea
\ea I played tennis yesterday.\\
\ex I played tennis when I was in high school.
                       \z
\z


For telic events, and in particular for accomplishments, the end-point or culmination is an intrinsic part of the event; so a perfective description of that event should be false if the culmination is not in fact attained. This prediction holds true for English, as illustrated in example (\ref{ex:}b) above, and for many other languages. However, a number of languages have been identified in which this culmination is only an implicature, rather than an entailment, for accomplishments expressed in the perfective. In Tagalog, for example, it is not a contradiction to say: ‘I removed the stain, but I ran out of soap, so I couldn’t remove it.’\footnote{\citet[186]{Dell1983}.} Other languages in which such “non-culminating accomplishments” are possible include Hindi, Mandarin, Thai, several Tibeto-Burman languages, various Philippine-type languages, and at least two Salish languages.\footnote{References: Hindi (\citealt{Singh1991,Singh1998}), Mandarin (\citealt{SohKuo2005}; \citealt{KoenigChief2008}), Thai (\citealt{KoenigMuansuwan2000}), Salish (Bar-el, \citealt{DavisMatthewson2005}), Tibeto-Burman (Larin Adams, p.c.).}



The exact conditions under which “non-culminating accomplishments” can occur vary from one language to another, but the existence of such cases might suggest that we need to modify our definition of \textsc{perfective} in some way. Another alternative that we might consider starts with the recognition that accomplishments are composed of two “phases”: the first phase is a process or activity which leads to the second phase, a change of state.\footnote{\citet{KleinEtAl2000}.} In building a house, for example, the first phase would be doing the work of building and the second phase would be the coming into existence of a completed house. We might account for the difference between languages like English vs. languages like Chinese or Tagalog by recognizing that for languages of the latter type, there are certain conditions under which a VP that normally describes an accomplishment can be used to refer to just the first phase of the event, i.e. a process or activity.



This two-phase analysis also gives us a way of thinking about a puzzling fact concerning accomplishment predicates in English, which \citet{Dowty1979} refers to as the “imperfective paradox”. Building on Vendler’s (1957) discussion of these facts, Dowty points out that with state and activity predicates a statement in the imperfective (a, a) entails the corresponding statement in the perfective (b, b). With accomplishment predicates, however, this entailment does not hold (\ref{ex:}--\ref{ex:}).


\ea
\ea Arnold was wearing a wig.\\
\ex Arnold wore a wig.  [a entails b]
                       \z
\z

\ea
\ea George was speaking Etruscan.\\
\ex George spoke Etruscan.  [a entails b]
                       \z
\z

\ea
\ea Felix was writing a letter.\\
\ex Felix wrote a letter.  [a does not entail b]
                       \z
\z

\ea
\ea Sarah was running to the library.\\
\ex Sarah ran to the library.  [a does not entail b]
                       \z
\z


Dowty goes on to ask: Given the fact that accomplishments always have a natural end-point, how can the imperfective description of the event be considered true if that end-point was never achieved?\footnote{Dowty’s solution was to propose that the progressive encodes not only aspect but also modality, that is, quantification over a certain class of possible worlds. He designated the relevant class of possible worlds \textsc{inertia worlds}, which he defined as follows: an inertia world is a possible world which is exactly like the actual world under discussion up to and including the topic time, “and in which the future course of events after this time develops in ways most compatible with the past course of events” \citep[148]{Dowty1979}. In other words, inertia worlds are possible worlds in which the expected outcomes from a given situation are actually realized. Dowty then proposed a new definition of the progressive which says that \textit{John was X-ing} will be true when asserted about a time interval I just in case (i) there is some longer time interval I[2B9?] which contains I and extends beyond the end-point of I; and (ii) \textit{John X-ed} is true in all inertia worlds when asserted about time interval I[2B9?].} It seems that English, like Chinese and Tagalog, allows a shift in meaning so that a VP which normally describes an accomplishment can be used to refer to just the first phase of the event. In English, however, this shift seems to be possible only in the imperfective.


\section{6. Aspectual sensitivity and coercion effects}\label{sec:}

A predicate which normally describes one type of situation can sometimes be coerced into a different situation type (Aktionsart) by contextual factors. De \citet[360]{Swart1998} describes this process as follows:


\begin{quote}
Typically, coercion is triggered if there is a conflict between the aspectual character [i.e., Aktionsart—PK] of the eventuality description and the aspectual constraints of some other element in the context. The felicity of an aspectual reinterpretation is strongly dependent on linguistic context and knowledge of the world.
\end{quote}


In example (\ref{ex:}a), for example, a basically stative predicate (\textit{know the answer}) is coerced into a change-of-state (achievement) interpretation by the adverb \textit{suddenly}, which emphasizes the starting point of the state.\footnote{The examples in \REF{ex:} are adapted from De \citet[359]{Swart1998}.}


\ea
\ea Suddenly I knew the answer\\
\ex I read \textit{The Lord of the Rings} for a few minutes.\\
\ex John played the sonata for about eight hours.\\
\ex For months, the train arrived late.
                       \z
\z


Examples (\ref{ex:}b-c) both involve predicates which normally describe telic events (\textit{read} \textit{The Lord of the Rings} and \textit{play the sonata}), specifically accomplishments. In both cases an activity reading is coerced by an adverbial PP which specifies the duration of the event. In (\ref{ex:}b) the time span that is specified (\textit{a few minutes}) is much too short for the entire event of reading \textit{The Lord of the Rings} to be accomplished. As a result, we interpret the statement to mean that the speaker carried out a certain activity, namely reading portions of \textit{The Lord of the Rings}, for a few minutes. In (\ref{ex:}c) the time span that is specified (\textit{for about eight hours}) is much longer than it would normally take to play a sonata. The most natural interpretation is that John played the sonata over and over again for about eight hours. This iterative interpretation describes an activity, because it has no natural endpoint.



Example (\ref{ex:}d) involves a predicate (\textit{arrive}) which is both telic and instantaneous, i.e., an achievement. The instantaneous nature of the basic meaning conflicts with the long duration specified by the adverbial phrase (\textit{for months}), which results in a habitual interpretation: the train always or usually arrived late whenever it ran during those months. As mentioned above, habitual situations can be considered to be a type of state.



De \citet{Swart1998} points out that coercion effects are often triggered by a kind of selectional restriction that is associated with some tense and aspect markers. In sections 2 and 4.2 above we discussed examples of grammatical morphemes (the progressive aspect markers in English and Mandarin) which can normally be used only for describing events, and not for states. Similar restrictions are found in a number of other languages as well: certain tense or aspect markers may select for specific situation types (Aktionsart). De \citet{Swart1998} refers to selectional restrictions of this kind as \textsc{aspectual sensitivity}.



In \sectref{sec:2} we illustrated the principle that stative predicates cannot normally be expressed in the progressive with examples like those in (\ref{ex:}a-c). However, we noted there that some such examples might be acceptable with a coerced interpretation in certain contexts. The progressive in (\ref{ex:}d), for example, suggests that the described state is temporary and likely to last only a short time. The progressive form of (\ref{ex:}e) seems to coerce a basically stative proposition, which would be a tautology in the simple present (\textit{he is himself}), into an eventive (activity) interpretation, roughly ‘acting in a way typical of him’.


\ea
\ea *This room is being too warm.\\
\ex *I am knowing the answer.\\
\ex *George is loving sauerkraut.\\
\ex George is loving all the attention he is getting this week.\\
\ex Arthur is being himself.
                       \z
\z


De Swart discusses two past tense forms in French: the \textit{passé simple} vs. the \textit{imparfait}. She suggests that they differ primarily in terms of their aspectual sensitivity: the \textit{passé simple} occurs only with bounded situations, while the \textit{imparfait} occurs only with unbounded situations. The normal way of expressing a state that was true in the past is with the \textit{imparfait}, as in (\ref{ex:}a), because states are not naturally bounded. When the \textit{passé simple} is used for stative predicates, as in (\ref{ex:}b), the sentence must receive a bounded interpretation through some kind of coercion effect. Depending on context, it could be bounded either by referring to the beginning of the state (ingressive/inchoative reading), or by describing a state that held true only for a limited period of time.


\ea
\ea \gll  Anne  \textit{était}  triste.\\
Anne  was(\textsc{imp})  sad.\\
\glt ‘Anne was sad.’
\ex \gll Anne  \textit{fut}  triste\\
Anne  was(\textsc{ps})  sad.\\
\glt ‘Anne became sad.’ or: ‘Anne was sad for a while.’
\z \z


The use of the \textit{passé simple} in the second sentence of (\ref{ex:}a) causes the normally stative predicate to be interpreted as an event (change of state) which takes place subsequent to the previous event in the narrative. The use of the \textit{imparfait} in (\ref{ex:}b) is interpreted as describing a state which overlaps the event described in the preceding sentence.


\ea
\ea  \gll Georges  annonça  sa  résignation.  Anne  \textit{fut}  triste.\\
George  announced  his  resignation.  Anne  was(\textsc{ps})  sad.\\
\glt ‘George announced his resignation. Anne became sad (as a result).’
\ex \gll  Georges  annonça  sa  résignation.  Anne  \textit{était}  triste\\
George  announced  his  resignation.  Anne  was(\textsc{imp})  sad.\\
\glt ‘George announced his resignation. Anne was sad (during that time).’
\z \z


A similar contrast is illustrated in \REF{ex:}. The use of the \textit{passé simple} in the second clause of (\ref{ex:}a) causes ‘cross the street’ to be interpreted as a bounded event (an accomplishment) which takes place subsequent to the event in the previous clause. The use of the \textit{imparfait} in (\ref{ex:}b) is interpreted as describing an unbounded event (an activity) which overlaps with the event described in the previous clause.


\ea
\ea \gll Quand  elle  vit  Georges,  Anne  \textit{traversa}  la  rue.\\
when  she  saw  George  Anne  crossed(\textsc{ps})  the  street\\
\glt ‘When/after she saw George, Anne crossed the street.’
\ex \gll  Quand  elle  vit  Georges,  Anne  \textit{traversait}  la  rue.\\
when  she  saw  George  Anne  crossed(\textsc{imp})  the  street\\
\glt ‘When she saw George, Anne was crossing the street.’
\z \z


The adverbial phrase ‘for two hours’ in \REF{ex:} imposes bounds on an activity (playing the piano) which would otherwise be unbounded. In this context, the most natural description of a past event would use the \textit{passé simple}, as in (\ref{ex:}a). De Swart states that the use of the \textit{imparfait} in (\ref{ex:}b) cannot describe a single event of Anne playing the piano for two hours, but could receive a habitual interpretation: whenever she played the piano, she used to play for two hours.


\ea
\ea \gll  Anne  \textit{joua}  du  piano  pendant  deux  heures.\\
Anne  played(\textsc{ps})  the  piano  for  two  hours\\
\glt ‘Anne played the piano for two hours.’
\ex \gll  Anne  \textit{jouait}  du  piano  pendant  deux  heures.\\
Anne  played(\textsc{imp})  the  piano  for  two  hours\\
\glt ‘Anne used to play the piano for two hours.’
\z \z


We will see more examples of coercion effects arising from aspectual sensitivity in the next two chapters.


\section{Conclusion}\label{sec:} %7. /

\textit{Aktionsart} (situation aspect) is a way of classifying situations (events and states) on the basis of their temporal contour, that is, the shape of their “run time”. A state is a situation which is homogeneous over time (nothing changes within the time span being described), while an event involves some kind of change. The primary features which are used to distinguish different classes of events are duration and telicity (boundedness).



Grammatical aspect (or “viewpoint aspect”) is a choice that the speaker makes in describing a situation, part of the claim that is being made about the situation. It is expressed by grammatical morphemes which indicate the relation between the run time of the situation and the “Topic Time”, or time about which a claim is being made. The most basic distinction is between perfective aspect, which indicates that the situation time is contained within Topic Time, vs. imperfective aspect, which indicates that the situation time extends beyond the boundaries of Topic Time.



Some tense and aspect markers impose selectional restrictions on the types of situations which they can be used to describe. De \citet{Swart1998} refers to selectional restrictions of this kind as \textsc{aspectual sensitivity}. When the expected temporal contour of the described situation clashes with the aspectual sensitivity of the tense or aspect marker that is used in the description, or with some other element of the clause (e.g. an adverbial phrase), a new interpretation may be coerced that involves a different \textit{aktionsart}. This type of coercion is an important factor in explaining how the basic meanings (established sense(s)) of tense and aspect markers can account for their observed range of uses.



\furtherreading



\citet{Comrie1976} is still an excellent resource on the typology of grammatical aspect. Smith (1991/1997) is another foundational work on grammatical (or “viewpoint”) aspect and \textit{aktionsart} (situation aspect). \citet{Binnick2006} provides a helpful overview of these topics, and \citet{Klein2009} provides a helpful introduction to his theory of tense and aspect. \citet{Dowty1979} provides an very good description of the \textit{aktionsart} categories and summarizes a number of useful tests for identifying and distinguishing them.


\subsubsection{Discussion exercises:}\label{sec:}

\textbf{A:} Identify the most likely situation type (\textit{aktionsart}) for the following predicates (options: \textsc{state, activity, achievement, accomplishment, semelfactive}):\footnote{Patterned after Kearns (2000, p. 225).}

\begin{enumerate}
\item \begin{enumerate}
\item \itshape
swim
\item \itshape
be happy
\item \itshape
wake up
\item \itshape
snap your fingers
\item \itshape
compose a sonnet
\item \itshape
swim the English channel
\item \itshape
drink coffee
\item \itshape
drink two cups of coffee
\item \textit{expire} (e.g., visa, passport, etc.)
\item \textit{own} (e.g., \textit{John owns a parrot})
\end{enumerate}
\end{enumerate}
\subsubsection{Homework exercises:}\label{sec:}

\textbf{A:} Some English verbs are polysemous between a stative sense and a dynamic (eventive) sense. Show how the progressive aspect can be used to distinguish these two senses for each of the following five verbs: \textit{weigh, extend, surround, smell}, \textit{apply} (e.g. \textit{that law doesn’t apply} vs. \textit{apply for a job}).\footnote{Adapted from Saeed (2009, p. 147).}

\textsf{Model answer:} \textsf{\textit{have}}

\begin{enumerate}
\item \textsf{\textsc{Stative}}\textsf{: She has/\#is having four grown children.}
\item \textsf{\textsc{Dynamic}}\textsf{: She is having a baby.}
\end{enumerate}

\textbf{B:} Show how you would use time adverbials (e.g. \textit{for an hour} vs. \textit{in an hour}) to determine whether each of the following situations is telic or atelic:

\begin{enumerate}
\item \begin{enumerate}
\item \itshape
Walter laughed.
\item \itshape
Susan realized her mistake.
\item \itshape
Horace played piano sonatas.
\item \textit{Horace played Beethoven’s Pathétique sonata}.
\item \itshape
Martha resented George’s comment.
\end{enumerate}
\end{enumerate}

\textbf{C:} Describe the coercion effects in the following examples:

\ea
\ea As I walked through his door, \textit{I was instantly aware} of the quiet strength of mind\\
  Buzz possesses.\footnote{\url{http://www.vvoice.org/?module=displaystory & story_id=3725 & format=html}} \\
\ex William recited the \textit{Iliad} for a few minutes.\\
\ex John knocked on the door for ten minutes.\\
\ex The children of Atuler village in Sichuan have for many years been climbing\\
  up a sheer 800 meter cliff on rattan ladders in order to attend school.
\z \z

\chapter{{21}: Tense}

\section{Introduction}\label{sec:} %1. /

As we discussed in \chapref{sec:20}, tense markers are frequently described as locating a situation in time relative to the time of speaking (or some other reference time). However, we argued (following Klein and others) that tense actually indicates the location of the \textsc{topic time} (the time span which is currently under discussion), rather than the time of the situation itself. In this chapter we explore the kinds of meanings that can be expressed by tense markers.



In \sectref{sec:2} we will compare Klein’s theory of tense with some other well-known approaches. In \sectref{sec:3} we discuss in some detail the simple present tense in English. This turns out to be a useful case study, because it illustrates how a wide range of uses can be explained in terms of a single basic sense plus coercion effects triggered by selectional restrictions etc.



\sectref{sec:key:4} discusses the difference between \textsc{absolute tense}, which defines past, present, or future relative to the time of speaking, from \textsc{relative tense}, in which the reference point for tense marking is some time other than the time of speaking. Some languages also have \textsc{complex tense} forms, which combine absolute with relative time reference. In example \REF{ex:}, for example, the first clause specifies a topic time (3:15 pm) that is in the past relative to the time of speaking. That time becomes the reference point for the tense marking in the second clause, which specifies a new topic time (3:00 pm) that is in the past relative to this reference point. The form \textit{had left} is an example of a complex tense, namely “past-in-the-past”.


\ea
I managed to get to the station at 3:15 pm, but the train \textit{had left} promptly at 3:00.
\z


Most languages that have grammatical tense markers distinguish only relative order: past is before the time of speaking, future is after the time of speaking. Some, however, make finer distinctions. \sectref{sec:key:5} briefly illustrates some of these \textsc{metrical tense} systems, in which various degrees of past or future time are grammatically distinguished. 


\section{2. Tense relates Topic Time to the Time of Utterance}\label{sec:}

In \chapref{sec:20} we quoted the following standard definitions of tense:


\ea
\ea  “\textbf{\textsc{Tense}} is grammaticalised expression of location in time… [T]enses locate situations either at the same time as the present moment…, or prior to the present moment, or subsequent to the present moment.” (\citealt{Comrie1985}:9, 14)
\ex   “\textbf{\textsc{Tense}} refers to the grammatical expression of the time of the situation described in the proposition, relative to some other time.” \citep{Bybee1985}
\z \z


An important feature of these definitions is the restriction to “grammatical(ized) expressions” of location in time. Every language has a variety of content words which can be used to specify the time of an event. These may include NPs (\textit{last year, that week, the next day}), PPs (\textit{in the morning, after the election}), temporal adverbs (\textit{soon, later, then}), adverbial clauses (\textit{While Hitler was in Vienna,} …), etc. But not all languages have tense markers. The traditional use of the term \textsc{tense} in linguistics has been restricted to grammatical morphemes: inflectional affixes, auxiliary verbs, particles, etc.



One way to represent the “location” of a situation in time is to define logical operators (e.g. \textsc{past} and \textsc{future}) which will add tense information to a basic proposition. These tense operators can be defined as existential quantifiers over times, as suggested in \REF{ex:}.\footnote{\citet{Prior1957,Prior1967}.} This definition says that \textsc{past}(p) will be true at the time of speaking just in case there was some time prior to the time of speaking when p was true; and similarly for \textsc{future}(p). (The letter “t” stands for ‘time’, and “<” in this context means ‘prior to’. TU represents the time of speaking; this is typically the time for which the truth value of a statement is evaluated.)


\ea
\textsc{past}(p) is true at TU  iff  ${\exists}$\textsubscript{t}[t < TU $\wedge$ p is true at time t]\\
\textsc{future}(p) is true at TU  iff  ${\exists}$\textsubscript{t}[TU < t $\wedge$ p is true at time t]
\z


This system works fairly well in many cases, but \citet{Partee1973} points out that it leads to problems with examples like \REF{ex:}:


\ea
Wife to husband, as they drive away from their house: “\textit{I didn’t turn off the stove}.”
\z


If the positive statement \textit{I turned off the stove} is interpreted as shown in (\ref{ex:}a), there are two possible ways of interpreting the corresponding negative statement, depending on the scope of negation, as shown in (\ref{ex:}b). The first reading means that the speaker has never in her life turned off the stove, while the second reading means that there was at least one moment in her life when the speaker was not turning off the stove. Clearly neither of these captures the intended meaning.


\ea
\ea  \textit{I turned off the stove}.  ${\exists}$\textsubscript{t} [t < TU $\wedge$ TURN\_OFF(speaker, stove) is true at t]
\ex   \textit{I didn’t turn off the stove}.  \textit{¬}${\exists}$\textsubscript{t} [t < TU $\wedge$ TURN\_OFF(speaker, stove) is true at t]\\
  or:  ${\exists}$\textsubscript{t} [t < TU $\wedge$ \textit{¬}TURN\_OFF(speaker, stove) is true at t]
\z \z

In \chapref{sec:20} we introduced Klein’s analysis of tense, which crucially defines tense as indicating the location of Topic Time rather than the location in time of the situation itself. Under Klein’s analysis, the past tense in \REF{ex:} \textit{I didn’t turn off the stove} indicates that Topic Time is prior to the Time of Utterance. The Topic Time is determined by the context; in this situation, it would be the time immediately before leaving the house. The speaker is asserting that at that particular time, she didn’t turn off the stove. No assertion is made about other times. This analysis provides the correct interpretation.


To review, Klein defines tense and aspect as shown in \REF{ex:}, where TT = Topic Time (the time period about which the speaker is making a claim); TU = Time of Utterance (i.e., time of speaking); and TSit = the time of the event or situation which is being described.


\ea
\ea \textsc{Tense} indicates a temporal relation between TT and TU;\\
\ex \textsc{Aspect} indicates a temporal relation between TT and TSit.
                       \z
\z


So, for example, past tense can be defined as a grammatical marker which indicates that TT is prior to TU. Future tense can be defined as indicating that TU is prior to TT. Present tense might be defined as indicating that TU is contained within TT.



Klein’s framework is based on the very influential work of \citet{Reichenbach1947}. Reichenbach defined tense categories in terms of three cardinal points in time: \textsc{speech time} (S), the time of the utterance; \textsc{event time} (E), the time of the event or situation which is being described; and \textsc{reference time} (R). S and E correspond to Klein’s TU and TSit, respectively. Reichenbach’s “reference time” can be seen as analogous to Klein’s TT, although there is some disagreement as to what Reichenbach actually meant by this term. In the discussion that follows we will use Klein’s terminology, but Reichenbach’s terms (E, S, and R) are also widely used, and it will be helpful to be aware of these as well.



Because tense is (normally) marked relative to the time of the speech event, tense markers are considered to be deictic elements. It is helpful to remember that tense markers normally do not fully specify the location of the topic time; rather, they impose constraints on that location, such as TT < TU (for past tense). More specific time reference can be achieved by using temporal adverbs, adverbial clauses, etc.



Klein’s definition of tense as marking a temporal relation between TT and TU provides us with a foundation for analyzing the semantic content of specific tense markers. However, as Comrie (1985:26–29, 54–55) points out, tense markers can be associated with other kinds of meaning as well, including presuppositions, implicatures, idiomatic uses, and polysemous senses. These factors often combine to create a complex range of possible uses even for tense markers whose basic semantic content is relatively simple. We can illustrate some of the challenges involved in analyzing tense systems by looking at the simple present tense in English.


\section{Case study: English simple present tense}\label{sec:} %3. /

The simple present tense in English is notoriously puzzling, as \citet{Langacker2001} observes:


{}[T]he English present is notorious for the descriptive problems it poses. Some would even refer to it as “the so-called present tense in English”, so called because a characterization in terms of present time seems hopelessly unworkable. On the one hand, it typically cannot be used for events occurring at the time of speaking. To describe what I am doing right now, I cannot felicitously use sentence [a], with the simple present, but have to resort to the progressive, as in [b]. On the other hand, many uses of the so-called present do not refer to present time at all, but to the future [a], to the past [b], or to transcendent situations where time seems irrelevant [c]. It appears, in fact, that the present tense can be used for \textbf{anything but} the present time.

\ea
\ea *I \textit{write} this paper right now.\\
\ex I am writing this paper right now.
                       \z
\z

\ea
\ea My brother \textit{leaves} for China next month.
\ex  I’m eating dinner last night when the phone \textit{rings}. I \textit{answer} it but there’s no\\
  response. Then I \textit{hear} this buzzing sound.
\ex  The area of a circle \textit{equals} pi times the square of its radius. 
\z \z


The concept of \textsc{aspectual sensitivity} (the potential for tense forms to select specific situation types or Aktionsart), which we introduced in \chapref{sec:20}, can help us to explain at least some of these puzzles.\footnote{Much of the discussion in this section is based on \citet{Michaelis2006}.} Suppose that the basic meaning of the English simple present tense is, in fact, present tense: it indicates that TU is contained within TT. In addition, suppose that the simple present imposes a selectional restriction on the described situation: only states may be described using this form of the verb. This would immediately explain why eventive (non-stative) situations that are happening at the time of speaking cannot normally be expressed in the simple present but require the progressive, as illustrated in \REF{ex:}.



What happens when an event-type predicate is expressed in the simple present? Eventive predicates in the progressive can be interpreted as referring to specific events occurring at the time of speaking, as seen in (\ref{ex:}a) and (\ref{ex:}a), but this interpretation is not available for the simple present because of the aspectual sensitivity described in the preceding paragraph. For this reason, an event-type predicate in the simple present frequently gets a habitual interpretation, as seen in examples (\ref{ex:}b) and (\ref{ex:}b).\footnote{Examples (\ref{ex:}b) and (\ref{ex:}b) are taken from \citet[185]{Smith1997}.}


\ea
\ea Mary is playing tennis.\\
\ex Mary \textit{plays} tennis.
                       \z
\z

\ea
\ea Sam is feeding the cat.\\
\ex Sam \textit{feeds} the cat.
                       \z
\z

As discussed in \chapref{sec:20}, habitual aspect describes a recurring event or on-going state which is a characteristic property of a certain period of time.\footnote{Comrie (1976: 27–28).} Examples (\ref{ex:}b) and (\ref{ex:}b) describe not what Mary and Sam are doing at the time of speaking, but characteristic properties of Mary and Sam; thus these sentences actually refer to states, not events. This is an example of coercion: since the aspectual sensitivity of the simple present blocks the normal eventive sense of these predicates, they take on a stative meaning in this context. The fact that the habitual readings encode states rather than events can be seen in the fact that (\ref{ex:}b) and (\ref{ex:}b) cannot be appropriately used to answer the question, “What is happening?”


A very similar use of the simple present is for \textsc{gnomic} (or universal) statements, like those in \REF{ex:}; see also (\ref{ex:}c). Again, even though the verbs used in \REF{ex:} are eventive, these sentences do not refer to specific events but to general properties.


\ea
\ea Pandas \textit{eat} bamboo shoots.\\
\ex Water \textit{boils} at 100°C.\\
\ex Work \textit{expands} to fill the time available.\\
\ex Absolute power \textit{corrupts} absolutely.
                       \z
\z

As Langacker illustrated in (\ref{ex:}a), the simple present can also be used to refer to events in the future. Additional examples are provided in \REF{ex:}. This “futurate present” usage presents two puzzles. First, we need to explain the shift in time reference. Second, we would like to account for the apparent violation of the aspectual restriction noted above: the simple present can be used to refer to specific events in the future, whereas this is normally impossible for events in the present.

\ea
\ea The Foreign Minister \textit{flies} to Paris on Tuesday (but you could see him on Monday).\\
\ex Brazil \textit{hosts} the World Cup next year.\\
\ex This offer \textit{ends} at midnight tonight, and will not be repeated.
                       \z
\z

\citet[47]{Comrie1976} notes that “there is a heavy constraint on the use of the present tense with future reference, namely that the situation referred to must be one that is scheduled.” He illustrates this constraint with the examples in \REF{ex:}. Comrie notes that (\ref{ex:}b) would only be acceptable if God is talking, or if humans develop new technology that allows them to schedule rain.

\ea
\ea The train departs at five o’clock tomorrow morning.\\
\ex ?\#It rains tomorrow.
                       \z
\z

Note also that the future interpretation of the simple present is not available within the scope of a conditional or temporal adverbial clause, as seen in (\ref{ex:}b), since these seem to block the inference that the event is independently scheduled.

\ea
\ea If/When you touch me, I will scream.  (main clause refers to specific event)\\
\ex If/When you touch me, I \textit{scream}.  (only gnomic/universal interpretation is possible)
                       \z
\z

We might explain these facts by suggesting that the futurate present is not a description of a future event, but rather an assertion that a particular event is “on the schedule” at the moment of speaking. It describes a state, specifically a property of events: the property of being scheduled. This represents another pattern of coercion. The habitual reading discussed above is unavailable because of the adverbial expressions which specify a definite future time. The scheduled future reading allows these sentences to be interpreted in a way which does not violate the aspectual sensitivity of the simple present.


There are other eventive uses of the simple present, however, which are not so easy to explain. The “historical present” illustrated in (\ref{ex:}b) seems to be allowed primarily in a specific genre of discourse, namely informal narrative. This usage seems to involve a shift in the deictic reference point, from the current time of speaking to the time line of the narrative. We need to recognize that such shifts are possible in order to deal with examples like \REF{ex:}, which should be a contradiction but is often heard on telephone answering machines.


\ea
I’m not here right now.
\z


In this example the identity of the speaker and location of the speech event are interpreted in the normal way, but the hearer is expected to interpret the deictic \textit{right now} as referring to the time when the recording is played, the time of hearing, rather than the original time of speaking. More study is needed to understand why this shift should license an apparent violation of the aspectual restrictions discussed above.



Other eventive uses of the simple present include explicit performatives, play-by-play reports by sportscasters, stage directions in the scripts of plays, etc.\footnote{See \citet{Klein2009} for a discussion of other special uses of the present.} For now, we will simply consider these to be idiosyncratic exceptions to the general rule, that is, idiomatic uses of the simple present form.


\section{Relative tense}\label{sec:} %4. /

As noted in the definitions we cited from Comrie and Bybee, tense systems typically specify location in time relative to the time of the current utterance (TU). This type of tense marking is called \textsc{absolute tense}. For certain tense markers, however, some other, contextually determined reference point is used. This type of tense marking is called \textsc{relative tense}. Because absolute tense marking is anchored to the time of the current utterance, absolute tenses are deictic elements; relative tenses might be considered anaphoric rather than deictic. A Brazilian Portuguese example is presented in (\ref{ex:}a).


\ea
\ea \gll Quando  você  chegar,\footnotemark  eu  já  saí.\\
when  2sg  arrive.\textsc{fut.sbjnctv}  1sg  already  leave.\textsc{past}\\
\glt ‘When you arrive, I will already have left.’   [Brazilian Portuguese; \citealt{Comrie1985}:31]
\ex   *When you arrive, I already left.
\z \z
\footnotetext{The future subjunctive is homophonous with the infinitive paradigm for most verbs, including \textit{chegar}; but the paradigms are distinct for certain irregular verbs, including \textit{ter} ‘have’, \textit{haver} ‘have’, \textit{ser} ‘be’, \textit{estar} ‘be’, \textit{querer} ‘want’, \textit{trazer} ‘bring’, \textit{ver} ‘see’, \textit{vir} ‘come’ (Jeff Shrum, p.c.).}


The simple past tense form \textit{saí} ‘left’ would normally have past reference; but in this context it gets a relative tense interpretation, indicating that the event described in the main clause is located in the past relative to the time of the event described in the adverbial clause. So in this context a verb marked for past tense can refer to an event which is actually in the future relative to the time of the speech event (TU). As demonstrated in (\ref{ex:}b), the literal English translation of this sentence is ungrammatical, because the simple past tense in English normally does not allow this kind of relative tense interpretation.



We will refer to the contextually determined reference point of a relative tense marker as the \textsc{perspective time} (PT).\footnote{This terminology follows \citet{Kiparsky2002} and \citet{Bohnemeyer2014}.} Absolute tense constrains the relationship between TT and TU, while relative tense constrains the relationship between TT and PT. In example (\ref{ex:}a), the adverbial clause (‘When you arrive’) establishes the perspective time, which is understood to be in the future relative to the time of speaking. The past tense on the main verb \textit{saí} ‘left’ gets a relative tense interpretation in this context, indicating that the topic time (i.e., the time about which an assertion is being made) is in the past relative to the perspective time.



The most likely interpretation for ex. (\ref{ex:}a) is diagrammed in \REF{ex:}. Relative past tense imposes the constraint that TT < PT, but does not specify whether TT is before or after TU. The fact that TT is later than TU is a pragmatic inference; if the speaker had already left before the time of speaking, it would be more natural and informative to simply say ‘I have already left.’ (The relationship between TT and TSit is determined by the perfective aspect of the simple past form, as discussed in \chapref{sec:20}.)


\ea
TU  [  TT  ]  PT  \\
      \textbf{{\textbar}}TSit: my departure\textbf{{\textbar}}    [ your arrival ]
\z


In Imbabura Quechua, main clause verbs have absolute tense reference.\footnote{\citet{Cole1982}, \citet[61]{Comrie1985}.} Most subordinate verbs use a distinct set of tense affixes which get a relative tense interpretation.\footnote{Verbs in relative clauses use the main-clause tense markers with absolute tense reference.} In the following examples, the subordinate verb ‘live’ is marked for relative past, present or future tense according to whether it refers to a situation which existed before, during or after the situation named by the main verb, which determines the perspective time. Since the main verb is marked for past tense, the actual time referred to by the subordinate verb may have been before the time of the utterance even when it is marked for ‘future’ tense, as in (\ref{ex:}c):


\ea
\textbf{Imbabura Quechua} (Peru; \citealt{Cole1982}:143)
\z

\ea
\ea  [Marya  Agatu-pi  kawsa-j]-ta  kri-rka-ni\\
Mary  Agato-in  live-\textsc{pres}-\textsc{acc}  believe-\textsc{past}-\textsc{1subj}\\
‘I believed that Mary was living (at that time) in Agato.’
\ex  [Marya  Agatu-pi  kawsa-shka]-ta  kri-rka-ni\\
Mary  Agato-in  live-\textsc{past}-\textsc{acc}  believe-\textsc{past}-\textsc{1subj}\\
‘I believed that Mary had lived (at some previous time) in Agato.’
\ex  [Marya  Agatu-pi  kawsa-na]-ta  kri-rka-ni\\
Mary  Agato-in  live-\textsc{fut}-\textsc{acc}  believe-\textsc{past}-\textsc{1subj}\\
‘I believed that Mary would (some day) live in Agato.’
\z \z


Relative past tense is sometimes referred to as \textsc{anterior} tense, relative future as \textsc{posterior} tense, and relative present as \textsc{simultaneous} tense. Relative tense is most common in subordinate clauses, but is also found in main clauses in some languages (e.g., classical Arabic). \citet{Comrie1985} points out that participles in many languages, including English and Latin, get a relative tense interpretation. Example \REF{ex:} illustrates the simultaneous meaning of the English present participle (\textit{flying}). Example \REF{ex:} illustrates the posterior meaning of the Latin future participle: the event of crossing the river is described for a topic time which is in the future relative to the perspective time defined by the main clause (the time when he failed to send over the provisions). Example \REF{ex:} illustrates the anterior meaning of the Latin past participle: the event of delaying is described for a topic time which is in the past relative to the perspective time defined by the main clause (the time when he orders them to give the signal).


\ea
\ea Last week passengers \textit{flying} with Qantas were given free tickets.\\
\ex Next week passengers \textit{flying} with Qantas will be given free tickets.
                       \z
\z

\ea
\textit{Tr\=aiect\=urus}  Rh\=enum  comme\=atum  n\=on  tr\=ansm\={\i}sit.\\
cross-\textsc{fut.prtcpl}  Rhine  provisions  \textsc{neg}  send.over-\textsc{past.pfctv.3}sg\\
‘Being about to cross the Rhine, he did not send over the provisions.’\\
{}[Suetonius; cited in \citealt{Comrie1985}:61]
\z

\ea
Paululum  \textit{commor\=atus},\footnote{The past participle in Latin, as in English, normally has a passive meaning; but the verb meaning ‘delay’ in Latin is a \textsc{deponent} verb, meaning that passive morphology does not create a passive meaning.}  s\={\i}gna  canere  iubet.\\
little.bit  delay-\textsc{pst.prtcpl}  signal.\textsc{pl}  to.sound  order-\textsc{pres.3}sg\\
‘Having delayed a little while, he orders them to give the signal.’\\
{}[Sallust, Catilina 59; cited in \citealt{AllenGreenough1931}, §496]
\z


The English \textit{be} \textit{going to} construction is sometimes identified as marking posterior tense. It can express future time relative to a perspective time in the past, as in (\ref{ex:}a), creating a “future in the past” meaning. It can express future time relative to some generic or habitual perspective time, which may be past or present relative to the time of speaking, as illustrated in (\ref{ex:}b-c).


\ea
\ea I was just \textit{going to tell} you when you first came in, only you began about\\
  Castle Richmond.\footnote{Anthony \citet{Trollope1860}, \textit{Castle Richmond}; cited at: \url{http://grammar.about.com/od/fh/g/Future-In-The-Past.htm}} \\
\ex John keeps saying that he is \textit{going to visit} Paris some day.\\
\ex Dibber always did tell me Pat was \textit{going to study} to be a doctor.\footnote{John Fante, “Horselaugh on Dibber Lannon”; cited at: \url{http://grammar.about.com/od/fh/g/Future-In-The-Past.htm}} \\
\ex John \textit{is going to visit} you very soon.
                       \z
\z


\citet{Comrie1985} points out that if a relative tense is used in contexts where the perspective time is equivalent to the time of speaking, then its meaning is equivalent to the corresponding absolute tense. For example, the interpretation of the posterior tense in (\ref{ex:}d) is equivalent to a simple future tense. English does not have a fully natural way of indicating “future in the future”. Comrie states that the closest equivalent would make use of the \textit{about to} construction, which marks immediate future: \textit{he will be about to X}. 


\subsection{Complex (“absolute-relative”) tense marking}\label{sec:} %4.1 /

The perspective time (PT) for relative tense markers like those discussed above is not grammatically specified, but is determined by contextual features. However, Comrie points out that some languages do have tense forms which grammatically specify both the location of PT (relative to TU) and the location of TT (relative to PT). Comrie refers to such cases as “absolute-relative” tense marking; we will use the term \textsc{complex tense}.



The English Pluperfect construction (\textit{I had eaten}) can be used to express “past in the past”, as illustrated in \REF{ex:}. In example (\ref{ex:}a), the event of Sam reaching the base camp is asserted to be true at a topic time which is in the past relative to a perspective time in the past, which is defined by the preceding clause (the time when the speaker arrived there). In example (\ref{ex:}b), the event of Einstein publishing a paper (in 1905) is asserted to be true at a topic time which is in the past relative to a perspective time in the past, i.e. the time at which he won the Nobel prize (1922).


\ea
\ea I reached the base camp Tuesday afternoon; Sam \textit{had arrived} the previous evening.\\
\ex Einstein was awarded the Nobel prize in 1922, for a paper that he \textit{had published}\\
  in 1905.
                       \z
\z


Similarly, the Future Perfect construction (\textit{I will have eaten}) can be used to express “past in the future”. In example (\ref{ex:}a), the event of Sam reaching the base camp is asserted to be true at a topic time which is in the past relative to a perspective time in the future (the time when the speaker arrives there). Another complex tense, “future in the past”, is illustrated in (\ref{ex:}b). This sentence asserts that the event of Einstein winning the Nobel prize (1922) was in the future relative to a perspective time in the past, i.e. the year in which he published four ground-breaking papers (1905).


\ea
\ea I expect to reach the base camp on Tuesday afternoon; Sam \textit{will have arrived} \\
  the previous evening.\\
\ex Einstein published four ground-breaking papers in 1905, including the one for which\\
  he \textit{would win} the Nobel prize in 1922.
                       \z
\z


The relative positions of TT, PT and TU for the italicized verbs in examples (\ref{ex:}b), (\ref{ex:}a), and (\ref{ex:}b) are shown in the diagrams in \REF{ex:}.


\ea
\ea      [  TT: 1905  ]  PT  TU   “past in the past” (b)\\
         \textbf{{\textbar}}TSit\textbf{{\textbar}}    (1922)  (now)
\ex    TU  [  TT: Mon. eve.  ]  PT  “past in the future” (a)\\
  (now)      \textbf{{\textbar}}TSit\textbf{{\textbar}}    (Tues. pm)
\ex    PT  [  TT: 1922  ]  TU  “future in the past” (b)\\
  (1905)      \textbf{{\textbar}}TSit\textbf{{\textbar}}   (now)
\z \z


As we will see in \chapref{sec:22}, the Pluperfect and Future Perfect forms are ambiguous. In addition to the complex tense readings illustrated in (\ref{ex:}--\ref{ex:}), they can also be used to indicate perfect aspect; but here we consider only their tense functions.\footnote{As discussed in \chapref{sec:22}, the temporal adverbs used here ensure that only the complex tense readings are available.}



\citet{Comrie1985} points out that cross-linguistically, most forms which express complex tense meanings are morphologically complex, i.e. involve combinations of two or more morphemes, like the English Pluperfect and Future Perfect constructions. However, occasional exceptions to this generalization do exist, e.g. the mono-morphemic pluperfect \textit{–ara} in literary Portuguese.


\subsection{4.2 Sequence of tenses in indirect speech}\label{sec:}

The difference between direct vs. indirect speech is that direct speech purports to be an exact quotation of the speaker’s words, as in (\ref{ex:}a), whereas indirect speech does not (\ref{ex:}b).\footnote{Most languages probably make a distinction between direct vs. indirect speech, but in some languages the difference is quite subtle. A number of languages are reported to have an intermediate form, “semi-direct speech”, in which some but not all of the deictic elements (especially pronouns and/or agreement markers) shift their reference point.}


\ea
\ea Yesterday Arthur told me, “I will meet you here again tomorrow.”  [\textsc{direct}]\\
\ex Yesterday Arthur told me that he would meet me there again today.  [\textsc{indirect}]
                       \z
\z


One of the most important differences between the two forms is seen in the use of the deictic elements. Deictics within the direct quote (\ref{ex:}a) are anchored to the perspective of the original speaker (Arthur) and the time and place of the original speech event: \textit{I} = Arthur; \textit{you} = the addressee in the original speech event, who is also the speaker in the current, reporting event; \textit{here} = place of the original speech event; \textit{tomorrow} = the day following the original speech event; etc. Deictics within the indirect quote (\ref{ex:}b) are anchored to the perspective of the speaker in the current, reporting event (= the addressee in the original speech event), and the time and place of the current speech event. So \textit{I} shifts to \textit{he}; \textit{you} shifts to \textit{me}; \textit{here} shifts to \textit{there}; \textit{tomorrow} shifts to \textit{today}; etc.



Notice that the tense of the verb also shifts: \textit{will meet} in the direct quote (\ref{ex:}a) becomes \textit{would meet} in the indirect quote (\ref{ex:}b). Since (absolute) tense is a deictic category, anchored to the time of speaking, this is hardly surprising. It would be natural to assume that this shift in tenses follows automatically from the shift in deictic reference point. This may in fact be the case in some languages, but in English and a number of other languages, the behavior of tense in indirect speech is more complex. (The same issues often arise in other types of finite complements, e.g. complements of verbs of thinking and knowing, in addition to verbs of saying.)



\citet{Comrie1985} presents an interesting contrast between the use of tense in indirect speech in English vs. Russian. In Russian, the tense of the verb in indirect speech is identical to the tense in direct speech, i.e., the tense that was used by the original speaker in the original speech act. However, all of the other deictic elements shift to the perspective of the current speaker, just as they do in English. An example is presented in \REF{ex:}, reporting a speech act by John at some unspecified time in the past:\footnote{Data from \citet[109]{Comrie1985}. The non-past tense used in these examples would be interpreted with future reference in this context.}


\ea
\ea  \gll Džon  skazal:  “Ja  ujdu  zavtra.”\\
John  said  1sg  will.leave  tomorrow\\
\glt John said, “I will leave tomorrow.”  [\textsc{direct}]
\ex \gll Džon  skazal,  čto  on  ujdet  na  sledujuščij  den.\\
John  said  \textsc{comp}  3sg  will.leave  on  next  day\\
\glt John said that he would leave (lit: will leave) on the following day.  [\textsc{indirect}]
\z \z


In other words, verbs in Russian indirect speech complements (and other finite complements) get relative tense marking: the reference point is not the current time of speaking, but the time of the reported speech event (or, more generally, the topic time of the main clause). English verbs behave differently in this regard. For example, in (\ref{ex:}b) and the English translation of (\ref{ex:}b), where the original speaker used a simple future tense (\textit{will leave}), the form used in indirect speech is the complex “future in the past” tense (\textit{would leave}). As noted above, this is what we would expect to happen due to the shift in the deictic reference point, from the time of the original speech event to the time of the current, reporting speech event. However, there are other contexts where this shift by itself cannot account for the English tense forms.



The examples in \REF{ex:} suggest that the form of the complement verb depends on the tense of the matrix (main clause) verb. Assume that John’s actual words in both (\ref{ex:}a) and (\ref{ex:}b) use the present progressive form (\textit{I am studying}). When the matrix verb occurs in the future tense, as in (\ref{ex:}b), English seems to follow the same pattern as Russian: the tense of the complement verb in indirect speech is identical to the tense that would have been used by the original speaker. However, when the matrix verb occurs in the past tense, this is not always true: in (\ref{ex:}a), for example, we see the past progressive form (\textit{was studying}) instead of the present progressive (\textit{is studying}).


\ea
\ea Yesterday I asked John what he was doing, and he said that he \textit{was/*is studying}.\\
\ex If I ask him the same thing tomorrow, he will say that he \textit{is/*will be studying}.
                       \z
\z


Some additional examples illustrating this contrast are presented below. One general pattern that emerges is that, when the complement clause contains an auxiliary verb, that auxiliary retains its original tense form if the matrix verb occurs in the future (b, b, b). However, if the matrix verb occurs in the past, the auxiliary is normally “back-shifted”, i.e., replaced by the corresponding past tense form, as seen in (a, a, b).\footnote{Many of the examples in the remainder of this section are adapted from \citet{Declerck1991}.}


\ea
\ea Yesterday I invited John to go out for pizza, but he said that he \textit{had/*has} just \textit{eaten}.\\
\ex If you invite him for pizza tomorrow, he will say that he \textit{has/*will have} just \textit{eaten}.
                       \z
\z

\ea
(spoken in 1998):\\
\ea  {In 2008} Ebenezer will say, “I \textit{will} get tenure in 2011.”\\
\ex  {In 2008} Ebenezer will say that he \textit{will} get tenure in 2011.
                       \z
\z

\ea
(spoken in 1998):\\
\ea {In 1987} Ebenezer said, “I \textit{will} get tenure in 1992.”\\
\ex {In 1987} Ebenezer said that he \textit{would}/*\textit{will} get tenure in 1992.
                       \z
\z


When the original, reported utterance contains a verb in the simple past tense, the original tense form is again retained if the matrix verb occurs in the future \REF{ex:}. This can result in a past tense form being used to describe an event which is in the future relative to the current time of speaking, as in (\ref{ex:}b). Back-shifting of a simple past form is often optional when the matrix verb occurs in the past, as in \REF{ex:}.


\ea
(spoken in 1998):\\
\ea {In 2008} Ebenezer will say, “I \textit{got} tenure in 2004.”\\
\ex {In 2008} Ebenezer will say that he \textit{got}/*\textit{will get} tenure in 2004.
                       \z
\z

\ea
(spoken in 1998):\\
\ea {In 1987} Ebenezer said, “I \textit{got} tenure in 1982.”\\
\ex {In 1987} Ebenezer said that he \textit{got}/\textit{had gotten} tenure in 1982.
                       \z
\z


There are certain other contexts where back-shifting appears to be optional as well. For example, if the matrix verb occurs in the past and the complement clause describes a situation which is still true at the current time of speaking, either past or present can often be used for the complement verb in place of the present tense used by the original speaker \REF{ex:}. However, even in this context back-shifting is sometimes obligatory, as illustrated in \REF{ex:}.


\ea
\ea Yesterday the mayor revealed that he \textit{is/was} terminally ill.\\
\ex Last week John told me that he \textit{likes/liked} you.\\
\ex The ancient Babylonians did not know that the earth \textit{circles}/\textit{circled} the sun.
                       \z
\z

\ea
\ea I \textsc{knew} you \textit{liked}/*\textit{like} her.\\
\ex This is John’s wife.\\
  — Yes, I \textsc{thought} he \textit{was}/*\textit{is} married.
                       \z
\z


The set of rules which determine the tense forms in indirect speech complements is traditionally referred to as the “sequence of tenses.” A full discussion of the sequence of tenses in English is beyond the scope of this chapter. Scholars disagree as to whether the sequence of tenses in English can be explained on semantic grounds. Some (e.g. \citealt{Comrie1985}) argue that the rules are purely grammatical, and cannot be predicted from the semantic content of the tense forms. Others (e.g. \citealt{Declerck1991}) argue that a semantic analysis is possible, though the rules would need to be fairly complex.



Our purpose in this section has been to show that verb forms in indirect speech complements may require special treatment: these verbs may not exhibit the same kind of relative tense marking found in other kinds of subordinate clauses within the same language, and the normal shift in deictic reference point may not explain the usage of the tenses. Finally, this is an area where even closely related languages can exhibit significant differences from each other.


\section{5. Temporal Remoteness markers (“metrical tense”)}\label{sec:}

Among languages in which tense is marked morphologically, the most common tense systems involve a two-way distinction: either past vs. non-past or future vs. non-future.\footnote{Chung and \citet{Timberlake1985}.} A three-way morphological distinction, like the Lithuanian past vs. present vs. future paradigm mentioned in \chapref{sec:20} (and repeated here as \REF{ex:}) is actually somewhat unusual.


\textbf{Lithuanian tense marking} (\citealt{ChungTimberlake1985}:204)

\begin{tabularx}{\textwidth}{XXX}
\lsptoprule
a. & dirb-\textit{au}\\
work-1sg\textsc{.past} & ‘I worked/ was working’\\
b. & dirb-\textit{u}\\
work-1sg\textsc{.present} & ‘I work/ am working’\\
c. & dirb-\textit{s}-iu\\
work-\textsc{future-}1sg & ‘I will work/ will be working’\\
\lspbottomrule
\end{tabularx}

However, a number of languages have verbal affixes which distinguish more than one degree of past and/or future time reference, e.g. ‘immediate past’ vs. ‘near past’ vs. ‘distant past’. Such systems are especially well-known among the Bantu languages. Example \REF{ex:} presents a paradigm from the Bantu language ChiBemba, which has (in addition to the present tense, not shown here) a symmetric set of four past and four future time markers.


\begin{tabularx}{\textwidth}{XXXXX}
\lsptoprule
& \multicolumn{3}{c}{\textbf{ChiBemba (Bantu)} (Chung and \citealt{Timberlake1985}:208, based on \citealt{Givón1972})} & \\
a. & \scshape remote past & ba-\textit{àlí} -bomb-\textit{ele} & \multicolumn{2}{c}{‘they worked (before yesterday)’}\\
b. & \scshape removed past & ba-\textit{àlíí} -bomba & \multicolumn{2}{c}{‘they worked (yesterday)’}\\
c. & \scshape near past & ba-\textit{àcí} -bomba & \multicolumn{2}{c}{‘they worked (today)’}\\
d. & \scshape immediate past & ba-\textit{á} -bomba & \multicolumn{2}{c}{‘they worked (within the last 3 hours)’}\\
e. & \scshape immediate future & ba-\textit{áláá} -bomba & \multicolumn{2}{c}{‘they’ll work (within the next 3 hours)’}\\
f. & \scshape near future & ba-\textit{léé} -bomba & \multicolumn{2}{c}{‘they’ll work (later today)’}\\
g. & \scshape removed future & ba-\textit{kà} -bomba & \multicolumn{2}{c}{‘they’ll work (tomorrow)’}\\
h. & \scshape remote future & ba-\textit{ká} -bomba & \multicolumn{2}{c}{‘they’ll work (after tomorrow)’}\\
\lspbottomrule
\end{tabularx}

A slightly less complex system is found in Grebo (Niger-Kordufanian), as illustrated in \REF{ex:}:


\begin{tabularx}{\textwidth}{XXXXX}
\lsptoprule
& \multicolumn{3}{c}{\textbf{Grebo (Niger-Kordufanian)} (\citealt{Frawley1992}:365–7; based on {In nes1966})} & \\
a. & \scshape remote past & ne du-\textit{da} bla & \multicolumn{2}{c}{‘I pounded rice (before yesterday)’}\\
b. & {\scshape yesterday past}

ne du-\textit{d[259?]} bla & ‘I pounded rice (yesterday)’ & \multicolumn{2}{c}{}\\
c. & \scshape today (past or fut) & ne du-\textit{e} bla & \multicolumn{2}{c}{‘I pounded/will pound rice (today)’}\\
d. & \scshape tomorrow future & ne du-\textit{a} bla & \multicolumn{2}{c}{‘I will pound rice (tomorrow)’}\\
e. & \scshape remote future & ne du-\textit{d[259?]\textsubscript{2}} bla & \multicolumn{2}{c}{‘I will pound rice (after tomorrow)’}\\
\lspbottomrule
\end{tabularx}

These systems are sometimes referred to as “metrical tense” or “graded tense” systems. However, some recent research has argued that at least in some languages, these markers indicate the location of the situation time (TSit), rather than the topic time (TT), relative to the time of speaking.\footnote{\citet{Cable2013}; LaCross (2016 ms.).} If this is true, then these markers would not fit Klein’s definition of tense. The widely-used label \textsc{Temporal Remoteness} is general enough to include this type as well.



As examples (\ref{ex:}--\ref{ex:}) illustrate, Temporal Remoteness systems frequently make distinctions such as ‘today’ vs. ‘yesterday’, ‘yesterday’ vs. ‘before yesterday’, etc. In such systems, the “today” category is sometimes referred to as \textsc{Hodiernal}, and the “yesterday past” category is sometimes referred to as \textsc{Hesternal}, based on the Latin words for ‘today’ and ‘yesterday’. In some languages, temporal remoteness is measured in other units of time, e.g. months or years; and in some, there can be a shift in the choice of unit depending on which unit would be contextually most relevant. Some languages make other kinds of distinctions, e.g. between remembered past vs. non-remembered past.\footnote{\citet{Botne2012}.}



The ChiBemba and Grebo systems illustrated above are both symmetrical, with equal numbers of past and future categories. It is also fairly common for a language with Temporal Remoteness markers to make more distinctions in the past than in the future. \citet{Nurse2008} reports that in his sample of 210 Bantu languages, about half have only a single future category, whereas 80\% have more then one degree of past time marking. 



When languages do have multiple contrastive future markers, it is not uncommon for one or more to take on secondary meanings relating to degree of certainty (remote future marking less certainty). Such secondary meanings are also associated with past time markers in some languages, with remoteness indicating reduced certainty.\footnote{\citet{Botne2012}; \citet{Nurse2008}.}


\section{6. Conclusion}\label{sec:}

We have adopted Klein’s definition of (absolute) tense as indicating a temporal relation between TT and TU, and aspect as indicating a temporal relation between TT and TSit. We assume further that relative tense indicates a temporal relation between TT and some perspective time (PT), which is determined by context. It is important to remember that the observed uses of tense-aspect markers do not depend only on the semantic content of these morphemes. When we seek to analyze the meanings of these markers, we need to consider the following additional factors as well:


\begin{enumerate}
\item 
\textit{aspectual sensitivity} (restriction to specific aktionsart/situation types);
\item 
potential for different semantic functions in different situation types;
\item 
coercion effects;
\item 
potential for different uses in main vs. subordinate clauses;
\item 
presuppositions triggered by the marker;
\item 
implicatures which may add extra meaning;
\item 
potential polysemy and/or idiomatic senses.
\end{enumerate}

Several of these points were illustrated in our discussion of the simple present tense in English.



\furtherreading



Comrie (1985, ch. 1) provides a good introduction to the study of tense, and (in sec. 1.8) a good discussion of the importance of distinguishing meaning from usage, for tense markers in particular. \citet{Michaelis2006} is another helpful introduction, focusing primarily on English. \citet{Botne2012} summarizes what we know about “metrical tense” systems. 


\subsubsection{Discussion exercises:}\label{sec:}

\textbf{A:} Draw time-line diagrams and provide an appropriate label for the italicized verb in the following sentences:

\textsf{Model answer:\\
I managed to get to the station at 3:15 pm, but the train} \textsf{\textit{had left}}\textsf{ promptly at 3:00.}

\ea
      {}[  TT: 3:00  ]  PT  TU     “past in the past”\\
         \textbf{{\textbar}}TSit\textbf{{\textbar}}    (3:15)
\z

\ea
\ea When I got home from the hospital, my wife \textit{wrote} a letter to my doctor.\\
\ex When I got home from the hospital, my wife \textit{was writing} a letter to my doctor.\\
\ex I fled from the Khmer Rouge in 1976; my brother \textit{would escape} two years later.\\
\ex I can get to the station by 5:00 pm, but the train \textit{will} \textit{have departed} at 3:00 pm.\\
\ex This morning the President rescinded an executive order that he \textit{had issued}\\
  just 12 hours earlier.
\z \z

\subsubsection{Homework exercises:}\label{sec:}

\textbf{A:} Draw time-line diagrams for the clauses which contain the italicized verb forms, and name the tense/aspect expressed by those forms:

\sffamily
Model answer:

\textsf{Einstein published four ground-breaking papers in 1905, including the one for which\\
  he} \textsf{\textit{would win}}\textsf{ the Nobel prize in 1922.}

\ea
    PT  [  TT: 1922  ]  TU     “future in the past”\\
  (1905)    \textbf{{\textbar}}TSit\textbf{{\textbar}}    (2016)


\ea When I got back from my trip, a family of stray cats \textit{were living} in my garage.\\
\ex The new President will move into the White House on Jan. 20\textsuperscript{th}; the previous President\\
  and his family \textit{will have vacated} the premises on Jan. 19\textsuperscript{th}.\\
\ex Kipling was sent back to England at the age of five; he \textit{would return} to India\\
  eleven years later to work as a journalist.

\ex The road to Fort Driant began for the United States Third Army when it landed on Utah Beach at 3 pm on August 5, 1944. The Third Army \textit{had been activated} four days earlier in England under the command of Lt. Gen. George S. Patton Jr.\footnote{\url{http://warfarehistorynetwork.com/daily/wwii/general-george-s-pattons-lost-battle/}} 
\z
\z

\chapter{{22}: Varieties of the Perfect}

\section{1. Introduction: \textsc{perfect} vs. \textsc{perfective}}\label{sec:}

The terms \textsc{perfect} and \textsc{perfective} are often confused, or used interchangeably, but there is an important difference between them. The contrast between the perfect (e.g. \textit{have eaten}) and perfective (\textit{ate}) in English is illustrated in the examples in \REF{ex:}. In some contexts there seems to be very little difference in meaning between the two, as illustrated in (\ref{ex:}a--b). In other contexts, however, the two are not interchangeable (\ref{ex:}c-e). For example, the perfect cannot be used with certain kinds of time adverbials which are fine with the perfective (\ref{ex:}c--d). We will discuss this very interesting restriction in \sectref{sec:3}.


\ea
\ea I just \textit{ate} a whole pizza.\\
\ex I \textit{have} just \textit{eaten} a whole pizza.\\
\ex Last night I \textit{ate}/\#\textit{have eaten} a whole pizza.\\
\ex When I was a small boy, I \textit{broke}/\#\textit{have broken} my leg.\\
\ex Gutenberg \textit{discovered}/\#\textit{has discovered} the art of printing.\footnote{\citet{McCoard1978}.}
                       \z
\z


Notice that the English perfect can be combined with imperfective (specifically progressive) aspect, as in \REF{ex:}. This shows clearly that perfect and perfective are distinct categories, because perfective and imperfective are incompatible and could not co-occur in the same clause.


\ea
\ea I \textit{have been standing} in this line for the past four hours.\\
\ex Smith \textit{has been paying} a lot of visits to New York lately.  (\citealt{Grice1975})\\
\ex Nixon \textit{has been writing} an autobiography.
                       \z
\z


There is a large measure of agreement about the basic meaning of the perfective. As stated in \chapref{sec:20}, it is an aspectual category which refers to an entire event as a whole, or as completely contained within Topic Time. In contrast, the meaning of the perfect has been and remains a highly contentious issue.



We will begin our discussion in \sectref{sec:2} by illustrating four or five well-known uses or readings of the perfect. Whether or not all of these uses can be explained in terms of a single core meaning remains one of the issues in the controversy. In \sectref{sec:3} we examine a much-discussed puzzle concerning the co-occurrence of time adverbials with the English present perfect. In \sectref{sec:4} we will review some of the properties of the various readings which have been cited as evidence supporting the claim that the English present perfect form is in fact polysemous. In sections 5–6 we examine the properties of perfect markers in two non-Indo-European languages.


\section{Uses of the perfect}\label{sec:} %2. /

\citet{McCawley1971}, \citet{Comrie1976}, and others identify four major uses, or semantic functions, of the present perfect in English: (i) experiential (or Existential) perfect, illustrated in \REF{ex:}; (ii) perfect of persistent situation (also known as the Universal reading), illustrated in \REF{ex:}; (iii) perfect of continuing result, illustrated in \REF{ex:}; (iv) perfect of recent past (the “hot news” reading), illustrated in \REF{ex:}. Similar uses are found in a number of other languages.


\ea
Experiential (or Existential) reading\\
\ea Have you ever tasted fresh durian?\\
\ex I have climbed Mt. Fuji twice in the past six months.
                       \z
\z

\ea
Perfect of persistent situation (Universal perfect)\\
\ea He has lived in Canberra since 1975.\\
\ex I have been waiting for three days.
                       \z
\z

\ea
Perfect of continuing result\\
\ea I have lost my glasses, so I can’t read this telegram.\\
\ex The governor has fainted; don’t let the press know until he regains consciousness.
                       \z
\z

\ea
Recent past (or “hot news”) reading\\
\ea A group of former city employees has just abducted the Mayor.\\
\ex The American president has announced new trade sanctions against the Vatican.
                       \z
\z


\citet{Kiparsky2002} mentions a fifth use of the perfect, attested in languages such as Swahili, Sanskrit, and ancient Greek, which he calls the Stative Present. In these languages, the perfect form can be used to refer to events, as in English; but it can also be used to refer to the state that results from an event. Some Swahili examples are provided in \REF{ex:}.\footnote{Kiparsky notes that this reading is available in English only with a single verb: \textit{I’ve got (=I have) five dollars in my pocket} (cf. \citealt{Jesperson1931}: 47). Comrie treats the Stative Present as a sub-type of his “perfect of result”.}


\ea
\textbf{Swahili} \citet{Ashton1944}
\z

\begin{tabularx}{\textwidth}{XXXX}
\lsptoprule
\bfseries\scshape Root &  & \bfseries\scshape Perfect form & \\
-fika & ‘arrive’ & a-me-fika & ‘he has arrived’\\
-iva & ‘ripen’ & ki-me-iva & ‘it is ripe’\\
-choka & ‘get tired’ & a-me-choka & ‘he is tired’\\
-simama & ‘stand up’ & a-me-simama & ‘he is standing’\\
-sikia & ‘hear, feel’ & a-me-sikia & ‘he understands’\\
\lspbottomrule
\end{tabularx}

We will focus our discussion on the four uses illustrated in (\ref{ex:}--\ref{ex:}). \citet{Comrie1976} and others have attempted to unify these four readings under a single definition in terms of “current relevance”. Comrie says that the perfect is used to express a past event which is relevant to the present situation. That is, it signals that some event in the past has produced a state of affairs which continues to be true and significant at the present moment.



Other authors have suggested that what the various uses of the perfect share is reference to an indefinite past time. \citet{Klein1992,Klein1994} for example, building on the analysis of \citet{Reichenbach1947}, suggests that the perfect indicates that Time of Situation precedes Topic Time. A number of other authors have adopted some version of Reichenbach’s analysis as well, often arguing that the different readings arise through various pragmatic inferences.



A very influential proposal by \citet{McCoard1978} argues that the meaning of the perfect locates the described event within the “Extended Now”, an interval of time which begins in the past and includes the utterance time.


\begin{quote}
“The intuitive idea of the Extended Now is that we typically count a longer stretch of time than the momentary “now” as the present for conversational purposes. Its exact duration is contextually determined, since what we count as “the present” in this sense may vary depending on the conversational topic.” (\citealt{Portner2003}) 
\end{quote}


Some authors, however, including \citet{McCawley1971,McCawley1981b}, \citet{Michaelis1994,Michaelis1998}, and \citet{Kiparsky2002}, have argued that the English perfect is polysemous, and that at least some of the readings listed above must be recognized as fully distinct senses. We will discuss some of the evidence which supports this claim in \sectref{sec:4}.


\section{3. Tense vs. aspect uses of English \textit{have} + participle}\label{sec:}
\subsection{3.1 The + perfect puzzle}\footnotemark{}\label{sec:}
\footnotetext{See \citet{Klein1992} for a detailed discussion of this topic.}

As illustrated in (\ref{ex:}c--d), the English present perfect cannot normally co-occur with adverbial phrases which name the time in the past when the event occurred; further evidence is provided in (\ref{ex:}b).


\ea
\ea George left for Paris \textit{yesterday/last week}.\\
\ex George has left for Paris (*\textit{yesterday/}*\textit{last week}).
                       \z
\z


This constraint may seem puzzling, since the use of the present perfect clearly indicates that the described event took place in the past. Klein’s definition of perfect aspect as indicating that Time of Situation precedes Topic Time may offer at least a partial explanation. 



In English, the perfect can combine with the various tenses to create the present perfect, past perfect, and future perfect forms: the present perfect combines present tense with perfect aspect, and so forth. Recall that tense indicates the position of Topic Time relative to the Time of Utterance; so in the present perfect, the Topic Time equals or includes the Time of Utterance. This helps to explain the “current relevance” constraint on the use of the present perfect: if the Topic Time is now, then in using the perfect to describe an event or situation in the past, we are actually “talking about” or making an assertion about the present moment.



Time adverbials like those in \REF{ex:} and \REF{ex:} generally modify the Topic Time. In the present perfect, the Topic Time is “now”; so time adverbials which locate the Topic Time in the past will be incompatible with the present perfect. The present perfect is, however, compatible with time adverbials which include the present moment, as illustrated in \REF{ex:}.


\ea
\ea I have \textit{now} built hospitals on five continents.\\
\ex I have interviewed ten students \textit{today}/*\textit{yesterday}.\\
\ex I have built five hospitals \textit{this year}/*\textit{last year}.
                       \z
\z


The use of the perfect aspect constrains the Time of Situation by indicating that it precedes Topic Time, but it does not provide a precise location in time for the Time of Situation. The result is an “indefinite past” interpretation, which stands in contrast to the simple past form of the verb. The simple past tense indicates that Topic Time precedes the Time of Utterance (past tense) and contains the Time of Situation (perfective aspect). Topic Time must be identifiable by the hearer, and so will generally be specified, with whatever degree of precision is required, by some combination of adverbial phrases, contextual clues, etc.



\citet[55]{Comrie1976} points out that past time adverbials actually can be used with the present perfect form of the verb in certain contexts, such as in non-finite clauses (\ref{ex:}a-c), or in the presence of a modal auxiliary (\ref{ex:}d-f).


\ea
\ea \textit{Having eaten} a whole pizza last night, I skipped breakfast this morning.\\
\ex Einstein’s \textit{having visited} Princeton in 1921 eventually led to his permanent\\
  appointment there.\\
\ex Charlie Chaplin was believed \textit{to have been born} on April 16, 1889.\\
\ex I should not \textit{have eaten} a whole pizza last night.\\
\ex Einstein must \textit{have visited} Princeton in 1921.\\
\ex Charlie Chaplin may \textit{have been born} on April 16, 1889.
                       \z
\z


\citet[101]{McCawley1971} observes that these environments have something in common: in these contexts past tense cannot be morphologically expressed by the normal past tense suffix \textit{-ed}. This suggests that the perfect form (\textit{have} + V\textit{-en}) may have a different function in such contexts, namely as a marker of past time, i.e., an “allomorph” of past tense.



The acceptability of the time adverbials in \REF{ex:} shows that the italicized verbs in these sentences do not have the interpretation normally associated with the present perfect form. But of course, it is also possible for a true perfect to occur with modals or in non-finite clauses, as illustrated in \REF{ex:}. So in these contexts the perfect form is ambiguous: it may either mark past tense, as in \REF{ex:}, or perfect aspect, as in \REF{ex:}. The two uses are distinguished by the interpretation of the time adverbials: if the time adverbs specify the time of the situation itself, as in \REF{ex:}, we are dealing with past tense.


\ea
\ea \textit{Having lived} in Tokyo since 1965, I know the city fairly well.\\
\ex Arthur was believed \textit{to have climbed} Mt. Fuji four times.\\
\ex Einstein must \textit{have visited} Princeton several times before he emigrated to America.
                       \z
\z


The same ambiguity can be observed in the past perfect and future perfect as well. The examples in \REF{ex:} involve true perfect aspect. The underlined time adverbials in these examples refer to Topic Time, which precedes the time of speaking in the past perfect (\ref{ex:}a) and follows the time of speaking in the future perfect (\ref{ex:}b). In both cases, perfect aspect indicates that the Situation Time (the time when Mt. Fuji is climbed) occurs before Topic Time.


\ea
\ea {In 1987}, when I first met Arthur, he \textit{had} (already) \textit{climbed} Mt. Fuji four times.\\
\ex Next Christmas, when you come to see me, I \textit{will} \textit{have} \textit{climbed} Mt. Fuji four times.
                       \z
\z


The examples in \REF{ex:} illustrate the use of the perfect form as a tense marker. In these examples the underlined time adverbials refer to the time when the event actually took place. The perfect form is used to locate the situation prior to some perspective time which is different from the time of speaking. The result is a compound tense, as discussed in \chapref{sec:21}: “past in the past” in (\ref{ex:}a), “past in the future” in (\ref{ex:}b).


\ea
\ea Einstein was awarded the Nobel prize in 1922, for a paper that he \textit{had published}\\
  in 1905.\\
\ex I will reach Tokyo at 6:00 pm, but George \textit{will} \textit{have arrived} at noon.
                       \z
\z

\subsection{Distinguishing perfect aspect vs. relative tense}\label{sec:} %3.2 /

There is a long tradition of regarding the aspectual vs. complex tense uses of the past perfect and future perfect forms as instances of polysemy.\footnote{\citet{Jespersen1924}; \citet{Comrie1976}.} However, some authors disagree with this view. \citet{Klein1994} for example argues that both the “perfect in the past” (\ref{ex:}a) and the “past in the past” (\ref{ex:}a) interpretations of the pluperfect (= past perfect) form can be assigned to a single basic sense: TSit<TT<TU. He states, “The notion of relative tense is not necessary to account for the Pluperfect nor for the Future Perfect” (1994:131).



\citet{Bohnemeyer2014} argues that perfect aspect does need to be distinguished from anterior (relative past) tense. The empirical basis for this claim is that some languages (e.g. Kalaallisut (=~West Greenlandic) and Yucatec Maya) have a perfect aspect that cannot be used to express anterior tense, while other languages (e.g. Japanese, Kituba, and Korean) have anterior tenses that cannot be used to express perfect aspect. The critical diagnostic that Bohnemeyer uses is the interpretation of time adverbials. Time adverbials can be used with the perfect aspect in Kalaallisut and Yucatec Maya to specify a topic time before which the event had occurred, as illustrated in \REF{ex:}, but not to specify the time of the event itself, as in \REF{ex:}. The opposite pattern holds for the anterior tense forms in Japanese, Kituba, and Korean: these are compatible with time adverbials that specify the time of the event itself, as in \REF{ex:}, but not with time adverbials that specify a topic time before which the event had occurred, as in \REF{ex:}.


\ea
\textsc{perfect aspect:}
\z

\ea
\ea  {In 1912}, when Theodore Roosevelt challenged William Howard Taft for the Republican nomination, both men \textit{had been elected} President of the United States. Taft was now an unpopular incumbent, Roosevelt his beloved predecessor.
\ex  When you see me again next Christmas, I \textit{will} \textit{have} \textit{graduated} from law school.
\z
\z 

\ea
\textsc{anterior tense:}\\
\ea  Arthur’s theft of government documents was discovered on May 21\textsuperscript{st}, but he \textit{had left} \\
the country on April 16\textsuperscript{th}.
\ex  I expect to reach the base camp on Tuesday afternoon; Sam \textit{will have arrived} \\
the previous evening.
\z \z


The crucial difference between perfect aspect vs anterior (= relative past) tense is this: With relative past tense the time of the described situation can be specified precisely, as seen in \REF{ex:}, because TSit must overlap with Topic Time. With perfect aspect, however, the time of the described situation is generally not specified precisely; all we know is that TSit must be sometime prior to Topic Time, as illustrated in \REF{ex:}.


\section{Arguments for polysemous aspectual senses of the English Perfect}\label{sec:} %4. /

As noted above, \citet{McCawley1971,McCawley1981}, \citet{Michaelis1994,Michaelis1998}, and \citet{Kiparsky2002} have argued that the various aspectual uses of the English perfect are in fact distinct polysemous senses. In this section we discuss some of the evidence that has been proposed in support of this hypothesis.


McCawley observed that the existential reading presupposes that a similar event could happen again, i.e., is currently possible. “In particular, the referents of the NP arguments must exist at [the time of speaking], and the event must be of a repeatable type” \citep{Kiparsky2002}. The examples in (\ref{ex:}b-c) are odd because the subject NPs are no longer alive at the time of speaking. Example \REF{ex:} is odd because the described event clearly cannot happen again. 


\ea
\ea I have never tasted fresh durian.\\
\ex \#Julius Caesar has never tasted fresh durian.\\
\ex \#Einstein has visited Princeton. (spoken after he died) (\citealt{Chomsky1970})
                       \z
\z

\ea
\#Fred has been born in Paris.  (\citealt{Kiparsky2002})
\z


\citet[33]{Leech1971} notes that the perfect form in (\ref{ex:}a) would be appropriate if the Gauguin exhibition is still running, so the addressee could still attend. Once the exhibit has closed for good, however, only (\ref{ex:}b) would be felicitous. \citet[107]{McCawley1971} points out that other circumstances could also make (\ref{ex:}a) infelicitous, for example if the addressee has “recently suffered an injury which will keep him in the hospital until long after the exhibition closes.”


\ea
\ea Have you visited the Gauguin exhibition?\\
\ex Did you visit the Gauguin exhibition?
                       \z
\z


The examples in (\ref{ex:}a--b) and (\ref{ex:}a) show that the “current possibility” requirement is a presupposition, because it applies even to negative statements and questions. They also give us reason to believe that this presupposition is better stated in terms of current possibility than repeatability, since neither sentence assumes that the event has happened in the past.



\citet[66--67]{Jespersen1931} notes that the choice between perfect and perfective can be significant because of this presupposition: “The difference between the reference to a dead man and to one still living is seen in the following quotation [] which must have been written between 1859, when Macaulay died, and 1881, when Carlyle died (note also Mr. before the latter name).”\footnote{Jespersen also points out that topicality can affect the use of the perfect: “Thus we may say: \textit{Newton has explained the movements of the moon} (i.e. in a way that is still known or thought to be correct, while \textit{Newton explained the movements of the moon from the attraction of the earth} would imply that the explanation has since been given up). On the other hand, we must use the preterit in \textit{Newton believed in an omnipotent God}, because we are not thinking of any effect his belief may have on the present age” \citep[66]{Jespersen1931}. The “effect on the present age” is relevant because the Topic Time of the present perfect is the time of speaking. Topicality also seems to be responsible for the contrast which \citet{Chomsky1970} noted between \textit{Einstein has visited Princeton}, which seems to imply that Einstein is still alive, vs. \textit{Princeton has been visited by Einstein}, which can still be felicitous after Einstein’s death.}


\ea
Macaulay \textit{did not impress} the very soul of English feeling as Mr. Carlyle, for example, \textit{has done}. [attributed to McCarthy]
\z


Kiparsky points out that the presupposition of current possibility does not attach to the recent past (or “hot news”) reading, as illustrated in \REF{ex:}. He cites this contrast as evidence that the existential and “hot news” readings are in fact distinct senses.


\ea
\ea Fred has just eaten the last doughnut.  (\citealt{Kiparsky2002})\\
\ex Einstein has just died.
                       \z
\z


A second argument is based on the observation that the various readings listed above do not all have the same truth conditions. Kiparsky notes that sentence \REF{ex:} is ambiguous between the existential vs. universal (or persistent situation) readings, and that these two readings have different truth conditions. The universal reading asserts that at all times from 1977 to the present, the speaker was in Hyderabad; it is false if there were any times within that period at which he was elsewhere. The existential reading asserts only that there was at least one time between 1977 and the present moment at which the speaker was in Hyderabad. We could easily construct a context in which the existential reading is true and the universal reading false. This suggests that we are dealing with true semantic ambiguity, rather than mere vagueness or generality.


\ea
I have been in Hyderabad since 1977.
\z


Third, the various readings have different translation equivalents in other languages. Kiparsky notes that some languages which have a perfect, e.g. German and modern Greek, would use the simple present tense rather than the perfect to express the universal reading.\footnote{See also \citet{Comrie1976}, \citet{Klein2009}.} In addition, some languages (e.g. Hungarian and Najdi Arabic) have a distinct form which expresses only the existential/experiential perfect. Mandarin seems to be another such language; see \sectref{sec:6} below.



A fourth type of evidence is seen in the following play on words (often attributed to Groucho Marx, but probably first spoken by someone else) which seems to demonstrate an antagonism between the (expected) “hot news” sense and the (unexpected) existential sense of the perfect:


\ea
I’ve had a perfectly wonderful evening, but this wasn’t it.
\z


Authors supporting the polysemy of the perfect have also pointed out that the various readings have different aspectual requirements. The universal reading, in contrast to all other uses of the perfect, is possible only with atelic situations. This would include states or activities (\ref{ex:}a--b), coerced states such as habituals (\ref{ex:}c), and accomplishments expressed in the imperfective (thus involving an atelic assertion, d). Telic situations like those in \REF{ex:} cannot normally be expressed in the universal perfect. In contrast, the perfect of continuing result illustrated in \REF{ex:} is possible only with telic events (achievements or accomplishments).


\ea
\ea I have loved Charlie Chaplin ever since I saw \textit{Modern Times}.\\
\ex Fred has carried the food pack for the past 3 hours, and needs a rest.\\
\ex I have attended All Saints Cathedral since 1983.\\
\ex I’ve been writing a history of Nepal for the past six years, and haven’t had time\\
  to work on anything else.
                       \z
\z

\ea
\ea \#Fred has arrived at the summit for the past 3 hours.\\
\ex \#I have written a history of Nepal for the past six years.
                       \z
\z


This correlation between situation type and “sense” of the perfect is clearly an important fact which any analysis needs to account for; but by itself it does not necessarily prove that the perfect is polysemous. We have already seen several cases where a single sense of a tense or aspect marker gives rise to different interpretations with different situation types (\textit{Aktionsart}), so this is a possibility that we should consider with the perfect as well. Here we leave our discussion of the English perfect, in order to examine the uses of the perfect in two other languages.


\section{Case study: Perfect aspect in Baraïn (Chadic)}\label{sec:} %5. /

Baraïn is an East Chadic (Afroasiatic) language spoken by about 6,000 people in the Republic of Chad. \citet{Lovestrand2012} discusses the contrast between perfect vs. perfective in Baraïn. He shows that the perfect form of the verb can be used for four of the five common uses of the perfect discussed above, specifically all but the experiential perfect. Examples of the four possible uses are presented in \REF{ex:}.


\ea
\ea  \textbf{Resultative}:\\
\glll kà  gūsē    ándì\\
kà  gūs-  -ē  ándì\\
S:3.m  go.out  \textsc{prf}  Andi\\
\glt ‘He has left Andi (and has not returned).’
\ex   \textbf{Universal}:\\
\glll kà  súlē    máŋgò  wàlè[25F?]ì    kúr\\
kà  súl-  -ē  móŋgò  wālō  -[25F?]i\`{}   kúr\\
S:3.m  sit  \textsc{prf}  Mongo  year  \textsc{poss}:3.m  ten\\
\glt ‘He has lived in Mongo for ten years (and lives there now).’
\ex   \textbf{Recent past}:\\
\glll kà  kólē    sòndé  kāj\\
kà  kól-  -ē  sòndé  kājē\\
S:3.m  go  \textsc{prf}  now  here\\
\glt ‘He has just left this moment.’
\ex   \textbf{Present state}:\\
\glll rámà  āt[2D0?]ē    màlpì\\
rámà  ǎt[2D0?]-  -ē  màlpì\\
Rama  remain  \textsc{prf}  Melfi\\
\glt ‘Rama has stayed in Melfi (and is there now).’\\
French: \textit{Il est resté à Melfi}.
\z \z


Lovestrand states that “what is labeled the ‘existential’ or ‘experiential’ perfect is not expressed with the Perfect, but instead with the Perfective marker.” An example is presented in \REF{ex:}.


\ea
\glll kì  kólá    āt[2D0?]á  ān[2D0?]áŋ  [272?][25F?]àménà\\
ki\`{}   kól-  -à  āt[2D0?]á  ān[2D0?]áŋ  [272?][25F?]amena\\
S:2.s  go  \textsc{pfv}  time  how.many  N’Djamena\\
\glt ‘How many times have you been to N’Djamena?’
\z


The perfect in Baraïn, in all four of its uses, entails that the situation is still true or the result state still holds at the time of speaking. Semelfactives, which do not have a result state, cannot be expressed in the perfect:


\ea
\ea \glll  kà  ás[2D0?]á    tā  āt[2D0?]á  pańi\'{ŋ}\\
kà  ás[2D0?]-  -à  tā  āt[2D0?]á  pańi\'{ŋ}\\
S:3.m  cough  \textsc{pfv}  \textsc{prtcl}  time  one\\
\glt ‘He coughed once.’
\ex \glll  \#kà  as[2D0?]e    āt[2D0?]á  pańi\'{ŋ}\\
  kà  ás[2D0?]-  -ē  āt[2D0?]á  pańi\'{ŋ}\\
S:3.m  cough  \textsc{prf}  time  one\\
\z \z


The requirement that the result state still hold true at the time of speaking is illustrated in (\ref{ex:}a). If the same event is described in the perfective, as in (\ref{ex:}b), it implies that the result state is no longer true.


\ea
\ea  \glll kà  kólá    wò  kà  láawē\\
kà  kól-  -à  wò  kà  láaw-  -ē\\
S:3.m  go  \textsc{pfv}  and  S:3.m  return  \textsc{prf}\\
\glt ‘He left but he has returned (and is still here).’
\ex \glll   kà  kólá    wò  kà  láawá    tā\\
kà  kól-  -à  wò  kà  láaw-  -à  tā\\
S:3.m  go  \textsc{pfv}  and  S:3.m  return  \textsc{pfv}  \textsc{prtcl}\\
\glt ‘He left and he returned (but he is not here now).’
\z \z


Events which result in a permanent change of state, like those in (\ref{ex:}a) and (\ref{ex:}a), must normally be expressed in the perfect. If these events are described in the perfective, as in (\ref{ex:}b) and (\ref{ex:}b), it implies that some extraordinary event has taken place to undo the result state of the described event.


\ea
\ea  \glll át[2D0?]ù    tōklē\\
át[2D0?]á  -[25F?]ù  tǒkl-  -ē\\
arm  \textsc{poss}:1.s  remove  \textsc{prf}\\
\glt ‘My arm was removed.’
\ex \glll  át[2D0?]ù    tòklá    tā\\
át[2D0?]á  -[25F?]ù  tǒkl-  -à  tā\\
arm  \textsc{poss}:1.s  remove  \textsc{pfv}  \textsc{prtcl}\\
\glt ‘My arm was removed once (but somebody reattached it).’
\z \z

\ea
\ea  \glll kà  mótē\\
kà  mót-  -ē\\
S:3.m  die  \textsc{prf}\\
\glt ‘He died.’
\ex \glll ?ka  mota\\
 kà  mót-  -à\\
S:3.m  die  \textsc{pfv}\\
\glt ‘He \textit{was} dead (but is miraculously no longer dead).’
\z \z


The inference illustrated in (\ref{ex:}--\ref{ex:}), by which the perfective signals that the result state is no longer true, seems to be an implicature triggered by the speaker’s choice not to use the perfect, where that would be possible. This inference does not arise in all contexts. For example, verbs describing main-line events in a narrative sequence can occur in the perfective without any implication that the result state is no longer true. In contrast, the requirement that the result state of an event in the perfect hold true at the time of speaking is an entailment which cannot be cancelled, as demonstrated in (\ref{ex:}b).


\ea
\ea  \glll kà  mótá    tā  wò  kà  [272?]\={\i}rē\\
kà  mót-  -à  tā  wò  kà  [272?]i\={r}  -ē\\
S:3.m  die  \textsc{pfv}  \textsc{prtcl}  and  S:3.m  resurrect  \textsc{prf}\\
\glt ‘He died, but he has been resurrected.’
\ex \glll  \#ka  mote    wo  ka  [272?]ire\\
  kà  mót-  -ē  wò  kà  [272?]i\={r}  -ē\\
S:3.m  die  \textsc{prf}  and  S:3.m  resurrect  \textsc{prf}\\
\glt (intended: ‘He has died, but he has been resurrected.’)
\z \z

\section{6. Case study: Experiential \textit{–guo} in Mandarin}\label{sec:}

In our discussion of the English perfect we noted that some languages have a perfect marker which expresses only the existential/experiential sense. Mandarin is one such language. The meaning of the verbal suffix \textit{-guo} is in some ways the polar opposite of the meaning of the perfect marker in Baraïn. While the perfect in Baraïn can express all of the standard perfect readings except the experiential, \textit{-guo} expresses only the experiential perfect. While the perfect in Baraïn requires that the result state of the event still holds true at the time of speaking, \textit{-guo} requires that the result state no longer holds true at the time of speaking.



The meaning of Mandarin \textit{guo} is similar in many ways to the existential/experiential perfect in English; but there are important differences as well. \citet{Chao1968} refers to the suffix \textit{-guo} as a marker of “indefinite past aspect”. \citet[226]{LiThompson1981} identify \textit{–guo} as a marker of “experiential aspect”, stating that it indicates that the situation has been experienced at least once, at some indefinite time in the past.\footnote{Some authors take the term “experiential aspect” quite literally, assuming that an animate experiencer must be involved. For example, \citet[144]{XiaoMcEnery2004} write: “The distinguishing feature of \textit{–guo} is that in conveys a mentally experienced situation.” \citet[267]{Smith1997} states that “sentences with\textit{–guo} ascribe to an experiencer the property of having participated in the situation.” However, \textit{–guo} can also be used in clauses which contain no animate arguments.} They provide the following minimal pair illustrating the contrast between the perfective (\ref{ex:}a), in which the described event occurs within Topic Time, vs. the experiential (\ref{ex:}b), in which the described event occurs at some arbitrary time prior to Topic Time.


\ea
\ea  \gll n[1D0?]  kànjian-le  w[1D2?]=de  y[1CE?]njìng  ma?\\
2sg  see-\textsc{pfv}  1sg=\textsc{poss}  glasses  Q\\
\glt ‘Did you see my glasses?’ (recently; I’m looking for them)
\ex \gll  n[1D0?]  kànjian-guo  w[1D2?]=de  y[1CE?]njìng  ma?\\
2sg  see-\textsc{exper}  1sg=\textsc{poss}  glasses  Q\\
\glt ‘Have you ever seen my glasses?’  [\citealt{Ma1977}:19; \citealt{LiThompson1981}:227]
\z \z


\citet{Wu2009} states: “an eventuality presented by \textit{–guo} is temporally independent of others in the same discourse.” This constraint follows from the fact that normally \textit{–guo} has indefinite time reference, and so does not establish a new Topic Time to which other clauses or sentences can refer. As a result, clauses marked with \textit{–guo} are not interpreted as a narrative sequence of events. \citet[308]{Iljic1990} provides the following contrast between the two verbal suffixes \textit{–le} and \textit{-guo}, showing that a series of clauses marked with \textit{–le} is interpreted as a chronological sequence, while the same series of clauses marked with \textit{–guo} is interpreted as a mere inventory of activities.


\ea

\ea \gll  Qùnián  w[1D2?]  \textit{zuò-le}  m[1CE?]imài,  \textit{xué-le}  jìsuànj\={\i},  \textit{shàng-le}  yèdàxué.\\
last.year  1sg  do-\textsc{pfv}  business  study-\textsc{pfv}  computer  go-\textsc{pfv}  evening.university\\
\glt ‘Last year I \textit{did} some business, (then) \textit{studied} computers, (then) \textit{attended} evening university.’ (chronological perspective)
\ex \gll  Qùnián  w[1D2?]  \textit{zuò-guo}  m[1CE?]imài,  \textit{xué-guo}  jìsuànj\={\i},  \textit{shàng-guo}  yèdàxué.\\
last.year  1sg  do-\textsc{exper}  business  study-\textsc{exper}  computer  go-\textsc{exper}  evening.          university\\
\glt ‘Last year I \textit{did} some business, (and) \textit{studied} computers, (and) \textit{attended} evening university.’ (inventory perspective)
\z \z


Examples like \REF{ex:} are sometimes cited as counter-examples to the generalization that \textit{–guo} marks indefinite time in the past. The speaker in this sentence is clearly not just claiming to have eaten food at some time in the past, but rather is stating that he has finished eating the most recently scheduled meal.


\ea
\gll W[1D2?]  ch\={\i}-guò  fàn  le.\\
1sg  eat-finish  rice  \textsc{cos}\\
\glt ‘I have already eaten.’  (\citealt{Ma1977})
\z


\citet[251]{Chao1968}, \citet[59]{Comrie1976} and Xiao \& McEnery (2004:139 ff) state that the \textit{–guò} in such examples is not the aspectual suffix but a verb root occurring as the second member of a compound verb. Both of these forms are derived from the verb \textit{guò} ‘to pass by’, and both are written with the same Chinese character. However, the aspectual suffix can be distinguished from the compound verb by phonological and morphological evidence. Phonologically, the aspectual suffix is always toneless (i.e., takes neutral tone) whereas the compound verb takes an optional 4\textsuperscript{th} tone, as marked in \REF{ex:}.\footnote{Comrie states that this 4\textsuperscript{th} tone is optional but is usually pronounced.} Morphologically, the compound verb \textit{–guò} can be followed by the perfective suffix \textit{-le}, whereas the aspectual suffix \textit{-guo} cannot. Chu (1998:39–40) shows that temporal adverbial clauses like the first clause of \REF{ex:} are another context where the compound verb \textit{–guò} rather than the aspectual suffix \textit{–guo} is used. Some authors introduce unnecessary complexity into the discussion of aspectual \textit{-guo} by failing to make this distinction.


\ea
\gll N[1D0?]  míngtian  kàn-guò  jiù  zh\={\i}dao  le.\\
2sg  tomorrow  see-finish  then  know  \textsc{cos}\\
\glt ‘After you see it tomorrow, you will know.’  (\citealt{Chen1979})
\z


Many authors have noted an interesting semantic restriction on the use of the aspectual suffix \textit{-guo}: as first observed by Chao (1968:439; cf. \citealt{Yeh1996}), there must be a “discontinuity” between Situation Time and Topic Time. If the described event produces a result state, the result state must be over before Topic Time, as seen in (\ref{ex:}a). We might represent this discontinuity as follows: TSit ${\cap}$ TT = ⌀ (here we assume that the result state is included in TSit). Some authors (e.g. \citealt{Iljic1990}, \citealt{Yeh1996}) have suggested that this discontinuity effect is merely an “inference”; but examples (\ref{ex:}a) and (\ref{ex:}a) seem to indicate that the requirement is an entailment and not just an implicature.


\ea
\ea  \gll W[1D2?]  shu\=ai-duàn-guo  tu[1D0?].\\
1sg  fall-break-\textsc{exper}  leg\\
\glt ‘I have broken my leg (before).’ (It has healed since.) [\citealt{Chao1968}; \citealt{Ma1977}:25]
\ex \gll W[1D2?]  shu\=ai-duàn-le  tu[1D0?].\\
1sg  fall-break-\textsc{pfv}  leg\\
\glt ‘I broke my leg.’  (It may be still in a cast.)  [\citealt{Chao1968}; \citealt{Ma1977}:25]
\z \z

\ea
\ea \gll T\=a  qùnián  dào  Zh\=ongguó  qù-guo,  (\#xiànzai  hái  zài  nàr  ne).\\
3sg  last.year  to  China  go-\textsc{exper}    now  still  at  there  \textsc{prtcl}\\
\glt ‘He has been to China sometime last year (\#and is still there now).’  (\citealt{Ma1977}:18)
\ex \gll  T\=a  qùnián  dào  Zh\=ongguó  qù-le,  (xiànzai  hái  zài  nàr  ne).\\
3sg  last.year  to  China  go-\textsc{pfv} now  still  at  there  \textsc{prtcl}\\
\glt ‘He went to China last year (and is still there now).’  (\citealt{Ma1977}:18)
\z \z

\ea
\ea \gll  T\=a  ài-guo  Huáng  Xi[1CE?]ojie,  (\#xiànzai  hái  ài-zhe  t\=a  ne).\\
3sg  love-\textsc{exper}  Huang  Miss now  still  love-\textsc{cont}  there  \textsc{prtcl}\\
\glt ‘He once loved Miss Huang (\#and he still loves her now).’  (\citealt{Ma1977}:18)
\ex \gll  T\=a  ài  Huáng  Xi[1CE?]ojie  le,  (xiànzai  hái  ài-zhe  t\=a  ne).\\
3sg  love  Huang  Miss \textsc{prtcl} now  still  love-\textsc{cont}  there  \textsc{prtcl}\\
\glt ‘He has fallen in love with Miss Huang (and he still loves her now).’  (\citealt{Ma1977}:18)
\z \z


Interestingly, this discontinuity requirement is (partially?) dependent on the definiteness of the affected argument (\citealt{Lin2007}; \citealt{Wu2008}; \citealt{Chen2009}). When the patient is definite, as in (\ref{ex:}a), the use of \textit{–guo} indicates that the result state no longer obtains; but when the patient is indefinite, as in (\ref{ex:}b), there is no such implication/entailment.


\ea
\ea  \gll L[1D0?]sì  nòng-huài-guo  zhè  bù  b[1D0?]jìxíng-diànn[1CE?]o.\\
Lisi  make-broken-\textsc{exper}  this  CL  laptop\\
\glt ‘Lisi has broken this laptop before.’\\
(strongly implies that the laptop has been fixed at the time of speech)
\ex \gll  L[1D0?]sì  nòng-huài-guo  y\={\i}  bù  b[1D0?]jìxíng-diànn[1CE?]o.\\
Lisi  make-broken-\textsc{exper}  one  CL  laptop\\
\glt ‘Lisi has broken a laptop before.’  [\citealt{Chen2009}; cf. \citealt{Lin2007}]\\
(no commitment as to whether the laptop has been fixed or not)
\z \z


A number of authors\footnote{\citet{Ma1977}; \citet[230]{LiThompson1981}; \citet{Yeh1996}; \citet[268]{Smith1997}.} have claimed that the situation marked by \textit{–guo} must be repeatable. If it is an event, there must be a possibility for the same kind of event to happen again. This is a well-known property of the experiential (or existential) perfect in English, and its applicability to \textit{-guo} is supported by examples like \REF{ex:}. However, this claim has been challenged by number of other authors.\footnote{\citet{Chen1979}, \citet{Iljic1990}, Xiao and McEnery (2004: 147–48), Pan and \citet{Lee2004}, \citet{Lin2007}.} Consider the contrast in \REF{ex:}. Neither being old nor young are states that are repeatable for a single individual. The contrast between the two sentences seems best explained in terms of the discontinuity requirement: a person who is no longer young can still be alive, but not a person who is no longer old.


\ea
\gll *T\=a  s[1D0?]-guo.\\
 3sg  die-\textsc{exper}\\
\glt (intended: ‘He has died before.’)  (\citealt{Ma1977}:15)
\z

\ea
\ea  \gll N[1D0?]  yě  niánq\={\i}ng-guo\\
you  also  young-\textsc{exper}\\
\glt ‘You also have been young before.’ 
\ex \gll *N[1D0?]  yě  l[1CE?]o-guo\\
you  also  old-\textsc{exper}\\
\glt ‘You have also been old before.’
\z \z


It appears that all of the data which has been proposed in support of the repeatability hypothesis can equally well be explained in terms of the discontinuity requirement. Support for the idea that discontinuity, rather than repeatability, is the operative factor comes from the observation that “repeatability” effects are sensitive to definiteness in exactly the same way as demonstrated above for the discontinuity requirement; this is illustrated in \REF{ex:}. The fact that it is possible to use \textit{–guo} when talking about the actions of dead people, as in (\ref{ex:}b), gives further support to the claim that there is no repeatability requirement in Mandarin. Such examples are normally quite unnatural in the English experiential/existential perfect.


\ea
\ea \gll  *Columbus  faxian-guo  meizhou\\
  Columbus  discover-\textsc{exper}  America\\
\glt (intended: ‘Columbus has discovered America before.’)  (\citealt{Yeh1996}:153)
 \ex \gll Columbus  faxian-guo  yi  ge  xiao  dao.\\
Columbus  discover-\textsc{exper}  one  CL  small  island\\
\glt ‘Columbus has discovered a small island before.’  (\citealt{Yeh1996}:163)
\z \z


It is useful to compare the semantic effect of the aspectual suffix \textit{–guo} in various situation types (\textit{Aktionsart}). With stative predicates, \textit{–guo} indicates that the state no longer exists \REF{ex:}. Therefore, permanent states cannot normally be marked with \textit{–guo} \REF{ex:}.


\ea
\ea \gll  Zh\=ang  Xi[1CE?]ojie  guòqù  pàng-guo.\\
Zhang  Miss  in.past  fat-\textsc{exper}\\
\glt ‘Miss Zhang has been fat.’ (implying she is not fat now)  (\citealt{Ma1977}:23)
\ex \gll Měiguo  níuròu  yě  guì-guo.\\
America  beef  also  expensive-\textsc{exper}\\
\glt ‘Beef America has also been expensive (in the past but not now).’  (\citealt{Ma1977}:23)
\ex \gll  T\=a  zài  Zh\=ongguó  zhù-guo  s\=an  nián.\\
3sg  at  China  live-\textsc{exper} three  year\\
\glt ‘He has lived in China for three years before (but does not live there now).’  (\citealt{Ma1977}:20)
\z \z

\ea
\gll *Dangdi  nongmin  zhidao-guo  na  gezha  youdu.\\
  local  farmer  know-\textsc{exper}  that  chrome.dreg  poisonous\\
\glt (intended: ‘Local farmers knew that those chrome dregs were poisonous.’)\\
   {}[Xiao \& \citealt{McEnery2004}:149]
\z


With atelic events such as activities \REF{ex:} and non-culminating accomplishments \REF{ex:}, the suffix -\textit{guo} has the same interpretation as the perfective suffix \textit{–le}, indicating that the event has terminated.


\ea
\gll Lisi  da-guo  wangqiu.\\
Lisi  play-\textsc{exper}  tennis\\
\glt ‘Lisi has played tennis before.’  (\citealt{Smith1997}:267)
\z

\ea
\ea \gll \#Wo  xie-guo  gei  Wang  de  xin,  hai  zai  xie.\\
 1sg  write-\textsc{guo}  to  wang  \textsc{lnk} letter,  still  \textsc{prog}  write\\
\glt ‘I wrote Wang’s letter and am still writing it.’\\
\ex \gll  Wo  xie-guo  gei  Wang  de  xin,  keshi  mei  xie-wan.\\
1sg  write-\textsc{guo}  to  wang  \textsc{lnk} letter,  but  not  write-finish\\
\glt ‘I wrote Wang’s letter but didn’t finish it.’  (\citealt{Smith1997}:267)
\z \z


In light of what we have said above, we would predict that the aspectual suffix \textit{–guo} cannot occur with telic predicates whose result state is permanent, because this would mean that discontinuity with Topic time is impossible. This prediction turns out to be true when the patient (or affected argument) is definite. However, as noted above, the discontinuity requirement does not apply when the patient is indefinite; so the aspectual suffix \textit{–guo} is possible in such contexts.



The examples in (\ref{ex:}a--b) contain a Result Compound Verb (RCV), which means that the culmination of the event is entailed. As predicted, \textit{–guo} is not allowed when the object NP is definite (\ref{ex:}a), but is possible when the object NP is indefinite (\ref{ex:}b). However, (\ref{ex:}c) contains the simple root ‘kill’ with no RCV, and so the culmination of the event would normally be implicated but not entailed. In this example, \textit{–guo} functions as an explicit indicator that the result state was not achieved.


\ea
\ea \gll *T\=a  sh\=a-s[1D0?]-guo  nèi-ge  rén.\\
 3sg  kill-die-\textsc{exper}  that-\textsc{class}  person\\
\glt (intended: ‘He has killed that person.’)  (\citealt{Ma1977}:23)

\ex \gll  T\=a  sh\=a-s[1D0?]-guo  s\=an-ge  rén.\\
3sg  kill-die-\textsc{exper}  three-\textsc{class}  person\\
\glt ‘He has killed three people.’  (\citealt{Ma1977}:23)
\ex \gll  T\=a  sh\=a-guo  nèi-ge  rén.\\
3sg  kill-\textsc{exper}  that-\textsc{class}  person\\
\glt ‘He tried (at least once) to kill that person (without success).’  (\citealt{Ma1977}:23)
\z \z


A similar pattern is seen in \REF{ex:}. the aspectual suffix \textit{–guo} can occur with the predicate ‘die’ only when the patient is indefinite (\ref{ex:}c). In (\ref{ex:}d), which \citet{Chu1998} and Xiao \& \citet{McEnery2004} describe as a figurative use of the word ‘die’, \textit{–guo} functions as an indicator that the result state was not achieved.


\ea
\ea \gll  *T\=a  s[1D0?]-guo.\\
 3sg  die-\textsc{exper}\\
\glt (intended: ‘He has died before.’)  (\citealt{Ma1977}:15)
\ex \gll T\=a  s[1D0?]-le\\
3sg  die-\textsc{pfv}\\
\glt ‘He died.’  (\citealt{Ma1977}:15)
\ex \gll  You  rén  zai  zhe  tiao  he  li  yan-si-guo.\\
have  person  at  this  CL  river  in  drown-die-\textsc{exper}\\
\glt ‘Someone has drowned in this river (before).’  (\citealt{Yeh1996}:163)
\ex \gll  W[1D2?]  s[1D0?]-guo  hao-ji-ci\\
1sg  die-\textsc{pfv} quite.a.few.times\\
\glt ‘I almost died quite a few times.’  (\citealt{Chu1998}:41)
\z \z


\citet{HuangDavis1989} point out that \textit{–guo} can also be used to indicate partial affectedness of a definite object, another way in which the culmination of the event might not be achieved:


\ea
\ea \gll  Gou  gangcai  chi-le  ni  de  pingguo.\\
dog  just.now  eat-\textsc{pfv}  you  \textsc{poss}  apple\\
\glt ‘The dog just ate your apple.’
\ex \gll  Gou  gangcai  chi-guo  ni  de  pingguo.\\
dog  just.now  eat-\textsc{exper}  you  \textsc{poss}  apple\\
\glt ‘The dog just took a bite of your apple.’  [\citealt{HuangDavis1989}:151]
\z \z

\section{Conclusion}\label{sec:} %7. /

We have discussed a number of different uses of the perfect in various languages. What all of these various uses have in common is the fact that (all or part of) the Situation Time precedes Topic Time. As mentioned at the beginning of this chapter, this is the component of meaning which \citet{Klein1992} identifies as the defining feature of perfect aspect.



\furtherreading



Comrie (1976, ch. 3) is a foundational work, and still a good place to start. \citet{Portner2011} and \citet{Ritz2012} provide good overviews of the empirical challenges and competing analyses for the perfect.


\subsubsection{Discussion exercises:}\label{sec:}

\textbf{A:} Identify the sub-type (i.e., the semantic function: \textsc{Experiential, Universal, Result}, or “\textsc{hot news}”) of the present perfect forms in the following examples:

\begin{enumerate}
\item Russia \textit{has} just \textit{accused} the American curling team of doping.
\item Donald \textit{has visited} Brazil three times.
\item Horace \textit{has been playing} that same sonata since four o’clock.
\item The Prime Minister \textit{has resigned}; it happened several weeks ago, but we still don’t know who the next Prime Minister will be.
\item Martha \textit{has known} about George’s false teeth for several years.
\end{enumerate}
\subsubsection{Homework exercise: Tok Pisin Tense-Aspect markers}\label{sec:}

Based on the examples provided below, describe the Tok Pisin Tense-Aspect system and suggest an appropriate label for each of the five italicized grammatical markers (e.g. \textit{subjunctive mood}, \textit{iterative aspect}, etc.). These markers are glossed simply as ‘\textsc{aux’}. Some of these forms can also be used as independent verbs, but you should consider those those meanings (shown in the section headings) to be distinct senses. Base your description on the ‘\textsc{aux’} functions only. You can ignore the somewhat mysterious “predicate marker” \textit{i}.


\paragraph{A. \textit{bin}}
\ea
\gll   Bung  i  \textit{bin}  stat  long  Mande  na  \textit{bai}  pinis  long  Fraide.\\
meeting  \textsc{pred}  \textsc{aux}  start  at  Monday  and  \textsc{aux}  end  at  Friday\\
\glt ‘The meeting began on Monday and will finish on Friday, April 22.’ [\citealt{Sebba1997}: p.21] 
\z

\ea
\gll  Asde/\#Tumora  mi  \textit{bin}  lukim  tumbuna  bilong  mi.\\
yesterday/\#tomorrow  1sg  \textsc{aux}  see  grandparent  \textsc{poss}  1sg\\
\glt ‘Yesterday/\#tomorrow I saw my grandparent.’
\z

\ea
\gll  Wanem  taim  sik  i  \textit{bin}  kamap  nupela?\\
what  time  illness  \textsc{pred}  \textsc{aux}  appear  new\\
\glt ‘When did the illness first appear?’  [Verhaar]
\z

\ea
\gll   Ol  tumbuna  i  no  \textit{bin}  wari  long  dispela.\\
\textsc{pl}  ancestor  \textsc{pred}  not  \textsc{aux}  worry  about  this\\
\glt ‘The ancestors did not worry about this.’  [Verhaar]
\z

\ea
\gll  ol  i  \textit{bin}  slip  long  haus  bilong  mi.\\
3pl  \textsc{pred}  \textsc{aux}  sleep  at  house  \textsc{poss}  1sg\\
\glt ‘They were sleeping in(side) my house.’  [Wohlgemuth]
\z

\paragraph{B. \textit{bai}}
\ea
\gll Long  wanem  taim  \textit{bai}  yu  go?\\
at  what  time  \textsc{aux}  2sg  go\\
\glt ‘At what time will you go?’  (\citealt{Dutton1973}:52)
\z

\ea
\gll Tumora/\#Asde  \textit{bai}  mi  askim  em.\\
tomorrow/\#yesterday  \textsc{aux}  1sg  ask  3sg\\
\glt ‘Tomorrow/\#yesterday I will ask him/her.’
\z

\ea
\gll Ating  apinun  \textit{bai}  mi  traim  pilai  ping-pong  namba.wan  taim.\\
maybe  afternoon  \textsc{aux}  1sg  try  play  ping-pong  first  time.’\\
\glt ‘Maybe this afternoon I will try to play ping-pong for the first time.’  [Verhaar 153]
\z

\ea
\gll   Sapos  yu  kaikai  planti  pinat  \textit{bai}  yu  kamap  strong  olsem  phantom.\\
if  2sg  eat  much  peanut  \textsc{aux}  2sg  become  strong  like  phantom\\
\glt ‘If you eat many peanuts, you will become strong like the phantom.’ [Wohlgemuth]
\z

\paragraph{C. \textit{save} (short form: \textit{sa})  [main verb sense: ‘know’]}
\ea
\gll Mipela  i  no  \textit{save}  kaikai  bulmakau.\\
1pl.\textsc{excl}  \textsc{pred}  \textsc{neg}  \textsc{aux}  eat  cow\\
\glt ‘We don’t (customarily) eat beef.’  (\citealt{Dutton1973}:64)
\z

\ea
\gll Mi  \textit{save}  wokabaut  go  wok.\\
1sg  \textsc{aux}  walk  go  work\\
\glt ‘I always walk to work.’  [Faraclas; cited in \citealt{Holm2000}:184]
\z

\ea
\gll  Long  nait  mi  slip  na  ol  natnat  i  \textit{save}  kam  long  haus  bilong  mi.\\
at  night  1sg  sleep,  and  \textsc{pl}  mosquitoes  \textsc{pred}  \textsc{aux}  come  to  house  \textsc{poss}  1sg\\
\glt ‘At night I sleep, and then the mosquitoes come into my house.’  [Verhaar]
\z

\ea
\gll Mipla  stap  lo(ng)  skul,  ol  ami  ol  \textit{sa}  pait  wantem  ol  man  ia.\\
1pl.\textsc{excl}  be  in  school,  \textsc{pl}  soldier  3\textsc{pl}  \textsc{aux}  fight  with  \textsc{pl}  man  here\\
\glt ‘When we were in school, the soldiers used to fight with the men (rebels).’\\
{}[\citealt{Smith2002}:132; East New Britain]
\z

\paragraph{D. \textit{stap}  [main verb sense: ‘be, stay, remain’]}
\ea
\gll Ol  i  kaikai  i \textit{stap}.\\
3pl  \textsc{pred}  eat  \textsc{pred}  \textsc{aux}\\
\glt ‘They are/were eating.’  [Verhaar 107]
\z

\ea
\gll   Ol  lapun  meri  i  subim  ka  i  go  i \textit{stap}.\\
pl  old  woman  \textsc{pred}  push  car  \textsc{pred}  go  \textsc{pred}  \textsc{aux}\\
\glt ‘The old women are/were pushing a car.’  (\citealt{Dutton1973}:149)
\z

\ea
\gll  Dua  i  op  nating  i \textit{stap}.\\
door  \textsc{pred}  open  just  \textsc{pred}  \textsc{aux}\\
\glt ‘The door was just open like that…’  [Verhaar 108]
\z

\ea
\gll Em  i tisa  i \textit{stap}  yet.\\
3sg  \textsc{pred} teacher  \textsc{pred  aux}  still\\
\glt ‘He is still a teacher.’  (\citealt{Dutton1973}:148)
\z

\ea
\gll  Hamas  de  pikinini  i  sik  i \textit{stap}?\\
how.many  day  child  \textsc{pred}  sick  \textsc{pred}  \textsc{aux}\\
\glt ‘How many days has the child been sick?’  [Verhaar 108]
\z

\ea
\gll  Taim  em  i  kam  i lukim  Dogare\\
time  3sg  \textsc{pred}  come  \textsc{pred}  see  (name)\\
\z

\ea
\gll   i  sindaun  tanim  smok  i  \textit{stap}.\\
\textsc{pred}  sit  roll  smoke  \textsc{pred}  \textsc{aux}\\
\glt ‘When he came he saw Dogare sitting down rolling a cigarette.’ (\citealt{Dutton1973}:145)
\z

\ea \gll  \textit{Bai}  sampela  ol  i  toktok  i \textit{stap}  na\\
\textsc{aux}  some  3pl  \textsc{pred}  talk  \textsc{pred}  \textsc{aux}  and\\
\z

\ea
 \gll  ol  i  no  harim  gut  tok  bilong  yu.\\
3pl  \textsc{pred}  not  listen  well  talk  \textsc{poss}  2sg\\
\glt ‘Some of them will be talking and not listen well to your speech.’ [Wohlgemuth]
\z

\paragraph{E. \textit{pinis}  [main verb sense: ‘finish, stop, complete’]}
\ea
\gll  Mipela  i  wokim  sampela  haus  \textit{pinis}.\\
1pl.\textsc{excl}  \textsc{pred}  build  some  house  \textsc{aux}\\
\glt ‘We [excl.] have built some houses.’  [Verhaar]
\z

\ea
\gll  Gavman  i  putim  \textit{pinis}  planti  didiman.\\
government  \textsc{pred} place  \textsc{aux}  many  agricultural.officer\\
\glt ‘The government has appointed many agricultural officers.’  [Verhaar]
\z

\ea
\gll  Dok  i  dai  \textit{pinis}.\\
dog  \textsc{pred}  die  \textsc{aux}\\
\glt ‘The dog has died/is dead.’  [Verhaar]
\z

\ea
\gll  Mi  lapun  \textit{pinis}.\\
1sg  old.person  \textsc{aux}\\
\glt ‘I am already old.’ Or: ‘I have grown old.’  [Verhaar]
\z

\ea \ea
\gll a. Ol  i  bikpela  \textit{pinis}.\\
3pl  \textsc{pred}  big  \textsc{aux}\\
\glt ‘They have become big/are grown-ups (now).’  [Verhaar 117]
\ex \gll  *Ol  i  liklik  \textit{pinis}.\\
 3pl  \textsc{pred}  small  \textsc{aux}\\
\glt (intended: ‘They are already small’ or ‘they were small once.’)
\z \z

\ea
\gll Pen  i  stap  longpela  taim  \textit{pinis},  o,  nau  tasol  em  i  kamap?\\
pain  \textsc{pred}  exist  long  time  \textsc{aux}  or  now  only  3sg  \textsc{pred}  become\\
\glt ‘Has the pain been there for a long time, or has it just started?  [Verhaar]
\z

\ea
\gll  Em  i  kamap  meija  \textit{pinis}  taim  mipela  i  harim  dispela  stori  hia.\\
3sg  \textsc{pred}  become  major  \textsc{aux}  time  1pl.\textsc{excl}  \textsc{pred}  hear  this  story  here\\
\glt ‘He had become a major by the time we [excl.] heard this story.’  [Verhaar]
\z

\ea
\ea \gll  Esra  i  sanap  long  dispela  ples\\
Esra  \textsc{pred}  stand  at  this  place\\
\glt ‘Ezra stood on this platform (while reading the Law).’  [Neh. 8:4]
\ex \gll  Man  i  sanap  \textit{pinis}.\\
man  \textsc{pred}  stand  \textsc{aux}\\
\glt ‘The man has stood up (and is standing now).’  [Liisa Berghäll]
\z \z

\ea
\ea \gll wanpela  diwai  i  sanap  namel  tru\\
one  tree  \textsc{pred}  stand  middle  very\\
\glt ‘One tree stood right in the middle (of the Garden).’  [Gen. 3:3]
\ex \gll  \#Diwai  i  sanap  \textit{pinis}.\\
  tree  \textsc{pred}  stand  \textsc{aux}\\
\glt ‘The tree has stood up (and is standing now).’  [Liisa Berghäll]
\z \z



\begin{verbatim}%%move bib entries to  localbibliography.bib
\begin{styleStylei}
Bibliography


Abbott, Barbara. 2010. \textit{Reference}. Oxford Surveys in Semantics and Pragmatics 2. Oxford University Press.



Adams, Ernest W. 1970. Subjunctive and indicative conditionals. \textit{Foundations of Language} 6: 89–94.



Aikhenvald, Alexandra Y. 2004. \textit{Evidentiality}. Oxford: Oxford University Press.



Aikhenvald, Alexandra Y. 2009. Semantics and grammar in clause linking. In: Dixon, R.M.W., and Alexandra Aikhenvald (eds.), \textit{The Semantics of Clause Linking: a cross-linguistic typology}. Oxford, UK: Oxford University Press, pp. 380–402.\\
\url{https://espaces.edu.au/tla/directors/alexandra-aikhenvald/selected-papers/typology-and-related-issues/the-semantics-of-clause-linking/view} 



Akatsuka, Noriko. 1985. Conditionals and the epistemic scale. Language 61.3:625–639.



Allwood, Jens, Lars-Gunnar Andersson, \& Östen Dahl. 1977. \textit{Logic in linguistics}. Cambridge \& New York: Cambridge University Press.



Arka, I Wayan. 2005. Speech levels, social predicates and pragmatic structure in Balinese: a lexical approach. \textit{Pragmatics} 15:2/3.169-203.  \url{http://elanguage.net/journals/pragmatics/article/download/490/418} 



Ashton, E.O. 1944. \textit{Swahili grammar (including intonation)}. London: Longman



Austin, J.L. 1956. Ifs and cans. \textit{Proceedings of the British Academy} 42:107–132.



Austin, J.L. 1961. Performative Utterances. In \textit{Philosophical Papers}, ed. by J. O. Urmson \& G. J. Warnock. Oxford: Oxford University Press (2\textsuperscript{nd} edition, 1970).



Austin, J. L. 1962. \textit{How to Do Things with Words}, ed. J. O. Urmson and Marina Sbisá. Oxford: Clarendon Press and Cambridge, MA: Harvard University Press.



Bach, Emmon. 1989. \textit{Informal lectures on formal semantics}. Albany, NY: State University of New York Press.



Bach, Kent. 1994. Conversational Impliciture. \textit{Mind and Language} 9: 124–162.



Bach, Kent. 2010. Impliciture vs explicature: What’s the difference? In Soria \& Romero (eds.), \textit{Explicit Communication: Robyn Carston’s Pragmatics}, 126–137. Palgrave.



Baker, Mark C. 1995. On the absence of certain quantifiers in Mohawk. In Bach, E., E. Jelinek, A. Kratzer, and B.H. Partee (eds), \textit{Quantification in Natural Languages}, 21–58. Dordrecht \& Boston: Kluwer Academic Publishers.



Bar-el, Leora, Henry Davis, and Lisa Matthewson. 2005. On non-culminating accomplishments. \textit{Proceedings of the North Eastern Linguistics Society 35}. Amherst, MA: GLSA. \url{http://faculty.arts.ubc.ca/lmatthewson/pdf/accomplishments.pdf} 



Bariki, Ozidi. 2008. On the relationship between translation and pragmatics. \textit{International Journal of Translation} 20:1-2.67–75.



Barker, Chris. 2002. Lexical Semantics. \textit{Encyclopedia of Cognitive Science}. Macmillan Reference Ltd.



Barnes, Janet. 1984. Evidentials in the Tuyuca verb. \textit{International Journal of American Linguistics} 50: 255–271.



Barwise, Jon \& Robin Cooper. 1981. Generalized quantifiers in natural language. \textit{Linguistics and Philosophy} 4:159–219.



Beekman, John, \& John Callow. 1974. \textit{Translating the Word of God}. Grand Rapids, MI: Zondervan.



Bennett, Jonathan. 1982. Even if. \textit{Linguistics and Philosophy} 5: 403–18.



Bennett, Jonathan. 2003. \textit{A philosophical guide to conditionals}. Oxford University Press.



Bhatt, R. and R. \citet{Pancheva2006}. “Conditionals”, \textit{The Blackwell Companion to Syntax, v. 1}, Blackwell, pp. 638–687.



Binnick, Robert I. 2006. Aspect and aspectuality. In Bas Aarts \& April McMahon (eds), \textit{The Handbook of English Linguistics}, 244–268. Malden, MA: Blackwell Publishing.



Birner, Betty J. 2012/2013. \textit{Introduction to Pragmatics}. Malden, MA \& Chichester, UK:Wiley-Blackwell.



Bittner, Maria. 1995. Quantification in Eskimo: A challenge for compositional semantics. In Bach, E., E. Jelinek, A. Kratzer, and B.H. Partee (eds), \textit{Quantification in Natural Languages}, 59–80. Dordrecht \& Boston: Kluwer Academic Publishers.



Blackburn, P., M. de Rijke, \& Y. Venema. 2008. \textit{Modal Logic}. Cambridge University Press.



Blum-Kulka, S., J. House \& G. Kasper (eds). 1989. \textit{Cross-Cultural Pragmatics: Requests and Apologies}. Advances in Discourse Processes, Volume XXXI. New Jersey: Ablex Publishing Corporation.



Bohnemeyer, Jürgen. 2014. Aspect vs. relative tense: The case reopened. \textit{Natural Language and Linguistic Theory} 32:917–954.



Bolinger, Dwight. 1967. Adjectives in English: Attribution and predication. \textit{Lingua} 18: 1–34.



Botne, Robert. 2012. Remoteness distinctions. In R. Binnick (ed.), \textit{The Handbook of Tense and Aspect}, 536–562. Oxford University Press.



Brown, P. \& S. Levinson. 1978. Universals in language usage: Politeness phenomena. In E. N. Goody (ed.), \textit{Questions and Politeness: Strategies of Social Interaction} (pp. 56–289). Cambridge: Cambridge University Press.



Bross, Fabian. 2012. German modal particles and the common ground. In: \textit{Helikon. A Multidisciplinary Online Journal}, 2: 182–209. \\
\url{http://helikon-online.de/2012/Bross_Particles.pdf} 



Bybee, Joan L., Revere Perkins, and William Pagliuca. 1994 \textit{The Evolution of Grammar: Tense, Aspect, and Modality in the Languages of the World}. Chicago: University of Chicago Press.



Bybee, Joan. 1985. \textit{Morphology: a study of the relation between meaning and form}. Amsterdam: John Benjamins.



Cable, Seth. 2013. Beyond the past, present, and future: towards the semantics of ‘graded tense’ in G\~\ik\~uy\~u. \textit{Natural Language Semantics} 21(3): 219–276.



Cann, Ronnie. 2011. Sense Relations. \textit{Semantics: An International Handbook of Natural Language and Meaning, Vol. 1}, ed. by Claudia Maienborn, Klaus von Heusinger, and Paul Portner. Walter de Gruyter.



Carlson, Greg. 1977. Reference to Kinds in English. PhD thesis, MIT.



Carston, Robyn. 1988. Implicature, explicature and truth-theoretic semantics.  In Kempson, R. (ed.), \textit{Mental representations: The interface between language and reality}. Cambridge: Cambridge University Press. 155–181. Reprinted in Davis, S. (ed.) 1991.



Carston, Robyn. 1998. Informativeness, relevance and scalar implicature. In R. Carston \& S. Uchida (eds.), \textit{Relevance Theory: Applications and Implications}, 179–236. Amsterdam: John Benjamins.



Carston, Robyn. 2002. \textit{Thoughts and Utterances: The Pragmatics of Explicit Communication}. Oxford: Blackwell.



Carston, Robyn. 2004. Relevance theory and the saying/implicating distinction. In L. Horn and G. Ward (eds.), \textit{The Handbook of Pragmatics}, 633–656. Oxford: Blackwell.



Carston, Robyn \& Alison Hall. 2012. Implicature and explicature. In Hans-Jörg Schmid (ed.) \textit{Cognitive Pragmatics}, Vol. 4 of \textit{Handbooks in Pragmatics}, 47–84. Berlin: Mouton de Gruyter.



Chafe, Wallace L. 1976. Givenness, contrastiveness, definiteness, subjects, topics and point of view. In Charles N. Li (ed.), \textit{Subject and Topic}. New York: Academic Press, 27–55.



Chang, Chen Chung \& H. Jerome Keisler. 1990. \textit{Model Theory}. Studies in Logic and the Foundations of Mathematics, Volume 73 (3\textsuperscript{rd} edition). Amsterdam: Elsevier. (First edition, 1973)



Chao, Yuen-Ren. 1968. \textit{A Grammar of Spoken Chinese}. University of California Press, Berkeley.



Chen, Chien-chou. 2009. Experientiality and reversibility of the aspectual morpheme \textit{guo} in Mandarin Chinese: Temporal and atemporal perspectives. \textit{Dong Hwa Journal of Humanities} ([6771?][83EF?][4EBA?][6587?][5B78?][5831?]) 1:247–94.  \url{http://ir.ndhu.edu.tw/bitstream/987654321/4910/1/14-247-294.pdf} 



Chen, Gwang-tsai. 1979. The aspect markers \textit{le}, \textit{guo}, and \textit{zhe} in Mandarin Chinese. \textit{Journal of the Chinese Language Teachers Association} 14(2): 27–46.



Cherchia, Gennaro and Sally McConnell-Ginet. 1990. \textit{Meaning and Grammar: An introduction to semantics}. Cambridge MA: MIT Press.



Chomsky, Noam. 1970. Deep structure, surface structure, and semantic interpretation. In R. Jakobson and S. Kawamoto (eds.), \textit{Studies in General and Oriental Linguistics}, pp. 62–119. Tokyo: TEC Corporation.



Chu, Chauncey. 1998. \textit{A discourse grammar of Mandarin Chinese}. New York: Peter Lang Publishing.



Chung, Sandra and Alan Timberlake. 1985. Tense, aspect, and mood. In Shopen (ed.) vol. 3, pp. 202–258.



Cohen, L. Jonathan. 1971. The logical particles of natural language. In Y. Bar-Hillel (ed.), \textit{Pragmatics of Natural Language}. Dordrecht: Reidel, 50-68.



Cole, Peter. 1982. \textit{Imbabura Quechua}. Lingua descriptive studies, no. 5. Amsterdam: North-Holland.



Comrie, Bernard. 1976. \textit{Aspect}. Cambridge University Press.



Comrie, Bernard. 1985. \textit{Tense}. Cambridge University Press.



Comrie, Bernard. 1986. Conditionals: a typology. In: Traugott, Elizatbeth C., Alice ter Meulen, Judy Snitzer Reilly, \& Charles Ferguson (eds.), \textit{On Conditionals}. Cambridge: Cambridge University Press, 77–99.



Coppock, Elizabeth. 2016. \textit{Semantics boot camp}. Unpublished ms.\\
\url{http://eecoppock.info/semantics-boot-camp.pdf} 



Cotterell, Peter and Max Turner. 1989. \textit{Linguistics and Biblical Interpretation}. Illinois: InterVarsity Press.



Croft, William, \& D. Alan Cruse. 2004. \textit{Cognitive Linguistics}. Cambridge: Cambridge University Press.



Cruse, D. Alan. 1986. \textit{Lexical Semantics}. Cambridge: Cambridge University Press.



Cruse, D. Alan. 2000. \textit{Meaning in language: An introduction to semantics and pragmatics}. New York \& Oxford: Oxford University Press.



Cruse, D. Alan. 2004. “Lexical facets and metonymy.” \textit{Revista Ilha do Desterro: A Journal of English Language, Literatures in English and Cultural Studies} 47:73–96.\\
\url{http://www.periodicos.ufsc.br/index.php/desterro/article/view/7348/6770}



de Haan, Ferdinand. 1997. \textit{The interaction of modality and negation}. New York: Garland.



de Haan, Ferdinand. 1999. Evidentiality and epistemic modality: Setting boundaries. \textit{Southwest Journal of Linguistics} 18, 83–101.



de Haan, Ferdinand. 2005. Encoding speaker perspective: Evidentials. In Z. Frajzyngier \& D. Rood (eds.), \textit{Linguistic diversity and language theories}, 379–97. Amsterdam: Benjamins.



de Haan, Ferdinand. 2006. Typological approaches to modality. In William Frawley (ed.), \textit{Modality}, 27–69. Berlin: Mouton de Gruyter.



de Haan, Ferdinand. 2012. Evidentiality and mirativity. In  Robert I. Binnick (ed.), \textit{The Oxford Handbook of Tense and Aspect}, pp. 1020–1046. Oxford University Press.



De Swart, Henriette. 1998. Aspect Shift and Coercion. \textit{Natural Language and Linguistic Theory} 16: 347–385.



Declerck, Renaat. 1991. \textit{Tense in English}. London: Routledge.



DeLancey, Scott. 1995. Verbal Case Frames in English and Tibetan. unpublished ms., Department of Linguistics, University of Oregon, Eugene, OR.



DeLancey, Scott. 2000. The universal basis of case. \textit{Logos and Language} 1:2, 1-15.



Dell, Francois. 1983. An aspectual distinction in Tagalog. \textit{Oceanic Linguistics} 22-23:175–206.



DeRose, Keith \& Richard E. Grandy. 1999. Conditional assertions and “biscuit” conditionals. \textit{Noûs} 33(3). 405–420.



Dowty, David R. 1979. \textit{Word meaning and Montague grammar: the semantics of verbs and times in generative semantics and in Montague's PTQ}.  Dordrecht \& Boston: Reidel.



Dowty, David. 1991. Thematic proto-roles and argument selection. \textit{Language} 67:547–619.



Dowty, David R., Robert E. Wall, and Stanley Peters. 1981. \textit{Introduction to Montague Semantics}. Dordrecht: Reidel.



Drubig, Hans B. 2001. On the syntactic form of epistemic modality. Unpublished ms. Tübingen, University of Tübingen.



Ebert, Christian, Cornelia Endriss, and Stefan Hinterwimmer. 2008. A unified analysis of indicative and biscuit conditionals as topics. In \textit{Proceedings of Semantics and Linguistic Theory 18}, ed. Tova Friedman and Satoshi Ito.



Egner, Inge. 2002. The speech act of promising in an intercultural perspective. \textit{SIL Electronic Working Papers}. \url{http://www.sil.org/silewp/2002/001/SILEWP2002-001.pdf} .



Enderton, Herbert B. 1977. \textit{Elements of set theory}. New York: Academic Press.



Engelberg, Stefan. 2011. Lexical Decomposition: Foundational Issues. In Claudia Maienborn, Klaus von Heusinger, Paul Portner (eds.), \textit{Semantics: An international handbook of natural language meaning, vol. 1}. Berlin, New York: de Gruyter, 122–142.



Ernst, Thomas. 2009. Speaker-oriented adverbs. \textit{Natural Language and Linguistic Theory} 27: 497–544. DOI 10.1007/s11049-009-9069-1



Faller, Martina. 2002. Semantics and pragmatics of evidentials in Cuzco Quechua. Unpublished PhD thesis, Stanford: Department of Linguistics, Stanford University.\\
\url{http://personalpages.manchester.ac.uk/staff/martina.t.faller/documents/thesis-a4.pdf} 



Faller, Martina. 2003. “Propositional- and illocutionary-level evidentiality in Cuzco Quechua.” In Jan Anderssen, Paula Menendez-Benito, and Adam Werle, eds. \textit{The Proceedings of SULA 2(The Second Conference on the Semantics of Under-Represented Languages in the Americas)},Vancouver, BC: 19–33. GLSA, University of Massachusetts, Amherst.\\
\url{http://www.umass.edu/linguist/events/SULA/SULA_2003_cd/files/faller.pdf}



Faller, Martina. 2006. Evidentiality above and below speech acts. Unpublished ms. \url{http://semanticsarchive.net/Archive/GZiZjBhO/} 



Fillmore, Charles. 1970. The grammar of \textit{hitting} and \textit{breakin}g. In \textit{Readings in English transformational grammar}, ed. by Roderick Jacobs and Peter Rosenbaum, 120–33. Washington, DC: Georgetown University Press.



Fillmore, Charles. 1977. The case for case reopened. In \textit{Grammatical relations}, ed. by Peter Cole and Jerrold Sadock, 59–81. Syntax and Semantics 8. New York: Academic Press.



Fillmore, Charles, \& Beryl T. Atkins. 2000. Describing Polysemy: The Case of ‘Crawl.’ In Ravin, Yael \& Claudia Leacock (eds.), \textit{Polysemy: Theoretical and computational approaches}, pp 1–29. Oxford: Oxford University Press.



Finegan, Edward. 1999. \textit{Language: Its Structure and Use}, 3rd edition. Fort Worth: Harcourt Brace College Publishers.



Fodor, Janet \& Ivan Sag. 1982. Referential and quantificational indefinites. \textit{Linguistics and Philosophy} 5: 355–398.



Fodor, Jerry A. 1975. \textit{The Language of Thought}. New York: Crowell.



Fortescue, Michael. 1984. \textit{West Greenlandic}. Croom Helm Descriptive Grammars. London: Croom Helm.



Fortin, Antonio. 2011. The Morphology and Semantics of Expressive Affixes. PhD thesis, University of Oxford.



Franke, Michael. 2007. The pragmatics of biscuit conditionals. In: \textit{Proceedings of the 16th Amsterdam Colloquium}, ed. by Maria Aloni, Paul Dekker, and Floris Roelofsen, pp. 91–96. \url{http://www.sfs.uni-tuebingen.de/~mfranke/Papers/AC07_Paper.pdf} 



Frawley, William. 1992. \textit{Linguistic semantics}. Hillsdale, NJ, Hove \& London: Lawrence Erlbaum.



Frege, Gottlob. 1892. “Über Sinn und Bedeutung.” In \textit{Zeitschrift für Philosophie und Philosophische Kritik}, 25-50. Trans. as ‘On sense and reference’, in P. Geach \& M. Black, eds., \textit{Translations from the philosophical writings of Gottlob Frege}. Oxford: Blackwell, 56–78.



Frege, Gottlob. 1918-19. The thought: A logical inquiry. Translated by Peter Geach in \textit{Mind} 55 (1956): 289-311.



Gamut, L. T. F. 1991a. \textit{Logic, Language and Meaning, Volume I: Introduction to Logic}. University of Chicago Press.



Gamut, L. T. F. 1991b. \textit{Logic, Language and Meaning, Volume II: Intensional Logic and Logical Grammar}. University of Chicago Press.



Garson, James. 2016. Modal Logic. In Edward N. Zalta~(ed.), \textit{The Stanford Encyclopedia of Philosophy} (\citealt{Spring2016} Edition), URL = <https://plato.stanford.edu/archives/spr2016/entries/logic-modal/>.



Gass, Susan \& Joyce Neu (eds.). 2006. \textit{Speech acts across cultures: Challenges to communication in a second language}. Walter de Gruyter.



Gazdar, Gerald. 1979. \textit{Pragmatics: Implicature, presupposition, and logical form}. New York: Academic Press.



Geurts, Bart. 2011. \textit{Quantity implicatures}. Cambridge University Press.



Geurts, Bart and David Beaver. 2011. Presupposition. In E. Zalta, ed., \textit{The Stanford Encyclopedia of Philosophy}. Stanford University. \url{http://plato.stanford.edu/entries/presupposition/} \\
revised version: Beaver, David and Bart Geurts. 2012. Presupposition. In: Klaus von Heusinger, Claudia Maienborn, and Paul Portner (eds.), \textit{Semantics: an international handbook of natural language meaning. Vol. 3}, pp. 2432–2460. Berlin: Mouton de Gruyter.



Giannakidou, Anastasia. 2011. (Non)veridicality and mood choice: subjunctive, polarity, and time. In Musan, Renate and Monika Rathert (eds.), \textit{Tense Across Languages}, 59–90. Berlin, Boston: De Gruyter.



Gillon, \citealt{Brendan1990}. Ambiguity, Generality and Indeterminacy: Tests and Definitions. \textit{Synthese} 85, 391–416.



Givón, Talmy. 1972. Studies in ChiBemba and Bantu grammar. \textit{Studies in African linguistics}, supplement 3:1–247.



Goldberg, Adele E. 2015. Compositionality. In Nick Reimer (ed), \textit{The Routledge Handbook of Semantics}, pp. 419-433.\\
\url{http://semanticsarchive.net/Archive/jcyZDc1Y/Goldberg.Compositionality.RoutledgeHandbook.pdf}



Gomes, Gilberto. 2008. Three types of conditionals and their verb forms in English and Portuguese. \textit{Cognitive Linguistics} 19 (2): 219–40.



Green, Georgia. 1990. The universality of Gricean interpretation. \textit{Berkeley Linguistics Society} \textit{16, Parasession on the Legacy of Grice}, pp. 411–428.



Greenberg, Joseph H. 1963. Some universals of grammar with particular reference to the order of meaningful elements. In Greenberg (ed.), \textit{Universals of language}. Cambridge MA: MIT Press, pp. 73–113. (2\textsuperscript{nd} edition, 1966)



Grice, H. Paul. 1961. The causal theory of perception. \textit{Arisotelian Society Supplement} 35. 121–152. URL: \url{http://www.jstor.org/stable/4106682}



Grice, H. Paul. 1975. Logic and conversation. In Cole, Peter and Jerry L. Morgan (eds.), \textit{Syntax and semantics 3: Speech acts} (41-58). New York: Academic Press.



Grice, H. Paul. 1978. Further notes on logic and conversation. In Cole, Peter and Jerry L. Morgan (eds), \textit{Syntax and Semantics 9: Pragmatics}. New York, Academic Press, 113–127.



Grice, H. Paul. 1981. Presupposition and conversational implicature. In: Peter Cole, ed., \textit{Radical pragmatics}, 183–198. New York: Academic Press.



Guerssel, Mohamed, Kenneth Hale, Mary Laughren, Beth Levin, and Josie White Eagle. 1985. A crosslinguistic study of transitivity alternations. In \textit{Papers from the parasession on causatives and agentivity at the 21st regional meeting}, ed. by William Eilfort, Paul Kroeber, and Karen Peterson, 48–63. Chicago: Chicago Linguistic Society.



Gutierrez-Rexach, Javier. 2013.  Quantification. In Philipp Strazny (ed.), \textit{Encyclopedia of Linguistics}, pp. 885–887. Routledge.



Gutzmann, Daniel. 2015. \textit{Use-conditional meaning: Studies in multidimensional semantics}. Oxford Studies in Semantics and Pragmatics 6. Oxford: Oxford University Press.



Hacquard, Valentine. 2007. Speaker-Oriented vs. Subject-Oriented Modals: A Split in Implicative Behavior. \textit{Proceedings of Sinn und Bedeutung 11}, E. Puig-Waldmüller (ed.), Barcelona: Universitat Pompeu Fabra, pp. 305–319.



Hacquard, Valentine. 2011. Modality. In \textit{Semantics: An International Handbook of Natural Language and Meaning, Vol. 2}, ed. by Claudia Maienborn, Klaus von Heusinger, and Paul Portner, pp. 1484–1515. HSK 33.2. Berlin: Walter de Gruyter..



Haegeman, Liliane. 2010a. The internal syntax of adverbial clauses. \textit{Lingua} 120.628–648.



Hale, Kenneth L., and Samuel J. Keyser. 1987. \textit{A view from the middle}. Lexicon Project Working Papers 10. Cambridge, MA: Center for Cognitive Science, MIT Press.



Halmos, Paul. 1960. \textit{Naive set theory}. Princeton, NJ: D. Van Nostrand Company. Reprinted by Springer-Verlag, New York, 1974.



Harada, S.-I. 1976. Honorifics. In Masayoshi Shibatani (ed.), \textit{Syntax and Semantics 5: Japanese Generative Grammar}, 499-561. New York: Academic Press.



Hardy, Heather K. \& Lynn Gordon. 1980. Types of adverbial and modal constructions in Tolkapaya. \textit{International Journal of American Linguistics} 46.3:183–196.



Hartmann, R. R. K. \& [200E?]Gregory James. 1998. \textit{Dictionary of Lexicography}. London and New York: Routledge.



Haspelmath, M. 1995. ‘The converb as a cross-linguistically valid category.’ in M. Haspelmath and E. König (eds) \textit{Converbs in cross-linguistic perspective: structure and meaning of adverbial verb forms – adverbial participles, gerunds}, pp. 1-55. Berlin: Mouton de Gruyter.



Heim, Irene \& Angelika Kratzer. 1998. \textit{Semantics in Generative Grammar}. Oxford: Basil Blackwell.



Hjelmslev, Louis. 1953[1943]. \textit{Prolegomena to a Theory of Language}. Baltimore: Indiana University Publications in Anthropology and Linguistics (IJAL Memoir, 7). [first published in Danish in 1943; 2nd English edition (slightly rev.): Madison: University of Wisconsin Press, 1961]



Hockett, Charles F. 1958. \textit{A course in modern linguistics}. New York: Macmillan.



Hockett, Charles F. 1960. The origin of speech. \textit{Scientific American} 203: 88–96.



Hodges, Wilfrid. 1997. \textit{A shorter model theory}. Cambridge: Cambridge University Press.



Hodges, Wilfrid. 2013. Model Theory. In Edward N. Zalta~(ed.), \textit{The Stanford Encyclopedia of Philosophy} (\citealt{Fall2013} Edition). URL = <https://plato.stanford.edu/archives/fall2013/entries/model-theory/>.



Horn, Laurence R. 1972. On the semantic properties of logical operators in English. Ph.D. dissertation, University of California, Los Angeles.



Horn, Laurence R. 1985. Metalinguistic negation and pragmatic ambiguity. \textit{Language} 61.121–174.



Horn, Laurence R. 1989. \textit{A Natural History of Negation}. Chicago: University of Chicago Press.



Horn, Laurence R. 1992. The Said and the Unsaid. In Chris Barker \& David Dowty (eds.),\\
\textbf{\textmd{\textit{SALT II: Proceedings of the 2nd Semantics and Linguistic Theory Conference}}}, 163–192\textbf{\textmd{\textit{. Ohio State University Working Papers in Linguistics, Volume 40.}}}  ISSN 2163-5951. Available at: <\url{http://journals.linguisticsociety.org/proceedings/index.php/SALT/article/view/3039/2782}>. doi:\url{http://dx.doi.org/10.3765/salt.v2i0.3039}.



Horn, Laurence R. 1997. Presupposition and implicature. In Lappin, Shalom. (ed.). \textit{The Handbook of Contemporary Semantic Theory}, 209–319. Blackwell Publishing.



Horn, Laurence R. 2004. Implicature. In L. R. Horn \& G. Ward (ed.), \textit{The Handbook of Pragmatics}, pp. 3–28. Oxford: Blackwell Publishing.



Huang, Lillian Meei Jin and Davis, Philip W. 1989. An aspectual system in Mandarin Chinese. \textit{Journal of Chinese Linguistics} 17:128–66.



Rodney Huddleston \& Geoffrey K. Pullum. 2002. \textit{The Cambridge Grammar of the English Language}. Cambridge: Cambridge University Press.



Iatridou, Sabine. 1991. Topics in conditionals. Ph.D. thesis, MIT.



Idris, A.A. 1980. Modality in Malay. \textit{Kansas Working Papers in Linguistics} 5.1: 1-14.\\
\url{http://kuscholarworks.ku.edu/dspace/handle/1808/531} 



Iljic, Robert. 1990. The verbal suffix \textit{–guo} in Mandarin Chinese and the notion of recurrence. \textit{Lingua} 81: 301–326.



Innes, Gordon. 1966. \textit{An introduction to Grebo}. London: School of Oriental and African Studies, University of London.



Izvorski, Roumyana. 1997. The present perfect as an epistemic modal. In Aaron Lawson (ed.), \textit{Proceedings from Semantics and Linguistic Theory VII}, pp. 222–239. Ithaca, NY: Cornell University. \url{http://www-bcf.usc.edu/~pancheva/evidentialperfect.pdf} 



Jackendoff, Ray. 1976. Toward an explanatory semantic representation. \textit{Linguistic Inquiry} 7.1:89–150.



Jackendoff, Ray. 1983. \textit{Semantics and cognition}. Cambridge, MA: MIT Press.



Jesperson, Otto. 1924. \textit{The philosophy of grammar}. London: Allen\& Unwin.



Jespersen, Otto. 1931. \textit{A modern English grammar on historical principles, Part 4/ Syntax, Third Volume: Time and tense}. London: G. Allen \& Unwin; Copenhagen: E. Munksgaard.



Jespersen, Otto. 1933. \textit{Essentials of English grammar}. London: G. Allen \& Unwin; Reprinted by University of Alabama Press, 1964.



Johnston, Michael. 1994. The syntax and semantics of adverbial adjuncts. Ph.D. Dissertation. University of California Santa Cruz.



Kalisz, Roman. 1992. Different cultures, different languages, and different speech acts revisited. \textit{Poznan Studies in Contemporary Linguistics} 27: 107–118.\\
\url{http://wa.amu.edu.pl/psicl/files/27/07Kalisz.pdf} 



Karagjosova, Elena. 2000. A unified approach to the meaning and function of modal particles in dialogue. In Catherine Pilire, editor, Proceedings of the ESSLLI 2000 Student Session, August 6-18, University of Birmingham, UK.



Karttunen, Lauri and Stanley Peters. 1979. Conventional implicature. \textit{Syntax and Semantics, Volume 11: Presupposition}, ed. by Choon-Kyu Oh and David Dinneen, 1–56. New York: Academic Press.



Katz, Jerrold J. 1972. \textit{Semantic Theory}. New York: Harper \& Row.



Katz, Jerrold J. 1978. Effability and Translation. In F. Guenthner and M. Guenthner-Reutter (eds.), \textit{Meaning and Translation}, pp. 191-234. New York: New York University Press.



Katz, Jerrold J. \& Jerry A. \citealt{Fodor1963}. The structure of a semantic theory. \textit{Language} 39.170–210.



Kearns, Kate. 2000. \textit{Semantics}. (Modern Linguistics series.) New York: St. Martin’s Press.



Kearns, Kate. 2011. \textit{Semantics} (second edition). Palgrave Macmillan.



Keenan, Elinor O. 1974. The universality of conversational implicatures. In Ralph W. Fasold and Roger W. Shuy (eds.), \textit{Studies in Linguistic Variation: Semantics, Syntax, Phonology, Pragmatics, Social Situations, Ethnographic Approaches}, 255–268, Georgetown University Press.



Kempson, Ruth. M. 1975. \textit{Presupposition and the delimitation of semantics}. Cambridge University Press.



Kempson, Ruth M. 1977. \textit{Semantic Theory}. (Cambridge Textbooks in Linguistics.) Cambridge: Cambridge University Press.



Kennedy, Christopher. 2011. Ambiguity and Vagueness: An Overview. In Claudia Maienborn, Klaus von Heusinger, \& Paul Portner (eds.), \textit{Semantics: An International Handbook of Natural Language Meaning, Vol. 1}, 507–535. Berlin: Mouton de Gruyter.



Kim, Jong-Bok and Peter Sells. 2007. Korean honorification: a kind of expressive meaning. \textit{Journal of East Asian Linguistics} Volume 16, Issue 4 \citep{December2007}, pp 303-336.  \url{http://web.khu.ac.kr/~jongbok/research/final-papers/kor-hon-kim-sells.pdf}



Kiparsky, Paul. 2002. Event structure and the perfect. In David I. Beaver, Luis D. Casillas Martínez, Brady Z. Clark, and Stefan Kaufmann (eds.), \textit{The Construction of Meaning}. Stanford CA: CSLI Publications.



Kiparsky, Paul and Carol Kiparsky. 1970. Fact. \textit{Progress in linguistics}, ed. by Manfred Bierwisch \& Karl Heidolph, 143–173. The Hague: Mouton. Reprinted in \textit{Semantics: An Interdisciplinary Reader}, ed. by L. Jakobovits and D. Steinberg. Cambridge University Press, 1971.



Kitis, Eliza. 2006. Causality and subjectivity: the causal connectives of Modern Greek. In \textit{Language and memory: Aspects of knowledge representation}, ed. by Hanna Pishwa, pp. 223–267. Berlin \& New York: Mouton de Gruyter.



Klein, Wolfgang. 1992. The present perfect puzzle. \textit{Language} 68: 525–552.



Klein, Wolfgang. 1994. \textit{Time in Language}. London: Routledge.



Klein, Wolfgang. 2009. How time is encoded. In \textit{The Expression of Time}, edited by Wolfgang Klein and Ping Li, 5–38. Berlin: De Gruyter.



Klein, Wolfgang, Ping Li and Henriette Hendriks. 2000. Aspect and assertion in Mandarin Chinese. \textit{Natural Language and Linguistic Theory} 18. 723–70.



Koenig, J.-P. \& L.C. Chief. 2008. Scalarity and state-changes in Mandarin (and other languages. in O. Bonami and P. Cabredo Hofherr, eds., \textit{Empirical Issues in Syntax and Semantics} 7, 241–262. \url{http://www.cssp.cnrs.fr/eiss7/koenig-chief-eiss7.pdf} 



Koenig, Jean-Pierre \& Karin Michelson. 2010. How to quantify over entities in Iroquoian. Paper presented at the Society for the Study of the Indigenous Languages of the Americas, Baltimore, MD.



Koenig, J.-P. \& N. Muansuwan. 2000. How to end without ever finishing: Thai semi-perfectivity. \textit{Journal of Semantics} 17, 147–184.



König, Ekkehard. 1991. The meaning of focus particles: A comparative perspective. London: Routledge.



König, Ekkehard, Detlef Stark, \& Susanne Requardt. 1990. \textit{Adverbien und Partikeln: Ein deutschenglisches Wörterbuch} [Adverbs and Particles: A German–English Dictionary]. Heidelberg: Julius Groos.



Kratzer, Angelika. 1981. The notional category of modality. In: H.-J. Eikmeyer \& H. Rieser (eds.), \textit{Words, worlds, and contexts: New approaches in word semantics}. Berlin: Mouton de Gruyter, 38-74.



Kratzer, Angelika. 1986. Conditionals. \textit{Chicago Linguistics Society}, 22(2): 1–15.



Kratzer, Angelika. 1991. Modality. In \textit{Semantics: An International Handbook of Contemporary Research}, edited by Arnim von Stechow \& Dieter Wunderlich, pp. 639–650. Berlin: Mouton de Gruyter.



Kratzer, Angelika. 1995. Stage-level/individual-level predicates. In \textit{The generic book}, ed. G.N. Carlson and F.J. Pelletier, 125–175. University of Chicago Press.



Kratzer, Angelika. 1999. Beyond ‘Ouch’ and ‘Oops’: How descriptive and expressive meaning interact. Comment on Kaplan’s paper at the Cornell Conference on Context Dependency. \url{http://semanticsarchive.net/Archive/WEwNGUyO/} 



Kroeger, Paul. 2005. \textit{Analyzing grammar: An introduction}.  Cambridge University Press.



Kroeger, Paul. 2010. The grammar of \textit{hitting}, \textit{breaking} and \textit{cutting} in Kimaragang Dusun. \textit{Oceanic Linguistics} 49:2–20.



Kroeger, Paul. 2017. Frustration, culmination, and inertia in Kimaragang grammar. \textit{Glossa: a journal of general linguistics} 2(1): 56. 1–29.



LaCross, Lisa. 2016 ms. Past temporal remoteness in Kimanianga. Paper presented at the Texas Linguistic Society conference, Feb. 2016, Austin TX.



Lakoff, George. 1970. A note on vagueness and ambiguity. \textit{Linguistic Inquiry} 1: 357-359.



Langacker, Ronald W. 2001. The English present tense. \textit{English Language and Linguistics}, 5 , pp 251–272.



Lee, Sooman Noah. 2008. \textit{A grammar of Iranian Azerbaijani}. Seoul: The Altaic Society of Korea.



Leech, G. N. 1971. \textit{Meaning and the English verb}. London: Longman.



Levin, Beth. 1993. \textit{English verb classes and alternations: A preliminary investigation}. Chicago, IL: University of Chicago Press.



Levin, Beth. 2015. Verb classes within and across languages. In A. Malchukov \& B. Comrie (eds.), \textit{Valency classes: A comparative handbook}, 1627–1670. Berlin: De Gruyter.



Levinson, Stephen C. 1983. \textit{Pragmatics}. Cambridge University Press.



Levinson, Stephen C. 1995. “Three levels of meaning.” In: F. R. Palmer (ed.). Grammar and Meaning: Essays in Honour of Sir John Lyons. Cambridge: Cambridge University Press, 90-115.



Levinson, Stephen C. 2000. \textit{Presumptive Meanings: The Theory of Generalized Conversational Implicature}. Cambridge, MA: MIT Press.



Levinson, Stephen. C., \& Annamalai, E. 1992. Why presuppositions aren’t conventional. \textit{Language and text: Studies in honour of Ashok R. Kelkar}, ed. by R. N. Srivastava, 227–242. Dehli: Kalinga Publications.



Lewis, C.I. 1918. \textit{Survey of symbolic logic}. University of California Press.



Lewis, David. 1973a. \textit{Counterfactuals}. Oxford: Blackwell.



Lewis, David. 1973b. Causation. \textit{Journal of Philosophy} 70: 556–567.



Lewis, David. 1975. Adverbs of quantification. In Edward Keenan (ed.), \textit{Formal semantics of natural language}, 3–15. Cambridge University Press.



Lewis, David. 2000. Causation as influence. \textit{Journal of Philosophy} 97: 182–97.



Li, Charles and Sandra Thompson. 1981. \textit{Mandarin Chinese: a functional reference grammar}. Berkeley, CA: University of California Press.



Lin, Jo-Wang. 2007. Predicate restriction, discontinuity property and the meaning of the perfective marker \textit{guo} in Mandarin Chinese. \textit{Journal of East Asian Linguistics} 16: 237–257.



Lovestrand, Joseph. 2012. \textit{The Linguistic Structure of Baraïn (Chadic)}. MA thesis, Graduate Institute of Applied Linguistics.\\
\url{http://www.gial.edu/images/theses/Lovestrand_Joseph-thesis.pdf} 



Lyons, John. 1977. \textit{Semantics (vol. 1-2)}. Cambridge University Press.



Lyons, John. 1995. \textit{Linguistic semantics: An introduction}. Cambridge University Press.



Ma, Jing Sheng. 1977. Some Aspects of the Teaching of \textit{-guo} and -\textit{le}. \textit{Journal of the Chinese Language Teachers Association} 12(1), 14–26.



Marques, Rui. 2004. On the system of mood in European and Brazilian Portuguese. \textit{Journal of Portuguese Linguistics} 3.1, 89-109.



Martin, John N. 1987. \textit{Elements of formal semantics: An introduction to logic for students of language}. Orlando: Academic Press.



Martin, Samuel E. 1992. \textit{A Reference Grammar of Korean: A Complete Guide to the Grammar and History of the Korean Language}. Rutland, Vermont, \& Tokyo: Charles E. Tuttle Company.



Matsumoto, Yo. 1995. The conversational condition on Horn scales. \textit{Linguistics and Philosophy} 18:21–60.



Matthewson, Lisa. 2006. Presupposition and cross-linguistic variation. In \textit{Proceedings of the 26th Meeting of the North-Eastern Linguistic Society}, pp. 63–76.



Matthewson, Lisa. 2010. Cross-linguistic variation in modality systems: the role of mood. \textit{Semantics and Pragmatics} 3:1-74.



Matthewson, Lisa. 2016. Modality. In Maria Aloni and Paul Dekker (eds.), \textit{Cambridge Handbook of Formal Semantics}, 525–559. Cambridge: Cambridge University Press.



Matthewson, Lisa, Henry Davis, and Hotze Rullmann. 2007. Evidentials as epistemic modals: Evidence from St’át’imcets. \textit{Linguistic Variation Yearbook} 7: 201-254.



McCawley, James D. 1968. Concerning the base component of a transformational grammar. \textit{Foundations of Language} 4. 243–269.



McCawley, James D. 1971. Tense and time reference in English. In Langendoen and Fillmore (eds.), \textit{Studies in linguistic semantics}, 97–113. New York: Holt Rinehart. 



McCawley, James D. 1981a. \textit{Everything that linguists have always wanted to know about logic but were ashamed to ask}. Chicago: University of Chicago Press.



McCawley, James D. 1981b. Notes on the English perfect. \textit{Australian Journal of Linguistics} 1.81–90.



McCloskey, James. 2001. The morphosyntax of WH-extraction in Irish. \textit{Journal of Linguistics} 37:67–100.



McCoard, Robert W. 1978. \textit{The English perfect: Tense choice and pragmatic inferences}. Amsterdam: North-Holland Publishing Company.



Meibauer, Jörg. 2005. Lying and falsely implicating. \textit{Journal of Pragmatics} 37:1373–1399.



Mellinkoff, Ruth. 1970. \textit{The Horned Moses in Medieval Art and Thought} (California Studies in the History of Art, 14). University of California Press.



Michaelis, Laura A. 1994. The ambiguity of the English present perfect. \textit{Journal of Linguistics} 30.111-157.



Michaelis, Laura A. 1998. \textit{Aspectual Grammar and Past-Time Reference}. London: Routledge.



Michaelis, Laura A. 2006. Time and tense. In B. Aarts and A. MacMahon, (eds.), \textit{The Handbook of English Linguistics}, 220–234. Oxford: Blackwell.



Milsark, Gary L. 1977. Toward an explanation of certain peculiarities of the existential construction in English. \textit{Linguistic Analysis} 3.1-30.



Moens, Marc \& Mark Steedman. 1988. Temporal ontology and temporal reference. \textit{Computational Linguistics} 14(2), 15–28.



Morton, Adam. 2004. Indicative versus subjunctive future conditionals. \textit{Analysis} 64(4):289–293.



Morzycki, Marcin. 2013 ms. \textit{Modification}. Unpublished ms, Michigan State University. In preparation for the Cambridge University Press series \textit{Key Topics in Semantics and Pragmatics}.



Murray, Sarah E. 2010. Evidentiality and the structure of speech acts. Ph.D. dissertation, Rutgers, The State University of New Jersey.\\
\url{http://conf.ling.cornell.edu/sem/Murray_Thesis-Rutgers-2010.pdf} 



Nida, Eugene. 1951. A system for the description of semantic elements. \textit{Word} 7: 1-14.



Nunberg, Geoffrey. 1979. The non-uniqueness of semantic solutions: Polysemy. \textit{Linguistics and Philosophy} 3.2, 143-184



Nunberg, Geoffrey. 1995. Transfers of meaning. \textit{Journal of Semantics} 12.2:109–32.  \url{http://people.ischool.berkeley.edu/~nunberg/JOS.pdf}



Nunberg, Geoffrey and Annie Zaenen. 1992. Systematic polysemy in lexicology and lexicography. In H. Tommola, K. Varantola, T. Salmi-Tolonen and J. Schopp (eds.), \textit{Proceedings of Euralex II}, 387-398. Tampere: University of Tampere.\\
\url{http://www.euralex.org/elx_proceedings/Euralex1992_2/011_Geoffrey Nunberg & Annie Zaenen -Systematic polysemy in lexicology and lexicography.pdf} 



Nurse, Derek. 2008. \textit{Tense and Aspect in Bantu}. New York: Oxford University Press.



Olshtain E. \& A.D. Cohen. 1989. Speech act behaviour across languages. In Hans W. Dechert \& Manfred Raupach (eds.), \textit{Transfer in language production}, 53–68. Norwood, N.J.: Ablex.



Pagin, Peter \& Dag Westerståhl. 2010. Compositionality II: Arguments and problems. \textit{Philosophy Compass} 5.3: 265–282.



Pak, Miok. 2008. Types of clauses and sentence end particles in Korean. \textit{Korean Linguistics} 14:113-155.



Pak, Miok, Paul Portner and Raffaella Zanuttini. 2013. Politeness, formality and main clause phenomena. Handout for talk at LSA annual meeting, Jan. 2013.



Palmer, F.R. 1986. \textit{Mood and modality}. Cambridge University Press.



Pan, H. \& P. Lee. 2004. The role of pragmatics in interpreting the Chinese perfective markers \textit{-guo} and \textit{-le}. \textit{Journal of Pragmatics} 36: 441–466.



Papafragou, Anna. 2006. Epistemic modality and truth conditions. \textit{Lingua} 116: 1688–1702.\\
\url{http://papafragou.psych.udel.edu/papers/Lingua-epmodality.pdf} 



Partee, Barbara Hall. 1973. Some structural analogies between tenses and pronouns in English. \textit{The} \textit{Journal of Philosophy} 70: 601–609.



Partee, Barbara Hall. 1995. “Lexical Semantics and Compositionality.” In \textit{Invitation to Cognitive Science, 2nd edition}. Daniel Osherson, general editor; in Part I: \textit{Language}, Lila Gleitman and Mark Liberman, eds. MIT Press, Cambridge MA, pp. 311-360.\\
prepublication version is available online at: \url{http://semanticsarchive.net/Archive/jhjMGYwM/BHP95Lexical SemanticsAndCompositionality.pdf})



Partee, Barbara Hall. 2007. Compositionality and coercion in semantics: The dynamics of adjective meaning. In Bouma, Gerlof et al. (eds). \textit{Cognitive Foundations of Interpretation}. Amsterdam: Royal Netherlands Academy of Arts and Sciences, pp. 145–161.



Partee, Barbara Hall. 2008. Negation, Intensionality, and Aspect: Interaction with NP Semantics. In: Rothstein, Susan D. (ed.), \textit{Theoretical and Crosslinguistic Approaches to the Semantics of Aspect}, pp. 291-320. Amsterdam \& Philadelphia: John Benjamins.



Pelletier, Francis. 2001. Did Frege believe Frege’s principle. \textit{Journal of Logic, Language and Information} 10:87–114.



Peters, Stanley \& Dag Westerståhl. 2006. \textit{Quantifiers in language and logic}. Oxford: Clarendon Press.



Pinker, Steven. 1994. \textit{The language instinct}. New York: W. Morrow and Co.



Pit, Mirna. 2003. \textit{How to express yourself with a causal connective: Subjectivity and causal connectives in Dutch, German and French}. Amsterdam: Rodopi.



Podlesskaya, Vera. 2001. “76. Conditional constructions.” In Haspelmath, Martin; König, Ekkehard; Oesterreicher, Wulf; \& Raible, Wolfgang (eds.), \textit{Language Typology and Language Universals}, vol. 2, pp. 998–1010. Walter de Gruyter.



Portner, Paul. 2003. The temporal semantics and modal pragmatics of the perfect. \textit{Linguistics and Philosophy} 26: 459–510.



Portner, Paul. 2011. Perfect and progressive. In \textit{Semantics: An international handbook of natural language meaning}, eds. Claudia Maienborn, Klaus von Heusinger, and Paul Portner, 1217–1261. Berlin: Mouton de Gruyter.



Posner, Roland. 1980. Semantics and pragmatics of sentence connectives in natural language. In: Searle, John R., Ferenc Kiefer \& Manfred Bierwisch (eds.), \textit{Speech act theory and pragmatics}, 168–203. Dordrecht: Reidel.



Potts, Christopher. 2005. \textit{The logic of conventional implicatures}. Oxford Studies in Theoretical Linguistics. Oxford: Oxford University Press.



Potts, Christopher. 2007a. Into the conventional-implicature dimension. \textit{Philosophy Compass} 4(2):665–679.\\
\url{http://www.stanford.edu/~cgpotts/papers/potts-conventional-implicature-compass.pdf}



Potts, Christopher. 2007b. Conventional implicatures, a distinguished class of meanings. In Gillian Ramchand \& Charles Reiss (eds.), \textit{The Oxford handbook of linguistic interfaces}. \textit{Studies in Theoretical Linguistics}, 475–501. Oxford: Oxford University Press.\\
http://web.stanford.edu/{\textasciitilde}cgpotts/papers/potts-interfaces.pdf



Potts, Christopher. 2007c. The expressive dimension. \textit{Theoretical Linguistics} 33(2):165–197.



Potts, Christopher. 2012. The pragmatics of conventional implicature and expressive content. In Claudia Maienborn, Klaus von Heusinger \& Paul Portner (eds.), \textit{Semantics: An international handbook of natural language meaning, vol. 3}, 2516–2536. Berlin: Mouton de Gruyter.



Potts, Christopher. 2015. Presupposition and implicature. In Shalom Lappin and Chris Fox (eds.), \textit{Wiley-Blackwell Handbook of Contemporary Semantics, second edition}, 168-202. Oxford: Wiley-Blackwell.\\
\url{http://web.stanford.edu/~cgpotts/manuscripts/potts-blackwellsemantics.pdf} 



Prince, Ellen F. 1982. Grice and universality: A reappraisal. URL ftp://babel.ling.upenn.edu/papers/faculty/ellen\_prince/grice.ps, ms, University of Pennsylvania.



Prior, A. N. 1957. \textit{Time and modality}. Oxford University Press.



Prior, A. N. 1967. \textit{Past, present and future}. Oxford University Press.



Pulte, William. 1985. The experienced and non-experienced past in Cherokee. \textit{International Journal of American Linguistics} 4:543-44.



Pustejovsky, \citealt{James1995}. \textit{The Generative Lexicon}. Cambridge, MA: The MIT Press.



Quine, W.V. 1956. Quantifiers and propositional attitudes. \textit{Journal of Philosophy} 53:177-187.



Quine, Willard van Orman. 1960. \textit{Word and Object}. Cambridge, MA: MIT Press. **check**



Rappaport Hovav, Malka and Beth Levin. 1998. Building verb meanings. In \textit{The projection of arguments: lexical and compositional factors}, ed. by Miriam Butt and Wilhelm Geuder, 97–134. Stanford, CA: CSLI Publications.



Ravin, Yael \& Claudia Leacock. 2000. Polysemy: An overview. In Ravin \& Leacock (eds.), \textit{Polysemy: Theoretical and computational approaches}, pp 91–110. Oxford: Oxford University Press.



Reichenbach, Hans. 1947. \textit{Elements of Symbolic Logic}. London: Macmillan.



Recanati, François. 2004. Pragmatics and semantics. In L. R. Horn and G. Ward (eds.), \textit{The Handbook of Pragmatics}, 442–462. Oxford: Blackwell.



Ritz, Marie-Eve. 2012. Perfect tense and aspect. In Binnick, Robert I. (ed.), \textit{The Oxford Handbook of Tense and Aspect}, 881–907. Oxford University Press.



Rundell, Michael. 2006. More than one way to skin a cat: why full-sentence definitions have not been universally adopted. In Corino E., Marello C., Onesti C. (eds.), \textit{Proceedings of 12th EURALEX International Congress}. Alessandria: Edizioni Dell’Orso.



Russell, Bertrand. 1905. On denoting. \textit{Mind} 14: 479–493.



Sadock, Jerry, 1978. On testing for conversational implicature. In: Cole, Peter (ed.), \textit{Syntax and Semantics Volume 9: Pragmatics}. Academic Press, New York, pp. 281–297.



Sæbø, Kjell Johan. 1991. Causal and purposive clauses. In Arnim von Stechow \& Dieter Wunderlich (eds.), \textit{Semantics: an International Handbook of Contemporary Research}, 623–631. Berlin: Walter de Gruyter.



Sæbø, Kjell Johan. 2011. Adverbial clauses. In Maienborn, von Heusinger and Portner (eds.), \textit{Semantics: An international handbook of natural language meaning} (Volume 2), 1420–1441. Berlin and New York: De Gruyter Mouton.\\
\url{http://folk.uio.no/kjelljs/060AdverbialClauses.pdf} 



Saeed, John. 2009. \textit{Semantics (3\textsuperscript{rd}} \textit{edition)}. Chichester, UK: Wiley-Blackwell.



Schachter, Paul and Fe T. Otanes. 1972. \textit{Tagalog Reference Grammar}. Berkeley CA: University of California Press.



Scheffler, Tatjana. 2005. Syntax and semantics of causal \textit{denn} in German. In \textit{Proceedings of the 15th Amsterdam Colloquium}, Amsterdam, Netherlands.



Scheffler, Tatjana. 2008. Semantic operators in different dimensions. PhD Dissertation, University of Pennsylvania.



Scheffler, Tatjana. 2013. \textit{Two-dimensional semantics: Clausal adjuncts and complements}. Berlin, Boston: De Gruyter Mouton.



Searle, John. 1969. \textit{Speech Acts: An Essay in the Philosophy of Language}. Cambridge: Cambridge University Press.



Searle, John. 1975. Indirect speech acts. In \textit{Syntax and Semantics, 3: Speech Acts}, ed. P. Cole \& J. L. Morgan, pp. 59–82. New York: Academic Press.



Siegel, Muffy. 1976. Capturing the Adjective. Ph.D. dissertation, University of Massachusetts.



Singh, Mona. 1991. The perfective paradox or how to eat your cake and have it too. \textit{Proceedings of the Seventeenth Annual Meeting of the Berkeley Linguistics Society: General Session and Parasession on The Grammar of Event Structure}, 469–479.\\
\url{http://elanguage.net/journals/bls/article/viewFile/2738/2719} 



Singh, Mona. 1998. On the semantics of the perfective aspect. \textit{Natural Language Semantics} 6:171-199.



Smith, Carlota. 1997. \textit{The Parameter of Aspect} (2\textsuperscript{nd} edition). Dordrecht: Kluwer Academic. (1\textsuperscript{st} edition 1991)



Soh, Hooi Ling and Jenny Yi-Chun Kuo. 2005. Perfective aspect and accomplishment situations in Mandarin Chinese. In Angeliek van Hout, Henriette de Swart and Henk Verkuyl (eds.), \textit{Perspectives on Aspect}, 199–216. Dordrecht: Springer.



Sohn, Ho-Min. 1999. \textit{The Korean Language}. New York, Cambridge University Press.



Stalnaker, Robert. 1968. A theory of conditionals. In Nicholas Rescher (ed.), \textit{Studies in Logical Theory}, 98–112. American Philosophical Quarterly Monograph Series \#2. Oxford: Basil Blackwell.



Stalnaker, Robert. 1973. Presuppositions. \textit{Journal of Philosophical Logic} 2: 447-457.



Stalnaker, Robert. 1974. Pragmatic Presuppositions. In Milton K. Munitz and Peter K. Unger (eds.), \textit{Semantics and Philosophy}, pp. 197–213. New York: New York University Press.



Strawson, Peter. 1950. On referring. \textit{Mind} 59, 320-344. Reprinted in A.P. \citealt{Martinich1990}, \textit{The Philosophy of Language}. New York: Oxford University Press, pp. 219–234.



Strawson, Peter. 1952: \textit{Introduction to Logical Theory}. London: Methuen.



Sun, Chen-Chen. 2011. \textit{Variations in the ba} \textit{construction and its relevance to DP: A Minimalist perspective}. ProQuest, UMI Dissertation Publishing.



Svensén, Bo. 2009. \textit{A Handbook of Lexicography: The theory and practice of dictionary-making}. Cambridge: Cambridge University Press.



Sweetser, Eve. 1990. \textit{From etymology to pragmatics: Metaphorical and cultural aspects of semantic structure}. Cambridge: Cambridge University Press.



Szabolcsi, Anna. 2015. Varieties of quantification. In Nick Riemer, ed., \textit{The Routledge Handbook of Semantics}, pp. 320–337. Routledge.



Tannen, Deborah. 1975. Communication Mix and Mixup, or How Linguistics Can Ruin a Marriage. \textit{San Jose State Occasional Papers in Linguistics}, pp. 205-211. San Jose State University.



Tannen, Deborah. 1981. Indirectness in discourse: Ethnicity as conversational style. \textit{Discourse Processes} 4:3.221–238. \url{http://faculty.georgetown.edu/tannend/pdfs/Indirectness_in_discourse.pdf} 



Tannen, Deborah. 1986. \textit{That’s not what I meant!: How conversational style makes or breaks relationships}. Ballantine.



Tenny, Carol. 1987. Grammaticalizing aspect and affectedness. Ph.D. dissertation, Department of Linguistics and Philosophy, MIT, Cambridge, Massachusetts.



Thompson, Sandra A., Robert E. Longacre, \& Shin Ja J. Hwang. 2007. Adverbial clauses. In \textit{Language typology and syntactic description, vol. 2}, second edition, ed. by Timothy Shopen, 237-300. Cambridge: Cambridge University Press.



Tonhauser, Judith, David Beaver, Craige Roberts, and Mandy Simons. 2013. Towards a taxonomy of projective content. \textit{Language} 89.1:66-109.



Tuggy, David. 1993. “Ambiguity, polysemy, and vagueness.” \textit{Cognitive Linguistics} 4:3.273–290.



van Benthem, Johan. 1988. \textit{A Manual of Intensional Logic}, 2nd edition, Stanford, CA: CSLI.



van Benthem, Johan. 2010. \textit{Modal Logic for open minds}. Stanford, CA: CSLI.



van der Auwera, Johan. 1986. Conditionals and speech-acts. In Traugott, Elizatbeth C., Alice ter Meulen, Judy Snitzer Reilly, \& Charles Ferguson (eds.), \textit{On Conditionals}, 197–214. Cambridge: Cambridge University Press



van der Auwera, Johan and Andreas Ammann. 2013. Overlap between situational and epistemic modal marking. In: Dryer, Matthew S. \& Haspelmath, Martin (eds.), The \textit{World Atlas of Language Structures Online}. Leipzig: Max Planck Institute for Evolutionary Anthropology. (Available online at \url{http://wals.info/chapter/76} )



Van Valin, Robert D. and Randy J. LaPolla. 1997. \textit{Syntax: Structure, meaning, and function}. Cambridge: Cambridge University Press.



Vendler, Zeno. 1957. Verbs and Times. \textit{Philosophical Review} 66:143–160. [Reprinted in Z. \citet{Vendler1967}, \textit{Linguistics in Philosophy}, Ithaca: Cornell University Press, 97–121.].



Vogel, A.R. 2005. Jarawara Verb Classes. Doctoral dissertation, University of Pittsburgh, Pittsburgh, PA.



von Fintel, Kai. 2004. Would you believe it? The king of France is back! Presuppositions and truth-value intuitions. In Anne Bezuidenhout \& Marga Reimer (eds.), \textit{Descriptions and beyond: An interdisciplinary collection of essays on definite and indefinite descriptions and other related phenomena}, 315–341. Oxford University Press.



von Fintel, Kai. 2006. Modality and Language. In Donald M. Borchert (ed.), \textit{Encyclopedia of Philosophy} – Second Edition, vol. 10, pp. 19–26. Detroit: MacMillan Reference USA. Online at \url{http://mit.edu/fintel/www/modality.pdf}



von Fintel, Kai. 2011. Conditionals. In \textit{Semantics: An international handbook of meaning}, edited by Klaus von Heusinger, Claudia Maienborn, and Paul Portner.\\
Preprint: \url{http://mit.edu/fintel/fintel-2009-hsk-conditionals.pdf} 



von Fintel, Kai. 2012. Subjunctive conditionals. In Gillian Russell \& Delia Graff Fara (eds.), \textit{The Routledge companion to philosophy of language}, 466–477. New York: Routledge.



von Fintel, Kai and Lisa Matthewson. 2008. Universals in Semantics. \textit{The Linguistic Review} 25.139–201.



von Heusinger, \citealt{Klaus2011}. Specificity. In: K. von Heusinger \& C. Maienborng \& P. Portner (eds.). \textit{Semantics: An International Handbook of Natural Language Meaning}. Vol 2. Berlin: de Gruyter, 1024-1057.



Waltereit, Richard, 2001. “Modal particles and their functional equivalents: a speech-act-theoretic approach.” \textit{Journal of Pragmatics} 33, 9, 1391-1417.



Weber, David J. 1989. \textit{A grammar of Huallaga (Huánuco) Quechua}. University of California Publications in Linguistics, 112. Berkeley: University of California Press. xxv, 490 p.



Wierzbicka, Anna. 1985. Different languages, different cultures, different speech acts: English vs. Polish. \textit{Journal of Pragmatics} 9: 145–178.



Wu, Jiun-Shiung. 2008. Terminability, wholeness and the semantics of the experiential \textit{guo}. \textit{Journal of East Asian Linguistics} 17.1:1–32.



Wu, Jiun-Shiung. 2009. Aspectual influence on temporal relations: a case study of the experiential \textit{guo} in Mandarin. \textit{Taiwan Journal of Linguistics} 7.2 1–24.\\
\url{http://tjl.nccu.edu.tw/volume7-2/7.2-1Wu.pdf} 



Xiao, Richard \& Tony McEnery. 2004. \textit{Aspect in Mandarin Chinese: a corpus-based study}. Amsterdam/Philadelphia: John Benjamins Publishing Company.



Yeh, Meng. 1993. Stative Situations and the imperfective \textit{-zhe} in Mandarin. \textit{Journal of the Chinese Language Teachers Association} 28:69–98.



Yeh, Meng. 1996. An analysis of the experiential \textit{guo} in Mandarin: A temporal quantifier. \textit{Journal of East Asian Linguistics} 5:151–182.



Zalta, Edward N. 2017. Gottlob Frege. In Edward N. Zalta~(ed.), \textit{The Stanford Encyclopedia of Philosophy} (\citealt{Spring2017} Edition), URL = <https://plato.stanford.edu/archives/spr2017/entries/frege/>.



Zimmermann, Malte. 2011. Discourse Particles. In P. Portner, C. Maienborn und K. von Heusinger (eds.), \textit{Semantics}. (= \textit{Handbücher zur Sprach- und Kommunikationswissenschaft HSK 33.2}). Berlin, Mouton de Gruyter. pp. 2011-2038.\\
\url{http://www.ling.uni-potsdam.de/~mzimmermann/papers/MZ2011-Particles-HSK.pdf} 



Zimmermann, Thomas \& Wolfgang Sternefeld. 2013. \textit{Introduction to semantics: An essential guide to the composition of meaning}. Berlin: DeGryter Mouton.



Zwicky, Arnold, and Jerry Sadock. 1975. Ambiguity tests and how to fail them. In J. P. Kimball (ed.), \textit{Syntax and Semantics, Volume 4}, pp. 1–36. New York: Academic Press.


\end{verbatim}  %add a percentage sign in front of the line to exclude this chapter from book
\chapter{Referring, denoting, and expressing}\label{sec:2}

\section{Talking about the world}\label{sec:2.1}
\largerpage[2]
In this chapter and the next we will think about how speakers use language to talk about the world. Referring to a particular individual, e.g. by using expressions such as \textit{Abraham Lincoln} or \textit{my father}, is one important way in which we talk about the world. Another important way is to describe situations in the world, i.e., to claim that a certain state of affairs exists. These claims are judged to be true if our description matches the actual state of the world, and false otherwise. For example, if I were to say \textit{It is raining} at a time and place where no rain is falling, I would be making a false statement.

We will focus on truth in the next chapter, but in this chapter our primary focus is on issues relating to reference. We begin in \sectref{sec:2.2} with a very brief description of two ways of studying linguistic meaning. One of these looks primarily at how a speaker’s words are related to the thoughts or concepts he is trying to express. The other approach looks primarily at how a speaker’s words are related to the situation in the world that he is trying to describe. This second approach will be assumed in most of this book.

In \sectref{sec:2.3} we will think about what it means to “refer” to things in the world, and discuss various kinds of expressions that speakers can use to refer to things. In \sectref{sec:2.4} we will see that we cannot account for meaning, or even reference, by looking only at reference. To preview that discussion, we might begin with the observation that people talk about the “meaning” of words in two different ways, as illustrated in \REF{ex:2.1}. In (\ref{ex:2.1}a), the word \textit{meant} is used to specify the reference of a phrase when it was used on a particular occasion, whereas in (\ref{ex:2.1}b-c), the word \textit{means} is used to specify the kind of meaning that we might look up in a dictionary.


\ea \label{ex:2.1}
\ea When Jones said that he was meeting “a close friend” for dinner, he meant his lawyer.\\
\ex \textit{Salamat} means ‘thank you’ in \ili{Tagalog}.\\
\ex \textit{Usufruct} means ‘the right of one individual to use and enjoy the property of another.’\footnote{\url{http://legal-dictionary.thefreedictionary.com/usufruct}}
\z
\z


We will introduce the term \textsc{sense} for the kind of meaning illustrated in (\ref{ex:2.1}b-c), the kind of meaning that we might look up in a dictionary. One crucial difference between sense and reference is that reference depends on the specific context in which a word or phrase is used, whereas sense does not depend on context in this way.



In \sectref{sec:2.5} we discuss various types of \textsc{ambiguity}, that is, ways in which a word, phrase or sentence can have more than one sense. The existence of ambiguity is an important fact about all human languages, and accounting for ambiguity is an important goal in semantic analysis.



In \sectref{sec:2.6} we discuss a kind of meaning that does not seem to involve either reference to the world, or objective claims about the world. \textsc{Expressive} meaning (e.g. the meanings of words like \textit{ouch} and \textit{oops}) reflects the speaker’s feelings or attitudes at the time of speaking. We will list a number of ways in which expressive meaning is different from “normal” \textsc{descriptive} meaning.


\section{Denotational semantics vs. cognitive semantics}\label{sec:2.2}

Let us begin by discussing the relationships between a speaker’s words, the situation in the world, and the thoughts or concepts associated with those words. These relationships are indicated in the figure in \REF{ex:2.2}, which is a version of a diagram that is sometimes referred to as the Semiotic Triangle.



 

\eabox{ \label{ex:2.2}
(one version of) the Semiotic Triangle\\
 \begin{tikzpicture}
  \node[regular polygon, regular polygon sides=3, minimum size=3cm,draw] (polygon3) {};
  \node[shift=(polygon3.corner 1),above] {\sffamily Mind}; 
  \node[shift=(polygon3.corner 3),below right] {\sffamily World}; 
  \node[shift=(polygon3.corner 2),below left] {\sffamily Language}; 
 \end{tikzpicture}
% %  \caption{
% }
}


Semiotics is the study of the relationship between signs and their meanings. In this book we are interested in the relationship between forms and meanings in certain kinds of symbolic systems, namely human languages. The diagram is a way of illustrating how speakers use language to describe things, events, and situations in the world. As we will see when we begin to look at word meanings, what speakers actually describe is a particular \textsc{construal} of, or way of thinking about, the situation. Now the speaker’s linguistic description rarely if ever includes everything that the speaker knows or believes about the situation, and what the speaker believes about the situation may not match the actual state of the world. Thus there is no one-to-one correspondence between the speaker’s mental representation and either the actual situation in the world or the linguistic expressions used to describe that situation. However, there are strong links or associations connecting each of these domains with the others.



The basic approach we adopt in this book focuses on the link between linguistic expressions and the world. This approach is often referred to as \textsc{denotational} semantics. (We will discuss what \textsc{denotation} means in \sectref{sec:2.4} below.) An important alternative approach, \textsc{cognitive semantics}, focuses on the link between linguistic expressions and mental representations. Of course, both approaches recognize that all three corners of the Semiotic Triangle are involved in any act of linguistic communication. One motivation for adopting a denotational approach comes from the fact that it is very hard to find direct evidence about what is really going on in a speaker’s mind. A second motivation is the fact that this approach has proven to be quite successful at accounting for compositionality (how meanings of complex expressions, e.g. sentences, are related to the meanings of their parts).



The two foundational concepts for denotational semantics, i.e. for talking about how linguistic expressions are related to the world, are \textsc{truth} and \textsc{reference}. As we mentioned in \chapref{sec:1}, we will say that a sentence is true if it corresponds to the actual situation in the world which it is intended to describe. It turns out that native speakers are fairly good at judging whether a given sentence would be true in a particular situation; such judgments provide an important source of evidence for all semantic analysis. Truth will be the focus of attention in \chapref{sec:3}. In the next several sections of this chapter we focus on issues relating to reference.


\section{Types of referring expressions}\label{sec:2.3}

Philosophers have found it hard to agree on a precise definition for \textit{reference}, but intuitively we are talking about the speaker’s use of words to “point to” something in the world; that is, to direct the hearer’s attention to something, or to enable the hearer to identify something. Suppose we are told that Brazilians used to “refer to” Pelé as \textit{o rei} ‘the king’.\footnote{Of course, Pelé rose to fame long after Brazil became a republic, so there was no king ruling the country at that time.} This means that speakers used the phrase \textit{o rei} to direct their hearers’ attention to a particular individual, namely the most famous soccer player of the 20\textsuperscript{th} century. Similarly, we might read that amyotrophic lateral sclerosis (ALS) is often “referred to” as Lou Gehrig’s Disease, in honor of the famous American baseball player who died of this disease. This means that people use the phrase \textit{Lou Gehrig’s Disease} to direct their hearers’ attention to that particular disease.



A \textsc{referring expression} is an expression (normally some kind of noun phrase) which a speaker uses to refer to something. The identity of the referent is determined in different ways for different kinds of referring expressions. A proper name like \textit{King Henry VIII}, \textit{Abraham Lincoln}, or \textit{Mao Zedong}, always refers to the same individual. (In saying this, of course, we are ignoring various complicating factors, such as the fact that two people may have the same name. We will focus for the moment on the most common or basic way of using proper names, namely in contexts where they have a single unambiguous referent.) For this reason, they are sometimes referred to as \textsc{rigid designators}. “Natural kind” terms, e.g. names of species (\textit{camel, octopus, durian}) or substances (\textit{gold, salt, methane}), are similar. When they are used to refer to the species as a whole, or the substance in general, rather than any specific instance, these terms are also rigid designators: their referent does not depend on the context in which they are used. Some examples of this usage are presented in \REF{ex:2.3}.


\ea \label{ex:2.3}
\ea \textit{The octopus} has eight tentacles and is quite intelligent.\\
\ex \textit{Camels} can travel long distances without drinking.\\
\ex \textit{Methane} is lighter than air and highly flammable.
\z
\z


For most other referring expressions, reference does depend on the context of use. \textsc{deictic} elements (sometimes called \textsc{indexicals}) are words which refer to something in the speech situation itself. For example, the pronoun \textit{I} refers to the current speaker, while \textit{you} refers to the current addressee. \textit{Here} typically refers to the place of the speech event, while \textit{now} typically refers to the time of the speech event.



Third person pronouns can be used with deictic reference, e.g. “Who is \textit{he}?” (while pointing); but more often are used anaphorically. An \textsc{anaphoric} element is one whose reference depends on the reference of another NP within the same discourse. (This other NP is called the \textsc{antecedent}.) The pronoun \textit{he} in sentence \REF{ex:2.4} is used anaphorically, taking \textit{George} as its antecedent.


\ea \label{ex:2.4}
Susan refuses to marry George\textsubscript{i} because he\textsubscript{i} smokes.
\z


Pronouns can be used with quantifier phrases, like the pronoun \textit{his} in sentence (\ref{ex:2.5}a); but in this context, the pronoun does not actually refer to any specific individual. So in this context, the pronoun is not a referring expression.\footnote{Pronouns used in this way are functioning as “bound variables”, as described in \chapref{sec:4}.} For the same reason, quantifier phrases are not referring expressions, as illustrated in (\ref{ex:2.5}b). (The symbol “\#” in (\ref{ex:2.5}b) indicates that the sentence is grammatical but unacceptable on semantic or pragmatic grounds.)


\ea \label{ex:2.5}
\ea{} [Every boy]\textsubscript{i} should respect his\textsubscript{i} mother.\\        
\ex{} [Every American male]\textsubscript{i} loves football; \#he\textsubscript{i} watched three games last weekend.
\z
\z

Some additional examples that illustrate why quantified noun phrases cannot be treated as referring expressions are presented in (\ref{ex:2.6}--\ref{ex:2.8}). As example (\ref{ex:2.6}a) illustrates, reflexive pronouns are normally interpreted as having the same reference as their antecedent; but this principle does not hold when the antecedent is a quantified noun phrase (\ref{ex:2.6}b).


\ea \label{ex:2.6}
\ea \textit{John trusts himself}  is equivalent to:  \textit{John trusts John}.\\
\ex \textit{Everyone trusts himself}  is not equivalent to:  \textit{Everyone trusts everyone}.
\z
\z


As we discuss in \chapref{sec:3}, a sentence of the form \textit{X is Estonian and X is not Estonian} is a contradiction; it can never be true, whether X refers to an individual as in (\ref{ex:2.7}b) or a group of individuals as in (\ref{ex:2.7}c). However, when X is replaced by certain quantified noun phrases, e.g. those beginning with \textit{some} or \textit{many}, the sentence could be true. This shows that these quantified noun phrases cannot be interpreted as referring to either individuals or groups of individuals.\footnote{\citet[49–52]{PetersWesterståhl2006} present a mathematical proof showing that quantified noun phrases cannot be interpreted as referring to sets of individuals.}


\ea  \label{ex:2.7}
\ea \#X is Estonian and X is not Estonian.\\
\ex \#John is Estonian and John is not Estonian.\\
\ex \#My parents are Estonian and my parents are not Estonian.\\
\ex Some/many people are Estonian and some/many people are not Estonian.
\z
\z


As a final example, the contrast in \REF{ex:2.8} suggests that neither \textit{every student} nor \textit{all students} can be interpreted as referring to the set of all students, e.g. at a particular school. There is much more to be said about quantifiers. We will give a brief introduction to this topic in \chapref{sec:3}, and discuss them in more detail in \chapref{sec:14}.


\ea \label{ex:2.8}
\ea The student body outnumbers the faculty.\\                
\ex \#Every student outnumbers the faculty.\\
\ex \#All students outnumbers the faculty.
\z
\z


Common noun phrases may or may not refer to anything. Definite noun phrases (sometimes called \textsc{definite descriptions}) like those in \REF{ex:2.9} are normally used in contexts where the hearer is able to identify a unique referent. But definite descriptions can also be used generically, without referring to any specific individual, like the italicized phrases in \REF{ex:2.10}.


\ea \label{ex:2.9}
\ea this book\\
\ex the sixteenth President of the United States\\
\ex my eldest brother
                       \z
\z

\ea \label{ex:2.10}
Life’s battles don’t always go\\
\hspace{5mm}   To \textit{the stronger or faster man},\\
But sooner or later \textit{the man who wins}\\
\hspace{5mm}   Is \textit{the one who thinks he can}.\footnote{From the poem “Thinking” by Walter D. Wintle, first published 1905(?). This poem is widely copied and often mis-attributed. Authors wrongly credited with the poem include Napoleon Hill, C.W. Longenecker, and the great American football coach Vince Lombardi.}
\z


\textsc{Indefinite descriptions} may be used to refer to a specific individual, like the object NP in (\ref{ex:2.11}a); or they may be non-specific, like the object NP in (\ref{ex:2.11}b). Specific indefinites are referring expressions, while non-specific indefinites are not.


\ea \label{ex:2.11}
\ea My sister has just married \textit{a cowboy}.\\
\ex My sister would never marry \textit{a cowboy}.\\
\ex My sister wants to marry \textit{a cowboy}.
                       \z
\z


In some contexts, like (\ref{ex:2.11}c), an indefinite NP may be ambiguous between a specific vs. a non-specific interpretation. Under the specific interpretation, (\ref{ex:2.11}c) says that my sister wants to marry a particular individual, who happens to be a cowboy. Under the non-specific interpretation, (\ref{ex:2.11}c) says that my sister would like the man she marries to be a cowboy, but doesn’t have any particular individual in mind yet. We will discuss this kind of ambiguity in more detail in \chapref{sec:12}.


\section{Sense vs. denotation}\label{sec:2.4}
\largerpage
In \sectref{sec:2.1} we noted that when people talk about what a word or phrase “means”, they may have in mind either its dictionary definition or its referent in a particular context. The German logician Gottlob Frege (1848–1925) was one of the first people to demonstrate the importance of making this distinction. He used the German term \textit{Sinn} (English \textsc{sense}) for those aspects of meaning which do not depend on the context of use, the kind of meaning we might look up in a dictionary.



Frege used the term \textit{Bedeutung} (English \textsc{denotation})\footnote{The term \textit{Bedeutung} is often translated into English as \textit{reference}, but this can lead to confusion when dealing with non-referring expressions which nevertheless do have a denotation.} for the other sort of meaning, which does depend on the context. The denotation of a referring expression, such as a proper name or definite NP, will normally be its referent. The denotation of a content word (e.g. an adjective, verb, or common noun) is the set of all the things in the current universe of discourse which the word could be used to describe. For example, the denotation of \textit{yellow} is the set of all yellow things, the denotation of \textit{tree} is the set of all trees, the denotation of the intransitive verb \textit{snore} is the set of all creatures that snore, etc. Frege proposed that the denotation of a sentence is its truth value. We will discuss his reasons for making this proposal in \chapref{sec:12}; in this section we focus on the denotations of words and phrases.



We have said that denotations are context-dependent. This is not so easy to see in the case of proper names, because they always refer to the same individual. Other referring expressions, however, will refer to different individuals or entities in different contexts. For example, the definite NP \textit{the Prime Minister} can normally be used to identify a specific individual. Which particular individual is referred to, however, depends on the time and place. The denotation of this phrase in Singapore in 1975 would have been Lee Kuan Yew; in England in 1975 it would have been Harold Wilson; and in England in 1989 it would have been Margaret Thatcher. Similarly, the denotation of phrases like \textit{my favorite color} or \textit{your father} will depend on the identity of the speaker and/or addressee.



The denotation of a content word depends on the situation or universe of discourse in which it is used. In our world, the denotation set of \textit{talks} will include most people, certain mechanical devices (computers, GPS systems, etc.) and (perhaps) some parrots. In Wonderland, as described by Lewis Carroll, it will include playing cards, chess pieces, at least one white rabbit, at least one cat, a dodo bird, etc. In Narnia, as described by C.S. Lewis, it will include beavers, badgers, wolves, some trees, etc.



For each situation, the sense determines a denotation set, and knowing the sense of the word allows speakers to identify the members of this set. When Alice first hears the white rabbit talking, she may be surprised. However, her response would not be, “What is that rabbit doing?” or “Has the meaning of \textit{talk} changed?” but rather “How can that rabbit be talking?” It is not the language that has changed, but the world. Sense is a fact about the language, denotation is a fact about the world or situation under discussion.



Two expressions that have different senses may still have the same denotation in a particular situation. For example, the phrases \textit{the largest land mammal} and \textit{the African bush elephant} refer to the same organism in our present world (early in the 21\textsuperscript{st} century). But in a fictional universe of discourse (e.g., the movie \textit{King Kong}), or in an earlier time period of our own world (e.g., 30 million BC, when the gigantic \textit{Paraceratherium} —estimated weight about 20,000 kg— walked the earth), these two phrases could have different denotations. If two expressions can have different denotations in any context, they do not have the same sense.



Such examples demonstrate that two expressions which have different senses \textsc{may} have the same denotation in certain situations. However, two expressions that have the same sense (i.e., \textsc{synonymous} expressions) must \textsc{always} have the same denotation in any possible situation. For example, the phrases \textit{my mother-in-law} and \textit{the mother of my spouse} seem to be perfect synonyms (i.e., identical in sense). If this is true, then it will be impossible to find any situation where they would refer to different individuals when spoken by the same (monogamous) speaker under exactly the same conditions.



So, while we have said that we will adopt a primarily “denotational” approach to semantics, this does not mean that we are only interested in denotations, or that we believe that denotation is all there is to meaning. If meaning was just denotation, then phrases like those in \REF{ex:2.12}, which have no referent in our world at the present time, would all either mean the same thing, or be meaningless. But clearly they are not meaningless, and they do not all mean the same thing; they simply fail to refer. 


\ea \label{ex:2.12}
\ea the present King of France\\
\ex the largest prime number\\
\ex the diamond as big as the Ritz\\
\ex the unicorn in the garden
                       \z
\z


Frege’s distinction allows us to see that non-referring expressions like those in \REF{ex:2.12} may not have a referent, but they do have a sense, and that sense is derived in a predictable way by the normal rules of the language.


\section{Ambiguity}\label{sec:2.5}

It is possible for a single word to have more than one sense. For example, the word \textit{hand} can refer to the body part at the end of our arms; the pointer on the dial of a clock; a bunch of bananas; the group of cards held by a single player in a card game; or a hired worker. Words that have two or more senses are said to be \textsc{ambiguous} (more precisely, \textsc{polysemous}; see \chapref{sec:5}).



A deictic expression such as \textit{my father} will refer to different individuals when spoken by different speakers, but this does not make it ambiguous. As emphasized above, the fact that a word or phrase can have different denotations in different contexts does not mean that it has multiple senses, and it is important to distinguish these two cases. We will discuss the basis for making this distinction in \chapref{sec:5}.



If a phrase or sentence contains an ambiguous word, the phrase or sentence will normally be ambiguous as well, as illustrated in \REF{ex:2.13}.


\ea \label{ex:2.13}
\textsc{lexical ambiguity}\\
\ea A boiled egg is hard to \textit{beat}.\\
\ex The farmer allows walkers to cross the field for free, but the bull \textit{charges}.\\
\ex I just turned 51, but I have a nice new \textit{organ} which I enjoy tremendously.\footnote{From e-mail newsletter, 2011.}
                       \z
\z


An ambiguous sentence is one that has more than one sense, or “reading”. A sentence which has only a single sense may have different truth values in different contexts, but will always have one consistent truth value in any specific context. With an ambiguous sentence, however, there must be at least one conceivable context in which the two senses would have different truth values. For example, one reading of (\ref{ex:2.13}b) would be true at the same time that the other reading is false if there is a bull in the field which is aggressive but not financially sophisticated.\largerpage[3]



In addition to lexical ambiguity of the kind illustrated in \REF{ex:2.13}, there are various other ways in which a sentence can be ambiguous. One of these is referred to as \textsc{structural ambiguity}, illustrated in (\ref{ex:2.14}a--d). In such cases, the two senses (or readings) arise because the grammar of the language can assign two different structures to the same string of words, even though none of those words is itself ambiguous. The two different structures for (\ref{ex:2.14}d) are shown by the bracketing in (\ref{ex:2.14}e), which corresponds to the expected reading, and (\ref{ex:2.14}f) which corresponds to the Groucho Marx reading. Of course, some sentences involve both structural and lexical ambiguity, as is the case in (\ref{ex:2.14}c).


\ea \label{ex:2.14}
\textsc{structural ambiguity}\footnote{These examples are taken from \citet[102]{Pinker1994}. The first three are said to be actual newspaper examples.}\\
\ea Two cars were reported stolen by the Groveton police yesterday.\\
\ex The license fee for altered dogs with a certificate will be \$3 and for pets owned by senior citizens who have not been altered the fee will be \$1.50.\\
\ex For sale: mixing bowl set designed to please a cook with round bottom for efficient beating.\\
\ex One morning I shot an elephant in my pajamas. How he got into my pajamas I’ll never know.\footnote{Groucho Marx, in the movie \textit{Animal Crackers}.}\\
\ex One morning I [shot an elephant] [in my pajamas].\\
\ex One morning I shot [an elephant in my pajamas].
                       \z
\z


Structural ambiguity shows us something important about meaning, namely that meanings are not assigned to strings of phonological material but to syntactic objects.\footnote{\citet[514]{Kennedy2011}.} In other words, syntactic structure makes a crucial contribution to the meaning of an expression. The two readings for (\ref{ex:2.14}d) involve the same string of words but not the same syntactic object.



A third type of ambiguity which we will mention here is \textsc{referential ambiguity}. (We will discuss additional types of ambiguity in later chapters.) It is fairly common to hear people using pronouns in a way that permits more than one possible antecedent, e.g. \textit{Adams wrote frequently to Jefferson while he was in Paris}. The pronoun \textit{he} in this sentence has ambiguous reference; it could refer either to John Adams or to Thomas Jefferson. It is also possible for other types of NP to have ambiguous reference. For example, if I am teaching a class of 14 students, and I say to the Dean \textit{My student has won a Rhodes scholarship}, there are multiple possible referents for the subject NP.\largerpage[1]



A famous example of referential ambiguity occurs in a prophecy from the oracle at Delphi, in ancient Greece. The Lydian king Croesus asked the oracle whether he should fight against the Persians. The oracle’s reply was that if Croesus made war on the Persians, he would destroy a mighty empire. Croesus took this to be a positive answer and attacked the Persians, who were led by Cyrus the Great. The Lydians were defeated and Croesus was captured; the empire which Croesus destroyed turned out to be his own.


\section{Expressive meaning: \textit{Ouch} and \textit{oops}}\label{sec:2.6}

Words like \textit{ouch} and \textit{oops}, often referred to as \textsc{expressives}, present an interesting challenge to the “denotational” approach outlined above. They convey a certain kind of meaning, yet they neither refer to things in the world, nor help to determine the conditions under which a sentence would be true. In fact, it is hard to claim that they even form part of a sentence; they seem to stand on their own, as one-word utterances. The kind of meaning that such words convey is called \textsc{expressive meaning}, which \citet[44]{Lyons1995} defines as “the kind of meaning by virtue of which speakers express, rather than describe, their beliefs, attitudes and feelings.” Expressive meaning is different from \textsc{descriptive meaning} (also called \textsc{propositional meaning} or \textsc{truth-} \textsc{conditional} \textsc{meaning}), the “normal” type of meaning which determines reference and truth values. If someone says \textit{I just felt a sudden sharp pain}, he is describing what he feels; but when he says \textit{Ouch!}, he is expressing that feeling.



Words like \textit{ouch} and \textit{oops} carry only expressive meaning, and seem to be unique in other ways as well. They may not necessarily be intended to communicate. If I hurt myself when I am working alone, I will very likely say \textit{ouch} (or some other expressive with similar meaning) even though there is no one present to hear me. Such expressions seem almost like involuntary reactions, although the specific forms are learned as part of a particular language. But it is important to be aware of the distinction between expressive vs. descriptive meaning, because many “normal” words carry both types of meaning at once.



For example, the word \textit{garrulous} means essentially the same thing as \textit{talkative}, but carries additional information about the speaker’s negative attitude towards this behavior.\footnote{\citet{Barker2002}.} There are many other pairs of words which seem to convey the same descriptive meaning but differ in terms of their expressive meaning: \textit{father} vs. \textit{dad}; \textit{woman} vs. \textit{broad}; \textit{horse} vs. \textit{nag}; \textit{alcohol} vs. \textit{booze}; etc. In each case either member of the pair could be used to refer to the same kinds of things in the world; the speaker’s choice of which term to use indicates varying degrees of intimacy, respect, appreciation or approval, formality, etc.


\largerpage
The remainder of this section discusses some of the properties which distinguish expressive meaning from descriptive meaning.\footnote{Much of this discussion is based on \citet{Cruse1986,Cruse2000} and \citet{Potts2007c}.} These properties can be used as diagnostics when we are unsure which type of meaning we are dealing with.


\subsection{Independence}\label{sec:2.6.1}

Expressive meaning is independent of descriptive meaning in the sense that expressive meaning does not affect the denotation of a noun phrase or the truth value of a sentence. For example, the addressee might agree with the descriptive meaning of \REF{ex:2.15} without sharing the speaker’s negative attitude indicated by the expressive term \textit{jerk}. Similarly, the addressee in \REF{ex:2.16} might agree with the descriptive content of the sentence without sharing the speaker’s negative attitude indicated by the pejorative suffix \textit{-aco}.


\ea \label{ex:2.15}
That \textit{jerk} Peterson is the only real economist on this committee.
\z

\ea \label{ex:2.16}
\gll Los  vecinos  tienen  un  pajarr-\textit{aco}  como  mascota.  [\ili{Spanish}]\\
the  neighbors  have  a  bird-\textsc{pejor}  as  pet\\
\glt Descriptive: The neighbors have a pet bird.\\
Expressive: The speaker has a negative attitude towards the bird.\footnote{\citet{Fortin2011}.}
\z

\subsection{Nondisplaceability}\label{sec:2.6.2}

\citet{Hockett1958,Hockett1960} used the term \textsc{Displacement} to refer to the fact that speakers can use human languages to describe events and situations which are separated in space and time from the speech event itself. Hockett listed this ability as one of the distinctive properties of human language, one which distinguishes it, for example, from most types of animal communication.



\citet[272]{Cruse1986} notes that this capacity for displacement holds only for descriptive meaning, and not for expressive meaning. A person can describe his own feelings in the past or future, e.g. \textit{Last month I felt a sharp pain in my chest}, or \textit{I will probably feel a lot of pain when the dentist drills my tooth tomorrow}; or the feelings of other people, e.g. \textit{She was in} \textit{a lot of pain}. But when a person says \textit{Ouch!}, it must normally express pain that is felt by the speaker at the moment of speaking.


\subsection{Immunity}\label{sec:2.6.3}

Descriptive meaning can be negated (\ref{ex:2.17}a), questioned (\ref{ex:2.17}b), or challenged (\ref{ex:2.17}c). Expressive meaning is “immune” to all of these things, as illustrated in \REF{ex:2.18}. As we will see in later chapters, negation, questioning, and challenging are three of the standard tests for identifying truth-conditional meaning. The fact that expressive meaning cannot be negated, questioned, or challenged shows that it is not part of the truth-conditional meaning of the sentence.


\ea \label{ex:2.17}
\ea  {I am not feeling any pain.}\\
\ex  {Are you feeling any pain?}\\
\ex  \textsc{patient}:  {I just felt a sudden sharp pain.}\\
  \textsc{dentist}:  {That’s a lie — I gave you a double dose of Novocain.}\\
    (\citealt{Cruse1986}: 271)
\z
                       \z

\ea \label{ex:2.18}
\ea  {*Not ouch.}\\
\ex  {*Ouch?}  (can only be interpreted as an elliptical form of the question:\\
     \textit{Did you say “Ouch”?})\\
\ex  \textsc{patient}:  {Ouch!}\\
  \textsc{dentist}: \#{That’s a lie.}
                       \z
\z

\subsection{Scalability and repeatability}\label{sec:2.6.4}

Expressive meaning can be intensified through repetition (as seen in line g of \tabref{extab:2.21} below), or by the use of intonational features such as pitch, length or loudness. Descriptive meaning is generally expressible in discrete units which correspond to the lexical semantic content of individual words. Repetition of descriptive meaning tends to produce redundancy, though we should note that a number of languages do use reduplication to encode plural number, repeated actions, etc.


\subsection{Descriptive ineffability}\label{sec:2.6.5}

“Effability” means ‘expressibility’. The \textsc{effability hypothesis} claims that “Each proposition can be expressed by some sentence in any natural language”;\footnote{\citet[209]{Katz1978}.} or in other words, “Whatever can be meant can be said.”\footnote{\citet[18]{Searle1969}; see also \citet[18--24]{Katz1972}; \citet[33]{Carston2002}.}



\citet{Potts2007c} uses the phrase “descriptive ineffability” to indicate that expressive meaning often cannot be adequately stated in terms of descriptive meaning. A paraphrase based on descriptive meaning (e.g. \textit{young dog} for \textit{puppy}) is often interchangeable with the original expression, as illustrated in \REF{ex:2.19}. Whenever (\ref{ex:2.19}a) is true, (\ref{ex:2.19}b) must be true as well, and vice versa. Moreover, this substitution is equally possible in questions, commands, negated sentences, etc. This is not the case with expressives, even where a descriptive paraphrase is possible, as illustrated in (\ref{ex:2.17}--\ref{ex:2.18}) above.


\ea \label{ex:2.19}
\ea  {Yesterday my son brought home a puppy.}\\
\ex  {Yesterday my son brought home a young dog.}
                       \z
\z


For many expressives there is no descriptive paraphrase available, and speakers often find it difficult to explain the meaning of the expressive form in descriptive terms. For example, most dictionaries do not attempt to paraphrase the meaning of \textit{oops}, but rather “define” it by describing the contexts in which it is normally used:


\ea

\ea “used typically to express mild apology, surprise, or dismay”\footnote{\url{http://www.merriam-webster.com}}\\
%  (http://www.merriam-webster.com)\\
\ex “an exclamation of surprise or of apology as when someone drops something or makes a mistake”\footnote{\textit{Collins English Dictionary – Complete and Unabridged}, ©HarperCollins Publishers 1991, 1994, 1998, 2000, 2003.} \\ 
% \\ (Collins English Dictionary – Complete and Unabridged, © HarperCollins Publishers 1991, 1994, 1998, 2000, 2003)
\z
\z


This limited expressibility correlates with limited translatability. The descriptive meaning conveyed by a sentence in one language is generally expressible in other languages as well. (Whether this is always the case, as predicted by strong forms of the Effability Hypothesis, is a controversial issue.) However, it is often difficult to find an adequate translation equivalent for expressive meaning. One well known example is the ancient \ili{Aramaic} term of contempt \textit{raka}, which appears in the \ili{Greek} text of Matthew 5:22 (and in many English translations), presumably because there was no adequate translation equivalent in Koine \ili{Greek}. (Some of the English equivalents which have been suggested include: \textit{good-for-nothing}, \textit{rascal}, \textit{empty head}, \textit{stupid}, \textit{ignorant}.) In 393 AD, St. Augustine offered the following explanation:


\begin{quote}
Hence the view is more probable which I heard from a certain Hebrew whom I had asked about it; for he said that the word does not mean anything, but merely expresses the emotion of an angry mind. Grammarians call those particles of speech which express an affection of an agitated mind \textsc{interjections}; as when it is said by one who is grieved, ‘Alas,’ or by one who is angry, ‘Hah.’ And these words in all languages are proper names, and are not easily translated into another language; and this cause certainly compelled alike the \ili{Greek} and the \ili{Latin} translators to put the word itself, inasmuch as they could find no way of translating it.”\footnote{\textit{On the Sermon on the Mount}, Book I, ch. 9, §23; \url{http://www.newadvent.org/fathers/16011.htm}} 
\end{quote}


Whether or not Augustine was correct in his view that \textit{raka} was a pure expressive, he provides an excellent description of this class of words and the difficulty of translating them from one language to another. This quote also demonstrates that the challenges posed by expressives have been recognized for a very long time.



A similar translation problem helped to create an international incident in 1993 when the Malaysian Prime Minister, Dr. Mahathir Mohamad, declined an invitation to attend the first Asia-Pacific Economic Cooperation (APEC) summit. {Australian} Prime Minister Paul Keating, when asked for a comment, replied: “APEC is bigger than all of us; Australia, the US and Malaysia and Dr Mahathir and any other recalcitrants.” Bilateral relations were severely strained, and both Malaysian government policies and Malaysian public opinion towards Australia were negatively affected for a long period of time. A significant factor in this reaction was the fact that the word \textit{recalcitrant} was translated in the Malaysian press by the \ili{Malay} idiom \textit{keras kepala}, literally ‘hard headed’. The two expressions have a similar range of descriptive meaning (‘stubborn, obstinate, defiant of authority’), but the \ili{Malay} idiom carries expressive meaning which makes the sense of insult and disrespect much stronger than in the English original. \textit{Keras kepala} would be appropriate in scolding a child or subordinate, but not in referring to a head-of-government.


\subsection{Case study: Expressive uses of diminutives}\label{sec:2.6.6}

Diminutives are grammatical markers whose primary or literal meaning is to indicate small size; but diminutives often have secondary uses as well, and often these involve expressive content. Anna \citet{Wierzbicka1985} describes one common use of diminutives in \ili{Polish} as follows:\largerpage


\begin{quote}
In \ili{Polish}, warm hospitality is expressed as much by the use of diminutives as it is by the ‘hectoring’ style of offers and suggestions. Characteristically, the food items offered to the guest are often referred to by the host by their diminutive names. Thus… one might say in \ili{Polish}: \textit{Wei jeszcze Sledzika! Koniecznie!} ‘Take some more dear-little-herring (\textsc{dim}). You must!’ The diminutive praises the quality of the food and minimizes the quantity pushed onto the guest’s plate. The speaker insinuates: “Don’t resist! It is a small thing I’m asking you to do — and a good thing!”. The target of the praise is in fact vague: the praise seems to embrace the food, the guest, and the action of the guest desired by the host. The diminutive and the imperative work hand in hand in the cordial, solicitous attempt to get the guest to eat more.
\end{quote}


Markers of expressive meaning often have several possible meanings, which depend heavily on context, and this is true for the \ili{Spanish} diminutive suffixes as illustrated in 
% \REF{ex:2.21}.
\tabref{extab:2.21}.
Notice that the same diminutive suffix can have nearly opposite meanings (deprecation vs. appreciation; exactness vs. approximation; attenuation vs. intensification) in different contexts (and, in some cases, different dialects). These examples also illustrate the “scalability” of expressive meaning, the fact that it can be intensified through repetition, as in \textit{chiqu-it-it-o}.

% \ea \label{ex:2.21}
% The expressive uses of \ili{Spanish} diminutive suffixes \citep{Fortin2011}

% \ea Deprecation

% \textit{mujer-zuela}  woman\textsc{-dim}  ‘disreputable woman’ + disdain/mockery

 % \ex Appreciation

% \textit{niñ-ito}  boy-\textsc{dim}  ‘boy’ + endearment/affection

% \ex  Hypocorism (nick-name, pet name)

% \textit{Carol-ita}  Carol-\textsc{dim}  ‘Carol’ + endearment

% \ex Exactness

% \textit{igual-ito}  the.same-\textsc{dim}  ‘exactly the same’

% \ex Approximation

% \textit{floj-illo}  lazy-\textsc{dim}  ‘kind of lazy, lazy-ish’

% \ex Attenuation

% \textit{ahor-ita}  now-\textsc{dim}  ‘soon, in a little while’ (Caribbean \ili{Spanish})

% \ex Intensification

% \textit{ahor-ita}  now-\textsc{dim}  ‘immediately, right now’ (\ili{Latin} American \ili{Spanish})

% \textit{chiqu-it-o}  small-\textsc{dim-masc}  ‘very small’\\
% \textit{chiqu-it-it-o}  small-\textsc{dim-dim-masc}  ‘very, very small/teeny-weeny’\\
% \textit{chiqu-it-it-…-it-o}  small-\textsc{dim}-\textsc{dim}-\textsc{…}-\textsc{dim}-\textsc{masc} ‘very, very, …, very, small’
% \z
% \z

\begin{table}
\caption{The expressive uses of Spanish diminutive suffixes. (Data from \citealt{Fortin2011}.)}
\label{extab:2.21}
\begin{tabularx}{\textwidth}{lp{4.5cm}@{}Q}
\lsptoprule
a. &  Deprecation\\
   &\textit{mujer-zuela}\newline woman\textsc{-dim}  	&  ‘disreputable woman’      + disdain/mockery\\
\tablevspace   
b. &  Appreciation\\
   &\textit{niñ-ito} \newline 	 boy-\textsc{dim}  	&  ‘boy’    + endearment/affection\\
\tablevspace   
c. &  \multicolumn{2}{l}{Hypocorism (nick-name, pet name)}\\
   & \textit{Carol-ita} \newline  Carol-\textsc{dim} 	& ‘Carol’  + endearment\\
\tablevspace   
d. &  Exactness\\
   & \textit{igual-ito}  \newline the.same-\textsc{dim}& ‘exactly the same’\\
\tablevspace   
e. & Approximation\\
   & \textit{floj-illo} \newline  lazy-\textsc{dim} 	& ‘kind of lazy, lazy-ish’\\
\tablevspace   
f. &  Attenuation\\
   & \textit{ahor-ita} \newline  now-\textsc{dim}  	& ‘soon, in a little while’\newline   (Caribbean \ili{Spanish})\\ 
\tablevspace   
g. & Intensification\\
   & \textit{ahor-ita} \newline  now-\textsc{dim}  	& ‘immediately, right now’\newline   ( {Latin} American \ili{Spanish})\\  
\tablevspace   
   & \textit{chiqu-it-o}\newline small-\textsc{dim-masc}& ‘very small’\\
\tablevspace   
   & \textit{chiqu-it-it-o} \newline small-\textsc{dim-dim-masc} & ‘very, very small/teeny-weeny’\\
\tablevspace   
   & \textit{chiqu-it-it-…-it-o}\newline  small-\textsc{dim}-\textsc{dim}-\textsc{…}-\textsc{dim}-\textsc{masc} & ‘very, very, …, very, small’\\ 
\lspbottomrule
\end{tabularx}
\end{table}


\section{Conclusion}\label{sec:2.7}

In this chapter we started with the observation that speakers use language to talk about the world, for example by referring to things or describing states of affairs. We introduced the distinction between sense and denotation, which is of fundamental importance in all that follows. Knowing the sense of a word is what makes it possible for speakers of a language to identify the denotation of that word in a particular context of use. In a similar way, as we discuss in \chapref{sec:3}, knowing the sense of a sentence is what makes it possible for speakers of a language to judge whether or not that sentence is true in a particular context of use. The issue of ambiguity (a single word, phrase, or sentence with more than one sense) is one that we will return to often in the chapters that follow. Finally, we demonstrated a number of ways in which this kind of descriptive meaning (talking about the world) is different from expressive meaning (expressing the speaker’s emotions or attitudes). In the rest of this book, we will focus primarily on descriptive meaning rather than expressive meaning; but it is important to remember that both “dimensions” of meaning are involved in many (if not most) utterances.\clearpage



\furtherreading{



\citet[Ch.~4]{Birner20122013} provides a helpful overview of reference and various related issues.  \citet[Ch.~2]{Abbott2010} provides a good summary of early work by Frege and other philosophers on the distinction between sense and denotation; later chapters provide in-depth discussions of various types of referring expressions.  For additional discussion of expressive meaning see \citet{Cruse1986,Cruse2000}, \citet{Potts2007b}, and \citet{Kratzer1999}.
}
% \subsection*{Discussion exercises}
\discussionexercises{
\paragraph*{A: Sense vs. denotation.}

Which of the following pairs of expressions have the same sense? Which have the same denotation? Explain your answer.

\medskip 
\noindent
\fittable{
\begin{tabular}{lll}
a. &  \small cordates (=‘animals with hearts’)  & \small renates (= ‘animals with kidneys’)  \\
b. & \small animals with gills and scales & \small fish \\
c. & \small your first-born son & \small your oldest male offspring \\
d. & \small Ronald Reagan & \small the Governor of California \\
e. & \small my oldest sister & \small your Aunt Betty \\
f. & \small my pupils & \small the students that I teach \\
g. & \small the man who invented the phonograph & \small the man who invented the light-bulb \\
\end{tabular}
}


\medskip
\modelanswer{Model answer for (a)}{
\begin{quote} In our world at the present time, all species that have hearts also have kidneys; so these two words have the same denotation in our world at the present time. They do not have the same sense, however, because we can imagine a world in which some species had hearts without kidneys, or kidneys without hearts; so the two words do not have the same denotation in all possible situations. \end{quote}
}

\paragraph*{B: Referring expressions.}

% \ea
Which of the following NPs are being used to \textit{refer} to something?
% \z

\begin{enumerate}[label=\alph*.]
\item I never promised you \textit{a rose garden}.
\item St. Benedict, the father of Western monasticism, planted \textit{a rose garden} at his early monastery in Subiaco near Rome.\footnote{\url{http://www.scu.edu/stclaregarden/ethno/medievalgardens.cfm}}
\item My sister wants to marry \textit{a policeman}.
\item My sister married \textit{a policeman}.
\item Leibniz searched for \textit{the solution to the equation}.
\item Leibniz discovered \textit{the solution to the equation}.
\item \textit{No cat} likes being bathed.
\item \textit{All musicians} are temperamental.
\end{enumerate}
}
% \subsection*{Homework exercises}
\homeworkexercises{
\paragraph*{A: Idiomatic meaning.}

Try to find one phrasal idiom (an idiom consisting of two or more words) in a language other than English; give a word-for-word translation and explain its idiomatic meaning.

\paragraph*{B: Expressive meaning.}

Try to find a word in a language other than English which has purely expressive meaning, like \textit{oops} and \textit{ouch}; and explain how it is used.

\paragraph*{C: Referring expressions.}

For each of the following sentences, state\\ whether or not the nominal expression in italics is being used to refer.

\begin{enumerate}[label=\alph*.]
\item Abraham Lincoln was very close to \textit{his step-mother}.\\[1em]
\modelanswer[.8]{Model answer}{The phrase \textit{his step-mother} is used to refer to a specific person,\\
  namely {Sarah Bush} Lincoln, so it does refer}
\item  I’m so hungry I could eat \textit{a horse}.
\item  {Senate Majority Leader Curt Bramble, R-Provo, was back in the hospital this weekend after getting} kicked by \textit{a horse}.\footnote{Provo, UT \textit{Daily Herald} Jan. 29, 2007.}
\item  Police searched the house for 6 hours but found \textit{no drugs}.
\item  Edward hopes that his on-line match-making service will help him find   \textit{the girl of his dreams}.
\item  Susan married \textit{the first man who proposed to her}.
\item  \textit{Every city} has pollution problems.
\end{enumerate}
}
\chapter{Kombinationen aus \textit{ja} und \textit{doch}}
\label{chapter:jud}
\section{Distribution von \textit{ja}, \textit{doch} und \textit{ja doch}}
\label{sec:distributionjd}
Es ist aus der Literatur bekannt, dass sich bestimmte MPn nur in bestimmten Domänen kombinieren lassen, wobei \textit{Domäne} hier im Sinne von \textit{Satztyp}, \textit{Satzmodus} bzw. \textit{Illokutionstyp} zu verstehen ist. Welches der drei genannten Konzepte das Kriterium ist, ist unklar. Die Annahmen unterscheiden sich je nach betrachte\-ten MPn. Dazu kommt, dass es in der Literatur keinen Konsens in Bezug auf die Auffassung der drei Konzepte gibt sowie dass im Einzelfall Uneinigkeit darüber besteht, wie bestimmte Konstruktionstypen zu klassifizieren sind. Auf konkrete Fälle dieser generellen Problematik verweise ich im Laufe des Kapitels. Trotz dieser Problematik muss einer Analyse der Abfolge von MPn in Kombinationen aber eine Bestimmung der beteiligten sprachlichen Kontexte, in denen die Kombination der betrachteten MPn überhaupt möglich ist, vorangehen. Neben der Festlegung derartiger (eher) struktureller Domänen ist dazu verschiedentlich beobachtet worden, dass auch semantische und pragmatische Faktoren Einfluss auf zulässige Domänen der Kombination nehmen. 

\subsection{Syntaktische Schnittmengenbedingung}
\label{sec:synschnitt}
\citet[20]{Thurmair1991} definiert die Beobachtung, dass sich MPn nur in bestimmten Umgebungen miteinander kombinieren lassen, über eine satzmodale Schnittmengenbedingung \is{satzmodale Schnittmengenbedingung}:
\begin{quotation}
[...] a modal particle A, which may only appear with sentence mood Z, and a modal particle B, which may only appear with sentence mood Y (Z $\neq$ Y), should not be combinable, and [...] a modal particle A, which appears in sentence mood X and Y, may co-occur with a modal particle B, which appears in sentence moods Y and Z, only in sentence mood Y, that is in the intersection. 
\end{quotation}
MPn können demnach nur in den Satzmodi \is{Satzmodus} miteinander kombiniert werden, in denen sie auch allein auftreten können. Thurmair geht hierbei von der Satzmoduskonzeption nach \citet{Altmann1984, Altmann1987} aus. Er trennt bei der Beschreibung von Sätzen/Äußerungen zwischen Form und Funktion; der Satzmodus ist die regelmäßige Verbindung aus Formtyp \is{Formtyp} und Funktionstyp \is{Funktionstyp} (vgl. \citealt[22]{Altmann1987}). Die von ihm angenommenen Formtypen lassen sich entlang grammatischer Eigenschaften (wie Verbstellung, Verbmorphologie, Tonmuster, Obligatorizität eines w-Elements, Exklamativakzent) beschreiben. Diesen Formtypen wird je\-weils ein Funktionstyp zugeordnet.\footnote{Ich verwende hier im Folgenden Thurmairs Terminologie, da ich mich an ihrer Darstellung orientiere, wähle ansonsten in der Arbeit aber selbst die üblichen Satzmodusbezeichnungen \textit{Deklarativ}-, \textit{Interrogativ}-, \textit{Imperativ}-, \textit{Optativ}- und \textit{Exklamativsatz} bzw. nehme die Betrachtung ab Abschnitt~\ref{sec:nonkan} aus der Perspektive des Diskursbeitrags, und damit der Illokution, der Äußerungen vor. Für die Fälle, in denen man es mit einer 1:1-Zuordnung von Satztyp, Satzmodus und Illokution zu tun hat, ist die Unterscheidung zwischen den Konzepten auch unerheblich für meine Argumentation.}

\begin{exe}
\ex\label{262}
\begin{tabular}[t]{lll}
  	\textbf{Formtypen} & & \textbf{Funktionstypen}\\
  	Aussagesatz & \scriptsize{\textit{Anja hat eine schöne Wohnung.}} & Assertion\\
  	E-Fragesatz & \scriptsize{\textit{Wohnt Anja alleine hier?}} & Frage\\
  	w-Fragesatz & \scriptsize{\textit{Wer wohnt in dieser Wohnung?}} & Frage\\
  	Imperativsatz & \scriptsize{\textit{Putz die Wohnung!}} & Aufforderung\\
  	Wunschsatz & \scriptsize{\textit{Hätte ich doch auch so eine Wohnung!}} & Wunsch\\
  	(Satz)Exklamativsatz & \scriptsize{\textit{Hast DU eine schöne Wohnung!}} & Exklamativ\\
  	w-Exklamativsatz & \scriptsize{\textit{Wie HELL ist die Wohnung!}} & Exklamativ
\end{tabular}
\end{exe}
Basierend auf den Formtypen setzt \citet[49]{Thurmair1989} die Verteilung von \textit{ja} und \textit{doch} (und damit das gemeinsame Auftreten) wie in (\ref{263}) an.\footnote{Übersichten dieser Art finden sich z.B. auch in \citet[59]{Karagjosova2004} und \citet[183]{Kwon2005}. Wie anfänglich erwähnt, können derartige Klassifikationen je nach Arbeit anders aussehen. Zu den Gründen, die ich dafür anführe, s.o. Am Beispiel der Verteilung von \textit{ja} und \textit{doch} lassen sich konkrete Unterschiede in der Zuordnung illustrieren: Bei \citet{Karagjosova2004} beispielsweise tritt \textit{ja} (anders als bei \citealt{Thurmair1989}) in Satz- und \textit{dass}-Exklamativsätzen auf. Bei \citet[157]{Hentschel1986} stellen hingegen nicht Formtypen, sondern Funktionstypen (Assertionen und Exkla\-mationen) die zulässige Domäne dar.}

\begin{exe}
	\ex\label{263}
	\tiny
     \begin{tabular}[t]{|l|l|l|l|l|l|l|l|}
     		\hline
     		& Aussagesatz & E-Fragesatz & w-Fragesatz & Imperativsatz & Wunschsatz & (Satz-)Exklamativsatz & w-Exklamativsatz\\
            \hline
            \textit{doch} & $\plus$ & $\minus$ & $\plus$ & $\plus$ & $\plus$ & $\minus$ & $\plus$\\
             \hline
             \textit{ja} & $\plus$ & $\minus$ & $\minus$ & $\minus$ & $\minus$ & $\minus$ & $\minus$\\
             \hline
      \end{tabular}\\
\end{exe}
Nach Thurmairs satzmodaler Schnittmengenbedingung (s.o.) sollten \textit{ja} und \textit{doch} gemeinsam nur im Aussagesatz auftreten können. Die Beispiele in den folgenden zwei Abschnitten bestätigen diese Vorhersage.

\subsubsection{Deklarativ-, Interrogativ-, Imperativ- und Optativ- und Exklamativsatz}
\label{sec:exkl}
Da weder \textit{doch} noch \textit{ja} im E-Fragesatz auftreten kann, ist auch die Kombination ausgeschlossen (vgl. (\ref{264})).

\begin{exe}
	\ex\label{264} 
	*Hast du \textbf{doch}/\textbf{ja}/\textbf{ja doch} am Wochenende Zeit?
\end{exe}
Der w-Fragesatz \is{w-Fragesatz}, Imperativsatz \is{Imperativsatz} und Wunschsatz \is{Wunschsatz} erlauben das Auftreten von \textit{doch}. Die Unzulässigkeit der Kombination folgt jedoch, da \textit{ja} in diesen satzmodalen Umgebungen nicht lizensiert ist (vgl. (\ref{265}) bis (\ref{267})).
	
\begin{exe}
	\ex\label{265} 
	Wie heißt \textbf{doch gleich}/*\textbf{ja gleich}/*\textbf{ja doch gleich} die Straße, in der du wohnst?
\end{exe}	
\vspace{-0.65cm}	
\begin{exe}
	\ex\label{266} 
	Mach' \textbf{doch}/*\textbf{ja}/*\textbf{ja doch} die Heizung an!
\end{exe}
\vspace{-0.65cm}	
\begin{exe}
	\ex\label{267} 
	Hätte ich \textbf{doch}/*\textbf{ja}/*\textbf{ja doch} am Gewinnspiel teilgenommen!
\end{exe}
In allen Fällen aus (\ref{265}) bis (\ref{267}) bleibt die Schnittmenge der Satzmodi, in denen die beiden MPn je in Isolation auftreten können, leer, weshalb die Kombination von \textit{ja} und \textit{doch} im Einvernehmen mit Thurmairs Beschränkung (s.o.) nicht akzep\-tabel ist.

Da \textit{ja} und \textit{doch} aber unabhängig voneinander in Aussagesätzen stehen können, führt auch ihr kombiniertes Auftreten zu einer wohlgeformten Struktur (vgl. (\ref{268})).

\begin{exe}
	\ex\label{268} 
	In Hamburg ist \textbf{doch}/\textbf{ja}/\textbf{ja doch} Hafenfest.
\end{exe}
Die in (\ref{265}) bis (\ref{268}) illustrierten Distributionsverhältnisse von \textit{ja}, \textit{doch} und \textit{ja doch} in E-Frage, w-Frage-, Imperativ-, Wunsch- und Aussagesätzen sind als unkontrovers einzustufen. Ausführlichere Ausführungen sind nötig in Bezug auf die Satz- und w-Exklamativsätze aus der Übersicht in (\ref{262}).\\

\noindent
In w-Exklamativsätzen, \is{w-Exklamativsatz} die nach \citet[45]{Thurmair1989} charakterisiert werden durch indikativischen Verbmodus, ein w-Element im Vorfeld, das mit einem wertenden Adjektiv/Adverb verbunden ist, das einen Akzent trägt, sowie fallenden Tonverlauf, kann \textit{doch}, nicht aber \textit{ja} auftreten. Dies gilt sowohl für w-V2-Exklamativsätze als auch für die VL-Variante dieses Konstruktionstyps (vgl. (\ref{269}) und (\ref{270})).

\begin{exe}
	\ex\label{269} 
		\begin{xlist}	
			\ex\label{269a} Was BIST du \textbf{doch}/*\textbf{ja bloß} für ein Mensch!
			\ex\label{269b} Was für eine Wohltat ist \textbf{doch}/*\textbf{ja} dieses Buch!
		\end{xlist}
\end{exe}

\begin{exe}
	\ex\label{270} 
		\begin{xlist}	
			\ex\label{270a} Wie SCHÖN du \textbf{doch}/*\textbf{ja} bist!	
			\hfill\hbox {\citet[218-219]{Rinas2006}}
			\ex\label{270b} Was für ein TOLLER KERL er \textbf{doch}/*\textbf{ja} ist!
			\newline
			\hbox{}\hfill\hbox {\citet[37]{Kwon2005}, (TAZ, 02.10.1996, VI)}
		\end{xlist}
\end{exe}
Da die Schnittmenge der Satzmodi hinsichtlich des Einzelauftretens von \textit{ja} und \textit{doch} leer ist, ist folglich auch die Kombination nicht möglich (vgl. (\ref{271})).

\begin{exe}
	\ex\label{271} 
		\begin{xlist}	
			\ex\label{271a} *Wie schön du \textbf{ja doch} bist!	
			\ex\label{271b} *Was für ein TOLLER KERL er \textbf{ja doch} ist!
			\ex\label{271c} *Was BIST du \textbf{ja doch bloß} für ein Mensch!	
			\ex\label{271d} *Was für eine Wohltat ist \textbf{ja doch} dieses Buch!
		\end{xlist}
\end{exe}
Zu den Satzexklamativsätzen \is{Satzexklamativsatz} zählen bei Thurmair Strukturen wie in (\ref{272}).

\begin{exe}
	\ex\label{272} 
		\begin{xlist}	
			\ex\label{272a} Hast DU eine schöne Wohnung!	
			\ex\label{271b} DU hast (vielleicht) eine schöne Wohnung!
		\end{xlist}
\end{exe}
Dieser Satztyp weist nach \citet[45]{Thurmair1989} die folgenden Charakteristika auf: V1- oder V2-Stellung, indikativischer Verbmodus, fallendes Tonmuster, Exklamativakzent. \citet[49]{Thurmair1989} vertritt die Annahme, dass in diesem satzmodalen Kontext weder \textit{ja} noch \textit{doch} auftreten können (vgl. auch \citealt[40]{Altmann1987}). 

Die Anwesenheit von \textit{doch} scheint wirklich ausgeschlossen (vgl. (\ref{273}) (vs. (\ref{274}))).

\begin{exe}
	\ex\label{273} 
	*DER hat \textbf{doch} einen Bart!
\end{exe}
\vspace{-0.65cm}	
\begin{exe}
	\ex\label{274} 
	DER hat \textbf{aber}/\textbf{vielleicht} einen Bart!	
			\hfill\hbox {\citet[218]{Rinas2006}}
\end{exe}																	     
\citet[141]{Hentschel1986} hält die Verwendung von \textit{doch} hier für veraltet. Nach \citet[26]{Weydt1969} ist (\ref{275}) grammatisch.

\begin{exe}
	\ex\label{275} 
	Kommst du \textbf{doch} spät!
\end{exe}
\citet[141]{Hentschel1986} hält die Sätze in (\ref{276}) für \glqq möglich\grqq{}, mit Akzent auf \textit{das} allerdings für  \glqq problematisch\grqq{}. 

\begin{exe}
	\ex\label{276} 
		\begin{xlist}	
			\ex\label{276a} War das \textbf{doch} ein Fest!	
			\ex\label{276b} Ist das \textbf{doch} schön!
		\end{xlist}
\end{exe}
M.E. sind die Strukturen in (\ref{273}), (\ref{275}) und (\ref{276}) alle ungrammatisch, wie auch schon von Thurmair und Rinas derart angenommen. Thurmairs Annahme widersprechend scheint mir \textit{ja} in diesem Kontext aber problemlos auftreten zu können (vgl. (\ref{277}) sowie den authentischen Beleg in (\ref{278a}), bei dem \textit{der} in diesem Kontext plausiblerweise akzentuiert wird).

\begin{exe}
	\ex\label{277} 
	DER hat \textbf{ja} einen Bart!	
			\hfill\hbox {\citet[221]{Rinas2006}}
\end{exe}
\vspace{-0.65cm}
\begin{exe}
	\ex\label{278a}
	\scriptsize
	 Bevor der Unterricht begann, mussten sich die Mädchen Kittelschürzen anziehen und die Jungen Westen, denn früher durften sich die Kinder nicht 			schmutzig machen. Außerdem bekamen wir alle Holzpantinen. Dann zeigte uns das Fräulein Lehrerin den Klassenraum. \glqq \textbf{Der ist \underline{ja} 		klein!}\grqq{}, riefen einige Kinder. 
	\hfill\hbox {(BRZ09/APR.06684 Braunschweiger Zeitung, 17.04.2009)} 
\end{exe}	
Im Einvernehmen mit Thurmairs satzmodaler Schnittmengenbedingung lassen sich \textit{ja} und \textit{doch} in dieser Domäne aufgrund der leeren Schnittmenge hinsichtlich ihres Einzelauftretens in Satzexklamativsätzen nicht kombinieren.\footnote{Interessanterweise findet man in diesem Typ von Exklamativsatz gelegentlich die Partikel \textit{mal} (vgl. (\ref{278})).

\begin{exe}
	\ex\label{278} 
	Dick sein ist doch gesund?\\
	Na, \textbf{das ist \underline{ja mal} eine Schlagzeile.} \glqq Übergewichtige Patienten leben länger\grqq{}.\\ Zu diesem Schluss kommt ein deutscher 		Arzt bei der diesjährigen Jahrestagung der \glqq Österreichischen Adipositas Gesellschaft\grqq{}. 	
	\newline
	\hbox{}\hfill\hbox {(BVZ08/NOV.02111 Burgenländische Volkszeitung, 26.11.2008)}
\end{exe}
Parallele Fälle findet man auch mit \textit{doch} (vgl. (\ref{279})).

\begin{exe}
	\ex\label{279} 
	STUTTGART Udo Jürgens statt Queen – \textbf{das ist \underline{doch mal} ein Tausch!} Ab Herbst 2010 soll nach Medienberichten \glqq Ich war noch 			niemals in New York\grqq{} in Stuttgart aufgeführt werden und die \glqq We Will Rock You\grqq{}-Show ersetzen [...].      	
	\newline
	\hbox{}\hfill\hbox {(HMP09/DEZ.00272 Hamburger Morgenpost, 03.12.2009)}
\end{exe}					             
Im Einklang mit Thurmairs Bedingung über zulässiges kombiniertes Auftreten halte ich die Kombination aus \textit{ja} $\plus$ \textit{doch} $\plus$ \textit{mal} (vgl. (\ref{280}) und (\ref{281})) wieder für grammatisch.

\begin{exe}
	\ex\label{280} 
	Na, \textbf{das ist \underline{ja doch mal} eine Schlagzeile.} \glqq Übergewichtige Patienten leben länger\grqq{}.
\end{exe}
\vspace{-0.4cm}
\begin{exe}
	\ex\label{281} 
	Udo Jürgens statt Queen – \textbf{das ist \underline{ja doch mal} ein Tausch!} 
\end{exe}
An dieser Stelle kann ich den Verweis auf derartige akzeptable Strukturen nur als Beobachtung stehen lassen und muss offen lassen, welchen Einfluss das \textit{mal} in diesem Kontext nimmt und welche Konsequenzen die Existenz dieser Struktur für die weitere Untersuchung der Abfolge von \textit{ja} und \textit{doch} in Kombinationen nimmt. Für die Betrachtung in Abschnitt~\ref{sec:unmarkiert} würde dies bedeuten, dass sie auch derartige exklamative Satztypen und damit deren Effekt auf den Äußerungskontext in die Analyse aufnehmen müsste.

Ähnliche Beobachtungen lassen sich dazu auch für Wunschsätze machen. Wie schon gezeigt, kann \textit{doch} im Gegensatz zu \textit{ja} in diesem Satzkontext auftreten (vgl. (\ref{282}) und (\ref{283})).

\begin{exe}
	\ex\label{282} 
	Wenn \textbf{doch} schon alles vorbei wäre!
	\hfill\hbox {\citet[140]{Hentschel1986}}
\end{exe}
\vspace{-0.5cm}
\begin{exe}
	\ex\label{283} 
	*Wenn \textbf{ja} schon alles vorbei wäre!
\end{exe}
Tritt \textit{mal} hinzu, scheint das Auftreten von \textit{ja} wiederum lizensiert (vgl. (\ref{284})).

\begin{exe}
	\ex\label{284} 
	Guten Morgen Perfektes Wetter...\\
	\textbf{Wenn das \underline{ja mal} immer so ginge!} Aber für unsere Leserinnen und Leser tun wir schließlich alles.
	\hfill\hbox{(RHZ05/OKT.11812 Rhein-Zeitung, 10.10.2005)}	
\end{exe}
Da \textit{doch} ebenfalls in dieser Umgebung stehen kann (vgl. (\ref{285})), steht auch dem gemeinsamen Auftreten von \textit{ja} und \textit{doch} nichts im Wege (vgl. (\ref{286}) und (\ref{287})).
		
\begin{exe}
	\ex\label{285} 
	Aufrechter sitzen! Mitschwingen! Schultern mitnehmen! Diese Signalwörter kennt wohl jeder aus dem Unterricht. Ach, \textbf{wenn das \underline{doch 		mal} so einfach wäre!} 
	\newline
	\hbox{}\hfill\hbox{(www.pferdemarkt.de/?page\_id=292999), (Google-Suche, eingesehen am 07.05.2013)}	
\end{exe}		

\begin{exe}
	\ex\label{286} 
	Wenn das \underline{\textbf{ja doch mal}} immer so ginge!
\end{exe}
\vspace{-0.4cm}
\begin{exe}
	\ex\label{287} 
	Ach, wenn das \underline{\textbf{ja doch mal}} so einfach wäre!
\end{exe}
Eine Erklärung muss an dieser Stelle offen bleiben.} Die Genera\-lisierung, dass \textit{ja} und \textit{doch} in Satzexklamativsätzen nicht kombiniert werden können, bliebe ebenfalls bestehen, wenn auch \textit{ja} (wie Thurmair argumentiert) in dieser Umgebung nicht auftreten könnte. 
									        
Aus dieser Verteilung ergibt sich, dass Exklamativsätze für \textit{ja} und \textit{doch} keine Umgebung darstellen, in der sie sich kombinieren können.\footnote{\label{Fn4}An dieser Stelle sei angemerkt, dass Thurmairs Unterscheidung zwischen w- und Satzexklamativsätzen auf formalen und nicht interpretatorischen Kriterien beruht. Aus semantischer Perspektive ist es eine gängige Annahme, Exklamativsätze, die ein Staunen über das Wie (d.h. zu welchem Grad/Ausmaß etwas der Fall ist) ausdrücken, zu unterscheiden von solchen, die ein Staunen über das Dass ausdrücken (d.h. über die Tatsache, dass ein Sachverhalt an sich gültig ist). w-Exklamative können prinzipiell nur Erstaunen über das Wie ausdrücken. Die angeführten Satzexklamative sind jedoch nicht (wie man vermuten könnte) der Lesart \glq staunen, dass\grq {} zuzuordnen. Auch sie drücken Erstaunen über Grad oder Ausmaß aus. Aufgrund der Distribution von \textit{ja} und \textit{doch} in Bezug auf diese zwei Interpretationen von Exklamativsätzen kann die Annahme aus \citet[161]{Hentschel1986}, dass sowohl \textit{ja} als auch \textit{doch} nicht auf das Wie, sondern nur das Dass Bezug nehmen können, nicht aufrechterhalten werden.}
	
Für \textit{dass}-Exklamativsätze \is{dass-Exklamativsatz} wie in (\ref{288}) gilt, dass \textit{doch}, aber nicht \textit{ja} auftreten kann (vgl. (\ref{289}) und (\ref{290})) (vgl. auch \citealt[109, Fn 49]{Doherty1985}, \citealt[152]{Zaefferer1988}, \citealt[135]{Meibauer1994}, \citealt[222]{Kwon2005}).

\begin{exe}
	\ex\label{288} 
		\begin{xlist}	
			\ex\label{288a} Daß ich dich hier treffen würde!
			\ex\label{288b} Daß ich das noch erleben muß!
		\end{xlist}
\end{exe}

\begin{exe}
	\ex\label{289} 
		\begin{xlist}	
			\ex\label{289a} Daß du dir das \textbf{doch} nie merken kannst!
			\hfill\hbox {\citet[56]{Thurmair1989}}
			\ex\label{289b} *Dass du dir das \textbf{ja} nie merken kannst!
		\end{xlist}
\end{exe}

\begin{exe}
	\ex\label{290} 
		\begin{xlist}	
			\ex\label{290a} Daß der mir \textbf{doch} die Vorfahrt nimmt!
			\ex\label{290b} *Dass der mir \textbf{ja} die Vorfahrt nimmt!
			\hfill\hbox {\citet[152]{Zaefferer1988}}
		\end{xlist}
\end{exe}
\citet[40-41]{Altmann1987} geht stattdessen davon aus, dass \textit{ja} und \textit{doch} in \textit{dass}-VL-Exklamativsätzen \is{dass-VL-Exklamativsatz} gar nicht (und somit auch nicht kombiniert) stehen können. In der Übersicht in \citet[59]{Karagjosova2004} findet man in dieser Domäne auch \textit{ja}. In ihrer Arbeit führt sie für diese Auftretensweise aber kein Beispiel an. Die Suche nach authentischen Belegen für \textit{ja} in dieser Umgebung im DeReKo und COW-Korpus (wobei es sich um die größten mir zugänglichen Korpora handelt), gestaltet sich erfolglos.

\textit{Dass-ja}-VL-Exklamativsätze sind anscheinend nicht aufzufinden und mit Ausnahme von \citet{Karagjosova2004} ist mir keine Arbeit bekannt, die von ihrer Existenz ausgeht. Allerdings ist an dieser Stelle anzumerken, dass auch \textit{doch}-VL-Exklamativsätze nur wenig frequent auftreten. Über das COW-Korpus lassen sich lediglich acht Treffer ausfindig machen, im DeReKo befindet sich gar kein Beleg. Und selbst da, wo Belege ausfindig zu machen sind, handelt es sich überwiegend um alte Texte (vgl. (\ref{291}) bis (\ref{293})).
 
\begin{exe}
	\ex\label{291} 
	\scriptsize
		An meiner Beschreibung dazu war ein Druckfehler, der die Säulen des Napoleon mit der troianischen in Rom verglich, das interessanteste. 					\textbf{Daß \underline{doch} immer die Setzer die witzigsten sind!}
			\newline
			\hbox{}\hfill\hbox{(http://www.hausen-im-wiesental.de/jphebel/briefe/brief\_hendel\_sch\%C3\%BCtz\_1809\_I.htm)}
			\newline
			\hbox{}\hfill\hbox{(1809: Brief von J.P. Hebel)}	
\end{exe}
\vspace{-0.65cm}
\begin{exe}
	\ex\label{292} 
	\scriptsize
		Daß mir \textbf{doch} dies alles so lebendig geblieben ist!
			\newline
			\hbox{}\hfill\hbox{(http://bfriends.brigitte.de/foren/allgemeines-forum/89544-gedicht-des-tages-17-a-18-print.html)}
			\newline
			\hbox{}\hfill\hbox{(19./erstes Drittel 20 Jhd.: Arno Holz)}		 
\end{exe}
			   
\begin{exe}
	\ex\label{293} 
	\scriptsize
		Täglich wird mir die Geschichte theurer. Ich habe diese Woche eine Geschichte des dreißigjährigen Kriegs gelesen, und mein Kopf ist mir noch ganz warm davon. \textbf{Daß \underline{doch} die Epoche des höchsten Nationen-Elends auch zugleich die glänzendste Epoche menschlicher Kraft ist!} Wie viele große Männer giengen aus dieser Nacht hervor! 		
			\newline
			\hbox{}\hfill\hbox{(http://www.kuehnle-online.de/literatur/schiller/briefe/1786/178604152.htm)} 
			\newline
			\hbox{}\hfill\hbox{(1786: Brief von Schiller)}			
\end{exe}
Die Struktur scheint folglich antiquiert, was vermutlich auch für den Satztyp \textit{dass}-VL-Exklamativsatz insgesamt angenommen werden kann. Auch \citet[140]{Hentschel1986} merkt schon an, dass \textit{dass-doch}-VL-Sätze außer in einigen wenigen konventionalisierten Verwendungen archaisch sind.

Ein weiterer Satztyp, in dem \textit{doch} auftritt und für den Thurmair annimmt, dass er zwar nicht unter die Exklamativsätze fällt, aber nah an diesen Typ heranreicht, ist in (\ref{294}) illustriert.

\begin{exe}
	\ex\label{294} 
	Stellt der \textbf{doch} glatt den Rotwein in den Kühlschrank!
			\hfill\hbox {\citet[115]{Thurmair1989}}
\end{exe}
Derartige Sätze weisen in der Regel V1-Stellung auf, V2-Stellung ist aber auch möglich (vgl. (\ref{295})).

\begin{exe}
	\ex\label{295} 
		\begin{xlist}	
			\ex\label{295a} (Da) LÄSST der sich \textbf{doch einfach} 'nen Bart wachsen!
			\ex\label{295b} BOHRT der sich \textbf{doch} in aller Öffentlichkeit in der Nase!
			\hfill\hbox {\citet[217]{Rinas2006}}
		\end{xlist}
\end{exe}
Die Partikel \textit{ja} kann in diesen Sätzen nicht auftreten wie (\ref{296}) zeigt.

\begin{exe}
	\ex\label{296} 
		\begin{xlist}	
			\ex\label{296a} *(Da) LÄSST der sich \textbf{ja einfach} 'nen Bart wachsen!
			\ex\label{296b} *BOHRT der sich \textbf{ja} in aller Öffentlichkeit in der Nase!
			\hfill\hbox {\citet[218]{Rinas2006}}
		\end{xlist}
\end{exe}
Wie von Thurmairs Schnittmengenbedingung vorhergesagt, ist aufgrund der Distribution von \textit{ja} und \textit{doch} das kombinierte Auftreten der beiden MPn ausgeschlossen (vgl. (\ref{297})).

\begin{exe}
	\ex\label{297} 
	*Bohrt der sich \textbf{ja doch} in aller Öffentlichkeit in der Nase!  
	\hfill\hbox {\citet[235]{Rinas2006}}
\end{exe}

\subsubsection{Emphatische Aussagen}
\label{sec:empha}
\citet{Thurmair1989} nimmt über die im letzten Abschnitt beschriebenen Satz- und w-Exklamativsätze hinaus keine weiteren Typen von Exklamativsätzen an. Doch verweist sie auf Sätze der Art \is{emphatische Aussage} in (\ref{298}) bis (\ref{302}).

\begin{exe}
	\ex\label{298} 
	Das Kind kommt vom Spielen heim. Die Mutter: \glqq Du blutest \textbf{ja}!\grqq{}
\end{exe}

\begin{exe}
	\ex\label{299} 
	Du hast \textbf{ja} grüne Augen!
\end{exe}	

\begin{exe}
	\ex\label{300} 
	Nelli: Feiert ihr vor Silvester noch ma?\\
	Bea: Ja!\\
	Nelli: Ach, du has \textbf{ja} Geburtstag!
\end{exe}	

\begin{exe}
	\ex\label{301} 
	Anna: Wir ham nach fünfeinhalb Jahren noch keins [Kind].\\
	Lisa: Mensch, ist das schon solange her?\\
	Anna: Ja!\\
	Lisa: Poh! Is \textbf{ja} Wahnsinn!
\end{exe}	
	
\begin{exe}
	\ex\label{302} 
	Da kommt \textbf{ja} der Heinz!
	\hfill\hbox {\citet[107-108/215]{Thurmair1989}}
\end{exe}
In den Klassifikationen mancher Autoren (vgl. z.B. \citealt[13]{Weydt1983b}, \citealt[157]{Hentschel1986}, \citealt[313]{Foolen1989}, \citealt[166-167]{Helbig1990}, \citealt[197]{Karagjosova2004}, \citealt[160-168]{Rinas2006}) zählen auch diese zu den Exklamativsätzen, die Staunen über das Dass ausdrücken.\footnote{Hier liegt ein weiterer konkreter Fall der anfänglich angeführten Problematik in der Angabe der Distribution von MPn und ihrer Kombinationen vor. Wenn die hier genannten Autoren Sätze der Art in (\ref{298}) bis (\ref{302}) als Exklamativsätze einstufen, ergeben sich für das Auftreten von \textit{ja} und \textit{doch} natürlich andere Klassifikationen. An dieser Stelle ist folglich Vorsicht geboten, wenn es zu entscheiden gilt, ob sich Ansätze tatsächlich widersprechen. Gerade im Bereich der Exklamativsätze gehen die Klassifikationen deutlich auseinander.} \citet[107-108]{Thurmair1989} zufolge (genauso auch \citealt[77-80]{Doherty1985}, \citealt[37]{Kwon2005}) liegen hier keine Exklamativsätze vor, sondern Aussagesätze – wenngleich sich die Funktion in Richtung Ausruf bewegt. Das gilt für alle Typen in (\ref{298}) bis (\ref{302}), der Typ in (\ref{298}) bis (\ref{300}) ist aufgrund der Unmöglichkeit der Kombination von \textit{ja} und \textit{doch} in dieser Umgebung für die weitere Untersuchung allerdings weniger interessant. Thurmair begründet ihre Entscheidung rein auf der Basis grammatischer Merkmale, d.h. Merkmale, die sie für Satzex\-klamativsätze als konstitutiv ansetzt, treffen nicht zu. Da sich die Formtypen, die Teil des Form-Funktionspaares des jeweiligen Satzmodus sind, ebenfalls auf der Basis bestimmter grammatischer Eigenschaften konstituieren, ist diese Argumentation plausibel und kohärent. So weisen Strukturen wie in (\ref{298}) bis (\ref{302}) keinen Exklamativakzent \is{Exklamativakzent} auf (wie etwa (\ref{303})).

\begin{exe}
	\ex\label{303} 
		\begin{xlist}	
			\ex\label{303a} Hast DU \textbf{aber} große Füße!
			\ex\label{303b} Hat DIE \textbf{vielleicht} einen kurzen Rock an!
			\ex\label{303c} Was hast DU \textbf{bloß} für komische Ansichten!
		\end{xlist}
\end{exe}	
V1-und V2-Stellung lässt sich nicht variieren (vgl. (\ref{304}) vs. (\ref{305})).

\begin{exe}
	\ex\label{304} 
		\begin{xlist}	
			\ex\label{304a} *Blutest du \textbf{ja}!	
			\ex\label{304b} *Hast du \textbf{ja} grüne Augen!
			\ex\label{304c} *Hast du \textbf{ja} Geburtstag!
			\ex\label{304d} *Ist das \textbf{ja} Wahnsinn!
			\ex\label{304e} *Kommt da \textbf{ja} der Heinz!
		\end{xlist}
\end{exe}
\begin{exe}
	\ex\label{305} 
		\begin{xlist}	
			\ex\label{305a} Hast DU eine schöne Wohnung!	
			\ex\label{305b} DU hast \textbf{(vielleicht)} eine schöne Wohnung!
		\end{xlist}
\end{exe}
Da sie für ihre beiden Typen von Exklamativsätzen \is{Exklamativsatz} ansetzt, dass sie die Sprecher\-einstellung kodieren, dass der Sprecher erstaunt ist, in welchem Maße etwas der Fall ist, handelt es sich auch aus dieser Perspektive nicht um Exklamativsätze. Sofern Staunen involviert ist, handelt es sich um das Staunen darüber, dass der beschriebene Sachverhalt zutrifft (aber vgl. Fußnote~\ref{Fn4}). Zuletzt kann \textit{ja} in den typi\-schen Exklamativsätzen (\textit{was für}/\textit{welch}/\textit{wie}-V2/VL) nicht auftreten (s.o. und (\ref{306})), was dann entsprechend merkwürdig erschiene, wenn es sich bei (\ref{298}) bis (\ref{302}) tatsächlich um Exklamativsätze handelte.

\begin{exe}
	\ex\label{306} 
		\begin{xlist}	
			\ex\label{306a} Was für eine Wohltat ist (*\textbf{ja}) dieses Buch!
			\ex\label{306b} Welch eine Simulation war (*\textbf{ja}) diese Wirklichkeit!	
			\hfill\hbox {\citet[37]{Kwon2005}}
		\end{xlist}
\end{exe}
Weitere Stützung für Thurmairs Entschluss, (\ref{298}) bis (\ref{302}) nicht zu den Exklamativsätzen zu zählen, liefert die Tatsache, dass das Kriterium des Staunens über das Dass zudem nicht einmal für alle diese Sätze gilt. Bei Äußerungen der Art in (\ref{301}) geht es nicht um den Ausdruck von Staunen, dass ein Sachverhalt besteht, sondern ein Sachverhalt wird bewertet. Der Sachverhalt, auf den sich die Be\-wertung bezieht, wird in dem Satz selbst nicht einmal ausgedrückt.
 
Für Thurmair sind die Sätze in (\ref{298}) bis (\ref{302}) \is{emphatische Aussage} \textit{emphatische Aussagen}. Z.T. finden sich hier auch zu (\ref{298}) bis (\ref{302}) parallele Fälle mit \textit{doch}. An dieser Stelle ist vor dem Hintergrund von Thurmairs satzmodaler Schnittmengenbedingung anzuführen, dass \textit{ja} und \textit{doch} in den Typen von emphatischen Aussagen, in denen sie prinzi\-piell in Isolation auftreten können, auch in Kombination zulässig sind. Dies trifft beispielsweise auf Bewertungen wie in (\ref{301}) zu. In derartigen Kontexten, in denen der Sprecher eine(n) erstaunliche(n) Vorgängerhandlung/Sachverhalt kommentiert/bewertet, kann \textit{doch} problemlos stehen (vgl. (\ref{307}) bis (\ref{309})).

\begin{exe}
	\ex\label{307} 
	Bin ich ja froh, dass das Kind nich mitgefahren war, stell dir das mal vor. Is doch Wahnsinn! (...) Das ist \textbf{doch} Wahnsinn!
\end{exe}
\vspace{-0.65cm}	
\begin{exe}
	\ex\label{308} 
	Das ist \textbf{doch} das Letzte!
\end{exe}	
\vspace{-0.65cm}
\begin{exe}
	\ex\label{309} 
	Jetzt nörgel nicht so rum! Das ist \textbf{doch} wirklich traumhaft hier!
	\newline
	\hbox{}\hfill\hbox{\citet[114-115]{Thurmair1989}}	
\end{exe}
Im Einklang mit Thurmairs Schnittmengenbedingung können sich \textit{ja} und \textit{doch} hier auch kombinieren. \citet[235]{Rinas2006} hält (\ref{310}) für fraglich. M.E. sind derartige Belege völlig akzeptabel.

\begin{exe}
	\ex\label{310} 
	?Das ist \textbf{ja doch} die Höhe!	
\end{exe}
Äußerungen der Art in (\ref{310}) lassen sich dazu recht einfach sowohl im heutigen Deutsch (vgl. (\ref{311}) und (\ref{312})) als auch in älteren Quellen finden (vgl. (\ref{313})).
\begin{exe}
	\ex\label{311} 
	\scriptsize
	\textbf{Oh man das ist \underline{ja doch} der Hammer.} Da sieht man mal wieder das Arzt nicht gleich Arzt ist. 
	\newline
	\hbox{}\hfill\hbox{(http://www.meerschwein-community.de/archive/index.php?thread-2607.html)}	
	\newline
	\hbox{}\hfill\hbox{(Google-Suche, eingesehen am 31.3.2013)}	
\end{exe}
	      
\begin{exe}
	\ex\label{312} 
	\scriptsize
	\textbf{Das ist \underline{ja doch} die Höhe}, meint eine großformatige, bunte deutsche Zeitung: \glqq Wie ein leichtes Mädchen\grqq{} habe sich Sarah 	Ferguson alias Fergie \glqq verkauft\grqq{}. 
	\hfill\hbox{(K96/NOV.24120 Kleine Zeitung, 03.11.1996)}	
\end{exe}
	
\begin{exe}
	\ex\label{313} 
	\scriptsize
	Es wurde ihm unbehaglich heiß.\\
	- \textbf{Aber das ist \underline{ja doch} niederträchtig!} Das ist ja Diebstahl! Pfui Teufel! ...
	- Und, wenn sie's beim Abrechnen merken? ...
	\newline
	\hbox{}\hfill\hbox{(Otto Julius Bierbaum: Stilpe – Kapitel 11, 1897) (eingesehen über: http://gutenberg.spiegel.de/)}	
\end{exe}								                       
Inwieweit sich die Funktion und Verwendung derartiger emphatischer Aussagen von denen von typischeren Aussagesätzen (vgl. Abschnitt~\ref{sec:synschnitt}) unterscheidet, wird in Abschnitt~\ref{sec:markiert} diskutiert. An dieser Stelle ist die Annahme entscheidend, dass diese Satztypen nicht zu den Exklamativsätzen zu rechnen sind, so dass an der Generalisierung festgehalten werden kann, dass sich \textit{ja} und \textit{doch} im Formtyp des Aussagesatzes kombinieren lassen.

\subsection{Semantische/pragmatische Schnittmengenbedingung}
Neben der im letzten Abschnitt erläuterten syntaktisch-distributionellen Schnitt\-mengenbedingung gibt es weitere Auftretensbeschränkungen, die Autoren (vgl. z.B. \citealt[218, 222-223]{Dahl1988}, \citealt[25-31]{Thurmair1991}) zu einer Verschärfung der Schnittmengenbedingung veranlasst haben. Es handelt sich hierbei um die Beobachtung, dass die Kompatibilität auch auf Ebene der Interpretation der Einzelpartikeln gegeben sein muss, damit die MP-Kombination zulässig ist. \citet[26-27]{Thurmair1991} führt beispielsweise die folgenden Sätze an.	 

\begin{exe}
	\ex\label{314} 
		\begin{xlist}	
			\ex\label{314a} Ist das Kleid \textbf{auch} durchsichtig?
			\ex\label{314b} Ist das Kleid \textbf{etwa} durchsichtig?
		\end{xlist}
\end{exe}

\begin{exe}
	\ex\label{315} 
		\begin{xlist}	
			\ex\label{315a} *Ist das Kleid \textbf{etwa auch} durchsichtig?
			\hfill\hbox {\citet[27]{Thurmair1991}}
			\ex\label{315b} *Ist das Kleid \textbf{auch etwa} durchsichtig?
		\end{xlist}
\end{exe}
Obwohl \textit{auch} und \textit{etwa} jeweils für sich in E-Fragesätzen auftreten können (vgl. (\ref{314})), ist ihre Kombination in diesem Satzmodus trotzdem ausgeschlossen (vgl. (\ref{315})). Verhältnisse wie in (\ref{314}) und (\ref{315}) legen deshalb nahe, dass eine geteilte satzmodale Umgebung als Kriterium nicht ausreichend ist. Thurmairs Anwendung der semantischen/pragmatischen Schnittmengenbedingung basiert auf den unterschiedlichen Antworterwartungen, die E-Fragen mit \textit{auch} bzw. \textit{etwa} mit sich bringen. Eine Frage wie (\ref{314a}) weist Thurmair zufolge eine positive Erwartung auf, d.h. sie legt die Reaktion der Zustimmung (bevorzugte Antwort  \textit{Ja.}) nahe. Die Frage in (\ref{314b}) hingegen bringt eine negative Erwartung mit sich, d.h. die bevorzugte Antwort ist \textit{Nein}. Die Unverträglichkeit der beiden MPn in E-Fragesät\-zen führt die Autorin nun darauf zurück, dass es keine sinnvolle Frage geben kann, die gleichzeitig eine positive und negative Antworterwartung aufweist. Auf dieser Ebene sind die Bedeutungen der beiden MPn inkompatibel und produzieren eine leere Schnittmenge.			        
								               
Auf ähnliche Art, wie das Kriterium der Kompatibilität der an einer Kombination beteiligten MPn in (\ref{315a}) und (\ref{315b}) über die rein satzmodale Verträglichkeit hinausgeht, lassen sich auch für das gemeinsame Auftreten von \textit{ja} und \textit{doch} restringiertere Auftretenskontexte formulieren als die (strukturelle) Domäne des Aussagesatzes. Im Folgenden wird dies aufgezeigt anhand der im letzten Abschnitt bereits angesprochenen \is{emphatische Aussage} emphatischen Aussagen. Diese Aussagen unterscheiden sich von typischeren Aussagen darin, dass sie auf funktionaler Ebene Eigenschaften mit Exklamativsätzen zu teilen scheinen, weil sie eine gewisse emotionale Färbung aufweisen, d.h. der Sprecher zu einem gewissen Grad Überraschung/Erstaunen über einen unmittelbar wahrgenommenen Sachverhalt zum Ausdruck bringt.
			   					         
In der Literatur werden kontroverse Annahmen zu der prinzipiellen Frage gemacht, ob \textit{doch} in Äußerungen, die sich durch eine gewisse Spontanität, den Bezug auf direkt Wahrgenommenes und den Bedeutungsas\-pekt der Überraschung charakterisieren, auftreten kann (vgl. für diese Annahme z.B. \citealt[26]{Weydt1969} (\textit{Kommst du \textbf{doch} spät!}), \citealt[37]{Helbig1977} (\textit{Kommst du \textbf{doch} unpünktlich!}, \textit{War das \textbf{doch} eine Überraschung!}), \citealt[116]{Helbig1990} (\textit{Du SCHNARCHST \textbf{doch}!}, \textit{Das war \textbf{doch} unsere ehemalige Studentin!})\footnote{Bei einigen der zitierten Strukturen handelt es sich um echte V1-Exklamativsätze, wenn man an den Kriterien zur Klassifikation von emphatischen Aussagen (s.o.) festhält. Dennoch eignen sich auch diese Daten dazu, zu diskutieren, ob man es bei der Interaktion von Staunen und dem MP-Beitrag von \textit{doch} mit einer semantischen/pragmatischen Inkompatibilität zu tun hat, die das kombinierte Auftreten von \textit{ja} und \textit{doch} in den emphatischen Aussagen deshalb unterbindet.} vs. dagegen \citealt[141]{Hentschel1986} (Beispiele von \citealt[26]{Weydt1969}, \citealt[37]{Helbig1977} [s.o.], \citealt[216]{Rinas2006} (*\textit{Du ISST \textbf{doch} gar nichts!})). \citet[141]{Hentschel1986} zufolge handelt es sich hierbei \glqq um einen veralteten Gebrauch von \textit{doch} [...], der zwar möglich, in der modernen Umgangssprache aber nicht mehr zu beobachten ist\grqq{}. Ob \textit{doch} in dieser Umgebung tatsächlich kate\-gorisch ausgeschlossen werden kann oder den Intuitionen aus \citet{Weydt1969} und \citet{Helbig1977, Helbig1990} zuzustimmen ist, bleibt noch zu klären (s.u.). Die Einschätzung der genannten Autoren, die sich gegen die Verträglichkeit von \textit{doch} und dem Aspekt des Erstaunens äußern, läuft konform mit der Annahme aus \citet[193]{Lindner1991}, dass \textit{doch} (anders als \textit{ja}) nicht auftreten kann, wenn die Äußerung das Bedeutungsmoment der Überraschung trägt. Die Beobachtung Lindners basiert auf der Adäquatheit der \textit{ja}-Äußerung und der Inadäquatheit der \textit{doch}-Äußerung im Kontext in (\ref{316}).  
	         
\begin{exe}
	\ex\label{316} 
	Im Kaufhaus: Ein kleines Mädchen betrachtet sich mit ihrem neuen Kleid im Spiegel. Ihre Mutter beobachtet sie.\\
	Mutter (überrascht):\\
	Das ist \textbf{ja}/*\textbf{doch}/*\textbf{ja doc}h ein hübsches Kleid. 
	\hfill\hbox {nach \citet[193]{Lindner1991}}
\end{exe}							
Angenommen, \textit{doch} kann in Kontexten, die die Komponente der Überraschung beinhalten, (anders als \textit{ja}) tatsächlich nicht auftreten, hätte man es hier wo\-möglich mit einem Fall zu tun, bei dem die beiden MPn nicht syntaktisch, sondern (ähnlich wie in (\ref{315a}) und (\ref{315b})) semantisch/pragmatisch inkompatibel sind, da nicht beide mit dem Bedeutungsas\-pekt von Überraschung verträglich sind. Im Einvernehmen mit dieser auf die Kompatibilität von Interpretation Bezug nehmenden (leeren) Schnittmengenbildung lassen sich \textit{ja} und \textit{doch} in dieser Umgebung auch nicht kombinieren.

Die Einschätzung Lindners hinsichtlich der Adäquatheit der \textit{ja}-, \textit{doch}- bzw. \textit{ja doch}-Äußerungen in (\ref{316}) ist zu teilen. Dennoch stellt sich die Frage, ob hier tatsächlich eine prinzipielle Unverträglichkeit der MP \textit{doch} mit dem Bedeutungsaspekt der Überraschung vorliegt. Dies scheint schon unplausibel vor dem Hintergrund, dass \textit{doch} in w-Exklamativsätzen (wie gesehen in Abschnitt~\ref{sec:exkl}) im Gegensatz zu \textit{ja} auftreten kann. Betrachtet man verschiedene Typen von emphatischen Aussagen (vgl. (\ref{298}) bis (\ref{302})), für die anzunehmen ist, dass ein Überraschungsmoment/Erstaunen/Spontanität beteiligt ist (s.o.), spricht die Datenlage für eine subtilere Verteilung von \textit{doch} in derartigen \glq Staunens\grq {}kontexten.  

In emphatischen Aussagen der Art in (\ref{317}) scheint \textit{doch} tatsächlich nicht auftre\-ten zu können. Für \textit{ja} lassen sich hier leicht Belege finden (vgl. z.B. (\ref{318}) und (\ref{319}) für Strukturen der Art \textit{Der hat \textbf{ja} x!}). 

\begin{exe}
	\ex\label{317}
	Du hast \textbf{ja} grüne Augen! 
	\hfill\hbox {\citet[108]{Thurmair1989}}
\end{exe}

\begin{exe}
	\ex\label{318}
	\scriptsize 
	Das war kein Gewitter! Das war Müller, und seine hellblauen Augen funkelten wie Laserblitze.
	\glqq Auweia, wie sieht der denn aus? \textbf{Der hat \underline{ja} weiße Wimpern!}\grqq{}, stellte Alfi fest.
	\newline
	\hbox{}\hfill\hbox{(RZ08/AUG.07411 Braunschweiger Zeitung, 16.08.2008)}	
\end{exe}

\begin{exe}
	\ex\label{319}
	\scriptsize 
	Bei der Kanzel müssen die Eltern ihre Kerzen hochhalten, damit alle die Mutter Gottes auf der Mondsichel sehen können. Beim Engel der Hoffnung fällt 		einem Kind auf: \glqq \textbf{Der hat \underline{ja} einen Anker in der Hand!}\grqq{}      
	\hfill\hbox{(BRZ09/APR.07349 Braunschweiger Zeitung, 18.04.2009)}	
\end{exe}							                                      							      
Für \textit{doch} liefern parallele Suchanfragen zwar durchaus Treffer (vgl. z.B. (\ref{320})). Allerdings deutet man in der Tat keine der zu findenden \textit{doch}-Äußerungen als spontane Reaktion des Erstaunens über das direkt Wahrgenommene.

\begin{exe}
	\ex\label{320}
	\scriptsize 
	Seinen großen Auftritt bei dieser WM hatte Thomas Hitzlsperger beim Achtelfinale gegen Schweden. Hä? Hitzlsperger?! \textbf{Der hat \underline{doch} 		gar nicht gespielt!}     
	\newline
	\hbox{}\hfill\hbox{(HMP06/JUN.02920 Hamburger Morgenpost, 27.06.2006)}	
\end{exe}						   
Die gleiche Beobachtung trifft auch auf strukturell anders, funktional aber sehr ähnliche Äußerungen zu. Strukturen des Musters \textit{Du bist ja x!} finden sich recht einfach (vgl. (\ref{321}) und (\ref{322})). Mit der gleichen Suchanfrage lassen sich jedoch keine vergleichbaren \textit{doch}-Äußerungen finden.

\begin{exe}
	\ex\label{321}
	\scriptsize 
	 \glqq Ist alles in Ordnung mit dir, Missy? \textbf{Du bist \underline{ja} ganz weiß im Gesicht!}\grqq{}    
	\newline
	\hbox{}\hfill\hbox{(BRZ08/JUL.06127 Braunschweiger Zeitung, 11.07.2008)}	
\end{exe}
\vspace{-0.65cm}
\begin{exe}
	\ex\label{322}
	\scriptsize 
	 \glqq \textbf{Du bist \underline{ja} völlig außer Atem!} Was ist passiert?\grqq{}    
	\newline
	\hbox{}\hfill\hbox{(DIV/APR.00001 Planert, Angela: Rubor Seleno. – Föritz, 2005 $[$S. 46$]$)}	
\end{exe}	 
Es ist nicht möglich, \textit{ja} in den belegten Beispielen durch \textit{doch} zu ersetzen unter Beibehaltung der Interpretation der Äußerung im Kontext (vgl. (\ref{323}) bis (\ref{326})). Und auch die Kombination aus \textit{ja} und \textit{doch} ist in diesen Kontexten auszuschließen.

\begin{exe}
	\ex\label{323}
	\glqq Auweia, wie sieht der denn aus? \#\textbf{Der hat \underline{doch}/\underline{ja doch} weiße Wimpern!}\grqq{}, stellte Alfi fest.
\end{exe}	

\begin{exe}
	\ex\label{324}
	Beim Engel der Hoffnung fällt einem Kind auf: \glqq \#\textbf{Der hat \underline{doch}/\underline{ja doch} einen Anker in der Hand!}\grqq{}
\end{exe}	

\begin{exe}
	\ex\label{325}
	\glqq Ist alles in Ordnung mit dir, Missy? \#\textbf{Du bist \underline{doch}/\underline{ja doch} ganz weiß im Gesicht!}\grqq{}
\end{exe}

\begin{exe}
	\ex\label{326}
	\glqq \#\textbf{Du bist \underline{doch}/\underline{ja doch} völlig außer Atem!} Was ist passiert?\grqq{}
\end{exe}
Hinsichtlich dieses Typus von emphatischer Aussage \is{emphatische Aussage} ist den Annahmen der oben angeführten Autoren hinsichtlich der Verträglichkeit von \textit{doch} und dem Aspekt der Überraschung somit zuzustimmen. 

Für den Typ der emphatischen Aussage in (\ref{327}) finden sich allerdings durchaus auch \textit{doch}-Belege.
	
\begin{exe}
	\ex\label{327} 
	Da kommt \textbf{ja} der Heinz!
	\hfill\hbox {\citet[215]{Thurmair1989}}
\end{exe}		            		
(\ref{328}) und (\ref{329}) sind Beispiele aus der Literatur.

\begin{exe}
	\ex\label{328} 
	Das ist \textbf{doch} mein ehemaliger Klassenlehrer!	
	\hfill\hbox {\citet[196]{Rinas2006}}
\end{exe}
\vspace{-0.65cm}
\begin{exe}
	\ex\label{329} 
	Das sind \textbf{doch} Klaus und Maria!	
	\hfill\hbox {\citet[86]{Dahl1988}}
\end{exe}											         
Funktional entsprechen die Sätze in (\ref{328}) und (\ref{329}) den emphatischen Aussagen aus (\ref{317}) bis (\ref{319}). Äußert ein Sprecher einen Satz der Art in (\ref{328}) oder (\ref{329}), bringt er in der Situation unmittelbar Erstaunen/Überraschung zum Ausdruck, die ge\-nannten Personen zu sehen/zu erkennen, genauso wie er mit Äußerungen wie in (\ref{317}) bis (\ref{319}) und (\ref{321}) und (\ref{322}) mit Erstaunen unmittelbar wahrnimmt und spontan die Erkenntnis/Beobachtung ausdrückt, dass die Person außer Atem ist, weiß im Gesicht ist, weiße Wimpern hat etc. Für diese Art der emphatischen Aussage finden sich auch authentische Belege (vgl. z.B. (\ref{330}) und (\ref{331})), die weder auf das Muster \textit{Das ist doch + Eigenname/definite Kennzeichnung!} noch auf die Struktur \textit{Das ist doch x!} beschränkt sind, wie (\ref{332}) und (\ref{333}) illustrieren.
	
\begin{exe}
	\ex\label{330}
	\scriptsize 
	Da kommt jemand hinter dem nächsten Pfeiler hervor und geht geradewegs auf mich zu. Es ist eine kleine rundliche Frau mit einem auffallend rosigen 			Gesicht. Sie trägt eine dunkelrote Wintermütze. \textbf{Das ist \underline{doch} Fräulein M.!}     
	\hfill\hbox{(NUZ05/FEB.02073 Nürnberger Zeitung, 18.02.2005)}	
\end{exe}	
	
\begin{exe}
	\ex\label{331}
	\scriptsize 
	Der Reisegruppe aus Thüringen ist sehr schnell klar, daß da politische Prominenz anrollt: Ein Troß Männer in dunklen Anzügen, umrandet von sportlichen 	Sicherheitsleuten [...].  \glqq \textbf{Das ist \underline{doch} der von der FDP!}\grqq{}, schallt es aus der Thüringer Touristengruppe, als der 			hessische Ministerpräsident und seine Begleitung vorbeihetzen.	     
	\hfill\hbox{(R99/JUL.54611 Frankfurter Rundschau, 09.07.1999)}	
\end{exe}				
							
\begin{exe}
	\ex\label{332}
	\scriptsize 
	Es ist die zweite Gemeinsamkeit, die Auma am Bruder entdeckt, den sie erst mit Ende 20 kennenlernt, weil er bei der Mutter auf Hawaii aufwächst, sie beim Vater in Kenia. Die erste flatterte mit der Post aus Amerika ins Haus. \glqq \textbf{Das ist \underline{doch} Vaters Schrift!}\grqq{}, durchfuhr es sie, als sie 1984 Baracks ersten Brief in der Hand hielt.        
	\hfill\hbox{(HMP10/SEP.02590 Hamburger Morgenpost, 26.09.2010)}	
\end{exe}
		
\begin{exe}
	\ex\label{333}
	\scriptsize 
	Mensch, \textbf{da geht \underline{doch} der Johnny Depp!} Kreisch, das ist er! – ruhig Blut, er ist es nicht.    
	\newline
	\hbox{}\hfill\hbox{(RHZ09/DEZ.02188 Rhein-Zeitung, 03.12.2009)}	
\end{exe}	                                                                
Da das \textit{doch} in (\ref{330}) bis (\ref{333}) jeweils problemlos durch \textit{ja} ersetzt werden kann, ist auch die Kombination wieder möglich. Beide MPn scheinen isoliert in diesem \glq Überraschungs\grq {}kontext auftreten zu können, so dass auch der Kombination nichts im Wege steht (vgl. (\ref{334}) bis (\ref{337}) sowie den Beleg in (\ref{338})).

\begin{exe}
	\ex\label{334}
	\scriptsize 
	Da kommt jemand hinter dem nächsten Pfeiler hervor und geht geradewegs auf mich zu. [...] \textbf{Das ist \underline{ja doch} Fräulein M.!}	
\end{exe}	

\begin{exe}
	\ex\label{335}
	\scriptsize 
	Der Reisegruppe aus Thüringen ist sehr schnell klar, daß da politische Prominenz anrollt: [...]  \glqq \textbf{Das ist \underline{ja doch} der von der FDP!}			\grqq{}, schallt es aus der Thüringer Touristengruppe [...].
\end{exe}

\begin{exe}
	\ex\label{336}
	\scriptsize 
	Es ist die zweite Gemeinsamkeit, die Auma am Bruder entdeckt [...]. Die erste flatterte mit der Post aus Amerika ins Haus. \glqq \textbf{Das ist 			\underline{ja doch} Vaters Schrift!}\grqq{}, durchfuhr es sie, als sie 1984 Baracks ersten Brief in der Hand hielt.
\end{exe}
	
\begin{exe}
	\ex\label{337}
	\scriptsize 
	Mensch, \textbf{da geht \underline{ja doch} der Johnny Depp!} Kreisch, das ist er! – ruhig Blut, er ist es nicht.
\end{exe}

\begin{exe}
	\ex\label{338}
	\scriptsize 
	Re: Kirmes in Wissel 2012\\
	\glqq Antwort \# 2 am: 24. Juli2012, 19:22:11\grqq{}\\
	\textbf{Das ist \underline{ja doch} der Star Light Petter, der Autoscooter.} Der sieht auch immer wieder klasse aus. Danke für die Bilder!	
	\hfill\hbox{(http://rummelforum.de/index.php?topic=8294.0)}
	\newline	
	\hbox{}\hfill\hbox{(Google-Suche, eingesehen am 30.05.2013)}	
\end{exe}									       
Der Beitrag in (\ref{338}) folgt unmittelbar als Reaktion auf eine Reihe von Fotos der Kirmes mit der einzigen Information desjenigen, der die Bilder eingestellt hat, dass und wann er diese Kirmes besucht hat, zzgl. einiger weniger Angaben zum Ort. D.h. andere Aspekte der Kirmes (insbesondere den Autoscooter betreffend) werden nicht diskutiert. Es lässt sich somit annehmen, dass die \textit{ja doch}-Äußerung in (\ref{338}) ebenso wie die Sätze in (\ref{334}) bis (\ref{337}) direkt Bezug nimmt auf das unmittelbar zuvor gesehene Bild und der Sprecher das Sehen des bestimmten Autoscooter spontan und unter Beteiligung eines gewissen Erstauntseins über das Wahrgenommene zum Ausdruck bringt.

Ein weiterer (möglicher) Fall, bei dem man es im Kontext der Kombination von \textit{ja} und \textit{doch} aufgrund der semantischen/pragmatischen Verteilung der beiden Partikeln mit einer leeren Schnittmenge zu tun hat, findet sich in (\ref{339}).

\begin{exe}
	\ex\label{339} 
		\begin{xlist}	
			\ex\label{339a} Da IST er \textbf{ja}!
			\ex\label{339b} \glqq Da bist du \textbf{ja}, Walter, ich dachte schon, du bist zu deinem Campari verschwunden!\grqq{}  	         
			\hfill\hbox {Frisch 1957: 136, zitiert nach \citet[167]{Rinas2006}}
		\end{xlist}
\end{exe}										      	     
Die Verwendung von \textit{ja} ist in diesem Kontext obligatorisch (vgl. (\ref{340})). 

\begin{exe}
	\ex\label{340} 
		\begin{xlist}	
			\ex\label{340a} Da IST er \textbf{ja}!/*Da IST er.
			\ex\label{340b} Weißt du, wo mein Stift ist? Ich kann ihn nirgendwo finden. – Ach, da IST er \textbf{ja}!/*Ach, da IST er!	         
			\hfill\hbox {\citet[167]{Rinas2006}}
		\end{xlist}
\end{exe}
Die MP kann \citet[168]{Rinas2006} zufolge nur ausgelassen werden, wenn der Haupt\-akzent verschoben wird (vgl. (\ref{341})). 

\begin{exe}
	\ex\label{341} 
	Weißt du, wo mein Stift ist? Ich kann ihn nirgendwo finden. – Ach, DA ist er!
	\hfill\hbox{\citet[168]{Rinas2006}}	
\end{exe}
Nach \citet[217]{Rinas2006} kann \textit{doch} in dieser Umgebung nicht auftreten, d.h. in den angeführten Verwendungen lässt sich \textit{ja} nicht durch \textit{doch} austauschen. Interessanterweise ist der interpretatorische Faktor, den Rinas als entscheidend ausmacht, wiederum das Staunen. Auch in Abgrenzung zum akzeptablen (\ref{342}) kann ihm zufolge \textit{doch} dann nicht auftreten, wenn der Satz allein Staunen ausdrücken soll. 

\begin{exe}
	\ex\label{342} 	
	Da IST er \textbf{ja}/*\textbf{doch}!	
	\hfill\hbox {\citet[217]{Rinas2006}}
\end{exe}											     
In Isolation oder mit wenig Kontext erscheint mir ein Urteil über eine \textit{da} $\plus$ Kopula $\plus$ Subjekt $\plus$ \textit{doch}-Äußerung schwierig, Recherchen bestätigen Rinas Annahme jedoch. Es lassen sich Belege für \textit{ja}-Äußerungen der Art in (\ref{340}) finden wie z.B. in (\ref{343}).

\begin{exe}
	\ex\label{343} 
	\scriptsize
	Ein elektronischer Poltergeist\\
	Es war genau 23 Uhr, als ein leises Piepsen mich aufschrecken ließ. Es kam vom Telefontisch her, steigerte sich in Lautstärke und Frequenz und 				verstummte erst nach einer Minute. Ich wunderte mich kurz und vergaß es über dem extrakniffligen Kreuzworträtsel. Bis zum nächsten Abend um 23 Uhr. Da 	piepste es wieder. Ich hob den Telefonhörer ab, drückte sämtliche Knöpfe auch an Feststation und Anrufbeantworter. [...]
	Vielleicht mal die Akkus tauschen, meinte er. Oder den Gerätehersteller fragen. Das mit den Akkus erwies sich als glatte Fehlinvestition. Es piepste 		weiter. Was Kundenservice, Konstruktionsabteilung, Verkaufsleitung und PR-Zentrale von Siemens glatt in Abrede stellten. – Aber es piepste. So langsam 	wollte ich schon an Poltergeister glauben. Bis gestern Abend, 23 Uhr. Da betrat meine Frau das Wohnzimmer, hörte das Piepsen, rollte das 					Telefontischchen beiseite, bückte sich und präsentierte ihn, mit einem strahlenden \glqq \textbf{Da ist er \underline{ja}!}\grqq{}: den seit Tagen 			vermissten Mini-Funkwecker... 
	\hfill\hbox{(RHZ01/OKT.07303 Rhein-Zeitung, 10.10.2001)}	
\end{exe}					
Die MP \textit{ja} scheint mir hier in der Tat nicht durch \textit{doch} ersetzt werden zu können. Im Einklang mit der Schnittmengenbedingung ist eine \textit{ja doch}-Äußerung in allen Kontexten, in denen \textit{doch} alleine nicht auftreten kann, ebenfalls nicht akzeptabel. Dies lässt sich durch Modifikation von (\ref{343}) nachweisen (vgl. (\ref{344})).

\begin{exe}
	\ex\label{344} 
	\scriptsize
	Da betrat meine Frau das Wohnzimmer, hörte das Piepsen, rollte das Telefontischchen beiseite, bückte sich und präsentierte ihn, mit einem strahlenden  	\glqq *\textbf{Da ist er \underline{ja doch}!}\grqq{}: den seit Tagen vermissten Mini-Funkwecker... 
\end{exe}	
Da auch hier der Bedeutungsaspekt \glq Staunen\grq {} als relevant ausgemacht wird, stellt sich die Frage, was die emphatischen Aussagen des Typs in (\ref{345}) und (\ref{346}) von den anderen angeführten Kontexten unterscheidet und was die Lizensierung des \textit{doch} – trotz der Bedeutungsaspekte von Erstaunen/Überraschung/spontaner Reaktion auf direkt Wahrgenommenes – lizensiert. Ich habe derzeit keine Antwort auf diese Frage.

\begin{exe}
	\ex\label{345} 
	Das ist \textbf{doch} mein ehemaliger Klassenlehrer!
	\hfill\hbox {\citet[196]{Rinas2006}}
\end{exe}
\vspace{-0.65cm}
\begin{exe}
	\ex\label{346} 
	Das sind \textbf{doch} Klaus und Maria!	
	\hfill\hbox {\citet[86]{Dahl1988}}
\end{exe}											         
Die in diesem Abschnitt betrachteten Daten zeigen deutlich, dass auch für die MPn \textit{ja} und \textit{doch} gilt, dass ihre prinzipielle Kombinierbarkeit nicht ausschließlich durch eine satzmodale Kompatibilität gesteuert wird, sondern die Forderung nach Kompatibilität auf interpretatorischer Ebene verschärft sein kann. Diese (zweite) semantische/pragmatische Schnittmengenbedingung greift, wenn die (erste) syn\-taktisch-distributionelle Voraussetzung bereits erfüllt ist. Die Verhältnisse bei der Kombination von \textit{ja} und \textit{doch}, wie hier beschrieben, entsprechen somit den Verhältnissen anderer Kombinationen, anhand derer \citet[26-27]{Thurmair1991} (vgl. auch \citealt[Kapitel 3]{Thurmair1989}) das gestaffelte Greifen dieser Schnittmengenbedingungen aufzeigt. 

Die Betrachtung in Abschnitt~\ref{sec:distributionjd} zeigt, dass die geteilte Umgebung von \textit{ja} und \textit{doch} (wenngleich es weitere interpretatorische Beschrän\-kungen gibt) Aussagesätze sind. Man hat es mit dem Formtyp zu tun, der sich durch ein $[\minus$w$]$ gefülltes Vorfeld, V2-Stellung, ein nicht-imperativisches finites Verb, keinen Exklamativakzent und einen fallenden Tonhöhenverlauf (vgl. \citealt[44]{Thurmair1989}, \citealt[176-177]{Oppenrieder1987}) auszeichnet. Der dem Aussagesatz zugeordnete Funktionstyp ist im Altmann'schen (\citeyear{Altmann1984, Altmann1987}) Satzmodusmodell die Assertion. Meine weitere Betrachtung untersucht die Funktion von \textit{ja}-, \textit{doch}- und \textit{ja doch}-Äußerun\-gen im Diskurs, d.h. ihren Effekt auf den Kontext sowie ihre kommunikativen Absichten. Die Generalisierung, auf die sich die folgende Argumentation im weiteren Verlauf stützt, ist deshalb, dass die relevante Domäne der Kombination dieser MPn aus Sicht des Funktionstyps die Assertion \is{Assertion} ist.

Die bisherigen Illustrationen zeigen allerdings auch, dass man es bei den Aussagesätzen, in denen sich die beiden Partikeln kombinieren lassen, nicht aus\-schließlich mit Standardassertionen (vgl. (\ref{347}), (\ref{348})) zu tun hat, sondern dass Äuße\-rungen auftreten, die funktional in den Bereich der Exklamativsätze (vgl. Abschnitt~\ref{sec:empha}) übergehen (vgl. (\ref{349}), (\ref{350})). Mit \citet[77-80]{Doherty1985}, \citet[107-108]{Thurmair1989} und \citet[37]{Kwon2005} (s.o.) gehe ich davon aus, dass es sich hierbei um Aussagen und somit ihrer Funktion nach ebenfalls Assertionen handelt.

\begin{exe}
	\ex\label{347} 
	Konrad ist \textbf{ja doch} verreist.	
\end{exe}
\vspace{-0.65cm}
\begin{exe}
	\ex\label{348} 
	\scriptsize
	A: Darfst das Bierglas nicht anfassen, weil du sonst eine Sehnenscheidenentzündung bekommst.\\
	B: \textbf{Die Sehnenscheidenentzündung war \underline{ja doch} von der Gitarre.}
	\newline
	\hbox{}\hfill\hbox{(FOLK\_E\_00039\_SE\_01\_T\_02) (Beleg bereinigt S.M.)}	
\end{exe}
\vspace{-0.65cm}
\begin{exe}
	\ex\label{349} 
	Das sind \textbf{ja doch} Paul und Maria! Was machen die denn hier?	
\end{exe}		
\vspace{-0.65cm}
\begin{exe}
	\ex\label{350} 
	Das ist \textbf{ja doch} die Höhe!
\end{exe}	
Inwiefern sich das assertive Potential in (\ref{347}) und (\ref{348}) von dem in (\ref{349}) und (\ref{350}) beteiligten unterscheidet, ist ein Aspekt, der in der weiteren Betrachtung aufgegriffen und in der Analyse der Abfolge von \textit{ja} und \textit{doch} entscheidend wird. 

Dass sich innerhalb der Assertionen weitere Untertypen ausmachen lassen, lässt sich auch beobachten, wenn man neben dem klassischen (assertiven) Formtyp auch andere assertive selbständige Sätze in die Betrachtung miteinbezieht, in denen die beiden MPn \textit{ja} und \textit{doch} jeweils einzeln und somit auch gemeinsam auftreten können. Auch diese Typen von Assertionen und ihre konkreten Funktionen spielen bei der Diskussion der (un)zulässigen Abfolgen der zwei Partikeln in Abschnitt~\ref{sec:unmarkiert} und \ref{sec:markiert} eine entscheidende Rolle.
				  
\subsection{Non-kanonische Deklarativsätze}
\label{sec:nonkan}
Eine Randerscheinung selbständiger Aussagesätze sind V1-Sätze des Typs in (\ref{351}) und (\ref{352}).

\begin{exe}
	\ex\label{351} 
	Zählt das Münster \textbf{doch} zu den wenigen romanischen Gotteshäusern, welche ...
	\hfill\hbox{\citet[294]{Oennerfors1997}}	
\end{exe}

\begin{exe}
	\ex\label{352} 
	\scriptsize
	Aufschlussreich sind Haiders Ausfälle allemal. \textbf{Bestätigt er \underline{doch} einmal mehr sein originelles Demo\-kratieverständnis:} Er glaubt 		immer noch, alles anordnen zu können, selbst Fußballsiege.  
	\newline
	\hbox{}\hfill\hbox{(A01/Nov. 40827, St. Galler Tagblatt, 06.11.2001) \citet[157]{Pittner2011}}	
\end{exe}
In Arbeiten zu diesen Strukturen wird angenommen, dass das Auftreten von \textit{doch} in diesem Satztyp obligatorisch ist (vgl. z.B. \citealt[295]{Oennerfors1997}, \citealt[172]{Pittner2011}). \citet[174]{Rinas2006} möchte für (satzintegrierte) Sätze dieses Typs aufgrund der Obligatorizität von \textit{doch} von einer idiomatisierten Konstruktion ausgehen (vgl. allerdings \citealt[162-169]{Pittner2011}, die aufzeigt, dass der Beitrag von \textit{doch} in dieser Umgebung kein anderer ist als in anderen Auftretenskontexten der Partikel, vgl. auch meine eigenen Ausführungen in Kapitel~\ref{chapter:dua}, Abschnitt~\ref{sec:Rand} sowie \citealt{MuellerimDruck})).

\citet[174]{Rinas2006} wie auch \citet[90]{Kwon2005} nehmen an, dass \textit{ja} in V1-Aussagesät\-zen \is{V1-Deklarativsatz} nicht möglich ist. \citet[158]{Oennerfors1997} geht davon aus, dass \textit{ja} in älteren Sprachstufen in dieser Umgebung auftreten konnte. Er verweist für Nachweise derartiger Beispiele auf alte Arbeiten wie \citet[74-75]{Sanders1883}, \citet[36-37]{Stenstad1917} und \citet[70-71]{Mattausch1965}. Entsprechende Daten sind aber auch durch\-aus im heutigen Deutschen zu finden wie die Belege in (\ref{353}) und (\ref{354}) aufzeigen.

\begin{exe}
	\ex\label{353} 
	\scriptsize
	\glqq Ich habe gute Kontakte zum Canisianum in Innsbruck\grqq{}, erzählt Kaplan Emil Bonetti. \textbf{Hat er dort \underline{ja} selbst während seinen 	Studienjahren gewohnt. }
	\newline
	\hbox{}\hfill\hbox{(V00/DEZ.60670 Vorarlberger Nachrichten, 04.12.2000)}	
\end{exe}

\begin{exe}
	\ex\label{354} 
	\scriptsize
	Dass die Züge voll sind, ist ein Problem, aber eigentlich ein gutes. \textbf{War es \underline{ja} das erklärte Ziel der Verkehrspolitik der letzten 		Jahre der Umwelt zuliebe die Bahn zu fördern.}
	\newline
	\hbox{}\hfill\hbox{(A11/APR.01450 St. Galler Tagblatt, 05.04.2011)}	
\end{exe}
Die Fälle sind eindeutig abzugrenzen von V2-Sätzen mit Vorfeld-Ellipse, die auf den ersten Blick für V1-Aussagesätze gehalten werden können und in denen das im Vorfeld ausgelassene Argument entweder später im Satz rechtsversetzt (vgl. (\ref{355})) oder gar nicht genannt wird (vgl. \ref{(356})).
	
\begin{exe}
	\ex\label{355} 
	\scriptsize
	Seit einigen Tagen weist ein brandneues Schild in den Signalfarben Rot und Weiß [...] auf die steilste Standseilbahn Deutschlands hin. \textbf{Ist 			\underline{ja} auch ein Schmuckstückchen, die Kurwaldbahn,} und noch dazu befördert sie die Fahrgäste in luftige Höhen über Bad Ems [...].
	\newline
	\hbox{}\hfill\hbox{(RHZ12/MAI.12144 Rhein-Zeitung, 11.05.2012)}	
\end{exe}

\begin{exe}
	\ex\label{356} 
	\scriptsize
	Bin gespannt, ob die jungen Alten vier Wochen durchhalten. Der Bundes-Berti ist davon überzeugt. \textbf{Hat \underline{ja} auch aus gutem Grund alle Positionen 		doppelt besetzt. }
	\newline
	\hbox{}\hfill\hbox{(RHZ98/JUN.14334 Rhein-Zeitung, 09.06.1998)}	
\end{exe}					                                   
Da auch \textit{ja} in diesem Randtyp von Aussagesatz auftreten kann, ist es nicht verwunderlich, dass in dieser Umgebung auch die Kombination von \textit{ja} und \textit{doch} zulässig ist. (\ref{357}) und (\ref{358}) sind völlig problemlos möglich und man muss auch hier nicht davon ausgehen, dass selbständige assertive V1-Sätze nur auf älteren Sprachstufen die \textit{ja doch}-Kombination erlaubten, wie das Beispiel in (\ref{359}) aus \citet{Oppenrieder1987} nahelegen könnte, auf das \citet[158, Fn 186]{Oennerfors1997} verweist.
 
\begin{exe}
	\ex\label{357} 
	\scriptsize
	Gründler geht davon aus, dass Peter S. die damals 27-jährige Arzthelferin am Tatmorgen in der Tiefgarage abfing, um sie in dieser Angelegenheit zur Rede zu stellen und letztlich mundtot zu machen. \textbf{\textit{Musste} S. \underline{ja doch} befürchten, dass Susanne M. tatsächlich Anzeige gegen ihn erstatten würde. }		   
	\newline
	\hbox{}\hfill\hbox{(NUZ09/DEZ.01570 Nürnberger Zeitung, 15.12.2009) (modifiziert durch S.M.)}	
\end{exe} 		
 		
\begin{exe}
	\ex\label{358} 
	\scriptsize
	Verliehen wird die Auszeichnung an Bürger, die sich besondere Verdienste erworben haben. Und da fand der Bürstädter Bürgermeister Alfons Haag viel 			Kurzweiliges. \textbf{\textit{Ist} Deckenbach \underline{ja doch} weit über die dörfliche Grenze hinaus als \glqq Kerwe-Opa\grqq{} bekannt}, der seit 		vielen Jahren dem \glqq Kerwevadder\grqq{} die Reden reimt.		   
	\hfill\hbox{(M09/DEZ.97728 Mannheimer Morgen, 09.12.2009) (modifiziert durch S.M.)}	
\end{exe} 	
							                               	
\begin{exe}
	\ex\label{359} 
	Stahl sie \textbf{ja doch}, dem Gatten zulieb, die Götzen des finsteren Vaters.	   
	\newline
	\hbox{}\hfill\hbox{\citet[187, Anm. 26]{Oppenrieder1987}}	
\end{exe}		
Wenn Önnerfors davon ausgeht, dass \textit{ja }nur zu früheren Zeiten hier auftreten konnte, würde sich auch das nur damalige kombinierte Auftreten plausibel in seine Argumentation einfügen. (\ref{353}) und (\ref{354}) sowie die akzeptablen (\ref{357}) und (\ref{358}) weisen allerdings darauf hin, dass die Beschränkung auf ältere Sprachstufen nicht angenommen werden muss. 

Ein anderer Randtyp eines Aussagesatzes ist \is{Wo-VL-Deklarativsatz} der \textit{wo}-VL-Satz, der hier weder lokal noch temporal interpretiert wird (vgl. auch zu diesem Satztyp ausführlich Kapitel~\ref{chapter:dua}, Abschnitt~\ref{sec:Rand} sowie \citealt{MuellerimDruck}). (\ref{360}) zeigt, dass \textit{doch} in dieser Umgebung aufzufinden ist, wenngleich es in diesem Satztyp nicht obligatorisch ist (vgl. z.B. \citealt[152]{Pasch1999} sowie \citealt[324]{Guenthner2002}).

\begin{exe}
	\ex\label{360} 
	\scriptsize
	Wobei sich die zuständige Stadträtin Karin Gutmann scherzhaft beschwerte: \glqq Es ist beinahe unfair, dass die Männer einen meist niedrigeren 				Cholesterinwert haben als die Frauen. \textbf{\textit{Wo} sie \underline{doch} sonst nicht so auf Gesundheit bedacht sind ...}\grqq{}    
	\newline
	\hbox{}\hfill\hbox{(NON12/APR.18615 Niederösterreichische Nachrichten, 26.04.2012)}	
\end{exe} 	
\textit{Ja} lässt sich ebenfalls belegen, auch wenn beispielsweise \citet[194]{Kwon2005} Gegenteiliges behauptet.

\begin{exe}
	\ex\label{361} 
	\scriptsize
	\glqq Wir sind schon sehr stolz darauf, dass wir das in dieser Saison erreicht haben. \textbf{\textit{Wo} wir \underline{ja} zu Beginn als Abstiegskandidat gehandelt wurden}\grqq{}, sagte Coach Mario Graf.    
	\newline
	\hbox{}\hfill\hbox{(NON07/MAI.03947 Niederösterreichische Nachrichten, 07.05.2007)}	
\end{exe}

\begin{exe}
	\ex\label{362} 
	\scriptsize
	Diese Heimspielrechnung wollte Bayer-Coach Klaus Toppmöller nicht mit aufmachen. \textbf{\textit{Wo} Bayer \underline{ja} auch nur noch ein Spiel in 		der BayArena hat} [...].   
	\hfill\hbox{(RHZ02/FEB.15088 Rhein-Zeitung, 21.02.2002)}	
\end{exe}
Das \textit{wo} wird in den Sätzen in (\ref{360}) bis (\ref{362}) jeweils im Sinne von \textit{denn}, \textit{weil}, \textit{da}, \textit{zumal} bzw. \textit{obwohl} verstanden, d.h. es wird kausal bzw. konzessiv interpretiert.

Im Einklang mit der syntaktischen Schnittmengenbedingung lassen sich \textit{ja} und \textit{doch} auch in diesem Typ von Aussagesatz kombinieren (vgl. (\ref{363})).

\begin{exe}
	\ex\label{363} 
	\scriptsize
	Es könnte ja komisch anmuten, sich ins Kino zu setzen und sich das jugendliche Leben reinzuziehen. \textbf{\textit{Wo} man es \underline{ja doch} 			täglich in Plattengeschäften, in Einkaufspassagen oder auf Pausenplätzen vorgeführt bekommt.}
	\hfill\hbox{(E97/SEP.22830 Zürcher Tagesanzeiger, 24.09.1997)}	
\end{exe}
Ausgehend von der Generalisierung, dass Assertionen die zulässige Domäne für Kombinationen aus \textit{ja} und \textit{doch} darstellen, zeigt der letzte Abschnitt, dass auch die Form der Assertionen von Standardassertionen in Gestalt von $[ \minus \textrm{w}]$, V2-Sätzen abweichen kann. Auch die Randtypen von V1- bzw. \textit{wo}-VL-Aussagesätzen sind in die Betrachtung einzubeziehen. In Abschnitt~\ref{sec:markiert} werden Besonderheiten in der Interpretation dieser Sätze genauer beleuchtet. 

Formuliert man die zulässige Domäne des Auftretens der MP-Kombination unter Bezug auf den Funktionstyp der Assertion, bringt dies den Vorteil mit sich, dass auch Nebensätze, in denen \textit{ja}, \textit{doch} und somit \textit{ja doch} auftreten können, in die Untersuchung einbezogen werden können. 

\subsection{Eingebettete Kontexte}
\label{sec:eingkon}
Die beiden Partikeln können z.B. in kausalen Nebensätzen\footnote{Ich erhebe an dieser Stelle keinen Anspruch auf Vollständigkeit der Auftretenskontexte. Zweck ist, das Auftreten in Nebensätzen soweit zu motivieren, wie es für die Ausführungen in Abschnitt~\ref{sec:markiert} relevant ist. Neben den verschiedenen Kausalsätzen erlauben auch Konzessiv-, non-restriktive Relativsätze sowie manche Komplementsätze das Auftreten von \textit{ja}, \textit{doch} und \textit{ja doch}. Ich klammere diese Typen aus der Darstellung aus, das sie für die weitere Betrachtung nur eine untergeordnete Rolle spielen. Gleiches gilt für unzulässige Kontexte wie restriktive Re\-lativsätze, Temporalsätze oder Konditionalsätze (vgl. z.B.: \citealt[11]{Borst1985}, \citealt[76]{Thurmair1989} \citealt[166]{Helbig1994}, \citealt[115]{Coniglio2007}, \citealt[139/152/147/152]{Coniglio2011}, \citealt[59]{Frey2011}).} auftreten, die durch verschiedene Konnektoren eingeleitet sein können.

\begin{exe}
	\ex\label{364} 
	Hans sollte Inge lieber nicht reizen, \textit{\textbf{da}} sie ihn \textbf{ja}/\textbf{doch} neulich geohrfeigt hat.
\end{exe}
\vspace{-0.65cm}	
\begin{exe}
	\ex\label{365} 
	Sie fürchtet sich, \textit{\textbf{weil}} ihr Vater sie \textbf{ja}/\textbf{doch} immer schlägt.
	\newline
	\hbox{}\hfill\hbox {nach \citet[77/81]{Borst1985}}
\end{exe}
\vspace{-0.65cm}
\begin{exe}
	\ex\label{366} 
	Der Kurzbesuch ist deinen Eltern hoch anzurechnen, \textit{\textbf{zumal}} sie \textbf{ja}/\textbf{doch} eine Anreise von 200km zurückzulegen haben.
\end{exe}
Im Einvernehmen mit der Schnittmengenbedingung ist dann jeweils auch das kombinierte Auftreten zulässig (vgl. (\ref{367}) bis (\ref{369})).

\begin{exe}
	\ex\label{367} 
	\scriptsize
	Ich würde eine schriftliche(e-mail reicht auch)Nachfrist setzen und dabei gleich mein Befremden zum Ausdruck bringen, über die Abwicklung und 				bezüglich der Ehrlichkeit gegenüber dem Kunden, \textbf{\textit{zumal}} du \textbf{ja doch} ein Stammkunde bist.
	\hfill\hbox {(DECOW 2012: 1188552685)}
\end{exe}
\vspace{-0.65cm}
\begin{exe}
	\ex\label{368} 
	da ist mir jedesmal zum heulen zumute wenn ich das sehe \textbf{\textit{da}} ich \textbf{ja doch} so leidenschaftlich koche.	
	\hfill\hbox {(DECOW 2012: 4194727)}
\end{exe}									                                 
\vspace{-0.65cm}
\begin{exe}
	\ex\label{369} 
	Das wäre schade, \textbf{\textit{weil}} das Trio \textbf{ja doch} genauso schön ist wie der Hauptteil.	
	\newline
	\hbox{}\hfill\hbox {(DECOW 2012: 199984396)}
\end{exe}		                                       
Gleiche Verhältnisse stellen sich bei kausalen \textit{wo}-Sätzen ein: \textit{doch} kann auftreten, genauso wie \textit{ja} (Gegenteiliges behaupten \citealt[63]{Thurmair1989} und \citealt[213]{Rinas2006}), weshalb auch das kombinierte Auftreten keine Überraschung darstellt.

\begin{exe}
	\ex\label{370} 
	Hans sollte Inge lieber nicht reizen, \textbf{\textit{wo}} sie ihn \textbf{doch} neulich geohrfeigt hat.	
	\hfill\hbox {\citet[77]{Borst1985}}
\end{exe}	

\begin{exe}
	\ex\label{371} 
	\scriptsize
	Es ist daher sehr unwahrscheinlich, dass Ihr Kleiner auf die Früchte in unseren Gläschen reagiert, \textit{\textbf{wo}} er sie \textbf{ja} sogar roh 		bereits vertragen hat.	
	\hfill\hbox {(http://www.hipp.de/forum/viewtopic.php?f=12\&t=4608)}
	\newline
	\hbox{}\hfill\hbox {(Google-Recherche 23.4.2012)}
\end{exe}

\begin{exe}
	\ex\label{372} 
	Können die mir denn eine Karte überhaupt verweigern \textbf{\textit{wo}} ich \textbf{ja doch} versichert bin?
	\hfill\hbox {(DECOW 2012: 1109601892)}
\end{exe}
Zu diesem Typ von Kausalsatz wurde in Abschnitt~\ref{sec:nonkan} bereits die selbständige Variante angeführt. Auch der dort im gleichen Zuge eingeführte selbständige V1-Satz weist ein unselbständiges Pendant auf. \textit{Doch} ist in diesen Sätzen möglich (nach Ansicht von \citealt[77-80]{Borst1985} und \citealt[13]{Rinas2006} sogar obligatorisch). Entgegen der Einschätzung von \citet[213]{Rinas2006} lässt sich auch \textit{ja} gut belegen, so dass auch dem kombinierten Auftreten nichts im Wege steht.
	
\begin{exe}
	\ex\label{373} 
	\scriptsize
	Auch die praktische Arbeit und Anschauung kamen nicht zu kurz, \textbf{\textit{verfügte}} Neustadt \textbf{doch} über eine günstige Lage [...].	
	\hfill\hbox {(DECOW-2012: 218534182)}
\end{exe}	
	
\begin{exe}
	\ex\label{374} 
	\scriptsize
	Eigentlich schade, den Bass nicht deutlicher in den Vordergrund zu stellen, \textbf{\textit{war}} Dixon \textbf{ja} nicht nur Komponist, sondern vor 		allem auch Bassist. 
	\hfill\hbox {(M11/OKT.08329 Mannheimer Morgen, 26.10.2011)}
\end{exe}

\begin{exe}
	\ex\label{375} 
	\scriptsize
	Öffentliche Gebäude gehören vermehrt genutzt, \textbf{\textit{werden}} sie \textbf{ja doch} vom Steuergeld unserer Bür\-gerInnen errichtet. 	
	\hfill\hbox {(NON08/MAI.05354 Niederösterreichische Nachrichten, 12.05.2008)}
\end{exe}
In der Literatur werden diejenigen Nebensätze, die MPn erlauben, zu den sogenannten \textit{peripheren Nebensätzen} \is{peripherer Nebensatz} gezählt.\footnote{Ich werde diese globale Zuordnung in Kapitel~\ref{chapter:hue} im Rahmen der Untersuchung des Auftretens von \textit{halt} und \textit{eben} in Relativsätzen in Frage stellen.} Diesen (adverbialen) Nebensätzen will man eine gewisse illokutive Selbständigkeit zuschreiben, während dem \glq Pendant\grq {} - den \textit{zentralen Adverbialsätzen} \is{zentraler Nebensatz} - diese Eigenschaft abgesprochen wird (vgl. \citealt{Haegeman2002, Haegeman2004, Haegeman2006}). Die Annahme, dass peripheren Nebensätzen eine gewisse illokutionäre Selbständigkeit zugeschrieben wird, geht auf die Beobachtung zurück, dass bestimmte Phänomene (die sogenannten \is{Wurzelphänomen} \textit{Wurzelphänomene}, die eigentlich auf Hauptsätze beschränkt sind, (je nach Sprache mehr oder weniger restringiert) auch in diesen Nebensätzen vorkommen können (vgl. \citealt{Emonds1969}, \citealt{Rutherford1970}, \citealt{Hooper1973}, für einen Überblick vgl. \citealt{Heycock2005}). \citet{Coniglio2011} und \citet{Abraham2012} vertreten, dass auch MPn zu den Wurzelphänomenen zu zählen sind. Coniglios Erklärung dafür ist, dass MPn illokutive/diskursive Funktion haben (Sprechereinstellungen kodieren, Illokuti\-onstypen modifizieren) und deshalb für die Domäne des Auftretens fordern, dass illokutive Selbständigkeit besteht. Gilt es nun zu entscheiden, welche Illokution in den die MPn lizensierenden Nebensätzen vorliegt, handelt es sich plausiblerweise um Assertivität. \citet[120]{Doherty1987} und \citet[93]{Kwon2005} behaupten auch schon vor der Diskussion um Wurzelphänomene und der Unterscheidung zwi\-schen peripheren und zentralen Nebensätzen -- ohne große Elaboration dieses Aspektes --, dass man es hier mit assertiven Kontexten zu tun hat.\\

\noindent
Es lässt sich also festhalten, dass mit \textit{ja}, \textit{doch} und \textit{ja doch} in selbständigen und eingebetteten assertiven Kontexten zu rechnen ist (vgl. auch schon \citealt[167-170]{Mueller2014a}; \citeyear[205-207]{Mueller2017b}). Da die Festlegung auf Assertionen möglich ist, wird auch meine eigene Ableitung der Abfolge von \textit{ja} und \textit{doch} auf Eigenschaften von Assertionen abheben. Der folgende Abschnitt skizziert auf diesem Weg zunächst bestehende Ansätze aus der Literatur zu dieser Frage. 

\section{Die Abfolge \textit{ja doch} und bestehende Erklärungsversuche}
\label{sec:abfolgejd}
In Arbeiten, die sich mit der konkreten Kombination aus \textit{ja} und \textit{doch} beschäftigen, fallen die Urteile so aus, dass das \textit{ja} dem \textit{doch} vorangeht (vgl. z.B. (\ref{376}) bis (\ref{378}), vgl. z.B. auch die Bewertungen in \citealt[101]{Meibauer1994}, \citealt[93]{Ormelius-Sandblom1997}, \citealt[431]{Rinas2007}).
\setcounter{equation}{0}
\begin{exe}
	\ex\label{376} 
	Konrad ist \textbf{ja doch}/*\textbf{doch ja} verreist.
\hfill\hbox {\citet[114]{Doherty1987}}
\end{exe}
\vspace{-0.5cm}
\begin{exe}
	\ex\label{377} 
	Er hat \textbf{ja doch}/??\textbf{doch ja} getanzt.
\hfill\hbox {\citet[20]{Struckmeier2014}}
\end{exe}
\vspace{-0.5cm}
\begin{exe}
	\ex\label{378} 
	Er hat sich \textbf{ja doch}/?\textbf{doch ja} sehr um sie bemüht.
\hfill\hbox {\citet[157]{Jacobs1991}}
\end{exe}
Wenngleich die Einschätzungen zwischen *, ?? und ? etwas schwanken, gilt für alle mir bekannten Betrachtungen, die Erklärungsvorschläge anbieten, dass sie beabsichtigen, die Abfolge \textit{doch ja} als ungrammatisch auszuschließen, und in der Regel im Rahmen der Modelle auch keine Möglichkeit besteht, die umgekehrte Reihung überhaupt zuzulassen.

Um diesen Aspekt der Ausgangslage meiner eigenen Betrachtung zu verdeutlichen, seien im Folgenden vor diesem Hintergrund einige Vorschläge (erneut) skizziert (zu ausführlichen Darstellungen s. Abschnitt~\ref{sec:forschung} in Kapitel~\ref{chapter:hintergrund}). \citet{Doherty1985, Doherty1987}, \citet{Ickler1994}, \citet{Ormelius-Sandblom1997} und \citet{Rinas2007}, die Erklärungsmodelle für die Ordnung von \textit{ja} und \textit{doch} vorschlagen, argumentieren dabei auf verschiedene Art unter Bezug auf die Interpretation der MP-Kombination, während \citet{Lindner1991} eine phonologische Beschränkung verantwortlich macht. Der entscheidende Punkt an dieser Stelle ist, dass eigentlich nur Lindner überhaupt die Möglichkeit für die Existenz der \textit{doch ja}-Abfolge zulässt (wenngleich sie sie auch nicht in Betracht zieht).

\citet{Doherty1985, Doherty1987} zufolge ordnen sich die Partikeln \textit{ja} und \textit{doch} entlang des Kriteriums der \is{assertive Stärke} abnehmenden assertiven Stärke, d.h. entlang unterschiedlicher Grade der Verbindlichkeit der Sprecherhaltung. \textit{Ja}-Äußerungen weisen einen höheren Grad an assertiver Stärke auf als \textit{doch}-Äußerungen, so dass die MP \textit{ja} der MP \textit{doch} stets vorangeht. Doherty nimmt diese Zuschreibung auf der Basis des jeweils angenommenen MP-Beitrags vor (vgl. (\ref{379})): \textit{ja} legt den Sprecher anders als \textit{doch} stets auf eine assertive Haltung zur Einstellung im Skopus der Partikel fest.

\begin{exe}
	\ex\label{379} 
		\textit{ja}: Ass$(\textrm{E}_{\textrm{S}}$(\textrm{p})$)$ und I\textrm{M}$(\textrm{E}_{\textrm{X}}$(\textrm{p})$)$\\
		\textit{doch}: $\textrm{Ass}^{\prime} (\textrm{E}_{\textrm{S}}(\textrm{p})) \ \textrm{und IM(neg}_{\textrm{X}}(\textrm{p}))$
		\hfill\hbox {\citet[80/71]{Doherty1985}}
\end{exe}
Da sie hier die inhärente, wörtliche Bedeutung der MPn modelliert, die durch nichts beeinflussbar sein sollte, ist in ihrem Modell nicht angelegt, dass sich an diesen Verhältnissen unterschiedlicher Grade von assertiver Stärke etwas ändert. Sobald von der MP-Ordnung \textit{ja doch} abgewichen wird, folgen die MPn auch nicht mehr der Stärkehierarchie von Assertivität und es sollte zu Akzeptabilitätsverlusten kommen.

Dieser Aspekt der Invarianz ihres Kriteriums bei der Ableitung der Abfolgen wird noch deutlicher, wenn man in ihre tiefere Ausdeutung dieses oberflächennahen Kriteriums schaut, die Bezug nimmt auf die Interaktion verschiedener Typen von Einstellungen (vgl. Abschnitt~\ref{subsec:input}). Der Beitrag von \textit{ja} ist es, in einer Äußerung die Einstellung in ihrem Skopus zu bestätigen Ass$(\textrm{E}_{\textrm{S}}$(\textrm{p})$)$. \textit{Ja} benötigt als Argument ein Objekt des Typs E. Da \textit{doch} keine Einstellung ausdrückt, sondern eine \underline{Haltung} zu einer Einstellung (in Assertionen ebenfalls \textit{Ass}), eignet sich der Ausdruck, der \textit{doch(p)} entspricht, nicht als Input für den Ausdruck von \textit{ja(p)}. Beide MPn assertieren, wenn sie gemeinsam auftreten, deshalb die Einstellung in ihrem Skopus. Sie nehmen den gleichen Skopus. Liegt die umgekehrte Abfolge vor, assertiert das \textit{ja} zunächst die Einstellung in seinem Skopus Ass$(\textrm{E}_{\textrm{S}}$(\textrm{p})$)$. Genau wie \textit{ja} benötigt \textit{doch} eine Einstellung als Argument, die es im Fall einer Assertion bestätigt. \textit{Ja(p)} kann an dieser Stelle der Berechnung jedoch nur den assertiven Einstellungs\underline{modus} beisteuern, der sich als Objekt im Skopus \is{Skopus} von \textit{doch} nicht eignet. Doherty zufolge scheidet diese MP-Abfolge somit aus, weil an diesem Punkt die Bedeutungszuschreibung scheitert (vgl. \citealt[84-85]{Doherty1985}). Wie in Abschnitt~\ref{subsec:input} ausgeführt, lässt sich Dohertys Ableitung als Erfüllung bzw. Verstoß von durch die beteiligten Objekte geforderten Inputbedingungen auffassen. Da es sich hierbei um semantische Anforderungen handelt, besteht keine Möglichkeit, den Verstoß im Falle der Abfolge \textit{doch ja} zu umgehen. Sie sollte folglich stets als ungrammatisch herausgefiltert werden.

Mit der Annahme, dass für \textit{doch} in der Abfolge \textit{doch ja} nach Applikation von \textit{ja} kein geeignetes semantisches Objekt vorliegt, auf das \textit{doch} Bezug nehmen kann, benennt Doherty einen Grund für den Ausschluss von \textit{doch ja}. Es bleibt aber ungeklärt, warum es hier nicht ebenfalls möglich ist, dass sich – wie im Falle von \textit{ja doch} (wo unter Skopus das gleiche Problem auftritt) – beide Partikeln auf die Einstellung E beziehen können.\footnote{Die einzige Erklärung, die mir plausibel scheint, ist, dass der Einstellungsmodus EM im Falle von \textit{doch ja} im Stadium der Verrechnung von \textit{doch} bereits festgelegt ist (\textit{Ass} durch \textit{ja}) und der Zugriff auf E in dessen Skopus nicht mehr möglich ist. Im Falle von \textit{ja doch} hingegen ist der EM nach der Integration von \textit{doch} noch nicht entschieden und \textit{ja} kann deshalb auf E zugreifen.}

Sowohl \citet{Ormelius-Sandblom1997} als auch \citet{Rinas2006, Rinas2007} schließen die Abfolge \textit{doch ja} aus, indem sie auf die vorliegenden syntaktischen/semantischen Skopusverhältnisse eingehen. Beide gehen davon aus, dass eine direkte Korrelation zwischen syntaktischer Struktur und semantischem Skopus besteht. Strukturell tiefer positionierte MPn stehen im Skopus von hierarchisch höher verankerten MPn. (\ref{380}) und (\ref{381}) zeigen erneut die Modellierung der Bedeutung der Einzelpartikeln nach \citet[420]{Rinas2007}.

\begin{exe}
	\ex\label{380} 
	\textit{ja}: JA(p) $»$ NICHT-GLAUBT(H, NICHT-p)
\end{exe}
\vspace{-0.65cm}
\begin{exe}
	\ex\label{381} 	
	\textit{doch}: DOCH(p) $»$ WIDERSPRICHT(p,q) \& (KENNT(H,p) $\lor$ KENNT(H,q))
	\newline
	\hbox{}\hfill\hbox{\citet[425/420]{Rinas2007}}	
\end{exe}
(\ref{382}) gibt nach Rinas die Bedeutung der MP-Kombination \textit{ja doch} adäquat wieder.
\begin{exe}
	\ex\label{382} 
		\begin{xlist}	
			\ex\label{382a} JA(DOCH(p) $»$ WIDERSPRICHT(p,q) \& KENNT(H,p) $\lor$ KENNT(H,q))
				$»$ NICHT GLAUBT(H, NICHT(DOCH(p) $»$ WIDERSPRICHT(p,q) \& \\ KENNT(H,p) $\lor$ KENNT(H,q)))
			\ex\label{382b} \glq Es ist unkontrovers (es ist nicht der Fall, dass der Hörer daran zweifelt), dass p im 							Widerspruch zu einer anderen Proposition q steht und dass dem Hörer p oder q bekannt ist.\grq {}
			\hfill\hbox {\citet[431]{Rinas2007}}
		\end{xlist}
\end{exe}
Aus dieser Bedeutungszuschreibung, der zufolge \textit{ja} Skopus über \textit{doch} nimmt, folgt s.E., dass \textit{ja} in der Struktur hierarchisch höher positioniert ist als \textit{doch}, was sich wiederum in der linearen Abfolge \textit{ja doch} niederschlage. Aufgrund dieses hierarchischen/asymmetrischen Verhältnisses zwischen den beiden MPn sei abzuleiten, dass die Umkehr der Reihung von \textit{ja} und \textit{doch} nicht möglich sei. Die Abfolge \textit{doch ja} existiert nach Rinas folglich nicht, weil diese Reihung nicht die Interpretation widerspiegelt, in der \textit{doch} im Skopus von \textit{ja} steht.

Wenngleich von Doherty und Rinas vertreten wird, dass \textit{doch ja} als ungrammatisch auszuschließen ist, unterscheiden sich die beiden Ansätze m.E. insofern, als dass es in Rinas' Modell nichts gibt, was die umgekehrte Abfolge ausschließt. Wenn man von einer Korrelation zwischen Skopus und linearer Abfolge ausgeht, spricht nichts dagegen (und eher einiges dafür), dass bei anderer Anordnung derselben MPn schlichtweg eine andere Interpretation vorliegt. In diesem Fall nimmt dann die Partikel \textit{doch} die Partikel \textit{ja} in ihren Skopus. Diese mögliche Interpretation wird von Rinas nicht in Betracht gezogen, obwohl sie – wie (\ref{383}) zeigt – unter Verwendung seiner Bedeutungszuschreibungen ohne Weiteres zu modellieren wäre.

\begin{exe}
	\ex\label{383} 
		\begin{xlist}	
			\ex\label{383a} DOCH(JA(p) $»$ NICHT-GLAUBT(H,NICHT-p))
				$»$ \\ WIDERSPRICHT((JA(p) $»$ NICHT-GLAUBT(H,NICHT-p)),q) \& \\
				(KENNT(H,JA(p) $»$ NICHT-GLAUBT(H,NICHT-p)) 
				\\ $\lor$ KENNT(H,q))
			\ex\label{383b} \glq Die Tatsache, dass p unkontrovers ist, steht im Widerspruch zu einer anderen Proposition q und 				dem Hörer ist bekannt, dass p unkontrovers ist oder q. \grq {}
			\hfill\hbox {\citet[431]{Rinas2007}}
		\end{xlist}
\end{exe}
Wollte Rinas die Kombination \textit{doch ja} im Rahmen seines Modells ausschließen, müsste er prinzipiellere Gründe gegen die Bedeutungszuschreibung in (\ref{383}) anführen. Auch wenn zwischen \textit{ja} und \textit{doch} in der Kombination \textit{ja doch} semantisch ein Skopusverhältnis besteht, mit dem ein hierarchisch-syntaktischer Strukturunterschied bzw. eine bestimmte Linearisierung einhergeht, berechtigt die Logik seiner Ableitung eigentlich nicht den Ausschluss von \textit{doch ja} bzw. fehlt die Angabe eines echten Grundes für diese Entscheidung. Parallele Annahmen können auch für die Ableitung von \citet{Ormelius-Sandblom1997} gemacht werden. Mit einer anderen Modellierung der Einzelpartikeln kommt sie zum gleichen Schluss wie Rinas: \textit{ja} geht doch voran, weil diese Abfolge dem Skopusverhältnis zwischen den beiden MPn entspricht (vgl. \citealt[92-93]{Ormelius-Sandblom1997}).

Ihre Überlegung hinsichtlich dieses Zusammenhangs schließt aber ebenso we\-nig wie bei Rinas aus, dass \textit{doch ja} in der Bedeutung \textit{doch(ja)} auftritt.

Auch \citet{Ickler1994} geht davon aus, dass allein \textit{ja doch} die akzeptable Kombination der zwei Partikeln darstellt. Er macht ebenfalls die Interpretation der Kombination für diese Annahme verantwortlich. Und es gilt ebenfalls, dass zwischen den MPn ein Skopusverhältnis vorliegt. Aus seinen Ausführungen ist allerdings ein Grund für den Ausschluss von \textit{doch ja} herauszulesen. Wie auch \citet{Vismans1994} geht \citet{Ickler1994} davon aus, dass verschiedene MPn auf unterschiedlichen (interpretatorischen) Ebenen wirken. Dadurch, dass die angenommenen Ebenen relativ zueinander hierarchisch angeordnet sind, ergeben sich für die diesen Ebenen zugeordneten MPn invariante Abfolgen. In diesem Sinne sind nur bestimmte MP-Abfolgen überhaupt interpretierbar.

In seiner Modellierung wirkt \textit{ja} auf der Ebene der Argumentation: Die Partikel weise einer Aussage in einer Argumentation ihren Stellenwert zu, indem diese bestätigt und bekräftigt werde (\citeyear[399]{Ickler1994}). \textit{Doch} zeige einen inhaltlichen Gegensatz an (\citeyear[401]{Ickler1994}), weshalb diese Partikel der Inhaltsebene zugeordnet ist. Die beiden beteiligten Ebenen sind derart geordnet, dass die Argumentationsebene \is{Argumentationsebene} der Inhaltsebene \is{Inhaltsebene} übergeordnet ist und erstere letztere somit in ihren Wirkungsbereich nimmt. In einer \textit{ja doch}-Äußerung werde durch \textit{doch} auf einen inhaltlichen Gegensatz verwiesen, \textit{ja} bekräftige diese Aussage anschließend und weise ihr damit ihren Stellenwert in der Argumentation zu. Interpretatorisch geht die Inhaltsebene der Argumentationsebene voran. Aus der Hierarchie \textit{Argumentation(Inhalt)} folgt dann (neben der (Skopus-)Interpretation \textit{ja(doch))} die Abfolge \textit{ja doch}. Der Ausschluss der umgekehrten Abfolge ist in Icklers Zugang auf die Art motiviert, dass er das umgekehrte Zusammenspiel zwischen den zwei Ebenen nicht für möglich hält (\citeyear[404]{Ickler1994}). Wenn zwischen den Ebenen das Ordnungsverhältnis \textit{Argumentation(Inhalt)} besteht, ist nicht denkbar, dass die mit der höheren Ebene assoziierte MP (\textit{ja}) in den Wirkungsbereich der der tieferen Ebene zugeschriebenen MP (\textit{doch}) fällt. Dies würde eine Umkehr des zwischen Inhalts- und Argumentationsebene bestehenden hierarchischen Verhältnisses bedeuten. Da diese Ordnung sowie die Zuordnung der beiden Partikeln zu den Ebenen invariant ist, ist die umgekehrte Abfolge \textit{doch ja} generell ausgeschlossen.

Anders als die vier angeführten Arbeiten, argumentiert \citet{Lindner1991} unter Bezug auf ein phonologisches Kriterium. Entscheidend für ihre konkrete Er\-klärung ist zum einen die \textit{Hierarchie konsonantischer Stärke} \is{konsonantische Stärke} und zum anderen drei \textit{Präferenzgesetze} \is{Präferenzgesetz} für den Silbenbau und die Silbenverkettung, die auf diesem Konzept basieren (vgl. \citealt[283/284]{Vennemann1982}). (\ref{384}) zeigt die Skala konsonantischer Stärke, die genau entgegen der üblicheren \textit{Sonoritätshierarchie} \is{Sonoritätshierarchie} verläuft.

\begin{exe}
	\ex\label{384} 
	Hierarchie konsonantischer Stärke\\
	Vokale < Liquide < Nasale < stimmhafte Frikative < stimmhafte Plosive/\\stimmlose Frikative < stimmlose Plosive
\end{exe}
Zu den sechs Präferenzgesetzen, die Vennemann formuliert, gehört z.B. \is{Silbenkontaktgesetz} das \textit{Silbenkontaktgesetz}. Es besagt, dass wenn zwei Silben aufeinander treffen, der letzte Laut der ersten Silbe und der erste Laut der zweiten Silbe eine große Differenz hinsichtlich ihrer konsonantischen Stärke aufweisen. Das \textit{Endrandgesetz} \is{Endrandgesetz} und das \textit{Anfangsrandgesetz} \is{Anfangsrandgesetz} beschreiben ferner, dass der letzte Laut der ersten Silbe wenig und der erste Laut der zweiten Silbe viel konsonantische Stärke aufweist.

Die Präferenz für die Abfolge \textit{ja doch} (gegenüber \textit{doch ja}) führt Lindner nun darauf zurück, dass der Silbenkontakt im ersten Fall besser ist als im zweiten: Sie klassifiziert /x/ als stimmlosen velaren Frikativ und /j/ als stimmhaften palatalen Frikativ. Nach der Hierarchie in (\ref{384}) folgen der stimmhafte und stimmlose Frikativ direkt aufeinander, während zwischen dem Vokal /a:/ und dem stimmhaften Plosiv /d/ drei Stufen liegen. Die Differenz konsonantischer Stärke zwischen dem Endrand der Vorsilbe und dem Anfangsrand der Folgesilbe ist bei \textit{ja doch} somit deutlich größer, so dass ein nahezu optimaler Silbenkontakt vorliegt. Nach \citet{Lindner1991} ist folglich der schlechtere Silbenkontakt bei \textit{doch ja} und der bessere Silbenkontakt im Falle von \textit{ja doch} dafür verantwortlich, dass die Abfolge \textit{ja doch} die präferierte ist.

An dieser Stelle erachte ich es für relevant, dass Lindners Ableitung des bevor\-zugten \textit{ja doch} (im Gegensatz zu den Modellen von Doherty, Ickler, Rinas und Ormelius-Sandblom) die umgekehr\-te Abfolge nicht kategorisch ausschließt. Sie selbst geht auf diese Möglichkeit zwar nicht ein, doch ihr Erklärungsmodell erlaubt sie prinzi\-piell, wenngleich es (korrekterweise) die Bevorzugung von \textit{ja doch} vorhersagt. Nicht-optimaler Silbenkontakt lässt sich schließlich generell \\ durchaus beobachten (vgl. z.B. \textit{himmlisch} $[/m/–/l/]$, \textit{Ampel} $[/m/–/p/]$ und \textit{Alpen} $[/l/–/p/]$). Und wie \citet[430]{Rinas2007}) bemerkt, ist auch das direkte Aufein\-andertreffen von /x/ und /j/ zu belegen: \textit{Ach ja!}, \textit{Ich mach ja gar nichts.}

Dieser erneute Blick auf Ansätze, die sich mit der MP-Kombination aus \textit{ja} und \textit{doch} beschäftigen, zeigt, dass die Arbeiten derart ausgerichtet sind, Gründe für die Abfolge \textit{ja doch} anzuführen. Die Sequenzierung \textit{doch ja} wird als inakzeptable Reihung angesehen, die es gilt herauszufiltern. Die Existenz der Reihung \textit{doch ja} wird in keiner der mir bekannten Arbeiten in Betracht gezogen und sie lassen in der Regel auch keine Option, die ihre E\-xistenz zulassen würde (vgl. auch schon \citealt[170-175]{Mueller2014a}).

\section{Die Distribution von \textit{doch ja}}
\label{sec:distributiondj}
Auf der Basis der Betrachtung großer Datenmengen in verschiedenen Korpora (DeReKo, DECOW, deWac, DWDS, Projekt Gutenberg) sowie \glq wilderen\grq {} Google-Suchen möchte ich neue Daten ins Feld führen, die die Annahme, dass die umge\-kehrte Abfolge \textit{doch ja} non-existent und kategorisch auszuschließen ist, herausfordern. Es gibt rund um diese Daten Aspekte, die als gesichert gelten können. An anderen Stellen sind auch sicherlich Fragen offen, die ich hier nicht alle beantworten kann und die auch die Grenzen der empirischen Möglichkeiten aufzeigen. Der Punkt des folgenden Abschnitts ist, dass sich deskriptiv drei Klassen ergeben, in die sich die Belege einordnen lassen. Basis dieser Klassenbildung sind ca. 150 Belege, die größtenteils, aber nicht ausschließlich, aus Webdaten stammen (DECOW). Da die Annahme der Belegbarkeit der umgekehrten Abfolge bisher nicht unternommen wurde, gebe ich viele Beispiele an. Ich bin nicht der Meinung, dass \textit{doch ja} in den angegebenen Kontexten akzeptabler ist als \textit{ja doch}. Diese Abfolge wird auch in diesen Kontexten deutlich bevorzugt. Auch ist die umgekehrte Abfolge in Korpora (im Vergleich zu den \textit{ja doch}-Treffern) unterrepräsentiert. Die Belege, die man finden kann, unterliegen aber entscheidenderweise einer Syste\-matik, was sie für mich zu einem interessanten Untersuchungsgegenstand macht. In Abschnitt~\ref{sec:status} diskutiere ich einige Aspekte und Fragen zum Status dieser Abfolge.
  
Ich gehe von den folgenden drei Klassen aus (vgl. auch schon \citealt[175-177]{Mueller2014a}; \citeyear[207-210]{Mueller2017b}): a) \is{Bewertung} Bewertungen, b) \is{epistemische Modalisierung} epistemisch modalisierte Sätze, c) \is{epistemischer Kausalsatz} \is{illokutionärer Kausalsatz} epistemisch bzw. illokutionär interpretierte Kausalsätze. In (\ref{385}) bis (\ref{387}) finden sich einige Beispiele für jede Klasse. Wie ich in Abschnitt~\ref{sec:markiert} erläutern werde, kann man verschiedene konkrete Realisierungen dieser Klassen ausmachen, d.h. das Auftreten verschiedener lexikalischer Mittel kann für die jeweilige Klassenzuordnung verantwortlich sein. 

In die erste Klasse der Bewertungen fallen Äußerungen, mit denen der Sprecher ein qualitatives Urteil über einen Sachverhalt abgibt, in (\ref{385}) bis (\ref{387}) anhand \is{evaluatives Adjektiv} eva\-luativer Adjektive.

\begin{exe}
	\ex\label{385}
	\scriptsize
	 \textbf{Das ist \underline{doch ja} wieder \textit{typisch}.} Ein \glqq Nerd\grqq{} läuft Amok wegen Frust auf Weib \&
 		Lehrer.
 	\newline
	\hbox{}\hfill\hbox{(http://webcache.googleusercontent.com/search?q=ca\_ch4Rz\\lhWolBoJ:identi.ca/notice/}
	\newline
	\hbox{}\hfill\hbox{66906092+\%22doc\_h+ja\%22\&cd=825\&hl=de\&ct=clnk\&gl=de\&source=www.google.-de)}
	\newline
	\hbox{}\hfill\hbox{(Beitrag vom 13.03.2011) (eingesehen über WebasCorpus am 21.09.2011)}						
	\newline
	\hbox{}\hfill\hbox{\citet[176]{Mueller2014a}}	 
\end{exe}

\begin{exe}
	\ex\label{386}
	\scriptsize
	Dann habe ich noch ein Problem. jetzt werden meine Beiträge die ich geschrieben habe auch als neu erkannt. \textbf{Das ist \underline{doch ja} 				\textit{falsch}}, für mich sind sie nicht neu. 
	\hfill\hbox{(DECOW2012-03X: 141373874)}	 
\end{exe}

\begin{exe}
	\ex\label{387}
	\scriptsize
	Ich denke, auch die meisten Frauen merken schon irgendwann rechtzeitig, dass sie selbst weiblich sind und wenn sie dann mal als Mann angesprochen 			werden, ist es wohl auch keine Kränkung. \glqq Sehr geehrte Frau Minister!\grqq{} \textbf{Ist \underline{doch ja} auch ganz \textit{hübsch}.} 
	\hfill\hbox{(DECOW2012-06: 697189512)}	 
	\newline
	\hbox{}\hfill\hbox{\citet[201]{Mueller2014a}}	
\end{exe}
Die zweite Klasse der epistemisch modalisierten Sätze wird durch (\ref{388}) bis (\ref{392}) illustriert. Hier tritt die \textit{doch ja}-Abfolge in Kontexten auf, die beispielsweise durch \is{epistemisches Modalverb} epistemisch interpretierte Modalverben (wie in (\ref{388}) \textit{sollten}, in (\ref{389}) \textit{müssten}), \is{epistemisches Adverb} modalisierende Adverbien (vgl. (\ref{390}), (\ref{391})) oder \is{Tag-Frage} auch Tag-Fragen (vgl. (\ref{392})) charakterisiert sind.
													         	     
\begin{exe}
	\ex\label{388}
	\scriptsize
	Nur zur Vollständigkeit: Was muss beim löschen des Computerkontos im AD denn noch beachtet werden? Einfach danach wieder in die Domäne bringen und 			fertig?\\
	\textbf{Benutzerrechte \textit{sollten} [sich] \underline{doch ja} nicht ändern} – gibt es noch Fallen? 
	\newline
	\hbox{}\hfill\hbox{(http://www.benutzer.de/Neuer\_Client\_unter\_gleichem\_}	
	\newline
	\hbox{}\hfill\hbox{Namen\_wieder\_in\_die\_Dom\%C3\%A4ne\_-\_geht\_das)}	
	\newline
	\hbox{}\hfill\hbox{(Google-Suche, eingesehen am 09.06.2012)}
\end{exe}																   	

\begin{exe}
	\ex\label{389}
	\scriptsize
	Wenn der Overmind schon Psi hat, wieso kann er nicht einen kollektiven Psi-Pool erstellen? \textbf{Die Eigenschaften \textit{müssten} \underline{doch 		ja} vererbbar sein...} Wieso können die Hellen Tepmler dann nicht kollektiv einen Zerebraten angreifen?                                               
	\hfill\hbox{(DECOW2012-05: 1062987308)}	
\end{exe}
	
\begin{exe}
	\ex\label{390}
	\scriptsize
	Wer aber sitzt, sieht kaum noch etwas von den stilvollen Leinwandfotos. Man sollte also einmal versuchen, seine Wandbilder etwas niedriger zu hängen. 		\textbf{Das trifft \underline{doch ja} \textit{eigentlich} auf jegliche Deko zu.} Schließlich stellt man um ein Beispiel zu nennen eine Blumenvase ja 		auch auf keinen Fall auf den Schrank, sondern drapiert sie auf einem schicken Sideboard. 
	\newline
	\hbox{}\hfill\hbox{(http://neueinrichtenjedentag.want2blog.net/2012/07/25/keine}	
	\newline
	\hbox{}\hfill\hbox{-behausung-ohne-wandbilder/, Beitrag vom 11.10.2005)}	
	\newline
	\hbox{}\hfill\hbox{(Google-Suche, eingesehen am 24.07.2012)}
\end{exe}	
	
\begin{exe}
	\ex\label{391}
	\scriptsize
	Oh je wer das wirklich denkt ist Blind...!!
	\textbf{Im VISA Werbespot das ist \underline{doch ja} \textit{nun wirklich} eindeutig ein S2.}
	Die Rahmenbreite stimmt, vordere Cam, Lautsprecher und Schrift stimmen, Powerknopf an der Seite stimmt und die Klinkenbuchse stimmt auch!
	\newline
	\hbox{}\hfill\hbox{(http://www.areamobile.de/b/1546-samsung-galaxy-}	
	\newline
	\hbox{}\hfill\hbox{s3-die-angeblichen-leaks-und-konzepte)}	
	\newline
	\hbox{}\hfill\hbox{(Google-Suche, eingesehen am 24.07.2012)}
\end{exe}						
							               	
\begin{exe}
	\ex\label{392}
	\scriptsize
	Meine alte Grafikkarte hatte ich so ausgebaut, ohne vorher auf irgendwelche Deinstallationen zu achten. Allerdings habe ich dann die Festplatte 			formatiert und ein frisches Windows installiert. \textbf{Demnach habe ich \underline{doch ja} beim Grafikkartenwechsel nichts falsch gemacht, 				\textit{oder??}}
	\newline
	\hbox{}\hfill\hbox{(http://www.drwindows.de/hardware-and-treiber/53 }	
	\newline
	\hbox{}\hfill\hbox{120-treiber-fuer-radeon-hd7850-mit-aero.html)}	
	\newline
	\hbox{}\hfill\hbox{(Google-Suche, eingesehen am 24.07.2012)}
	\newline
	\hbox{}\hfill\hbox{\citet[201]{Mueller2014a}}
\end{exe}		     
Es handelt sich hierbei um sprachliche Elemente, die ein eingeschränktes Sprecherbekenntnis kodieren. Oftmals treten derartige Markierungen auch in Kombination auf, wie in (\ref{393}) z.B. (hier: Adverb \textit{eigentlich} + epistemisches Modalverb \textit{sollte} + MP \textit{wohl} + \is{Diskursmarker} Diskursmarker \textit{denke ich}).
\begin{exe}
	\ex\label{393}
	\scriptsize
	Ich persönlich halte \glqq Alles für eine umfassendere und nicht ausschließende Einstellung. Aber bestätigen kann ich Dir das erst nach vielen Testfahrten ;-). \textbf{\textit{Eigentlich sollte} sich ein solches Gerät bei der Einstellung \glqq Sendersuche: automatisch\grqq{} \underline{doch ja} \textit{wohl} den besten, aber empfangbaren Sender nehmen, so \textit{denke ich}.}
	\hfill\hbox{(http://www.gopal-navigator.de/archive/index.php/t-5689.html)                                                                      }
	\newline
	\hbox{}\hfill\hbox{120-treiber-fuer-radeon-hd7850-mit-aero.html)}	
	\newline
	\hbox{}\hfill\hbox{(Google-Suche, eingesehen am 21.07.2014)}
	\newline
	\hbox{}\hfill\hbox{\citet[208]{Mueller2017b}}
\end{exe}
Die dritte Klasse konstituiert sich durch Kausalsätze, die keine Ursache-Wirkung-Relation ausdrücken, sondern erklären, warum der Sprecher eine bestimmte Annahme macht (bzw. allgemeiner eine bestimmte Einstellung vertritt) oder warum er einen bestimmten Sprechakt ausführt. 

In (\ref{394}) begründet der \textit{denn}-Satz die Einschätzung aus dem ersten Teil des Satzes, der zudem epistemisch modalisiert ist (\textit{dürfte}), wie es für Hauptsätze, auf die sich epistemische Kausalsätze beziehen, typisch ist. 
        
\begin{exe}
	\ex\label{394}
	\scriptsize
	Bei seiner letzen Sendung am 21.01.09 schalten wieder 10,63 Millionen TV  Zuschauer eine Sendung ein. Zur gleichen Zeit lief bei RTL \glqq Ich bin ein 	Star (Rentnerin \glqq Ingrid van Bergen\grqq{}) holt mich hier raus\grqq{}. \emph{Am 28.02.09 dürfte er dann noch einige junge Zuschauer weniger haben}, 		\textbf{\textit{denn} diese möchten \underline{doch ja} garantiert DSDS sehen}, da geht es ja um \glqq Alles oder Nichts\grqq{} bei den 15 DSDS 				Sternchen. 	
	\newline
	\hbox{}\hfill\hbox{(http://www.deutschlands-superstar.de/2009/02/27/                                                                                     	wettendass-gegen-dsds-die-entscheidung/)}	
	\newline
	\hbox{}\hfill\hbox{(eingesehen am 09.06.2011)}
	\newline
	\hbox{}\hfill\hbox{\citet[176]{Mueller2014a}}
\end{exe}        
Ähnlich begründet der Sprecher in (\ref{395}) seine Annahme, dass für die Liga kein Problem entstünde.

\begin{exe}
	\ex\label{395}
	\scriptsize
	Wenn Mannheim jetzt in die OL durchmüsste... \emph{wär das doch für die RL kein Problem...} \textbf{dann kann die \underline{doch ja} auch wieder mit 		18 Teams spielen}... je nachdem was mit den Abstiegern geregelt wär... ach warum nur alles so kompliziert!? 	
	\hfill\hbox{(DECOW2012-01: 956951068)}
\end{exe} 
(\ref{396}) und (\ref{397}) zeigen Beispiele für illokutionär \is{illokutionärer Kausalsatz} verwendete Kausalsätze.
 
\begin{exe}
	\ex\label{396}
	\scriptsize
	Für das Anfang Januar durch ein Feuer stark beschädigte CapaHaus in der Jahnallee 61 besteht immer noch akute Einsturzgefahr. [...] 
	Um das Capa-Haus schlussendlich zu retten, müsste es umfassend saniert werden, unter anderem durch die Abdichtung des Daches und die Instandsetzung 		der durch das Feuer beschädigten Teile. Die Veranlassung dieser Arbeiten obliegt aber dem Eigentümer, so das Amt weiter.\\
	Gepostet: 01.02.2012 12:31 anonym\\
	\emph{einfach abreißen!} 
	\textbf{\textit{denn} es fehlen \underline{doch ja} schon sämtliche zwischen decken} [...] 		
	\newline
	\hbox{}\hfill\hbox{(http://www.leipzigfernsehen.de/default.aspx?ID=5846\&showNews=}	
	\newline
	\hbox{}\hfill\hbox{1107962, Beitrag vom 01.02.2012) (eingesehen am 24.07.2012)}
	\newline
	\hbox{}\hfill\hbox{\citet[176]{Mueller2014a}}
\end{exe}        

\begin{exe}
	\ex\label{397}
	\scriptsize
	\emph{zu \glqq wie vielt\grqq{} müssen wir sein, damits als guild run zählt}, \textbf{\textit{weil} mc \underline{doch ja} solo machbar wär}^^ -– aber 	nichtsdesdotrotz bin ich dabei -– wird sicher n 	spaß!
	\newline
	\hbox{}\hfill\hbox{(http://webcache.googleusercontent.com/search?q=cache:ta5}	
	\newline
	\hbox{}\hfill\hbox{\%3Ff\%3D63\%26p\%3D29269+\%22doch+ja\%22\&cd=440\&}
	\newline
	\hbox{}\hfill\hbox{hl=de\&ct=clnk\&gl=de\&source=www.google.de)}
	\newline
	\hbox{}\hfill\hbox{(Google-Suche, eingesehen am 09.06.2012)}
\end{exe} 

In (\ref{396}) macht der Sprecher den Vorschlag, das Gebäude abzureißen und gibt dann an, warum er diesen Vorschlag macht: Die Zwischendecken fehlen schon. Hier wird folglich ein direktiver Sprechakt begründet. In (\ref{397}) handelt es sich bei dem zu begründenden Sprechakt um eine Frage: Die Aussage, dass das Beispiel allein machbar sei, ist das Motiv für die Frage, wie viele Spieler benötigt werden.\\

\noindent	
Ich nehme an, dass es nicht der Fall ist, dass \textit{ja doch} die grammatische Abfolge und \textit{doch ja} die ungrammatische Reihung ist, wie in den in Abschnitt~\ref{sec:abfolgejd} skizzier\-ten Ansätzen und überhaupt allen mir bekannten Arbeiten vertreten wird, sondern dass \textit{ja doch} die \underline{unmarkierte} \is{Markiertheit} und \textit{doch ja} die \underline{markierte} Abfolge ist. Letztere tritt deshalb seltener auf und ist vor allem auf bestimmte Kontexte beschränkt, die sich präzise benennen und auf der Basis von kookurrierendem sprachlichen Material charakterisieren lassen. Ich nehme an, dass \textit{ja doch} in jedem Kontext auftreten kann, in dem auch \textit{doch ja} auftritt. Allerdings kann \textit{doch ja} nicht in jedem Kontext auftreten wie \textit{ja doch}. Dies zeigen die Beispiele aus der Literatur in (\ref{376}) bis (\ref{378}) (hier wiederholt in (\ref{398}) bis (\ref{400})), deren Beurteilung ich teile.

\begin{exe}
	\ex\label{398} 
	Konrad ist \textbf{ja doch}/*\textbf{doch ja} verreist.
	\hfill\hbox {\citet[114]{Doherty1987}}
\end{exe}
\vspace{-0.5cm}	
\begin{exe}
	\ex\label{399} 
	Er hat \textbf{ja doch}/??\textbf{doch ja} getanzt.
	\hfill\hbox {\citet[20]{Struckmeier2014}}
\end{exe}
\vspace{-0.5cm}	
\begin{exe}
	\ex\label{400} 
	Er hat sich \textbf{ja doch}/?\textbf{doch ja} sehr um sie bemüht.
	\hfill\hbox {\citet[157]{Jacobs1991}}
\end{exe}
Die zwei Fragen, die ich im Folgenden auf der Basis dieser Ausgangslage beantworten möchte, sind: a) Warum ist \textit{ja doch} die unmarkierte Abfolge, die in allen assertiven Kontexten auftreten kann? b) Warum lässt sich die Abfolge in den drei charakterisierten Kontexten umkehren?
	
\section{Der interpretatorische Beitrag von \textit{ja}, \textit{doch} sowie ihrer Kombination}		
\label{sec:interpretationjd}			
Die im Folgenden beschriebene Analyse (vgl. auch \citealt[183-197]{Mueller2014a}; \citeyear[214-223]{Mueller2017b}) fußt zum einen auf dem Effekt, den MP-lose Assertionen auf den Diskurs\-kontext nehmen und zum anderen auf einer Modellierung des kontextuellen Effektes von \textit{ja} und \textit{doch} in Isolation, auf den ich die Interpretation der Kombination aufbauen werde. In Abschnitt~\ref{sec:diskursmodell}, Kapitel~\ref{chapter:hintergrund} wird das zugrunde gelegte formale Diskursmodell nach \citet{Farkas2010} ausführlich eingeführt. Der besseren Lesbarkeit halber seien an dieser Stelle die für die weitere Argumentation relevanten Aspekte des Modells bzw. der MP-Auffassung erneut angeführt (vgl. die Abschnitte~\ref{sec:mplass} und \ref{sec:mpn}), bevor sich Abschnitt~\ref{sec:inkdm} mit dem Bedeutungseffekt von \textit{ja}- und \textit{doch}-Äußerungen beschäftigt und in Abschnitt~\ref{sec:kombi} das kombinierte Auftreten betrachtet wird.
								
\subsection{Modalpartikellose Assertionen}
\label{sec:mplass}
Wird eine Assertion wie beispielsweise in (\ref{402}) im Diskurs geäußert, führt dies nach \citet{Farkas2010} vor dem Hintergrund des Ausgangszustands K$_{1}$ (vgl. (\ref{401})) die im Folgenden beschriebenen Kontextveränderungen mit sich.
     
\newcolumntype{C}[1]{>{\centering}p{#1}} 
\begin{exe}
\ex\label{401} K$_1$: initialer Kontextzustand\\[-0.6em]
\begin{tabular}[t]{| C{6em}|p{6em}|p{6em}|  C{6em}|}
\hline
 $\textrm{DC}_{\textrm{A}}$ & \multicolumn{2}{C{12em}|}{Tisch} &  $\textrm{DC}_{\textrm{B}}$ \tabularnewline
\hline
{} & \multicolumn{2}{p{12em}|}{} & {}  \tabularnewline
\hline
\multicolumn{2}{|p{12em}|}{cg $\textrm{s}_{1}$}&\multicolumn{2}{|p{12em}|}{$\textrm{ps}_{1} = \lbrace \textrm{s}_{1} \rbrace$} \tabularnewline
\hline
\end{tabular}
\end{exe}		                              
								  
\begin{exe}
	\ex\label{402} 
	A: Die Spieler der 2. Geige sind die wahren Geiger.
\end{exe}									         						   
Die ausgedrückte Proposition p wird dem Diskursbekenntnissystem \is{Diskursbekenntnissystem (discourse commitment set)} (\textit{discourse commitment set}) von Diskursteilnehmer A (DC$_{\textrm{A}}$) hinzugefügt. Die Proposition wird samt ihrer Negation auf den Tisch \is{Tisch} obenauf gelegt. Der cg-Zustand bleibt im Vergleich zu K$_{1}$ unverändert. Die projizierte Zukunft des cg, das \is{Projektionsmenge (projected set)} \textit{projected set} (ps), wird um p erweitert, da die kanonische Reaktion auf eine Assertion die Bestätigung dieser Assertion ist.

\newcolumntype{C}[1]{>{\centering}p{#1}} 
\begin{exe}
\ex\label{403} K$_{2}$: A hat assertiert: \textit{Die Spieler der 2. Geige sind die wahren Geiger.} relativ zu K$_{1}$\\[-0.6em]
\begin{tabular}[t]{| C{6em}|p{6em}|p{6em}|  C{6em}|}
\hline
 $\textrm{DC}_{\textrm{A}}$ & \multicolumn{2}{C{12em}|}{Tisch} &  $\textrm{DC}_{\textrm{B}}$ \tabularnewline
\hline
p & \multicolumn{2}{C{12em}|}{p $\vee$ $\neg$p} & {}  \tabularnewline
\hline
\multicolumn{2}{|p{12em}|}{cg $\textrm{s}_{2}$ = $\textrm{s}_{1}$}&\multicolumn{2}{|p{12em}|}{$\textrm{ps}_{2} = \lbrace \textrm{s}_{1} \cup \lbrace \textrm{p} \rbrace \rbrace$} \tabularnewline
\hline
\end{tabular}
\end{exe}		
Wenn Diskursteilnehmer B den Inhalt der Äußerung in (\ref{402}) annimmt/bestä\-tigt/ak\-zeptiert, wird p ebenfalls zu einem Diskursbekenntnis von Gesprächs\-teilnehmer B (vgl. (\ref{404a})). Als bewusst geteiltes öffentliches Diskursbekenntnis wird p zu einem cg-Inhalt und das Thema wird (da p $\vee$ $\neg$p nun nicht mehr zur Diskussion steht) vom Tisch geräumt (vgl. (\ref{404b})).


\newcolumntype{C}[1]{>{\centering}p{#1}} 
\begin{exe}
\ex\label{404} K$_{3}$: B bestätigt As Beitrag \\[-1em]
\begin{xlist}	
			\ex\label{404a} 
			Teil 1\\[-1em]
			\begin{tabular}[t]{| C{6em}|p{6em}|p{6em}|  C{6em}|}
			\hline
 			$\textrm{DC}_{\textrm{A}}$ & \multicolumn{2}{C{12em}|}{Tisch} &  $\textrm{DC}_{\textrm{B}}$ \tabularnewline
			\hline
			p & \multicolumn{2}{C{12em}|}{p $\vee$ $\neg$ p} & {p}  \tabularnewline
			\hline
			\multicolumn{2}{|p{12em}|}{cg $\textrm{s}_{3}$ = $\textrm{s}_{1}$}&\multicolumn{2}{|p{12em}|}{$\textrm{ps}_{3} = \textrm{ps}_{2}$} 							\tabularnewline
			\hline
			\end{tabular}
\end{xlist}

\begin{xlist}	
			\ex\label{404b} 
			Teil 2\\[-1em]
			\begin{tabular}[t]{| C{6em}|p{6em}|p{6em}|  C{6em}|}
			\hline
 			$\textrm{DC}_{\textrm{A}}$ & \multicolumn{2}{C{12em}|}{Tisch} &  $\textrm{DC}_{\textrm{B}}$ \tabularnewline
			\hline
			{} & \multicolumn{2}{C{12em}|}{} & {}  \tabularnewline
			\hline
			\multicolumn{2}{|p{12em}|}{cg $\textrm{s}_{4} = \lbrace \textrm{s}_{1} \cup \lbrace \textrm{p} \rbrace \rbrace$}&\multicolumn{2}{|p{12em}|}{$				\textrm{ps}_{4} = \lbrace \textrm{s}_{4} \rbrace$} \tabularnewline
			\hline
			\end{tabular}
\end{xlist}
\end{exe}
Entscheidend für meine Antwort auf die zwei Fragen vom Anfang ist Farkas \& Bruce' Überlegung zu kanonischen Verhaltensweisen in der Konversation (vgl. auch hierzu ausführlicher Abschnitt~\ref{sec:diskursmodell} in Kapitel~\ref{chapter:hintergrund}). Prinzipiell wird Konversation ihrer Ansicht nach durch die zwei Aspekte in (\ref{405}) getrieben (vgl. \citealt [87]{Farkas2010}).

\begin{exe}
	\ex\label{405} 
		Zwei fundamentale Antriebe für Gespräche
		\begin{xlist}	
			\ex\label{405a} Erweiterung des cg
			\ex\label{405b} Herstellung eines stabilen Kontextzustands
		\end{xlist}
\end{exe}
Gesprächsteilnehmer folgen dem Bedürfnis, den cg zu erweitern. Aus diesem Grund legen sie Elemente auf dem Tisch ab. Ferner streben sie danach, einen Kontextzustand zu schaffen, in dem sich kein Element auf dem Tisch befindet.

Neben diesen allgemeinen Diskursbestreben formulieren die Autoren zudem kanoni\-sche Reaktionen, die mit einzelnen Sprechakten verbunden sind. Die kanonische Reaktion auf Assertionen ist die Bestätigung der Assertion. In dem Sinne, dass der Sprecher der Assertion ein öffentliches Diskursbekenntnis zu ihrem Inhalt macht und die enthaltene Proposition mit ihrer Alternative auf den Tisch gelegt wird, machen Assertionen Vorschläge. Durch die Annahme dieses Vor\-schlags, d.h. die Akzeptanz/Zustimmung durch den Gesprächsteilnehmer, wird diese Proposition auf direktem Wege zu einem cg-Inhalt. Diese Überlegung lässt sich auffassen als Vorliegen einer Art von konversationellem Druck, den cg anzurei\-chern, indem veröffentlichte Bekenntnisse zu geteilten Bekenntnissen gemacht werden. Um dies zu erfüllen, muss der Gesprächspartner den Inhalt annehmen. In diesem Sinne hat eine Assertion eine Voreingenommenheit zugunsten der ausgedrückten Proposition. Vor dem Hintergrund dieser Überlegungen formulieren die Autoren Kriterien, die von prototypischen und weniger prototypischen Assertionen \is{prototypische Assertion} in unterschiedlichem Maße erfüllt werden. (\ref{406}) zeigt die formalisierte Version der Autoren, (\ref{407}) eine Paraphrasierung dieser Darstellung.

\begin{exe}
	\ex\label{406} 
		S($[\textrm{D}], \textrm{a}, \textrm{K}_{\textrm{i}}) = \textrm{K}_{\textrm{o}}$ so that
		\begin{xlist}	
			\ex\label{406a} $\textrm{DC}_{\textrm{a,o}} = \textrm{DC}_{\textrm{a, i}} \cup \lbrace\textrm{p}\rbrace$
			\ex\label{406b} $\textrm{T}_{\textrm{o}} = \textrm{push}(\langle \textrm{S}[\textrm{D}]; \lbrace \textrm{p} \rbrace \rangle, \textrm{T}_{\textrm{i}})$
			\ex\label{406c} $\textrm{ps}_{\textrm{o}} =  \textrm{ps}_{\textrm{i}} \ \overline\cup \ \lbrace \textrm{p} \rbrace$
			\hfill\hbox {\citet[92]{Farkas2010}}
		\end{xlist}
\end{exe}

\begin{exe}
	\ex\label{407} 
	Nachdem ein Sprecher eine Assertion mit Proposition p geäußert hat, gilt:
		\begin{xlist}	
			\ex\label{407a} Die neue Diskursbekenntnismenge des Sprechers beinhaltet p.
			\ex\label{407b} Die Proposition p (vs. $\neg$p) wird oben auf den Stapel des alten Tisches gelegt.
			\ex\label{407c} Die projizierte Zukunft des alten cg beinhaltet p (unter Bewahrung der Konsistenz des cg).
		\end{xlist}
\end{exe}
Wichtig für meine Ableitung der Markiertheit von \textit{doch ja} und Unmarkiertheit von \textit{ja doch} ist, dass der Effekt in (\ref{407a}) auf alle Assertionen zutrifft, d.h. alle Äußerungen, die man als assertiv einstuft, haben den Effekt im Kontext, dass die Diskursbekenntnismenge des Autors der Assertion mit der ausgedrückten Proposition angereichert wird. Dieser Effekt trifft auf prototypische Assertionen (s.u.) genauso zu wie auf weniger prototypische und auf selbständige Sätze genauso wie auf assertive Nebensätze. Eine als Standardassertion oder mit anderen Worten \textit{prototypische Assertion} eingestufte Äußerung muss hingegen alle drei Kriterien erfüllen: Der Sprecher der Assertion bekennt sich zu ihrem propositionalen Inhalt, die Assertion eröffnet ein Thema, indem sie ein Element auf den Tisch legt und sie lenkt die Konversation in die Richtung einer Auflösung des Themas durch die Bestätigung der ausgedrückten Proposition. Eine kano\-nische Assertion führt zu einem Kontext, der kategorisch voreingenommen ist hinsichtlich einer Bestätigung der assertierten Proposition. 

Ein Aspekt, den ich in der Beschreibung des Diskursmodells in Abschnitt~\ref{sec:diskursmodell} ausgelassen habe, betrifft die Art von Information, die auf dem Tisch liegt und somit im Diskurs zur Diskussion stehen kann. In den bisher angeführten Beispielen handelte es sich um wörtliche Bedeutungsanteile, die mitgeteilt und zum Thema des Gesprächs werden. Assertionen und andere Sprechakte sind bekannt\-lich nicht nur mit wörtlichem Inhalt assoziiert, sondern beispielsweise auch mit implikatiertem. \citet[94]{Farkas2010} nehmen an, dass auch dieser auf dem Tisch abgelegt werden kann. Weitere Diskursbeiträge können sowohl auf wörtlichen als auch implikatierten Inhalt Bezug nehmen.\footnote{\citet[94]{Farkas2010} machen für die Zwecke des von ihnen untersuchten Phänomens einen Unterschied hinsichtlich der Art, mit der derartige Inhalte auf dem Tisch liegen. Im\-plikatierter Inhalt sei zwar verbunden mit, aber getrennt von wörtlichem Inhalt. Damit meinen die Autoren, dass nur der wörtliche Inhalt mit syntaktischem Material auf dem Tisch verbunden ist. Da die syntaktische Information für die von mir untersuchten Strukturen nicht relevant ist, ist es für meine Argumentation unerheblich, ob implikatierte Inhalte auf die gleiche oder eine andere Art auf dem Tisch verfügbar sind wie wörtliche Inhalte.} Dass implikatierte Inhalte auf dem Tisch liegen können, lässt sich leicht dadurch zeigen, dass Implikaturen \is{Implikatur} thematisiert und in Frage gestellt werden können (vgl. (\ref{408})).

\begin{exe}
	\ex\label{408} 
	A: Peter hat Maria mit einem Mann in der Stadt gesehen.\\
	B: Aha. Interessant. Vor allem, dass sie nicht mit ihrem Ehemann in die Stadt geht.\\
	A: Oh nein nein, es war schon ihr Ehemann.
\end{exe}
In (\ref{408}) hat die Äußerung von Sprecher A (mit p = dass Peter Maria mit einem Mann in der Stadt gesehen hat) die Implikatur $\neq$q (= dass Peter Maria nicht mit ihrem Ehemann in der Stadt gesehen hat). Genau diesen Inhalt kann B thematisieren und somit gelangt über p auch $\neg$q auf den Tisch und q $\vee$ $\neg$q kann ein potenzielles Thema sein.\footnote{\label{Fn16}Eine offene Frage ist, ob man annehmen möchte, dass Sprecher sich auch stets zu den Implikaturen ihrer Äußerungen bekennen, d.h. ob A in diesem Fall auch ein Bekenntnis zu $\neg$q hat. Ich halte ein Bekenntnis zu implikatierten Inhalten nicht für plausibel und zwar insbesondere aus dem Grund, dass sich die Frage zuspitzt, je pragmatischer (d.h. auch kontextgebundener) derartige Inferenzen werden. Die gleiche Frage kann man stellen zu Präsuppositionen und konventionellen Implikaturen. Wenn ein öffentliches Diskursbekenntnis auch beinhaltet, dass ein Sprecher sich dieser gegebenen Information bewusst ist, möchte ich doch in Frage stellen, dass ein Sprecher sich aller Schlüsse seiner Äußerungen bewusst ist. Die Antwort auf diese Frage ist an dieser Stelle aber auch nicht entscheidend, anders als die unkontroverse Annahme, dass eine derartige implikatierte (oder auch präsupponierte) Information ebenfalls auf dem Tisch liegen kann und die Gesprächsteilnehmer auf diese reagieren können.}
	
\begin{exe}
	\ex\label{409}
     \begin{tabular}[t]{ll}
     		A: & Peter hat Maria mit einem Mann in der Stadt gesehen.\\
            DC$_{\textrm{A}}$: & p\\
            Tisch: & p $\vee$ $\neg$p\\
 			 & q $\vee$ $\neg$q\\
 			DC$_{\textrm{B}}$: & --
      \end{tabular}
\end{exe}	
Denn Sprecher B kann auch auf diese implikatierte \is{Implikatur} Information reagieren. Nach Bs Äußerung in (\ref{408}) liegt ein Bekenntnis von B zu p vor sowie zu $\neg$q. Da beide Diskursteilnehmer ein Bekenntnis zu p haben, wird diese Proposition zu einem cg-Inhalt. Sprecher A streitet $\neg$q ab, weshalb das Thema q vs. $\neg$q unentschieden auf dem Tisch zurückbleibt. 

\begin{exe}
	\ex\label{410}
     \begin{tabular}[t]{lp{20em}}
     		B: & Aha. Interessant. Vor allem, dass sie nicht mit ihrem Ehe-\\
     		& mann in die Stadt geht.\\
            DC$_{\textrm{A}}$: & p\\
            Tisch: & q $\vee$ $\neg$q\\
 			DC$_{\textrm{B}}$: & p, $\neg$q\\
 			cg = & $\lbrace \textrm{p} \rbrace$
      \end{tabular}
\end{exe}	
Um den Effekt von MP-Äußerungen auf den Kontext auf die gleiche Art wie den Diskursbeitrag von MP-losen Assertionen zu be\-schreiben, werde ich im Folgenden die MP-Auffassung und -Modellierung nach Arbeiten von Diewald in das Diskursmodell nach Farkas \& Bruce integrieren.

\subsection{Die rückverweisende Funktion von Modalpartikeln}
\label{sec:mpn}
Wie in Abschnitt~\ref{sec:zugang} in Kapitel~\ref{chapter:hintergrund} ausführlich beschrieben, folgt Diewald in einer Reihe von Arbeiten der generellen Auffassung, dass MPn wie andere grammatische Ka\-tegorien eine relationale Komponente aufweisen: Sie nehmen Bezug auf pragmatisch präsupponierte Einheiten im Vorgängerkontext der tatsächlichen MP-Äußerung (vgl. \citealt[414-415]{Diewald2006}). Die genau vorliegende Relation unterscheidet sich je nach MP. Als Operationalisierung dieser grundsätzlichen Funktion verwendet Diewald ein dreistufiges Beschreibungsmodell. 

Für \textit{ja} und \textit{doch} nehmen \citet{Diewald1998} die folgenden konkreten Relationen an. Als Beispiel dient die \textit{doch}-Äußerung in (\ref{411}).

\begin{exe}
	\ex\label{411} 
	Das war \textbf{doch} richtig.
\end{exe}
\vspace{-0.65cm}	
\begin{exe}
	\ex\label{412} Bedeutungsschema der MP \textit{doch}\\[-0.6em]
     \begin{tabular}[t]{|l|p{7cm}|}
     	\hline
      	pragmatischer Prätext & im Raum steht: ob das richtig war\\
        \hline
        relevante Situation & ich denke: das war richtig\\
        \hline
        $\rightarrow$ Äußerung & Das war \textbf{doch} richtig.\\
        \hline
     \end{tabular}\\
     \hbox{}\hfill\hbox{\citet[92]{Diewald1998}}
\end{exe}
Einer \textit{doch}-Äußerung geht eine Situation voran, in der die Frage offen ist, ob p gilt, d.h. hier ist möglich \glq es war richtig\grq {} oder \glq es war nicht richtig\grq {}. Vor dem Hintergrund dieser zwei im Kontext bestehenden Alternativen vertritt der Sprecher eine der beiden Möglichkeiten. Die Partikel \textit{doch} indiziert hier \\ letztlich eine konzessive Relation: Der Sprecher entscheidet sich für die in seiner Äußerung ausgedrückte Proposition, obwohl die gegenteilige Annahme ebenfalls kontextuell aktiviert ist.

 \begin{exe}
	\ex\label{413} 	
	Es soll \textbf{ja} fliegen.
	\hfill\hbox {\citet[93-94]{Diewald1998}}
\end{exe}
\vspace{-0.65cm}	
\begin{exe}
	\ex\label{414} Bedeutungsschema der MP \textit{ja}\\[-0.6em]
     \begin{tabular}[t]{|l|p{7cm}|}
     	\hline
      	pragmatischer Prätext & im Raum steht: jmd. denkt, dass es fliegen soll\\
        \hline
        relevante Situation & ich denke: es soll fliegen\\
        \hline
        $\rightarrow$ Äußerung & Es soll \textbf{ja} fliegen.\\
        \hline
     \end{tabular}\\
     \hbox{}\hfill\hbox{\citet[84]{Diewald1998}}
\end{exe}
Für eine \textit{ja}-Äußerung wie in (\ref{413}) setzen die Autoren an, dass im Kontext vor der MP-Äußerung jemand anders (in der Regel der Angesprochene) genau die Proposition vertritt, die in der (kommenden) \textit{ja}-Assertion enthalten ist. Der Sprecher schließt sich somit einer bereits bestehenden Annahme an.

\subsection{Das Einzelauftreten von \textit{ja} und \textit{doch}}
\label{sec:inkdm}
Um im Rahmen des Modells von Farkas \& Bruce auch MP-Äußerungen zu beschrei\-ben, gebe ich im Folgenden zwei Kontextzustände an: die Beschaffenheit des Kontextes vor der MP-Äußerung und nach der MP-Äußerung.\footnote{Der Übersichtlichkeit halber verzichte ich auf die Füllung des \textit{projected set}.}

\subsubsection{\textit{doch}}
\label{sec:doch1}
Für eine \textit{doch}-Äußerung nehme ich an, dass im Kontext vor der MP-Äußerung die Bipartition p $\vee$ $\neg$p auf dem Tisch liegt (entspricht bei Diewald: es steht im Raum, ob p) (vgl. (\ref{415})). Der cg ist unverändert im Vergleich zum Vorgängerkontext.

\begin{exe}
	\ex\label{415} Kontext vor der \textit{doch}-Äußerung\\[-1em]	
 \begin{tabular}[t]{|C{6em}|C{6em}|C{6em}|} 
 \hline 	
   $\textrm{DC}_{\textrm{A}}$ & {Tisch} & $\textrm{DC}_{\textrm{B}}$ \tabularnewline
  \hline
    & p $\vee$ $\neg$p & \tabularnewline
  \hline      
   \multicolumn{3}{|l|}{cg s$_{1}$} \tabularnewline 
   \hline
 \end{tabular}
\end{exe}
Im Kontext nach der Äußerung einer \textit{doch}-Assertion liegt (durch die Assertion) ein Sprecherbekenntnis zu p vor, das der Sprecher vor dem Hintergrund macht, dass zwei Möglichkeiten offen sind und somit auch das gegenteilige Bekenntnis denkbar wäre (entspricht bei Diewald: \textit{ich denke p, obwohl die beiden Alternativen offen sind}). Der cg bleibt unverändert.

\begin{exe}
	\ex\label{416} Kontext nach der \textit{doch}-Äußerung\\[-1em]	
 \begin{tabular}[t]{|C{6em}|C{6em}|C{6em}|} 
 \hline 	
   $\textrm{DC}_{\textrm{A}}$ & {Tisch} & $\textrm{DC}_{\textrm{B}}$ \tabularnewline
  \hline
    p & p $\vee$ $\neg$p & \tabularnewline
  \hline      
   \multicolumn{3}{|l|}{cg s$_{2}$ = s$_{1}$} \tabularnewline 
   \hline
 \end{tabular}
\end{exe}
Ein wichtiger Punkt ist an dieser Stelle, dass die Modellierungen in (\ref{415}) und (\ref{416}) Minimalanforderungen beschreiben. In konkreten Dialogen kann es durchaus so sein, dass ggf. auch die anderen Systeme \underline{zusätzlich} gefüllt sind. Die Charak\-terisierungen beabsichtigen nicht, jegliche denkbaren Szenarien abzudecken. Auch können mit der Füllung der Komponenten im konkreten Kontext weitere pragmatische Effekte verbunden sein (z.B. direkter Widerspruch, Erinnerung). Meine Hypothese hinsichtlich der Modellierung des Diskursbeitrags der beiden betrachte\-ten MPn ist allein, dass die Komponenten mit diesen Inhalten minimal beteiligt sind. Man hat es hier folglich (wie auch bei Diewalds Beschreibung) mit einer \textit{bedeutungsminimalistischen} (vs. -\textit{maximalistischen}) Auffassung \is{Bedeutungsminimalismus/-maximalismus} zu tun.

Eine abstrakte Bedeutungszuschreibung, wie ich sie vorschlage, bleibt natürlich den Nachweis schuldig, dass sie den Beitrag der MP-Äußerungen in konkreten Fällen erfassen kann.

Typische Beispiele für eine \textit{doch}-Assertion finden sich in (\ref{417}) bis (\ref{419}).

\begin{exe}
	\ex\label{417} 
	A: Patrick ist nicht zu Hause.\\
	B: Aber sein Auto ist \textbf{doch} da.	
	\hfill\hbox {\citet[83]{Ormelius-Sandblom1997}}
\end{exe}

\begin{exe}
	\ex\label{418} 
	A: Peter wird auch mitkommen.\\
	B: Er ist \textbf{doch} krank.
	\hfill\hbox {\citet[126]{Egg2013}}
\end{exe}

\begin{exe}
	\ex\label{419} 
	Ich habe wieder Schnupfen. Dabei lebe ich \textbf{doch} ganz vernünftig.
	\newline
	\hbox{}\hfill\hbox{\citet[84]{Dahl1988}}	
\end{exe}
Die \textit{doch}-Äußerung bezieht sich hier in gewissem Sinne auf Konsequenzen aus der ersten Äußerung. 

\begin{exe}
	\ex\label{420} 
	Patrick ist nicht zu Hause. ($\neg$p) $>$ Patricks Auto ist nicht da. ($\neg$q)\\
	Wenn Patrick nicht zu Hause ist, ist sein Auto normalerweise nicht da. 	
\end{exe}

\begin{exe}
	\ex\label{421} 
	Peter wird auch mitkommen. $>$ Peter ist nicht krank.\\
	Wenn eine Person mitkommt, ist sie normalerweise nicht krank. 
\end{exe}		
		 
\begin{exe}
	\ex\label{422} 
	Ich habe wieder Schnupfen. $>$ Ich lebe nicht vernünftig.\\
	Wenn eine Person wiederholt erkältet ist, lebt sie normalerweise nicht vernünftig.  	
\end{exe}
Die beiden beteiligten Propositionen stehen sicherlich nicht in der Relation einer logischen Folgerung. Ich schließe mich Ormelius-Sandbloms Annahme an, dass ein pragmatischer Schluss beteiligt ist, der die zwei Propositionen hier in einen plausiblen Zusammenhang bringt und auf Hintergrund- oder Weltwissen basiert. Da der abgeleitete Inhalt in diesem Fall leicht tilgbar oder verstärkbar ist und Wissen um Patricks Fortbewegungsvorlieben vonnöten sind, handelt es sich vermutlich um eine \is{konversationelle Implikatur} konversationelle Implikatur. Der Bezug der \textit{doch}-Äußerung auf die Implikatur der ersten Äußerung lässt sich mit der in Abschnitt~\ref{sec:mplass} vorgeschlagenen Modellierung gut erfassen, da - wie wir gesehen haben - auch implikatierte Inhalte auf dem Tisch liegen können.

\begin{exe}
	\ex\label{423} Kontext nach As Äußerung\\
	A: Patrick ist nicht zu Hause. (= $\neg$p) $>$ Patricks Auto ist nicht da. (= $\neg$q)\\[-1em]	
 \begin{tabular}[t]{|C{6em}|C{6em}|C{6em}|} 
 \hline 	
   $\textrm{DC}_{\textrm{A}}$ & {Tisch} & $\textrm{DC}_{\textrm{B}}$ \tabularnewline
  \hline
    $\neg$p & p $\vee$ $\neg$p & \tabularnewline
    {} & q $\vee$ $\neg$q & \tabularnewline
  \hline      
   \multicolumn{3}{|l|}{cg s$_{1}$} \tabularnewline   
   \hline
 \end{tabular}
\end{exe}
In (\ref{423}) hat Sprecher A ein Bekenntnis zu $\neg$p, so dass auf dem Tisch die Frage eröffnet wird, ob p $\vee$ $\neg$p. Da aus $\neg$p $\neg$q ableitbar ist, öffnet sich auf dem Tisch auch die Frage q $\vee$ $\neg$q. Der cg ist nicht beeinflusst. B reagiert nun mit seinem öffentlichen Bekenntnis zu q auf die offene Frage q $\vee$ $\neg$q auf dem Tisch  (vgl. (\ref{424})).

\begin{exe}
	\ex\label{424} Kontext nach Bs Äußerung\\
	B: Aber sein Auto ist \textbf{doch} da. (= q)\\[-1em]	
 \begin{tabular}[t]{|C{6em}|C{6em}|C{6em}|} 
 \hline 	
   $\textrm{DC}_{\textrm{A}}$ & {Tisch} & $\textrm{DC}_{\textrm{B}}$ \tabularnewline
  \hline
    $\neg$p & p $\vee$ $\neg$p & \tabularnewline
    {} & q $\vee$ $\neg$q & q\tabularnewline
  \hline      
   \multicolumn{3}{|l|}{cg s$_{2}$ = s$_{1}$} \tabularnewline   
   \hline
 \end{tabular}
\end{exe}
Eine \textit{doch}-Äußerung kann sich auch auf Glückensbedingungen \is{Glückensbedingung} eines vorangehenden Sprechaktes \is{Sprechakt} beziehen.
\begin{exe}
	\ex\label{425} 
	A: Übersetze mir bitte diesen Brief.\\
	B: Ich kann \textbf{doch} kein Englisch. ($\neg$q)
\end{exe}

\begin{exe}
	\ex\label{426} 
	A: Bestell dir die Schweinshaxe.\\
	B: Ich bin \textbf{doch} Vegetarier.(q)
	\hfill\hbox {\citet[133]{Egg2013}}
\end{exe}
Die \textit{doch}-Äußerung thematisiert hier die 1. Einleitungsregel für Aufforderungen bzw. Raten.

\begin{exe}
	\ex\label{427} 
	1. Einleitungsregel Aufforderung \is{Aufforderung}\\
	\glqq H ist in der Lage, A zu tun. S glaubt, daß H in der Lage ist, A zu tun.\grqq{}
\end{exe}

\begin{exe}
	\ex\label{428} 
	1. Einleitungsregel Raten \is{Rat}\\
	\glqq S hat Grund zu glauben, daß A H nützen wird.\grqq{}
	\hfill\hbox {\citet[100/104]{Searle1971}}
\end{exe}		  
Nimmt man an, dass nach As Aufforderung bzw. Rat q $\vee$ $\neg$q auf dem Tisch liegt, bezieht sich B mit seiner \textit{doch}-Äußerung auf eine dieser beiden Propositionen. Dass es sich hierbei um die Disjunktionen \textit{Du kannst Englisch.} vs. \textit{Du kannst kein Englisch.} und \textit{Du bist Vegetarier.} vs. \textit{Du bist kein Vegetarier.} handelt, ergibt sich über weitere Schlussprozesse: Wenn der Brief auf Englisch ist und B den Brief übersetzen soll, beherrscht B in As Augen Englisch. Wenn die Empfehlung an B lautet, Schweinshaxe zu bestellen, nimmt S an, dass B kein Vegetarier ist.

Neben Implikaturen und Sprechaktbedingungen kann sich eine \textit{doch}-Assertion auch auf eine Implikation \is{Implikation} beziehen, wie das Beispiel in (\ref{429}) nahelegt.

\begin{exe}
	\ex\label{429} 
	A: Ich bin oft krank.(= p)\\
	B: Du bist \textbf{doch} immer gesund.(= q)
	\hfill\hbox {\citet[132]{Egg2013}}
\end{exe}
$\neg$q ist eine logische Folgerung aus p. Durch die Assertion von p gelangt neben p $\vee$ $\neg$p auch q $\vee$ $\neg$q auf den Tisch. Die \textit{doch}-Assertion von B nimmt in der Reaktion auf As Beitrag Bezug auf q.

\textit{Doch}-Äußerungen können nicht nur reaktiv auftreten, sondern es gibt auch Verwendungen wie in (\ref{430}) und (\ref{431}).

\begin{exe}
	\ex\label{430} 
	Dein Bruder ist \textbf{doch} Arzt. Ich habe nämlich da dieses Ziehen im Arm... 
	\newline
	\hbox{}\hfill\hbox {\citet[133-134]{Hentschel1986}}
\end{exe}	
\vspace{-0.65cm}
\begin{exe}
	\ex\label{431} 
	Sie sind \textbf{doch} Paul Meier.
	\hfill\hbox {\citet[126]{Egg2013}}
\end{exe}
Die \textit{doch}-Äußerungen in (\ref{430}) und (\ref{431}) treten diskursinitial auf. Zumindest explizit (s.u.) fehlt ihnen das reaktive Moment. Es handelt sich hierbei um ein Auftreten von \textit{doch}, das in jeder mir bekannten \textit{doch}-Beschreibung zu Problemen führt und oftmals auf eher fragwürdige Art in ansonsten plausible Analysen integriert wird.\footnote{Ein Beispiel für einen solchen Fall ist die Bedeutungsmodellierung von \textit{doch} bei \citet[68]{Koenig1997}. Prinzipiell nimmt er an, dass die MP \textit{doch} auf einen Widerspruch verweist. Zu Verwendungen der Art in (\ref{430}) und (\ref{431}) schreibt er nach der Diskussion klassischer Beispiele:  \glqq Da es sich $[$...$]$ hier nicht um einen reaktiven Gebrauch von \textit{doch} handelt, kann es noch keine Widersprüche geben, die durch das Verhaltens $[$sic!$]$ des Hörers ausgelöst wurden. Was in diesen Fällen geschieht, ist eine Ausformulierung des Kontextes, die für spätere Züge des Hö\-rers relevant sind. $[$...$]$ Im Unterschied zu den Standardfällen der Verwendung von \textit{doch} erfolgt $[$...$]$ kein Rückverweis auf Inkonsistenzen zwischen neuer Information und bestehenden Annahmen, sondern ein Vorverweis auf mögliche Inkonsistenzen zwischen Zustimmung zu einer Äußerung und späteren Interaktionsschritten.\grqq{}}

Wenngleich eine Vorgängeräußerung, auf die sich die \textit{doch}-Äußerung bezieht, hier sicherlich nicht explizit vorliegt, gilt es als ein bekanntes Phänomen, dass MPn auch verwendet werden können, um das Vorhandensein anderer Information im Kontext vorzugeben, d.h. zu suggerieren. Genau dies trifft auf die Fälle in (\ref{430}) und (\ref{431}) zu. \citet[Fn 14]{Repp2013} beschreibt den Eindruck, dass derartige Äußerungen ohne \textit{doch} sehr unhöflich wirken würden. \citet[138]{Egg2013} schreibt, der Sprecher teile dem Diskurspartner diesem bereits Bekanntes mit. Diese Eindrücke lassen sich auffangen, wenn man annimmt, dass der Sprecher die Offenheit des Themas (der Bruder ist Arzt vs. der Bruder ist nicht Arzt bzw. es ist Paul Meier vs. es ist nicht Paul Meier) suggeriert und sich mit der \textit{doch}-Äußerung zu p bekennt. Mein Eindruck ist, dass derartige Assertionen defensiver wirken. Und diese Wirkungsweise folgt aus der suggerierten reaktiven Verwendung. Der Sprecher sagt etwas, das der Gesprächspartner weiß, präsentiert es aber als Reaktion auf ein bereits eröffnetes Thema, zu dem seine Reaktion gewünscht ist. Er hat dieses Thema im Diskurs folglich nicht selbst eröffnet. Mit einer MP-losen Assertion würde er sich bei diskursinitialer Verwendung der Assertion zu p bekennen, damit selbst ein neues Thema eröffnen und den Hörer dazu auffordern, eine Antwort zu geben bzw. zu drängen, die Proposition zu bestätigen. Es ist recht plausibel, dass es unhöflich wirkt, wenn ein Sprecher Information, von der er weiß, dass sie bekannt ist, von sich aus präsentiert, als stünde sie zur Diskussion. Der Eindruck der Unhöflichkeit und der Bekanntheitsstatus des Asser\-tionsgehaltes hängen meiner Beschreibung der Situation nach folglich zusammen. Darüber hinaus ist jede Gesprächseröffnung potenziell gesichtsbedrohend, da der Angesprochene das Vorhaben des Sprechers auch abblocken kann. Ein suggeriertes offenes Thema weicht dieser Gefahr aus, da der Gesprächsrahmen zwi\-schen Sprecher und Hörer als bereits bestehend ausgegeben wird und nicht erst etabliert werden muss.

Eine Unklarheit/Kontroverse, auf die ich an dieser Stelle hinweisen möchte, die ich allerdings nicht beseitigen kann, ist, ob davon auszugehen ist, dass die Disjunktion, für die ich annehme, dass sie im Kontext vor der \textit{doch}-Äußerung auf dem Tisch liegt, zum offenen Thema wird, weil der Sprecher sich zu p bzw. $\neg$p bekennt (vgl. auch \ref{Fn16}). Ich halte diese Frage nicht für leicht entscheidbar: Für die diskursinitialen \textit{doch}-Assertionen scheint mir dies äußerst unplausibel. Sicherlich möchte man hier nicht annehmen, dass der Sprecher beim Gesprächs\-partner ein Bekenntnis suggeriert, dass sein Bruder nicht Arzt ist oder er nicht Paul Meier ist. Für die anderen angeführten Beispiele wäre dies eher denkbar. In (\ref{425}) und (\ref{429}) spricht erstmal nichts dagegen, dass Sprecher A im Zuge seiner Äußerung auch ein Diskursbekenntnis zu q (\textit{Du kannst Englisch.}) und $\neg$q (\textit{Der Sprecher ist nicht immer gesund.}) macht. Fraglicher scheint dies schon wieder in (\ref{419}) und (\ref{417}). In (\ref{419}) müsste der Sprecher sich dann zuerst dazu bekennen, dass er nicht vernünftig lebt, um im direkten Anschluss das Gegenteil mitzuteilen. In (\ref{417}) ist es auch durchaus denkbar, dass A gar nicht um den Zusammenhang weiß, aufgrund dessen B seine Äußerung macht. 

Die generellere Frage, die sich hier auftut, ist, ob der Sprecher einer Äußerung sich auch zu mit dieser eingeführten \is{Implikation} Implikationen, Implikaturen \is{Implikatur} und Sprechaktbedingungen \is{Sprech\-aktbedingung} bekennt bzw. ob es sich hierbei um Bekenntnisse des Gesprächspartners oder sogar cg-Inhalt handelt. Diese steuern in den hier angeführten Beispielen die Proposition bei, auf deren entgegengesetzte Polarität sich die \textit{doch}-Äuße\-rung bezieht. Wenn es sich bei Diskursbekenntnissen \is{Diskursbekenntnis} um Inhalte handelt, derer der Sprecher sich bewusst ist, halte ich es (zumindest bei Implikaturen) nicht für plausibel, anzunehmen, dass ein Sprecher stets um diese vermittelten Inhalte weiß. Jingyang Xue (p.c.) hat mich darauf hingewiesen, dass es ebenfalls möglich ist, dass der Gesprächspartner die Implikatur erst durch die \textit{doch}-Äußerung etabliert. Für (\ref{417}) wäre dies denkbar. Dann könnte nicht davon ausgegangen werden, dass A mit seiner Äußerung auch ein Diskursbekenntnis zur implikatierten Proposition macht, aber B von diesem Zusammenhang ausgeht und sich aus seiner Sicht das Thema eröffnet.

Folglich ist im Einzelfall zu entscheiden, in welchem der Systeme die Proposition, die an der Disjunktion auf dem Tisch beteiligt ist, zu verankern ist. Die Beobachtung, dass je nach Szenario aber DC$_{\textrm{A}}$, DC$_\textrm{B}$ und/oder der cg in Frage kommen, spricht deutlich dafür, diese Verankerung der der \textit{doch}-Proposition entgegengesetzten Proposition nicht zur Bedeutungsbeschreibung von \textit{doch} zu machen. Viele Arbeiten (z.B. \citealt[71]{Doherty1985}, \citealt[83]{Ormelius-Sandblom1997}) gehen davon aus, dass aus der Vorgängeräußerung die Negation der \textit{doch}-Asser\-tion abzuleiten ist und demnach ein adversatives Moment beteiligt ist. Die obigen Überlegungen zeigen, dass davon nicht generell auszugehen ist und diese Formulierung deshalb zu stark ist. Entscheidend und klar scheint mir aber zu sein, dass die Disjunktion auf dem Tisch zu liegen kommt, d.h. die Frage steht zur Diskussion und kann durch die \textit{doch}-Äußerung thematisiert werden. Das Vorhandensein dieser beiden offenen Alternativen im Kontext – unabhängig von ihrem Zustandekommen – ist die Anforderung, die meiner Modellierung des Diskursbeitrags der \textit{doch}-Assertion nach erfüllt sein muss. In Abschnitt~\ref{sec:direktive} in Kapitel~\ref{chapter:dua} wird sich zeigen, dass sich auch weitere Auftretensweisen von \textit{doch} mit dieser Modellierung erfassen lassen.

\subsubsection{\textit{ja}}
\label{sec:ja}
Um auch den Diskursbeitrag einer \textit{ja}-Äußerung im Rahmen des Modells von Farkas \& Bruce unter Rückbezug auf Diewalds MP-Charakterisierung aufzufangen, übersetze ich Diewalds \textit{jemand denkt, dass p} im Prätext derart, dass der Diskuspartner im Kontext vor der \textit{ja}-Äußerung ein öffentliches Bekenntnis zu p hat.
\begin{exe}
	\ex\label{432} Kontext vor der \textit{ja}-Äußerung \\[-1em]	
 		\begin{tabular}[t]{|C{6em}|C{6em}|C{6em}|} 
 		\hline 	
   		$\textrm{DC}_{\textrm{A}}$ & {Tisch} & $\textrm{DC}_{\textrm{B}}$ \tabularnewline
  		\hline
   		{} & {} & p \tabularnewline
  		\hline      
   		\multicolumn{3}{|l|}{cg s$_{1}$} \tabularnewline   
  		 \hline
 		\end{tabular}
\end{exe}
Wie bei \textit{doch} verstehe ich dies als die minimale Anforderung an den Kontext und schließe damit aber nicht aus, dass in konkreten Dialogen weitere Komponenten beteiligt sind. Mehr Anforderungen müssen allerdings nicht erfüllt sein. Im Kontext nach der \textit{ja}-Äußerung entspricht Diewalds \textit{ich denke p} in meiner Mo\-dellierung dem öffentlichen Diskursbekenntnis des Sprechers zu p. Durch dieses Bekenntnis öffnet sich auf dem Tisch das Thema p $\vee$ $\neg$p (Teil 1) (vgl. (\ref{433a})) und wird aber sofort vom Tisch entfernt, da B bereits das Bekenntnis zu p hat, das benötigt wird, um diese Proposition zum Teil des cg zu machen. Die Frage auf dem Tisch wird somit sofort zugunsten von p entschieden (Teil 2) (vgl. (\ref{433b})).

\begin{exe}
	\ex\label{433} Kontext nach der \textit{ja}-Äußerung\\[-1em]
	\begin{xlist}
		\ex\label{433a} Teil 1\\[-1em]
 			\begin{tabular}[t]{|C{6em}|C{6em}|C{6em}|} 
 			\hline 	
   			$\textrm{DC}_{\textrm{A}}$ & {Tisch} & $\textrm{DC}_{\textrm{B}}$ \tabularnewline
  			\hline
   			p & p $\vee$ $\neg$p & p \tabularnewline
  			\hline      
   			\multicolumn{3}{|l|}{cg s$_{2}$ = s$_{1}$} \tabularnewline   
  			 \hline
 			\end{tabular}
 		\ex\label{433b}	Teil 2\\[-1em]
 			\begin{tabular}[t]{|C{6em}|C{6em}|C{6em}|} 
 			\hline 	
   			$\textrm{DC}_{\textrm{A}}$ & {Tisch} & $\textrm{DC}_{\textrm{B}}$ \tabularnewline
  			\hline
   			{} & {} & {} \tabularnewline
  			\hline      
   			\multicolumn{3}{|l|}{cg $\textrm{s}_{2} = \lbrace \textrm{s}_{1} \cup \lbrace \textrm{p} \rbrace \rbrace$} \tabularnewline   
  			 \hline
 			\end{tabular} 			
 	\end{xlist}	
\end{exe}
Bei der Auftretensweise von \textit{ja} lassen sich drei prinzipielle Fälle unterscheiden.

In der ersten Situation hat der Diskurspartner tatsächlich ein Bekenntnis zu p und p wird nach der Äußerung in den cg aufgenommen. Es handelt sich hierbei um Sequenzen, in denen es Gründe dafür gibt, dem Gesprächspartner ein Diskursbekenntnis zuschreiben zu können. Dies kann beispielsweise daran liegen, dass Sprecher und Hörer die Wahrnehmungsebene teilen, wie im folgenden Beispiel.

\begin{exe}
	\ex\label{433} 
	Oh dann mußt du es \textbf{ja} nochmal abmachen.
	\hfill\hbox {\citet[93]{Diewald1998}}
\end{exe}																		        
Die Äußerung erfolgt in einem Kontext, in dem ein Sprecher nach Anweisung eines anderen ein Flugzeugmodell zusammenbaut. Beide Diskursteilnehmer sehen dabei die gleichen Dinge. Der Sprecher der Assertion in (\ref{433}) weiß folglich, dass der Angesprochene (ebenfalls) sehen kann, dass das angesprochene (oder ein späteres) Teil nicht passt und das bezeichnete Teil deshalb abgemacht werden muss. Nur weil die Gesprächsteilnehmer die gleiche Wahrnehmungssituation haben, kann der Sprecher dem Adressaten das öffentliche Bekenntnis zur in (\ref{433}) ausgedrückten Proposition zuschreiben. Hierbei handelt es sich nach meiner Modellierung in (\ref{432}) und (\ref{433}) um die Voraussetzung für die Verwendung von \textit{ja}.

Unter diese Verwendungsweise von \textit{ja} fasse ich auch Beispiele wie in (\ref{434}).

\begin{exe}
	\ex\label{434} 
	\begin{xlist}
		\ex\label{434a} Du bist \textbf{ja} ganz nass.	
 		\ex\label{434b}	Du hast \textbf{ja} einen Fleck auf dem Hemd.
 		\ex\label{434c}	Du blutest \textbf{ja}.				
 	\end{xlist}	
\end{exe}	
Derartige Äußerungen beziehen sich auf Sachverhalte, die der Sprecher in der Situation feststellt und bei denen er davon ausgehen kann, dass diese für die Wahrnehmung des Hörers ebenfalls aktuell zugänglich sind. An diesen Beispielen sieht man auch, dass das vorweggenommene Hörerbekenntnis nicht nur durch Äußerungen zu kommunizieren ist. Wenn der Angesprochene z.B. in (\ref{434a}) nass in einen Raum kommt, kann ihm auch das Bekenntnis zugeschrieben werden, nass zu sein; genauso wie ihm in (\ref{434b}) transparent das Bekenntnis einen Fleck auf dem Hemd zu haben oder in (\ref{434c}) zu bluten zuzuschreiben ist. Möchte der Sprecher diese Inhalte im aktuellen Gespräch offiziell zu geteiltem Wissen machen, würde es auch recht merkwürdig anmuten, wenn er diese Sachverhalte als für den Gesprächspartner neue Information in den Diskurs einführen würde und als offenes Thema zur Diskussion stellen würde, ohne dem Angesprochenen eine Haltung zu diesem Sachverhalt zuzuschreiben.
		
In einer weiteren Verwendung von \textit{ja} \underline{unterstellt} der Sprecher dem Adressaten ein Bekenntnis zu p und p wird ebenfalls zu einem cg-Inhalt. Es handelt sich hierbei um Situationen, in denen der Sprecher sich darauf beruft, dass der Angesprochene die gleiche Annahme teilt, d.h. ein öffentliches Bekenntnis hat, obwohl der Kontext nahelegt, dass er dieses Diskursbekenntnis gerade nicht vorweist, sondern es ihm vom Sprecher unterstellt wird. (\ref{435}) zeigt ein Beispiel, in dem diese Unterstellung einen harmlosen Effekt mit sich führt. Es handelt sich um den Ausschnitt aus einem Interview. Die Gesprächspartner kennen sich \\ folglich nicht. Da die Proposition \textit{dass ich als Halbwaise aufgewachsen bin} im Kontext nicht vorerwähnt ist, ist davon auszugehen, dass der Reporter p nicht wissen kann.

\begin{exe}
	\ex\label{435} 
	\scriptsize
	[...] Meine Frau hat sowas mit den Kindern geklärt. Sie war für die Erziehung zuständig, ich habe mich nur eingemischt, wenn es wichtig war. \textbf{Ich selbst bin \underline{ja} als Halbwaise aufgewachsen.} Mein Vater starb, als ich zwölf war.   
	\hfill\hbox {\citet[204]{Rinas2006}}
\end{exe}
Obwohl klar ist, dass der Interviewer nicht schon wissen kann, dass der Sprecher Halbwaise ist, beabsichtigt der Sprecher hier nicht, den Inhalt zur Diskussion zu stellen. Er verfolgt das Ziel, ihn auf direktem Weg in den cg einzufügen, was er dadurch erreicht, dass er das Bekenntnis des Hörers zu p voraussetzt. Ich bezeichne diesen Gebrauch von \textit{ja} oben als \glqq harmlos\grqq{}, da es auch perfide Verwendungen (vgl. \citealt{Reiter1980}) gibt wie in (\ref{436}) und (\ref{437}).

\begin{exe}
	\ex\label{436} 
	\begin{xlist}
		\ex\label{436a} Du bist \textbf{ja} ein Versager.	
		\hfill\hbox {\citet[134]{Dahl1988}}	
 		\ex\label{436b}	Bist \textbf{ja} doof.		
 		\hfill\hbox {\citet[345]{Reiter1980}}		
 	\end{xlist}	
\end{exe}	

\begin{exe}
	\ex\label{437} 
(gesungen + Tanz) Ätschibätsch, wir gehn \textbf{ja} heut ins Kino.	
	\newline
 	\hbox{}\hfill\hbox {nach \citet[346]{Reiter1980}}		
\end{exe}	
Der Effekt des unterstellten Bekenntnisses wird hier in dem Sinne \glq ausgebeutet\grq {}, dass ein Hörerbekenntnis angenommen wird, obwohl der Adressat sich in (\ref{436}) mit Sicherheit nicht dazu bekannt hat oder freiwillig bekennen wird, dass er ein Versager oder doof ist. Ähnlich erlangt die Äußerung in (\ref{437}) ihre abwertende kommunikative Funktion erst dadurch, dass der Sprecher auf Seiten des Hörers ein Wissen ansetzt, das dieser sicherlich nicht hat, weil er nicht wissen kann, was der Sprecher vorhat. Der Sprecher suggeriert dem Adressaten absichtlich, dass er etwas weiß, was er de facto nicht weiß, um ihn als ausgeschlossen hinzustellen.

In der dritten Gebrauchweise von \textit{ja}, die ich unterscheiden möchte und die oft als Standardfall behandelt wird, ist die ausgedrückte Proposition bereits Teil des cg und der Sprecher verweist auf diese, um sie zu bestimmten rhetorischen Zwecken in der aktuellen Gesprächssituation zu aktivieren. In diesem Fall befindet sich p im Kontext vor der \textit{ja}-Äußerung unter den Diskursbekenntnissen beider Gesprächsteilnehmer. Der Sprecher macht p somit zum Thema im Gespräch, indem er auf eine Proposition verweist, für die im Diskurs bereits Einigkeit unter den Beteiligten hergestellt worden ist. Die Proposition wird auf den Tisch gelegt, die offene Frage wird aber im gleichen Zug reduziert, da p in diesem Kontext nicht mehr zur Diskussion gestellt werden muss.

Es scheint mir schwierig, Fälle anzuführen, die eindeutig und ausschließlich diese Verwendung widerspiegeln. Ein plausibler Gesprächszug sind Begründungen der Art in (\ref{438}).

\begin{exe}
	\ex\label{438} 
	Ich gehe nicht schwimmen. Das Wasser ist \textbf{ja} noch zu kalt.  
 		\hfill\hbox {\citet[132]{Dahl1988}}		
\end{exe}																						
Sicherlich muss dieses Beispiel nicht so interpretiert werden (Kontext 1 und 2 sind auch denkbar), doch handelt es sich um eine plausible Interpretation, dass Sprecher und Adressat sich hinsichtlich des Sachverhalts, dass das Wasser kalt ist, einig sind und dies zu einem späteren Zeitpunkt des Gesprächs als Begründung für das Nicht-Schwimmen wieder aktivieren. Die Bekanntheit des Inhalts einer Äußerung, die als Begründung beabsichtigt ist, scheint mir für den intendierten Argumentationsverlauf durchaus förderlich. 

Unter diese Gebrauchsweise von \textit{ja} können auch Äußerungen gefasst werden, die der Fremderinnerung dienen. 

\begin{exe}
	\ex\label{439} 
	\begin{xlist}
		\ex\label{439a} Du warst \textbf{ja} damals nicht dabei.	
 		\ex\label{439b}	Ich hab's dir \textbf{ja} gesagt.		
 		\hfill\hbox {\citet[344/345]{Reiter1980}}		
 	\end{xlist}	
\end{exe}
Hier kann davon ausgegangen werden, dass der Sprecher auf Sachverhalte verweist, die dem Gegenüber tatsächlich bekannt sein sollten, weil sie Teil der gemeinsamen Wissensbasis sind. 

Plausibel wäre die Annahme, dass p tatsächlich eine cg-Information ist, auch in (\ref{440}).

\begin{exe}
	\ex\label{440} 
	A: Soll ich dir beim Tragen helfen?\\
	B: Das ist \textbf{ja} viel zu schwer für dich.
	\hfill\hbox {\citet[141]{Dahl1988}}
\end{exe}
Hier wird von B eine Voraussetzung zurückgewiesen, die der Partner macht, indem er seinen Sprechakt äußert. Es ist durchaus denkbar, dass die Gesprächspartner sich tatsächlich schon geeinigt haben, welche Last A tragen kann und B diese Assertion an dieser Stelle im Gespräch anführt, um eine Voraussetzung des Angebots von A und damit das Angebot selbst zurückzuweisen. Zu den Voraussetzungen der Annahme eines Angebots kann plausiblerweise gezählt werden, dass der Inhalt des Angebots für den Sprecher machbar ist und keine Nachteile mit sich bringt. Ein kooperativer bzw. anständiger Sprecher wird den Diskurspartner auf die Nicht-Erfülltheit dieses Aspektes hinweisen. Auch in diesem Fall nutzt der Sprecher den cg-Inhalt für die Zwecke seiner Argumentation: Unter Ausdruck der Einigung in Bezug auf den Sachverhalt, dass die Sache für den Hörer zu schwer ist, kann das Angebot sofort als zurückgewiesen gelten.\footnote{Diese Modellierung baut auf der gleichen Bedeutungszuschreibung auf wie sie z.B: auch von \citet[101]{Doherty1987}, \citet[104]{Thurmair1989} oder \citet[425]{Rinas2007} vertreten wird. Andere Autoren setzen eine leicht andere \textit{ja}-Bedeutung an, die etwas schwächer nicht von der Bekannt\-heit der Proposition auf Seiten des Hörers ausgeht. \citet[146]{Jacobs1991} und \citet[178]{Lindner1991} formulieren den Beitrag der Partikel z.B. derart, dass der Hörer nicht Gegenteiliges glaubt und in der aktuellen Situation auch nicht die Falschheit von p überhaupt in Betracht zieht. Ich glaube nicht, dass meine Modellierung zu stark ist und diesen Charakterisierungen nicht nachkommen kann: Das Diskursmodell erlaubt vielmehr gerade die Abbildung dieser Situation, in der die Zustimmung auf Seiten des Hörers durch den Sprecher präsupponiert wird, und p somit akkommodiert  wird, weil der Sprecher nicht mit der Inanspruchnahme des Hö\-rers rechnet. Problematischer scheint es mir, bedeutungsminimalistisch nur den Aspekt der Kontroverse zu modellieren, der dann in vielen Fällen zum stärkeren Bedeutungsmoment der Bekanntheit erweitert werden muss.}

Mit dem Diskursmodell nach \citet{Farkas2010}, das es vermag, den Einfluss von Assertionen auf den Diskurskontext zu erfassen, sowie einer MP-Beschrei\-bung, die ich in dieses Modell integriert habe, liegen die zwei Komponenten vor, auf deren Basis ich im Folgenden meinen Vorschlag zur Beantwortung der zwei Fragen aus Abschnitt~\ref{sec:distributiondj} erläutern werde.

Meine Absicht ist es, ein Erklärungsmodell für die Beschränktheit der MP-Abfolge vorzuschlagen. Da sich diese Ableitung auf die Interpretation der Kombination gründen soll, setzt dies eine Klärung der Frage voraus, wie die MP-Kombination(en) aus \textit{ja} und \textit{doch} interpretiert werden. 	

\subsection{Das kombinierte Auftreten von \textit{ja} und \textit{doch}}
\label{sec:kombi}
Wie bereits in Abschnitt~\ref{sec:transparenz} in Kapitel~\ref{chapter:hintergrund} ausgeführt, ist ein in der Literatur kontrovers diskutierter Aspekt in diesem Zusammenhang, welche Skopusrelationen \is{Skopus} in MP-Kombi\-nationen vorliegen. Unzweifelhaft ist, dass beide MPn Skopus nehmen, was sie bei isoliertem Auftreten schließlich auch tun (vgl. (\ref{441})).

\begin{exe}
	\ex\label{441} 
	\begin{xlist}
		\ex\label{441a} Fritz wurde von einem Golfball getroffen. (p)	
 		\ex\label{441b}	Fritz wurde \textbf{ja} von einem Golfball getroffen. (ja(p))		
		\ex\label{441c}	Fritz wurde \textbf{doch} von einem Golfball getroffen. (doch(p))	
 	\end{xlist}	
\end{exe}
Die offene Frage ist allerdings, ob bei ihrem kombinierten Auftreten (vgl. (\ref{442})) zwischen ihnen eine hierarchische Relation besteht (wie in (\ref{443})) oder ob die beiden MPn den gleichen Skopus nehmen, wie in (\ref{444})\footnote{In Abschnitt~\ref{sec:transparenz} habe ich im Falle der additiven Interpretation auch die Möglichkeit erwogen, dass sich die Partikeln gleichzeitig auf p beziehen. Da ich beabsichtige, aus der Applikation der Partikeln die Abfolge abzuleiten, führe ich diese Möglichkeit nicht weiter an. Die grund\-sätzliche Interpretation, von der ich ausgehe, ist die Interpretation in (\ref{444}): Die Bedeutungen der beiden MPn addieren sich. Erneut möchte ich darauf hinweisen, dass ich nicht beabsichtige, anzunehmen, dass p selbst mehrfach in die Bedeutungskonstitution eingeht \textit{(ja \& doch)(p)} suggeriert m.E. die Annahme eines MP-Clusters, von dem ich nicht ausgehe.}.

\begin{exe}
	\ex\label{442} 
	Fritz wurde \textbf{ja doch} von einem Golfball getroffen.
\end{exe}
\vspace{-0.65cm}	
\begin{exe}
	\ex\label{443} Skopusrelation zwischen \textit{ja} und \textit{doch}\\[-1em]
	\begin{xlist}
		\ex\label{443a} ja(doch(p))	
 		\ex\label{443b}	doch(ja(p))		
 	\end{xlist}	
\end{exe}

\begin{exe}
	\ex\label{444} Keine Skopusrelation zwischen \textit{ja} und \textit{doch}\\[-1em]
	\begin{xlist}
		\ex\label{443a} 1. ja(p), 2. doch(p)	
 		\ex\label{443b}	1. doch(p), 2. ja(p)		
 	\end{xlist}	
\end{exe}
Bei den Alternativen in (\ref{443}) und (\ref{444}) handelt es sich um vier prinzipiell denkbare Möglichkeiten der Relation zwischen zwei Partikeln, vorausgesetzt sie nehmen beide Skopus in einer Äußerung, in der sie kombiniert auftreten. Jede dieser Möglichkeiten entspricht dabei einer eigenen Interpretation. Umso verwunderlicher ist es, dass verschiedene dieser Bedeutungen für die konkrete Kombination \textit{ja doch} vorgeschlagen wurden (vgl. die Zuordnungen in (\ref{445})).

\begin{exe}
	\ex\label{445} Keine Skopusrelation zwischen \textit{ja} und \textit{doch}\\[-1em]
	\begin{xlist}
		\ex\label{445a} ja(doch(p))
		\hfill\hbox {\citet{Ormelius-Sandblom1997}, \citet{Rinas2007}}	
 		\ex\label{445b}	doch(ja(p))	
 		\hfill\hbox {\citet{Lindner1991}}	
 		\ex\label{445c} 1. ja(p), 2. doch(p)
 		\hfill\hbox {\citet{Thurmair1989}}		
 		\ex\label{445d}	1. doch(p), 2. ja(p) 	
 		\hfill\hbox {\citet{Doherty1985}}		
 	\end{xlist}	
\end{exe}									
Selbst wenn die in (\ref{445}) zugeordneten Arbeiten sich natürlich in ihren Bedeutungszuschreibungen an die Einzelpartikeln unterscheiden, halte ich es für äußerst unplausibel, dass alle vier Interpretationen zutreffen können. Eine Analyse kann erst dann eine Entscheidung zwischen den Möglichkeiten in (\ref{445}) treffen, wenn sie in der Lage ist, alle diese Fälle abzubilden und wenn sie sie nach ihrer Gene\-rierung an konkreten Beispielen für das kombinierte Auftreten von \textit{ja} und \textit{doch} gegeneinander abwägt, um zu entscheiden, welcher Skopusverlauf den Diskursbeitrag am passendsten erfasst. Für bestehende Ansätze, die sich hinsichtlich dieser Frage äußern, gilt, dass sie teilweise aufgrund ihres Verfahrens der Model\-lierung der MP-Bedeutung nicht alle vier Möglichkeiten abbilden können, d.h. I\-diosynkrasien der Erklärungsmodelle schließen Lesarten aus, gegen die es prinzi\-piell keine Einwände gäbe. Bzw. es werden manche Lesarten von den Autoren nicht in Betracht gezogen, obwohl diese Bedeutungen sich mit dem verwendeten Instrumentarium durchaus formulieren ließen. 

Ich möchte im Folgenden versuchen, mit meiner eigenen Analyse diesen Forde\-rungen nachzukommen (vgl. auch schon \citealt[192-197]{Mueller2014a}; \citeyear[221-223]{Mueller2017b}). Ich halte es für möglich, mit meinem Modell alle vier Varianten aus (\ref{445}) abzubilden.
	
(\ref{446}) bis (\ref{449}) bilden die vier denkbaren Interaktionen zwischen den Skopoi von \textit{ja} und \textit{doch} im Kontext vor der \textit{ja doch}-Äußerung ab.

Dass \textit{ja} über \textit{doch} Skopus \is{Skopus} nimmt (vgl. (\ref{446})), würde dann bedeuten, dass der Kontextzustand vor der MP-Äußerung derart beschaffen ist, dass Gesprächsteilnehmer B ein Bekenntnis dazu hat, dass p $\vee$ $\neg$p auf dem Tisch liegt. Zunächst appliziert \textit{doch} auf p (p $\vee$ $\neg$p liegt auf dem Tisch), anschließend dient diese komplexe Struktur als Input für \textit{ja} (B hat ein Bekenntnis zu p bzw. hier dazu, dass p $\vee$ $\neg$p auf dem Tisch liegt).
 
\begin{exe}
	\ex\label{446} Kontext vor der MP-Äußerung: ja(doch(p))\\[-1em]	
 		\begin{tabular}[t]{|C{6em}|C{6em}|C{6em}|} 
 		\hline 	
   		$\textrm{DC}_{\textrm{A}}$ & {Tisch} & $\textrm{DC}_{\textrm{B}}$ \tabularnewline
  		\hline
   		{} & {} & (p $\vee$ $\neg$p) $\in$ T \tabularnewline
  		\hline      
   		\multicolumn{3}{|l|}{cg s$_{1}$} \tabularnewline   
  		 \hline
 		\end{tabular}
\end{exe}
Das umgekehrte Skopusverhältnis führt zu der Lesart, dass auf dem Tisch liegt, dass B ein Bekenntnis zu p hat oder dass B kein Bekenntnis zu p hat (vgl. (\ref{447})). Das zuerst zur Anwendung kommende \textit{ja} schreibt B ein Bekenntnis zu p zu, wes\-halb dieser Ausdruck im Skopus von \textit{doch} dazu führt, dass er zum Teil der Disjunktion auf dem Tisch wird.

\begin{exe}
	\ex\label{447} Kontext vor der MP-Äußerung: doch(ja(p))\\[-1em]	
 		\begin{tabular}[t]{|C{6em}|C{6em}|C{6em}|} 
 		\hline 	
   		$\textrm{DC}_{\textrm{A}}$ & {Tisch} & $\textrm{DC}_{\textrm{B}}$ \tabularnewline
  		\hline
   		{} & {} & (p $\in$ $\textrm{DC}_{\textrm{B}}$) $\vee$ $\neg$(p $\in$ $\textrm{DC}_{\textrm{B}}$) \tabularnewline
  		\hline      
   		\multicolumn{3}{|l|}{cg s$_{1}$} \tabularnewline   
  		 \hline
 		\end{tabular}
\end{exe}
Haben \textit{ja} und \textit{doch} den gleichen Wirkungsbereich und applizieren in dieser Reihenfolge, führt dies zu einem Kontextzustand, in dem Diskursteilnehmer B ein Bekenntnis zu p hat (Beitrag von \textit{ja}) und zusätzlich auf dem Tisch liegt p $\vee$ $\neg$p (Beitrag von \textit{doch}) (vgl. (\ref{448})).

\begin{exe}
	\ex\label{448} Kontext vor der MP-Äußerung: 1. ja(p), 2. doch(p)\\[-1em]	
 		\begin{tabular}[t]{|C{6em}|C{6em}|C{6em}|} 
 		\hline 	
   		$\textrm{DC}_{\textrm{A}}$ & {Tisch} & $\textrm{DC}_{\textrm{B}}$ \tabularnewline
  		\hline
   		{} & p $\vee$ $\neg$p & p \tabularnewline
  		\hline      
   		\multicolumn{3}{|l|}{cg s$_{1}$} \tabularnewline   
  		 \hline
 		\end{tabular}
\end{exe}
Wirkt erst \textit{doch} und dann \textit{ja} (vgl. (\ref{449})), ist die Situation, was den hier jeweils beschriebenen Vorzustand angeht, die gleiche wie in (\ref{448}), da bei Zuständen als statische Objekte die Reihenfolge erst einmal keine Rolle spielt. Auch in diesem Fall hat Teilnehmer B ein Bekenntnis zu p und auf dem Tisch liegt p $\vee$ $\neg$p. 

\begin{exe}
	\ex\label{449} Kontext vor der MP-Äußerung: 1. doch(p), 2. ja(p)\\[-1em]	
 		\begin{tabular}[t]{|C{6em}|C{6em}|C{6em}|} 
 		\hline 	
   		$\textrm{DC}_{\textrm{A}}$ & {Tisch} & $\textrm{DC}_{\textrm{B}}$ \tabularnewline
  		\hline
   		{} & p $\vee$ $\neg$p & p \tabularnewline
  		\hline      
   		\multicolumn{3}{|l|}{cg s$_{1}$} \tabularnewline   
  		 \hline
 		\end{tabular}
\end{exe}
(\ref{446}) bis (\ref{449}) sowie meine Erläuterungen zeigen folglich, dass meine Beschreibung der Einzelpartikeln es vermag, die denkbaren Skopusverläufe und -interaktionen prinzipiell abzubilden.

Im Folgenden möchte ich dafür argumentieren, dass die Interpretation unter gleichem Skopus zu der angemessenen Interpretation führt, wenn man die Verwendung von \textit{ja doch}-Äußerungen im Kontext betrachtet (s.u.). Als prinzipielles Argument gegen die Skopusinterpretation ist auch die Tatsache anzusehen, dass die umgekehr\-te Abfolge der MPn belegt ist. Autoren, die (\ref{445a}) als die adäquate Interpretation ansehen, argumentieren, dass die (angeblich) feste Abfolge ein Reflex der Skopusrelation \is{Skopus} ist bzw. dass deshalb nur die Interpretation unter Skopus in Frage kommt, weil die MPn in dieser Abfolge auftreten. Wenn es so ist, dass in der Abfolge \textit{ja doch} das \textit{ja} über \textit{doch} Skopus nimmt, müssten die Vertreter dieser Annahme plausiblerweise auch annehmen, dass in der Abfolge \textit{doch ja} das \textit{doch} über \textit{ja} Skopus nimmt. Da es sich meiner Meinung nach so verhält, dass sich \textit{doch ja} stets durch \textit{ja doch} austauschen lässt, der umgekehrte Ersatz aber nicht gegeben ist, halte ich es nicht für haltbar, dass für die beiden Abfolgen im Sinne der beiden Skopusmöglichkeiten eine andere Bedeutung anzunehmen ist. Wie ich in Abschnitt~\ref{sec:unmarkiert} und \ref{sec:markiert} ausführen werde, möchte ich die verschiedenen Abfolgen auf verschiedene kommunikative Absichten zurückführen, die sich in der Reihenfolge der Applikation der beiden MPn widerspiegeln. Die Bedeutung im Sinne des reinen Diskursbeitrags ist aber in beiden Fällen die gleiche. Ich gehe dabei konkret von der Interpretation in (\ref{448}) bzw. (\ref{449}) aus. Warum ich diese Interpretation für passend halte, zeige ich im Folgenden anhand zweier Auftretensweisen der MP-Kombination. 

Der erste Fall, der auch schon in \citet{Lindner1991} und \citet{Rinas2007} diskutiert worden ist, ist ein Ausschnitt aus Hofmannsthals \textit{Der Schwierige}.

\begin{exe}
\scriptsize
	\ex\label{450} Teil eines Gesprächs zwischen Graf Hans Karl Bühl und der Magd der Dame Agathe.\\
			(Komödie \textit{Der Schwierige} von Hugo von Hofmannsthal, Akt 1, Szene 6)\\
			\begin{tabular}[t]{ll} 
 				Hans Karl: & Guten Abend, Agathe. \tabularnewline
				Agathe: & Daß ich Sie sehe, Eure Gnaden Erlaucht! Ich zittre ja. \tabularnewline
				Hans Karl: & Wollen Sie sich nicht setzen? \tabularnewline
				Agathe (stehend): & Oh, Euer Gnaden, seien nur nicht ungehalten darüber,  \tabularnewline
				& daß ich gekommen bin statt dem Brandstätter. \tabularnewline
				Hans Karl: & Aber liebe Agathe, \textbf{wir sind \underline{ja doch} alte Bekannte}.  \tabularnewline
				& Was bringt Sie denn zu mir? \tabularnewline
 				Agathe: & Mein Gott, das wissen doch Erlaucht. Ich komm \tabularnewline 	
 				& wegen der Briefe.  		
 			\end{tabular}		
\end{exe}
Ich sehe die zwei Propositionen in (\ref{451}) an der Analyse als beteiligt an.

\begin{exe}
	\ex\label{451} 
	\begin{xlist}
		\ex\label{451a} p = dass Agathe sich ergeben zeigen muss
 		\ex\label{451b}	q = dass wir alte Bekannte sind	
 	\end{xlist}	
\end{exe}		
Die \textit{doch}-Äußerung nimmt Bezug auf q. Agathe macht in dieser Szene durch ihr Verhalten (sie zittert, sie sagt, dass Hans Karl nicht ungehalten sein soll, weil sie gekommen ist und nicht der Brandstätter) ein Diskursbekenntnis zu p, d.h. sie muss sich ergeben zeigen. Dadurch öffnet sich auf dem Tisch p $\vee$ $\neg$p. Ihre devote Haltung bringt nun die plausible Voraussetzung mit sich, dass Grund zu einer solchen Haltung besteht. Eine derartige Voraussetzung für devotes Verhalten ist z.B., dass die Beteiligten nicht auf einer Ebene stehen. Dies ist hier in der vorliegenden sozialen Hierarchie tatsächlich der Fall. Ein solcher Unterschied kann sich nun aus anderen Gründen auflösen. Hierzu kann beispielsweise gehören, dass man sich gut kennt und deshalb den offiziellen Gepflogenheiten nicht nachkommen muss. Für die hier betrachtete Szene heißt dies, dass wenn Hans Karl und Agathe alte Bekannte sind, sie ihm gegenüber nicht ergeben sein muss. Da sie diesen offiziellen Gepflogenheiten aber nachkommt und sich devot gibt, lässt sich aus der auf dem Tisch liegenden Proposition p ableiten, dass sie nicht alte Bekannte sind ($\neg$q). Diese Proposition beschreibt den Sachverhalt, der eine pragmatische Folgerung von p ist (wenn sich jemand devot verhält, kennt er den Interaktionspartner nicht gut) (p $>$ $\neg$q) bzw. handelt es sich hier (schwächer) um einen Sachverhalt, der in Frage gestellt wird (Kennen sie sich gut?). Sie zeigt definitiv kein Verhalten, aus dem q pragmatisch folgen würde. Das in dieser Situation offene Thema q $\vee$ $\neg$q ist auch problemlos angreifbar (vgl. (\ref{452})), wodurch Agathes untergebenes Verhalten als unangebracht zurückgewiesen werden kann.

\begin{exe}
	\ex\label{452} 
	Benimm dich mal nicht so. Wir kennen uns schon so lange. Warum machst du dich hier so klein?
\end{exe}
Ich gehe auf der Basis der obigen Erläuterungen sowie (\ref{452}) deshalb davon aus, dass auch q $\vee$ $\neg$q auf den Tisch gelangt. Die Folgerung aus p ($\neg$q) ist diskutierbar. Für diesen konkreten Dialog ist auch anzunehmen, dass der Schluss p $>$ $\neg$q Teil des cg ist. Agathe macht mit dem Bekenntnis zu p somit ebenfalls ein Diskursbekenntnis zu $\neg$q. Vor diesem Hintergrund äußert Hans Karl (\ref{453}).

\begin{exe}
	\ex\label{453} 
	Aber liebe Agathe, wir sind \textbf{ja doch} alte Bekannte.
\end{exe}
Es stellt sich der Eindruck ein, dass er damit Agathes ergebenes Verhalten als unnötig zurückweist. Die MP-Äußerung bezieht sich folglich auf die Folgerung aus der Devotheit, die mit q $\vee$ $\neg$q offenes Thema ist und die Hans Karl dadurch angreift, dass er aus dieser Disjunktion q auswählt. Ich habe den Eindruck, dass eine \textit{ja doch}-Äußerung stärker wirkt als die gleiche Assertion ohne \textit{ja}. Ich führe dies darauf zurück bzw. erfasse dies dadurch, dass q im Kontext vor der \textit{ja doch}-Äußerung in Agathes Bekenntnissystem verankert ist. Dadurch macht Hans Karl q im Zuge der \textit{ja doch}-Assertion zu einem cg-Inhalt bzw. verweist er in diesem Dialog sogar wiederholt darauf, dass sie eigentlich weiß, dass q der Fall ist. Die Proposition q scheint einen Sachverhalt zu beschreiben, über den sie sich eigentlich einig sind. Dies lässt sich schließen aus vorangehenden und folgenden Szenen: Beispielsweise sagt Agathe später, dass sie den Sekretär von früher kennt. Hans Karl nennt sie immer wieder \glqq liebe Agathe\grqq{}. Agathe weiß um die Briefe, die sie von der Gräfin überbringt. Und wenn sie sagt, die Gräfin habe ihr eingeschärft, sie solle nichts verderben, ist klar, dass sie mit der Gräfin in Kontakt steht. Aus den Anschlussszenen ist abzulesen, dass Agathe und Hans Karl sich bereits lange kennen und sich vertrauter sind als dies zwischen Leuten dieser Standesunterschiede der Fall sein müsste. In diesem Sinne kann q hier auch plausibel bereits im cg sein und zu Zwecken der Argumentation hervorgeholt werden.

\begin{exe}
	\ex\label{454} Kontext vor der \textit{ja doch}-Äußerung\\[-1em]	
 		\begin{tabular}[t]{|C{6em}|C{6em}|C{6em}|} 
 		\hline 	
   		$\textrm{DC}_{\textrm{Hans Karl}}$ & {Tisch} & $\textrm{DC}_{\textrm{Agathe}}$ \tabularnewline
  		\hline
   		{} & p $\vee$ $\neg$p & p \tabularnewline
   		{} & q $\vee$ $\neg$q & $\neg$q \tabularnewline
  		\hline      
   		\multicolumn{3}{|l|}{cg s$_{1}$ = $\lbrace$ p $<$ $\neg$q, q $\rbrace$} \tabularnewline   
  		 \hline
 		\end{tabular}
\end{exe}
\pagebreak
\begin{exe}
	\ex\label{455} Kontext nach der \textit{ja doch}-Äußerung\\[-1em]
		\begin{xlist}	
			\ex\label{455a} Teil 1\\[-1em]
 				\begin{tabular}[t]{|C{6em}|C{6em}|C{6em}|} 
 				\hline 	
   				$\textrm{DC}_{\textrm{Hans Karl}}$ & {Tisch} & $\textrm{DC}_{\textrm{Agathe}}$ \tabularnewline
  				\hline
   				{} & p $\vee$ $\neg$p & p \tabularnewline
   				q & q $\vee$ $\neg$q & $\neg$q \tabularnewline
  				\hline      
   				\multicolumn{3}{|l|}{cg s$_{2}$ = s$_{1}$} \tabularnewline   
  				 \hline
 				\end{tabular}
 			\ex\label{455b} Teil 2\\[-1em]	
 				\begin{tabular}[t]{|C{6em}|C{6em}|C{6em}|} 
 				\hline 	
   				$\textrm{DC}_{\textrm{Hans Karl}}$ & {Tisch} & $\textrm{DC}_{\textrm{Agathe}}$ \tabularnewline
  				\hline
   				{} & p $\vee$ $\neg$p & p \tabularnewline
  				\hline      
   				\multicolumn{3}{|l|}{cg s$_{3}$ = s$_{1}$} \tabularnewline   
  				 \hline
 				\end{tabular}
 		\end{xlist}		
\end{exe}
Ein in (\ref{454}) und (\ref{455}) unberücksichtigter Effekt, der für die Interpretation der MP-Äußerung aber nicht direkt relevant ist, ist, dass wenn q im cg enthalten ist bzw. zu einem cg-Inhalt wird, und dazu der Schluss p $>$ $\neg$q im cg ist, auch $\neg$q im cg sein müsste: Hans Karl \glq überschreibt\grq {} dann sowohl Agathes Bekenntnis zu $\neg$q als auch ihr Bekenntnis zu p. In diesem Sinne findet sich der Eindruck wieder, dass Hans Karl Agathes von ihr vermittelte Ergebenheit als unnötig zurückweist.

Ich möchte folglich vertreten, dass eine \textit{ja doch}-Äußerung eine sinnvolle Interpretation erfährt, wenn man davon ausgeht, dass die beteiligten Partikeln sich beide gleichermaßen auf die gleiche Proposition beziehen, d.h. den gleichen Skopus nehmen. Dies entspricht der Bedeutungszuschreibung aus (\ref{445c}/\ref{445d}) bzw. den diskursstrukturellen Modellierungen in (\ref{448}) und (\ref{449}).

In diesem Sinne soll die Verwendung der beiden MPn in dem Kontext in (\ref{450}) nicht darauf verweisen, dass zur Diskussion steht, ob Agathe ein Bekenntnis zu q oder $\neg$q hat. Dies entspricht der Lesart, in der \textit{doch} Skopus über \textit{ja} nimmt. Für genauso unpassend halte ich die Interpretation, unter der Hans Karl sich mit seiner \textit{ja doch}-Äußerung auf ein Bekenntnis von Agathe bezieht, das beinhaltet, dass q vs. $\neg$q auf dem Tisch liegt. Diese Lesart entsteht, wenn \textit{doch} im Skopus von \textit{ja} steht. In beiden Fällen müsste das Thema (Sind sie alte Bekannte?) nicht einmal wirklich auf dem Tisch liegen. Es würde sich allein um die von ihr vertretene Annahme handeln bzw. es stünde zur Diskussion, was sie vertritt. Mir scheinen diese Lesarten zu schwach. Das Thema q $\vee$ $\neg$q steht hier tatsächlich zur Diskussion, auf der Basis ihres Verhaltens. Auch unter der von mir zugeschriebenen Bedeutung kann Agathe durchaus ebenfalls annehmen, dass auf dem Tisch liegt, ob q gilt bzw. dass (aufgrund ihres Bekenntnisses zu $\neg$q) zur Diskussion steht, ob sie q annimmt oder nicht. Beides beschreibt aber nicht den erforderlichen Kontextzustand für die MP-Äußerung von Hans Karl. Setzt man die additive Lesart an, können diese Verhältnisse ebenfalls eintreten. Sie können aber nicht als Minimalanforderung an den Vorkontext einer \textit{ja doch}-Assertion angesehen werden. Ich halte hier die non-Skopus-Lesart für korrekt. Und dies ist genau die Frage, die es zu entscheiden gilt: Wie sieht der Kontext aus, auf den ein Sprecher mit einer \textit{ja doch}-Äußerung reagiert? Mit anderen Worten, was muss vorliegen, damit Hans Karl sich zu dieser Äußerung veranlasst sieht? Da die Kontextanforderungen, wie ich sie formuliert habe, Minimalanforderungen abbilden, gilt dies gleichermaßen für die Modellierungsmöglichkeiten der Kombinationen.

(\ref{456}) zeigt ein authentisches Beispiel einer \textit{ja doch}-Assertion aus dem FOLK-Korpus der DGD2. EL und NO, die dem Gespräch nach zu urteilen beide Friseure sind, unterhalten sich über Sehnenscheidenentzündungen und das Risiko für ihre berufliche Tätigkeit.

EL hat bisher nur mit leichten Entzündungen zu tun gehabt, NO berichtet, dass er sich jeden Abend den Arm mit Voltaren eincremt. Schließlich kommt das Thema auf eine Sehnenscheidenentzündung, wobei nicht klar ersichtlich ist, wen von beiden sie betraf.\footnote{Für meine Analyse ist dies allerdings auch unerheblich. Ich gehe davon aus, dass NO betroffen war.}

\begin{exe}
	\ex\label{456} 
		\begin{tabular}[t]{lll} 
 		1172 & NO & oder darfst\_s (.) bierglas nich falsch anfassen kriegst ooch \tabularnewline
 		& & $[$ne sehnenschei$]$denent$[$zündung (.) ne$]$ \tabularnewline
 		1173 & EL & $[$hm\_hm$]$ \tabularnewline
 		1174 & EL & $[$((lacht))$]$ \tabularnewline
 		1175 &	& (0.31) \tabularnewline
		1176 & EL &	\textbf{na det war \underline{ja doch} (.) von der gitarre} \tabularnewline
		1177 &	& (0.31) \tabularnewline
		1178 & NO & ((lacht)) \tabularnewline
 		1179 &	& (1.84) \tabularnewline
		1180 & EL & die sehnenscheidengitarre \tabularnewline
		1181 & & (0.2) \tabularnewline
		1182 & NO & hm\_h$[$m \tabularnewline
		1183 & EL & $[$((lacht))
		\hfill\hbox {(FOLK\_E\_00039\_SE\_01\_T\_02)}			
  		\end{tabular} 						
\end{exe}	
NOs Äußerung suggeriert, dass besagte Entzündung vom Heben eines Bierglases stammt (= p). Damit macht er ein Bekenntnis zu p und legt p mit seiner Alternative auf den Tisch. Die Proposition p \is{Implikation} impliziert, dass die Entzündung nicht auf einen anderen Umstand zurückzuführen ist, wie z.B. auf eine Gitarre ($\neg$q) (wenn man davon ausgeht, dass es nur einen Grund gibt). p $\rightarrow$ $\neg$q kann als cg-Wissen angesehen werden. Durch das Bekenntnis zu p bekennt NO sich auch zu $\neg$q, wodurch sich auch das Thema q $\vee$ $\neg$q eröffnet. Aus dem Folgekontext ist zudem klar, dass beide wissen, dass die Entzündung durch die Gitarre bzw. ein als solches benutztes Objekt zustandegekommen ist, d.h. q ist im cg.

\begin{exe}
	\ex\label{457} Kontext vor der \textit{ja doch}-Äußerung\\[-1em]	
 		\begin{tabular}[t]{|C{6em}|C{6em}|C{6em}|} 
 		\hline 	
   		$\textrm{DC}_{\textrm{EL}}$ & {Tisch} & $\textrm{DC}_{\textrm{NO}}$ \tabularnewline
  		\hline
   		{} & p $\vee$ $\neg$p & p \tabularnewline
   		{} & q $\vee$ $\neg$q & $\neg$q \tabularnewline
  		\hline      
   		\multicolumn{3}{|l|}{cg s$_{1}$ = $\lbrace$ p $\rightarrow$ $\neg$q, q $\rbrace$} \tabularnewline   
  		 \hline
 		\end{tabular}
\end{exe}
EL äußert die \textit{ja doch}-Assertion und reagiert damit auf das offene Thema, ob es die Gitarre war. q kann transparent als bekannt ausgegeben werden, wodurch NOs Bekenntnis zu p, das überhaupt Anlass für die \textit{doch}-Verwendung ist, \glq überschrieben\grq {} und zurückgewiesen wird. Wenn Einigkeit hinsichtlich der Implikation \is{Implikation} besteht und hinsichtlich q, kann nur $\neg$p gelten. Dies entspricht auch der Realität, weil NO eigentlich sehr genau weiß, dass die Gitarre der Grund für die Entzündung ist, d.h. er meint seinen Beitrag sicherlich nicht ganz ernst.

Die Non-Skopus-Lesart führt erneut zu einer adäquaten Interpretation der Szene. EL reagiert nicht derart, weil NO annimmt, dass zur Diskussion steht, ob die Sehnenscheidenentzündung von der Gitarre herrührt oder nicht (ja(doch(q))) und auch nicht, weil auf dem Tisch liegt, ob NO davon ausgeht, dass die Gitarre der Grund ist oder dass er nicht davon ausgeht (doch(ja(q))). Wie in (\ref{450}) ist mein Punkt nicht, zu vertreten, dass diese Bedeutungsaspekte undenkbar und unmöglich beteiligt sein können. Aber ich bin der Meinung, dass ja(q) \& doch(q) die Absichten der MP-Äußerung am besten auffängt. Es geht in erster Linie darum, im Kontext q selbst zu klären.\\

\noindent
Anhand dieser zwei \textit{ja doch}-Assertionen im Kontext möchte ich folglich dafür argumentieren, dass auch – neben der Tatsache, dass die umgekehrte Abfolge in nicht abweichender Skopuslesart belegbar ist – die Betrachtung derartiger Äußerungen im Kontext die Entscheidung zugunsten der additiven Lesart nahelegt.

Angenommen die additive Lesart bildet die Interpretation der \textit{ja doch}-Äußerung korrekt ab, bleibt dennoch die Frage bestehen, wie die Präferenz der Reihung \textit{ja doch} zustande kommt. Unabhängig davon, ob zuerst \textit{ja} und anschließend \textit{doch} seinen Beitrag leistet oder \textit{doch} vor \textit{ja} appliziert, bleibt die Interpretation hinsichtlich der Bezugsbereiche schließlich gleich.

\section{Erklärung der unmarkierten Abfolge}
\label{sec:unmarkiert} 
\subsection{Ikonizität}
Den Markiertheitsunterschied zwischen \textit{ja doch} und \textit{doch ja} möchte ich im Folgenden über die Annahme einer Form von \textit{Ikonizität} \is{Ikonizität} ableiten (vgl. \citealt[197-200]{Mueller2014a}; \citeyear[223-226]{Mueller2017b}). Ich gehe davon aus, dass die syntaktische Oberflächenabfolge die Verwendung der Elemente widerspiegelt. Diese prinzipielle Überlegung wird auch in anderen funktionalen Erklärungen zur Wortstellung vertreten. \citet[399]{Dik1997} formuliert beispielsweise das Prinzip in (\ref{458}).
\setcounter{equation}{0}
\begin{exe}
	\ex\label{458} Generelles Prinzip 1\\
 		The Principle of Iconic Ordering\\
		Constituents \is{The Principle of Iconic Ordering} conform to (GP1) when their ordering in one way or another iconically reflects the semantic 				content of the expression in which they occur. 
	\hfill\hbox {\citet[399]{Dik1997}}
\end{exe}
Ich habe in Abschnitt~\ref{sec:ikonizität} in Kapitel~\ref{chapter:hintergrund} diese Form von Ikonizität als \textit{diagrammatische ikonische Motivation} (\citealt[516]{Haiman1980}) \is{diagrammatische ikonische Motivation} als einen Typus von Ikonizität eingeordnet (zu einer ausführlicheren Darstellung des Konzeptes vgl. diesen Abschnitt).

Dieses Prinzip äußert sich nach \citet[399]{Dik1997} z.B. darin, dass die Ordnung von Sätzen in einem Text im unmarkierten Fall die Reihenfolge der Ereignisse widerspiegelt, die sie beschreiben. Je nach temporaler Konjunktion \is{Konjunktion} ergeben sich z.B. Unterschiede hinsichtlich der Markiertheit der Abfolge von Haupt- und Nebensatz. Im unmarkierten Fall gehen Nebensätze mit der Bedeutung \glq nachdem p\grq {} dem Hauptsatz deshalb voran (vgl. (\ref{459})), während Nebensätze mit der Bedeutung \glq bevor p\grq {} dem Hauptsatz folgen (vgl. (\ref{460}), vgl. auch \citealt{Diessel2008}).

\begin{exe}
	\ex\label{459} 
		\begin{xlist}
			\ex\label{459a} After John had arrived, the meeting started. (unmarkiert)
 			\ex\label{459b}	The meeting started after John had arrived. (markiert)
 		\end{xlist}		
\end{exe}

\begin{exe}
	\ex\label{460}
 		\begin{xlist}
			\ex\label{460a} The meeting started before John arrived. (unmarkiert)
 			\ex\label{460b}	Before John arrived, the meeting started. (markiert)	
 		\hfill\hbox {\citet[400]{Dik1997}}	
 		\end{xlist}				
\end{exe}
\citet{Dik1997} argumentiert ähnlich für markierte und unmarkierte Abfolgen von Haupt- und Nebensatz bei Konditional- und \is{Konditionalsatz} \is{Finalsatz} Finalsätzen (vgl. (\ref{461}), (\ref{462})).

\begin{exe}
	\ex\label{461}
 		\begin{xlist}
			\ex\label{461a} If you are hungry, you must eat. (unmarkiert)
 			\ex\label{461b}	You must eat if you are hungry. (markiert)
 		\end{xlist}			
\end{exe}

\begin{exe}
	\ex\label{462}
 		\begin{xlist}
			\ex\label{462a} John went to the forest in order to catch a deer. (unmarkiert)
 			\ex\label{462b}	In order to catch a deer, John went to the forest. (markiert)	
 		\end{xlist}			
	\hfill\hbox {\citet[400]{Dik1997}}
\end{exe}
Die Überlegung ist, dass die Bedingung der Konsequenz in gewissem Sinne konzep\-tuell überlegen ist, ähnlich wie die Ausführung der Handlung ihrem finalen Ziel.

In Abschnitt~\ref{sec:ikonizität} habe ich auch gezeigt, dass ikonische Erklärungen nicht nur für prinzipiell akzeptable, lediglich weniger frequente Strukturen herangezogen worden sind, sondern durchaus auch stärkere Akzeptabilitätsabfälle zu beobachten sind (vgl. (\ref{464}) und (\ref{465})).

\begin{exe}
	\ex\label{464}
 		\begin{xlist}
			\ex\label{464a} *He killed and shot her.
 			\ex\label{464b}	He shot and killed her.	
 		\hfill\hbox {\citet[92]{Givon1991}}	
 		\end{xlist}				
\end{exe}

\begin{exe}
	\ex\label{465}
 		\begin{xlist}
			\ex\label{465a} Herz und Nieren
 			\ex\label{465b}	*Nieren und Herz
 			\hfill\hbox {\citet[140]{Plank1979}}	
 		\end{xlist}			
\end{exe}
Im Folgenden möchte ich der Idee nachgehen, zu sagen, dass die Abfolge der MPn in einem Sinne motiviert ist, in dem motiviert ist, warum z.B. temporale Abfolgen Einfluss auf markierte und unmarkierte Anordnungen von Haupt- und Nebensatz nehmen. In dem von mir untersuchten Fall sind nicht temporale Relationen beteiligt oder Konzeptualisierungen von oben nach unten, die sich in \is{irreversible Binomiale} Binomialen spiegeln, sondern es geht um die direkteste Abbildung des Diskursverlaufs, d.h. die Reihenfolge der Kontextaktualisierungen mit den beiden MPn.

\subsection{Stabile und instabile Kontextzustände}
In Abschnitt~\ref{sec:mplass} habe ich die zwei generellen Antriebe für Gespräche nach \citet{Farkas2010} angeführt (vgl. (\ref{466})).

\begin{exe}
	\ex\label{466} Zwei fundamentale Antriebe für Gespräche
 		\begin{xlist}
			\ex\label{466a} Erweiterung des cg
 			\ex\label{466b}	Herstellen eines stabilen Kontextzustands
 		\end{xlist}			
\end{exe}
Zum einen folgen Teilnehmer dem Bedürfnis, den cg zu erweitern. Zum anderen streben sie danach, einen stabilen Kontextzustand zu erreichen, d.h. einen Zustand, in dem kein offenes Thema zur Diskussion auf dem Tisch liegt. Die Gesprächsteilnehmer beabsichtigen somit, die Elemente auf die Art vom Tisch zu entfernen, dass der cg erweitert wird. 

Betrachtet man die Diskursbeiträge, die ich in Abschnitt~\ref{sec:inkdm} für \textit{ja}- und \textit{doch}-Äußerungen formuliert habe, vor dem Hintergrund der Stabilität von Kontextzuständen, bezieht sich \textit{doch} stets auf einen instabilen Kontextzu\-stand: Die Disjunktion p $\vee$ $\neg$p liegt vor der MP-Äußerung auf dem Tisch. Der Sprecher bekennt sich im Zuge der \textit{doch}-Äußerung zu einer der beiden Propositionen. Mit einer \textit{doch}-Äußerung kann aber nie Einigung hergestellt werden, so dass das Thema vor und nach der MP-Äußerung offen ist. Der Kontext bleibt instabil. Die Verwendung von \textit{ja} resultiert hingegen immer in einem stabilen Kontextzustand. Im Kontext vor der MP-Äußerung ist im Diskursbekenntnissystem des Gesprächspartners bereits genau die Annahme enthalten, die die Assertion im nächsten Kontextzustand einführen wird. Als Resultat haben Sprecher und Hörer das gleiche öffentliche Bekenntnis und die Proposition gelangt in den cg. Die Partikel \textit{ja} involviert in diesem Sinne stets einen stabilen Kontextzustand, \textit{ja} fordert nie, dass p zur Debatte steht. 

Wenn \textit{ja} und \textit{doch} zusammen auftreten, ist somit immer ein Element beteiligt, das einen stabilen Kontextzustand herstellt (\textit{ja}) und ein Element, das auf einen instabilen Zu\-stand Bezug nimmt (\textit{doch}), der auch bestehen bleiben würde, wenn es allein aufträte (d.h. ohne \textit{ja}).

\subsection{Diskursstrukturelle Ikonizität}
Wenn es nun die oberste Absicht eines Gespräches ist, den cg zu erweitern und einen stabilen Kontextzustand zu erreichen, kommt ein Sprecher diesem obersten kommunikativen Ziel am direktesten nach, wenn er das Element, das die Stabilität des Kontextes herbeiführen kann und die Proposition zu cg-Material machen kann, sofort einführt und zur Wirkung bringen lässt.

Führt er erst das \textit{ja} ein, wird das, was er wünscht, nämlich Stabilität, direkt hergestellt, da er dadurch ausdrückt, dass die Diskursteilnehmer sich hinsichtlich der zur Diskussion stehenden Proposition einig sind. Führt der Sprecher zuerst das \textit{doch} ein, bringt er zunächst nur die konzessive Relation zum Ausdruck (trotz der beiden zur Diskussion stehenden Optionen p $\vee$ $\neg$p vertritt der Sprecher p). Und erst im nächsten Schritt vermittelt er, dass es sich bei diesem Inhalt um eine Annahme handelt, die auch der Gesprächspartner vertritt, weshalb sie sich hinsichtlich p einig sind und p Teil des cgs ist. 

Vor diesem Hintergrund halte ich es für unmarkiert, weil ikonisch, das \textit{ja} vor dem \textit{doch} einzuführen, da das \textit{ja} den von \textit{doch} vorausgesetzten instabilen Zustand sofort auflöst. Der eigentliche Diskursbeitrag ist zwar unter beiden Abfolgen der gleiche, am direktesten, d.h. \is{Isomorphie} isomorph, kommt aber die Reihenfolge \textit{ja doch} dem kommunikativen Ziel nach. Diese Reihenfolge ist folglich motiviert in dem Sinne, dass sie unter Auftreten dieser beiden MPn die direkteste Möglichkeit darstellt, den gewünschten Kontextzustand herbeizuführen.

Die Anordnung der MPn leite ich hier aus der Annahme ab, dass die Anrei\-cherung des cg sowie die Herstellung eines stabilen Kontextzustandes von den Gesprächsteilnehmern beabsichtigt ist. Hierbei handelt es sich um ein übergeordnetes Prinzip, das im Diskurs wirkt. Es macht eine sehr allgemeine Annahme über Diskursabsichten, die zunächst nicht an bestimmte Sprechakte oder Konstruktionstypen gebunden ist. Deshalb verwundert es nicht, dass dieses Prinzip über alle assertiven Kontexte hinweg greift, wann immer \textit{ja} und \textit{doch} gemeinsam auftreten. Die Reihung \textit{ja doch} ist folglich stets die bevorzugte Abfolge. Dies gilt sowohl für die Beispiele, die ich zu Beginn von Abschnitt~\ref{sec:abfolgejd} aus der Literatur angeführt habe, als auch für die sprachlichen Kontexte, in denen sich die umgekehrte Abfolge \textit{doch ja} finden lässt (vgl. Abschnitt~\ref{sec:distributiondj}). 

\subsection{Prototypische Assertionen}
Wenngleich es sich bei (\ref{466}) um ein übergeordnetes Prinzip kommunikativer Absichten handelt, wird es dennoch durch konkrete Konstruktions- bzw. Sprechakttypen \is{Sprechakt} realisiert.

Diese Typen sind zwar in dem Sinne gleich, dass sie assertiv \is{Assertivität} sind, sie unterscheiden sich aber auch auf die Art, dass sie auch eigene Absichten mitbringen (s.u.). Genau diese Anforderungen oder kommunikativen Absichten von Äußerungen sind der Aspekt, über den ich ableiten möchte, warum die Abfolge von \textit{ja} und \textit{doch} vornehmlich in ganz bestimmten Fällen umkehrbar zu sein scheint. Die Idee ist, dass es Sprechakttypen gibt, deren Eigenschaften sowieso – d.h. unabhängig des Auftretens jeglicher MPn – den Diskurseigenschaften entsprechen, die die \textit{ja doch}-Abfolge dem allgemeinen Prinzip in (\ref{466}) nach widerspiegelt. Hierbei handelt es sich \is{prototypische Assertion} um prototypische Assertionen, d.h. Assertionen, die alle drei Kriterien aus (\ref{467}) erfüllen.

\begin{exe}
	\ex\label{467} Prototypische Assertion
 		\begin{xlist}
			\ex\label{467a} Bekenntnis des Autors zu p.
 			\ex\label{467b}	p (vs. non-p) wird auf dem Tisch oben auf gelegt.
 			\ex\label{467c}	Projektion eines zukünftigen cg, der p beinhaltet.
 		\end{xlist}			
 		\hfill\hbox{\citet[92]{Farkas2010}}
\end{exe}
Dieser Typ von Assertion hält sich, wenn \textit{ja} und \textit{doch} zusammen auftreten, sehr einfach an das übergeordnete Diskursprinzip, weil es dem Charakter dieses assertiven Typus entspricht: Es ist die Absicht einer solchen MP-losen Assertion, p zum Inhalt des cg zu machen. Da dieser Typ Assertion p sowieso in den cg einfügen möchte, ist es nur natürlich, dass – wenn zwei Lexeme auftreten, von denen eines diese Forderung erfüllen kann (\textit{ja}) und das andere nicht (\textit{doch}) – der Sprecher ersteres (das \textit{ja}) durch seine unmittelbare Einführung sofort zur Wirkung bringt, um dem Ziel der cg-Herstellung auf direktestem Wege nach\-zukommen.

\section{Erklärung der markierten Abfolge}
\label{sec:markiert} 
Nun ist die prototypische Assertion nur \underline{ein} Typ von Assertion im Diskurs. Meiner Meinung nach lassen sich manchen der als assertiv eingestuften Konstruktionen eigene Absichten zuschreiben, die von der prototypischen Assertion abweichen. In diesen Fällen ist es nicht die oberste Absicht der assertiven Äußerung, die ausgedrückte Proposition zu einem cg-Inhalt zu machen. Aus diesem Grund ist es auch nicht ihre oberste Absicht, den cg-Marker \textit{ja} möglichst früh applizieren zu lassen. Die eigenen Eigenschaften dieses Typs von Assertion entsprechen folg\-lich nicht sowieso der Diskursveränderung, die die \textit{ja doch}-Abfolge unmarkiert abbildet – dem möglichst unmittelbaren und direkten Schaffen eines stabilen Kontextes. Liegt ein solcher assertiver Typ vor, der nicht primär dieses Ziel verfolgt, lässt sich die Abfolge am leichtesten umkehren. Meine Vorhersage ist somit, dass die Abfolge \textit{doch ja} nicht in prototypischen Assertionen auftritt, deren Absicht es ist, die ausgedrückte Proposition zum Inhalt des cg zu machen, d.h. zu bewusst geteiltem Inhalt zwischen den Diskurspartnern.

Die Frage, die sich an dieser Stelle auftut, ist nun, inwiefern die drei Kontexte, für die ich in Abschnitt~\ref{sec:distributiondj} argumentiere, dass \textit{doch ja} dort aufzufinden ist, diese Überlegung bestätigen. Es gilt nachzuweisen, inwiefern es nicht das oberste kommunikative Ziel dieser Assertionen ist, die enthaltene Proposition zu geteiltem Inhalt zu machen. Ich werde im Folgenden die drei Kontexte nacheinander beschreiben, um anschließend ihren gemeinsamen Nenner im Sinne der obigen Überlegung zu formulieren (vgl. auch schon \citealt[200-204]{Mueller2014a}; \citealt[226-231]{Mueller2017b}). 

Der erste Kontext sind Bewertungen. (\ref{468}) bis (\ref{475}) zeigen erneut Beispiele aus Abschnitt~\ref{sec:distributiondj} bzw. weitere Belege.

\begin{exe}
	\ex\label{468} 
	\textbf{Das ist \underline{doch ja} wieder \textit{typisch}.} Ein \glqq Nerd\grqq{} läuft Amok wegen Frust auf Weib \& Lehrer. 
\end{exe}

\begin{exe}
	\ex\label{469} 
	\scriptsize
	Ich denke, auch die meisten Frauen merken schon irgendwann rechtzeitig, dass sie selbst weiblich sind und wenn sie dann mal als Mann angesprochen 			werden, ist es wohl auch keine Kränkung. \glqq Sehr geehrte Frau Minister!\grqq{} \textbf{Ist \underline{doch ja} auch ganz \textit{hübsch}.}		
\end{exe}	

\begin{exe}
	\ex\label{470} 
	\scriptsize
	die FR hat als bollwerkspresse in diesem fall mal wieder genau im richtigen moment die gelegenheit ergriffen zu initiieren! bravo, danke! dass bronski 	jetzt auch noch so genial ist das mit dem thema kultur überhaupt zu verflechetn, \textbf{ist \underline{doch ja} schon \textit{die speerspitze der europäischen bewegung}} (*grins)
	\newline		
	\hbox{}\hfill\hbox{(DECOW2012-00:B00: 331498526)}		
\end{exe}

\begin{exe}
	\ex\label{471} 
	\scriptsize
	Davon ab: für uns KLingonen seit IHR die Aliens! ( hrhr ) eben und ihr von der Förderation seit ja sooo bööööseeeeeeeeeee : D \textbf{Das ist 				\underline{doch ja} auch \textit{einer der interessanten Aspekte an Star Trek}}, dass es so eine klare Trennung zwischen Gut und Böse nicht gibt.		
	\hfill\hbox{(DECOW2012-03:B03: 254862193)}	
	\newline		
	\hbox{}\hfill\hbox{\citet[227]{Mueller2017b}}	
\end{exe}

\begin{exe}
	\ex\label{472} 
	\scriptsize
	Wir Verbraucher sind doch so leicht zu manipulieren, würden uns auch in der Wüste eine Heizung aufschwatzen lassen (\textbf{klar 8 h ist \underline{doch ja} auch \textit{mal recht kalt}}). HiHi wie doof der Otto-Normal Verbraucher doch zu weilen ist.		
	\hfill\hbox{(http://www.motor-talk.de/forum/ab-heute-}	
	\newline		
	\hbox{}\hfill\hbox{13-08-in-der--dacia-duster-gegen-lada-niva-}	
	\newline		
	\hbox{}\hfill\hbox{t2847278.html?page=2, Beitrag vom 20.08.2010)}	
	\newline		
	\hbox{}\hfill\hbox{(Google-Recherche vom 24.07.2012)}
\end{exe} 
							                    
\begin{exe}
	\ex\label{473} 
	warum sollte EA schwache Screenshots vom PC veröffentlichen – \textbf{das passt \underline{doch ja} \textit{irgendwie} nicht}?!
	\hbox{}\hfill\hbox{(DECOW2012-00:B00: 693856613)}
\end{exe} 
		               		 		        
\begin{exe}
	\ex\label{474} 
	Mittlerweile bin ich gar nicht mehr so abgeneigt. Frischer Wind kann ja nur gut tun und \textbf{viel schlimmer als Rosi letzte Saison \textit{wird} er \underline{doch ja} nich sein}.
	\hbox{}\hfill\hbox{(DECOW2012-00:B00: 972050785)}
\end{exe} 	
	
\begin{exe}
	\ex\label{475} 
	US Termin ist noch unbekannt, wird aber vor dem UK/ EU release sein. \textbf{aber das war \underline{doch ja} \textit{klar}} das es in eu kommen 			wuerde.    
	\newline		                                              
	\hbox{}\hfill\hbox{(DECOW2012-02:B02: 346714563)}
\end{exe} 
Mit allen diesen Äußerungen gibt der Sprecher eine Bewertung eines Sachverhalts ab. In fast allen Fällen dieser Art, die ich gefunden habe, handelt es sich um Kopula-Konstruktionen mit der Struktur \textit{das ist + X}. Typischerweise wird das Prädikat durch ein eva\-luatives Adjektiv realisiert, wie in den Beispielen vom Anfang (vgl. (\ref{468}) $[$\textit{typisch}$]$, (\ref{469}) $[$\textit{hübsch}$]$) oder wie in (\ref{475}) $[$\textit{klar}$]$). Es können hier aber auch Nominalphrasen stehen, die entweder selbst bewertende Bedeutungsanteile haben (vgl. (\ref{470}) $[$\textit{Speerspitze}$]$) oder die – da selbst semantisch relativ bedeutungsarm – in Verbindung mit einem bewertenden Adjektiv (vgl. (\ref{471}) mit dem semantisch relativ leeren Nomen \textit{Aspekte}) auftreten. Der (be)wertende Aspekt geht dann entweder auf das Nomen selbst zurück (vgl. (\ref{470})) oder auf ein bewertendes attribuierendes Adjektiv (vgl. (\ref{471})). Gelegentlich treten in Äußerungen dieser Art auch Abschwächer oder Relativierer auf wie in (\ref{472}) und (\ref{473}) \textit{mal}, \textit{recht} oder \textit{irgendwie}, die zu den \textit{Heckenausdrücken} (engl. \textit{hedge}) \is{Heckenausdruck (hedge)} gezählt werden können. Der Terminus geht ursprünglich auf \citet{Lakoff1973} zurück und be\-zeichnet Ausdrücke, die anzeigen, in welchem Maß ein Sprecher eine Sache einer Kategorie zuordnet. In auf Lakoffs Arbeit folgende Forschung ist die Klasse erweitert worden und – wie so häufig – hat man es hier mittlerweile mit den verschiedensten Klassifikationen und damit verbunden Abgrenzungs- und Terminologieproblemen zu tun (zu einem Überblick vgl. \citealt[Kapitel 3]{Clemen1998}). Für die in (\ref{472}) und (\ref{473}) auftretenden Ausdrücke scheint es mir allerdings unproblematisch, zu behaupten, dass sie, da sie eine \glqq gewisse Reserve gegenüber einer eindeutigen Einordnung\grqq{} (\citealt[10]{Clemen1998}) anzeigen, zur subjektiven Färbung \is{Subjektivierung} der Äußerung beitragen. 

In Beleg (\ref{474}) tritt auch \textit{werden} in epistemischem Gebrauch \is{epistemisches Modalverb} auf, für das angenommen wird, dass es eine \glqq subjektive Prognose\grqq{}  (\citealt[39]{Clemen1998}) oder eine \glqq Inferenz aus subjektiven Annahmen oder Überzeugungen\grqq{} (\citealt[1901]{Zifonun1997}) anzeigt. 

Bewertet ein Sprecher einen Sachverhalt, wie ich für derartige Beispiele annehmen möchte, dann verfolgt er mit dem Ausdruck solch eines Urteils nicht in erster Linie, den Hörer dazu zu bewegen, diese Bewertung zu teilen. Und in diesem Sinne ist es nicht das oberste Ziel dieses Typs von Äußerung, den cg um diese Information zu erweitern. Es besteht deshalb nicht die Notwendigkeit, die MP, die den Zustand der cg-Erweiterung direkt herbeiführen kann, frühst möglich zur Anwendung zu bringen. Der mit einer Bewertung \is{Bewertung} ausgedrückte Inhalt wird natürlich dennoch geteilte Information zwischen den Diskursteilnehmern (andernfalls müsste \textit{ja} schließlich überhaupt nicht verwendet werden), dies ist jedoch nicht das oberste Ziel solch einer Äußerung.

Ein weiterer Kontext, in dem sich die markierte Abfolge \textit{doch ja} belegen lässt, sind epistemisch \is{epistemische Modalisierung} modalisierte Sätze. Hier findet sich eine große Bandbreite der konkreten Realisierungen. In Abschnitt~\ref{sec:distributiondj} habe ich Beispiele angeführt, in denen \is{epistemisches Modalverb} epistemische Modalverben, modalisierende Adverbien \is{epistemisches Adverb} und Tag-Fragen \is{Tag-Frage} die epistemische Modalisierung bedingen. (\ref{476}) bis (\ref{485}) zeigt einige andere Beispiele, in denen diese sprachlichen Mittel auftreten.

\begin{exe}
	\ex\label{476} 
	\scriptsize
	Dazu kam also ein schlechtes Gewissen – ändern konnte ich ja nichts mehr daran, was ich den Ohren der Menschen \glqq angetan\grqq{} habe. Die Einzige, 	die mir da Mut gemacht hat, war meine Logopädin, die mir sagte, ich hätte auch ohne CI mein Sprechen gut kontrollierte, \textbf{es \textit{konnte} 			\underline{doch ja} also so schlimm nicht gewesen sein}. 	
	\hfill\hbox{http://www.kestner.de/n/elternhilfe/berichte/nf2.htm)}	
	\newline		
	\hbox{}\hfill\hbox{(eingesehen am 9.6.2012, Google-Recherche)}	
	\newline		
	\hbox{}\hfill\hbox{\citet[228]{Mueller2017b}}
\end{exe} 

\begin{exe}
	\ex\label{477} 
	\scriptsize
	Ist das der Luke der in der aktuellen Reptilia einen Bericht veröffentlicht hat? höchstwahrscheinlich wenn man die Nachnamen vergleicht war der schon 		out of order als er gefunden worden war? ich mein wenn \textbf{der \textit{musste} \underline{doch ja} noch zeit gehabt haben}, telefoniert oder 			jemanden bescheid gegeben haben das er gebissen worden ist. 	
	\hfill\hbox{(DECOW2012-04: 445498466)}	
\end{exe}

\begin{exe}
	\ex\label{478} 
	\scriptsize
	Exklusivinterview mit Josh Bazell\\
	Was danach kommt ... mhhh ... im Grunde finde ich den Gedanken ausgesprochen reizvoll, diese Figur bis zum Gehtnichtmehr auszuquetschen (lacht) und 		wenn ich es recht überlege, werde ich wahrscheinlich noch über Pietro schreiben, wenn ich achtzig bin (lacht noch mehr). Warum nicht? \textbf{\textit{Könnte} \underline{doch ja} gut sein}, dass die Leute dann immer noch seine Geschichten hören wollen ...			
	\newline		
	\hbox{}\hfill\hbox{(http://www.krimi-forum.de/Datenbank/Interviews/fi002279.html)}	
	\newline		
	\hbox{}\hfill\hbox{(Google-Recherche vom 24.07.2012)}	
\end{exe} 

\begin{exe}
	\ex\label{479} 
	\scriptsize
	Bei einem Abenteuer in z.B. Haelgard sollte ich ja jetzt nicht zuviel Gefahr haben, daß hier eine Orkarmee alles kurz und klein schlägt. \textbf{Wir 		haben \underline{doch ja} \textit{nun wirklich} bei Gott genügend weiße Flecken auf der Karte wo was plaziert werden kann und auch sollte.} 		
	\hfill\hbox{(DECOW2012-00: 618521882)}	
	\newline		
	\hbox{}\hfill\hbox{\citet[228]{Mueller2017b}}	
\end{exe} 		      

\begin{exe}
	\ex\label{480} 
	\scriptsize
	Aber damit hat man noch keinen Unterschied definiert. \textbf{Vorallem ist es \underline{doch ja} \textit{eigentlich} abhängig vom Hörenden ob jetzt ne Glocke schellt oder glocknet.}
	\newline
	\hbox{}\hfill\hbox{(http://kumanomori.wordpress.com/2008/08/20/glocke-triichle-schelle-und-balolzeli/)}	
	\newline		
	\hbox{}\hfill\hbox{(Google-Recherche vom 24.07.2012)}	
	\newline		
	\hbox{}\hfill\hbox{\citet[201]{Mueller2014a}}	
\end{exe} 	
\vspace{-0.65cm}
\begin{exe}
\ex\label{481}
\scriptsize
\begin{tabular}[t]{ll}
	SPIEGEL: & Aber die Grenze selbst war noch nicht erreicht? \tabularnewline
	SCHILLER: & Ich sage ja. entlang der Grenze. \tabularnewline 
	SPIEGEL: &	Können Sie uns ein paar Grenzsteine nennen. \textbf{Sie haben \underline{doch ja} auch \textit{gewiß} Vor-} \tabularnewline
	& stellungen, wo die stehen. \tabularnewline
	SCHILLER: & Sicherlich, aber die leuchtet man nicht an. 
\end{tabular}
	\newline		
	\hbox{}\hfill\hbox{(http://www.spiegel.de/spiegel/print/d-42928443.html)}
	\newline		
	\hbox{}\hfill\hbox{(eingesehen am 05.10.2015)}
\end{exe}

\begin{exe}
	\ex\label{482} 
	\scriptsize
	Während der Installation wurde dann mein Benutzername gefragt (\textbf{\textit{eigentlich} hatte ich \underline{doch ja} schon einen!?}).
	\hfill\hbox{(http://www.computerhilfen.de/hilfen-5-86354-0.html, Beitrag vom 11.10.2005)}	
	\newline		
	\hbox{}\hfill\hbox{(eingesehen am 31.7.2014)}	
\end{exe} 		
	
\begin{exe}
	\ex\label{483} 
	\scriptsize
	Wir sollen was machen? Aufgabe? Pflicht? Wirkung? Jetzt? Ähm, liebe Öffentlichkeit ... ... ... ... tut doch bitte einfach so, als wären wir nicht da . 	: ) \textbf{Wir sind \underline{doch ja} auch gar nicht da, \textit{oder}?}
	\newline		
	\hbox{}\hfill\hbox{(DECOW2012-00: 384726199)}	
\end{exe} 		
				        		                        
\begin{exe}
	\ex\label{484} 
	\scriptsize
	sagt mal es gibt ja eine rassen kunde für hunde und katzen und nager, \textbf{für schweinchen gibt es das \underline{doch ja} auch \textit{oder}?}
	\hfill\hbox{(DECOW2012-03:B03: 64250787)}	
\end{exe} 									                        
 
\begin{exe}
	\ex\label{485} 
	\scriptsize
	Hoffe das sich hier in geraumer Zeit etwas ändert!. Gegen wenn soll sich die Bluray durchsetzen? \textbf{Sie hat \underline{doch ja} schon gegen HDDVD 	gewonnen \textit{oder}?}    
	\hfill\hbox{(DECOW2012-02: 568050573)}	
	\newline		
	\hbox{}\hfill\hbox{(Google-Recherche vom 24.07.2012)}	
	\newline		
	\hbox{}\hfill\hbox{\citet[228]{Mueller2017b}}	
\end{exe} 									   		                 
Der Effekt von derartigen epistemischen Modalausdrücken ist, anzuzeigen, dass der Sprecher der Realisierung des beschriebenen Sachverhalts eine größere oder kleinere Wahrscheinlichkeit zuschreibt. Im Zentrum solcher Sätze steht für den Sprecher nicht, p zu geteiltem Wissen zu machen, sondern seine Einschätzung hinsichtlich der Proposition kund zu tun. In diesem Sinne ähneln die in (\ref{476}) bis (\ref{485}) auftretenden Mittel den expliziten Bewertungen des oben angeführten ersten \textit{doch ja}-Kontextes, in dem sich das Bedeutungsmoment der Bewertung aus dem Inhalt ergibt. Diese Verhältnisse, die ich im Folgenden für die einzelnen sprachlichen Mittel konkreter ausführen werde, liefern die Begründung, anzunehmen, dass das Eigenbedürfnis des Äußerungstyps nicht derart beschaffen ist, unabhängig die Voranstellung von \textit{ja} und dessen direkte Anwendung vor der anderen MP (\textit{doch}) zu präferieren, weil es die oberste Absicht des Sprechers ist, cg herzustellen.

In (\ref{476}) bis (\ref{478}) treten epistemische Modalverben \is{epistemisches Modalverb} auf. \citet[26]{Mache2009} zählt zu diesen Verben: \textit{kann}, \textit{könnte}, \textit{muss}, \textit{müsste}, \textit{dürfte}, \textit{sollte}, \textit{mag}, (\textit{will}) und (\textit{möchte}). (\ref{486}) zeigt einige Beispiele für die Verwendung dieser Verben.
	
\begin{exe}
	\ex\label{486} 
		\begin{xlist}	
			\ex\label{486a} Sie \textit{\textbf{dürfte}} inzwischen fertig sein.
			\ex\label{486b} Sie \textbf{\textit{kann}} mit dem Auto gefahren sein.
			\ex\label{486c} Sie \textit{\textbf{mag}} recht haben.
			\ex\label{486d} Sie \textit{\textbf{muß}} in der Stadt sein.	
			\hfill\hbox {\citet[220]{Diewald1999b}}
			\ex\label{486e} Morgen \textit{\textbf{dürfte}}/\textit{\textbf{sollte}} das Wetter besser sein.			
		\end{xlist}
\end{exe}	
Trotz ansonsten deutlich abweichender Beschreibung und Analyse der Modalverben in der Literatur, scheint in Bezug auf die Klasse der epistemischen Modalverben \is{Modalverb} Einigkeit zu bestehen, dass sie vermitteln, dass der Sprecher dem Sachverhalt eine mehr oder weniger große Wahrscheinlichkeit zuschreibt. \citet[25]{Diewald1997} beispielsweise spricht von einer \glqq sprecherabhängige$[$n$]$ Einschätzung der Rea\-lität des dargestellten Sachverhalts\grqq{}. Ähnlich heißt es bei \citet[28]{Oehlschlaeger1989} \glqq $[$...$]$ daß die Modalverben hier eine Einstellung des Sprechers hinsichtlich des Bestehens eines Sachverhalts ausdrücken, Grade der Gewissheit des Sprechers, daß ein bestimmter Sachverhalt besteht\grqq{}. \citet[350]{Loetscher1991} fasst den Bedeutungsbeitrag als \glqq durch epistemische Inferenzen gewonnene relative Sicherheitsgrade bezüglich einer Behauptung\grqq{}. Die verschiedenen Modalverben werden dann gerne entlang einer Skala der Gewissheit, wie z.B. in (\ref{487}) geordnet. Hier nimmt der Gewissheitsgrad von links nach rechts ab. 

\begin{exe}
	\ex\label{487} 
	müssen $>$ werden $>$ dürfen $>$ mögen $>$ können
	\hfill\hbox {\citet[206]{Oehlschlaeger1989}}
\end{exe}
\citet[108]{Liedke2000} spricht hier mit \citet[21]{Buscha1981[1971]} von der \glqq Vermutungsbedeutung\grqq{}. Dazu wird manchmal davon ausgegangen, dass man es generell mit einer \glqq Abschwächung des Wahrheitsanspruchs\grqq{} (\citealt[109]{Liedke2000}) zu tun hat (vgl. auch \citealt[222]{Diewald1993}, \citealt[205]{Diewald1999b}).

In den \textit{doch ja}-Beispielen (s.o.) treten mit \textit{können} und \textit{müssen} Verben auf, deren epistemische Verwendung unter obiger Interpretation niemand anzweifeln würde. Es finden sich aber auch Belege mit \textit{sollen} in einem Gebrauch wie in (\ref{488}) und (\ref{489}) (vgl. schon Abschnitt~\ref{sec:distributiondj}).

\begin{exe}
	\ex\label{488} 
	\scriptsize
	Nur zur Vollständigkeit: Was muss beim löschen des Computerkontos im AD denn noch beachtet werden? Einfach danach wieder in die Domäne bringen und fertig? \textbf{Benutzerrechte \textit{sollten} [sich] \underline{doch ja} nicht ändern} – gibt es noch Fallen?
\end{exe}

\begin{exe}
	\ex\label{489} 
	\scriptsize
	Ich persönlich halte \glqq Alles für eine umfassendere und nicht ausschließende Einstellung. Aber bestätigen kann ich Dir das erst nach vielen Testfahrten ;-). \textbf{\textit{Eigentlich} \textit{sollte} sich ein solches Gerät bei der Einstellung \glqq Sendersuche: automatisch\grqq{} \underline{doch ja} \textit{wohl} den besten, aber empfangbaren Sender nehmen, so \textit{denke ich}.}
\end{exe}
Mit \citet[33]{Heine1995} schreibe ich auch \textit{sollen} in dieser Verwendung eine epistemische Interpretation zu (vgl. auch die Beispiele in \citealt[350]{Loetscher1991}; vgl. auch \citealt[27, Fn 4]{Mache2009}). Heine führt hier das Beispiel in (\ref{490}) an, dessen epistemische Lesart er paraphrasiert als \glq Ich habe Grund zur Annahme, dass das vor mir stehende Bier kalt ist.\grq {} (im Gegensatz zur non-epistemischen Interpretation \glq Ich möchte, dass das Bier kalt ist. Deshalb solltest du es besser wieder in den Kühlschrank tun.\grq {}).

\begin{exe}
	\ex\label{490} 
	Das Bier \textbf{\textit{sollte}} kalt sein.
\end{exe}
\citet[202, Fn 32-34]{Diewald1999b} möchte dieses epistemische \textit{sollen} anders nicht anerkennen.

Modalverben stellen ein \is{epistemisches Modalverb} sprachliches Mittel dar, um eine epistemische Modalisierung zu bewirken. Eine andere Möglichkeit ist die Verwendung von \is{epistemisches Adverb} Adverbien (vgl. (\ref{491})).

\begin{exe}
	\ex\label{491} 
		\begin{xlist}	
			\ex\label{491a} Sie ist \textit{\textbf{wahrscheinlich}}/\textit{\textbf{vermutlich}} inzwischen zu hause.
			\newline
			\hbox{}\hfill\hbox {\citet[29]{Diewald1997}}
			\ex\label{491b} \textit{\textbf{Vielleicht}} habe ich mich getäuscht.	
			\hfill\hbox {\citet[278]{Diewald1999b}}
			\ex\label{491c} Hier hat es \textit{\textbf{sicher}} mal Wasser gegeben.
			\hfill\hbox {\citet[67]{Dietrich1992}}			
		\end{xlist}
\end{exe}
Die \glqq sprecherbasierte Faktizitätsbewertung\grqq{} (\citealt[14]{Diewald1999b}), die ich oben mit verschiedenen Autoren den epistemischen Modalverben zugeschrieben habe, ist hier noch offensichtlicher auf die lexikalische Bedeutung der Ausdrücke zurückzuführen (bei den Modalverben beeinflussen auch weitere grammatische und kontextuelle Faktoren $[$vgl. z.B. \citealt[223-229]{Diewald1993}, \citealt[23-33]{Heine1995}$]$). Die Adverbien dienen dem Zweck, die Proposition als mehr oder weniger (un)gewiss auszuzeichnen. Für die Exemplare in (\ref{491}) ist wohl die Skala in (\ref{492}) anzunehmen.

\begin{exe}
	\ex\label{492} 
	sicher $>$ wahrscheinlich/vermutlich $>$ vielleicht
\end{exe}
Ohne behaupten zu wollen, dass zwischen den (a)-, (b)- und (c)-Beispielen in (\ref{491}) und (\ref{493}) jeweils Bedeutungsidentität besteht, werden diese Adverbien gerade herangezogen, um den epistemischen Gebrauch der Modalverben wiederzugeben.

\begin{exe}
	\ex\label{493} 
		\begin{xlist}	
			\ex\label{493a} Sie \textit{\textbf{dürfte}} inzwischen zu hause sein.
			\hfill\hbox {\citet[29]{Diewald1997}}
			\ex\label{493b} Ich \textit{\textbf{kann}} mich getäuscht haben.	
			\hfill\hbox {\citet[278]{Diewald1999b}}
			\ex\label{493c} Hier \textit{\textbf{muß}} es mal Wasser gegeben haben.
			\hfill\hbox {\citet[67]{Dietrich1992}}			
		\end{xlist}
\end{exe}
Ein Adverb, das sehr auffällig vertreten ist unter den \textit{doch ja}-Belegen, ist \textit{eigentlich}. Die Charakterisierungen seiner Verwendung als Satzadverb \is{Satzadverb} aus der Literatur fügen sich gut in meine Überlegung, warum die Abfolge der Partikeln bei einem Kovorkommen hier in umgekehrter Abfolge auftritt. \textit{Eigentlich} wird in dieser Verwendung die Funktion zugeschrieben, den Gültigkeitsanspruch des dargestellten Sachverhalts einzuschränken oder zu relativieren (vgl. z.B. \citealt[26]{Albrecht1977}, \citealt[77]{Koenig1990}). Es trete eine \glqq Abschwächung der Assertion\grqq{} ein (\citealt[26]{Albrecht1977}) und es würden Behauptungen eingeleitet, \glqq die dem Sprecher selbst $[$...$]$ zweifelhaft vorkommen\grqq{} (\citealt[340]{Reiners1943}. Der Sprecher bekennt sich folglich nicht vorbehaltlos zum ausgedrückten Sachverhalt, von dem er den Adressaten zu überzeugen beabsichtigt. Vielmehr nimmt er eine reserviertere Haltung ein und präsentiert somit – wie im Falle der obigen klassischen epistemischen Modalisierungen – seine subjektive \is{Subjektivierung} Sicht auf die Dinge. Die Nähe zu Modalisierungen zeigt sich auch in der von \citet[26]{Albrecht1977} vorgeschlagenen Paraphrasierung von \textit{eigentlich p} durch \textit{p sollte gelten}. Da es deshalb auch in diesem Fall nicht das oberste Ziel der Äußerungen ist, p zu einem cg-Inhalt zu machen, erklärt sich, weshalb \textit{ja} in der Kombination auch erst später zur Wirkung gebracht werden \underline{kann}.
 
In (\ref{483}) bis (\ref{485}) treten Tag-Fragen \is{Tag-Frage} auf, die in deutschen Arbeiten als \glqq Rückversicherungssignale\grqq{} (\citealt{Schwitalla2002}), \glqq Vergewisserungssignale\grqq{} (\citealt{Weinrich2005[1993]}) oder \glqq Vergewisserungsfragen\grqq{} (\citealt{Willkop1988}) bezeichnet werden. Willkop sieht derartige \glq Fragen\grq {} (worunter sie keinen formalen Fragetyp verstanden wissen will) durch die \glqq frageähnlich verwendete$[$n$]$ Partikeln\grqq{} bzw. \glqq ste\-reotype$[$n$]$ Wendungen\grqq{} (\citeyear[70]{Willkop1988}) \textit{ne}, \textit{nich}, \textit{nicht}, \textit{nicht wahr}, \textit{gell}, \textit{ge}, \textit{oder}, \textit{ja}, \textit{hm}, \textit{nein} realisiert (\citeyear[71]{Willkop1988}). Es lassen sich auch regionale (\textit{gell?}, \textit{odr?}, \textit{wa?}, \textit{woll?}) oder jugendsprachliche (\textit{ey}) Varianten ausmachen (vgl. \citealt[265]{Schwitalla2002}, \citealt[128]{Imo2011}, \citealt{Frey2010}).
 
Als die zwei Kernfunktionen derartiger Ausdrücke gelten a) das Einleiten eines Sprecherwechsels und b) die Anregung zu einem Hörer-Feedback (ohne Wechsel der Sprecherrolle). Während b) das Heischen um Aufmerksamkeit oder die Verständnissicherung zum Zweck hat, ist mit a) die Erwartung verbunden, dass dem Sprecher fehlendes Wissen geliefert wird oder der Angesprochene die Ansicht des Sprechers teilt (nach \citealt[146]{Hagemann2009}, vgl. auch die metakommunikativen Fragen, die \citealt[73]{Willkop1988} den Vergewisserungssignalen zuordnet $[$z.B. \textit{Stimmt es, daß...}; \textit{Hörst du mir zu?}$]$). 

Insbesondere aufgrund der beim Adressaten eingeforderten Bestätigung der Sicht/Einstellung des Sprechers werden Frage-Tags \is{Tag-Frage} mitunter (vorschnell) kategorisch mit epistemischer Unsicherheit in Verbindung gebracht, wie z.B. im folgenden Zitat aus \citet[146]{Imo2011} 146): \glqq Das eigene Gesicht wird dadurch gewahrt, dass man durch ein Vergewisserungssignal die eigene Aussage zur Diskussion stellt und ihren Wahrheitsanspruch abschwächt.\grqq{} Es ist allerdings vielmehr so, dass hier zwischen verschiedenen Tags unterschieden werden muss, da sie Gebrauchsunterschiede aufweisen (z.B. verschiedenen Äußerungstypen angehängt werden können) und es je nach Ausdruck gerade nicht der Fall ist, dass der Sprecher die Absicht verfolgt, sich nicht zum Gesagten bekennen zu wollen (vgl. die Einzelanalysen in \citealt[125-134]{Bublitz1978}, \citealt[253-261, 262-270, 271-275]{Willkop1988}). \citet{Hagemann2009} macht auch darauf aufmerksam, dass die Position eines Tags (redezugintern vs. -final) Einfluss auf seine interaktive Funktion nimmt. Es gibt durchaus Ausdrücke, die gerade Nachdrücklichkeit (vs. ein abgeschwächtes Be\-kenntnis) indizieren und die die Sprecherrolle sichern (vs. sie übergeben) wollen. 

Die alleinige Tatsache, dass die Abfolge \textit{doch ja} in meinen Belegen auch gerne von Frage-Tags begleitet wird, fügt sich deshalb noch nicht in das Bild, dass epistemische Modalisierungen \is{epistemische Modalisierung} die Umkehr der unmarkierten \textit{ja doch}-Abfolge begünstigen. Entscheidend ist, dass es sich um das Vergewisserungssignal \textit{oder?} handelt (teilweise auch begleitet von (übermäßig) vielen Fragezeichen oder Frage- und Ausrufezeichen).

\textit{Oder?} wird in seiner Funktion als Frage-Tag nämlich genau mit der Ver\-mittlung epistemischer Unsicherheit auf Seiten des Sprechers in Verbindung gebracht. \citet[273]{Willkop1988} beschreibt den Effekt dieses Vergewisserungssignals derart, dass die vorweggehende Behauptung teilweise zurückgenommen werde. Der Sachverhalt werde \glqq nachträglich als lediglich wahrscheinlich gekennzeichnet\grqq{} (\citeyear[271]{Willkop1988}) und der Sprecher markiere seine Behauptung als Vermutung (\citeyear[272]{Willkop1988}). In diese Charakterisierung fügt sich die Einschätzung aus \citet[131]{Bublitz1978}, dass \textit{oder?} nicht auftreten kann, wenn der Sprecher von seiner Aussage sehr überzeugt ist. Es deute eine \glqq Alternative zum Vordersatz an\grqq{} (\citeyear[126]{Bublitz1978}) (wobei \citealt[276]{Willkop1988} auch betont, dass – wie im Falle anderer Vergewisserungsfragen – eine affirmative Reaktion erwartet werde). Aus diesem Grund könne \textit{oder?} ebenfalls nicht gut stehen, wenn die Alternative unmittelbar zuvor ausgeschlossen worden ist.

\begin{exe}
	\ex\label{494} 
	So, das freut mich; ihr habt jetzt also doch endlich geheiratet, \textit{\textbf{ja}}?/*\textit{\textbf{oder}}?
	\newline
	\hbox{}\hfill\hbox {\citet[127]{Bublitz1978}}			
\end{exe}
Der Sprecher ist sich der Gültigkeit seiner Assertion nicht sicher, es handelt sich um eine \is{Assertion} abgeschwächte Assertion. Mit dem sprecherseitig vertretenen einge\-schränkten Gültigkeitsanspruch gegenüber p und der durch den Adressaten benö\-tigten Bestätigung geht Willkops weitere Beschreibung einher, dass vom Adressaten eine Stellungnahme konkret erwartet werde (\citeyear[272]{Willkop1988}), in dem Sinne, dass sein Gebrauch i.d.R. der Übergabe der Sprecherrolle diene. Der Sprecher wünsche eine Auflösung durch den Hörer. Sie sieht hier sogar die Nähe zu Informationsfragen (auch im Unterschied zu \textit{ne}, \textit{gell}).

Wie bei den Bewertungen und epistemischen Modalisierungen durch Modalverben oder Adverbien sowie beim Auftreten von \textit{eigentlich} liegt folglich wiederum eine gewisse Distanzierung des Sprechers vom ausgedrückten Inhalt vor, was meiner Ansicht nach die Umkehr der unmarkierten MP-Abfolge ermöglicht. Es handelt sich zwar um die Einschätzung des Sprechers, dass p gilt, er beabsichtigt aber kein cg-Update mit der beteiligten Proposition. Er vertritt eine abwartendere Haltung als mit einer unmodalisierten Assertion, in dem Sinne, dass die Entscheidung, ob p Gültigkeit hat, von der Hörerreaktion abhängig ist. Prinzipiell erlaubt der Sprecher auch noch die Alternative, was bei einer Standardassertion ausgeschlossen ist. Es steht somit nicht im Mittelpunkt, p zu geteiltem Wissen zu machen, was durch das nachgestellte \textit{ja} (und nicht seine frühest mögliche Ap\-plikation) gespiegelt wird.

Neben diesen drei sprachlichen Mitteln, für die ich in Abschnitt~\ref{sec:distributiondj} bereits Beispie\-le angeführt habe, finden sich auch funktional verwandte Erscheinungen, die durch (\ref{495}) bis (\ref{500}) illustriert werden.

\begin{exe}
	\ex\label{495} 
	\scriptsize
	Das gibt ja n Problem, wenn ich evtl. noch gar kein Zimmer habe, wenn das Studium schon losgehtb:crazy: \textbf{V.a. müssen wir \underline{doch ja} 		nicht nur zur Uni, sondern auch mal ins Oberwiesenfeld oder nach Oberschleißheim, \textit{seh ich das richtig?}}
	\hfill\hbox{(DECOW2012-07: 515043714)}	
\end{exe}

\begin{exe}
	\ex\label{496} 
	\scriptsize
	Sowie Gast 5 meine ich das. Man sollte ja versuchen die angegebene Auflösung zu nutzen. \textbf{Das trifft \underline{doch ja} wohl auch für Spiele 		zu, \textit{\textbf{oder irre ich da?}}}   
	\hfill\hbox{(DECOW2012-07: 333693059)}	
\end{exe}
In (\ref{495}) und (\ref{496}) treten andere Rückfragen \is{Rückfrage} auf, die durch ganze Phrasen realisiert sind, die gerade lexikalisch die fragende Interpretation nahelegen, die anzeigt, dass der Sprecher vom dargestellten Sachverhalt nicht vollends überzeugt ist und ihn nicht – ohne ein Hörerstatement abzuwarten – zu geteiltem Wissen machen möchte. 

Ein weiteres Realisierungsmittel epistemischer Modalisierung, das in Ab\-schnitt~\ref{sec:distributiondj} bei einem kombinierten Vorkommen erwähnt wurde, sind \is{Diskursmarker} Diskursmarker. In (\ref{497}) tritt ein solcher Ausdruck in Isolation auf und markiert die vorweggehende Äußerung offensichtlich einerseits überhaupt als subjektive Einschätzung des Sprechers und stuft sie andererseits in der Mitte einer Commitment-Skala ein (vgl. \citealt[18]{Aijmer1997}, vgl. auch \citeyear[24]{Aijmer1997} zu \textit{I think} in finaler Position).	

\begin{exe}
	\ex\label{497} 
	\scriptsize
	So rein spekulativ nur, aber von der Theorie her, damit ich weiß ob ich es richtig verstehe. Aber, so ganz sinnvoll würde mir das ja nicht erscheinen, 	\textbf{würde \underline{doch ja} auch die Notebooks betreffen, \textit{denke ich}}, ich kenn mich da nicht aus $[$...$]$.  
	\hfill\hbox{(DECOW201200: 304543630)}	
\end{exe}
Ebenfalls zu den Mitteln der Realisierung epistemischer Modalität möchte ich Distanzierungen zählen, mit Hilfe derer ein Sprecher anzeigt, dass die Quelle der Information ein anderer Sprecher ist. Ausgedrückt werden kann diese \is{quotative Lesart} quotative/reportative \is{reportative Lesart}Interpretation beispielsweise durch eine Lesart des Modalverbs \textit{sollen} (wie in (\ref{498}) und (\ref{500})) sowie durch Adverbien wie \textit{angeblich} (vgl. (\ref{499}), (\ref{500})).

\begin{exe}
	\ex\label{498} 
	\scriptsize
	Das Tempo in dem Du diese Quallität ablieferst ist einfach atemberaubend. Vielen Dank für die tolle Beschreibung. \textbf{\textit{Soll} 					\underline{doch ja} eigentlich ganz einfach sein}, wenn man Deinen SBS folgt.  
	\newline
	\hbox{}\hfill\hbox{(DECOW2012-01: 931561866)}	
\end{exe}

\begin{exe}
	\ex\label{499} 
	\scriptsize
	Mit welchen Recht darf nur die Frau allein entscheiden, ob sie sich einem Kind mit Herztönen erledigen möchte? Und die Babyklappe? Warum wird in allen 	Punkten der Vater nicht in die Entscheidung mit einbezogen? \textbf{Er hat \underline{doch ja} \textit{angeblich} auch das \glqq Sorgerrecht\grqq{}}, 		zumindest wenn man verheiratet ist.
	\hfill\hbox{(DECOW2012-01: 253859941)}	
\end{exe}
												                    
\begin{exe}
	\ex\label{500} 
	\scriptsize
	Im Horizontalmodus war ich bisher nicht unterwegs da er mir trotz sauber eingestellter TS immer sehr stark nach links weg driftet. Eigenartigerweise 		macht er das im Pos Mod oberhalb 15 meter nicht so stark, \textbf{wo er \underline{doch ja} \textit{angeblich} dann in den Horizontalmod automatisch 		umschalten \textit{soll}}. 	
	\newline
	\hbox{}\hfill\hbox{(DECOW2012-00: 536949567)}	
\end{exe}		      								   				  
Neben \textit{sollen} hat auch \textit{wollen} eine ähnliche quotative Verwendung (vgl. (\ref{501})).

\begin{exe}
	\ex\label{501} 
	Sie \textbf{\textit{will}} den Betrag vor einer Woche überwiesen haben.
\end{exe}	
Je nach Autor werden diese reportativen Gebrauchsweisen auch zu den epistemischen Lesarten \is{epistemisches Modalverb} gezählt (vgl. z.B. \citealt[235]{Oehlschlaeger1989}, \citealt[218, 119-220]{Diewald1993}, \citealt[20]{Heine1995}). In dem Sinne, dass epistemische Modalverben \glqq besagen, daß es dem Sprecher nicht möglich ist, den dargestellten Sachverhalt als faktisch einzuschätzen\grqq{} (\citealt[221]{Diewald1993}), ist dies sicherlich zutreffend, wenn\-gleich hier natürlich keine Vermutungseinschätzung kund getan wird (was in anderen Arbeiten wiederum der Grund ist, diese Verwendungen nicht als i.e.S. epistemisch einzustufen $[$vgl. z.B. \citealt[41]{Mache2009}$]$), sondern die Verantwortung für die Faktizität des Inhalts einer anderen Person zugeschrieben wird. Diese ausdrückliche Entbindung von der Verpflichtung, für die Gültigkeit des Sachverhalts selbst einzutreten, kann man als Distanzierung (vgl. \citealt[49]{Bruenner1983}) bzw. \glqq Abschwächung der Assertion\grqq{} (\citealt[103]{Glas1984}) verstehen. Diese Charakterisierung von \textit{sollen} und \textit{wollen} in der epistemischen Lesart lässt sich auf \textit{angeblich} übertragen, dessen Bedeutung durch \glq wie behauptet/gesagt wird\grq {}, \glq scheinbar\grq {}, \glq wohl\grq {} charakterisiert wird (vgl. den Eintrag auf www.duden.de zu \textit{angeblich}). 

Auch durch diese Ausdrücke wird folglich ein eingeschränktes Sprecherbekennt\-nis zur Proposition bewirkt, was zu dem Eindruck einer abgeschwächten Assertion führt, deren Inhalt der Sprecher nicht vorbehaltlos zu geteiltem Wissen zwischen den Diskurspartnern zu machen beabsichtigt.

Und wie auch eingangs bereits erwähnt, finden sich zudem zahlreiche Beispiele, in denen diese sprachlichen Mittel kombiniert werden. Teilweise betrifft dies schon die obigen Belege. Weitere Fälle kombinierten Auftretens zeigen die folgenden Beispiele.

\begin{exe}
	\ex\label{502} 
	\scriptsize
	das fragst ausgerechnet ! a u s g e r e c h n e t ! du ? ! \textbf{\textit{müsstest} du \underline{doch ja} \textit{eigentlich} am besten wissen} 			WARUM man provoziert und stänkert.		 		 
	\hfill\hbox{(DECOW2012-02: 359345024)}	
\end{exe}

\begin{exe}
	\ex\label{503} 
	\scriptsize
	Nur habe ich jetzt was vergessen worauf ich auch noch achten muss? \textbf{Und eine logo/sps darf \underline{doch ja} \textit{eig.} nicht die Stanze 		direkt ansteuern \textit{oder?}} Bzw. das Ventil und das ventil die Stanze.		 		 
	\newline
	\hbox{}\hfill\hbox{(DECOW2012-05: 986319955)}	
\end{exe}
			              
\begin{exe}
	\ex\label{504} 
	\scriptsize
	Habe mal günstig Hering in Sahnesoße ergattert....nun läuft er in 4 Tagen ab...wollte fragen ob man diesen einfrieren kann?
	hm, \textbf{es kann \underline{doch ja} \textit{eigentlich} nichts passieren \textit{oder?????}} 
	\newline
	\hbox{}\hfill\hbox {(http://www.chefkoch.de/forum/2,56,281434/Hering-in-Sahnesosse-einfrieren.html)}
	\newline
	\hbox{}\hfill\hbox {(Google-Recherche vom 24.07.2012)}
\end{exe}	

\begin{exe}
	\ex\label{505} 
	\scriptsize
	Ihr fragt euch wahrscheinlich, weshalb ich in der letzten Zeit so wenige News über JC schreibe. \textbf{Der \textit{sollte} \underline{doch ja} 			\textit{eigentlich} fleissig im Training sein und riesen Fortschritte machen, \textit{\textbf{oder}?}} Leider nein.	
	\hfill\hbox {(http://www.myreininghorse.ch/, Beitrag vom 11.02.2012)}
	\newline
	\hbox{}\hfill\hbox {(eingesehen am 24.07.2012)}
	\newline
	\hbox{}\hfill\hbox {\citet[176]{Mueller2014a}}
\end{exe}        	        
Generell sehe ich die fördernde Wirkung epistemischer Modalisierungen (unter die ich hier verschiedene sprachliche Mittel gefasst habe) auf das Zulassen der umgekehrten Abfolge von \textit{ja} und \textit{doch} folglich – wie bei den expliziten Bewertungen – darin, dass der Sprecher kein uneingeschränktes Bekenntnis zu p abgibt, das er in den cg hinzufügen möchte, sondern, dass seine subjektive Einschätzung \is{Bewertung} im Mittelpunkt steht. Es besteht deshalb kein erhöhter Bedarf, die Partikel ja, die ein cg-Update bewirkt, unmittelbar einzuführen.		
	
Der dritte Kontext, in dem die umgekehrte Abfolge zu finden ist, sind \is{modaler Kausalsatz} modal interpretierte Kausalsätze (vgl. (\ref{506}) bis (\ref{510})), d.h. Kausalsätze, die keine Sachverhalte, sondern Annahmen, Einstellungen oder Sprechakte begründen.

\begin{exe}
	\ex\label{506} 
	\scriptsize
	Terror, Afghanistan-Krieg, Koalitionskrise: \emph{Ärgert Sie}, dass das Jubiläum unter diesen Vorzeichen steht? \emph{Nein}\textbf{, \textit{denn} das ist \underline{doch ja} gerade das Spannende am Bundespresseball.} Er ist aufregend, weil es immer ein Überraschungsmoment gibt, da man nie weiß, unter welchen Vorzeichen der Abend stattfinden wird.	 		 
	\hfill\hbox{(unbekannt, in: Der Tagespiegel 2001-11-14, S. -1)}	
\end{exe}

\begin{exe}
	\ex\label{507} 
	\scriptsize
	Trotzdem nehm ich mich vom Vorwurf aus, verhätschelt zu sein, gewiss nicht :D \emph{Mir schlägt dieses Thema sauer auf}\textbf{, \textit{da} viele \underline{doch ja} heutzutage bis 25 nur Party, Spaß und Blödsinn im Kopf haben}, im Alter von 16,17,18 ganz zu schweigen.  		 
	\hfill\hbox{(DECOW2012-02: 355741630)}	
\end{exe}			 

\begin{exe}
	\ex\label{508} 
	\scriptsize
	\textbf{\emph{warum}} hat keiner mal vorher die Gerüchte über Krell Morat überprüft und untersucht, \textbf{\textit{wo} \underline{doch ja} 				anscheinend alle Daten vorhanden waren? }
	\newline
	\hbox{}\hfill\hbox{(http://www.scifi-forum.de/archive/index.php/t-227.html)}
	\newline
	\hbox{}\hfill\hbox{(eingesehen am 09.06.2012)}	
	\newline
	\hbox{}\hfill\hbox{\citet[203]{Mueller2014a}}	
\end{exe}	                 

\begin{exe}
	\ex\label{509} 
	\scriptsize
	Ich habe mir ja auch sagen lassen, dass man auch mit MS-Frontpage gute Homepages machen kann. Habt ihr schon mit dem Erfahrungen gemacht? 					\emph{Ist MS-Frontpage für ne ganz einfache Homepage vor Dreamweaver zu bevorzugen.} \textbf{\textit{Denn} MS-Frontpage verwendet 					\underline{doch ja} auch ne HTML-Oberfläche.}
	\newline
	\hbox{}\hfill\hbox{(http://www.informatik-forum.at/showthread.php?43549-Typo3-vs-Dreamweaver)}
	\newline
	\hbox{}\hfill\hbox{(eingesehen am 09.06.2012)}	
\end{exe}	                 

\begin{exe}
	\ex\label{510} 
	\scriptsize
	Wenn ich bloß ne Erklärung hätte für die Ursache, aber ich weiß es nicht:mauer: PS: Auf gute Freundschaft Pietbear und Danke Als Anhang noch den Kleinschaden: @ Heli-Player, mal so ne Frage und an alle natürlich auch. \emph{Muss da echt das Heckrohr gewechselt werden?} \emph{Wenn der Riemen noch schön läuft?} \textbf{Die Heckabspannung ist \underline{doch ja} auch nur für die Optik.}
	\hfill\hbox{(DECOW2012-02: 318143690)}
	\newline
	\hbox{}\hfill\hbox{\citet[230]{Mueller2017b}}	
\end{exe}	                 
In (\ref{506}) ist dies die Einstellung des Sprechers, sich nicht zu ärgern, in (\ref{507}) die Haltung, warum ihn das Thema sauer macht. In (\ref{508}) motiviert die Angabe, dass alle Daten vorhanden zu sein schienen, die Frage, warum die Überprüfung ausblieb. In (\ref{509}) ist die Information, dass MS-Frontpage auch eine HTML-Oberfläche verwendet, das Motiv für die vorweggehende Frage, ob MS-Frontpage zu bevorzugen sei. In beiden Beispielen fehlen die Fragezeichen, die Äußerungen sind im Kontext aber als Fragen zu erkennen. Und auch in (\ref{510}) begründet die letzte Aussage die Frage nach der Notwendigkeit des Austausches des Heckrohrs.

Der Unterschied der oben angesprochenen drei Typen von Kausalsätzen lässt sich anhand von (\ref{507a}) erklären.

\begin{exe}
	\ex\label{507a} 
	Peter bleibt zu /HAU\textbackslash se, // weil es so stark /REG\textbackslash net. 		 
	\hfill\hbox{\citet[265]{Bluehdorn2006}}	
\end{exe}
Nach Blühdorn kann man diesen Satz auf drei Arten lesen, die er \is{dipositioneller Kausalsatz} als \textit{dispositionell} (eine geläufigere Bezeichnung ist \is{propositionaler Kausalsatz} \textit{propositional}), \textit{epistemisch} \is{epistemischer Kausalsatz} und \textit{deontisch-illokutionär} \is{deontisch-illokutionärer Kausalsatz} bezeichnet (m.E. ist \textit{illokutionär} passender, weil die deontische Lesart sicherlich nicht stets vorliegt). Die jeweiligen Interpretationen werden durch die Paraphrasen in (\ref{508}) erfasst.	
 		                   							   
\begin{exe}
	\ex\label{508} 
		\begin{xlist}	
			\ex\label{508a} dispositionell/propositional\\
			Peter bleibt zu Hause, und der Grund dafür ist die Tatsache, dass es so stark  regnet.
			\ex\label{508b} epistemisch\\
			Ich bin überzeugt davon, dass Peter zu Hause bleibt, und der Grund für diese Überzeugung ist mein Wissen um die Tatsache, dass es so stark 					regnet.	
			\ex\label{508c} deontisch-illokutionär/illokutionär\\
			Ich ordne an, dass Peter zu Hause bleibt, und der Grund für diese Anordnung ist meine Bewertung der Tatsache, dass es so stark regnet.
			\hfill\hbox {\citet[265]{Bluehdorn2006}}			
		\end{xlist}
\end{exe}	
Die epistemische und illokutionäre Lesart werden als \textit{modale} Lesarten \is{modaler Kausalsatz} zusammengefasst, im Gegensatz zur \textit{nicht-modalen} propositionalen Lesart (vgl. \citealt[265-266]{Bluehdorn2006}).

Es ist also die prominente Funktion der modalen Kausalsätze, ihren Bezugssatz zu begründen und zu motivieren. Oberstes Ziel ihres Auftretens ist aber nicht, zu bewirken, dass Einigung zwischen den Beteiligten in Bezug auf ihren eigenen Inhalt hergestellt wird. Wenngleich es ihre Funktion ist, zu stützen (womit einhergeht, dass keine neue Kontroverse geschaffen werden soll), steht die Vorgänger\-äußerung im Zentrum. In den Fällen, in denen im Vorgängersatz tatsächlich eine Proposition vorliegt, hinsichtlich derer eine gemeinsame Übereinkunft beabsichtigt ist, wäre der Sprecher vermutlich auch zufrieden, wenn der Hörer diese annähme – auch wenn er die Evidenz oder das Motiv zurückweisen würde. Dies sind andere Umstände als sie bei den propositional interpretierten Kausalsätzen mit Ursache-Wirkung-Relation vorliegen. In diesem Fall ist das kommunikative Ziel nämlich, tatsächlich p und q, d.h. den begründeten und begründenden Sachverhalt, zu geteiltem Wissen zu machen. Der Sprecher erhebt hier einen Wahrheitsanspruch auf p und q (vgl. auch \citealt[140]{Pasch1999}). Die modalen Kausalsätze tun das, was Äußerungen tun, denen man zuschreibt, \textit{illokutiv subsidiär} \is{illokutiv subsidiär} zu sein (vgl. \citealt[58]{Motsch1987}, \citealt[21]{Brandt1992a}, \citealt[54]{Pittner2007}, \citealt[178]{Pittner2011} zu diesem Konzept). Sie dienen der Erfolgssicherung eines anderen \is{Illokutionstyp} Illokutionstyps, womit genau meine Einschätzung aufgefangen wird, dass der Fokus auf der Äußerung liegt, für die die Kausalsätze Begründungen/Motive anführen und nicht auf ihnen selbst.
	
Für den Eigenbedarf dieser Äußerungen ist es – wie auch in den Kontexten 1 (Bewertungen) und 2 (epistemische Modalisierungen) – wieder nicht so entscheidend, dass \textit{ja} früh in der Kombination eingeführt wird: Die Einigung auf den Inhalt des Kausalsatzes ist nicht das prominente Diskursziel. Dies ermöglicht die Umkehr der Abfolge. 
	
In allen drei angeführten Fällen steht gerade nicht die Sachverhaltsbeschreibung im Mittelpunkt, sondern das Gewicht liegt auf der Sprecherhaltung. Es handelt sich in allen drei Fällen letztlich um epistemische Kontexte: Es liegt eine als solche markierte Bewertung/Einschätzung des Sprechers vor. Diese kann sich inhaltlich ergeben, die Sätze können epistemisch modalisiert sein oder es können modal interpretierte Kausalsätze auftreten. Die Kontexte weisen folglich alle einen gewissen Grad an Subjektivierung \is{Subjektivierung} auf.

Vorausgesetzt, für prototypische Assertionen gilt, dass der Sprecher sich zur Wahrheit des ausgedrückten Sachverhalts bekennt und die Absicht verfolgt, ihren Inhalt zu geteiltem Wissen zu machen (vgl. (\ref{509}) und (\ref{510})), teilen Äußerungen, in denen die Abfolge \textit{doch ja} zu finden ist, die Eigenschaft, keine prototypischen Assertionen zu sein.

\begin{exe}
	\ex\label{509} 
		S($[\textrm{D}], \textrm{a}, \textrm{K}_{\textrm{i}}) = \textrm{K}_{\textrm{o}}$ so that
		\begin{xlist}	
			\ex\label{509a} $\textrm{DC}_{\textrm{a,o}} = \textrm{DC}_{\textrm{a, i}} \cup \lbrace\textrm{p}\rbrace$
			\ex\label{509b} $\textrm{T}_{\textrm{o}} = \textrm{push}(\langle \textrm{S}[\textrm{D}]; \lbrace \textrm{p} \rbrace \rangle, \textrm{T}_{\textrm{i}})$
			\ex\label{509c} $\textrm{ps}_{\textrm{o}} = \textrm{ps}_{\textrm{i}} \ \overline\cup \ \lbrace \textrm{p} \rbrace$
			\hfill\hbox {\citet[92]{Farkas2010}}
		\end{xlist}
\end{exe}

\begin{exe}
	\ex\label{510} 
	Nachdem ein Sprecher eine Assertion mit Proposition p geäußert hat, gilt:
		\begin{xlist}	
			\ex\label{510a} Die neue Diskursbekenntnismenge des Sprechers beinhaltet p.
			\ex\label{510b} Die Proposition p (vs. $\neg$p) wird oben auf den Stapel des alten Tisches gelegt.
			\ex\label{510c} Die projizierte Zukunft des alten cg beinhaltet p (unter Bewahrung der Konsistenz des cg).
		\end{xlist}
\end{exe}
Es geht folglich nicht prominent darum, den Gesprächspartner von dieser Einschätzung zu überzeugen. Und wenn Einigung nicht das oberste Ziel ist (was \textit{ja} unmittelbar bewirkt), ist auch motiviert, warum es in der Kombination nicht vorn stehen muss. In allen drei Fällen ist aber nicht ausgeschlossen, dass der Sprecher dem übergeordneten Diskursprinzip (vgl. (\ref{511})) folgt und einen stabilen Kontextzustand herstellt, sobald es ihm möglich ist. 

\begin{exe}
	\ex\label{511} Zwei fundamentale Antriebe für Gespräche
		\begin{xlist}	
			\ex\label{511a} Erweiterung des cg
			\ex\label{511b} Herstellung eines stabilen Kontextzustands
			\hfill\hbox {\citet[87]{Farkas2010}}
		\end{xlist}
\end{exe}
In diesem Fall wählt er die unmarkierte Abfolge \textit{ja doch}.
								         
\section{Der Status der Abfolge \textit{doch ja}}
\label{sec:status}
Da sich meines Wissens nach noch kein Autor für die Existenz der umgekehrten Abfolge \textit{doch ja} ausgesprochen hat und ohnehin äußerst selten darauf hingewie\-sen wird, dass die Sequenzierungen nicht absolut sind (vgl. z.B. \citealt[289]{Thurmair1989}), bin ich über Kritik nicht überrascht. Ich möchte in diesem Kapitel deshalb einige Aspekte zum Status der Abfolge \textit{doch ja} ausführen. In Abschnitt~\ref{sec:ort} geht es insbesondere um den Fundort der Daten, d.h. die Frage, welche Bedeutung (positiv oder negativ) der Tatsache zuzuschreiben ist, dass viele Belege aus Webdaten stammen.\footnote{Ich danke den Gutachtern zu \citet{Mueller2017b} für einige dieser potenziellen Einwände.} Wenngleich ich von der Existenz der Abfolge \textit{doch ja} überzeugt bin und besagte Einwände nur für bedingt berechtigt halte, gibt es darüber hi\-naus auch Lücken, auf die ich verweisen möchte, weil ich sie mit meinen bisherigen empirischen Bemühungen noch nicht vollends schließen kann und deshalb wei\-tere Untersuchungen nötig sind. Abschnitt~\ref{sec:akz} präsentiert anschließend die Ergebnisse einer Akzeptabilitätsstudie, die zu einer der vorher aufgeworfenen Fragen einen Beitrag leistet.

\subsection{Der Fundort}
\label{sec:ort}
Mein Hauptargument für die Behauptung der Existenz der umgekehrten Abfolge von \textit{ja} und \textit{doch} ist entscheidenderweise, dass sich bei der Durchsicht der aufgefundenen Belege eine Systematik in Form der drei angeführten Kontexte offenbart. Aus diesem Grund greift meiner Ansicht nach der Einwand nicht, dass die Daten im Wesentlichen ‚nur‘ aus Webdaten stammen und deshalb als Unachtsamkeiten, Schlampigkeit oder Performanzfehler abgetan werden können, da sich im Internet sowieso nahezu alles finden lässt, wenn man nur danach sucht. Egal, um welches Phänomen es sich handelt, eine Systematik sollte aufhorchen lassen.

Man sieht, dass die Daten mitunter abweichende Orthografie und fehlende Interpunktion aufweisen sowie umgangssprachlichen Stil haben. Ich halte diesen Umstand nicht für ein wirkliches Problem, sondern eher einen Aspekt, den es in Kauf zu nehmen gilt, wenn man ein seltenes Phänomen der (konzeptionellen) Mündlichkeit untersuchen möchte. Scheinbar \glq seriösere\grq {} Alternativen wie Zei\-tungssprache oder Protokolle von politischen Reden/Debatten, die diesen \glq Makel\grq {} nicht aufweisen, eignen sich für meine Fragestellung nicht besser. MPn sind ein Phäno\-men der konzeptionellen Mündlichkeit und treten deshalb in der gesprochenen Umgangs\-sprache auf bzw. da, wo diese simuliert werden soll. Die einzig sinn\-volle Alternative wären gesprochene Daten. Hierbei hat man es aber zum einen ebenso mit \glq unsaubererer\grq {} Sprache zu tun, und zum anderen kann man auch nicht auf Orthografie und Interpunktion bauen. Ein zusätzliches Problem, das sich beim Arbeiten mit gesprochenen Daten einstellt, ist, dass diese im Ver\-gleich zu zugängli\-chen geschriebenen Daten in viel geringerer Anzahl vorliegen. Da MP-Kombina\-tionen an sich schon seltener sind als man meinen mag, ist es umso schwieriger, für eine markierte Struktur bei einem ohnehin markierten Phänomen genügend Belege zu finden. Die Datenmenge muss zunächst einmal genug \textit{ja doch}s zu Tage fördern, bevor mit einem \textit{doch ja} zu rechnen ist. Im \textit{FOLK}-Korpus der DGD2 (Gespräche von 34,5 Stunden zwischen 2007–2011) gibt es zwei relevante \textit{ja doch}-Treffer, im \textit{Wendekorpus} (Gespräche zwischen 1993–1996) finden sich 22 Belege und im \textit{Freiburger Korpus} (Gespräche 68 Stunden zwischen 1960–1974) sind 37 \textit{ja doch}-Belege enthalten. Auch im \textit{Dortmunder Chat-Korpus}, das mit medial schriftlichen Daten bei konzeptioneller Mündlichkeit den Webdaten mehr entspricht als viele andere Korpora, findet man nur eine \textit{ja doch}-Äußerung.\footnote{Der Angabe dieser Zahlen geht natürlich die Betrachtung der Strukturen im Kontext bzw. eine Überprüfung der Hörbelege voraus, um gleichlautende Formen (insbesondere akzentuiertes \textit{doch}) auszuschließen.} Es liegt nahe, auf dieser Schiene gegen die Daten argumentieren zu wollen. Wenngleich natürlich jedes Vorgehen auch Mankos mit sich bringt (derer man sich bewusst sein sollte), bin ich der Meinung, dass man gerade das Potenzial dieser noch wenig genutzten Daten sehen sollte. Webdaten geben ein realistisches Bild von konzeptioneller Mündlichkeit \is{konzeptionelle Mündlichkeit} im geschriebenen Medium ab. Möchte man herausfinden, wie Spre\-cher Strukturen in authentischen Kontexten verwenden, muss man die obigen \glq Makel\grq {} der Daten wohl in Kauf nehmen. Dazu kommt, dass es auch sein kann, dass man es hier mit sprachlichen Veränderungen zu tun hat (s.u.). Unter dieser Prämisse liegt dann auch ein Genre vor, in dem man die Struktur beobachten sollte, bevor man sie in Zeitungen oder Protokollen findet. Nicht zuletzt erlauben Webdaten das Durchsuchen sehr großer Datenmengen, was in diesem Fall unbedingt vonnöten ist. 

Dass man die umgekehrten Abfolgen hier findet, ist meiner Ansicht nach folg\-lich nicht auf einen umgangssprachlichen Stil oder eine schlampige Sprache zurückzuführen, die man im Internet vermutet. Wer die Umgangssprache so sehr als negativen Aspekt wertet, hat möglicherweise auch eine falsche Vorstellung davon, wie im Internet geschrieben wird. Einerseits kann man von den umgangssprachli\-chen, orthografisch abweichenden und interpunktionsarmen Belegen nicht behaupten, dass sie schwer ungrammatisch und unverständlich sind, weshalb sie nicht als Beispiele dienen können. Nur weil man Webdaten verwendet, verliert man nicht seinen Sachverstand, völlig abgebrochene Sätze etc. nicht zum Gegenstand der Betrachtung zu machen. Dies gilt analog für gesprochene Sprache. Andererseits werden mit Webkorpora oder auch schlichten Google-Suchen auch nicht nur Foren oder Wikipedia-Diskussionen durchsucht (woran man zunächst denken mag), sondern genauso sonstige Erzähltexte, Berichte, Artikel und Interviews, in denen MPn eben auftreten. Es gibt folglich auch innerhalb dieses Teilgenres Unterschiede zwischen verschiedenen Graden an \is{Umgangssprache} Umgangssprache, in der MPn vorkommen. Kritik allein auf der Ebene der Qualität der Daten möchte ich folglich zurückweisen. 

Es ist aber sicherlich so, dass das Phänomen der MPn/MP-Kombinationen stärker als andere Untersuchungsgegenstände für falsche Einschätzungen anfällig ist. Dieser Aspekt gilt allerdings relativ unabhängig vom verwendeten Datentyp.

Ein Klassiker, mit dem man sich bei der Untersuchung von MPn stets konfrontiert sieht, ist, dass man die Bestandteile der Kombination als MPn behandle, obwohl gleichlautende Formen anderer Wortarten vorlägen. Die einfachste Variante, gegen die Existenz der \textit{doch ja}-Abfolge zu argumentieren, ist aus dieser Perspektive, zu sagen, dass das auftretende \textit{doch} das betonte Adverb \is{Adverb} ist. \citet[209]{Thurmair1989} schreibt über \textit{ja doch}, dass schwer zu entscheiden ist, ob zwei MPn vorliegen. Sie meint sogar, dass das betonte \textit{doch} häufiger auftritt und die Kombination zweier Partikeln hier selten sei. Ich glaube, dass es generell fast unmöglich ist, das Vorkommen von \glq Dubletten\grq {} in schriftlich vorliegenden Daten auszuschließen. Es ist deshalb ebenso nahezu unmöglich, zu behaupten, dass man die Adverb-Verwendung von \textit{doch} ausschließen kann, und zwar in der Abfolge \textit{ja doch} wie in der Reihung \textit{doch ja}. Je nach MP-Kombination ist dieser Ausschluss unterschiedlich schwierig. Einfacher wird es, wenn sich die Bedeutungen von MP und Non-MP ferner sind als im Falle des betonten und unbetonten \textit{doch}. Trotz dieser Problematik ist für mich – genauso wie im Falle von \textit{ja doch} – relevant, ob die MP-Lesart plausibel verfügbar ist. Wenn man weiß, dass die Formen in den unmarkierten MP-Abfolgen prinzipiell ambig sein können, kann man nicht für die umgekehrte Abfolge fordern, dass dies nicht so ist. Die Argumentation kann nicht so verlaufen, dass man sich in der umgekehrten Abfolge sicher ist, dass die Form \underline{nicht} die MP ist, während man der unmarkierten Ordnung potenzielle Ambiguität zuschreibt. Ich habe viele Belege angeführt, so dass es dem Leser frei steht, diese dahingehend durchzusehen, ob er sich sicher ist, dass es sich stets eindeutig um das betonte \textit{doch} handelt. Ich glaube dies nicht und halte die Lesart in manchen Fällen sogar für unmöglich. Ich gehe folglich nicht davon aus, dass dieser Aspekt, der die Beschäftigung mit MPn generell begleitet, die \textit{doch ja}-Belege erklärt. Das Problem lässt sich verkleinern mit gesprochenen Daten oder konstruierten Beispielen. Im ersten Fall findet man aus Frequenzgründen nicht genügend Treffer. Letztere eignen sich nicht gut, um für eine Struktur zu argumentieren, deren Existenz bisher abgesprochen wurde. Mein Punkt ist es, zu zeigen, dass beide Abfolgen verwendet werden (s.u. zu Akzeptabilitäts\-urteilen), weshalb auch auf Korpusdaten zurückgegriffen werden muss. In Kapitel~\ref{chapter:dua} wird sich bei der Beschäftigung mit Kombinationen aus \textit{doch} und \textit{auch} zeigen, dass die gleichlautenden Adverbien unbedingt im Blick behalten werden müssen. Im vorliegenden Fall halte ich das Intervenieren des Adverbs \is{Adverb} allerdings für weniger wahrscheinlich. Über \textit{doch halt}/\textit{eben} und \textit{halt}/\textit{eben doch} schreibt \citet[216]{Thurmair1989}, dass die Voranstellung von \textit{doch} mit seiner Unakzentuiertheit einhergeht, die Nachstellung mit der betonten Form.\footnote{Ich kann dies bestätigen für die Kombination aus \textit{doch} und \textit{auch}, deren Untersuchung dadurch erschwert wird, dass beide Partikeln eine Verwendung als Adverb haben. Das weiter rechts stehende Element wird i.d.R. als das Adverb interpretiert.} Die Frage wäre also, warum sich \textit{doch} in Verbindung mit \textit{ja} in dieser Hinsicht anders verhalten sollte. \citet[215-216]{Thurmair1989} nimmt zudem an, dass sowohl \textit{doch halt}/\textit{eben} als auch \textit{halt}/\textit{eben doch} zwei Partikeln enthalten können. Die MP-Lesart sei klarer beim vorangestellten \textit{doch}. 

Die Argumentation eines Kritikers kann nun natürlich weiterlaufen unter Berufung auf Autoren, die vertreten, dass Adverbien scrambeln können. Dann hätte man es hier in allen \textit{doch ja}-Belegen mit einem gescrambelten \is{Scrambling} Adverb zu tun. Diese Annahme scheint grundsätzlich auch nicht unsinnig aus der Perspektive, dass Autoren den Akzent auf \textit{doch} mit Kontrast \is{Kontrast} in Verbindung gebracht haben (wie z.B. \citealt{Meibauer1994}) und kontrastiertes Material im Mittelfeld umgestellt werden kann. Es ergibt sich dann aber die Frage, warum \textit{doch} gerade in diesen drei Kontexten scrambelt und nicht auch in Standardassertionen. Bringt man diesen Aspekt mit der obigen Annahme zu schlampiger Sprache im Internet zusammen, würde man hier auch einen Zusammenhang zu gescrambelten Strukturen aufmachen, den man vermutlich nicht eröffnen möchte. Da man auch hier sicherlich eher davon ausgehen möchte, dass diese Umstellung durch einen gewissen Informationsstatus/Prosodie etc. bedingt ist und nicht Performanzfehler vorliegen, sollte die Annahme um vorliegendes Scrambling den obigen Einwand zur Qua\-lität der Daten auch auflösen. Abgesehen davon, dass ich nicht glaube, dass stets das betonte \textit{doch} auftritt, scheint mir auch nicht genug über die Interaktion von Adverbien und MPn bekannt zu sein, um die Möglichkeit gescrambelter Adverbien als sicheres Gegenargument vertreten zu können. Ich halte die Datenlage hier eher für unklar (vgl. z.B. \citealt{Coniglio2007}). Auch gibt es durchaus Ansätze, die nicht von gescrambelten Adverbien, sondern umgestellten MPn ausgehen (vgl. \citealt{Coniglio2007}). Unter diesem Blickwinkel müsste man dann zunächst um die Basisposition von betontem \textit{doch} wissen, um entscheiden zu können, wie \textit{ja} relativ zu diesem positioniert werden kann.

Ein anderer Einwand zu \citet{Mueller2017b}) war, dass es sich nicht um die Sequenz \textit{doch ja} handelt, sondern \is{Konstruktion} um \textit{Konstruktionen}, in denen \textit{ja} mit dem folgenden Material eine Einheit bildet: \textit{xxx doch + ja xxx}. Ich möchte nicht gänz\-lich zurückweisen, dass Umstände dieser Art eine Rolle spielen können (s. auch meine Ausführungen zu \textit{auch doch} in Abschnitt~\ref{sec:distributionad} in Kapitel~\ref{chapter:dua}). Man müsste im Fall von \textit{ja} und \textit{doch} aber dann natürlich eine handhabbare Definition dazu geben können, dass/wann/ob/warum eine Konstrution (nicht) vorliegt. Schaut man auf die Daten, die ich angeführt habe, müssten ziemlich viele Abfolgen als Konstruktionen klassifiziert werden (\textit{ist-\textbf{doch}} + \textit{\textbf{ja}-wieder-typisch}, \textit{müssen-\textbf{doch}} + \textit{\textbf{ja}-vererbbar}, \textit{trifft-\textbf{doch}} + \textit{\textbf{ja}-eigentlich}, \textit{habe-ich-\textbf{doch}} + \textit{\textbf{ja}-beim-Grafikkartenwech\-sel}, \textit{möchte-\textbf{doch}} + \textit{\textbf{ja}-garantiert}, \textit{kann-die-\textbf{doch}} + \textit{\textbf{ja}-auch}, \textit{weil-mc-\textbf{doch}} + \textit{\textbf{ja}-solo} usw.). Ich halte das sprachliche Material, das ich aufgedeckt habe, für zu vielfältig, um hier von Konstruktionen zu sprechen. Aber natürlich ist es auch schwierig, zu sagen, wo ein/e Muster/Systematik endet und eine eingefrorene Konstruktion anfängt. Ich gehe gerade vom Kovorkommen bestimmten lexikali\-schen Materials aus, und diese sprachlichen Mittel sind z.T. auch an feste Positionen gebunden, so dass nicht weiter verwunderlich ist, dass \textit{ja} oftmals einem Adjektiv oder Adverb vorangeht. Dazu kommt, dass \textit{ja} dann auch nur über seine Konstruktion Skopus nehmen würde. Ich denke aber, dass beide Partikeln weiten Skopus \is{Skopus} nehmen; enger Skopus wäre für MPn generell sehr unüblich.

Aus der Diskussion ist mitzunehmen, dass aufgrund der manchmal unklaren Wortartzugehörigkeit/Funktionen der Elemente bei der Untersuchung von MPn sicherlich Vorsicht geboten ist. Dies spricht aber nicht gegen die Arbeit mit Webdaten. Genauso wenig ist die umgekehrte Abfolge eine Illusion, deren Nachweis sich anderweitig erklären lässt.\footnote{Vgl. auch \citet[232]{Mueller2017b} zu Gegenargumenten zum Vorschlag, es handle sich bei \textit{doch ja} um eine Selbst-Korrektur des Sprechers/Schreibers.}

In der Forschung wird von vielen Autoren (z.B. \citealt{Abraham1991b}, \citealt{Diewald1997}, \citealt{Wegener2002}) vertreten, dass MPn das Ergebnis eines Grammatikalisierungspro\-zesses sind. Mir ist keine Untersuchung zur Diachronie von MP-Kombinationen im Deutschen bekannt (zum Niederländischen vgl. \citealt{Hoeksema2008}). Ich sehe aber keinen Grund, warum hier nicht ebenfalls mit Veränderungen zu rechnen sein sollte. Bevor man die Untersuchung nicht vorgenommen hat, ist nicht auszu\-schließen, dass die umgekehrte Abfolge deshalb in den Webdaten so präsent ist, weil es sich um eine neue Entwicklung handelt. Potenzielle Ambiguität \is{Ambiguität} gilt als wichtiger Zwischenschritt in \is{Grammatikalisierung} Grammatikalisierungsprozessen (vgl. z.B. \citealt[137-138, 141, 144]{Diewald2008})  und ist demzufolge auch in den Kombinationen zu erwarten. Es ist aber auch möglich, die umgekehrte Abfolge in älteren Daten zu belegen. Die ältesten Belege, die ich gefunden habe, sind von 1545, vermehrte Tref\-fer habe ich ab dem 19. Jahrhundert finden können (vornehmlich über \textit{Projekt Gutenberg}, \textit{zeno.org} bzw. fanden sich Belege unter den DECOW-Treffern). (\ref{512}) bis (\ref{522}) zeigen einige Beispiele aus verschiedenen Jahrhunderten.

\begin{exe}
	\ex\label{512} 
	\scriptsize
	Aber vmb deines Namens willen / \emph{las vns nicht geschendet werden} / \emph{Las den Thron deiner Herrligkeit nicht verspottet 			werden} / \emph{Gedenck doch} / \emph{vnd las deinen Bund mit vns} / \emph{nicht auffhören}. 22 \textbf{Es ist \underline{doch ja} vnter der Heiden Götzen keiner} / der Regen künd geben / 
	\hfill\hbox{Jeremia 14.22}
\end{exe}	

\begin{exe}
	\ex\label{513} 
	\scriptsize
	\emph{Warumb stellestu dich} / \emph{als ein Helt der verzagt ist} / \emph{vnd als ein Rise} / \emph{der nicht 						helffen kan?} \textbf{Du bist \underline{doch ja} vnter vns HERR} / vnd wir heissen nach deinem Namen / verlas vns nicht. 
	\hfill\hbox{Jeremia 14.9}
\end{exe}	

\begin{exe}
	\ex\label{514} 
	\scriptsize
	Sie gebirt vnd hat doch keine wehe / als were sie nicht schwanger? \emph{Kan auch} / \emph{ehe denn ein Land die wehe kriegt} / 			\emph{ein Volck zu gleich geborn werden?} \textbf{Nu hat \underline{doch ja} Zion jre Kinder on die wehe geboren.}Jesaja 66.8			
	\hfill\hbox{(1545, Martin Luther: Luther-Bibel)}
	\newline
	\hbox{}\hfill\hbox{(eingesehen über http://www.bibel-online.net/)}
\end{exe}
\vspace{-0.65cm}	
\begin{exe}
	\ex\label{515} 
	\scriptsize
	O Jesu! der du hoch am Creutz stehst aufgericht:\\
	Mein Heyland! \emph{ach neig  ab dein blutig Angesicht!}\\
	\textbf{Ich bin \underline{doch ja} der Preiß}/ \textbf{um welchen du gerungen}/\\
	Als du durch deinen Tod hast meinen Tod verschlungen.
	\newline
	\hbox{}\hfill\hbox{(Literatur im Volltext: Andreas Gryphius: Gesamtausgabe der}
	\newline
	\hbox{}\hfill\hbox{deutschsprachigen Werke. Band 3, Tübingen 1963, S. 93-95.:}
	\newline
	\hbox{}\hfill\hbox{So walt es Gott) (eingesehen über zeno.org)}
\end{exe}           
	
\begin{exe}
	\ex\label{516} 
	\scriptsize
	Sie betrachteten diesen Mann, dem ein so großer Ruf vorangegangen war, \emph{vielleicht} nicht mit geringerem Interesse als wir, wenn wir die kaiserlichen oder königlichen Söhne des Mars die Dienste eines Feldherrn verrichten sahen. \textbf{\textit{Knüpft} sich \underline{doch ja} gerade an die Person eines ausgezeichneten Führers das Interesse, das dem ganzen Heer gilt}, ja, wir meinen oft, die Schlachten, von denen uns die Sage oder die öffentlichen Blätter erzählen, um so deutlicher zu verstehen, wenn wir uns die Gestalt des Heerführers vor das Auge zurückrufen können.
	\newline
	\hbox{}\hfill\hbox{(1826 $[$1981$]$ Wilhelm Hauff: Lichtenstein – Kapitel 10)}
	\newline
	\hbox{}\hfill\hbox{(eingesehen über http://gutenberg.spiegel.de/)}
\end{exe}		
							  
\begin{exe}
	\ex\label{517} 
	\scriptsize
	Seine Sorgen hatte er zurückgelassen, sie folgten ihm nicht durch das Tor der Träume; nur liebliche Erinnerungen verschmolzen und mischten sich zu 			neuen reizenden Bildern; das Mädchen aus der St.-Séverin-Straße mit ihrer schmelzenden Stimme schwebte zu ihm her, und erzählte ihm von ihrer Mutter; 		\emph{er schalt sie, daß sie so lange auf sich habe warten lassen}, \textbf{\textit{da} er \underline{doch ja} den Ersten und Fünf\-zehnten gekommen sei}; er wollte sie küssen zur Strafe, sie sträubte sich, er hob den Schleier auf, er hob das schöne Gesichtchen am Kinn empor, und siehe – 		es war Don Pedro, der sich in des Mädchens Gewänder gesteckt hatte, und Diego sein Diener wollte sich totlachen über den herrlichen Spaß. 
	\newline
	\hbox{}\hfill\hbox{(1828 $[$1970$]$ Wilhelm Hauff: Novellen – Kapitel 31)}
	\newline
	\hbox{}\hfill\hbox{(eingesehen über http://gutenberg.spiegel.de/)}
\end{exe}	
						    	        
\begin{exe}
	\ex\label{518} 
	\scriptsize
	11$]$ \emph{Daher also geht mit Mir}, damit ihr zu eurem großen Troste solches alles ehedem erfahrt denn alle die andern in der Hütte der 			Purista; \textbf{\textit{denn} ihr habt für die Errettung der Tiefe vor dem Untergange Meines Wissens \underline{doch ja} auch in dieser Zeit am 			meisten und am lebendigsten zu Gott Tag und Nacht gefleht!}	
	\hfill\hbox{(1842 $[$k.A.$]$ Jakob Lorber: Die Haushaltung Gottes Bd.)}
	\newline
	\hbox{}\hfill\hbox{(DECOW2012-02: 263034439)}
\end{exe}								                        
											  
\begin{exe}
	\ex\label{519} 
	\scriptsize
	Ob eine solche Empfänglichkeit sich in Tat oder auch nur in Begierde äußert, hängt von Nebenumständen ab, Umständen, die von der Tugend unabhängig 			sind. \textbf{So viel \textit{wird} \underline{doch ja} Herr Dr. TAUBERT uns zugeben}, daß das Wesen der Tugend eine Gesinnung ist.	
	\newline
	\hbox{}\hfill\hbox{(1874 $[$k.A.$]$ Frederickn Anthony Hartsen: Die Moral des Pessimismus)}
	\newline
	\hbox{}\hfill\hbox{(DE2012-02: 858568178)}
\end{exe}						                        
		      
\begin{exe}
	\ex\label{520} 
	\scriptsize
	Er sollte gleich am ersten Tage herausfühlen, daß ich nicht so dumm sei, seine Natur, seine Gaben, seine Vorzüge zu knechten und zu knebeln. 				\textbf{\textit{Das war} \underline{doch ja} \textit{das beste Mittel}, diese Gaben und Vorzüge kennen zu lernen!}			
	\hfill\hbox{(1887 $[$k.A.$]$ Karl May: Deutsche Herzen, deutsche Helden)}
	\newline
	\hbox{}\hfill\hbox{(DECOW2012-02: 248350430)}
\end{exe}									 

\begin{exe}
	\ex\label{521} 
	\scriptsize
	Der erstaunte Graf tröstete sie freundlich, zeigte ihr den Grafenring und sagte, sie solle gleich das Hennenmädel in den Saal kommen lassen. – \glqq 		Aber, mein lieber Himmel! \textbf{\textit{die ist} \underline{doch ja} \textit{so garstig und schmutzig}!}\grqq{} meinte die Köchin.
	\newline
	\hbox{}\hfill\hbox{(1911 Ignaz und Josef Zingerle: Kinder- und Hausmärchen aus Tirol)}
	\newline
	\hbox{}\hfill\hbox{(DECOW2012-07: 750624)}
\end{exe}				             								            
  
\begin{exe}
	\ex\label{522} 
	\scriptsize
	\glqq Konstantin Dmitritsch,\grqq{} begann sie zu Lewin, \glqq sagt mir doch, ich bitte recht schön – was hat das zu 				bedeuten – \textbf{Ihr wißt 			\underline{doch ja} alles.} Bei uns in Kaluga haben alle die Bauern und alle die 			Weiber alles vertrunken, was sie hatten, und zahlen uns jetzt keine Steuern mehr. Was hat das zu bedeuten? Ihr lobt doch die 	Bauern sonst stets!\grqq{} – 
	\hfill\hbox{(1920 [1920] Lew Tolstoi: Anna Karenina – 1 – Kapitel 15)}
	\newline
	\hbox{}\hfill\hbox{ (eingesehen über http://gutenberg.spiegel.de/)}
\end{exe}					 
In (\ref{512}) bis (\ref{515}) und (\ref{522}) tritt \textit{doch ja} in nicht weiter sprachlich markierten V2-Deklarativsätzen auf, die plausibel als Begründungen des vorweggehenden Sprech\-aktes interpretiert werden können. In (\ref{512}), (\ref{515}) und (\ref{522}) ist dies ein Direktiv, in (\ref{513}) und (\ref{514}) gehen jeweils Fragen vorweg. Als Sprechaktbegründungen \is{illokutionärer Kausalsatz} lassen sich auch der \textit{da}- und \textit{denn}-Satz in (\ref{517}) und (\ref{518}) einstufen. In (\ref{517}) wird dann der wiedergegebene \is{Vorwurf} Vorwurf, in (\ref{518}) ebenfalls ein Direktiv \is{Direktiv} motiviert. In (\ref{516}) tritt ein V1-Satz auf, der die vorangehende Annahme (markiert durch \textit{vielleicht}) begründet. In (\ref{519}) liegt mit dem Modalverb \textit{werden} eine epistemische Modalisierung \is{epistemische Modalisierung} vor, während die \textit{doch ja}-Äußerungen in (\ref{520}) und (\ref{521}) als lexikalisch als solche markierten Bewertungen gelesen werden können. 

Ich habe nicht den Eindruck, dass hier jeweils klar das betonte \textit{doch} vorliegt. In (\ref{523}) habe ich einige weitere Werke gelistet, in denen ich \textit{doch ja}-Belege gefunden habe.
								                        
\begin{exe}
	\ex\label{523}
	\tiny
     \begin{tabular}[t]{|p{5em}|p{18em}|p{5em}|}
     		\hline
     		Jahr & Quelle & Kontext\\
            \hline
            1545 $[$1888$]$ & Martin Luther: Wider das Papsttum zu Rom, vom Teufel gestiftet & Bewertung\\
            \hline
            1582 $[$1845$]$ & Anonym: Das Ambraser Liederbuch vom Jahre 1582. $[$Der mond der scheint so helle$]$  & 							illokutionär kausal\\
            \hline
            1658 $[$1996$]$ & Andreas Gryphius: Absurda Comica Oder Herr Peter Squentz – Kapitel 6  & Bewertung\\
            \hline
            1789 (eigentlich: 300–700 n. Chr.) & Hermes Trismegistos (dt.): Das andere Buch, Hermetis: Das Gemüt am Hermes & 					illokutionär kausal\\
            \hline
            1817 $[$1984$]$ & E.T.A. Hoffmann: Das Gelübde – Kapitel 1 & illokutionär kausal\\
            \hline
            1838 $[$1898$]$ & Jeremias Gotthelf: Leiden und Freuden eines Schulmeisters & Bewertung\\
            \hline
            1839 $[$k.A.$]$ & Ludwig I. von Bayern: Gedichte – Kapitel 125 & illokutionär kausal\\
            \hline
            1847 $[$1991$]$ & Franz Grillparzer: Libussa – Kapitel 4  & epistemisch kausal\\
            \hline
            19. Jhd. (1810–1976) $[$k.A.$]$ & Ferdinand Freiligrath: Ein Glaubensbekenntnis & illokutionär kausal\\
            \hline
            1909 $[$1985$]$ & Robert Walser: Jakob von Gunten & illokutionär kausal\\             
            \hline
      \end{tabular}\\
\end{exe}
Auch hier ist Vorsicht geboten, da ich nicht die Erstausgaben eingesehen habe, sondern die in den genannten Korpora eingespeisten Ausgaben (soweit bekannt in (\ref{512}) bis (\ref{522}) und (\ref{523}) in Klammern hinzugefügt). Inwiefern hier ggf. Text geändert wurde, kann ich nicht kontrollieren. Für diese wenigen alten Belege gilt, dass selten tatsächlich explizit sprachliches Material beteiligt ist, wie ich es in Abschnitt~\ref{sec:markiert} identifiziert habe. Es verhält sich eher so, dass sich die Interpretation anbietet – aus der Perspektive der Kontexte, in denen ich die Abfolge in aktuelleren Daten finde. Sprachliches Material stellt man in diesen Beispielen erstmalig 1826 fest. 

Ob die umgekehrte Abfolge neu ist, kann ich im Rahmen dieser Arbeit nicht klären. Belegbar ist sie bereits in älteren Quellen. Möglicherweise hat man es auch mit einer Frequenzzunahme zu tun (vgl. auch \citealt[2]{Imo2010} zum Aufdecken \glqq neuer\grqq{} Daten im Sinne von \glqq neu für die traditionellen Grammatiken\grqq{}; vgl. auch \citealt[279]{Freywald2008} zur \textit{dass}-V2-Struktur). Diese Aspekte muss man gesondert untersuchen. 

Neben der historischen Dimension, die ein Forschungsdesiderat darstellt, gibt es auch aus synchroner Perspektive offene empirische Fragen. Insbesondere halte ich für untersuchenswert, inwiefern die Kontexte, in denen ich die umgekehr\-te Abfolge finde, wirklich genuin mit dieser Abfolge verbunden sind. Zu sagen, man findet \textit{doch ja} in diesen Kontexten, ist eine Annahme, die ich für richtig halte. Die sich anschließende schwierige Frage ist, wie man nachweist, dass sie sich im gegenpoligen Kontext, d.h. in Strukturen, in denen der Sprecher es darauf anlegt, den Inhalt zu geteiltem Wissen zu machen, nicht auftreten kann – anders als \textit{ja doch}, das eine weitere Verwendung vorweist (vgl. Abschnitt~\ref{sec:akz} zu einer Untersuchung, die diese Absicht über Akzeptabilitätsurteile verfolgt). Für diesen Äußerungskontext kommt vor allem die V2-Standardassertion in Frage. Da ich nicht glaube (wie vielfach angenommen wird), dass MPn auf periphere Nebensätze beschränkt sind (vgl. hierzu detaillierter Abschnitt~\ref{sec:rs} in Kapitel~\ref{chapter:hue}), stellen auch propositionale \textit{weil}-Sätze einen solchen Testkontext dar.

Man findet \textit{doch ja} nicht in Standardassertionen und man würde generell erwarten, dass diese häufig vorkommen – zumal modalisierte Strukturen im Grunde auch bereits markierte Assertionen sind. Meine bisherigen Untersuchungen können noch nicht ausschließen, dass das auffällige Auftreten in diesen drei Kontexten nicht durch den Datentyp bzw. die Kombination dieser beiden MPn bedingt ist. Betrachtet man Beispiele aus der Literatur, sucht Belege oder konstruiert neue Beispiele, kann \textit{ja doch} natürlich prinzipiell in mehr Kontexten als den subjektivierten auftreten. In diesen erscheint \textit{doch ja} in konstruierten Beispielen merkwürdig, was entsprechend zur Annahme seines Ausschlusses in der Lite\-ratur geführt hat. Ich habe nun allerdings Verwendungsweisen untersucht. Und aus der Annahme, dass \textit{ja doch} prinzipiell auch in non-modalisierten Kontexten gebraucht werden \underline{kann}, ist nicht zu folgern, dass es auch so verwendet \underline{wird}. Wahrscheinlich tritt auch diese Abfolge gerne in modalisierten Kontexten auf. Es ist verschiedentlich beobachtet worden, dass sich Modalisierungen auch anderweitig modale Umgebungen suchen (vgl. z.B. \citealt[26]{Albrecht1977}, \citealt[26]{Aijmer1997}, \citealt[278-279]{Bluehdorn2006}). Die modalisierten Kontexte sind wiederum ggf. auch an die Textsorte gebunden, in denen man überhaupt MPn findet oder möglicherweise auch genau diese beiden. Ich glaube, dies ist ein Aspekt, der oft nicht berücksichtigt wird, wenn Behauptungen über das Auftreten bestimmter Phänomene in bestimmten Domänen gemacht werden (je nach Phänomen/Domä\-ne und Datensatz ist dieser Aspekt auch mit überschaubarem Aufwand zu kontrollieren, vgl. Abschnitt~\ref{sec:gebrauchheeh}). Die Kombinationen aus \textit{ja} und \textit{doch} betreffend, halte ich es aber für sehr schwierig, diesen Aspekt aufzufangen. Um festzustellen, ob \textit{ja doch} nicht auch bevorzugt (vielleicht gar ausschließlich) in den drei Kontexten auftritt, müsste man eine sehr große Menge von \textit{ja doch}s durchsehen, weil diese Abfolge sehr viel häufiger in den Korpora vertreten ist als \textit{doch ja}. Stellt man hier einen Unterschied fest, ist immer noch nicht geklärt, ob das Auftreten in diesen Kontexten nicht dadurch bedingt ist, dass sie sehr häufig in diesen Textsorten zu finden sind. Dies herauszufinden halte ich für fast nicht praktizierbar. Stellt man keinen Unterschied fest, ist die Nicht-Existenz der \textit{doch ja}-Abfolge dennoch nicht nachgewiesen. Auf der Basis meines Vorgehens, alle auffindbaren \textit{doch ja}-Treffer in verschiedenen Korpora zu sammeln, ist es zudem im Grunde nicht möglich, aus diesen Korpora auch alle \textit{ja doch}s zu untersuchen. Stichproben zu untersuchen gestaltet sich auch schwierig, wenn mehr als ein Korpus die Belege für \textit{doch ja} liefert. Man bräuchte eine finite Menge von Daten, in denen man nach \textit{ja doch} und \textit{doch ja} suchen kann, und in denen ei\-nerseits genug \textit{doch ja}s auftreten und man es andererseits mit einer durchsehbaren Menge von \textit{ja doch}s zu tun hat. Diese Menge ist aber schwierig zu bekommen, da man die relevante Menge von \textit{doch ja}s erst erreicht, wenn man sich sehr große Datenmengen anschaut.

Man könnte die Untersuchung zu den beiden MP-Abfolgen folglich definitiv noch ausbauen. Es gilt bei diesem Vorhaben dann, die schwierige Frage zu beantworten, ob die Prominenz der drei Kontexte von a) der Textsorte (Sind die drei Kontexte nicht sowieso typisch für \is{konzeptionelle Mündlichkeit} konzeptionelle Mündlichkeit?) und b) dem Auftreten von \textit{ja} und \textit{doch} (Sind dies nicht auch beliebte Kontexte für \textit{ja doch}?) gelöst werden kann.

Weitere Evidenz könnte auch die Untersuchung der Partikel-Kombinationen mit Distanzstellung erbringen (vgl. z.B. (\ref{524}) und (\ref{525})).
	
\begin{exe}
	\ex\label{524} 
	Aber das \textit{\textbf{wird}} \underline{\textbf{doch}} beim Internisten \underline{\textbf{ja}} gar nicht lange dauern 			\textbf{\textit{oder?}}
	\newline
	\hbox{}\hfill\hbox{(DECOW2012-06X: 685575138)}
\end{exe}		

\begin{exe}
	\ex\label{525} 
	wenn das klappt \textbf{\textit{müsste}} sich das \underline{\textbf{doch}} im prinzip \underline{\textbf{ja}} auf jedes 			model von u1 bis mw ausweiten lassen, \textbf{\textit{oder?}}                                                 
	\hfill\hbox{(DECOW2012-06X: 999332683)}
\end{exe}		
Es wird manchmal angeführt, dass bei Distanzstellung \is{Distanzstellung} Kombinationen akzep\-tabler werden, die es bei Kontaktstellung \is{Kontaktstellung} nicht sind (vgl. \citealt[219]{Dahl1988}). Es wäre eine Untersuchung wert, zu schauen, ob sich die drei Kontexte auch dann als Domäne herauskristallisieren, wenn die Abfolge auf Distanz ohnehin weniger Restriktionen unterliegt. Die beiden oben aufgeworfenen Fragen stehen dann aber immer noch im Raum.

Mit der bisher geleisteten Empirie kann ich noch nicht alle diese Fragen beantworten. Die Ausführungen stellen den derzeitigen Stand meiner Untersuchungen dar.

Der nächste Abschnitt stellt eine Akzeptabilitätsstudie\footnote{Ich danke Thomas Weskott für seinen Rat bei der Planung des Experiments sowie Sandra Pappert für ihre Hilfe bei der statistischen Auswertung.} vor, die mit der Absicht durchgeführt wurde, der obigen Frage nach dem ausgeschlossenen Kontext der \textit{doch ja}-Abfolge nachzugehen.

\subsection{Akzeptabilitätsurteile}
\label{sec:akz}
Getestet wurden ein \glq besserer\grq {} und ein \glq schlechterer\grq {} Kontext, d.h. in einem Kontext sollte sich die Abfolge schlechter umkehren lassen als im anderen. 70 deutsche Muttersprachler haben Urteile zu \textit{ja doch}- und \textit{doch ja}-Sätzen auf einer 5er Skala abgegeben. Der \glq schlechte\grq {} Kontext wurde durch eine Standardassertion \is{Standardassertion} repräsentiert, d.h. völlig unmarkierte $[$-w$]$, V2-Sätze. Insbesondere liegt keine Modalisierung vor und der Kontext schließt auch kausale Interpretationen aus. Epistemische oder illokutionäre Begründungen \is{epistemischer Kausalsatz} \is{illokutionärer Kausalsatz} bieten sich nicht an. Die Standardassertion zeichnet sich gerade dadurch aus, dass keine besonderen Markierungen vorliegen. Die Testitems sind so konstruiert, dass die MP-Äußerung auf eine Implikatur \is{Implikatur} aus dem Vorgängerkontext reagiert, d.h. es liegt ein typischer Kontext für eine \textit{doch}-Äußerung vor. Die Partikel \textit{ja} tritt hier in der Akkommoda\-tions-Verwendung \is{Akkommodation} auf. (\ref{528}) und (\ref{529}) zeigen zwei Testitems (mit den Implikaturen in (\ref{530}) und (\ref{531})).

\begin{exe}
	\ex\label{528} 
	\begin{tabular}[t]{p{2em} p{25em}}
	\multicolumn{2}{l}{D1 Ein wichtiges Spiel}\\
    Olli: & Am Samstag steigt das wichtigste Spiel der Saison. Der Kader ist zum Glück komplett. Alexander spielt im Sturm.\\
	Stefan: & \underline{Er ist ja doch noch verletzt}.\\
	& \underline{Er ist doch ja noch verletzt.} \\		
    \end{tabular}
\end{exe}

\begin{exe}
	\ex\label{529} 
	\begin{tabular}[t]{p{3em} p{25em}}
	\multicolumn{2}{l}{D8 Kochen}\\
    Marina: & Die Tomate dort auf dem Tisch kannst du auch noch waschen, schneiden und dann in den Salat geben.\\
	Sophie: & \underline{Sie ist ja doch ganz verfault.}\\
	& \underline{Sie ist doch ja ganz verfault.}\\	
    \end{tabular}
\end{exe}
	
\begin{exe}
	\ex\label{530}
	Alexander spielt. $>$ Er ist nicht verletzt.\\
	\glq Wenn man spielt, ist man normalerweise nicht verletzt.\grq {}
\end{exe}		
		
\begin{exe}
	\ex\label{531}
	Tomate zum Essen geben. $>$ Die Tomate ist nicht verfault.\\
	\glq Wenn eine Tomate mit in den Salat soll, ist sie normalerweise nicht verfault.\grq {}
\end{exe}
Das Material bestand aus acht Lexikalisierungen, wobei die Testsätze dem Muster in (\ref{532}) entsprechen.

\begin{exe}
	\ex\label{532}
	Pronomen $\plus$ ist $\plus$ \textit{ja doch}/\textit{doch ja} $\plus$ schwaches Element (\textit{noch}/\textit{ganz}) + 			Partizip Perfekt (zweite Silbe akzentuiert)
\end{exe}
Hierbei handelt es sich meiner Argumentation nach um einen \glq schlechten\grq {} Kontext, da eine normale Assertion vorliegt. Vor dem Hintergrund der Charakterisie\-rung nach \citet{Farkas2010} stellt sich die Situation ein, dass der Sprecher p mitzuteilen beabsichtigt und den Diskurspartner von dieser Information über\-zeugen möchte. In diesem Kontext sollte die Abfolge der beiden Partikeln folglich nicht (bzw. schlechter als an anderer Stelle) umkehrbar sein. Der \glq gute\grq {} Kontext wird durch illokutionär interpretierte Kausalsätze \is{illokutionärer Kausalsatz} dargestellt, d.h. durch Kausalsätze, die einen Sprechakt \is{Sprechakt} begründen: Der Sprecher nennt das Motiv, aufgrund dessen er den Sprechakt äußert. Für die möglichst eindeutige Modellierung derart interpretierter Kausalsätze habe ich mir die Erkenntnis aus der Literatur zu Nutze gemacht, dass mit manchen Konnektoren bestimmte kausale Lesarten einhergehen. Mit ihnen können nur Interpretationen auf manchen der drei Ebenen, auf denen kausale Zusammenhänge prinzipiell möglich sind, kodiert werden. Entscheidend ist hier, dass angenommen wurde, dass \textit{denn} keine propositionale Verwendung hat, d.h. es wirkt nicht auf der Sachverhaltsebene (vgl. \citealt[320]{Volodina2010}), bzw. – selbst wenn der Sachverhaltsbezug nicht generell auszuschließen ist (vgl. \citealt[141]{Schmidhauser1995}) – bevorzugt es zumindest die \is{epistemischer Kausalsatz} \is{illokutionärer Kausalsatz} epistemische/illokutionäre Lesart (vgl. \citealt[270]{Bluehdorn2006}; \citeyear[29]{Bluehdorn2008}). Wenn zusätzlich ein non-assertiver Sprechakt vorangeht, liegt sehr deutlich die illokutionäre Lesart vor.

(\ref{533}) und (\ref{534}) zeigen auch für diesen Kontext je ein Item.

\begin{exe}
	\ex\label{533} 
	\begin{tabular}[t]{p{3em} p{22em}}
	\multicolumn{2}{l}{C1 Go-Go-Tänzerin}\\
    Janina:	& Mein Chef hat herausbekommen, dass ich im Club frei\-tags abends als Go-Go-Tänzerin arbeite. Er hält das für unseriös und droht mit der 			Kündigung, wenn ich diese Tätigkeit nicht aufgebe. Jetzt habe ich aber schon zugesagt, am Wochenende auf der Bühne zu stehen. Was soll ich jetzt nur 		machen?\\
	Kristina: & Na, hör mal. Sag den Auftritt ab!\\
	& \underline{Denn du riskierst ja doch sonst deinen Job.}\\
	& \underline{Denn du riskierst doch ja sonst deinen Job.}
    \end{tabular}
\end{exe}

\begin{exe}
	\ex\label{534} 
	\begin{tabular}[t]{p{2em} p{25em}}
	\multicolumn{2}{l}{C5 Ansehen}\\
    Nina: & Erst gestern habe ich erfahren, dass Erik Gruber im Ort als jemand gilt, der jede Frau abschleppt und anschließend 			sofort wieder fallen lässt. Von dem sollte man sich also fernhalten, wenn man nicht ins Gerede kommen möchte. Nun habe ich 			aber nichts ahnend zugestimmt, als er mich für Freitag zum Cocktailtrinken eingeladen hat. Was soll ich jetzt nur machen?\\
	Ida: & Na, hör mal. Sag das Date ab!\\
	& \underline{Denn du ruinierst ja doch sonst deinen Ruf.}\\
	& \underline{Denn du ruinierst doch ja sonst deinen Ruf.}
    \end{tabular}
\end{exe}
Inhaltlich folgen die Lexikalisierungen für diesen Kontext dem Muster, dass der erste Sprecher ein Problem mitteilt und den Gesprächspartner fragt, was er tun soll. Der zweite Sprecher rät von der geplanten problematischen Handlung ab und begründet, warum er dies rät. Strukturell weisen die zu bewertenden Sätze (bzw. ihr unmittelbarer Vorkontext) die Form in (\ref{535}) auf.

\begin{exe}
	\ex\label{535} 
	(Was soll ich jetzt nur machen? – Na, hör mal. Sag X ab!\\
	\textit{Denn} $\plus$ \textit{du} $\plus$ finites Verb auf $\lbrace$-ierst$\rbrace$ (letzte Silbe akzentuiert) \textit{ja doch}/\textit{doch ja} $\plus$ \textit{sonst} $\plus$ \textit{deinen} $\plus$ Nomen (einsilbig)
\end{exe}
Da die beiden getesteten Kontexte sich gerade interpretatorisch und sprachlich unterscheiden müssen, ist die Bildung von ,echten\grq {} lexikalischen Sets unmöglich. Die Testzusammensetzung einer jeden Version ist beschaffen wie in (\ref{536}).

\begin{exe}
	\ex\label{536} Items pro Testant\\[-0.6em]
     \begin{tabular}[t]{|l|l|l|}
     		\hline
     		 \diagbox{Satzkontext:}{Partikelfolge:} &
     		 \textit{ja doch} & \textit{doch ja}\\
            \hline
            Standardassertion & 4 & 4\\
            \hline
            Sprechaktbegründung & 4 & 4\\
            \hline
      \end{tabular}\\
\end{exe}
Jeder Testant hat vier verschiedene Sätze in jeder der vier Bedingungen gesehen, d.h. insgesamt 16 Testitems. Dazu kamen 32 andere Dialoge. Unter den Fillern fanden sich ganz verschiedene Phänomene (aus den Bereichen Wortstellung, Nebensätze, Korrelate, Extraktionsstrukturen, Satzmodus, Flexion), für die zudem Bewertungen von \glq hart schlecht\grq {}, \glq schlecht\grq {} , \glq leicht schlecht\grq {}  und \glq gut\grq {} anzunehmen sind. Auf ähnliche Art (an die Darbietung der Testitems und Aufgabenstellung/Art der Bewertung angepasst) waren sie Teil der in \citet{Mueller2012} beschriebenen Studie zu w-Fragen mit imperativischer Verbmorphologie. Einige Sätze sind auch unter Bezug auf ein sechsstufiges Ranking, das in \citet{Featherston2009} getestet wurde, konstruiert. Ich gehe deshalb davon aus, dass die Filler nicht nur bestimmte Stufen der Skala abdecken.

(\ref{537}) zeigt die arithmetischen Mittel für alle Bedingungen über alle 70 Teilnehmer der Studie. Die Urteile erfolgten auf einer 5er Skala, wobei \glq 1\grq {} das untere Ende der Skala darstellte (\glq völlig inakzeptabel\grq {} ) und \glq 5\grq {}  das obere Ende (\glq völlig akzeptabel\grq {}).

\begin{exe}
	\ex\label{537} Arithmetisches Mittel\\[-0.6em]
     \begin{tabular}[t]{|l|l|l|l|}
     		\hline
     		70 Teilnehmer & \textit{ja doch} & \textit{doch ja} & Unterschied\\
            \hline
            Standardassertion & 2,44 & 1,86 & $-$0,58\\
            \hline
            Sprechaktbegründung & 2,51 & 2,28 & $-$0,23\\
            \hline
            Unterschied & $+$0,07 & $+$0,42 & \\
            \hline
      \end{tabular}\\
\end{exe}
Im schlechten Kontext beträgt der Unterschied zwischen \textit{ja doch} und \textit{doch ja} 0,58, im guten Kontext liegt er bei 0,23.

Wenngleich die Sätze insgesamt recht niedrige Urteile erhalten haben (dazu s.u.), zeigen die Bewertungen dennoch, dass der Unterschied zwischen der Abfolge \textit{ja doch} und \textit{doch ja} in der Sprechaktbegründung \is{illokutionärer Kausalsatz} kleiner ist als in der \is{Standardassertion} Standardassertion. Sprecher können die Abfolge im gutem Kontext scheinbar etwas besser umdrehen als im schlechten. Darüber hinaus liegt ein sehr kleiner Unterschied von 0,07 zwischen den \textit{ja doch}-Auftreten in den zwei Kontexten vor und ein Unterschied von 0,42 zwischen den \textit{doch ja}-Äußerungen in den beiden Kontexten. Sprecher akzeptieren \textit{doch ja} also scheinbar etwas leichter im guten als im schlechten Kontext.

Eine zweifaktorielle Varianzanalyse mit Messwiederholung ergab sowohl für die by-subject- als auch die by-item-Analyse einen Effekt von \glqq Kontext\grqq{} (F$_{1}$(1,69) = 10,24, p $<$ 0,01, $\eta_{\textrm{p}}^{2}$ = 0,13; F$_{2}$(1,14) = 21,28, p $<$ 0,001, $\eta_{\textrm{p}}^{2}$ = 0,60), \glqq Abfolge\grqq{} (F$_{1}$(1,69) = 44,04, p $<$ 0,001, $\eta_{\textrm{p}}^{2}$ = 0,39; F$_{2}$(1,14) = 64,33, p $<$ 0,001, $\eta_{\textrm{p}}^{2}$ = 0,82) und der Interaktion von \glqq Kontext\grqq{} und \glqq Abfolge\grqq{} (F$_{1}$(1,69) = 7,71, p $<$ 0,01, $\eta_{\textrm{p}}^{2}$ = 0,10; F$_{2}$(1,14) = 11,39, p $<$ 0,01, 
$\eta_{\textrm{p}}^{2}$ = 0,45).

Interessant für meine Argumentation ist insbesondere die Feststellung, dass die Interaktion der beiden Faktoren einen Effekt auf die Bewertung nimmt. Aufgrund dieser signifikanten Interaktion wurde ein Scheff\'{e}-Test (p $<$ 0,05) als Post-hoc-Test kalkuliert. 

Der Test gibt aus, dass \textit{ja doch} in Standardassertionen signifikant besser be\-wertet wird als \textit{doch ja} in Standardassertionen und dass \textit{ja doch} in illokutionär interpretierten Kausalsätzen höhere Bewertungen erhält als \textit{doch ja} in illokutionären Kausalsätzen. Dies sind die Verhältnisse, wie sie zu erwarten sind, wenn zwi\-schen \textit{ja doch} und \textit{doch ja} ein Markiertheitsunterschied \is{Markiertheit} besteht. Der für meine Hypothese wichtige Aspekt ist, dass ebenfalls \textit{doch ja} in Sprechaktbegründungen \is{illokutionärer Kausalsatz} besser bewertet wird als in \is{Standardassertion} Standardassertionen, während sich der Unterschied zwischen \textit{ja doch} in den beiden Kontexten nicht als signifikant herausstellt. Die Ergebnisse der Studie zeigen auf, dass die Unterschiede subtil sind\footnote{M.E. lehren die feinen Unterschiede auch, dass sich zwischen der unmarkierten und markierten Partikelabfolge keine sehr stark voneinander abweichenden Urteile einstellen, die zu der Annahme veranlassen müssten, dass die umgekehrte Abfolge im Gegensatz zur üblicheren Ordnung gar als ungrammatisch angesehen werden muss. Wie der Blick auf andere Ansätze in Abschnitt~\ref{sec:abfolgejd} gezeigt hat, ist dies aber eine Ansicht, die den Arbeiten zu Partikelabfolgen in der Regel zugrunde liegt. Letztlich scheint mir aber auch hier Vorsicht geboten, weil Unterschiede auch je nach Methode ggf. unterschiedlich deutlich erscheinen können. D.h. ein Unterschied, der auf der 5-Punkt-Skala klein aussieht, kann in Paarvergleichen ggf. nach einem deutlicheren Kontrast aussehen.} und \glq Abfolge\grq {}  klar ein dominanter Faktor ist. Doch da die umgekehrte Abfolge nur gefunden werden kann, wenn sehr große Datenmengen betrachtet werden, ist sicherlich auch nicht mit großen Effekten zu rechnen. Dies gilt insbesondere dann, wenn die Urteile zusätzlich Interpretationen im Kontext involvieren, die eine prinzi\-pielle Unsicherheit beisteuern. Nichtsdestotrotz weisen die statistischen Auswertungen darauf hin, dass man nicht annehmen kann, dass \glq Abfolge\grq {}  der einzige relevante Faktor ist. \glq Abfolge abhängig von Kontext\grq {} ist ebenfalls relevant für den Unterschied zwischen den Mittelwerten. Die Ergebnisse sind folglich konform mit meiner Hypothese: Es macht einen Unterschied, ob/wie gut sich die Abfolge umkehren lässt, abhängig vom Kontext. Der hier betrachtete Kontext sind Sprechaktbegründungen. Dies ist einer der Kontexte, in dem ich die Abfolge \textit{doch ja} (vornehmlich) in den Webdaten gefunden habe. Ich sehe das Ergebnis dieser Akzeptabilitätsstudie deshalb als Stützung meiner Annahme, dass es sich hierbei um einen Kontext handelt, in dem die Abfolge einfacher umgekehrt werden kann als in \is{prototypische Assertion} prototypischen Assertionen, die in den \textit{doch ja}-Treffern aus den Korpusdaten nicht belegt sind. Beide Untersuchungen weisen folglich in die gleiche Richtung. Die Frage nach der Qualität der Webdaten möchte ich in Bezug auf dieses konkrete Phänomen deshalb so beantworten, dass die Webdaten eine genauso gute oder schlechte Quelle für linguistische Analysen sind wie die Akzeptabilitätsbewertungen. 

Als Grund für die prinzipiell recht niedrigen Bewertungen kommen m.E. verschiedene Aspekte in Frage, die nicht alle mit dem Phänomen an sich zusammenhängen. Sprecher nutzen z.B. selten die gesamte Spanne der Skala aus. Wenige Sprecher verwenden 1 und wenige 5. Die Skala wird in der Realität folglich bei vielen Sprechern zu einer 3er Skala. Ein weiterer Aspekt ist, dass unter den Filler-Sätzen auch absolut wohlgeformte deutsche Sätze waren. Wenn Sprecher diesen lediglich eine Bewertung von 2 oder 3 zuordnen, erhalten MP-Äußerungen natürlich niedrigere Urteile. Was man auch nicht unterschätzen sollte, ist, dass MP-Kombinationen gar nicht so häufig auftreten, wie man denken mag. Es würde aus dieser Sicht nicht verwundern, wenn derartige Äußerungen für Sprecher \\ deshalb schon generell einen gewissen Markiertheitsgrad \is{Markiertheit} aufwiesen. Und selbst wenn die Bewertungen insgesamt etwas höher ausfielen, würde die Subtilität der Unterschiede dennoch bestehen bleiben. An dem grundsätzlich sicherlich kleinen Unterschied würde sich auch nichts ändern, wenn die Ergebnisse im 3er- oder 4er-Bereich der Skala lägen. Wie ich schon angeführt habe, ist mit größeren Unterschieden nicht zu rechnen. Vielleicht ist es deshalb auch nur die impressionistische Sicht, die sich über die recht niedrigen Bewertungen wundert. Statistisch ergeben sich schließlich nur an manchen (und zwar an durch meine Analyse vorhergesehenen) Stellen Effekte.\\

\noindent
Mit diesen Ausführungen zum Status der Abfolge \textit{doch ja} schließe ich die Betrachtung von \textit{ja}, \textit{doch} sowie ihren Kombinationen. Der nächs\-te Teil der Arbeit beschäftigt sich mit zwei anderen MPn: \textit{halt} und \textit{eben}. Zum einen stellen sich für ihr kombiniertes Auftreten gleiche Fragen, wie sie in Bezug auf \textit{ja doch} und \textit{doch ja} aufgeworfen wurden, zum anderen gibt es aber auch spezielle Themen, die der Betrachtung dieser beiden Partikeln eigen sind.


\chapter{Kombinationen aus \textit{halt} und \textit{eben}}
\label{chapter:hue}
\section{Die Einzelpartikeln \textit{halt} und \textit{eben} in der Literatur}
\label{sec:hueinliteratur}
M.E. lassen sich in der Literatur zu den Einzelpartikeln \textit{halt} und \textit{eben} zwei generel\-le Ansichten unterscheiden: Die eine vertritt die Idee, dass die Bedeutung der beiden MPn identisch ist und der Unterschied zwischen ihnen in ihrer regional verschiedenen Verwendung liegt. Die andere geht davon aus, dass die Bedeutung zwar ähnlich ist, es aber auch Unterschiede gibt. Ich schließe mich der letzteren Ansicht an, wenngleich man anerkennen muss, dass es früher regionale Unterschiede gegeben hat. Da der Aspekt der regionalen Unterschiede dem Thema anhaftet und es (wie wir in Abschnitt~\ref{sec:anlit} sehen werden) auch Aussagen über die Kombination von \textit{halt} und \textit{eben} vor diesem Hintergrund gibt, möchte ich im folgenden Abschnitt einige Annahmen zu dem Thema regionaler Varianten skizzieren.

\subsection{Regionale Varianten}
\label{sec:regio}
Es gibt ältere Untersuchungen, die zu zeigen behaupten, dass es regionale Unterschiede zwischen dem \textit{halt}- und \textit{eben}-Gebrauch im deutschsprachigen Raum gegeben hat. Die meines Wissens erste Beobachtung dieser Art geht zurück auf \citet{Eichhoff1978} im \textit{Wortatlas der deutschen Umgangssprache}. Es handelt sich hierbei um eine Erhebung (vor allem im Lexembereich), die die Absicht verfolgte, regionale Verwendungsunterschiede in Deutschland, Österreich und der Schweiz zu erfassen. In Form der Frage in (\ref{538}) waren auch \textit{halt} und \textit{eben} Teil dieser Untersuchung.

\begin{exe}
	\ex\label{538} Frage 113\\
    Setzt man \textit{halt} oder \textit{eben} oder einen anderen Ausdruck in einem Satz wie: \glqq Der Zug fährt erst in einer Stunde, da muß ich .......... so lange warten?\grqq{} Wenn ja, bitte setzen Sie ein.
\hfill\hbox{\citet[31]{Eichhoff1978}}    
\end{exe}
Die Ergebnisse werden in \citet{Eichhoff1978} jeweils durch Karten der Art in Abbildung \ref{Abbildung 1} präsentiert.

\begin{figure}[h]
\includegraphics[width=0.7\textwidth]{he1.png}
\caption{Abbildung 1}
\label{Abbildung 1}
\hbox{}\hfill\hbox{\citet[103]{Eichhoff1978}}
\end{figure}	
\noindent
Man sieht hier eine relativ klare Trennung mit \textit{eben} in der oberen Landeshälfte und \textit{halt} in der unteren. Die Grenze verläuft nördlich der Main-Linie. Wenn andere Autoren diese Ergebnisse zusammenfassen, heißt es für die Verteilung in Deutschland, dass \textit{eben} im Norden und \textit{halt} im Süden des Landes auftritt (vgl. z.B. \citealt[212]{Dittmar2000}). Allerdings sieht man an \ref{Abbildung 1} auch, dass \textit{eben} im Süden durchaus vertreten ist, während \textit{halt} nicht im Norden auftritt.

Im Hinblick auf meine eigenen Datenerhebungen, die ich in Abschnitt~\ref{sec:spu} vorstellen werde, ist zu bedenken, dass Eichhoffs Erhebung in den 70er Jahren durchgeführt wurde (1971–1978). D.h. die damaligen Testanten sind heutzutage mindestens 60 Jahre alt (vgl. auch \citealt[2]{Elspass2005}, der hier 2005 von der Eltern- und Großelterngeneration seiner Studenten ausgeht). 

\citet{Elspass2005} berichtet von einer Nacherhebung im Stile von \citet{Eichhoff1978} in Form einer Onlinebefragung, die 2002 stattgefunden hat. Diese Untersuchung ergab die Karte in \ref{Abbildung 2}.

\begin{figure}[h]
\includegraphics[width=0.7\textwidth]{he2.png}
\caption{Abbildung 2}
\label{Abbildung 2}
\hbox{}\hfill\hbox{\citet[51]{Elspass2005}}
\end{figure}
\noindent
Sie zeigt, dass sich die \textit{halt}-Verwendung in den Norden ausgeweitet hat, sowie dass \textit{eben} (wenn auch weniger) auch in anderen Gebieten auftritt.

Aus diesen Untersuchungen hat man abgeleitet, dass man es bei der Verwendung von \textit{halt} und \textit{eben} mit einem Nord-Süd-Gefälle zu tun hatte, das sich aber abgebaut zu haben scheint. Im Süden Deutschlands (bzw. in der Schweiz und Österreich) existieren \textit{halt} und \textit{eben} bereits zur Zeit der ersten Untersuchungen nebeneinander (wenngleich \textit{halt} nach \ref{Abbildung 1} zu überwiegen scheint $[$s.u. weiteres dazu$]$). Im Norden gab es Zeiten, in denen nur \textit{eben} gebräuchlich war; \textit{halt} hat aber Eingang in den Sprachgebrauch dort gefunden. Für das Gegenwartsdeutsche ist somit davon auszugehen, dass die beiden MPn im gesamten Sprachgebiet nebeneinander existieren. Dies schreiben sogar auch schon Autoren, die in den 80er Jahren auf die erste Untersuchung von Eichhoff blicken (z.B. \citealt[78]{Hartog1982}, \citealt[97]{Dahl1988}, \citealt[124]{Thurmair1989}; vgl. auch \citealt[93]{Autenrieth2002}). Die Etablierung des \textit{halt}-Gebrauchs in der oberen Landeshälfte scheint also in den 80er Jahren eingetreten zu sein.

Ähnliche Verteilungsunterschiede wurden auch für die West-Ost-Achse ange\-nommen. \citet{Protze1997} berichtet von einer Untersuchung aus der zweiten Hälfte der 70er Jahre bis 1980. Es handelt sich um Umfragen wie bei \citet{Eichhoff1977}; (\citeyear{Eichhoff1978}) die in 296 Städten der ehemaligen DDR durchgeführt wurden. Die \textit{eben}/\textit{halt}-Verteilungen, die sich auf der Basis des Testsatzes in (\ref{539}) ergaben, sind in der Karte in \ref{Abbildung 3} dargestellt.

\begin{figure}[h]
\includegraphics[width=0.5\textwidth]{he3.png}
\caption{Abbildung 3}
\label{Abbildung 3}
\hbox{}\hfill\hbox{\citet[266]{Protze1997}}
\end{figure}

\begin{exe}
	\ex\label{539} 
    Der Bus fährt erst in einer Stunde; da muß ich ... so lange warten. 
    \newline		
	\hbox{}\hfill\hbox{\citet[167]{Protze1997}}    
\end{exe}
Man sieht, dass in den neuen Bundesländern fast flächendeckend \textit{eben} verwendet wurde. Ausnahmen sind Südthüringen, das Vogtland (Südsachsen) und die Oberlausitz (Nord-Ost-Sachsen), wo \textit{halt} als gebräuchlichste Form genannt wurde.

Eine spätere Untersuchung zur unterschiedlichen Verteilung von \textit{halt} und \textit{eben} in Ost- und Westdeutschland findet sich in \citet{Dittmar2000}. Er beschäftigt sich mit den Auswirkungen von Mauerfall und Wiedervereinigung auf den Sprachgebrauch in den neuen Bundesländern. Die Datenbasis sind Interviews in Ost- und Westberlin zwischen 1993 und 1996, in denen die Befragten von ihren persönlichen Erfahrungen mit dem 09.11.1989, der Zeit danach sowie den Auswirkungen der Ereignisse auf ihre aktuelle Situation berichten. Der zentraler Punkt in \citet[213]{Dittmar2000} ist, dass die Verwendung von \textit{halt} vs. \textit{eben} einen soziolinguistischen Grund hat, in dem Sinne, dass \textit{halt} als sozialprestigeträchtig angesehen worden sei. Prinzipiell hätten die Westberliner in den Interviews häufiger \textit{halt} verwendet als die Ostberliner (s.u.). Dazu habe die \textit{halt}-Verwendung nach der \glq Wende \grq {} in Ost-Berlin und den neuen Bundesländern zugenommen. Evidenz für den hohen sozialen Status des \textit{halt}-Gebrauchs und die (daraus resultierende) Anpassung der östlichen an die westliche Sprachgemeinschaft geben metalinguistische Äußerungen wie die folgenden aus \citet[230]{Dittmar2000}.
\begin{exe}
	\ex\label{540} 
		\begin{tabular}[t]{ll} 
 		B & jo also mir' ich zucke auch zusammen eh wenn mit das \textbf{halt} raus-   \tabularnewline
 		 & rutscht;   \tabularnewline
 		I & warum? \tabularnewline
 		B & eh weil ich natürlich ganz genau weiß dieses \textbf{halt} is son marker, den' \tabularnewline
 		& den man eh bewusst setzen kann dass man \textbf{eben} sich integriert oder \tabularnewline
 		& oder dass es \textbf{eben} einfach dass es übernommen wird. \tabularnewline
 		I & wohin? \tabularnewline
 		B &	na in ein' in eine sprachgemeinschaft west;	
  		\end{tabular}				
\end{exe}																	
Dittmar gibt Häufigkeiten für jeden Sprecher zum Gebrauch von \textit{halt} und \textit{eben} in diesen Interviews an. In der Tabelle in (\ref{541}) lässt sich ablesen, dass sowohl die West- als auch die Ostberliner insgesamt häufiger \textit{eben} als \textit{halt} verwenden und dass dieses Gefälle bei den Ostberlinern außerdem stärker ausgeprägt war als bei den Westberlinern.

\begin{exe}
	\ex\label{541} 
		\begin{tabular}[t]{|l|l|l:cx{1pt}l|} 
		\hline
 		& \textit{\textbf{eben}} & \textit{\textbf{ebent}} & \textit{\textbf{eben}} & \textit{\textbf{halt}}   \tabularnewline
 		\hline
 		& & & (\textbf{gesamt}) & \tabularnewline
 		\hline
 		\textbf{31 Ostberliner} & 245 & 372 & 617 & 204 \tabularnewline
 		\hline
 		\textbf{25 Westberliner} & 213 & 96 & 309 & 161 \tabularnewline
 		\hline
  		\end{tabular}			
\end{exe}	
 		\hfill\hbox {\citet[221]{Dittmar2000}}\\
Dieser Eindruck bestätigt sich auch, wenn man die Verwendung bei den einzelnen Sprechern betrachtet. Von den 31 Ostberlinern verwenden 17, also mehr als die Hälfte, \textit{halt} gar nicht. Es gibt nur fünf Sprecher, die häufiger \textit{halt} als \textit{eben} gebrauchen, und lediglich drei, die sowohl \textit{eben} als auch \textit{halt} etwa gleich häufig verwenden. Unter den 25 Westberlinern befinden sich acht Sprecher (ca. ein Drittel), die \textit{halt} gar nicht äußern, fünf verwenden \textit{halt} häufiger als \textit{eben} und bei sechs weiteren Sprechern finden \textit{halt} und \textit{eben} etwa gleich häufig Verwendung (vgl. die Daten in \citealt[121-122]{Dittmar2000}).

Fasst man die Ergebnisse der hier skizzierten Untersuchungen zusammen, lässt sich – bis in die 70er Jahre rückblickend – sagen, dass \textit{halt} und \textit{eben} einmal regional verteilt waren: Das Süddeutsche kennt seit dieser Zeit sowohl \textit{halt} als auch \textit{eben}, das Nord- und Ostdeutsche verwendete \textit{eben} (mit einigen \textit{halt}-Flecken in Ostdeutschland). Dazu ist eine Bewegung von \textit{halt} von Süden nach Norden und von Westen nach Osten zu verzeichnen. Für das heutige Deutsch ist davon auszugehen, dass im gesamten deutschen Sprachgebiet \textit{halt} und \textit{eben} nebeneinander existieren.

\subsection{Zur Validität der dialektalen Erhebungen}
\label{sec:val}
Bei den hier skizzierten dialektalen Annahmen zu \textit{halt} und \textit{eben} handelt es sich um Aspekte, die sich bei dieser Thematik etabliert haben und die deshalb immer wieder in Arbeiten angeführt werden. Gerade deshalb halte ich es für wichtig, auch auf einige Kritikpunkte und fragliche Basen von Aussagen hinzuweisen, die z.T. auch bereits in anderen Arbeiten angeführt worden sind. Sie betreffen die Art der Fragestellung, die Auswertung und Darstellung der Ergebnisse, die Auswahl der Testsätze sowie die Repräsentativität der Ergebnisse.

Beispielsweise kritisiert \citet[16-17]{Elspass2005} mit \citet[174-178]{Hentschel1986} die Fragestellung von \citet{Eichhoff1978}, der danach fragte, ob \textit{halt}, \textit{eben} oder ein anderer Ausdruck in (\ref{542a}) verwendet werde.

\begin{exe}
	\ex\label{542a} 
 Der Zug fährt erst in einer Stunde, da muß ich .......... so lange warten.
\end{exe}																
Da es im Süddeutschen auch dialektale Formen gibt (wie \textit{ebe}, \textit{eaba}), ist es \citet[174]{Hentschel1986} zufolge unplausibel, dass der Gebrauch von \textit{halt} derart überwiegen soll. Sie nimmt an, dass \textit{halt} im Süden als mundartlicher gilt. Da die Testanten wussten, dass es sich um eine dialektale Erhebung handelt und dass \textit{halt} im Norddeutschen bzw. generell im Standarddeutschen unüblich ist, hätten sie sich (da sie sich für eine Form entscheiden mussten) für \textit{halt} entschieden, obwohl sie \textit{eben} gleichermaßen kannten. Unter dieser Sicht wären also die Art der Abfrage und die Umstände, unter denen sie stattfand, mit dafür verantwortlich, dass \textit{eben} auf der Karte in \ref{Abbildung 2} im süddeutschen Raum unterrepräsentiert zu sein scheint. Neben dem Bewusstsein um den dialektalen Status von \textit{halt}, führt \citet[176]{Hentschel1986} ebenfalls die Möglichkeit an, dass die Befragten sich für \textit{halt} entschieden, weil es positivere Konnotationen (wie \textit{weich}, \textit{warm}) aufweise. Um derartige mögliche Störfaktoren zu umgehen, beabsichtigte \citet{Elspass2005} eine Fragestellung, die die Testanten nicht zur Auswahl einer der beiden Formen zwang. Seine Formulierung lautete deshalb: \glqq Setzt man \textit{halt} oder \textit{eben}, beides oder einen anderen Ausdruck ...?\grqq{}  (\citealt[17, Fn 41]{Elspass2005} 17). Doch wie er selbst bemerkt, ist diese Version nicht eindeutig, da er mit \textit{beides} nicht auf die Nennung der Kombination der beiden MPn abzielen wollte (d.h. \textit{halt eben} oder \textit{eben halt}), sondern auf die Gleichwertigkeit der beiden Formen. Auch diese Formulierung ist nicht vollständig geglückt.

Wie die Art der Befragung bzw. die anschließende Aufarbeitung in Form der Karten Ergebnisse beeinflussen kann, zeigt auch die neueste Untersuchung zur Thematik von Elspaß \& Möller (2012) (unter http://www.atlas-alltagssprache.de/\\runde-9/). Betrachtet man die Karte in Abbildung \ref{Abbildung 4}, gewinnt man den Eindruck, \textit{eben} werde in Süddeutschland, Österreich und der Schweiz überhaupt nicht verwendet.
\begin{figure}[h]
\includegraphics[width=0.7\textwidth]{he4.png}
\caption{Abbildung 4}
\label{Abbildung 4}
\hbox{}\hfill\hbox{(http://www.atlas-alltagssprache.de/halt-eben/)}
\hbox{}\hfill\hbox{(eingesehen am 05.05.2014)}
\end{figure}
\noindent																                       										
Dies verwundert, zumal dies in keiner Weise den Ergebnissen/Darstellungen der älteren Untersuchungen (weder \citealt{Eichhoff1978} noch \citealt{Elspass2005}) entspricht. Die Lösung dieses überraschenden Ergebnisses präsentieren die Autoren selbst:			
\begin{quotation}														                       										Zur Kartierung: Bei verschiedenen Antworten aus ein und demselben Ort ist die häufiger genannte Variante kartiert, in vielen Karten daneben auch (kleiner) die seltener genannte Variante, sofern sie mehr als 35\% der Nennungen ausmacht. Bei gleich häufiger Nennung musste nach dem Zufalls\-prinzip entschieden werden, welche Variante als Erst- bzw. Zweitvariante kartiert wurde. In Karten, bei denen die Wiedergabe der Zweitvarianten vor allem zur Verunklarung der regionalen Unterschiede geführt hätte, wurde der Übersichtlichkeit zuliebe darauf verzichtet. Insofern erscheinen die Unterschiede hier – gegenüber dem tatsächlichen \glqq durchschnittlichen\grqq{} Gebrauch – etwas stärker betont; dies ergibt sich aber auch schon allein aus der Art der Befragung.
\hfill\hbox{(eingesehen am 05.05.2014)}
\newline
\hbox{}\hfill\hbox{(http://www.atlas-alltagssprache.de/halt-eben)}
\end{quotation}
Wieder ist also deutlich zu sehen, dass das Ergebnis (diesmal vornehmlich durch die Art der Auswertung) verzerrt, und zwar in diesem Fall überzeichnet, wird.

Einen weiteren Aspekt gilt es in der Darstellung aus \citet{Dittmar2000} zu berücksichtigen. Er weist selbst darauf hin, dass seine Ergebnisse nur bedingt aussagekräftig sind, da einerseits die Informantenzahl verschieden (25 West- vs. 31 Ostberliner) und andererseits die Interviewausschnitte unterschiedlich lang waren (vgl. \citealt[222]{Dittmar2000}). Vor diesem Kritikpunkt ist Dittmars Ergebnis, dass mehr Westberliner \textit{halt} überhaupt benutzen, allerdings auch umso bedeutender, da die Interviews mit den Ostberlinern sogar länger sind und sie somit mehr Möglichkeiten hatten, \textit{halt} zu verwenden (wenn sie es überhaupt verwenden) (vgl. \citealt[222]{Dittmar2000}).

Ein letzter Aspekt, der in den Untersuchungen von Eichhoff, Elspaß und Protze nicht genügend Beachtung findet, ist die Frage, ob in dem einen (!) vorgegebenen Testitem tatsächlich \textit{halt} und \textit{eben} gleichermaßen verwendet werden können. Ich möchte meine konkreten Bedenken auf Abschnitt~\ref{sec:spu} verschieben, um unter Bezug auf meine Bedeutungsmodellierung präziser argumentieren zu können. An dieser Stelle sei aber bereits angemerkt, dass der regionale Charakter (insbesondere bei prinzipieller Bekanntheit beider Formen) wohl nur nachweisbar ist, wenn die beiden MPn in der vorliegenden Äußerung tatsächlich beide gleichermaßen plausibel auftreten können. Im Rahmen ihrer Analyse der Interpretation von \textit{halt} und \textit{eben} bezweifelt auch schon \citet[174]{Hentschel1986} dies in Bezug auf Eichhoffs Testsatz. Nach der Autorin ist \textit{halt} emotionaler konnotiert als \textit{eben} und der Kontext in (\ref{542a}) lege diese Lesart nahe.

\subsection{Die Bedeutung von \textit{halt} und \textit{eben}}
\label{sec:bedhe}
Wenn Autoren annehmen, dass es die zwei Formen \textit{halt} und \textit{eben} deshalb gibt, weil sie regional verteilt sind, geht diese Sicht damit einher, dass sie ihre Bedeutung und Verwendung für identisch halten (vgl. z.B. \citealt[10]{Becker1978}, \citealt[358]{Luetten1977}, \citealt[81]{Bublitz1978}, \citealt[202]{Karagjosova2004}). Es wird dann eine Bedeutung für \textit{eben} angegeben, die auf \textit{halt} gleichermaßen zutreffen soll. Wenige Autoren beschäftigen sich hingegen mit der Bedeutung von \textit{halt} und betrachten es in Differenz zu \textit{eben}. Die Bedeutung, die von ersteren Autoren sowohl für \textit{halt} als auch für \textit{eben} angesetzt wird, ist dann in der Regel diejenige, die Autoren, die sich für einen Unterschied zwischen den beiden Partikeln aussprechen, für \textit{eben} formulieren.

\subsubsection{Gemeinsamkeiten}
Die Illokutionstypen, in denen \textit{halt} und \textit{eben} auftreten können, sind Assertionen \is{Assertion} und \is{Direktiv} Direktive.\footnote{Die beiden Partikeln können auch in verschiedenen Nebensatztypen auftreten, vor allem \textit{halt} scheint hier
sehr frei verwendet werden zu können (z.B. Kausalsätze, Konditionalsätze, Temporalsätze, Instrumentalsätze $[$zu einem Überblick vgl. \citealt[202-203]{Hentschel1986}$]$). Ich gehe auf die Nebensatzverwendungen an dieser Stelle nicht ein, betrachte in Abschnitt~\ref{sec:rs} allerdings ihr Auftreten in Relativsätzen sehr detailliert.}

\begin{exe}
	\ex\label{542} Assertion\\
	A: Peter sieht sehr schlecht aus. (= q)\\
	B: Er war \textbf{eben}/\textbf{halt} lange krank gewesen. (= p)		
\end{exe}	

\begin{exe}
	\ex\label{543} Direktiv\\
	A: Ich schaffe es nicht bis morgen! (= q)\\
	B: Arbeite eben/halt schneller! (= p)	
	\hfill\hbox{\citet[340/215]{Karagjosova2004}} 	
\end{exe}				    
Die klassischen Merkmale, die in Beschreibungen dieser zwei Partikeln genannt werden (vgl. auch \citealt[150-152]{Mueller2016a}; \citeyear[165-169]{Mueller2016b}, \citeyear[239-243]{Mueller2017a}), sind zum einen die \textit{Rückorientierung} von \textit{eben}/\textit{halt}-Äußerungen und zum anderen der Aspekt der \textit{Kategorizität}. Die Eigenschaft der Rückorientierung (vgl. z.B. \citealt[98, 224]{Dahl1988}, \citealt[120, 125-126]{Thurmair1989}, \citealt[340]{Karagjosova2003}; \citeyear[208]{Karagjosova2004}) zeigt sich generell darin, dass diese MP-Äußerungen immer Reaktionen auf eine vorausgehende Äußerung oder Situation sind. Wie in Abschnitt~\ref{sec:zugang} in Kapitel~\ref{chapter:hintergrund} erläutert, sind MPn prinzipiell reaktiver Natur. Diese Eigenschaft trifft auf \textit{halt} und \textit{eben} aber umso mehr zu, da die Bedingung bzw. das Begründende vorerwähnt sein muss. Sie können nicht redeeinleitend verwendet werden und eignen sich nicht zum Themenwechsel. Konkreter wird für diese Relation angenommen, dass die MP-Äußerung und eine andere Äußerung in einem kausalen Verhältnis \is{kausale Relation} zueinander stehen oder eine Bedingungs-Folge-Relation \is{Bedingungs-Folge-Relation} vorliegt (vgl. z.B. \citealt[40]{Weydt1969}, \citealt[101, 288 Fn 60, 125]{Dahl1988}, \citealt[121]{Helbig1990}, \citealt[67]{Koenig1997}). 

In (\ref{542}) beispielsweise ist das Kranksein die Begründung für das schlechte Aussehen. Die MP-Äußerung gibt hier den Grund für den Inhalt der vorausgehenden Äußerung an (vgl. (\ref{544})).

\begin{exe}
	\ex\label{544} Weil er lange krank gewesen ist (= p), sieht Peter schlecht aus (= q).		
\end{exe}
Mit dieser Funktion der \textit{eben}/\textit{halt}-Äußerung ist gut verträglich, dass einer solchen Äußerung gerne eine Frage nach dem Grund vorweggeht (vgl. (\ref{545}), (\ref{546}) (vgl. \citealt[121]{Thurmair1989})).
\begin{exe}
	\ex\label{545} 
	Hans: Warum sind nur die Frauen so hinter dir her?\\
	Peter: Ich bin \textbf{eben} unwiderstehlich.
	\hfill\hbox{\citet[121]{Thurmair1989}} 	
\end{exe}	

\begin{exe}
	\ex\label{546} 
	Warum bist du denn so fad?\\
	Ich bin \textbf{halt} krank.	
	\hfill\hbox{\citet[312]{Schlieben-Lange1979}} 	
\end{exe}	
Eine Äußerung ohne Partikel ist hier situativ unangemessen und das Auftreten einer kausalen Konjunktion ist in diesem Fall notwendig (vgl. (\ref{547})).

\begin{exe}
	\ex\label{547} 
	Hans: Warum sind nur die Frauen so hinter dir her?\\
	Peter: ?Ich bin unwiderstehlich./Weil ich unwiderstehlich bin.	
	\newline
	\hbox{}\hfill\hbox{\citet[121]{Thurmair1989}} 	
\end{exe}
Bei den \glq Begründungen\grq {} muss es sich dabei nicht um tatsächliche Begründungen handeln. In (\ref{548}) und (\ref{549}) liegen keine oder nur sehr oberflächliche Begründungen vor (vgl. \citealt[322]{Troemel-Ploetz1979}).

\begin{exe}
	\ex\label{548} 
	Wieso muss man denn hier fünf Fragebögen ausfüllen? – Das ist \textbf{eben} so.	
	\newline
	\hbox{}\hfill\hbox{\citet[312]{Schlieben-Lange1979}} 	
\end{exe}	
\begin{exe}
	\ex\label{549} 
	Warum esst ihr denn mit Stäbchen? – Das gefällt uns \textbf{eben}.
	\newline
	\hbox{}\hfill\hbox{\citet[322]{Troemel-Ploetz1979}} 	
\end{exe}																	              
Ein Beispiel für eine Bedingungs-Folge-Relation findet sich in (\ref{550}).

\begin{exe}
	\ex\label{550} 
	Evi: Du, das ist ganz blöd heute, ich hab noch so wahnsinnig viel zu tun. 
	\newline
	\hbox{}\hfill\hbox{(= q)}\\
	Pit: Gut, komm ich \textbf{halt}/\textbf{eben} morgen. (= p) So dringend ist es ja nicht.
	\newline
	\hbox{}\hfill\hbox{nach \citet[122]{Thurmair1989}} 	
\end{exe}

\begin{exe}
	\ex\label{551} 
	\glq Wenn du so viel zu tun hast (= q), dann komme ich morgen (= p).\grq {} 	
\end{exe}
Assertive Fälle können sowohl den Grund (vgl. (\ref{542})) als auch (wie in (\ref{550})) die Folge angeben. In Direktiven scheint mir der MP-Teilsatz immer die Folge auszumachen (vgl. (\ref{543}) und (\ref{552})).

\begin{exe}
	\ex\label{552} 
	Wenn du es nicht bis morgen schaffst (= q), musst du schneller arbeiten. 
	\newline
	\hbox{}\hfill\hbox{(= p)} 	
\end{exe}
Ein anderes Merkmal, das bei der Charakterisierung von \textit{eben} (und ggf. auch \textit{halt}) angeführt wird, ist das der \textit{Kategorizität}. 

Bei \citet[120-121]{Helbig1990} und \citet[340]{Karagjosova2003} heißt es über \textit{halt} und \textit{eben}, der Sachverhalt werde als \textit{unabänderlich}, \textit{kategorisch} und \textit{Thema beendend} ausgegeben. \citet[80/83]{Autenrieth2002} spricht von \textit{Absolutheit} und \textit{Kategorizität}. \citet[120]{Thurmair1989} schreibt, der Sachverhalt sei \textit{evident} (zu \textit{eben}). In Direktiven entspricht dies dem Verhältnis, dass die Handlung, zu der aufgefordert wird, \textit{offensichtlich} ist und die \textit{einzig mögliche Lösung des Problems} darstellt (vgl. \citealt[122]{Thurmair1989}, vgl. auch \citealt[169]{Hentschel1986}). Wie für alle mit MPn erreichten Effekte gilt auch hier, dass die Bedeutungsmomente \textit{Evidenz} und \textit{Faktum} auch nur vorgegeben sein können. \citet[320-323]{Troemel-Ploetz1979} nimmt für \textit{eben} an, die MP gebe die Proposition als \textit{generell gültig} und als \textit{Faktum} aus. Der Hörer könne die Annahme nicht in Frage stellen oder ihr widersprechen. Weitere Diskussionen würden sich erübrigen. Die Aussage werde als \textit{axiomatisch} dargestellt und die Behauptung somit \textit{immunisiert}. Auch die Auffassung von \citet[130]{Diewald1997} zu \textit{eben}, die Proposition sei \glqq bereits gegeben und aktuell wiederholt\grqq{} fügt sich in dieses Bild ein. \citet{Autenrieth2002} weist anhand von Beispielen der Art in (\ref{553}) darauf hin, dass die Charakterisierung \textit{unabänderlich} ggf. auch zu stark ausfalle, da auf solche Äußerungen eher verwandte, aber schwächere Bedeutungs\-zuschreibungen wie \textit{unspektakulär}, \textit{unproblematisch} und \textit{gewöhnlich} zutreffen würden.

\begin{exe}
	\ex\label{553} 
	(S zeigt Foto)\\
	Und auf DIEsem Foto ist Beates Kuchen dann \textbf{halt}/\textbf{eben} fertig und steht auf dem Tisch.		
	\hfill\hbox{nach \citet[99]{Autenrieth2002}} 
\end{exe}
Es ist auch darauf hingewiesen worden, dass sich die Bedeutungseffekte \textit{Evidenz} und \textit{Kategorizität} ebenfalls auf die Relation und nicht (nur) auf die Proposition der MP-Äußerung beziehen können (vgl. \citealt[99]{Dahl1988}). In (\ref{554}) wird nach Dahls Argumentation (zu \textit{eben} und \textit{halt}) nicht der Sachverhalt, dass der Nachbar Choleriker ist, als kategorisch/axiomatisch ausgegeben, sondern die Relation zwi\-schen den Propositionen (der Nachbar macht Krach und der Nachbar ist Choleriker).

\begin{exe}
	\ex\label{554} 
	A: Unser Nachbar hat heute wieder Krach gemacht.\\
	B: Er ist \textbf{eben} ein Choleriker.		
	\hfill\hbox{nach \citet[98]{Dahl1988}} 
\end{exe}
In (\ref{555}) ist der Zusammenhang bekannt:
\begin{exe}
	\ex\label{555} 
	Wenn jemand Choleriker ist, wird es bei ihm auch manchmal laut.
\end{exe}
Dahl nimmt dazu an, die \textit{eben}-Äußerung wirke \textit{abqualifizierend} (\citeyear[100]{Dahl1988}), in dem Sinne, dass B die Berechtigung der vorausgehenden Äußerung bestreitet. B widerspricht den Erwartungen, die A mit seiner Äußerung ausdrückt. Im Falle einer Assertion wie in (\ref{554}) ist dies die Annahme, dass er B etwas mitteilt, das wissenswert ist. Diese Erwartung wird in (\ref{554}) gestört, weil die Proposition aus As Äußerung ableitbar ist. Parallele Verhältnisse liegen auch in Direktiven wie in (\ref{556}) vor.
\begin{exe}
	\ex\label{556} 
	A: Mein Zahn tut mir weh.\\
	B: Dann geh \textbf{eben} zum Zahnarzt!
	\hfill\hbox{nach \citet[101]{Dahl1988}} 
\end{exe}
A müsste hier selbst zum Schluss der \textit{eben}-Äußerung kommen, da der Zusammenhang in (\ref{557}) bekannt ist.
\begin{exe}
	\ex\label{557} 
	Wenn dein Zahn weh tut, musst du zum Zahnarzt gehen.
\end{exe}
Es besteht kein Grund zum Klagen für A und die Mitteilung von A ist für die Eröffnung eines Problemlösungsszenarios irrelevant. Auch hier bestreitet B die Berechtigung des vorweggehenden Sprechaktes.

Diese Überlegungen stammen aus \citet[98-101]{Dahl1988} und werden in \citet[340]{Karagjosova2003}; (\citeyear[202-220]{Karagjosova2004}) übernommen, um die Bedeutung von \textit{eben} und \textit{halt} in einem formalen Diskursmodell zu erfassen. Da ich auf ihre Bedeutungszuschreibung in Abschnitt~\ref{sec:modellierung} zurückgreifen werde, seien ihre Annahmen an dieser Stelle bereits angeführt.

Auf der Basis der beiden Kriterien a) Rückorientierung und b) Kategorizität fasst die Autorin den Beitrag von \textit{eben} und \textit{halt} folgendermaßen auf: Zum einen ist die Proposition der MP-Äußerung bekannt. Dies kann p oder q sein (s.u.). Zum anderen ist unter den Diskurspartnern die Inferenzrelation p $>$ q (\glq Normalerweise, wenn p, dann q.\grq {}) bekannt. p $>$ q hat in \citet{Asher1991}, auf die sich Karagjosova hier bezieht, den Status \is{defeasible rule} einer \textit{defeasible rule}. Fungiert die \textit{eben}/\textit{halt}-Äußerung als Begründung der vorweggehenden Äußerung wie in (\ref{558}), ist p bekannt sowie die Inferenzrelation p $>$ q.
\begin{exe}
	\ex\label{558} 
	A: Unser Nachbar hat heute wieder Krach gemacht. (= q)\\
	B: Er ist \textbf{eben} ein Choleriker. (= p)
\end{exe}
Über \textit{modus ponens} \is{modus ponens} unter Beteiligung dieser (pragmatischen) Inferenzrelation, an dem nor\-malerweise ein \underline{logischer} Zusammenhang zwischen p und q beteiligt ist, ist abzuleiten, dass auch q bekannt ist (\textit{deafisible modus ponens} \is{deafisible modus ponens} in \citealt[387]{Asher1991}). 

Stellt die MP-Äußerung die Folge in einem Bedingungs-Folge-Gefüge dar wie in (\ref{559}), dreht sich die Inferenzrelation um. D.h. die Relation q $>$ p ist bekannt, ebenso die Proposition p.\footnote{Alternativ bleibt die Inferenzrelation p $>$ q und die MP bezieht sich auf q (und nicht p).}

\begin{exe}
	\ex\label{559} 
	A: Ich schaffe es nicht bis morgen. (= q)\\
	B: Arbeite \textbf{eben}/\textbf{halt} schneller! (= p)
\end{exe}	
Nach As Äußerung sind A und B sich hinsichtlich q einig. (A ist von q aufgrund seiner Äußerung überzeugt, B gibt einen Rat unter den Umständen von q, was auf seine Akzeptanz von q schließen lässt.) Ebenfalls bekannt ist der Zusammenhang zwischen q und p (\glq Wenn du etwas nicht bis zum nächsten Tag schaffst, musst du schneller arbeiten.\grq {}). Wenn q und q $>$ p bekannt ist, ist auch klar, dass zu p geraten wird.

Karagjosova setzt den hier beschriebenen Effekt sowohl für \textit{halt} als auch für \textit{eben} an. Wie eingangs angeführt, gibt es allerdings auch Autoren, die in ihrer Charakterisierung einen Unterschied zwischen den beiden Partikeln machen.

\subsubsection{Unterschiede}
\label{sec:untersch}
\citet[124]{Thurmair1989} geht davon aus, dass die beiden MPn zwar eine ähnliche Bedeutung haben, es sich aber nicht um Synonyme handelt. Auch \citet[254]{Dahl1988} schreibt schon, \glqq daß es sich eher um funktionale Ähnlichkeit als Äquivalenz\grqq{}  handelt. Eines der Argumente für diese Annahme ist bei \citet{Thurmair1989}, dass \textit{eben} und \textit{halt} nicht beliebig austauschbar sind. So gibt es Kontexte, in denen \textit{eben} im Gegensatz zu \textit{halt} nicht gut stehen kann (vgl. (\ref{5600}) und (\ref{561})).\footnote{Ein Gutachter weist korrekterweise darauf hin, dass ein Beispiel wie (\ref{5600}) zeigt, dass die beteiligte Begründung nicht einer vorher explizit genannten Proposition gelten muss, sondern eine \textit{halt}-/\textit{eben}-Äußerung sich auch auf eine Implikatur der Vorgängeräußerung beziehen kann. In (\ref{5600}) werde nicht die Proposition dass der Adressat seine Freunde mitbringen kann begründet, sondern der durch \textit{schon} ausgedrückte Vorbehalt gegenüber weiteren Gästen.}

\begin{exe}
	\ex\label{5600} Du kannst deine Freunde schon mitbringen.
		\begin{xlist}	
			\ex\label{560a} Wir haben \textbf{halt} kein Bier mehr.
			\ex\label{560b} *Wir haben \textbf{eben} kein Bier mehr.
		\end{xlist}
\end{exe}

\begin{exe}
	\ex\label{560} Monika will Hans um einen Gefallen bitten, zögert aber, ihn anzurufen. Nach einiger Zeit sagt ihre Freundin.
		\begin{xlist}	
			\ex\label{560a} Jetzt ruf den Hans \textbf{halt} an!
			\ex\label{560b} *Jetzt ruf den Hans \textbf{eben} an!
			\hfill\hbox {\citet[124]{Thurmair1989}}
		\end{xlist}
\end{exe}
Umgekehrt sind ihrer Ansicht nach schwieriger Fälle zu finden, in denen \textit{eben} angemessen ist und \textit{halt} unpassend erscheint. Thurmair gibt die Beispiele in (\ref{561}) und (\ref{562}).

\begin{exe}
	\ex\label{561} 
		\begin{xlist}	
			\ex\label{561a} Der Wal ist \textbf{eben} ein Säugetier.
			\ex\label{561b} ?Der Wal ist \textbf{halt} ein Säugetier.
		\end{xlist}
\end{exe}
\begin{exe}
	\ex\label{562} 
		\begin{xlist}	
			\ex\label{562a} Der Krieg ist \textbf{eben} unmoralisch.
			\ex\label{562b} ?Der Krieg ist \textbf{halt} unmoralisch.
			\hfill\hbox {\citet[124]{Thurmair1989}}
		\end{xlist}
\end{exe}
Und auch wenn beide Partikeln in einem Kontext prinzipiell stehen können (vgl. (\ref{563}), (\ref{564})), geht das Auftreten der ein oder anderen Partikel der Autorin zufolge mit einem Interpretationsunterschied einher.

\begin{exe}
	\ex\label{563} 
		\begin{xlist}	
			\ex\label{563a} Männer sind \textbf{eben} so.
			\ex\label{563b} Männer sind \textbf{halt} so.
		\end{xlist}
\end{exe}
\begin{exe}
	\ex\label{564} Ich hab dich doch gewarnt. 
		\begin{xlist}	
			\ex\label{564a} Horrorvideos sind \textbf{eben} grausam.	
			\ex\label{564b} Horrorvideos sind \textbf{halt} grausam.
			\hfill\hbox {\citet[124]{Thurmair1989}}
		\end{xlist}
\end{exe}
Eine \textit{halt}-Äußerung wirke im Vergleich zu einer \textit{eben}-Äußerung abgeschwächter und weniger apodiktisch (vgl. \citealt[125]{Thurmair1989}, vgl. auch schon \citealt[309]{Schlieben-Lange1979}). Wie oben erläutert, zeigen \textit{halt} und \textit{eben} hinsichtlich des Kriteriums des Rückbezugs dasselbe Verhalten. Im Falle von \textit{halt} ist der Sachverhalt (bzw. die Relation) nicht evident/kategorisch/die einzige Möglichkeit, sondern nur plausibel. Die \textit{halt}-Äußerung ist eine plausible Erklärung/Begründung für den Vorgängerbeitrag oder eine plausible Folge aus diesem. Im Falle einer \textit{halt}-Begründung/Erklärung sind alternative Begründungen zugelassen. Der Sprecher vertritt lediglich die von ihm präsentierte Erklärung. Insgesamt wirken \textit{halt}-Assertionen somit abgeschwächter im Vergleich zu \textit{eben}-Assertionen. \textit{Halt}-Auffor\-derungen, \is{Aufforderung} die die Folge aus dem zuvor gesagten darstellen, sind schwächer als \textit{eben}-Aufforderungen. Sie präsentieren nicht apodiktisch die einzig denkbare Lösung, sondern es handelt sich aus Sprechersicht um eine plausible Lösung für das Problem aus dem Vorgängerbeitrag (vgl. \citealt[125-126]{Thurmair1989}, vgl. auch \citealt[316]{Schlieben-Lange1979}, \citealt[74-75]{Hartog1982}, \citealt[235]{Meibauer1994}, \citealt[312]{Rost-Roth1998}, \citealt[98]{Autenrieth2002}).

Unter der Annahme dieses Unterschieds in der Interpretation von \textit{eben}- und \textit{halt}-Äußerungen leitet \citet{Thurmair1989} die Beispiele aus (\ref{560}) und (\ref{561}), in denen die beiden MPn situativ nicht gleichermaßen angemessen sind, auf die folgende Art ab: Sie argumentiert, dass \textit{eben} dann inakzeptabel ist, wenn Alternativen relevant sind und anerkannt werden. In Beispielen der Art in (\ref{560}) und (\ref{561}) stellt der Sachverhalt der MP-Äußerung eine plausible Begründung dar, der Sprecher bzw. der Adressat handelt aber nicht danach. In (\ref{560}) ist es beispielsweise plausibel für den Sprecher, dass, wenn man kein Bier mehr hat, man nicht weitere Gäste einlädt. Dies scheint aber nicht der einzige Zusammenhang zu sein: Wenn dies der einzige denkbare Zusammenhang wäre, könnte er die Freunde schließlich nicht zulassen. Ist die MP-Äußerung die Folge wie im Falle der Aufforderung in (\ref{561}), handelt es sich bei der Handlung, zu der geraten wird, um eine plausible, aber wiederum nicht einzig mögliche Lösung. Den Gesprächspartner anzurufen, ist \underline{eine} Möglichkeit, es ist aber nicht die einzige. Aus dem Kontext ist bekannt, dass der Vorschlag gerade nicht evident ist, da der Adressat genau diese Handlung nicht favorisiert (vgl. \citealt[127]{Thurmair1989}).

In Abschnitt~\ref{sec:modellierung} werde ich erläutern, wie sich diese Fälle in meine Bedeutung\-smodellierung fügen. Die entscheidende Erkenntnis von Thurmair an dieser Stelle, die es m.E. bei jeder Modellierung des Beitrags von \textit{halt}- und \textit{eben}-Äußerungen zu berücksichtigen gilt, ist, dass sich Beispiele finden lassen, in denen \textit{halt} stehen kann, in denen das Auftreten von \textit{eben} aber hingegen nicht angemessen ist. Es handelt sich um Situationen, in denen nicht von Evidenz/Kategorizität, sondern schwächer von Plausibilität auszugehen ist. Umgekehrt ist \textit{halt} dann nicht passend, wenn der ausgedrückte Sachverhalt evident ist. Es ist in diesem Fall nicht angemessen, Alternativen offen zu halten; \textit{halt} ist in diesem Sinne dann zu schwach. Es scheint äußerst unwahrscheinlich, dass nur der Sprecher davon ausgeht, dass der Wal ein Säugetier oder Krieg unmoralisch ist (vgl. (\ref{560}) und (\ref{561})). In Äußerungen dieser Art ist somit nur \textit{eben} zu verwenden. Die Fälle, in denen das Auftreten von \textit{halt} im Gegensatz zu \textit{eben} unangemessen ist, sind allerdings sehr selten (vgl. \citealt[128]{Thurmair1989}).

Die MP \textit{eben} lässt sich folglich in der Regel durch \textit{halt} ersetzen. Der umgekehrte Austausch ist hingegen nicht immer möglich (vgl. \citealt[128]{Thurmair1989}, \citealt[392]{Ickler1994}). Wenn \textit{eben} (mit Thurmair, s.o.) Evidenz anzeigt und \textit{halt} Plausibilität, sind die obigen Verhältnisse folgendermaßen zu erklären: Ein evidenter Sachverhalt ist auch plausibel, eine plausible Sachlage ist aber nicht notwendigerweise evident. In diesem Sinne schließt die Bedeutung von \textit{eben} die Bedeutung von \textit{halt} ein, der Beitrag von \textit{halt} schließt aber nicht den Beitrag von \textit{eben} ein (vgl. \citealt[128]{Thurmair1989}). Ich greife diese Zusammenhänge zu einem späteren Zeitpunkt der Betrachtung wieder auf. Sie sind zentral für meine Argumentation in Abschnitt~\ref{sec:impli}.

\section{Modellierung im Diskursmodell}
\label{sec:modellierung}
Ich bilde die Interpretation der MPn in dieser Arbeit ab, indem ich ihren Diskursbeitrag im Rahmen des formalen Diskursmodells aus \citet{Farkas2010} be\-schreibe. Im Falle von \textit{eben} und \textit{halt} baue ich dabei auf Annahmen und die Analyse von \citet{Karagjosova2003}; (\citeyear{Karagjosova2004}), die ebenfalls mit einem formalen Diskurs\-modell arbeitet. Im Gegensatz zu Karagjosova gehe ich allerdings von einem Unterschied zwischen \textit{eben} und \textit{halt} aus. 

\subsection{Direktive im Diskursmodell}
\label{sec:dirdm}
In der Form, wie von \citet{Farkas2010} entworfen, kann das Diskursmodell (vgl. Abschnitt~\ref{sec:diskursmodell}, Kapitel~\ref{chapter:hintergrund} für eine ausführliche Darstellung) nur Assertionen und E-Fragen erfassen. Da \textit{eben} und \textit{halt} auch in Direktiven \is{Direktiv} auftreten können, ist es nötig, das Modell zu erweitern, um auch diese Illokutionstypen im gleichen Rahmen behandeln zu können. Wenngleich Imperative noch nicht ausführlich aus der Perspektive ihres Diskurseffektes betrachtet worden sind, gibt es Autoren, die Diskursmodelle für diese Zwecke erweitert haben. Ich werde im Folgenden bei diesen Arbeiten Anleihen machen, ohne einem Ansatz konsequent zu folgen.\footnote{Dies gilt selbst für die Arbeit von \citet{Farkas2011}, die ihren eigenen Ansatz aus \citet{Farkas2010} ausbaut.} Der Hauptgrund für dieses Vorgehen ist, dass die Arbeiten – je nach behandeltem Phänomen – jeweils verschiedene Aspekte in ihre Modellierung aufnehmen und gleichzeitig andere ausklammern.

Bei \citet{Potts2003}, \citet{Portner2004}; (\citeyear{Portner2007}), \citet{Ninan2005}, \citet{Beyssade2006}, \citet{Farkas2011} (vgl. auch \citealt[211-215]{Roberts2004}) wird eine \textit{To-Do-Liste} \is{To-Do-Liste} eingeführt (vgl. auch schon das \textit{Plan Set} \is{Plan Set} bei \citet{Han1998}, um den Diskurseffekt von Direktiven aufzufangen. Die Etablierung dieser Komponente trägt der unkontroversen Annahme Rechnung, dass Direktive weniger an Wahrheitsbedingungen \is{Wahrheitsbedingungen}(und damit an Bekenntnisse zur beteiligten Proposition) gebunden sind, sondern vielmehr mit Erfüllensbedingungen \is{Erfüllensbedingungen} (und damit Vorhaben und Absichten) assoziiert werden.

Hinsichtlich der konkreten Ausbuchstabierungen dieser To-Do-Liste gibt es zwischen den Ansätzen zwar durchaus Abweichungen, die Grundidee lässt sich allerdings derart fassen, dass diese Komponente die Absicht des jeweiligen Diskurs\-teilnehmers beinhaltet, einen bestimmten Zustand hervorzubringen.\footnote{Die Arbeiten unterscheiden sich, je nachdem von welcher semantischen Natur die Inhalte der Liste sind. In Portners Arbeiten werden beispielsweise nicht \textit{Propositionen}, sondern \textit{Eigenschaften} in dieser Diskurskomponente gespeichert, während Han, Potts, Ninan und Farkas mit Propositionen arbeiten. Beyssade \& Marandin zufolge denotieren Imperative wieder anders \textit{outcomes}.}

Ich nehme an, dass, genauso wie es die Diskursbekenntnisse der einzelnen Gesprächsteilnehmer $\textrm{DC}_{\textrm{X}}$ gibt, jedem Diskurspartner X zusätzlich eine To-Do-Liste $\textrm{TDL}_{\textrm{X}}$ zugewiesen ist. Diese Liste beinhaltet Sachverhaltsbeschreibungen, deren Aktualisierung vom Inhaber der Liste abhängen. 

Wird im Kontext ein Direktiv geäußert, wird die ausgedrückte Proposition dieser Komponente hinzugefügt, sofern der Adressat keinen Einspruch einlegt.\footnote{\label{Fn6}Obwohl Autoren auf diesen Aspekt hin und wieder verweisen (vgl. z.B. \citealt[374]{Portner2007}, \citealt[214]{Roberts2004}), wird er m.E. nicht weiter beachtet. Dies erinnert an die früheren Charakterisierungen des Diskurseffektes von Assertionen, bei denen dem Adressaten die prinzi\-pielle Ablehnung zwar eingeräumt, aufgrund der Beschaffenheit der Diskursmodelle aber nie so recht ermöglicht wurde. Im Grunde müsste auch eine direktive Äußerung in dem Sinne zunächst als Vorschlag, die TDL zu erweitern, eingeführt werden können. Farkas selbst (\citeyear[323]{Farkas2011}) nimmt an, dass auch ein Imperativ auf den Tisch gelegt wird und das Gespräch in einen Zustand gelenkt wird, in dem p der TDL des Adressaten hinzugefügt wird. S.u. zu inwiefern ein Imperativ m.E. mit dem Tisch interagiert.}

Um diese Proposition notationell von anderen Propositionen abgrenzen zu können, setze ich ein Ausrufezeichen ! vor sie (vgl. \citealt[54-55, 59 Fn 27]{Beyssade2006}). Den Zusammenhang zwischen p und !p sehe ich mit den Autoren so, dass p wahr ist in einer Situation, in der !p erfüllt ist.\footnote{\label{Fn7}Man sieht hier am Beispiel meiner eigenen Modellierung, dass auch bei Autoren, die behaupten, die Elemente in TDL hätten den gleichen Status wie die Inhalte in $\textrm{DC}_{\textrm{X}}$ \glq durch die Hintertür\grq {} ein Unterschied eingeführt wird. M.E. tut sich nicht viel, ob man annimmt, dass die Komponenten sich nicht grundsätzlich unterscheiden, wohl aber die Elemente in den Mengen/Listen, oder ob man die Beschaffenheit der Komponenten für unterschiedlich hält und keine semantischen Unterschiede für die Objekte selbst postuliert. In TDL ist auch in meinem Ansatz schließlich !p enthalten und nicht p. Es führt m.E. kein Weg daran vorbei, einen Unterschied zwischen den wahrheitsbedingungsaffinen Assertionen und erfüllungsbedingungsnahen Direktiven abzubilden. Dieser Unterschied entsteht selbst bei \citet[323]{Farkas2011}, die betont, dass die Elemente in TDL bei ihr den parallelen Status zu Elementen in DC haben. In ihrem Fall tritt dieser Unterschied dadurch auf, dass sich in TDL$_{\textrm{B}}$ z.B. p befindet und sich der Autor des Direktivs (A) dann auch zu p bekennt (weil er davon ausgeht, dass der Adressat den Imperativ akzeptieren wird). In diesem Kontext schreibt sie aber: \glqq the author of the imperative [...] assumes that the future oriented proposition radical is true\grqq{}  (\citeyear[324]{Farkas2011}) Genau den gleichen Status hat p hier folglich auch nicht. Gleiches lässt sich ablesen aus ihrer Auffassung des Zusammenhangs zwischen p in TDL$_{\textrm{X}}$ und p in DC$_{X}$:  \glqq If a proposition p is an element of TODO$_{\textrm{X}}$ at a particular time t$_{1}$ in a conversation c, X is publicly committed to bringing about e$_{p}$, the minimal event that exemplifies p at some time t$_{n}$ that is subsequent to t$_{1}$. It follows that X is also publicly committed to the truth of p.\grqq{} }

Weist B den Direktiv nicht zurück, wird er als !p zum Teil seiner TDL der noch zur Realisierung stehenden Sachverhalte.

Wird ein Direktiv in den Kontext eingeführt, eröffnet sich m.E. auf dem Tisch auch die Frage, ob p Gültigkeit hat. p trifft zu, sobald !p realisiert wurde und p $\vee$ $\neg$p wird erst dann vom Tisch entfernt.

Äußert A \textit{Geh nach Hause!}, resultiert unter diesen Annahmen der Kontextzu\-stand in (\ref{565}) (s.u. zu Erweiterungen).

\newcolumntype{C}[1]{>{\centering}p{#1}}
\begin{exe}
\ex\label{565} K$_1$: A äußert Direktiv: Geh nach Hause! (= !p)\\[-0.6em]
\begin{tabular}[t]{|C{6em}|C{12em}|C{6em}|}
\hline
$\textrm{DC}_{\textrm{A}}$ & Tisch &  $\textrm{DC}_{\textrm{B}}$ \tabularnewline
\hline
{} & p $\vee$ $\neg$p & {}  \tabularnewline
\cline{1-1}\cline{3-3}
$\textrm{TDL}_{\textrm{A}}$ & {} & $\textrm{TDL}_{\textrm{B}}$  \tabularnewline
\cline{1-1}\cline{3-3}
{} & {} & {!p}  \tabularnewline
\hline
\multicolumn{3}{|l|}{cg s$_{1}$} \tabularnewline
\hline
\end{tabular}
\end{exe}
Die Annahme, dass die Äußerung von Direktiven einen Effekt auf den Tisch hat, mag etwas merkwürdig erscheinen (in \citealt[6]{Portner2004} und \citealt[60]{Beyssade2006} interagieren die Komponenten z.B. auch nicht). Insbesondere die Be\-trachtung von \textit{doch} in Imperativen in Abschnitt~\ref{sec:direktive} in Kapitel~\ref{chapter:dua} bietet allerdings Evidenz für diesen Effekt. Im Rahmen einer etwas anderen Diskursmodellierung wird diese Annahme auch in \citet{Gutzmann2011} motiviert. Die Autoren untersuchen den Diskurseffekt des \is{Verum Fokus} \textit{Verum-Fokus} (VF), der u.a. auftritt, wenn das finite Verb fokussiert wird. I.E. ist es der Beitrag verumfokussierter Äußerungen, die aktuell diskutierte Frage aus der Menge der \textit{Question under Discussion} \is{Question Under Discussion} (\textsc{qud}) (die als Komponente in anderen Diskursmodellen etablierter ist und den offenen Themen auf Farkas \& Bruce' \textit{Tisch} entspricht) zu entfernen. Es erfolgt letztlich die Instruktion/der Wunsch, die Diskussion um diesen Aspekt mit der angegebenen Lösung zu beenden (vgl. (\ref{566}) für die relevanten Charakterisierungen rund um die \textsc{qud}, vgl. \citealt[159-162]{Gutzmann2011} für Illustrationen dieses Effektes).

\begin{exe}
	\ex\label{566} Question under Discussion \is{Question Under Discussion}
		\begin{xlist}	
			\ex\label{566a} \textsc{qud}: A partially ordered set that specifies the currently discussable issues. If
	 		a question \textit{q} is \textsc{qud}, it is permissible to provide any information specific to q using (optionally) 			a short answer.
			\ex\label{566b} \textsc{qud} update: Put any question that arises from an utterance on \textsc{qud}.
			\ex\label{566c} \textsc{qud} downdate: When an answer a is uttered, remove all questions resolved by \textit{a} from 			\textsc{qud}.	
			\hfill\hbox {\citet[95]{Engdahl2006}}
		\end{xlist}
\end{exe}
Für eine adäquate Verwendung eines Satzes mit VF muss den beiden Autoren zufolge das Thema, zu dem die verumfokussierte Äußerung einen Beitrag leistet, bereits zur Debatte stehen, d.h. mit anderen Worten \textsc{qud} sein. Andernfalls könne ein verumfokussierter Satz nicht angemessen geäußert werden. Hieraus leiten sie z.B. die Inadäquatheit eines out-of-the-blue geäußerten Satzes mit VF ab. Interessant für die Integration von Direktiven in eine formale Beschreibung von Diskurs ist nun ihre Überlegung, dass auch ein VF-Imperativ bewirkt, dass die \textsc{qud} entfernt wird. Die Instruktion des \textit{Downdating} \is{Downdating} entlang von (\ref{566c}) setzt dann aber voraus, dass der Inhalt des Direktivs schon zur Diskussion steht. Ein Impe\-rativ mit VF ist u.a. akzeptabel, wenn die Anordnung bereits mehrfach gegeben wurde, wie in (\ref{567}).

\begin{exe}
	\ex\label{567}  
	A: John, please, take the chair.\\
	B: (No reaction)\\
	A: Honey, will you please take the chair?\\
	B: (No reaction)\\
	A: \textsc{nimm} dir endlich den Stuhl!
\hfill\hbox {\citet[163]{Gutzmann2011}}
\end{exe}
Sie nehmen deshalb an, dass nach der Äußerung eines Imperativs \is{Imperativ} wie in (\ref{568}) in der \textsc{qud} die Frage enthalten ist: \textit{Nimmt der Adressat den Stuhl oder nimmt er nicht den Stuhl?} (vgl. \citealt[163]{Gutzmann2011}).

\begin{exe}
	\ex\label{568}  
	Nimm den Stuhl!
\end{exe}
Sie gehen ferner davon aus, dass die Frage aus der \textsc{qud} genommen wird, sofern der Adressat mit \textit{Ja}. reagiert. M.E. kann das Thema aber erst als entschieden angesehen werden, wenn der Adressat den Sachverhalt tatsächlich realisiert hat. Da nach meiner Modellierung p wahr ist in einer Situation, in der !p realisiert ist, ist von der Wahrheit von p schließlich noch nicht auszugehen, wenn !p erfolg\-reich in der TDL verankert wird (vgl. auch \citealt[7]{Potts2003}).\footnote{Ist diese Überlegung plausibel, wäre zu überlegen, ob der Diskurseffekt des VF wirklich das Downdating ist, oder ob er nicht eher dem Adressaten die Möglichkeit des Widerspruchs nimmt und !p in dessen TDL verankert. Hierfür spricht auch die Paraphrase des Effektes von VF in Imperativen aus \citet[119]{Hoehle1992} (\textit{Mach es endlich wahr, dass p.}). Wie jeder Imperativ ist auch ein VF-Imperativ abhängig von seiner Realisierung.}

Wie in Fußnote \ref{Fn6} angeführt, legt \citet[323]{Farkas2011} den Direktiv (Imperativsatz + Denotat) auf den Tisch. Sie behandelt diesen Äußerungstyp somit parallel zu Assertionen wie in \citet{Farkas2010} entworfen. Ich habe ihre Modellierung in meiner Darstellung etwas abgewandelt, um das Verhältnis hervorzuheben, dass die Assertion von p die Frage eröffnet, ob p gilt. Da die Annahme bzw. Zurückweisung einer Assertion mit einem Bekenntnis des Adressaten zur ausgedrückten Proposition bzw. zur gleichen Proposition mit entgegengesetzter Polarität einhergeht, ergibt sich m.E. kein Unterschied zwischen der Zustimmung/Ablehnung der Assertion und der Aufnahme von p/$\neg$p in die Diskursbekenntnisse des Hörers bzw. der Entfernung des Deklarativsatzes plus seinem Denotat oder der Disjunktion p $\vee$ $\neg$p vom Tisch. 

Im Falle des Imperativsatzes \is{Imperativsatz} ergibt sich hier allerdings ein Unterschied. Wie oben beschrieben, argumentiere ich, dass sich durch die Äußerung eines Direktivs !p auf dem Tisch die Frage eröffnet, ob p gilt. Anders als bei der Assertion fällt die Akzeptanz des Direktivs hier aber nicht zusammen mit der Annahme von p. p entscheidet sich erst, wenn !p wahr gemacht worden ist. Möchte man die Möglichkeit der Zurückweisung des Direktivs auf dem Tisch abbilden, müsste man neben p $\vee$ $\neg$ p auf dem Tisch ablegen: Wird p zu einem Element von Bs TDL? (etwa p $\in$ $\textrm{TDL}^{\prime}_{\textrm{B}}$ $\vee$ $\neg$(p $\in$ $\textrm{TDL}^{\prime}_{\textrm{B}}$)). Akzeptiert B den Direktiv, nimmt er !p in seine TDL auf. Damit bekennt er sich nicht zu p, sondern zu der Tatsache, dass er p realisieren wird, d.h. dass !p Teil seiner TDL ist. Wenn !p auf Bs TDL steht, befindet sich folglich auch p $\in$ $\textrm{TDL}^{\prime}_{\textrm{B}}$ in $\textrm{DC}_{\textrm{B}}$. Ich denke, dass diese Überlegung letztlich die Motivation in \citet[223/324]{Farkas2011} war, die TDLen als Teilmengen der DC-Systeme anzusehen (vgl. auch schon Fußnote \ref{Fn7}):

\begin{quotation}
I assume that these lists are distinguished subsets of the discourse commitments of participants in the conversation. If a proposition p is an element of $\textrm{TODO}_{\textrm{X}}$ at a particular time $\textrm{t}_{1}$ in a conversation c, X is publicly committed to bringing about $\textrm{e}_{\textrm{p}}$, the minimal event that exemplifies p at some time $\textrm{t}_{\textrm{n}}$ that is subsequent to $\textrm{t}_{1}$. It follows that X is also publicly committed to the truth of p.\\
\newline
The propositional content of the sentence radical is also added to the discourse commitments of the author of the imperative since the author of the imperative assumes acceptance of the imperative by the addressee and therefore assumes that the future oriented proposition radical is true.
\end{quotation}
Angenommen der Sprecher des Imperativs bekennt sich im Zuge der Äußerung dazu, dass p Teil der TDL von B werden wird, kann nach Bs Annahme des Imperativs p $\in$ $\textrm{TDL}_{\textrm{B}}$ auch cg werden, da sich beide Diskursteilnehmer hinsichtlich dieses offenen Aspektes einig sind. Auf die Entscheidung p $\vee$ $\neg$p nimmt diese Kontextentwicklung aber gar keinen Einfluss. Bevor B !p nicht tatsächlich nach\-kommt, steht die Frage \textit{ob p} im Raum. p klärt sich somit erst, wenn B p realisiert hat und damit ein Bekenntnis zu p abgibt, das der andere Sprecher annehmen kann, z.B. wenn B wirklich geht und A die Handlung als solche akzeptiert. (\ref{569}) bis (\ref{571}) fassen die Kontexteffekte zusammen.

\newcolumntype{C}[1]{>{\centering}p{#1}}
\begin{exe}
\ex\label{569} K$_1$: Kontextzustand vor Äußerung des Direktivs\\[-1em]
\begin{tabular}[t]{|C{6em}|C{12em}|C{6em}|}
\hline
$\textrm{DC}_{\textrm{A}}$ & Tisch &  $\textrm{DC}_{\textrm{B}}$ \tabularnewline
\hline
{} & {} & {}  \tabularnewline
\cline{1-1}\cline{3-3}
$\textrm{TDL}_{\textrm{A}}$ & {} & $\textrm{TDL}_{\textrm{B}}$  \tabularnewline
\cline{1-1}\cline{3-3}
{} & {} & {}  \tabularnewline
\hline
\multicolumn{3}{|l|}{cg s$_{1}$} \tabularnewline
\hline
\end{tabular}
\end{exe}

\newcolumntype{C}[1]{>{\centering}p{#1}}
\begin{exe}
\ex\label{570} K$_2$: Kontextzustand nach Äußerung des Direktivs:\\ A: Geh nach Hause! (!p)\footnote{Streng genommen geht es darum, dass !p ein Element der \underline{aktualisierten} TDL wird, d.h. der TDL des Folgekontextzustands.}\\[-1em]
\begin{tabular}[t]{|C{6em}|C{12em}|C{6em}|}
\hline
$\textrm{DC}_{\textrm{A}}$ & Tisch &  $\textrm{DC}_{\textrm{B}}$ \tabularnewline
\hline
!p $\in$ $\textrm{TDL}_{\textrm{B}}$ & !p $\in$ $\textrm{TDL}_{\textrm{B}}$ $\vee$ $\neg$(!p $\in$ $\textrm{TDL}_{\textrm{B}}$)\\ p $\vee$ $\neg$p & {}  \tabularnewline
\cline{1-1}\cline{3-3}
$\textrm{TDL}_{\textrm{A}}$ & {} & $\textrm{TDL}_{\textrm{B}}$  \tabularnewline
\cline{1-1}\cline{3-3}
{} & {} & {}  \tabularnewline
\hline
\multicolumn{3}{|l|}{cg s$_{2}$ = s$_{1}$} \tabularnewline
\hline
\end{tabular}
\end{exe}
\pagebreak
\newcolumntype{C}[1]{>{\centering}p{#1}}
\begin{exe}
	\ex\label{571} Akzeptanz des Direktivs durch B\\[-1.75em]
		\begin{xlist}	
			\ex\label{571a} Teil 1\\[-1em]
			\begin{tabular}[t]{|C{6em}|C{12em}|C{6em}|}
			\hline
			$\textrm{DC}_{\textrm{A}}$ & Tisch &  $\textrm{DC}_{\textrm{B}}$ \tabularnewline
			\hline
			!p $\in$ $\textrm{TDL}_{\textrm{B}}$ & !p $\in$ $\textrm{TDL}_{\textrm{B}}$ $\vee$ $\neg$(!p $\in$ $\textrm{TDL}					_{\textrm{B}}$) {} & !p $\in$ $\textrm{TDL}_{\textrm{B}}$  \tabularnewline
			\cline{1-1}\cline{3-3}
			$\textrm{TDL}_{\textrm{A}}$ & p $\vee$ $\neg$p & $\textrm{TDL}_{\textrm{B}}$  \tabularnewline
			\cline{1-1}\cline{3-3}
			{} & {} & !p  \tabularnewline
			\hline
			\multicolumn{3}{|l|}{cg s$_{3}$ = s$_{2}$} \tabularnewline
			\hline
			\end{tabular}

			\ex\label{571b} Teil 2\\[-1em]
			\begin{tabular}[t]{|C{6em}|C{12em}|C{6em}|}
			\hline
			$\textrm{DC}_{\textrm{A}}$ & Tisch &  $\textrm{DC}_{\textrm{B}}$ \tabularnewline
			\hline
			{}  & p $\vee$ $\neg$p & {}  \tabularnewline
			\cline{1-1}\cline{3-3}
			$\textrm{TDL}_{\textrm{A}}$ & {} & $\textrm{TDL}_{\textrm{B}}$  \tabularnewline
			\cline{1-1}\cline{3-3}
			{} & {} & !p  \tabularnewline
			\hline
			\multicolumn{3}{|l|}{cg s$_{4}$ = $\lbrace \textrm{s}_{2} \cup \lbrace \textrm{!p} \in \textrm{TDL}_{\textrm{B}} 					\rbrace \rbrace$}		
			\tabularnewline
			\hline
			\end{tabular}	
		\end{xlist}
\end{exe}
Lehnt B den Direktiv ab, fügt er !$\neg$p in seine TDL ein (vgl. \citealt[325]{Farkas2011}). Die beiden Diskursteilnehmer können sich in diesem Fall hinsichtlich \textit{p $\in$ $\textrm{TDL}_{\textrm{B}}$}? nicht einigen.

Meine Modellierung bildet die Intuitionen aus den beiden Ansätzen (\citealt{Gutzmann2011} und \citealt{Farkas2011}) ab, die hinsichtlich des Diskurseffektes von Direktiven mehr beisteuern als die Komponente der TDL überhaupt einzuführen. Sie fängt gleichzeitig den Aspekt auf, den ich an beiden kritisiere, nämlich dass man zwischen der Frage, ob der Adressat den Direktiv akzeptiert, und der Frage, ob die im Direktiv enthaltene Proposition wahr ist, trennen sollte.

Die Erweiterung des Diskursmodells um die Komponente TDL erlaubt es mir also, sowohl den Einfluss von Assertionen als auch Direktiven auf den Kontext innerhalb desselben Modells abzubilden.\footnote{Sicherlich bleiben einige beteiligte Aspekte unberücksichtigt bei einer solchen Integration der Komponente der TDL in das Modell aus \citet{Farkas2010}. Insbesondere in \citet{Portner2007} findet sich eine formale Modellierung dieser Diskurskomponente. Er formuliert beispielsweise auch die Interaktion der TDLen mit dem cg und nimmt je nach Illokution unterschiedlich \glqq gefärbte\grqq{} TDLen an. Auch in \citet{Farkas2011} wird dieser Aspekt integriert und die Autorin spezifiziert (u.a. zu diesem Zweck) in den Imperativen weiter \textit{source} und \textit{dependent} nach \citet{Gunlogson2008}. Meine Absicht ist an dieser Stelle, den Diskurseffekt von Direktiven (vor allem in Kontrast zum Effekt von Assertionen) im Rahmen der hier vertretenen Modellierung von Diskursveränderungen zu beschreiben und nicht eine ausgefeilte Imperativsemantik abzubilden. Je nach untersuchtem Phänomen spielen die obigen Aspekte (und vermutlich auch einige weitere) ggf. eine Rolle.}

Abschnitt~\ref{sec:kontexte} diskutiert nun, inwiefern sich MP-lose Assertionen und Direktive von \textit{eben}/\textit{halt}-Assertionen und \textit{eben}/\textit{halt}-Direktiven unterscheiden.

\subsection{\textit{halt}-/\textit{eben}-Äußerungen und ihre Kontextzustände}
\label{sec:kontexte}
Meine Analyse von \textit{eben}- und \textit{halt}-Äußerungen baut auf den oben bereits angeführten Annahmen von \citet{Karagjosova2003}; (\citeyear{Karagjosova2004}) auf, die ebenfalls eine Modellierung in einem formalen Diskursmodell vorgeschlagen hat (vgl. auch \citealt[152-159]{Mueller2016a}). Für die Autorin leisten die beiden MPn einen identi\-schen Beitrag. Ich werde anders für einen Unterschied argumentieren und unter Bezug auf die in Abschnitt~\ref{sec:untersch} erläuterten deskriptiven Aspekte Differenzierungen ihres jeweiligen Diskurseffektes modellieren.

Die Kontextsituation, in der eine \textit{eben}-Äußerung wie in (\ref{572}) gemacht wird, modelliert \citet[208]{Karagjosova2004} derart, dass die \textit{eben}-Proposition (p) sowie die Inferenzrelation p $>$ q (\glq Wenn man lange krank war, sieht man schlecht aus.\grq {}) den Diskursteilnehmern bewusst bekannt ist, weshalb über den \textit{Modus Ponens} \is{modus ponens} auf der Basis von a) \textit{bekannt ist p} und b) \textit{bekannt ist p $>$ q} die Bekanntheit von q abzuleiten ist.\footnote{Wenn man es ganz genau nimmt, müsste man annehmen, dass die beiden im Dialog auftretenden Propositionen konkrete Instantiierungen des eigentlich generischen Zusammenhangs der Inferenzrelation sind. Ich vernachlässige diesen Aspekt im Folgenden aber ebenfalls. In den meisten Fällen ist die Inferenzrelation tatsächlich allgemeinerer Natur als die im Dialog beteiligten Propositionen.
}

\begin{exe}
	\ex\label{572} \textit{eben}-Begründung/Erklärung\\
	B: Peter sieht schlecht aus. (= q)\\
	A: Er war \textbf{eben} lange krank. (= p)	
\end{exe}
Im Kontext vor der \textit{eben}-Äußerung (vgl. (\ref{573})) ist die Inferenzrelation p $>$ q (\glq Wenn man lange krank war, sieht man schlecht aus.\grq {}) Teil des cg\footnote{Ein Gutachter wirft die Frage auf, warum eine derartige Relation nicht auch bei der Modellierung von \textit{doch} im cg angenommen wird. In Abschnitt~\ref{sec:doch1} habe ich gezeigt, dass die Offenheit des Themas auf verschiedene Arten zustandekommen kann. Es muss nicht eine Implikatur beteiligt sein. Im Fall von \textit{eben} ist eine derartige Relation m.E. aber immer beteiligt. Es wäre für meine Begriffe nicht korrekt, die Relation im cg zum Teil der Kontextanforderung (und somit inhärenten Bedeutung) einer \textit{doch}-Äußerung zu machen.} und in Bs Bekenntnissystem ist p (= er war lange krank) verankert. Dies sind die Füllungen der Komponenten wie sie von der MP verlangt werden. B assertiert q, wodurch er ein öffentliches Diskursbekenntnis zu q macht und auf dem Tisch die Frage eröffnet wird, ob q, d.h. dort liegt nach Bs Äußerung die Disjunktion q $\vee$ $\neg$q. 
\pagebreak
\newcolumntype{C}[1]{>{\centering}p{#1}}
\begin{exe}
	\ex\label{573} Kontext vor der \textit{eben}-Äußerung: B: \textit{Peter sieht schlecht aus.} (= q)\\[-1em]
 		\begin{tabular}[t]{|C{6em}|C{6em}|C{6em}|} 
 		\hline 	
   		$\textrm{DC}_{\textrm{A}}$ & {Tisch} & \textbf{$\textrm{DC}_{\textrm{B}}$} \tabularnewline
  		\hline
   		{} & q $\vee$ $\neg$q & \textbf{q}\\\textbf{p} \tabularnewline
  		\hline      
   		\multicolumn{3}{|l|}{\textbf{cg s$_{1}$ = $\lbrace$p $>$ q$\rbrace$}} \tabularnewline   
  		 \hline
 		\end{tabular}
\end{exe}
Vor dem Hintergrund dieses Kontextzustands äußert A p (= er war lange krank). Dadurch macht A ein öffentliches Diskursbekenntnis zu p und legt p (zunächst mit seiner Alternative) auf den Tisch (vgl. (\ref{574a})). Da B bereits ein Diskursbe\-kenntnis zu p hat (das gebraucht wird, um den Inhalt der Assertion von A zu einem cg-Inhalt zu machen), gelangt p direkt in den cg (vgl. (\ref{574b}). Dieses Verhältnis entspricht dem Beitrag von \textit{ja} (vgl. Abschnitt~\ref{sec:ja}). Da im cg ein pragmatischer Zusammenhang zwischen p und q besteht, ist qua Modus Ponens \is{modus ponens} auch q cg. A hat folglich auch ein Bekenntnis zu q, denn der cg ist eine Teilmenge der individuellen Diskursbekenntnisse der Gesprächsteilnehmer. D.h. es ist nicht möglich, dass sich Propositionen im cg befinden, aber nicht in den individuellen Bekenntnismengen.

Diese Verhältnisse bedeuten, q hat sowieso Gültigkeit und B hätte dies auch wissen können. Schließlich sind p und p $>$ q unter seinen Bekenntnissen. Da keine Themen mehr zur Verhandlung stehen, ist der Tisch leer und ebenso leeren sich die Diskursbekenntnissysteme von A und B (vgl. (\ref{574c})).

\newcolumntype{C}[1]{>{\centering}p{#1}}
\begin{exe}
	\ex\label{574} Kontext nach der \textit{eben}-Äußerung: A: \textit{Er war \textbf{eben} lange krank.} (= p)\\[-1.75em]
		\begin{xlist}	
			\ex\label{574a} Teil 1\\[-1em]
			\begin{tabular}[t]{|C{6em}|C{12em}|C{6em}|}
			\hline
			$\textrm{DC}_{\textrm{A}}$ & Tisch &  $\textrm{DC}_{\textrm{B}}$ \tabularnewline
			\hline
			p\\q  & p $\vee$ $\neg$p\\q $\vee$ $\neg$q & p\\q  \tabularnewline
			\cline{1-1}\cline{3-3}
			\hline
			\multicolumn{3}{|l|}{cg s$_{2}$ = s$_{1}$}		
			\tabularnewline
			\hline
			\end{tabular}	

			\ex\label{574b} Teil 2\\[-1em]
			\begin{tabular}[t]{|C{6em}|C{12em}|C{6em}|}
			\hline
			$\textrm{DC}_{\textrm{A}}$ & Tisch &  $\textrm{DC}_{\textrm{B}}$ \tabularnewline
			\hline
			q  & q $\vee$ $\neg$q & q  \tabularnewline
			\cline{1-1}\cline{3-3}
			\hline
			\multicolumn{3}{|l|}{cg $\textrm{s}_{3} = \lbrace \textrm{s}_{2} \cup \lbrace \textrm{p} \rbrace \rbrace$}		
			\tabularnewline
			\hline
			\end{tabular}
			
			\ex\label{574c} Teil 3\\[-1em]
			\begin{tabular}[t]{|C{6em}|C{12em}|C{6em}|}
			\hline
			$\textrm{DC}_{\textrm{A}}$ & Tisch &  $\textrm{DC}_{\textrm{B}}$ \tabularnewline
			\hline
			{} & {} & {}  \tabularnewline
			\cline{1-1}\cline{3-3}
			\hline
			\multicolumn{3}{|l|}{cg s$_{4}$ = $\lbrace$ p $>$ q, p, q$\rbrace$}		
			\tabularnewline
			\hline
			\end{tabular}			
		\end{xlist}
\end{exe}
(\ref{575}) und (\ref{576}) fassen den diskursiven Effekt der \textit{eben}-Assertion zusammen.
\pagebreak
\newcolumntype{C}[1]{>{\centering}p{#1}}
\begin{exe}
	\ex\label{575} Kontext vor der \textit{eben}-Äußerung\\[-1em]
			\begin{tabular}[t]{|C{6em}|C{12em}|C{6em}|}
			\hline
			$\textrm{DC}_{\textrm{A}}$ & Tisch &  $\textrm{DC}_{\textrm{B}}$ \tabularnewline
			\hline
			{} & {} & p \tabularnewline
			(q) & {} & (q)  \tabularnewline
			\hline
			\multicolumn{3}{|l|}{cg s$_{1}$ = $\lbrace$p $>$ q$\rbrace$}		
			\tabularnewline
			\hline
			\end{tabular}	
\end{exe}

\newcolumntype{C}[1]{>{\centering}p{#1}}
\begin{exe}
	\ex\label{576} Kontext nach der \textit{eben}-Äußerung\\[-1em]
			\begin{tabular}[t]{|C{6em}|C{12em}|C{6em}|}
			\hline
			$\textrm{DC}_{\textrm{A}}$ & Tisch &  $\textrm{DC}_{\textrm{B}}$ \tabularnewline
			\hline
			{}  & {} & {}  \tabularnewline
			\hline
			\multicolumn{3}{|l|}{cg s$_{2}$ = $\lbrace$s$_{1}$ $\cup$ $\lbrace$p$\rbrace\rbrace$}		
			\tabularnewline
			\hline
			\end{tabular}	
\end{exe}
Diese Modellierung erfasst die Eigenschaften, die ich in Abschnitt~\ref{sec:bedhe} deskriptiv angeführt habe. Der Aspekt der Rückorientierung wird durch die Inferenzrelation, an der eine andere Proposition beteiligt ist, aufgefangen.

In monologischen Kontexten (vgl. (\ref{578})), ist q Teil von $\textrm{DC}_{\textrm{A}}$, in dialogischen (vgl. (\ref{572})) von $\textrm{DC}_{\textrm{B}}$. 

\begin{exe}
	\ex\label{578} 						
	\glqq Um mich ausleben zu können, wie ich es will, bleibt mir nichts anderes übrig, als
 	das Alleinsein. (= q) Es ist eben so, daß Nähe mich tötet. (= p)\grqq{} 	
 	\newline
	\hbox{}\hfill\hbox {\citet[100]{Dahl1988}}	
 \end{exe}
Denkbar ist ebenfalls, dass q bereits im cg enthalten ist. In (\ref{579}), wobei es sich um einen ganz typischen Kontext für eine \textit{eben}-Assertion handelt, ist q durch die \textit{wieso}-Frage präsupponiert \is{Präsupposition} und somit Teil des cg.	Unter diesen Umständen befindet sich q sowohl in $\textrm{DC}_{\textrm{A}}$ als auch $\textrm{DC}_{\textrm{B}}$.

\begin{exe}
	\ex\label{579} 						
	B: Wieso braucht denn dein Mann immer so lang zum Abspülen?\\
	A: Er kann sich \textbf{eben} nicht konzentrieren.	
 	\hfill\hbox {\citet[322]{Troemel-Ploetz1979}}	
 \end{exe}
Die minimale Anforderung ist somit, dass q mindestens in einem der beiden DC-Systeme enthalten ist. Um welches es sich handelt, ist abhängig von der konkreten Dialogsituation.\footnote{Dieser Aspekt wird in (\ref{575}) durch die runden Klammern angezeigt.}

Die konkrete Art der Relation ist hier ein kausaler Zusammenhang. Urteile wie \textit{kategorisch}, \textit{absolut}, \textit{Thema beendend}, \textit{widerspruchslos}, \textit{evident} und \textit{wiederholt} in Bezug auf die \textit{eben}-Proposition bzw. die Relation p $>$ q ergeben sich dadurch, dass p und p $>$ q cg sind bzw. cg werden.\footnote{Man kann sich sicherlich auch fragen, ob die Inferenzrelation – genauso wie p – im Zuge der \textit{eben}-Assertion erst zum Teil des cg werden kann und dies nicht schon im Vorgängerkontext ist. Ich denke, dies ist prinzipiell möglich. In diesem Fall würde auch die Relation p $>$ q mit ihrer Alternative (dass sie nicht besteht: $\neg$(p $>$ q)) auf den Tisch gelegt und würde aufgrund des vorweggenommenen Hörerbekenntnisses zum Teil des cg werden. Diese Modellierung würde an meiner weiteren Argumentation auch nichts ändern. Es handelt sich bei der beteiligten Inferenzrelation in der Regel aber tatsächlich um stereotype Zusammenhänge, die deshalb als Teil des cg angesehen werden können. Und dies unterscheidet sie vom Status der Proposition p, die durchaus erst durch die \textit{eben}-Assertion zum Bestand des cg werden kann. Dies spiegelt sich auch in Beispieldialogen wider. Die Durchsicht der in der Literatur angeführten Beispiele führt mich zu der Ansicht, dass p vor der MP-Äußerung ggf. durchaus noch nicht cg ist, weil p vom Sprecher entweder akkommodiert wird oder einfach noch keine Einigkeit hinsichtlich dieser Proposition hergestellt worden ist, dass dies aber nicht in gleicher Weise für die Relation p $>$ q gilt. Es scheint mir beispielsweise nur schwer vorstellbar, dass ein Diskurs derart verläuft, dass der Hörer in einem bestimmten Stadium ein Bekenntnis zu p $>$ q äußert, dieser Aspekt aber im Kontext offen bleibt, und der Diskurspartner im späteren Verlauf eine \textit{eben}-Äußerung macht, durch die die Relation p $>$ q zu einem cg-Inhalt wird. Analoges halte ich aber für durchaus plausibel in Bezug auf p allein.}

Ebenfalls erfasst wird der von \citet{Dahl1988} geäußerte Eindruck, die vorweggehende Äußerung werde vom Sprecher der MP-Äußerung abgewertet: Da q cg ist bzw. cg wird, handelt es sich bei dieser Proposition nicht um einen Inhalt, den B mit seiner Assertion als neue und zur Verhandlung stehende Information in den Diskurs einführen kann (vgl. auch \citealt[340-341]{Karagjosova2003}; \citeyear[208-209]{Karagjosova2004}).

Aus der Modellierung, bei der \textit{eben} p zum Teil des cg macht bzw. p aus dem cg hervorholt, sind auch die Ergebnisse aus \citet[185]{Hentschel1986} zu emotionalen/ex\-pressiven Konnotationen von \textit{eben} abzuleiten (vgl. auch \citealt[125, Fn 47]{Thurmair1989}). Ihr zufolge assoziiert man mit \textit{eben} Adjektive wie \textit{hart}, \textit{klar}, \textit{stark}, \textit{aktiv}, \textit{selbstbewusst} und \textit{egoistisch}. Dies sind Eigenschaften, die zutreffend sind, wenn ein Sachverhalt als Teil des cg ausgegeben wird und deshalb nicht weiter zur Verhandlung steht.\\

\noindent
Die Modellierung der bei einer \textit{halt}-Äußerung (vgl. (\ref{580})) beteiligten Effekte ist meiner Auffassung nach leicht anders. Genau diese Änderung fängt aber die in Abschnitt~\ref{sec:untersch} beschriebenen unterschiedlichen Interpretationen auf, die ebenfalls nur leicht vom Beitrag der MP \textit{eben} abweichen.

\begin{exe}
	\ex\label{580} \textit{halt}-Begründung/Erklärung\\					
	B: Peter sieht schlecht aus. (= q)\\
	A: Er war \textbf{halt} lange krank. (= p)	
\end{exe}
(\ref{581}) zeigt, wie ich die Diskurskomponenten im Kontextzustand vor der \textit{halt}-Äußerung in (\ref{580}) gefüllt sehe.
\pagebreak
\newcolumntype{C}[1]{>{\centering}p{#1}}
\begin{exe}
	\ex\label{581} Kontext vor der \textit{halt}-Äußerung\\[-1em]
			\begin{tabular}[t]{|C{6em}|C{12em}|C{6em}|}
			\hline
			\textbf{$\textrm{DC}_{\textrm{A}}$} & Tisch &  \textbf{$\textrm{DC}_{\textrm{B}}$}\tabularnewline
			\hline
			\textbf{p $>$ q} & & \tabularnewline
			& q $\vee$ $\neg$q & \textbf{q}  \tabularnewline	
			\hline
			\multicolumn{3}{|l|}{cg s$_{1}$}		
			\tabularnewline
			\hline
			\end{tabular}	
\end{exe}
Im Bekenntnissystem von A ist die Inferenzrelation p $>$ q (\glq Wenn man lange krank war, sieht man normalerweise schlecht aus.\grq {}) enthalten. D.h. ich nehme an, dass es sich bei der Inferenzrelation anders als bei \textit{eben} nicht um einen Inhalt handelt, über den sich Sprecher und Hörer einig sind (zur Evidenz für diese Annahme s.u.). Die Relation ist deshalb nicht Teil des cg. Ich argumentiere dafür, dass sie nur dem Sprecher der MP-Äußerung zugeschrieben werden können muss. Ebenfalls ist p nicht im System von B verankert, so dass p von A im Zuge der MP-Äußerung erstmals in den Kontext eingeführt wird. Aufgrund von Bs vorweggehender Assertion mit propositionalem Gehalt q in der konkreten Dialogsituation in (\ref{580}), hat B im Kontext vor der MP-Äußerung hier ein öffentliches Diskursbekenntnis zu q. Durch die Assertion wird q auf den Tisch gelegt und seine Alternative öffnet sich.

Der Effekt von As sich anschließender \textit{halt}-Äußerung (mit p = er war lange krank) ist, dass p zu einem öffentlichen Diskursbekenntnis von A wird und p ($\vee$ $\neg$p) auf den Tisch gelangt (vgl. (\ref{582a})).

Wenn A p assertiert, führt dies mit sich, dass für A gilt, dass auch q der Fall ist. Aus As Sicht ist klar, dass Peter schlecht aussieht, weil er lange krank war. A bekennt sich folglich zu p und zu q, weil A q aus p und p $>$ q schlussfolgert. Diesen Zusammenhang hält aber (anders als bei \textit{eben}) nur A für gegeben, für B werden keine Entscheidungen mitgetroffen. 

Die Proposition q wird hier aber cg, da B ebenfalls ein Bekenntnis zu q hat. Dass q sowieso Gültigkeit hat, gilt hier allerdings nur für die Ansichten von A, d.h. es handelt sich in As Augen um keine Neuigkeit, dass Peter schlecht aussieht. Das Thema p $\vee$ $\neg$p (war Peter lange krank oder war er es nicht) ist folglich weiterhin offen (vgl. (\ref{582b})). A geht zwar von p aus, es gibt aber keinen Grund zur Annahme, dass B p auch annehmen muss.

\newcolumntype{C}[1]{>{\centering}p{#1}}
\begin{exe}
	\ex\label{582} Kontext nach der \textit{halt}-Assertion\\[-1.75em]
		\begin{xlist}
		\ex\label{582a} Teil 1\\[-1em]
			\begin{tabular}[t]{|C{6em}|C{12em}|C{6em}|}
			\hline
			$\textrm{DC}_{\textrm{A}}$ & Tisch & $\textrm{DC}_{\textrm{B}}$ \tabularnewline
			\hline
			p $>$ q & & \tabularnewline
			p & p $\vee$ $\neg$p & \tabularnewline
			q & p $\vee$ $\neg$q & q \tabularnewline
			\hline
			\multicolumn{3}{|l|}{cg s$_{2}$ = s$_{1}$}		
			\tabularnewline
			\hline
			\end{tabular}	
			
		\ex\label{582b} Teil 2\\[-1em]
			\begin{tabular}[t]{|C{6em}|C{12em}|C{6em}|}
			\hline
			$\textrm{DC}_{\textrm{A}}$ & Tisch & $\textrm{DC}_{\textrm{B}}$ \tabularnewline
			\hline
			p & p $\vee$ $\neg$p & \tabularnewline
			p $>$ q &  & \tabularnewline
			\hline
			\multicolumn{3}{|l|}{cg s$_{3}$ = $\lbrace$s$_{2}$ $\cup$ $\lbrace$q$\rbrace\rbrace$}		
			\tabularnewline
			\hline
			\end{tabular}		
		\end{xlist}
\end{exe}
Mit diesem Verhältnis geht der Effekt einher, dass der Informativitätswert \is{Informativität} von Bs Äußerung (die die Proposition q enthält) nicht – wie im Falle der \textit{eben}-Äußerung – derart herabgestuft wird. Wie oben illustriert, drückt A dort aus, dass auch für B q sowieso gelten sollte, weil p und p $>$ q im cg enthalten sind. Im Kontext einer \textit{halt}-Äußerung ist q für den Sprecher abzuleiten, nicht aber für B. Meine Füllung der Diskurskomponenten im Kontext einer \textit{halt}-Assertion erfasst somit, warum \textit{halt}-Äußerungen weniger harsch, schwächer und eher angreifbar etc. wirken (s.o.). Hier fügen sich auch die von \citet[193]{Hentschel1986} angenommenen Konnotationen wie \textit{warm}, \textit{herzlich}, \textit{persönlich}, \textit{anteilnehmend} ein. \textit{Halt}-Äußerungen sind sprecherorientierter und beziehen sich unaufdringlicher als \textit{eben}-Äußerungen allein auf die Annahmen des Sprechers. (\ref{583}) und (\ref{584}) fassen den \textit{halt}-Beitrag in den Diskurskomponenten zusammen.

\newcolumntype{C}[1]{>{\centering}p{#1}}
\begin{exe}
	\ex\label{583} Kontext vor der \textit{halt}-Assertion\\[-1em]
			\begin{tabular}[t]{|C{6em}|C{12em}|C{6em}|}
			\hline
			$\textrm{DC}_{\textrm{A}}$ & Tisch &  $\textrm{DC}_{\textrm{B}}$ \tabularnewline
			\hline
			p $>$ q & & \tabularnewline
			(q) & & (q) \tabularnewline
			\hline
			\multicolumn{3}{|l|}{cg s$_{1}$}		
			\tabularnewline
			\hline
			\end{tabular}	
\end{exe}

\newcolumntype{C}[1]{>{\centering}p{#1}}
\begin{exe}
	\ex\label{584} Kontext nach der \textit{halt}-Assertion\\[-1em]
			\begin{tabular}[t]{|C{6em}|C{12em}|C{6em}|}
			\hline
			$\textrm{DC}_{\textrm{A}}$ & Tisch &  $\textrm{DC}_{\textrm{B}}$ \tabularnewline
			\hline
			p $>$ q & p $\vee$ $\neg$p & \tabularnewline
			p & &  \tabularnewline
			q & & \tabularnewline
			\hline
			\multicolumn{3}{|l|}{cg s$_{2}$ = s$_{1}$}		
			\tabularnewline
			\hline
			\end{tabular}	
\end{exe}
Wie im Fall von \textit{eben} kann die für den (kausalen) Rückverweis notwendige Proposition q in DC$_{\textrm{A}}$, DC$_{B}$ oder im cg enthalten sein. Hinsichtlich der Verankerung dieser Proposition unterscheiden sich \textit{halt} und \textit{eben} folglich nicht. Aufgrund der unterschiedlichen Füllung der anderen Systeme, gelangt q allerdings in verschiedene Komponenten. Ist q Teil von DC$_{A}$ (in monologischen Fällen $[$vgl. (\ref{585})$]$, ist q nach der \textit{halt}-Äußerung lediglich in DC$_{A}$).

\begin{exe}
	\ex\label{585}
	Es war nur ein Rappel, meint der Doktor, nicht das Herz. (=q) Sie ist \textbf{halt} wetterfühlig (= p), und die senile 				Demenz wird auch schuld sein.  
	\newline
	\hbox{}\hfill\hbox {\citet[125]{Thurmair1989}}
\end{exe}	
Liegt ein dialogischer Kontext vor wie in (\ref{580}), ist q in DC$_{\textrm{B}}$ enthalten und wird im Zuge der \textit{halt}-Assertion cg. Aufgrund des Implikationsverhältnisses \is{Implikation} zwischen cg und DC$_{A/B}$ ist q dann aber auch auf jeden Fall unter As Bekenntnissen.

Die leichten Interpretationsunterschiede zwischen \textit{halt}- und \textit{eben}-Äußerungen (vgl. Abschnitt~\ref{sec:untersch}) fange ich dadurch auf, dass \textit{halt}-Äußerungen ihre Proposition \is{Assertion} \underline{assertieren}, während \textit{eben}-Äußerungen \is{Präsupposition} sie \underline{präsupponieren}. Im Falle von \textit{halt} macht der Sprecher ein Bekenntnis zu p und eröffnet damit das Thema. Mit \textit{eben} greift er auf eine bekannte Proposition zurück bzw. bewirkt die Aufnahme von p in den cg. Der Zusammenhang zwischen p und q ist dazu im einen Fall (bei \textit{eben}) geteiltes Wissen im cg, im anderen (bei \textit{halt}) allein Annahme des Spre\-chers. Das Kriterium, das die beiden MP-Äußerungen teilen (Rückbezug und (hier) Begründung/Erklärung), trifft auf \textit{halt} in meiner Modellierung genauso zu wie auf \textit{eben} (s.o.). Die Inferenzrelation \is{Inferenzrelation} p $>$ q ist schließlich jeweils gleichermaßen beteiligt, sie ist nur in einer anderen Komponente verankert. 

Wie schon in Kapitel~\ref{chapter:jud} in Abschnitt~\ref{sec:inkdm} erläutert, handelt es sich bei der Mo\-dellierung des Beitrags der MPn um einen \textit{bedeutungsminimalistischen} \is{Bedeutungsminimalismus/-maximalismus} Zugang, d.h. intendiert ist die denkbar restriktivste Füllung der Diskurskomponenten, die des\-halb auf alle \textit{halt}/\textit{eben}-Verwendungen zutrifft. Wie anhand der Abbildung der Beispiele in diesem Modell oben zu sehen war, werden die Komponenten in konkreten Äußerungssituationen natürlich ggf. weiter gefüllt. Die Annahme ist aber, dass die Füllungen aus (\ref{575}) und (\ref{576}) bzw. (\ref{583}) und (\ref{584}) immer beteiligt sind, während die übrigen diskursiven Verhältnisse ggf. auch variieren können. Ich würde deshalb beispielsweise auch nicht \textit{halt}-Verwendungen prinzipiell aus\-schließen, in denen die Relation zwischen p und q im cg enthalten ist. Dies ist in (\ref{586}) beispielsweise denkbar. Meiner Analyse nach genügt es allerdings, dass sie Teil der Sprecherbekenntnisse ist.	

\begin{exe}
	\ex\label{586} Du kannst deine Freunde schon mitbringen.
		\begin{xlist}	
			\ex\label{586a} Wir haben \textbf{halt} kein Bier mehr.
			\ex\label{586b} *Wir haben \textbf{eben} kein Bier mehr. 
		\end{xlist}
\end{exe}
Obwohl die \textit{eben}-Äußerung, bei der meiner Analyse nach p $>$ q im cg sein muss, unangemessen ist, kann hier dennoch Einigkeit zwischen den Diskursteilnehmern darüber bestehen, dass man keine weiteren Gäste einlädt, wenn kein Bier mehr da ist. In (\ref{586}) ergibt sich die Differenz zwischen der \textit{eben}- und der \textit{halt}-Verwendung dadurch, dass p (= wir haben kein Bier mehr) beim Gesprächspartner (aufgrund seiner Frage nach dem Mitbringen weiterer Gäste) nicht als bekannt angenommen werden kann. D.h. p wird vom Sprecher assertiert \is{Assertion} und kann aufgrund des Kontextes nicht plausibel präsupponiert \is{Präsupposition} sein bzw. dem Hörer als Bekenntnis zugeschrieben werden.

Die bisher in diesem Abschnitt besprochenen Fälle waren Beispiele, in denen die MP-Äußerung als Begründung/Erklärung fungiert. M.E. lassen sich unter Annahme der gleichen Diskurseffekte von \textit{eben} und \textit{halt} auch Beispiele erfassen, in denen die MP-Äußerungen die Folge in einem Bedingungs-Folge-Gefüge \is{Bedingungs-Folge-Gefüge} darstellen.

Folge ist die \textit{eben}-Äußerung z.B. in (\ref{587}).

\begin{exe}
	\ex\label{587} \textit{eben}-Folge\\
	B: Ich schaffe es nicht bis morgen! (= q)\\
	A: Arbeite \textbf{eben} schneller! (= !p)
\end{exe}	
Genauso wie im Falle der \textit{eben}-Begründung ist die Inferenzrelation q $>$ !p im Kontext vor der \textit{eben}-Äußerung Teil des cg (vgl. (\ref{588a})). B hat ein Bekenntnis zu q, weshalb sich auf dem Tisch die Frage eröffnet, ob von q auszugehen ist. Da q $>$ !p im cg ist und B sich zu q bekennt, folgt qua \is{modus ponens} Modus Ponens, dass auch !p auf Bs To-Do-Liste stehen sollte. In DC$_{\textrm{B}}$ ist deshalb auch das Bekenntnis enthalten, dass p in TDL$_{\textrm{B}}$ verankert ist. Sofern A q nicht ablehnt (und damit bestätigt) (vgl. (\ref{588b})), wird q in den cg aufgenommen (vgl. (\ref{588c})) und das Thema wird vom Tisch entfernt. A und B sind sich einig, dass das Problem besteht, dass B seine Arbeit nicht bis zum nächsten Tag fertig stellen wird.

\newcolumntype{C}[1]{>{\centering}p{#1}}
\begin{exe}
	\ex\label{588} Kontext vor der \textit{eben}-Äußerung: B: \textit{Ich schaffe es nicht bis morgen!} (= q)\\[-1.75em]
	\begin{xlist}
		\ex\label{588a} Teil 1\\[-1em]
			\begin{tabular}[t]{|C{6em}|C{12em}|C{6em}|}
			\hline
			$\textrm{DC}_{\textrm{A}}$ & Tisch &  \textbf{$\textrm{DC}_{\textrm{B}}$} \tabularnewline
			\hline
			{} & q $\vee$ $\neg$q & \textbf{q}\\!p $\in$ TDL$_{\textrm{B}}$  \tabularnewline
			\cline{1-1}\cline{3-3}
			$\textrm{TDL}_{\textrm{A}}$ & p $\vee$ $\neg$p & \textbf{$\textrm{TDL}_{\textrm{B}}$}  \tabularnewline
			\cline{1-1}\cline{3-3}
			{} & {} & {\textbf{!p}}  \tabularnewline
			\hline
			\multicolumn{3}{|l|}{\textbf{cg s$_{1}$ = $\lbrace$q $>$ !p$\rbrace$}} \tabularnewline
			\hline
			\end{tabular}

	\ex\label{588b} Teil 2\\[-1em]
			\begin{tabular}[t]{|C{6em}|C{12em}|C{6em}|}
			\hline
			$\textrm{DC}_{\textrm{A}}$ & Tisch &  $\textrm{DC}_{\textrm{B}}$ \tabularnewline
			\hline
			q & q $\vee$ $\neg$q & q\\!p $\in$ TDL$_{\textrm{B}}$  \tabularnewline
			\cline{1-1}\cline{3-3}
			$\textrm{TDL}_{\textrm{A}}$ & p $\vee$ $\neg$p & $\textrm{TDL}_{\textrm{B}}$  \tabularnewline
			\cline{1-1}\cline{3-3}
			{} & {} & !p  \tabularnewline
			\hline
			\multicolumn{3}{|l|}{cg s$_{2}$ = s$_{1}$} \tabularnewline
			\hline
			\end{tabular}

	\ex\label{588c} Teil 3\\[-1em]
			\begin{tabular}[t]{|C{6em}|C{12em}|C{6em}|}
			\hline
			$\textrm{DC}_{\textrm{A}}$ & Tisch &  $\textrm{DC}_{\textrm{B}}$ \tabularnewline
			\hline
			{} & {} & !p $\in$ TDL$_{\textrm{B}}$  \tabularnewline
			\cline{1-1}\cline{3-3}
			$\textrm{TDL}_{\textrm{A}}$ & p $\vee$ $\neg$p & $\textrm{TDL}_{\textrm{B}}$  \tabularnewline
			\cline{1-1}\cline{3-3}
			{} & {} & !p  \tabularnewline
			\hline
			\multicolumn{3}{|l|}{cg s$_{3}$ = $\lbrace$s$_{2}$ $\cup$ $\lbrace$q$\rbrace\rbrace$} \tabularnewline
			\hline
			\end{tabular}
\end{xlist}		
\end{exe}
Vor diesem Hintergrund äußert A den Direktiv !p und führt damit eine zu realisierende Proposition in den Kontext ein, die sich bereits auf der TDL von B befindet. A fordert B zu einer Handlung auf, die B sowieso vorhaben sollte zu tun. Da !p bereits in TDL$_{\textrm{B}}$ enthalten ist, besteht für B keine Möglichkeit der Ablehnung (vgl. (\ref{589a})). Da B ebenfalls bereits ein Bekenntnis dazu hat, dass !p auf seiner TDL steht, kann dieses Thema, das A mit seiner Äußerung des Direktivs eröffnet, auch direkt (d.h. ohne Bestätigung durch B abzuwarten) dem cg hinzugefügt werden.

\newcolumntype{C}[1]{>{\centering}p{#1}}
\begin{exe}
	\ex\label{589} Kontext nach der \textit{eben}-Äußerung: A: \textit{Arbeite \textbf{eben} schneller!} (= !p)\\[-1.75em]
	\begin{xlist}
		\ex\label{589a} Teil 1\\[-1em]
			\begin{tabular}[t]{|C{6em}|C{12em}|C{6em}|}
			\hline
			$\textrm{DC}_{\textrm{A}}$ & Tisch &  $\textrm{DC}_{\textrm{B}}$ \tabularnewline
			\hline
			!p $\in$ TDL$_{\textrm{B}}$ & p $\vee$ $\neg$p & !p $\in$ TDL$_{\textrm{B}}$  \tabularnewline
			\cline{1-1}\cline{3-3}
			$\textrm{TDL}_{\textrm{A}}$ & !p $\in$ TDL$_{\textrm{B}}$ $\vee$ $\neg$(!p $\in$ TDL$_{\textrm{B}}$) & $\textrm{TDL}_{\textrm{B}}$  \tabularnewline
			\cline{1-1}\cline{3-3}
			{} & {} & !p  \tabularnewline
			\hline
			\multicolumn{3}{|l|}{cg s$_{4}$ = $\lbrace$ q $>$ !p, q, !p $\in$ TDL$_{\textrm{B}}$ $\rbrace$} \tabularnewline
			\hline
			\end{tabular}
	\ex\label{588b} Teil 2\\[-1em]
			\begin{tabular}[t]{|C{6em}|C{12em}|C{6em}|}
			\hline
			$\textrm{DC}_{\textrm{A}}$ & Tisch &  $\textrm{DC}_{\textrm{B}}$ \tabularnewline
			\hline
			{} & p $\vee$ $\neg$p & {}  \tabularnewline
			\cline{1-1}\cline{3-3}
			$\textrm{TDL}_{\textrm{A}}$ & {} & $\textrm{TDL}_{\textrm{B}}$  \tabularnewline
			\cline{1-1}\cline{3-3}
			{} & {} & !p  \tabularnewline
			\hline
			\multicolumn{3}{|l|}{cg s$_{5}$ = s$_{4}$} \tabularnewline
			\hline
			\end{tabular}
\end{xlist}		
\end{exe}
Dass p zu tun ist, um sein Problem zu lösen, sollte B folglich selber wissen, sobald er q äußert. Er müsste dies schließlich selber ableiten können auf der Basis von q $>$ !p und seinem Anliegen q. Mit Ausnahme der Beteiligung einer neuen Diskurskomponente unterscheidet meine \textit{eben}-Modellierung folglich nicht zwi\-schen den beteiligten \is{Illokutionstyp} Illokutionstypen. 

Der entscheidende Unterschied ist, dass im \textit{eben}-Direktiv die Realisierung von !p durch B noch aussteht. Im Falle der \textit{eben}-Assertion gelangt p in den cg.\\

\noindent
Das Pendant zum monologischen Fall der \textit{ebe}n- (und auch \textit{halt}-)Assertion scheint es mir beim Direktiv nicht zu geben. Beispiele wie in (\ref{590}) sind ganz typisch in der Literatur.

\begin{exe}
	\ex\label{590} 
		\begin{xlist}	
			\ex\label{590a} B: Es sind keine sauberen Tassen mehr da.\\
							A: Dann nimm \textbf{eben} ein Glas.
			\ex\label{590b} B: Mein Zahn tut mir weh.\\
							A: Dann geh \textbf{eben} zum Zahnarzt.
			\ex\label{590c} B: Die Wohnzimmertür quietscht schrecklich.\\
							A: Dann öl sie \textbf{eben}!
			\hfill\hbox {\citet[105/101/101]{Dahl1988}}
		\end{xlist}
\end{exe}
Beispiele wie in (\ref{591}) sind denkbar. Die konditionalen Nebensätze nehmen hier ebenfalls Bezug auf ein zuvor vermitteltes (ggf. auch vom Adressaten geäußertes) Verhalten, Geschehen oder auf einen Umstand des Adressaten und werden nicht allein als Annahme des Sprechers gelesen.
\begin{exe}
	\ex\label{591} 
		\begin{xlist}	
			\ex\label{591a} Arbeite \textbf{eben} schneller! (wenn es nicht anders geht)
			\ex\label{591b} Dann steh \textbf{eben} etwas früher auf! (wenn du mehr schaffen willst)
			\ex\label{591c} Bleib \textbf{eben} zu Hause! (wenn du dich nicht wohl fühlst)
			\hfill\hbox {\citet[122]{Helbig1990}}
		\end{xlist}
\end{exe}
Zu diesem Eindruck passen auch die Beschreibungen zu \textit{eben}-Aufforderungen aus \citet[121]{Helbig1990}: \glqq die Handlung (zu der aufgefordert wird) $[$ergibt$]$ sich als (einzig mögliche) Konsequenz aus dem vorhergehenden Geschehen\grqq{}  und \citet[169]{Hentschel1986}: \glqq $[$es wird$]$ eine Handlung gekennzeichnet, die sich als einzig mögliche Konsequenz aus dem vorangegangen verbalen oder situativen Kontext ergibt\grqq{}. Plausiblerweise ist dieses Geschehen, das der \textit{eben}-Äußerung vorweggeht, Teil der Wahrnehmung dessen, an den der Direktiv gerichtet ist. Ich nehme deshalb an, dass q im Kontext vor dem \textit{ebe}n-Direktiv minimal unter Bs Bekenntnissen ist. Ein Szenario, in dem sich q nur unter As Bekenntnissen befindet und A den Direktiv an den Diskurspartner richtet, scheint eher abwegig. Ein Argument für diese Annahme lässt sich auch aus der konkreten Illokution der \textit{eben}-Direktive ableiten: \citet[102]{Dahl1988} und \citet[122]{Thurmair1989} zufolge handelt es sich weniger um Befehle \is{Befehl} denn um (rechthaberische/besserwisserische) \is{Ratschlag} Ratschläge. Ratschlag setzt ein Problem voraus, das auf Seiten des Adressaten vorhanden ist oder als existent vorausgesetzt wird (vgl. z.B. \citealt[186]{Rolf1997}: \glqq  Beim Erteilen eines Ratschlags präsupponiert der Sprecher, daß der Hörer ein technisch-praktisches oder moralisch-praktisches Problem hat $[$...$]$\grqq{} (vgl. ähnlich auch \citealt[409]{Hindelang1978}; \citeyear[59]{Hindelang2010}).

Wieder ist natürlich nicht ausgeschlossen, dass p bereits cg ist (vgl. (\ref{592}) und (\ref{593})).

\begin{exe}
	\ex\label{592}
	B: Es sind \textbf{\textit{ja}} gar keine sauberen Tassen mehr da.\\
	A: Dann nimm \textbf{eben} ein Glas!  
	\hfill\hbox {nach \citet[105]{Dahl1988}}
\end{exe}

\begin{exe}
	\ex\label{593}
	B: \textbf{\textit{Du weißt}}, dass ich heute morgen schon wieder die S-Bahn verpasst habe.\\
	A: Steh \textbf{eben} morgen früher auf!	  
	\hfill\hbox {nach \citet[122]{Thurmair1989}}
\end{exe}			
Die Eindrücke von \textit{Vorwurf}, \textit{Rechthaberei} und \textit{einzig möglicher Problemlösung}, die in deskriptiven Beschreibungen angeführt worden sind (s.o.), fängt man mit dieser Analyse auf, weil die Gesprächsteilnehmer sich einig sind, dass im Falle von q !p anstehen sollte und sie sich darüber hinaus auch hinsichtlich q einig sind/werden.\\
\newline
Wie im deskriptiven Teil angeführt, kann die Folgeinterpretation auch bei einer Assertion auftreten (vgl. (\ref{594})).
\begin{exe}
	\ex\label{594}
	B: Du, das ist ganz blöd heute, ich hab noch so wahnsinnig viel zu tun.
	\newline
	\hbox{}\hfill\hbox{(= q)} \\
	A: Gut, komm ich \textbf{eben} morgen. (= p)
	\hfill\hbox {\citet[121]{Thurmair1989}}
\end{exe}
Vor der \textit{eben}-Äußerung assertiert B q, wodurch sich q $\vee$ $\neg$q auf dem Tisch eröffnet.\footnote{Hier zeigt sich auch, dass es Inhalte gibt, die vermutlich nur schlecht ein Thema auf dem Tisch eröffnen. Bei q handelt es sich schließlich um einen Inhalt, der nur schwer zurückgewiesen werden könnte von A.} Da im cg eine Relation der Art enthalten ist, dass man Besuche verschiebt, wenn der Gastgeber keine Zeit hat, vertritt B auch p (vgl. (\ref{595a})). Im Zuge dessen, dass A mit seiner kommenden Äußerung auf den Umstand q reagiert, akzeptiert er ihn (vgl. (\ref{595b})).

\newcolumntype{C}[1]{>{\centering}p{#1}}
\begin{exe}
	\ex\label{595} Kontext vor einer \textit{eben}-Assertion (Folge)\\[-1.75em]
		\begin{xlist}	
			\ex\label{595a} Teil 1\\[-1em]
			\begin{tabular}[t]{|C{6em}|C{12em}|C{6em}|}
			\hline
			$\textrm{DC}_{\textrm{A}}$ & Tisch &  \textbf{$\textrm{DC}_{\textrm{B}}$} \tabularnewline
			\hline
			{}  & q $\vee$ $\neg$q & \textbf{q}  \tabularnewline
			{} & {} & p \tabularnewline
			\hline
			\multicolumn{3}{|l|}{cg s$_{1}$ = $\lbrace$q $>$ p$\rbrace$}		
			\tabularnewline
			\hline
			\end{tabular}	

			\ex\label{595b} Teil 2\\[-1em]
			\begin{tabular}[t]{|C{6em}|C{12em}|C{6em}|}
			\hline
			$\textrm{DC}_{\textrm{A}}$ & Tisch &  $\textrm{DC}_{\textrm{B}}$ \tabularnewline
			\hline
			q  & q $\vee$ $\neg$q & q  \tabularnewline
			{} & {} & p \tabularnewline
			\hline
			\multicolumn{3}{|l|}{cg $\textrm{s}_{2} = \lbrace \textrm{s}_{1} \cup \lbrace \textrm{q} \rbrace \rbrace$}		
			\tabularnewline
			\hline
			\end{tabular}			
		\end{xlist}
\end{exe}
Die \textit{eben}-Äußerung kann q nun als kategorisch ausgeben, weil A und B p ableiten können, da beide von q $>$ p und von p ausgehen. A legt p folglich auf den Tisch und entfernt das Thema sofort, weil das Hörerbekenntnis vorweggenommen (hier bereits cg) ist (vgl. (\ref{596})).

\newcolumntype{C}[1]{>{\centering}p{#1}}
\begin{exe}
	\ex\label{596} Kontext nach der \textit{eben}-Assertion (Folge)\\[-1.75em]
		\begin{xlist}	
			\ex\label{596a} Teil 1\\[-1em]
			\begin{tabular}[t]{|C{6em}|C{12em}|C{6em}|}
			\hline
			$\textrm{DC}_{\textrm{A}}$ & Tisch &  $\textrm{DC}_{\textrm{B}}$ \tabularnewline
			\hline
			p  & p $\vee$ $\neg$p & p  \tabularnewline
			\hline
			\multicolumn{3}{|l|}{cg s$_{3}$ = $\lbrace$q, q $>$ p, p$\rbrace$}		
			\tabularnewline
			\hline
			\end{tabular}	

			\ex\label{596b} Teil 2\\[-1em]
			\begin{tabular}[t]{|C{6em}|C{12em}|C{6em}|}
			\hline
			$\textrm{DC}_{\textrm{A}}$ & Tisch &  $\textrm{DC}_{\textrm{B}}$ \tabularnewline
			\hline
			{} & {} & {}  \tabularnewline
			\hline
			\multicolumn{3}{|l|}{cg $\textrm{s}_{4} = \textrm{s}_{3}$}		
			\tabularnewline
			\hline
			\end{tabular}			
		\end{xlist}
\end{exe}
Hier führt B q ein und A akzeptiert q. Es wäre genauso denkbar, dass hinsichtlich q bereits Einigkeit besteht. Ein Szenario, in dem A q assertiert und es sich nicht um Inhalt handelt, für den Bs Zustimmung (ggf. stillschweigend) angenommen wird, erscheint mir merkwürdig. Auffallend ist, dass im Falle der assertiven \textit{eben}- (und auch \textit{halt}-)Folge Handlungsvorschläge/-entscheidungen des Sprechers ausgedrückt werden. Es handelt sich somit um sprachliche Handlungen, die in die Nähe von \textit{Promissiv} \is{Promissiva} rücken (vgl. \citealt{Pak2008} zum Koreanischen). Möglicherweise wäre es deshalb plausibler, in diesen Fällen die To-Do-Liste des Sprechers zu aktualisieren und nicht seine Diskursbekenntnisse (vgl. \citealt[5, 11-12]{Portner2004}, \citealt[55]{Beyssade2006}). Es handelt sich bei den nicht-direktiven Folgen durchweg um Vorhaben des Sprechers, von Sprecher und Adressat (ähnlich Hortativen \is{Hortativ} $[$vgl. (\ref{597})$]$) oder auch um modalisierte Handlungsanweisungen an den Adressaten (vgl. (\ref{598})).

\begin{exe}
	\ex\label{597} 
		Macht nichts, (wenn die letzte U-Bahn schon weg ist,) nehmen wir \textbf{eben} ein Taxi.
		\hfill\hbox{\citet[169]{Hentschel1986}}	
\end{exe}

\begin{exe}
	\ex\label{598} 
		(A kommt völlig verkatert an den Frühstückstisch; B sagt zu ihm:\\
		Du darfst \textbf{eben} nicht so viel trinken.
			\hfill\hbox {\citet[287, Fn 55]{Dahl1988}}
\end{exe}			
Die assertiven \textit{eben}/\textit{halt}-Folgen ähneln den direktiven \textit{eben}-/\textit{halt}-Folgen mehr als den assertiven \textit{eben}-/\textit{halt}-Begründungen.	

Wie (\ref{599}), (\ref{600}) und (\ref{601}) im Vergleich zeigen, liegen in allen drei Fällen im Kontext vor der \textit{eben}-Äußerung parallele Verhältnisse vor.

\begin{exe}
	\ex\label{599} Kontext vor der \textit{eben}-Assertion (Begründung) (eben(p))\\[-1em]	
			\begin{tabular}[t]{|C{6em}|C{12em}|C{6em}|}
			\hline
			$\textrm{DC}_{\textrm{A}}$ & Tisch &  $\textrm{DC}_{\textrm{B}}$ \tabularnewline
			\hline
			(q) & {} & (q)  \tabularnewline
			{} & {} & p \tabularnewline
			\hline
			\multicolumn{3}{|l|}{cg s$_{1}$ = $\lbrace$p $>$ q$\rbrace$}		
			\tabularnewline
			\hline
			\end{tabular}	
\end{exe}

\begin{exe}
	\ex\label{600} Kontext vor der \textit{eben}-Assertion (Folge) (eben(p))\\[-1em]	
			\begin{tabular}[t]{|C{6em}|C{12em}|C{6em}|}
			\hline
			$\textrm{DC}_{\textrm{A}}$ & Tisch &  $\textrm{DC}_{\textrm{B}}$ \tabularnewline
			\hline
			{}  & {} & q  \tabularnewline
			{} & {} & p \tabularnewline
			\hline
			\multicolumn{3}{|l|}{cg s$_{1}$ = $\lbrace$q $>$ p$\rbrace$}		
			\tabularnewline
			\hline
			\end{tabular}	
\end{exe}

\newcolumntype{C}[1]{>{\centering}p{#1}}
\begin{exe}
\ex\label{601} Kontext vor einem \textit{eben}-Direktiv (Folge) (eben(p))\\[-1em]
\begin{tabular}[t]{|C{6em}|C{12em}|C{6em}|}
\hline
$\textrm{DC}_{\textrm{A}}$ & Tisch &  $\textrm{DC}_{\textrm{B}}$ \tabularnewline
\hline
{} & {} & q  \tabularnewline
\cline{1-1}\cline{3-3}
$\textrm{TDL}_{\textrm{A}}$ & {} & $\textrm{TDL}_{\textrm{B}}$  \tabularnewline
\cline{1-1}\cline{3-3}
{} & {} & !p  \tabularnewline
\hline
\multicolumn{3}{|l|}{cg s$_{1}$ = $\lbrace$q $>$ !p$\rbrace$} \tabularnewline
\hline
\end{tabular}
\end{exe}
Die Inferenzrelation ist jeweils Teil des cg. Die Proposition der \textit{eben}-Äußerung ist im relevanten System des Hörers bereits vor der MP-Äußerung verankert bzw. !p in DC$_{\textrm{B}}$ $[$(\ref{601})$]$). Die Proposition q der Vorgängeräußerung ist mindestens in DC$_{\textrm{B}}$ enthalten. Die Ausnahme sind monologische Begründungsfälle, in denen q auch allein ein Sprecherbekenntnis sein kann.
	
Parallel zu den \textit{halt}-Begründungen unterscheidet sich auch meine Modellierung der \textit{halt}-Folge (vgl. (\ref{602})) im Diskurs nur leicht von der Charakterisierung der \textit{eben}-Folge.

\begin{exe}
	\ex\label{602} \textit{halt}-Folge\\
	B: Ich schaffe es nicht bis morgen! (= q)\\
	A: Arbeite \textbf{halt} schneller! (= !p)		
\end{exe}
Im Kontext vor der \textit{halt}-Äußerung ist die Inferenzrelation q $>$ !p unter den Diskursbekenntnissen von A. Im konkreten Fall in (\ref{602}) hat B dazu ein Bekenntnis zu q, wodurch sich auf dem Tisch die Frage eröffnet, ob q im Kontext für die Beteiligten Gültigkeit hat. Widerspricht A nicht, gilt dies als Bestätigung von q und q wird somit in den cg aufgenommen. D.h. A und B sind sich einig darüber, dass es sich bei q um ein Problem von B handelt.
\pagebreak
\newcolumntype{C}[1]{>{\centering}p{#1}}
\begin{exe}
\ex\label{603} Kontext vor dem \textit{halt}-Direktiv (Folge)\\[-1.75em]
\begin{xlist}
\ex\label{603a} Teil 1\\[-1em]
\begin{tabular}[t]{|C{6em}|C{12em}|C{6em}|}
\hline
\textbf{$\textrm{DC}_{\textrm{A}}$} & Tisch &  \textbf{$\textrm{DC}_{\textrm{B}}$} \tabularnewline
\hline
\textbf{q $>$ !p} & q $\vee$ $\neg$q & \textbf{q}  \tabularnewline
\cline{1-1}\cline{3-3}
$\textrm{TDL}_{\textrm{A}}$ & {} & $\textrm{TDL}_{\textrm{B}}$  \tabularnewline
\cline{1-1}\cline{3-3}
{} & {} & {}  \tabularnewline
\hline
\multicolumn{3}{|l|}{cg s$_{1}$} \tabularnewline
\hline
\end{tabular}

\ex\label{603b} Teil 2\\[-1em]
\begin{tabular}[t]{|C{6em}|C{12em}|C{6em}|}
\hline
$\textrm{DC}_{\textrm{A}}$ & Tisch &  $\textrm{DC}_{\textrm{B}}$ \tabularnewline
\hline
q $>$ !p &  &   \tabularnewline
\cline{1-1}\cline{3-3}
$\textrm{TDL}_{\textrm{A}}$ & {} & $\textrm{TDL}_{\textrm{B}}$  \tabularnewline
\cline{1-1}\cline{3-3}
{} & {} & {}  \tabularnewline
\hline
\multicolumn{3}{|l|}{cg s$_{2}$ = $\lbrace$s$_{1}$ $\cup$ $\lbrace$q$\rbrace\rbrace$} \tabularnewline
\hline
\end{tabular}
\end{xlist}
\end{exe}
Im nächsten Schritt tätigt A den Direktiv !p und sofern B die Handlungsanweisung nicht zurückweist (vgl. (\ref{604a}) und (\ref{604b}), gelangt diese zu realisierende Proposition auf die TDL von B. Auf dem Tisch eröffnet sich somit die Frage, ob p gilt (vgl. (\ref{604c})).

\newcolumntype{C}[1]{>{\centering}p{#1}}
\begin{exe}
\ex\label{604} Kontext nach dem \textit{halt}-Direktiv (Folge)\\[-1.75em]
\begin{xlist}
\ex\label{604a} Teil 1\\[-1em]
\begin{tabular}[t]{|C{6em}|C{12em}|C{6em}|}
\hline
$\textrm{DC}_{\textrm{A}}$ & Tisch &  $\textrm{DC}_{\textrm{B}}$ \tabularnewline
\hline
q $>$ !p\\!p $\in$ TDL$_{\textrm{B}}$ & p $\vee$ $\neg$p\\!p $\in$ TDL$_{\textrm{B}}$ $\vee$ $\neg$(!p $\in$ TDL$_{\textrm{B}}$) & {}  \tabularnewline
\cline{1-1}\cline{3-3}
$\textrm{TDL}_{\textrm{A}}$ & {} & $\textrm{TDL}_{\textrm{B}}$  \tabularnewline
\cline{1-1}\cline{3-3}
{} & {} & {}  \tabularnewline
\hline
\multicolumn{3}{|l|}{cg s$_{3}$ = s$_{2}$} \tabularnewline
\hline
\end{tabular}

\ex\label{604b} Teil 2\\[-1em]
\begin{tabular}[t]{|C{6em}|C{12em}|C{6em}|}
\hline
$\textrm{DC}_{\textrm{A}}$ & Tisch &  $\textrm{DC}_{\textrm{B}}$ \tabularnewline
\hline
q $>$ !p\\!p $\in$ TDL$_{\textrm{B}}$ & p $\vee$ $\neg$p\\!p $\in$ TDL$_{\textrm{B}}$ $\vee$ $\neg$(!p $\in$ TDL$_{\textrm{B}}$) & {}\\!p $\in$ TDL$_{\textrm{B}}$  \tabularnewline
\cline{1-1}\cline{3-3}
$\textrm{TDL}_{\textrm{A}}$ & {} & $\textrm{TDL}_{\textrm{B}}$  \tabularnewline
\cline{1-1}\cline{3-3}
{} & {} & {!p}  \tabularnewline
\hline
\multicolumn{3}{|l|}{cg s$_{4}$ = s$_{3}$} \tabularnewline
\hline
\end{tabular}

\ex\label{604c} Teil 3\\[-1em]
\begin{tabular}[t]{|C{6em}|C{12em}|C{6em}|}
\hline
$\textrm{DC}_{\textrm{A}}$ & Tisch &  $\textrm{DC}_{\textrm{B}}$ \tabularnewline
\hline
q $>$ !p & p $\vee$ $\neg$p & {} \tabularnewline
\cline{1-1}\cline{3-3}
$\textrm{TDL}_{\textrm{A}}$ & {} & $\textrm{TDL}_{\textrm{B}}$  \tabularnewline
\cline{1-1}\cline{3-3}
{} & {} & {!p}  \tabularnewline
\hline
\multicolumn{3}{|l|}{cg s$_{5}$ = $\lbrace$s$_{4}$ $\cup$ $\lbrace$!p $\in$ TDL$_{\textrm{B}}\rbrace\rbrace$} \tabularnewline
\hline
\end{tabular}
\end{xlist}
\end{exe}
Unter Bezug auf (\ref{602}) bis (\ref{604}) ist es möglich, die Interpretationen zu erfassen, dass A eine \underline{mögliche} Lösung des Problems von B anbietet. Es handelt sich hierbei um eine Lösung, die A für plausibel hält, da A sowohl q $>$ !p als auch q annimmt. Es handelt sich bei !p aber (anders als im Fall von \textit{eben}) nicht um die einzige Lösung, weil q $>$ !p nicht im cg enthalten ist, d.h. B muss unter der Annahme von q nicht von sich aus auf !p kommen.

Aufgrund der gleichen Argumentation wie für \textit{eben}-Direktive scheint mir q stets mindestens unter den Diskursbekenntnissen von B zu sein. Die durch den Sprecher angebotene Lösung ist in seinen Augen plausibel, das entworfene Problem ist aber mindestens eines vom Adressaten als solches ausgegebenes.\\

\noindent
Ein Punkt, in dem sich die Beiträge von \textit{halt} und \textit{eben} nach meiner Modellierung voneinander unterscheiden, ist, dass sie den Eindruck widerspiegelt, dass Sprecher und Hörer sich im Falle von \textit{halt} hinsichtlich der Relation nicht einig sein müssen. Gute Evidenz für diesen Unterschied wären deshalb Kontexte, aus denen hervorgeht, dass die Diskurspartner tatsächlich entgegengesetzte Ansichten hinsichtlich dieser Relation vertreten. Für (\ref{605}) bietet sich eine solche Interpretation an.

\begin{exe}
	\ex\label{605} Monika will Hans um einen Gefallen bitten (= q), zögert aber, ihn anzurufen. Nach einiger Zeit sagt ihre 			Freundin
		\begin{xlist}	
			\ex\label{605a} Jetzt ruf den Hans \textbf{halt} an! (= !p)
			\ex\label{605b} \#Jetzt ruf den Hans \textbf{eben} an!
		\end{xlist}
\end{exe}
Wenngleich in diesem Beispiel Monika nicht explizit das Gegenteil hinsichtlich des Zusammenhangs \glq Wenn du jemanden um einen Gefallen bitten willst, ruf ihn an!\grq {} behauptet, ist aus der Kontextangabe, dass sie (anders als der Sprecher der \textit{halt}-Äußerung) zögert, Hans anzurufen, abzuleiten, dass sie gerade nicht für diese Relation einsteht. Andernfalls würde sie dieser Handlung schließlich nachkommen, um ihr Problem zu lösen. Da aus der Kontextinformation klar ist, dass Monika von q ausgeht, bietet sich als Erklärung der Differenz zwi\-schen (a) und (b) nicht die Ableitung an, die ich für Begründungsfälle wie in (\ref{606}) vorgeschlagen habe.

\begin{exe}
	\ex\label{606} 
		\begin{xlist}	
			\ex\label{606a} Wir haben \textbf{halt} kein Bier mehr.
			\ex\label{606b} \#Wir haben \textbf{eben} kein Bier mehr.
		\end{xlist}
\end{exe}
Dort habe ich angenommen, dass zwar Einigkeit hinsichtlich des Zusammenhangs der zwei beteiligten Propositionen bestehen kann (p $>$ q), dann jedoch nicht hinsichtlich p, was wiederum verhindert, dass der Schluss auf q für beide Partner zwingend wird. In (\ref{605}) vertritt B aber q, d.h. wenn A und B sich auch hinsichtlich q $>$ !p einig wären, müsste B !p ableiten können. A müsste folglich einen \textit{eben}-Direktiv, der genau darauf verweist, adäquat äußern können. Dies ist nicht der Fall und leicht zu erfassen, wenn q $>$ !p nicht cg sein kann, weil die Relation nur von A vertreten wird, der (anders als B) !p für einen angemessenen Vorschlag hält, um das Problem zu lösen.\\
\newline
Die Akzeptabilität der \textit{halt}-Äußerung und die Inakzeptabilität der \textit{eben}-Äußerung lassen sich unter Bezug auf die oben vorgeschlagene Charakterisierung der Kontextzustände in Dialogen wie in (\ref{605}) und (\ref{606}) folglich erklären. Gleiches gilt m.E. auch für die sehr seltenen Fälle, bei denen gegenüber (\ref{605}) und (\ref{606}) umgekehrte Verhältnisse vorliegen. Hierbei handelt es sich um Kontexte, in denen \textit{halt} im Gegensatz zu \textit{eben} unangemessen scheint (vgl. z.B. (\ref{607})).
	
\begin{exe}
	\ex\label{607} Du kannst deine Freunde schon mitbringen.
		\begin{xlist}	
			\ex\label{607a} Der Wal ist \textbf{eben} ein Säugetier.
			\ex\label{607b} ?Der Wal ist \textbf{halt} ein Säugetier.
		\end{xlist}
\end{exe}	
In Beispielen dieser Art ist der ausgedrückte Sachverhalt derart eindeutig, dass man ihn als Teil des cg ausgeben \underline{muss} und seine Assertion (und damit erstmalige Einführung in den Kontext) als eine zu schwache Behandlung im Diskurs anzunehmen ist. Da kein weiterer Kontext gegeben ist, lässt sich über die Relation, an der p teilhat, keine Aussage machen. In (\ref{608}) ist sie hingegen erkenntlich. Hier nehme ich an, dass \textit{halt} weniger angemessen ist als \textit{eben}, weil die Relation zwischen Goretex tragen und trocken bleiben derart bekannt ist, dass sie cg sein muss und es nicht ausreicht, diesen Zusammenhang allein als Sprecherannahme zu präsentieren.

\begin{exe}
	\ex\label{608} 
		Tim: Mensch, du bist ja ganz trocken! (= q)\\
		Hans: Das ist \textbf{eben} Goretex. vs. (?) Das ist \textbf{halt} Goretex. (= p)
		\newline
		\hbox{}\hfill\hbox{\citet[125]{Thurmair1989}}	
\end{exe}
Die Proposition p selbst (= das ist Goretex) scheint mir akkommodiert \is{Akkommodation} zu werden (d.h. ist vor der \textit{eben}-Äußerung in DC$_{\textrm{B}}$ verankert), da zumindest aus diesem Kontext heraus nicht ableitbar ist, dass der Hörer um die Beschaffenheit des Stoffes wissen muss.

Die Verhältnisse bei der assertiven \textit{halt}-Folge entsprechen denen unter Auftre\-ten von \textit{eben} in diesem Kontext. Der einzige Unterschied ist – wie bei den \textit{halt}- und \textit{eben}-Direktiven –, dass die Relation nur unter den Sprecherbekenntnissen ist. (\ref{609}) zeigt ein Beispiel.

\begin{exe}
	\ex\label{609} 
	\scriptsize
	Der Reim hätte gern den Mark angestellt aber auf ne (...) richtige Stelle und da hat sich . die Personalstelle 						quergelegt (...) Ja, und dann hab ich gesagt, dann geb ich \textbf{halt} dem Mark, bis die nächste Viertelstelle, die 				wär im 	März freigeworden, (...) Geb ich dem solang meine Viertelstelle. (BA, 64)
	\newline
	\hbox{}\hfill\hbox{\citet[125]{Thurmair1989}}
\end{exe}
Die \textit{halt}-Äußerung ist hier Teil eines Berichts. In der ursprünglichen Situation ist q (= Mark hat keine richtige Stelle) plausiblerweise im cg, d.h. die Beteiligten wissen um diesen Sachverhalt. Der Sprecher vertritt q $>$ p (mit p = ich gebe Mark meine Viertelstelle). Bei diesem Zusammenhang handelt es sich vermutlich nicht um eine etablierte Bedingungs-Folge-Relation, die prinzipiell oder am beteiligten Lehrstuhl als gesetzt anzusehen ist. Es ist vielmehr ein Zusammenhang, den der Sprecher individuell für sich sieht und sich deshalb auf der Basis von q für p entscheidet (vgl. (\ref{610})).\footnote{Auch bei p hat man es mit einem Inhalt zu tun, für den vielleicht gar nicht angenommen werden sollte, dass er zur Diskussion gestellt wird.}
\newcolumntype{C}[1]{>{\centering}p{#1}}
\begin{exe}
	\ex\label{610} Kontext nach der \textit{halt}-Assertion (Folge) (halt(p))\\[-1em]
 		\begin{tabular}[t]{|C{6em}|C{6em}|C{6em}|} 
 		\hline 	
   		$\textrm{DC}_{\textrm{A}}$ & {Tisch} & $\textrm{DC}_{\textrm{B}}$ \tabularnewline
  		\hline
   		{q $>$ p} & {} & {} \tabularnewline
   		p &  p $\vee$ $\neg$p & {} \tabularnewline
  		\hline      
   		\multicolumn{3}{|l|}{cg s$_{2}$ = $\lbrace$q $\rbrace$} \tabularnewline   
  		 \hline
 		\end{tabular}
\end{exe}
In (\ref{609}) befindet sich p zum Zeitpunkt der \textit{halt}-Äußerung bereits im cg. In (\ref{611}) führt der Sprecher der späteren \textit{halt}-Äußerung q selbst zunächst ein (die Bieringer kamen). 
			
\begin{exe}
	\ex\label{611} 
	\scriptsize
	Und dann kamen auch noch die Bieringer, plötzlich, vier Mann hoch. Da hab ich \textbf{halt} schnell noch einen Topf Nudeln 			gekocht und das Fleisch kleiner geschnitten. [...] 	
	\hfill\hbox {\citet[235]{Franck1980}}
\end{exe}			
Es ist jedoch nicht davon auszugehen, dass der Gesprächspartner q nicht zustimmen würde. Ich gehe deshalb auch für die assertive \textit{halt}-Folge davon aus, dass q mindestens in DC$_{\textrm{B}}$ enthalten ist.

Die erforderlichen Vorkontextzustände für eine sich anschließende \textit{halt}-Äuße\-rung unterscheiden sich zwischen Assertion und Direktiv bzw. Folge und Begründung somit unwesentlich (vgl. (\ref{612}), (\ref{613}) und (\ref{614})).

\newcolumntype{C}[1]{>{\centering}p{#1}}
\begin{exe}
	\ex\label{612} Kontext vor der \textit{halt}-Assertion (Begründung) (halt(p))\\[-1em]
 		\begin{tabular}[t]{|C{6em}|C{6em}|C{6em}|} 
 		\hline 	
   		$\textrm{DC}_{\textrm{A}}$ & {Tisch} & $\textrm{DC}_{\textrm{B}}$ \tabularnewline
  		\hline
  		(q) & {} & (q) \tabularnewline
   		{p $>$ q} & {} & {} \tabularnewline
  		\hline      
   		\multicolumn{3}{|l|}{cg s$_{1}$} \tabularnewline   
  		 \hline
 		\end{tabular}
\end{exe}

\newcolumntype{C}[1]{>{\centering}p{#1}}
\begin{exe}
	\ex\label{613} Kontext vor der \textit{halt}-Assertion (Folge) (halt(p))\\[-1em]
 		\begin{tabular}[t]{|C{6em}|C{6em}|C{6em}|} 
 		\hline 	
   		$\textrm{DC}_{\textrm{A}}$ & {Tisch} & $\textrm{DC}_{\textrm{B}}$ \tabularnewline
  		\hline
   		{q $>$ p} & {} & q \tabularnewline
  		\hline      
   		\multicolumn{3}{|l|}{cg s$_{1}$} \tabularnewline   
  		 \hline
 		\end{tabular}
\end{exe}

\newcolumntype{C}[1]{>{\centering}p{#1}}
\begin{exe}
\ex\label{614} Kontext vor einem \textit{halt}-Direktiv (Folge) (halt(p))\\[-1em]
\begin{tabular}[t]{|C{6em}|C{12em}|C{6em}|}
\hline
$\textrm{DC}_{\textrm{A}}$ & Tisch &  $\textrm{DC}_{\textrm{B}}$ \tabularnewline
\hline
q $>$ !p & {} & q  \tabularnewline
\cline{1-1}\cline{3-3}
$\textrm{TDL}_{\textrm{A}}$ & {} & $\textrm{TDL}_{\textrm{B}}$  \tabularnewline
\cline{1-1}\cline{3-3}
{} & {} & {}  \tabularnewline
\hline
\multicolumn{3}{|l|}{cg s$_{1}$} \tabularnewline
\hline
\end{tabular}
\end{exe}
Die Unterschiede, die sich einstellen, sind nicht auf verschiedene \textit{halt}-Beiträge, sondern auf die bei Begründung und Folge bzw. Assertion und Direktiv beteiligten Konstellationen zurückzuführen.

\section{Empirische Fragen zur Kombination von \textit{halt} und \textit{eben}}
\label{sec:empirie}
Für die Sequenzen aus diesen Einzelpartikeln gilt, dass prinzipiell beide denkbaren Kombinationen (\textit{eben halt} und \textit{halt eben}) mit Korpora einfach zu belegen sind (vgl. (\ref{615}) bis (\ref{618})).

\begin{exe}
	\ex\label{615} 
	\scriptsize
	2:1 vorne gegen neun Mann und einen Stürmer Jan Koller als Torhüter, zog es der Branchenführer vor, den Ball in den 				eigenen Reihen zu halten, statt für klare Verhältnisse zu sorgen. \textbf{In der Krise geht Vorsicht \underline{halt 				eben} über alles.}
	\hfill\hbox{(RHZ02/NOV.07350 Rhein-Zeitung, 11.11.2002)}
\end{exe}

\begin{exe}
	\ex\label{616} 
	\scriptsize
	\textbf{Wir kamen \underline{halt eben} eine Viertelstunde zu spät}, und wie wir reinkommen in die
 	Sch/ Klasse, da hat wohl schon der Lehrer die ersten Bänke das Pult zugemacht $[$...$]$. 	
 	\hfill\hbox{(OS$\minus\minus$\_E\_00067\_SE\_01\_T\_01)}
\end{exe}
	
\begin{exe}
	\ex\label{617} 
	\scriptsize
	Vorher ließ sich das Portal weder von meinem Rechner zu Hause aus, noch von meinem Rechner auf Arbeit öffnen. 						\textbf{So reibungslos wie dieses Forum funktioniert es \underline{eben halt} nicht.}
	\newline
	\hbox{}\hfill\hbox{(DECOW2012$-$00: 4616684)}
\end{exe}
											    		                			      			       
\begin{exe}
	\ex\label{618}
	\scriptsize 
	Ich bin entlassen worden, nicht? Und, äh, \textbf{da hat man \underline{eben halt} geguckt, wie man sich einmal 					vorläufig durchschlägt.}    
	\hfill\hbox{(ZW$\minus\minus$\_E\_00587\_SE\_01\_T\_01)}                         	
\end{exe}	
Auch in der Literatur finden sich zwar Beispiele für beide Abfolgen, es herrscht aber keine Einigkeit, wie mit der Varianz der Reihung umzugehen ist. Vielmehr ergeben sich aus den (wie im Folgenden gezeigt werden wird) sich widersprechenden Annahmen diverse empirische Fragen, wie z.B.: Wird eine (und wenn ja welche) Abfolge bevorzugt? Wie kommt die Varianz überhaupt zustande (d.h. gibt es z.B. regionale Gründe oder spielt der Satzmodus eine Rolle)? (vgl. auch \citealt[144-165]{Mueller2016b}; \citeyear[233-238]{Mueller2017a}).

\subsection{Annahmen in der Literatur}
\label{sec:anlit}
In einigen Arbeiten liest man, der Satzmodus \is{Satzmodus} sei ausschlaggebend für das Auftreten der ein oder anderen Abfolge. In \citet[1542-1543]{Zifonun1997} heißt es z.B., die Partikelfolge im \glqq Aussagemodus\grqq{} \is{Aussagemodus} sei \textit{eben halt}, im \is{Aufforderungsmodus}  \glqq Aufforderungsmodus\grqq{} \textit{halt eben}. An anderer Stelle liest man hingegen, sowohl im Aussage- als auch im Aufforderungsmodus sei die Reihung \textit{halt eben} (vgl. \citealt[908-909]{Zifonun1997}). \citet[227/230/234]{Dahl1988} wiederum geht davon aus, dass in \glqq Konstativsätzen\grqq{} \textit{halt eben} uneingeschränkt zulässig sei (\textit{eben halt} erhält ein Fragezeichen), während in \glqq Imperativsätzen\grqq{} die Wahl zwischen \textit{halt eben} und \textit{eben halt} bestehe. Er schreibt dazu: \glqq erstaunlicherweise sind \textit{eben halt} in Imperativsätzen eher akzeptabel als in Konstativsätzen\grqq{} (\citealt[250]{Dahl1988}). Diese wenigen miteinander inkompatiblen Aussagen lassen den Leser verwundert zurück und es ergibt sich die zunächst einmal empirisch zu klärende Frage, ob das Auftreten der Ordnungen \textit{halt eben} und \textit{eben halt} durch den je\-weiligen Satzmodus bzw. Illokutionstyp bedingt ist. Die Klärung dieser Frage ist im Rahmen dieser Arbeit ein zentraler Punkt. Ich beabsichtige eine Ableitung der Abfolgebeschränkungen von MPn basierend auf dem diskurssemantischen Beitrag der MP-Äußerung. Sollte es sich ergeben, dass der Illokutionstyp Einfluss auf die Sequenzierung von \textit{halt} und \textit{eben} nimmt, wäre dies ein Aspekt, den es in der Analyse zu berücksichtigen gilt.

Es gibt wenige Aussagen über die beiden Abfolgen \textit{halt eben} und \textit{eben halt} auf der Basis authentischer Daten. \citet[257, Fn 32]{Thurmair1989} hat in ihrem Korpus nur Treffer für \textit{halt eben}. Die Reihung \textit{eben halt} ist ihr zufolge in diesem Korpus nicht belegt und ihr auch nicht geläufig. Sie verweist auf \textit{eben halt}-Treffer bei \citet[256]{Hentschel1986}, die die Autorin in einem Diskurs vermehrt von einem Sprecher gehört hat. In \citet[78]{Hartog1982} finden sich zwei authenti\-sche Belege für \textit{eben halt} und auch \citet[310]{Rost-Roth1998} führt ein solches Beispiel an. \citet[221]{Dittmar2000} findet in seinem Korpus (vgl. Abschnitt~\ref{sec:regio}) überwiegend \textit{eben halt}. In den Interviews der 31 Ostberliner fallen 13 x \textit{eben halt} und kein \textit{halt eben}. Die 25 Westberliner verwenden 17 x \textit{eben halt} und 8 x \textit{halt eben}. \citet[468]{Braber2010} finden in dem von ihnen durchsuchten Korpus 23 Tref\-fer für \textit{eben halt} und einen für \textit{halt eben}. Es handelt sich bei \textit{eben halt} in ihrer Untersuchung um die zweit häufigst auftretende Zweierkombination (23 aus 286 Kombinationen). Bei den von Braber \& McLelland untersuchten Daten handelt es sich um die gleichen Daten wie bei Dittmar.\footnote{ Mir ist nicht ganz klar, wieso beide Studien nicht zu identischen Frequenzangaben gelangen. Ich gehe davon aus, dass \citet{Braber2010} nur ein Teil der ursprünglichen Daten von \citet{Dittmar2000} zur Verfügung stand.} Das Ergebnis verwundert demnach nicht. 

Auch diese Ergebnisse veranlassen nicht zu einer Generalisierung. Die Tatsache, dass \citet{Thurmair1989}, die nur \textit{halt eben} findet, Korpora des Bairischen benutzt und \citet[224]{Dittmar2000} (und somit auch \citealt{Braber2010}) in seinen Berliner Interviews überwiegend \textit{eben halt} nachweist, veranlasst \citet[17, Fn 41]{Elspass2005} zu der Aussage, diese Verhältnisse sprächen dafür, dass man es mit einem regionalen Unterschied zu tun habe. Er erläutert nicht näher, wie der Zusammenhang zwischen der (einmal vorherrschenden) Verteilung der Einzelpartikeln und der (seiner Ansicht nach) regionalen Verteilung der Kombination aussehen soll. Möchte man dieser Ansicht folgen, scheint es so zu sein, dass \textit{halt eben} da zu finden ist, wo einmal \textit{halt} überwog (Süddeutschland). Und \textit{eben halt} ist dort zu finden, wo einmal vorwiegend \textit{eben} bekannt war (Ostdeutschland). Plausiblerweise wäre Letzteres dann auch für den Norden Deutschlands anzunehmen.

Wenngleich die Annahme regionaler Unterschiede eine naheliegende Erklärung für die unterschiedlichen Korpusrechercheergebnisse zu sein scheint, möchte ich in Bezug auf die Ergebnisse aus \citet{Dittmar2000} einige Bedenken äußern. Zum einen ist der Unterschied von 13 (\textit{eben halt}) : 0 (\textit{halt eben}) (31 Ostberliner) und 17 (\textit{eben halt}) : 8 (\textit{halt eben}) (25 Westberliner) nur im ersten Fall statistisch signifikant. Die Abfolgen \textit{halt eben} und \textit{eben halt} treten nur bei den Ostberlinern überzufällig häufig auf ($\chi^{2}$(1, n = 13) = 11,077, p $<$ 0,001, V = 0,92). Der Unterschied von 17:8 ist nicht statistisch signifikant ($\chi^{2}$(1, n = 25) = 2,56, p = 0,1096). Meine späteren Korpusuntersuchungen werden zeigen, dass in vielen Fällen bei diesem Phänomen eher mit Tendenzen und höchstens kleineren statistischen Effekten zu rechnen ist. Dies ist deshalb nicht mein Hauptkritikpunkt. Zum anderen verteilen sich die auftretenden Kombinationen allerdings auch auf sehr wenige Sprecher. Der Übersicht auf Seite 221 lässt sich entnehmen, dass von den 31 Ost\-berlinern nur drei Sprecher die Kombination \textit{eben halt} verwenden, wobei von den 13 Treffern 11 auf einen dieser drei Sprecher zurückgehen (11–1–1). Bei den 25 Westberlinern verteilen sich die 17 \textit{eben halt}-Treffer auf nur zwei Sprecher, von denen einer 15x die Kombination verwendet. Die \textit{halt eben}-Belege gehen auch nur auf zwei Sprecher zurück (2–6). Man sollte mit Annahmen wie von \citet{Elspass2005} auf dieser Datenbasis folglich vorsichtig sein, da es genau genommen nur sieben Sprecher unter den 46 interviewten Berlinern gibt, die überhaupt eine Kombination aus \textit{halt} und \textit{eben} verwenden. Nur zwei dieser Sprecher benutzen die Sequenzen dazu mit einer gewissen Häufigkeit. Aus dieser Datenlage auf regionale Unterschiede zu schließen halte ich für etwas gewagt. Ich werde diese prinzipielle Überlegung allerdings dennoch am Rande in meine eigenen Untersuchungen bzw. ihre Auswertung einfließen lassen (vgl. Abschnitt~\ref{sec:spu}).

Eine letzte Annahme, die in der Literatur gemacht wird, ist, MP-Kombinationen als Fehlleistungen einzustufen. \citet[96]{Autenrieth2002} vertritt die Ansicht, keines der Beispiele von Thurmair sei \glqq wirklich\grqq{} akzeptabel . Die Autorin schreibt: \glqq Vielmehr scheint es sich um für die gesprochene Sprache typische \glq Fehlleistungen\grq {} zu handeln, die auf mangelnde Planung der Rede von Seiten des Sprechers zurückzuführen sind.\grqq{} Für diese Annahme spreche auch gerade, dass sich beide Abfolgen finden lassen, weil bei \glqq regelhaften MP-Kombinationen [...] die Reihenfolge [...] stets dieselbe\grqq{} sei.

Dieser Blick auf Annahmen, die in anderen Arbeiten zu Kombinationen aus \textit{halt} und \textit{eben} gemacht werden, zeigt einerseits, dass sich diese Ansichten widersprechen (spezifische Satzmodus-/Illokutionstypenverteilung, Relevanz der Be\-lege vs. Fehlleistung) bzw. andererseits, dass eine haltbare empirische Generalisierung zum Verhältnis von \textit{halt eben} und \textit{eben halt} bisher nicht vorliegt (Korpusrecher\-chen mit entgegengesetzten Ergebnissen). 

Um diese Fragen einer Lösung näher zu bringen, habe ich sowohl Korpusrecherchen durchgeführt (vgl. Abschnitt~\ref{sec:häufko}) als auch Sprecherurteile erhoben (vgl. die Abschnitt~\ref{sec:spu} und \ref{sec:stressclash}). Das Ziel dieser Untersuchungen soll eine empirisch valide Generalisierung über die beiden Kombinationen aus \textit{halt} und \textit{eben} sowie ihr Verhältnis zueinander sein, auf der Basis derer dann eine grammatiktheoretische Ableitung entwickelt werden kann.

\subsection{Häufigkeiten in Korpora}
\label{sec:häufko}
Um die auf der Basis bestehender Ansätze nicht zu beantwortenden Fragen anzugehen, habe ich anhand der größten mir zugänglichen Korpora das Verhältnis von \textit{halt eben}- und \textit{eben halt}-Äußerungen bestimmt, mit dem Ziel, zu sehen, ob die beiden Kombinationen gleichermaßen verwendet werden oder ob ein Häufigkeitsgefälle zu beobachten ist (vgl. auch \citealt[148-155]{Mueller2016b}; \citeyear[233-235]{Mueller2017a}). Im Einzelnen handelt es sich um das DeReKo, die DGD2 sowie das DECOW2012. Das DeReKo ist das größte zugängliche Korpus geschriebener Daten des Deutschen, DGD2 enthält die größte Menge gesprochener Daten, DECOW2012 ist ein Web\-korpus, das mit 8 Mio. Tokens die größte Datenmenge aufweist, die mir zum Zeitpunkt meiner Untersuchung zugänglich ist. DeReKo und DGD2 erfassen hier\-bei traditionellere Daten, während DECOW mit Webdaten zusätzlich einen neueren Datentyp in die Betrachtung einbezieht. Man sollte an dieser Stelle allerdings auch bedenken, dass gerade das DeReKo zwar medial schriftlich ist, die Daten konzeptionell aber auch durchaus mündlich einzustufen sind (z.B. Protokolle von Parlamentsreden, Zitate aus Interviews, Diskussionen aus Wikipedia). Nach Sichtung aller Belege und Aussortierung aller Fälle, in denen \textit{halt} und \textit{eben} nicht als MPn auftreten, liegen in den drei Korpora die Verteilungen in (\ref{619}) vor.

\begin{exe}
	\ex\label{619} Häufigkeiten \textit{halt eben}/\textit{eben halt} in allen Korpora\\[-1em]
     \begin{tabular}[t]{|l|l|l|}
     \hline
     & \textit{halt eben} & \textit{eben halt}\\
     \hline
     DeReKo & 715 & 117\\
     \hline
     DGD2 & 63 & 10\\
     \hline
     DECOW2012 & 7328 & 2291\\
     \hline
     \end{tabular}
\end{exe}
Der Unterschied zwischen den Häufigkeiten der beiden Abfolgen ist innerhalb der Korpora jeweils hochsignifikant (p $<$ 0,001) und es liegen in allen drei Fällen starke Effekte vor.\footnote{DeReKo: $\chi^2$(1, n = 832) = 429,81, p $<$ 0,001, V = 0,719\\
DGD2: $\chi^2$(1, n = 73) = 38,48, p $<$ 0,001, V = 0,726\\
DECOW2012: $\chi^2$(1, n = 9619) = 2637,63, p $<$ 0,001, V = 0,524} In allen drei Korpora findet man folglich klare Häufigkeits\-unterschiede: \textit{halt eben} ist jeweils deutlich häufiger belegt als \textit{eben halt}. Für die Daten, für die Informationen über die Herkunft der Sprecher verfügbar sind (dies betrifft Teile der gesprochenen Daten) habe ich entsprechende Aufschlüsselungen vorgenommen. Genauso habe ich die Zeitungen den deutschsprachigen Ländern zugeordnet (Deutschland, Österreich, Schweiz) und innerhalb Deutschlands Untergruppen mit Nord-/Ost-/Süd- und Westdeutschland eröffnet. Die Ergebnisse der Häufigkeitsunterschiede liefern mir keinen Grund, anzunehmen, dass es regionale Unterschiede in der Verteilung gibt. Die Tabelle in (\ref{620}) zeigt hier die Frequenzen für die Zeitungen, sortiert nach Ländern. In den Fällen, in denen genügend Daten vorliegen, dass statistische Auswertungen möglich sind (\glqq -\grqq{} zeigt hier an, wenn dies nicht gegeben ist), sind die Unterschiede (mit einer Ausnahme im Falle der \textit{Nürnberger Nachrichten}, in der aber auch sehr wenige Daten vorliegen) signifikant und die Effektstärken hoch.
\pagebreak
\begin{exe}
	\ex\label{620} Häufigkeiten \textit{halt eben} und \textit{eben halt} in DeReKo (Zeitungen)\\[-0.5em]
	\scriptsize
     \begin{tabular}[t]{|l|l|l|l|}
     \hline
	 Zeitungen & \textit{halt eben} & \textit{eben halt} &  \\
	 \hline
	 St. Galler Tagblatt & 112 & 5 & $\chi^2$(1, n = 117) = 97,855, p $<$ 0,001, V = 0,915\\
	 \hline
	 Die Südostschweiz & 49 &3 & $\chi^2$(1, n = 52) = 40,692, p $<$ 0,001, V = 0,885\\
	 \hline
	 Zürcher Tageszeitung & 20 & 2 & $\chi^2$(1, n = 22) = 13,136, p $<$ 0,001, V = 0,773\\
	 \hdashline
	 \textbf{Gesamt Schweiz} & \textbf{181} & \textbf{1}0 & $\chi^2$(1, n = 191) = 153,094, p $<$ 0,001, V = 0,895\\
	 \hline\hline
	 Vorarlberger Nachrichten & 14 & 1 & $\chi^2$(1, n = 15) = 9,6, p $<$ 0,01, V = 0,8\\
	 \hline
	 Kleine Zeitung & 1 & 1 & -\\
	 \hline
	 Die Presse & 6 & 1 & -\\
	 \hline
	 Tiroler Tageszeitung & 5 & 1 & -\\
	 \hline
	 Salzburger Nachrichten & 4 & - & - \\
	 \hline
	 Neue Kronen-Zeitung & 8 & - & -\\
	 \hline
	 Niederösterreichische Nachrichten & 6 & 1 & -\\
	 \hdashline
	 \textbf{Gesamt Österreich} & \textbf{44} & \textbf{5} & $\chi^2$(1, n = 49) = 31,041, p $<$ 0,001, V = 0,796\\
	 \hline\hline
	 Nürnberger Nachrichten & 8 & 5 & ns\\
	 \hline
	 Nürnberger Zeitung & 7 & - & -\\
	 \hline
	 Mannheimer Morgen & 17 & 2 & $\chi^2$(1, n = 19) = 10,316, p $<$ 0,01, V = 0,737\\
	 \hline
	 Frankfurter Rundschau & 6 & 1 & -\\
	 \hline
	 Rhein-Zeitung & 195 & 10 & $\chi^2$(1, n = 205) = 166,95, p $<$ 0,001, V = 0,902\\
	 \hline
	 Hannoversche Allgemeine & 1 & - & -\\
	 \hline
	 Braunschweiger Zeitung & 4 & 6 & -\\
	 \hline
	 Berliner Morgenpost & 1 & 1 & -\\
	 \hline
	 Hamburger Morgenpost & 1 & 1 & -\\
	 \hdashline
	 \textbf{Gesamt Deutschland} & \textbf{240} & \textbf{26} & $\chi^2$(1, n = 266) = 172,17, p $<$ 0,001, V = 0,805\\
	 \hline\hline
     \end{tabular}
\end{exe}
Die Zeitungen dürfen dabei wohl als standardsprachlich gelten (wenngleich eine Definition von \textit{Standardsprache} \is{Standardsprache} sicherlich auch nicht ohne Probleme ist), wes\-halb es vielleicht merkwürdig erscheint, aus Zeitungsdaten regionale Unterschiede ableiten zu wollen. Die Betrachtung beansprucht nicht, eine explizite dialektale/regiolektale Studie zu sein. Meine Überlegung ist, dass, wenn gewisse Ausdrücke in bestimmten Gebieten gar nicht bekannt wären,  auch davon auszugehen ist, dass sie nicht in entsprechenden Zeitungen auftreten (abgesehen von direkten Zitaten z.B.). Würde man bei der Durchsicht der Belege feststellen, dass \textit{halt eben} nur in südlichen und ggf. westlichen Zeitungen zu finden ist, während in allen nördlichen und östlichen Zeitungen \textit{eben halt} verwendet wird, hätte ich dies (mit aller Vorsicht) als Hinweis auf eine etwaige regionale Verteilung ge\-wertet, die man weiter verfolgen könnte.

Neben den Zeitungen aus dem DeReKo habe ich fünf westdeutsche und sechs nördlichere Zeitungen auf die Kombinationen hin durchsucht, da man in (\ref{620}) sieht, dass generell mehr Treffer für die Kombinationen aus südlichen Zeitungen stammen. Das Ergebnis sind 82 x \textit{halt eben} : 19 x \textit{eben halt} bei den westdeutschen Zeitungen und 22:11 bei den nördlicheren Zeitungen\footnote{Die Auswahl der Zeitungen habe ich auf der Basis praktischer Kriterien vorgenommen. Es sind Zeitungen, die im Internet frei zugänglich sind und die eine Stringsuche erlauben. Ich bin mir dessen bewusst, dass es weitere norddeutsche Zeitungen gibt, die ggf. auch besser geeignet sind.} (vgl. (\ref{621}) und (\ref{622}) für eine detailliertere Übersicht). 

\begin{exe}
	\ex\label{621}Häufigkeiten \textit{halt eben} und \textit{eben halt} westdeutsche Zeitungen\\[-0.6em]
	\scriptsize
     \begin{tabular}[t]{|l|l|l|l|}
     \hline
	 \textbf{Zeitungen (West)} & \textit{halt eben} & \textit{eben halt} & {} \\
	 \hline
	 Westdeutsche Zeitung & 5 & 2 & -\\
	 \hline
	 Westdeutsche Allgemeine Zeitung & 43 & 9 & $\chi^2$(1, n = 52) = 22,231, p $<$ 0,001, V = 0,654\\
	 \hline
	 Rheinische Post & 22 & 7 & $\chi^2$(1, n = 28) = 6,759, p $<$ 0,01, V = 0,483\\
	 \hline
	 Ruhrnachrichten & 5 & 1 & -\\
	 \hline
	 Kölner Stadtanzeiger & 11 & 2 & $\chi^2$(1, n = 13) = 4,923, p $<$ 0,05, V = 0,615\\
	 \hline
	 \textbf{West gesamt} & \textbf{82} & \textbf{19} & $\chi^2$(1, n = 101) = 39,30, p $<$ 0,001, V = 0,624\\
	 \hline 
     \end{tabular}
\end{exe}

\begin{exe}
	\ex\label{622}Häufigkeiten \textit{halt eben} und \textit{eben halt} norddeutsche Zeitungen\\[-0.6em]
	\scriptsize
     \begin{tabular}[t]{|l|l|l|l|}
     \hline
	 \textbf{Zeitungen (Nord)} & \textit{halt eben} & \textit{eben halt} & {} \\
	 \hline
	 Hamburger Morgenpost & 4 & 1 & -\\
	 \hline
	 Weser Kurier & 1 & 1 & -\\
	 \hline
	 Münsterländer Volkszeitung & 9 & 5 & ns\\
	 \hline
	 Münstersche Zeitung & 5 & 1 & -\\
	 \hline
	 Schleswig-Holsteinischer Zeitungsverlag & 1 & 3 & -\\
	 \hline
	 Hannoversche Allgemeine & 2 & 0 & -\\
	 \hline 
	 \textbf{Nord gesamt} & \textbf{22} & \textbf{11} & ns\\
	 \hline
     \end{tabular}
\end{exe}
Man sieht, dass auch hier das prinzipielle Verhältnis der \textit{halt eben}-Ordnung zur Sequenz \textit{eben halt} bestehen bleibt. Es scheint mir allerdings, als würden Kombinationen aus \textit{eben} und \textit{halt} in norddeutschen Zeitungen generell weniger verwendet werden. Das obige Verhältnis von 22:11 (das nicht signifikant ist) deutet auch darauf hin, dass das Gefälle zwischen den beiden Sequenzen möglicherweise weniger groß ist; wobei mir für diese Zeitungen auch die geringsten Datenmengen vorliegen. Es ist aber definitiv nicht so, dass sich das Verhältnis der beiden Kombinationen zueinander etwa umdreht.

Eine positive Aussage hinsichtlich möglicher regionaler Verteilungen bleibt auf dieser Datenbasis schwierig (s. auch oben). Sie wird dadurch erschwert, dass im DeReKo insgesamt eine deutlich größere Anzahl von Treffern in südliche Gebiete fällt (süddeutsche, österreichische, Schweizer Zeitungen). Meine Absicht ist es nicht, explizit gegen die regionale Annahme zu argumentieren. Es stellt sich mir lediglich die Frage, ob es sich hierbei um einen Aspekt handelt, den es zu berücksichtigen gilt – zumal es auch nicht wirklich empirische Evidenz \underline{für} diese Annahme gibt. Auch wenn die Unterschiede bei den nord-/ostdeutschen Zeitungen nicht derart deutlich sind, sehe ich sie nicht als Grund an, von der Annahme abzuweichen, die prinzipiell über das DeReKo abzuleiten ist, dass \textit{halt eben} deutlich häufiger auftritt als \textit{eben halt}.

Auch bei den gesprochenen Daten sind insgesamt mehr Treffer in südlichen Varietäten zu verzeichnen. Es sind allerdings auch westdeutsche Daten sowie ostdeutsche Dialekte vertreten. Und es ist nicht so, dass die \textit{eben halt}-Fälle, die auftreten, explizit nur von nord- und/oder ostdeutschen Sprechern produziert werden. (\ref{623}) zeigt die Verteilungen auch für diese Daten in der Übersicht. Sofern bekannt, habe ich in den Fußnoten den Aufnahmeort angegeben.\footnote{Drei Treffer aus dem Teilkorpus \glqq Biographische und Reiseerzählungen\grqq{} (1 x \textit{halt eben}, 2 x \textit{eben halt}) habe ich aufgrund der Korpusbeschreibung (\glqq überwiegend junge Frauen und Männer aus Ostdeutschland, Polen und der Tschechoslowakei\grqq{}) nicht in die Auswertung genommen, da mir nicht genügend Informationen vorliegen, um sagen zu können, ob der Beleg einem deutschen Sprecher zuzuschreiben ist.}

\begin{exe}
	\ex\label{623}Häufigkeiten \textit{halt eben} und \textit{eben halt} gesprochene Daten (DGD2)\\[-0.6em]
	\scriptsize
     \begin{tabular}[t]{|l|l|l|l|}
     \hline
	 \textbf{Korpora} & \textit{halt eben} & \textit{eben halt} & {} \\
	 \hline
	 \textbf{FOLK} (2006–2011) & 6\footnotemark & - & -\\
	 \hline
	 \textbf{elizitierte Konfliktgespräche}  & 4 & - & -\\
	 (1988–1990) & & & \\
	 \hline
	 \textbf{Deutsche Mundarten: ehemalige}  & 19\footnotemark & 1\footnotemark & $\chi^2$(1, n = 20) = 14,45, p $<$ 0,001, \\
	 \textbf{deutsche Ostgebiete} (1962–1965)\footnotemark & & & V = 0,85\\
	 \hline
	 \textbf{Dialogstrukturen} (1960–1977)\footnotemark & 2 & 1 & -\\
	 \hline
	 \textbf{Pfefferkorpus} (1961)\footnotemark & 2 & - & -\\
	 \hline
	 \textbf{Grundstrukturen:}  & - & 3\footnotemark & -\\
	 \textbf{Freiburger Korpus} (1969–1974)\footnotemark & & & \\
	 \hline
	 \textbf{Zwirner Korpus} & 28 & 5 & $\chi^2$(1, n = 33) = 14,67, p $<$ 0,001, \\
	 (1955–1972)\footnotemark & & & V = 0,667\\
	 \hline
	 \textbf{Elizitierte Konfliktgespräche}  & 2 & - & -\\
	 (1988–1990) & & &\\
	 \hline
	 \textbf{gesprochen gesamt} & \textbf{63} & \textbf{10} & $\chi^2$(1, n = 73) = 38,48, p $<$ 0,001,\\
	 & & & V = 0,726\\
	 \hline
     \end{tabular}
     \footnotetext[19]{alemannisch: 3 (Baden-Württemberg, Bayern), obersächsisch: 2 (Sachsen), rheinfränkisch: 1 (Hessen, 			 Baden-Württemberg, Saarland, Rheinland-Pfalz, Bayern)}
     \footnotetext[20] {\glqq Sprecher ost- und südostdeutscher Dialekte, die den Sprachstand von 1945 repräsentieren\grqq{}  		(Korpusinformation unter: http://dgd.ids-mannheim.de)}
     \footnotetext[21] {Die regionalen Angaben beziehen sich auf den Wohnort der Befragten zum Zeitpunkt der Befragung. Von 			 Interesse war bei der Erhebung, dass es sich bei den Sprechern um Bewohner der ehemaligen deutschen Ostgebiete 				handelte (Territorien östlich der Oder-Neiße-Linie, die seit nach dem 2. Weltkrieg zu Polen und Russland gehören). 				Nordrhein-Westfalen: 7, Bayern: 9, Hessen: 2, Schleswig-Holstein: 1}
     \footnotetext[22] {Bayern: 1}    
     \footnotetext[23] {\glqq Sprecher der Standardsprache bzw. standardnahen Sprache\grqq{}(Korpusbeschreibung auf \underline http://dgd.ids-mannheim.de)}
     \footnotetext[24] {Bayern (Nürnberg, München): 2}
     \footnotetext[25] {\glqq Sprecher der Standardsprache bzw. standardnahen Sprache\grqq{} (Korpusbeschreibung auf http://dgd.ids-mannheim.de)}
 	 \footnotetext[26] {Bayern: 4:1, Rheinland-Pfalz: 3:2, Baden-Württemberg: 16:1, Nordrhein-Westfalen: 3:1, Schleswig-				  Holstein: 2:0, Hessen: 1:0}
     \footnotetext[27] {Baden-Württemberg (Freiburg): 2}
\end{exe}
Für erwähnenswert halte ich im Zuge dieser Korpusuntersuchung auch, dass im DeReKo ein deutlich größerer Anteil der \textit{eben halt}-Vorkommen (76 aus 117 $[$ $\approx$ 65\%$]$) als der \textit{halt eben}-Vorkommen (249 aus 715 $[$ $\approx$ 35\%$]$) aus den Webdaten stammen, die Teil des DeReKo sind.\footnote{Ich möchte an dieser Stelle generell den Aspekt hervorheben, dass das DeReKo mit Wikipedia-Diskussionen Webdaten enthält. Das Korpus wird im Vergleich zu speziellen Webkorpora oder gar Google-Recherchen als seriöser angesehen. Wir sehen hier aber, dass auch in diesen Daten ein großer Anteil der MP-Kombinationen auf den Webdatentyp zurückgeht.} (\ref{624}) zeigt die Verhältnisse der beiden Abfolgen nach Auftreten in Zeitungen und in den Wikipedia-Diskussionen.

\begin{exe}
	\ex\label{624} Verteilungen in DeReKo\\[-1em]
     \begin{tabular}[t]{|l|l|l|}
     \hline
     DeReKo & \textit{halt eben} & \textit{eben halt}\\
     \hline
     Wikipedia & 249 & 76\\
     \hline
     Zeitungen & 466 & 41\\
     \hline
     \end{tabular}
\end{exe}
Der Unterschied zwischen den Häufigkeiten von \textit{halt eben} und \textit{eben halt} ist je\-weils statistisch signifikant und von hohen Effektstärken begleitet.\footnote{$\chi^2$(1, n = 325) = 92,09, p $<$ 0,001, V = 0,532), $\chi^2$(1, n = 507) = 356,26, p $<$ 0,001, V = 0,838} Dennoch sieht man, dass das Verhältnis der zwei Abfolgen zueinander in den Wikipedia-Daten ungefähr 77\%-23\% beträgt, während es in den Zeitungsdaten bei 92\%-8\% liegt.\footnote{$\chi^2$(1, n = 832) = 38,35, p $<$ 0,001, V = 0,215} Wenngleich auch möglicherweise nicht verwunderlich, liefern die Frequenzen einen Hinweis darauf, dass der Unterschied zwischen den zwei Abfolgen für das normorientiertere Medium noch größer ist.

(\ref{625}) zeigt erneut die Verteilungen der zwei Kombinationen innerhalb der Webdaten, die in DECOW2012 enthalten sind.\footnote{96 \textit{halt eben}-Belege habe ich nicht in die Darstellung der Verteilung aufgenommen. Es handelt sich hierbei um eine sehr extreme Verwendung durch ein und denselben Sprecher in einem Transkript eines Gespräches.}

\begin{exe}
	\ex\label{625} Verteilungen in DECOW2012\\[-1em]
     \begin{tabular}[t]{|l|l|l|}
     \hline
     & \textit{halt eben} & \textit{eben halt}\\
     \hline
     DECOW2012 & 7328 & 2291\\
     \hline
     \end{tabular}
\end{exe}
Bei den von mir durchsuchten Korpora handelt es sich um die größten Korpora für geschriebene, gesprochene sowie als neueren Datentyp webbasierte Daten. Das Ergebnis der Untersuchung zu der Kombination aus \textit{halt} und \textit{eben} auf dieser weiten Datenbasis ist, dass die Abfolge \textit{halt eben} der Reihung \textit{eben halt} deutlich (und zwar auch in statistischem Sinne relevant) überwiegt. Die Untersuchung macht wenige positive Aussagen zur Frage der regionalen Verteilung. Unter den Treffern sind wenige norddeutsche Daten. Ob dies darauf zurückzuführen ist, dass schlicht wenige norddeutsche Daten in den Korpora vertreten sind oder ob die Struktur weniger verwendet wird, kann ich nicht klären. Auch da, wo norddeutsche Daten auftreten, stellt man nicht fest, dass sich die Verhältnisse völlig umdrehen. Dazu liegen auch westdeutsche Daten in größerer Anzahl vor, in denen \textit{halt eben} gleichermaßen wie in den südlicheren Daten überwiegt. Auch in den ostdeutschen Daten überwiegt \textit{halt eben}, obwohl in Ostdeutschland in den 70er/80er Jahren weitestgehend \textit{eben} verbreitet gewesen sein soll (vgl. Abschnitt~\ref{sec:regio}). Wenn die gesprochenen Daten (\textit{Deutsche Mundarten: ehemalige deutsche Ostgebiete}) den Stand des Ostdeutschen um 1945 widerspiegeln, sollte es sich erst recht um eine Zeit handeln, in der \textit{halt} in Ostdeutschland wenig verbreitet war. Deutlich wird m.E. auch, dass in den Gebieten, in denen \textit{eben} und \textit{halt} den in Abschnitt~\ref{sec:regio} skizzierten Untersuchungen zufolge schon seit den 70er/80er Jahren verwendet werden, \textit{halt eben} bevorzugt wird und die weniger auftretenden \textit{eben halt}-Treffer aber ebenfalls von süddeutschen Sprechen stammen. D.h. es verhält sich nicht so, dass die vereinzelten \textit{eben halt}-Folgen Einstreuungen nord- oder ostdeutscher Sprecher sind. Von Interesse wäre es sicherlich, entsprechend große Mengen aktueller nord- und ostdeutscher Daten auf diese Fragestellung hin zu untersuchen. 

Die Verteilung der beiden Abfolgen bleibt auch bestehen, wenn man dialektale Formen mit in die Suche aufnimmt. Ich habe in DeReKo und DGD2 Suchen durchgeführt für Kombinationen aus \textit{haut} (Westen der deutschsprachigen Schweiz), \textit{hait}, \textit{hoet}, \textit{hoit}, \textit{halter} (Österreich), \textit{halch} (Sachsen) bzw. \textit{halt} und \textit{ebent} (Ost-Berlin),\textit{ eem}, \textit{ebe} (Schweiz $[$gesprochen$]$), \textit{eaba} (Süddeutsch) bzw. \textit{eben} (zu den Formen und ihren Zuordnungen vgl. \citealt[167]{Protze1997}, \citealt[16]{Elspass2005}, \citealt[31]{Eichhoff1978}, Elspaß \& Möller 2012). In der DGD finden sich zwar Treffer für \textit{ebent} und \textit{ebe}, aber keine Kombination mit \textit{halt} oder einer ihrer Varianten. Das DeReKo liefert vier Treffer für \textit{halt ebe} im Schweizer Deutschen, d.h. die Trefferzahl würde sich unter Hinzunahme dieser Formen sogar noch etwas erhöhen.

Für regionale Unterschiede im Gebrauch der zwei MP-Kombinationen aus \textit{halt} und \textit{eben} lässt sich in den von mir untersuchten Korpora keine Evidenz finden, wenngleich die Untersuchung natürlich keine dialektale/regiolektale Untersuchung ersetzen kann und soll, sondern ich diese Anmerkung am Rande machen möchte, um die regionale Information der Daten zu nutzen. Über die be\-trachteten sehr großen Datenmengen leite ich ab, dass hinsichtlich des Gebrauchs der MP-Kombi\-nationen \textit{halt eben} unmarkierter ist als \textit{eben halt}. Die von \citet{Autenrieth2002} hervorgebrachte Einschätzung, es handle sich bei den Kombinationen prinzipiell um Fehlleistungen, nämlich Planungsfehler, möchte ich entschieden ablehnen. Die Kombinationen treten in allen drei Korpora in sehr großer Anzahl auf. Von einem sporadischen Gebrauch, den man dann ggf. mit solcherart Fehlplanung in gesprochener Sprache assoziieren könnte, kann hier m.E. nicht die Rede sein. Dazu treten die Abfolgen auch nicht ausschließlich in medial mündlicher Sprache auf. Man müsste sich auch fragen, wie man die deutlich häufigere Verwendung der Abfolge \textit{halt eben} gegenüber der Reihung \textit{eben halt} unter dieser Sicht erfassen wollte. D.h. man müsste sich fragen, warum die Planungsfehler präferiert durch die Sequenz \textit{halt eben} realisiert werden. Es scheint mir deshalb plausibler, davon auszugehen, dass die Kombinationen aus \textit{halt} und \textit{eben} im Sprachsystem ernstzunehmende Formen darstellen, die ein Sprecher zu Zwecken ganz bestimmter kommunikativer Absichten im Gespräch verwendet. Die in Abschnitt~\ref{sec:anlit} eröff\-nete Frage, ob das Auftreten der Ordnung \textit{halt eben} oder \textit{eben halt} durch den Satzmodus \is{Satzmodus} bzw. den Illokutionstyp \is{Illokutionstyp} bedingt ist, lässt sich auf der Basis dieser Daten nicht gut beantworten, da u.a. prinzipiell sehr wenige Direktive auftreten. Dazu müsste man sich überlegen, mit welchen Strukturen man diese vergleichen wollte, um etwaige Verteilungsunterschiede auszumachen. Da \textit{halt} und \textit{eben} nur in Assertionen und Direktiven auftreten können, sind alle non-direktiven Äußerungen somit erst einmal assertiv. Man könnte sich deshalb vorstellen, \textit{halt eben}-Direktive \textit{halt eben}-Assertionen und \textit{eben halt}-Direktive \textit{eben halt}-Assertio\-nen gegenüberzustellen und die Verteilung der beiden Verteilungen zu vergleichen. Allerdings finden sich aber auch viele Nebensätze, Ellipsen, lose Anschlüsse, Nachträge oder modalisierte Strukturen unter den Daten, für die man klären müsste, ob man sie den Direktiven wirklich gegenüberstellen möchte, da es zu ihnen kein direktives \glq Pendant\grq {} gibt. Selbst wenn man hier einen Weg fände, ist eine gute Bewertung der festzustellenden Verteilungen von \textit{halt eben} und \textit{eben halt} dann nur möglich, wenn man angeben kann, wie das Verhältnis von Direktiven und Assertionen im Korpus überhaupt aussieht. Es erscheint mir nahezu unmöglich, hier einen Wert zu ermitteln, der eine sehr gute (und deshalb erst brauchbare) Näherung darstellt (d.h. mit einer möglichst kleinen Streuung). Für die Frage nach einer etwaigen satzmodalen bzw. illokutionär ge\-steuerten Verteilung erscheinen mir Korpusuntersuchungen aus den oben angeführten Gründen ungeeignet.

Um weitere Evidenz für den Markiertheitsstatus von \textit{eben halt} zu finden sowie eine Antwort auf die Frage nach dem potenziellen Einfluss des Satzmodus/Illoku\-tionstyps auf die Abfolge geben zu können, habe ich Akzeptabilitätsurteile erhoben, deren Ergebnisse Gegenstand des folgenden Abschnitts sind.

\subsection{Sprecherurteile}
\label{sec:spu}
Durchgeführt wurden zunächst zwei verschiedene Akzeptabilitätsstudien, die jeweils Urteile für 48 Sätze am Ende kleinerer Kontexte eingeholt haben (vgl. auch \citealt[155-161]{Mueller2016b}; \citeyear[235-238]{Mueller2017a}). Die 48 Sätze setzen sich zusammen aus 12 Testsätzen und 36 Fillern (s.u.). Es handelt sich jeweils um Paarvergleiche, d.h. den Testanten wurden Minimalpaare vorgelegt und sie hatten die Auswahl zu treffen zwischen \glqq Satz a) ist besser als Satz b)\grqq{} und \glqq Satz b) ist besser als Satz a)\grqq{}. Die 12 Testsätze setzen sich zusammen aus sechs Assertionen ($[$-w$]$, V2-Deklarativsätze) bzw. sechs Direktiven (Imperativsätze), in denen die Abfolgen \textit{halt eben} und \textit{eben halt} gegenübergestellt werden, und sechs Testitems, die ein Phänomen untersuchen, das ich in Abschnitt~\ref{sec:verstimpli} einführen werde.

Sowohl in den Deklarativ- als auch in den Imperativsätzen habe ich versucht, zu garantieren, dass die Partikeln \textit{halt} und \textit{eben} gleichermaßen plausibel stehen können. Ist dies nicht gewährleistet, kann die Präferenz für die ein oder andere Abfolge auch leicht dadurch zustande kommen, dass die Einzelverwendung der jeweiligen MP präferiert ist.

Im Falle der Assertion kann der in der MP-Äußerung ausgedrückte Sachverhalt plausibel als schon bekannte Information ausgegeben werden (\textit{eben}), er ist aber nicht explizit vorerwähnt (andernfalls wäre \textit{halt} ggf. unangemessen). D.h. p kann präsupponiert sein (\textit{eben}), es kann aber auch ebenso plausibel assertiert sein (\textit{halt}). Der Zusammenhang zwischen den zwei Propositionen p und q könnte prinzipiell im cg stehen (\textit{eben}), er ist aber nicht derart beschaffen, dass er dort stehen muss, d.h. es kann sich auch um zwei Sachverhalte handeln, die nur der Sprecher als zusammenhängend ansieht (\textit{halt}). 

Dazu wird die temporale Lesart von \textit{eben} dadurch ausgeschlossen, dass jeweils Sachverhalte ausgedrückt werden, die nicht ausschließlich temporär vorliegen können, sondern vielmehr inhärent anhaften. (\ref{626}) zeigt die Testsätze.

\begin{exe}
	\ex\label{626} 
	 Er ist \textbf{halt eben}/\textbf{eben halt} Maurer/Brite/Moslem/Waage/Schwabe/Kölner.
\end{exe}
Sternzeichen, Nationalitäten, Berufe, Religionen oder Herkunft gelten in der Regel nicht nur für eine überschaubare Zeit oder einen Moment \glqq kurz einmal\grqq{} oder \glqq gerade eben\grqq{}, d.h. \textit{eben} kann in (\ref{626}) nicht als temporales Adverb gelesen werden. Um den Zusammenhang p $>$ q in die Situation einzufügen, habe ich Stereotype in die Kontexte eingebaut, wie \glqq Schwaben sind sparsam\grqq{}, \glqq Menschen mit dem Sternzeichen Waage lieben die Gerechtigkeit\grqq{} oder \glqq Kölner feiern Karneval\grqq{}. Strukturell sind die Testsätze alle gleich aufgebaut (vgl. (\ref{627})).

\begin{exe}
	\ex\label{627} 
	 \textit{Er} $\plus$ \textit{ist} $\plus$ \textit{halt eben}/\textit{eben halt} $\plus$ nackte NP (zweisilbig, erste Silbe akzentuiert).
\end{exe}	
Ebenso sind die vorweggehenden Kontexte gleich beschaffen, indem der Gesprächs\-partner nach dem Grund für eine Eigenschaft fragt. Der Testsatz ist die Antwort auf diese Frage. (\ref{628}) und (\ref{629}) sind Beispiele für zwei Testitems.

\begin{exe}
	\ex\label{628} 
	 A5 Geiz\\
	Kathrin: Warum ist Andreas eigentlich immer so sparsam?\\
	Daniel: \underline{Er ist halt eben Schwabe.}/\underline{Er ist eben halt Schwabe.}
\end{exe}	

\begin{exe}
	\ex\label{629} 
	A1 Umbauarbeiten\\
	Fritz: Warum will Herr Dicke die Wand im Esszimmer eigentlich selber versetzen?\\
	Phillip: \underline{Er ist halt eben Maurer.}/\underline{Er ist eben halt Maurer.}
\end{exe}	
Die MP-Äußerung enthält p, der Sachverhalt q wird durch die Vorgangsfrage eingeführt. D.h. in (\ref{629}) z.B. gelten die konkreten Verhältnisse in (\ref{630}).

\begin{exe}
	\ex\label{630} 
	p = dass er Maurer ist\\
	Inferenzrelation: Wenn er Maurer ist (p), versetzt er die Wand im eigenen Esszimmer plausiblerweise selber (q). $[$p $>$q		$]$
\end{exe}
Fritz könnte prinzipiell wissen, welchen Beruf Herr Dicke ausübt ($\rightarrow$ \textit{eben}), diese Information ist aber auch nicht vorerwähnt und kann demzufolge neu assertiert werden ($\rightarrow$ \textit{halt}). Der Zusammenhang zwischen p und q ist denkbar, d.h. er kann durchaus im cg sein ($\rightarrow$ \textit{eben}), es muss aber auch nicht zwingend Einigkeit über ihn bestehen. Natürlich erledigen nicht alle Berufsgruppen Dinge, die sie tun können, selbst. 

Bei den Fillern handelt es sich zur Hälfte um grammatische Phänomene und zur Hälfte um \is{Kohärenzrelation} Kohärenzrelationen. Unter die 18 grammatischen Phänomene aus Syntax, Semantik und Pragmatik fallen z.B. die Selektion von verschiedenen Komplementsätzen, Korrelate, Tempora in Imperativen und die Wortstellung in Adverbialsätzen. In acht Fällen liegen deutliche Unterschiede vor, in sieben Fällen leichtere Unterschiede, in drei Fällen sind beide der gegenübergestellten Sätze gleich gut bzw. schlecht. In den übrigen 18 Fillersätzen liegen temporale bzw. kausale Kohärenzrelationen zwischen zwei Teilsätzen vor, die Auswirkungen auf die Abfolge der Sätze nehmen sollten. In fünf Fällen liegt kein Unterschied zwischen den zwei Sätzen vor. Der Grund für die Aufnahme der 18 Filler, die nicht im engeren Sinne grammatische Phänomene betreffen, wird deutlich, wenn in Abschnitt~\ref{sec:verstimpli} das mit den weiteren sechs Testitems untersuchte Phäno\-men in die Diskussion eingeführt wird.

Es gab acht verschiedene Versionen des Tests, in denen jeweils dieselben 12 Testitems zur Bewertung standen. Vier dieser acht Versionen wurden zudem in zwei verschiedenen Randomisierungen dargeboten. In diesen 12 resultierenden Varianten sind die Antworten a) und b) sowohl bei den Testsätzen als auch den Fillern ausbalanciert und wechseln zwischen den Tests.

Genauso wie in den Deklarativsätzen habe ich auch in den Imperativen die temporale Lesart von \textit{eben} dadurch möglichst ausgeschlossen, dass der Gesprächs\-partner im Beitrag vor der MP-Äußerung ein Problem ausdrückt, das fortwährend besteht. D.h. etwas ist \textit{immer}, \textit{seit Langem}, \textit{morgens}, \textit{jedes Mal} der Fall. Der Sprecher gibt daraufhin mit der MP-Äußerung einen Rat, was der Diskurspartner zur Lösung des Problems tun soll. Da das vorweggehende Problem als sich wiederholend oder andauernd beschrieben wird, scheint mir \textit{eben} nicht als temporales Adverb \is{Temporaladverb} interpretiert werden zu können. (\ref{631}) und (\ref{632}) sind Beispiele für Testitems.

\begin{exe}
	\ex\label{631} 
	C1 Probe\\
	Ferdinand: Ich komme jedes Mal zu spät zur Orchesterprobe.\\
	Wolfgang: \underline{Dann lauf halt eben früher los!}/\underline{Dann lauf eben halt früher los!}
\end{exe}

\begin{exe}
	\ex\label{632} 
	C4 Schulnöte\\
	Peter: Ich bin schon immer der schlechteste Schüler der Klasse.\\
	Astrid: \underline{Dann pass halt eben besser auf!}/\underline{Dann pass eben halt besser auf!}
\end{exe}
Die Struktur der zu bewertenden Sätze ist auch hier immer dieselbe (vgl. (\ref{633})).

\begin{exe}
	\ex\label{633} 
	\textit{Dann} $\plus$ finites Verb im Imperativ (einsilbig) $\plus$ \textit{halt eben}/\textit{eben halt} $\plus$ Adverb 		(zweisilbig, endet auf \{-er\}) $\plus$ Verbpartikel (einsilbig)!
\end{exe}
Ein inhaltliches Kriterium, das sicherstellen soll, dass \textit{halt} und \textit{eben} gleichermaßen verwendet werden können, ist, dass es Alternativen zu der vom Sprecher geratenen Handlung geben kann (\textit{halt}), dass die Lösung, die der Sprecher empfiehlt, aber eine naheliegende ist, auf die der Adressat auch selber hätte kommen können (\textit{eben}). In (\ref{632}) z.B. ist für den Hörer zwar einfach einzusehen, dass früheres Loslaufen (= !p) das erwünschte Resultat bringen kann, es sind aber auch alternative Zusammenhänge wie schnelleres Laufen, mit Rad/Bus/Bahn fahren oder einen anderen Weg nehmen denkbar, um das Problem des Zuspätkommens (= q) zu beseitigen. D.h. der Hörer muss mit dem Sprecher der MP-Äußerung nicht darin übereinstimmen, dass genau dieser Zusammenhang vorliegt und er deshalb früher loslaufen sollte (\textit{halt}).

Wie schon in Abschnitt~\ref{sec:val} angesprochen bin ich der Meinung, dass dieser Aspekt in den einzigen Testitems der dialektalen Erhebungen von \citet{Eichhoff1978}, \citet{Elspass2005} und \citet{Protze1997} bei der Untersuchung des Auftretens der Einzelpartikeln nicht genügend berücksichtigt worden ist. M.E. handelt es sich bei dem Satz in (\ref{634}), der in allen Studien verwendet wurde, um einen Kontext, in dem eher \textit{halt} verwendet wird.

\begin{exe}
	\ex\label{634} 
	Der Zug fährt erst in einer Stunde, da muß ich \hrulefill \ so lange warten.
\end{exe}
Diese Intuition teilt auch schon \citet[174]{Hentschel1986} unter Bezug auf die Konnotationen, die sie \textit{halt} und \textit{eben} zuschreibt (vgl. Abschnitt~\ref{sec:untersch}). Sie nimmt an, \textit{halt} sei emotionaler konnotiert als \textit{eben} und der Kontext in (\ref{634}) lege eine entsprechend emotionale Verbindung zum Geschehen nahe. Im Rahmen meiner Analyse von \textit{halt} und \textit{eben} ist \textit{halt} sprecherorientierter als \textit{eben}, weil es Bezug nimmt auf einen Zusammenhang, den der Sprecher als bestehend annimmt, über den sich die Dialogpartner (anders als im Falle von \textit{eben}) aber nicht einig sein müssen. Gilt es für einen Sprecher wie in (\ref{634}), sein \underline{eigenes} Problem zu lösen, und ist aus dem Kontext heraus nicht einmal ersichtlich, dass ein weiterer Gesprächs\-partner beteiligt ist, mit dem er einen cg teilt, scheint es mir naheliegender, \textit{halt} zu verwenden. Die Partikel \textit{eben} ist natürlich zulässig (der Zusammenhang ist ja tatsächlich recht offensichtlich), sie scheint mir aber dispräferiert zu sein. Die überwiegenden \textit{halt}-Nennungen im südlichen Sprachraum (trotz Bekanntheit von \textit{eben}) können folglich auch ein Reflex dieses hinsichtlich des \textit{halt}- und \textit{eben}-Gebrauchs nicht neutralen Kontextes sein. Kannten die Sprecher im Norden und Osten tatsächlich nur \textit{eben}, war dies für sie natürlich die einzig denkbare Nennung.

In die Auswertung von Test 1 (\textit{halt eben}-/\textit{eben halt}-Assertionen) \is{Assertion} sind die Bewertungen von 29 deutschen Muttersprachlern, bei denen es sich um Studierende an der Universität Bielefeld im WS 2013/2014 handelte (23 Germanistik-Studieren\-de, 6 Linguistik-Studierende), eingegangen.\footnote{Aus der Wertung ausgeschlossen wurden die Bögen von vier ausländischen Studierenden.} Alle Testanten waren naiv gegenüber dem getesteten Phänomen. Die Daten in Test 2 (\textit{halt eben}-/\textit{eben halt}-Direktive) stammen von 32 deutschen Muttersprachlern (ebenfalls Germanistik-Studierende im WS 2013/2014 an der Universität Bielefeld).\footnote{Auch hier mussten vier weitere Bögen von ausländischen Studierenden aus der Wertung ausgeschlossen werden.} \footnote{Ich bedanke mich bei Sandra Pappert und Jens Michaelis, dass sie den Fragebogen in ihren Kursen haben bearbeiten lassen sowie den Teilnehmern meiner eigenen Veranstaltungen im WS 2013/2014.} 

Aufgrund der Diskussion um dialektale Unterschiede bei der Verwendung von \textit{halt} und \textit{eben} sowie ihrem kombinierten Gebrauch sind die demografischen Anga\-ben bei dieser Untersuchung interessant. Es handelt sich bei den Testanten um relativ junge Menschen, die in Test 1 19 bis 32 Jahre ($\diameter 26$) und in Test 2 zwi\-schen 18 und 33 Jahre ($\diameter 23$) alt sind. Abgefragt wurden der Geburts\-ort, der Ort der Schulzeit sowie der aktuelle Wohnort. Letzterer war bei allen Teilnehmern Bielefeld oder die nähere (Pendel)umgebung, d.h. das nördliche Ruhrgebiet, Ost\-westfalen oder das südliche Niedersachsen. Als Kriterium für die sprachliche Verankerung der Sprecher diente der Wohnort der Schulzeit, da sie dort die längste Zeit ihres Lebens verbracht haben dürften. Entsprechend der Datenpunkte bei \citet{Eichhoff1978} stammen die Testanten so gut wie ausschließlich aus Be\-reich C und der Mitte bzw. dem Norden aus Bereich D. D.h. die Sprecher sind nach dem Kriterium des Ortes der Schulzeit im nördlichen Nordrhein-Westfalen und im südlichen Niedersachsen zu verorten. Diese Zuordnungen würden sich auch in den wenigsten Fällen verändern, wenn man als Kriterium den Geburts\-ort ansetzen würde. In Test 1 fallen 86\% der Sprecher in Bereich C, 14\% in den mittleren und nördlichen Bereich von D. In Test 2 sind 80\% dem Bereich C, 10\% dem nördlichen D und drei Sprecher anderen Gebieten zuzuordnen (1 x E, 1 x A, 1 x H).

Ich halte diese Angaben von daher für interessant, als dass die Teilnehmer dieser Studie ihrer Herkunft nach in den Bereich fallen, der ursprünglich einmal eine Gegend war, in der \textit{eben} verwendet wurde und \textit{halt} als ungebräuchlich oder sogar unbekannt galt. Wenn es (wie \citealt[17, Fn 41]{Elspass2005} spekuliert) so sein sollte, dass auch die zwei auffindbaren Abfolgen von \textit{halt} und \textit{eben} regional verteilt sind, würde es sich bei meinen Testanten um Sprecher handeln, für die gelten müsste, dass sie die Reihung \textit{eben halt} bevorzugen.\footnote{Wie bereits erwähnt, behauptet Elspaß dies nicht explizit, weil er für die genaue Gestalt des Zusammenhangs von Herkunft und Partikelgebrauch keinen Vorschlag macht. Die Zuordnung von \textit{eben halt} \& Nord/Ost sowie \textit{halt eben} \& Süd/West ist lediglich meine plausible Weiterführung seiner Annahme, basierend auf den zwei Studien, die regional verschiedene Daten zur Diskussion beisteuern (\citealt{Thurmair1989}: \textit{halt eben} $[$Bairisch$]$, \citealt{Dittmar2000}: \textit{eben halt} $[$Berlinerisch$]$).}

Für die Häufigkeitsverteilungen bei der Wahl zwischen den \textit{halt eben}- und \textit{eben halt}-Assertionen (Test 1) ergibt sich für jedes der sechs Testitems eine deutliche Bevorzugung der Abfolge \textit{halt eben} (vgl. (\ref{635})).
\pagebreak
\begin{exe}
\renewcommand{\arraystretch}{1.75}
\ex\label{635} Ergebnisse Häufigkeiten \emph{halt eben} \& \emph{eben halt} in Assertionen\\[-0.6em]
\scriptsize
\begin{tabular}[t]{|l|l|p{25em}|}
\hline 
\multicolumn{2}{|l|}{\textbf{\textit{Maurer}}}& \multirow{3}{25em}{\textbf{Umbauarbeiten}\newline Fritz: Warum will Herr Dicke die Wand im Esszimmer eigentlich selber versetzen?\newline Phillip:  \underline{Er ist halt eben Maurer.}/\underline{Er ist eben halt Maurer.}}\\
\cline{1-2}
\emph{halt eben}& \emph{eben halt} & {}\\
\cline{1-2}
24 & 4 & {}\\
\hline
\multicolumn{2}{|l|}{\textbf{\textit{Brite}}}& \multirow{3}{25em}{\textbf{Umgangsformen}\newline Verena: Warum ist dein neuer Freund eigentlich immer so höflich?\newline Sara: \underline{Er ist halt eben Brite.}/\underline{Er ist eben halt Brite.}}\\
\cline{1-2}
\emph{halt eben}& \emph{eben halt} & {}\\
\cline{1-2}
24 & 4 & {}\\
\hline
\multicolumn{2}{|l|}{\textbf{\textit{Moslem}}}& \multirow{3}{25em}{\textbf{Feierabendbier
}\newline Dolores: Warum kommt dein neuer Kollege eigentlich so selten mit in die Kneipe?\newline Sabrina: \underline{Er ist halt eben Moslem.}/\underline{Er ist eben halt Moslem.}}\\
\cline{1-2}
\emph{halt eben}& \emph{eben halt} & {}\\
\cline{1-2}
22 & 5 & {}\\
\hline
\multicolumn{2}{|l|}{\textbf{\textit{Waage}}}& \multirow{3}{25em}{\textbf{Sternzeichen}\newline Günther: Warum setzt sich Peter eigentlich immer so für Gerechtigkeit ein? \newline
Martin: \underline{Er ist halt eben Waage.}/\underline{Er ist eben halt Waage.}}\\
\cline{1-2}
\emph{halt eben}& \emph{eben halt} & {}\\
\cline{1-2}
25 & 3 & {}\\
\hline
\multicolumn{2}{|l|}{\textbf{\textit{Schwabe}}}& \multirow{3}{25em}{\textbf{Geiz
}\newline Kathrin: Warum ist Andreas eigentlich immer so sparsam? \newline
Daniel: \underline{Er ist halt eben Schwabe.}/\underline{Er ist eben halt Schwabe.}}\\
\cline{1-2}
\emph{halt eben}& \emph{eben halt} & {}\\
\cline{1-2}
22 & 6 & {}\\
\hline
\multicolumn{2}{|l|}{\textbf{\textit{Kölner}}}& \multirow{3}{25em}{\textbf{Interessen
}\newline Ralf: Warum liebt dein neuer Nachbar eigentlich den Karneval so 
sehr, obwohl das doch in Ostwestfalen kaum wen interessiert? \newline	
Martina: \underline{Er ist halt eben Kölner.}/\underline{Er ist eben halt Kölner.}}\\
\cline{1-2}
\emph{halt eben}& \emph{eben halt} & {}\\
\cline{1-2}
21 & 7 & {}\\
\hline
\end{tabular}
\end{exe}
Es wurde ein log-lineares gemischtes Modell gerechnet (\citealt{Baayen2008}) mit Abfolge als abhängiger Variable und Teilnehmern und Items als Zufallsvariablen (N = 167, log-Likelihood = $\minus$64,83). Das signifikante Interzept ($\beta$ = 2,750, SE = 0,543, Wald z = 5,07, p $<$ 0,001) zeigt, dass signifikant häufiger \textit{halt eben} als \textit{eben halt} verwendet wird.\footnote{Ich danke Sandra Pappert für ihre Hilfe mit der statistischen Auswertung der Experimente.}

Test 2, in dem \textit{halt eben}- und \textit{eben halt}-Direktive \is{Direktiv} die sechs Testitems ausmachen, entwirft das gleiche Bild: Auch in Direktiven ziehen die Sprecher die Abfolge \textit{halt eben} der Abfolge \textit{eben halt} deutlich vor (vgl. (\ref{636})).
\pagebreak
\begin{exe}
\renewcommand{\arraystretch}{1.75}
\ex\label{636} Ergebnisse Häufigkeiten \emph{halt eben} \& \emph{eben halt} in Direktiven\\[-0.6em]
\scriptsize
\begin{tabular}[t]{|l|l|p{30em}|}
\hline
\multicolumn{2}{|l|}{\textbf{\textit{fahr ab}}}& \multirow{3}{28em}{\textbf{Arbeitsbeginn}\newline Susanne: Ich bin jeden Tag immer viel zu früh im Büro.\newline Daniela: \underline{Dann fahr halt eben später ab!}/\underline{Dann fahr eben halt später ab!}}\\
\cline{1-2}
\emph{halt eben}& \emph{eben halt} & {}\\
\cline{1-2}
31 & 0 & {}\\
\hline
\multicolumn{2}{|l|}{\textbf{\textit{pass auf}}}& \multirow{3}{28em}{\textbf{Schulnöte}\newline Peter: Ich bin schon immer der schlechteste Schüler der Klasse. \newline Astrid: \underline{Dann pass halt eben besser auf!}/\underline{Dann pass eben halt besser auf!}}\\
\cline{1-2}
\emph{halt eben}& \emph{eben halt} & {}\\
\cline{1-2}
29 & 2 & {}\\
\hline
\multicolumn{2}{|l|}{\textbf{\textit{geh hin}}}& \multirow{3}{28em}{\textbf{Wissenslücken
}\newline Agnes: Ich verpasse schon seit Langem viel zu viel Stoff an der Uni.\newline
Caroline: \underline{Dann geh halt eben öfter hin!}/\underline{Dann geh eben halt öfter hin!}}\\
\cline{1-2}
\emph{halt eben}& \emph{eben halt} & {}\\
\cline{1-2}
28 & 3 & {}\\
\hline
\multicolumn{2}{|l|}{\textbf{\textit{schlag zu}}}& \multirow{3}{28em}{\textbf{Boxsport}\newline Bastian: Ich verliere andauernd, wirklich jeden meiner Kämpfe. \newline Julius: \underline{Dann schlag halt eben schneller zu!}/\underline{Dann schlag eben halt schneller zu!}}\\
\cline{1-2}
\emph{halt eben}& \emph{eben halt} & {}\\
\cline{1-2}
28 & 3 & {}\\
\hline
\multicolumn{2}{|l|}{\textbf{\textit{lauf los}}}& \multirow{3}{28em}{\textbf{Probe
}\newline Ferdinand: Ich komme jedes Mal zu spät zur Orchesterprobe. \newline Wolfgang: \underline{Dann lauf halt eben früher los!}/\underline{Dann lauf eben halt früher los!}}\\
\cline{1-2}
\emph{halt eben}& \emph{eben halt} & {}\\
\cline{1-2}
31 & 1 & {}\\
\hline
\multicolumn{2}{|l|}{\textbf{\textit{greif durch}}}& \multirow{3}{28em}{\textbf{Verhaltensprobleme
}\newline Tanja: Unser Hund macht ständig, was er will, d.h. bellt mich an, springt mich an, schnappt nach mir. \newline
Nils: \underline{Dann greif halt eben härter durch!}/\underline{Dann greif eben halt härter durch!}}\\
\cline{1-2}
\emph{halt eben}& \emph{eben halt} & {}\\
\cline{1-2}
30 & 2 & {}\\
\hline
\end{tabular}
\end{exe}
Auch hier wurde (wie bei Test 1) ein log-lineares gemischtes Modell gerechnet mit Abfolge als abhängiger Variable und Teilnehmern und Items als Zufallsvariablen (N = 188, log-Likelihood = $\minus$39,97). Das signifikante Interzept ($\beta$ = 3,931, SE = 0,561, Wald z = 7,00, p $<$ 0,001) zeigt, dass signifikant häufiger \textit{halt eben} als \textit{eben halt} verwendet wird.

Es zeigt sich, dass sich die Ergebnisse beider Experimente völlig konform zu den Korpusdaten verhalten: Die Abfolge \textit{halt eben} wird der Anordnung \textit{eben halt} gegenüber sowohl in Assertionen als auch Direktiven deutlich präferiert. Da beide Satzmodi, in denen \textit{halt} und \textit{eben} prinzipiell kombiniert auftreten können, in Bezug auf die präferierte Anordnung der beiden MPn gestestet wurden und beide Tests das gleiche Ergebnis produzieren, lässt sich folglich keine Evidenz für die in Abschnitt~\ref{sec:anlit} genannten Annahmen aus der Literatur anführen, dass der Satzmodus Einfluss auf die Abfolge der beiden MPn nimmt. D.h. weder die Annahme aus \citet[1542-1543]{Zifonun1997}, \textit{eben halt} stehe im Aussage- und \textit{halt eben} im Aufforderungsmodus, noch der Eindruck aus \citet[230]{Dahl1988}, dass in Imperativen die Wahl zwischen \textit{halt eben} und \textit{eben halt} bestehe, lässt sich nachweisen.

In einem weiteren Experiment (Experiment 4)\footnote{Ich beziehe mich mit der Nummerierung der Experimente auf ihren Zeitpunkt der Durchführung. Deshalb ist das in Abschnitt~\ref{sec:stressclash} beschriebene Experiment Nr. 3.} haben 64 deutsche Muttersprachler je sechs \textit{halt eben}- vs. \textit{eben halt}-Assertionen und -Direktive im glei\-chen Design bewertet (plus 36 Filler). Wenngleich die separate Betrachtung von Assertionen und Direktiven nicht für einen Einfluss des Satzmodus zu sprechen scheint, lässt sich dies aus den beiden Studien als Ergebnis nur schwer ableiten, da jeweils andere Sprecher die beiden Satzkontexte bewertet haben, d.h. dieser Aspekt ging nicht als Bedingung in die Tests ein. Dies ist aber in diesem weiteren Experiment der Fall.

Die Items, Strukturierung der Testbögen und die ausbalancierte Verteilung der Antworten auf die Möglichkeiten a) und b) in den Items und Fillern wurde aus den ersten beiden Tests übernommen (s.o.). Es gab acht verschiedene Testzusammensetzungen. Von
diesen Versionen wurden vier zusätzlich in zwei Randomisie\-rungen dargeboten (in Bezug
auf die Anordnung der Blöcke und Items innerhalb eines Blockes). Insgesamt lagen somit
12 Varianten vor. Bei den Testteilnehmern handelt es sich um Germanistik-Studierende der Universität Bielefeld im SoSe 2015, die gegenüber dem Phänomen naiv waren. Ihr Durchschnittsalter betrug 22 Jahre (von 19 bis 33) und auch die regionale Herkunft entsprach der Zusammensetzung aus den ersten beiden Studien: 81\% sind in Bereich C der Eichhoff'schen Einteilung zur Schule gegangen, 23\% in D (davon 60\% nördliches D), je ein Sprecher stammt aus E und B. Ein Sprecher hat lediglich die Angabe \glqq NRW\grqq{} gemacht. (\ref{637}) zeigt die Ergebnisse in der Übersicht.

\begin{exe}
        \ex\label{637} Häufigkeiten \textit{halt eben} vs. \textit{eben halt} in Assertionen und Direktiven\\[-1em]
    \begin{tabular}[t]{|l|l|l|l|}
    \hline
    \multicolumn{2}{|l|}{\textbf{Assertionen}} & \multicolumn{2}{l|}{\textbf{Direktive}}\\
    \hline
    Maurer & 58:6 & fahr später ab & 58:6\\
    \hline
    Brite & 55:9 & pass besser auf & 58:6\\
    \hline
    Moslem & 55:9 & geh öfter hin & 55:9\\
    \hline
    Waage & 57:7 & schlag schneller zu & 56:8\\
    \hline
    Schwabe & 59:5 & lauf früher los & 58:6\\
    \hline
    Kölner & 56:7 & greif härter durch & 56:8\\
    \hline
    \end{tabular}
\end{exe}
Es wurde eine log-lineares gemischtes Modell gerechnet (\citealt{Baayen2008}) mit Abfolge als abhängiger Variable, Satzmodus als Effekt-kodierte unabhängige Variable und Teilnehmern und Items als Zufallsvariablen mit dem Faktor Satzmodus in der Steigung (\citealt{Barr2013}) (N = 767, log-Likelihood = $\minus$227.9). Das signifikante Interzept ($\beta$ = 3,227, SE = 0,304, Wald z = 10,61, p $<$ 0,001) zeigt, dass signifikant häufiger \textit{halt eben} als \textit{eben halt} verwendet wird. Es findet sich keine Evidenz für einen Einfluss des Satzmodus ($\beta$ = 0,109, SE = 0,143, Wald z = 0,76, p = 0,45).

Wenngleich ich mit diesen zwei Experimenten keine dialektale Umfrage beabsichtige, kann ich – genauso wie im Zuge der Betrachtung von Korpusdaten – auch hier keinen Hinweis darauf finden, dass die Wahl der einen oder anderen Reihung dialektal bedingt ist. Unter der logischen Weiterführung der unspezifischen Aussage von \citet{Elspass2005} müssten Sprecher derjenigen Gebiete, in denen \textit{eben} überwog, zur Abfolge \textit{eben halt} tendieren (Nord- und Ostdeutschland), während Sprecher aus Gegenden, in denen \textit{halt} gebräuchlicher war, sich für \textit{halt eben} entscheiden müssten (Süddeutschland). Die befragten Testanten haben die längste Zeit ihres Lebens in einem Teil Deutschlands verbracht, der der Karte aus \citet{Eichhoff1978} zufolge zum \textit{eben}-Gebiet gehört (C, nördliches D).

Diese Sprecher entscheiden sich aber bei der direkten Wahl zwischen \textit{halt eben} und \textit{eben halt} sehr deutlich für \textit{halt eben}, was – wenn an der dialektalen Verteilung der Kombination etwas dran wäre – dem genau entgegengesetzten Verhalten entspräche.\\

\noindent
Auf der Basis der Korpusbetrachtung sowie dieser drei Befragungen gehe ich deshalb davon aus, dass im heutigen Deutsch beide Abfolgen von \textit{halt} und \textit{eben} in Kombination existieren und verwendet werden. In den Korpusdaten tritt \textit{eben halt} auch in großen Zahlen und nicht nur sporadisch auf. Die Häufigkeiten für die Sequenz \textit{halt eben} übersteigen die Anzahl der \textit{eben halt}-Auftretensweisen deutlich. Testanten präferieren die Kombination \textit{halt eben} sowohl in Assertionen als auch Direktiven im direkten Vergleich mit \textit{eben halt} (in Kontexten, in denen beide MPn in Isolation gleichermaßen stehen können) klar, wobei kein Einfluss des Satzkontextes festzustellen ist. Ich gehe deshalb in meiner weiteren Argumentation davon aus, dass es zwischen den beiden Abfolgen einen Markiertheits\-unterschied gibt. Die Abfolge \textit{halt eben} ist dabei die unmarkierte und \textit{eben halt} die markierte Reihung.

Die Frage, die ich in den folgenden Abschnitten deshalb zu beantworten beabsichtige, ist: Wie motiviert sich dieser Markiertheitsunterschied, d.h. was macht \textit{halt eben} zu der unmarkierten und \textit{eben halt} zu der markierten Ordnung? Wie schon in Kapitel~\ref{chapter:jud} für die Kombinationen aus \textit{ja} und \textit{doch} angenommen, leite ich auch hier den Markiertheitsunterschied aus der diskursiven Interpretation der ein oder anderen Sequenz heraus ab.

\section{Interpretation der Kombination}
\label{sec:interpretationkombi}
Voraussetzung für eine Ableitung des Markiertheitsunterschiedes zwischen \textit{halt eben} und \textit{eben halt}, die auf die Interpretation der MP-Äußerung Bezug nimmt, ist die Klärung der Frage, wie eine Äußerung, in der sowohl \textit{halt} als auch \textit{eben} auftreten, überhaupt zu interpretieren ist (vgl. auch \citealt[159-161]{Mueller2016a}; \citeyear[169-172]{Mueller2016b}; \citeyear[243-244]{Mueller2017a}). Die zentrale Frage ist dabei, ob/wie die Skopoi, die die MPn jeweils über die beteiligte Proposition nehmen, miteinander interagieren. Angenommen, man sagt, dass die Einzelpartikeln in (\ref{638}) Skopus über die Proposition p (= dass Peter krank ist) nehmen, ergeben sich für das kombinierte Auftreten der beiden MPn in (\ref{639}) die vier möglichen Skopusverhältnisse in (\ref{640}) und (\ref{641}).

\begin{exe}
	\ex\label{638} 
		\begin{xlist}	
			\ex\label{638a} Peter ist \textbf{halt} krank. $[$halt(p)$]$
			\ex\label{638b} Peter ist \textbf{eben} krank. $[$eben(p)$]$
		\end{xlist}
\end{exe}

\begin{exe}
\ex\label{639}
	Peter ist \textbf{halt eben}/\textbf{eben halt} krank.
\end{exe}

\begin{exe}
	\ex\label{640} Verschiedener Skopus\\[-1em]
		\begin{xlist}	
			\ex\label{640a} halt(eben(p))
			\ex\label{640b} eben(halt(p))
		\end{xlist}
\end{exe}

\begin{exe}
	\ex\label{641} Gleicher Skopus\\[-1em]
		\begin{xlist}	
			\ex\label{641a} halt(p) \& eben(p)
			\ex\label{641b} eben(p) \& halt(p)
		\end{xlist}
\end{exe}
In diese offene Debatte lässt sich nur Klärung bringen, indem man eine Beschreibung der Einzelpartikeln vorschlägt, die es erlaubt, die Möglichkeiten in (\ref{639}) und (\ref{640}) gleichermaßen abzubilden. 

(\ref{642}) bis (\ref{647}) zeigen die Füllungen der Diskurskomponenten in \textit{eben}-/\textit{halt}-Äußerungen, wie ich für sie in Abschnitt~\ref{sec:kontexte} argumentiert habe.

\begin{exe}
	\ex\label{642} Kontext vor der \textit{eben}-Assertion (Begründung)\\[-1em]	
			\begin{tabular}[t]{|C{6em}|C{12em}|C{6em}|}
			\hline
			$\textrm{DC}_{\textrm{A}}$ & Tisch &  $\textrm{DC}_{\textrm{B}}$ \tabularnewline
			\hline
			{} & {} & p \tabularnewline
			(q) & {} & (q)  \tabularnewline
			\hline
			\multicolumn{3}{|l|}{cg s$_{1}$ = $\lbrace$p $>$ q$\rbrace$}		
			\tabularnewline
			\hline
			\end{tabular}	
\end{exe}
\pagebreak
\begin{exe}
	\ex\label{643} Kontext vor der \textit{eben}-Assertion (Folge)\\[-1em]	
			\begin{tabular}[t]{|C{6em}|C{12em}|C{6em}|}
			\hline
			$\textrm{DC}_{\textrm{A}}$ & Tisch &  $\textrm{DC}_{\textrm{B}}$ \tabularnewline
			\hline
			{}  & {} & q  \tabularnewline
			{} & {} & p \tabularnewline
			\hline
			\multicolumn{3}{|l|}{cg s$_{1}$ = $\lbrace$q $>$ p$\rbrace$}		
			\tabularnewline
			\hline
			\end{tabular}	
\end{exe}

\newcolumntype{C}[1]{>{\centering}p{#1}}
\begin{exe}
\ex\label{644} Kontext vor dem \textit{eben}-Direktiv (Folge)\\[-1em]
\begin{tabular}[t]{|C{6em}|C{12em}|C{6em}|}
\hline
$\textrm{DC}_{\textrm{A}}$ & Tisch &  $\textrm{DC}_{\textrm{B}}$ \tabularnewline
\hline
{} & {} & q  \tabularnewline
\cline{1-1}\cline{3-3}
$\textrm{TDL}_{\textrm{A}}$ & {} & $\textrm{TDL}_{\textrm{B}}$  \tabularnewline
\cline{1-1}\cline{3-3}
{} & {} & !p  \tabularnewline
\hline
\multicolumn{3}{|l|}{cg s$_{1}$ = $\lbrace$q $>$ !p$\rbrace$} \tabularnewline
\hline
\end{tabular}
\end{exe}

\newcolumntype{C}[1]{>{\centering}p{#1}}
\begin{exe}
	\ex\label{645} Kontext vor der \textit{halt}-Assertion (Begründung)\\[-1em]
 		\begin{tabular}[t]{|C{6em}|C{6em}|C{6em}|} 
 		\hline 	
   		$\textrm{DC}_{\textrm{A}}$ & {Tisch} & $\textrm{DC}_{\textrm{B}}$ \tabularnewline
  		\hline
  		{p $>$ q} & {} & {} \tabularnewline
  		(q) & {} & (q) \tabularnewline
  		\hline      
   		\multicolumn{3}{|l|}{cg s$_{1}$} \tabularnewline   
  		 \hline
 		\end{tabular}
\end{exe}

\newcolumntype{C}[1]{>{\centering}p{#1}}
\begin{exe}
	\ex\label{646} Kontext vor der \textit{halt}-Assertion (Folge)\\[-1em]
 		\begin{tabular}[t]{|C{6em}|C{6em}|C{6em}|} 
 		\hline 	
   		$\textrm{DC}_{\textrm{A}}$ & {Tisch} & $\textrm{DC}_{\textrm{B}}$ \tabularnewline
  		\hline
   		{q $>$ p} & {} & q \tabularnewline
  		\hline      
   		\multicolumn{3}{|l|}{cg s$_{1}$} \tabularnewline   
  		 \hline
 		\end{tabular}
\end{exe}

\newcolumntype{C}[1]{>{\centering}p{#1}}
	\begin{exe}
	\ex\label{647} Kontext vor dem \textit{halt}-Direktiv (Folge)\\[-1em]
	\begin{tabular}[t]{|C{6em}|C{12em}|C{6em}|}
	\hline
	$\textrm{DC}_{\textrm{A}}$ & Tisch &  $\textrm{DC}_{\textrm{B}}$ \tabularnewline
	\hline
	q $>$ !p & {} & q  \tabularnewline
	\cline{1-1}\cline{3-3}
	$\textrm{TDL}_{\textrm{A}}$ & {} & $\textrm{TDL}_{\textrm{B}}$  \tabularnewline
	\cline{1-1}\cline{3-3}
	{} & {} & {}  \tabularnewline
	\hline
	\multicolumn{3}{|l|}{cg s$_{1}$} \tabularnewline
	\hline
	\end{tabular}
	\end{exe}
Basierend auf den Beschreibungen in (\ref{642}) bis (\ref{644}) und (\ref{645}) bis (\ref{647}) für die Einzelpartikeln ergeben sich für die vier Möglichkeiten aus (\ref{640}) und (\ref{641}) die vier konkreten Modellierungen in (\ref{648}) bis (\ref{650}).

Nimmt \textit{halt} in der Kombination aus \textit{halt} und \textit{eben} Skopus über \textit{eben}, resultieren (\ref{648a}), (\ref{648b}) bzw. (\ref{648c}).
\pagebreak
\begin{exe}
	\ex\label{648} \textit{halt} $>$ \textit{eben}: Kontextzustand vor der Äußerung\\[-1em]	
	\begin{xlist}
		\ex\label{648a} Assertion (Begründung) (halt(eben(p))\\[-1em]
			\begin{tabular}[t]{|C{8em}|C{8em}|C{8em}|}
			\hline
			$\textrm{DC}_{\textrm{A}}$ & Tisch &  $\textrm{DC}_{\textrm{B}}$ \tabularnewline
			\hline
			(q $\in$ $\textrm{DC}_{\textrm{A/B}}$ \& p $\in$ $\textrm{DC}_{\textrm{B}}$ \& cg = $\lbrace$p $>$ q$\rbrace$) $>$ 				q & {} & {} \tabularnewline
			{} & {} & {} \tabularnewline	
			(q) & {} & (q)  \tabularnewline			
			\hline
			\multicolumn{3}{|l|}{cg s$_{1}$}		
			\tabularnewline
			\hline
			\end{tabular}	
		\ex\label{648b} Assertion (Folge) (halt(eben(p))\\[-1em]
			\begin{tabular}[t]{|C{8em}|C{8em}|C{8em}|}
			\hline
			$\textrm{DC}_{\textrm{A}}$ & Tisch &  $\textrm{DC}_{\textrm{B}}$ \tabularnewline
			\hline
			q $>$ (cg = $\lbrace$q $>$ p$\rbrace$ \& q $\in$ $\textrm{DC}_{\textrm{B}}$ \& p $\in$ $\textrm{DC}_{\textrm{B}}				$) & {} & q
			\tabularnewline		
			\hline
			\multicolumn{3}{|l|}{cg s$_{1}$}		
			\tabularnewline
			\hline
			\end{tabular}	
			
		\ex\label{648c} Direktiv (halt(eben(p))\\[-1em]
		\begin{tabular}[t]{|C{8em}|C{8em}|C{8em}|}
		\hline
		$\textrm{DC}_{\textrm{A}}$ & Tisch &  $\textrm{DC}_{\textrm{B}}$ \tabularnewline
		\hline
		q $>$ !(cg = $\lbrace$q $>$ !p$\rbrace$ \& !p $\in$ $\textrm{TDL}_{\textrm{B}}$ \& q $\in$ $\textrm{DC}_{\textrm{B}}$) & 		{} & q \tabularnewline
		\cline{1-1}\cline{3-3}
		$\textrm{TDL}_{\textrm{A}}$ & {} & $\textrm{TDL}_{\textrm{B}}$  \tabularnewline
		\cline{1-1}\cline{3-3}
		{} & {} & {} \tabularnewline
		\hline
		\multicolumn{3}{|l|}{cg s$_{1}$} \tabularnewline
		\hline
		\end{tabular}			
\end{xlist}			
\end{exe}
Unter dieser Interpretation macht der Output von eben(p) den Input für halt(p) aus. D.h. erst appliziert \textit{eben} auf p und auf diesem Objekt appliziert \textit{halt}.

Die umgekehrten Skopusverhältnisse liegen in (\ref{649}) vor.

\begin{exe}
	\ex\label{649} \textit{eben} $>$ \textit{halt}: Kontextzustand vor der Äußerung\\[-1em]	
	\begin{xlist}
		\ex\label{649a} Assertion (Begründung) (eben(halt(p))\\[-1em]
			\begin{tabular}[t]{|C{8em}|C{8em}|C{8em}|}
			\hline
			$\textrm{DC}_{\textrm{A}}$ & Tisch &  $\textrm{DC}_{\textrm{B}}$ \tabularnewline
			\hline
			(q) & {} & (q) \tabularnewline
			{} & {} & q $\in$ $\textrm{DC}_{\textrm{A/B}}$ \tabularnewline
			{} & {} & (p $>$ q) $\in$ $\textrm{DC}_{\textrm{A}}$ \tabularnewline	
			\hline
			\multicolumn{3}{|l|}{cg s$_{1}$ = $\lbrace$((p $>$ q) $\in$ $\textrm{DC}_{\textrm{A}}$ \& q $\in$ $\textrm{DC}_{\textrm{A/B}}$) $>$ q$\rbrace$}		
			\tabularnewline
			\hline
			\end{tabular}	
\pagebreak			
		\ex\label{649b} Assertion (Folge) (eben(halt(p))\\[-1em]
			\begin{tabular}[t]{|C{8em}|C{8em}|C{8em}|}
			\hline
			$\textrm{DC}_{\textrm{A}}$ & Tisch &  $\textrm{DC}_{\textrm{B}}$ \tabularnewline
			\hline
			{} & {} & q \tabularnewline
			{} & {} & (q $>$ p) $\in$ $\textrm{DC}_{\textrm{A}}$ \tabularnewline
			{} & {} & q $\in$ $\textrm{DC}_{\textrm{B}}$ \tabularnewline				
			\hline
			\multicolumn{3}{|l|}{cg s$_{1}$ = $\lbrace$q $>$ ((q $>$ p) $\in$ $\textrm{DC}_{\textrm{A}}$ \& q $\in$ $						\textrm{DC}_{\textrm{B}}$}		
			\tabularnewline
			\hline
			\end{tabular}	
			
		\ex\label{649c} Direktiv (eben(halt(p))\\[-1em]
			\begin{tabular}[t]{|C{8em}|C{8em}|C{8em}|}
			\hline
			$\textrm{DC}_{\textrm{A}}$ & Tisch &  $\textrm{DC}_{\textrm{B}}$ \tabularnewline
			\hline
			{} & {} & q
			\tabularnewline
			\cline{1-1}\cline{3-3}
			$\textrm{TDL}_{\textrm{A}}$ & {} & $\textrm{TDL}_{\textrm{B}}$  \tabularnewline
			\cline{1-1}\cline{3-3}
			{} & {} & !((q $>$ !p) $\in$ $\textrm{DC}_{\textrm{A}}$ \& q $\in$ $\textrm{DC}_{\textrm{B}}$)		
			 \tabularnewline
			\hline
			\multicolumn{3}{|l|}{cg s$_{1}$ = $\lbrace$q $>$ !((q $>$ !p) $\in$ $\textrm{DC}_{\textrm{A}}$ \& q $\in$ $						\textrm{DC}_{\textrm{B}}$} \tabularnewline
			\hline
			\end{tabular}			
	\end{xlist}			
	\end{exe}
Nimmt \textit{eben} über \textit{halt} Skopus, handelt es sich bei halt(p) um die Eingabe für eben(p). 

Die Alternative zu der Interpretation, unter der die MPn in einem asymmetri\-schen Skopusverhältnis zueinander stehen, ist, dass sie den gleichen Skopus aufweisen. D.h. sie beziehen sich beide auf die in der Äußerung enthaltene Proposition. In diesem Fall resultieren die Verhältnisse in (\ref{650a}) bis (\ref{650c}).

\begin{exe}
	\ex\label{650} 1. halt \& 2. eben bzw. 1. eben \& 2. halt: Kontextzustand vor der Äußerung\\[-1em]	
	\begin{xlist}
		\ex\label{650a} Assertion (Begründung) (halt(p) \& eben(p)/eben(p) \& halt(p))\\[-1em]
			\begin{tabular}[t]{|C{8em}|C{8em}|C{8em}|}
			\hline
			$\textrm{DC}_{\textrm{A}}$ & Tisch &  $\textrm{DC}_{\textrm{B}}$ \tabularnewline
			\hline
			p $>$ q & {} & {} \tabularnewline
			(q) & {} & (q) \tabularnewline
			{} & & p \tabularnewline	
			\hline
			\multicolumn{3}{|l|}{cg s$_{1}$ = $\lbrace$p $>$ q$\rbrace$}		
			\tabularnewline
			\hline
			\end{tabular}	
			
		\ex\label{650b} Assertion (Folge) (halt(p) \& eben(p)/eben(p) \& halt(p))\\[-1em]
			\begin{tabular}[t]{|C{8em}|C{8em}|C{8em}|}
			\hline
			$\textrm{DC}_{\textrm{A}}$ & Tisch &  $\textrm{DC}_{\textrm{B}}$ \tabularnewline
			\hline
			q $>$ p & {} & q \tabularnewline
			{} & {} & p \tabularnewline
			\hline
			\multicolumn{3}{|l|}{cg s$_{1}$ = $\lbrace$p $>$ q$\rbrace$}		
			\tabularnewline
			\hline
			\end{tabular}	
\pagebreak			
		\ex\label{650c} Direktiv (halt(p) \& eben(p)/eben(p) \& halt(p)) \\[-1em]
			\begin{tabular}[t]{|C{8em}|C{8em}|C{8em}|}
			\hline
			$\textrm{DC}_{\textrm{A}}$ & Tisch &  $\textrm{DC}_{\textrm{B}}$ \tabularnewline
			\hline
			q $>$ !p & {} & {} \tabularnewline
			{} & {} & q \tabularnewline
			\cline{1-1}\cline{3-3}
			$\textrm{TDL}_{\textrm{A}}$ & {} & $\textrm{TDL}_{\textrm{B}}$  \tabularnewline
			\cline{1-1}\cline{3-3}
			{} & {} & !p	\tabularnewline
			\hline
			\multicolumn{3}{|l|}{cg s$_{1}$ = $\lbrace$q $>$ !p $\rbrace$} \tabularnewline
			\hline
			\end{tabular}			
	\end{xlist}			
	\end{exe}
Die Bedeutung ergibt sich in (\ref{650}) jeweils aus der additiven Verknüpfung des Bedeutungsbeitrags der Einzelpartikeln.

Wenn es sich bei (\ref{648}) bis (\ref{650}) um die vier möglichen Skopusverläufe in einer MP-Zweierkombination handelt, die auf der Basis der Einzelbeschreibungen von \textit{halt} und \textit{eben} in (\ref{642}) bis (\ref{647}) konkret ausbuchstabiert werden, stellt sich die Frage, welche dieser prinzipiellen Möglichkeiten für die Interpretation der Kombination aus \textit{halt} und \textit{eben} anzusetzen ist.

Ein prinzipielles Argument gegen eine der beiden Skopusinterpretationen ist, dass – wie wir gesehen haben – beide Abfolgen in relevanter Zahl belegt sind. (\ref{651}) bis (\ref{654}) zeigt erneut einige Beispiele.

\begin{exe}
	\ex\label{651} 
	Es kommen soooooo viele alte schöne Klinkerfassaden hinter dem alten DDR Putz 
	zum Vorschein, warum bei diesen Bauten nicht? Na, \textbf{weil diese Gebäude \underline{eben halt} keiner verputzt hat.}	
	\hfill\hbox {(DECOW2012-00: 13893583)}
\end{exe}	          	                              

\begin{exe}
	\ex\label{652} 
	Trotzdem soll es den Frühlings- und den Herbstmarkt auch in Zukunft geben, waren sich Lach und Matthews einig. 					\textbf{Entscheidend sei \underline{eben halt} das Wetter}, und deshalb gelte das Prinzip Hoffnung.      
	\newline
	\hbox{}\hfill\hbox {(Braunschweiger Zeitung, 08.04.2006)}
\end{exe}			

\begin{exe}
	\ex\label{653} 
	Mir hat sie gefallen und ich glaube auch, \textbf{dass das Outfit da \underline{halt eben} kein \glqq Bühnenoutfit\grqq{} war}, ganz einfach.  
	\hfill\hbox {(DECOW2012-02: 408756250)}
\end{exe}

\begin{exe}
	\ex\label{654} 
	Eishockey. So brutal kann Sport sein, oder wie sich der Uzwiler Coach Roger Bader ausdrückte:  \textbf{So ist 					\underline{halt eben} der Sport.}    
	\newline
	\hbox{}\hfill\hbox {(St. Galler Tagblatt, 20.10.2008)}
\end{exe}	                             
Möchte man für diese Interpretation der Kombination eine Skopuslesart ansetzen und dazu davon ausgehen (wie ich es in dieser Arbeit verfolge), dass die Interpretation der MP-Kombination mit der Abfolge der MPn in der Kombination zusammenhängt, müsste man plausiblerweise davon ausgehen, dass die zwei Ordnungen von \textit{halt} und \textit{eben} jeweils mit einer anderen Skopusinterpretation einhergehen. M.E. ist für die situativ angemessene Äußerung von (\ref{651}) bis (\ref{654}) aber nicht von verschiedenen Vorgängerkontexten auszugehen, wie sie durch (\ref{640}) bzw. (\ref{641}) im Rahmen meiner Modellierung beschrieben werden. Es scheint mir nicht der Fall zu sein, dass für \textit{halt eben}- und \textit{eben halt}-Äußerungen unterschiedliche Bedeutungen im Sinne von verschiedenen Skopusverhältnissen anzunehmen sind. Da sich unter anderem Skopus die Bezüge aber verändern, wäre dies eine Konsequenz aus der Zuschreibung der Skopusbedeutung. 

Betrachtet man die beiden Skopuslesarten im Detail, gewinnt man den Eindruck, dass diese im Falle der Direktive zu nahezu unsinnigen Interpretationen führen. 
	                    
Es fällt schwer, eine Diskurssituation zu konstruieren, in der der Sprecher annimmt, dass aus q normalerweise abzuleiten ist, dass der Adressat realisieren soll, dass q $>$ !p cg ist und dass !p eine Absicht des Hörers ist und dass q unter den Bekenntnissen des Hörers ist (vgl. (\ref{648c})).

Es mutet ebenfalls etwas merkwürdig an, dass für den Diskurspartner u.a. die Handlung aussteht, q $>$ !p zu einem Diskursbekenntnis des Sprechers zu machen (vgl. (\ref{649c})).

Auch frage ich mich, welches Szenario dazu führen soll, dass A annimmt, dass aus q normalerweise folgt, dass q $>$ p cg ist und q und p unter Bs Bekenntnissen sind (vgl. (\ref{650b})), oder dass Einigkeit besteht, dass aus q normalerweise folgt, dass die Inferenzrelation unter As Bekenntnissen ist und B von q ausgeht (vgl. (\ref{649b})).

Im Falle der begründenden Assertionen scheint mir die Interpretation, in der das \textit{halt} das \textit{eben} in seinen Skopus nimmt, zwar nicht zutreffend, aber doch weniger abwegig als die soeben angeführten Lesarten. Man könnte sich prinzi\-piell vorstellen, dass der Sprecher annimmt, dass wenn p $>$ q cg ist und der Hörer von p und q ausgeht, er auch normalerweise von q ausgeht (vgl. (\ref{648a})). Im monologischen Fall (wenn der Sprecher selbst von q ausgeht), bestünde die Relation trivialerweise.

Die durch (\ref{649a}) beschriebene Situation scheint allerdings ebenfalls wenig plausibel. Der Diskurspartner kann natürlich annehmen, dass der Sprecher einen Zusammenhang zwischen p und q sieht. Ich frage mich aber, aufgrund welcher Verhältnisse der cg eine Relation enthalten soll zwischen der Annahme des Sprechers, dass p und q zusammenhängen, und q.

Kurz gefasst, ich halte die konkreten Interpretationen aus (\ref{648}) und (\ref{649}) (neben dem prinzipiellen Argument des Auftretens beider Abfolgen) nicht für zutreffend. Da ich zudem nicht davon ausgehe, dass sich die Interpretation der MP-Kombinationen je nach Illokutionstyp unterscheidet (und deshalb begründende Assertionen das Skopusverhältnis widerspiegeln könnten, während auf Direktive die additive Lesart zutrifft), scheiden alle vier Skopuslesarten aus. Per Aus\-schlussverfahren würde nur die additive Lesart übrig bleiben. M.E. lässt sich für diesen Argumentationsschritt auch  positive Evidenz anführen.

Die Beschreibungen von Autoren hinsichtlich ihrer Auffassung zum Verständnis der Kombination scheint mir deutlich für die Interpretation unter gleichem Skopus zu sprechen. \citet[257]{Thurmair1989} schreibt beispielsweise: \glqq In dieser Kombination wird durch \textit{halt} der durch \textit{eben} angezeigte kategorische Charakter der Aussage zurückgenommen.\grqq{} Wenn \textit{halt} die axiomatische Wirkung von \textit{eben} entkräftet, sollten beide Partikeln den gleichen Bezugsbereich aufweisen. Der Einschätzung von Thurmair recht nahe kommt der Eindruck der Interpretation der Kombination von \citet[226]{Dittmar2000}. Er schreibt, dass \glqq das an \textit{eben} angehängte \textit{halt} $[$...$]$ offenbar die Härte von eben ab$[$schwächt$]$\grqq{}. Der Autor äußert sich nicht explizit zur \is{Skopus} Skopusfrage. Seine Beschreibung lässt sich aber vor dem Hintergrund der Möglichkeiten nur auffangen, wenn man davon ausgeht, dass die beiden MPn den gleichen Bezugsbereich haben. Andere konkrete Aussagen zur Interpretation dieser MP-Kombination sind meines Wissens in der Literatur nicht gemacht worden.

Unter Bezug auf diese beiden Zitate schließe ich mich für den folgenden Verlauf meiner Argumentation Thurmair an, die annimmt, dass die beiden MPn in der Sequenz den Beitrag leisten, den sie auch in Isolation jeweils alleine besteuern.  D.h. ich gehe von einer additiven Verknüpfung entlang von (\ref{650}) aus.

Treten in einer Äußerung sowohl \textit{halt} als auch \textit{eben} auf, wird folglich sowohl Plausibilität als auch Kategorizität ausgedrückt. Der Sprecher bringt zum Ausdruck, dass er selbst q annimmt (bzw. p/!p) aufgrund des auf seiner Seite vorausgesetzten Zusammenhangs zwischen p und q bzw. q und p/!p. Ebenfalls zeigt er an, dass er den Inhalt der MP-Äußerung als bekannt voraussetzt, genauso wie den Zusammenhang zwischen p und q bzw. q und p/!p. Sowohl er als auch beide Diskurspartner vertreten seines Erachtens nach p, p $>$ q/q $>$ p/!p und somit auch q, p bzw. !p. 

Meiner Argumentation nach gibt es im Sinne der Skopusverhältnisse keinen Bedeutungsunterschied zwischen den beiden Abfolgen \textit{halt eben} und \textit{eben halt}. Die Frage, die es im Folgenden zu klären gilt, ist deshalb, wie bei prinzipiell glei\-cher Bedeutung der beiden Abfolgen der Markiertheitsunterschied mit dem unmarkierten \textit{halt eben} und dem markierten \textit{eben halt} zustandekommt (vgl. auch \citealt[162-164]{Mueller2016a}; \citeyear[169-177]{Mueller2016b}; \citeyear[244-248]{Mueller2017a}).

\section{Erklärung der (un)markierten Abfolge}
\label{sec:erklärunghe}
\subsection{Implikation}
\label{sec:impli}
In Abschnitt~\ref{sec:untersch} habe ich mit \citet{Thurmair1989} illustriert, dass es Kontexte gibt, in denen \textit{eben} situativ weniger angemessen ist als \textit{halt}. Hingegen lassen sich nur sehr wenige Kontexte finden, in denen das umgekehrte Verhältnis vorliegt. 

Es lässt sich folglich annehmen, dass man \textit{eben} in der Regel durch \textit{halt} ersetzen kann, andersherum \textit{halt} aber nicht in jedem Kontext durch \textit{eben} (vgl. \citealt[128]{Thurmair1989}, \citealt[392]{Ickler1994}).

Wie in Abschnitt~\ref{sec:bedhe} erläutert, geht Thurmair von der folgenden Bedeutungszu\-weisung aus: \textit{eben} zeigt ihr zufolge Evidenz, \textit{halt} Plausibilität an. Die zu beobachtenden Ersetzbarkeiten erklärt sie, indem sie annimmt, dass Evidentes stets auch plausibel ist. Aufgrund dessen lässt sich \textit{eben} ($\rightarrow$ evident) immer auch durch \textit{halt} ($\rightarrow$ plausibel) austauschen. Das, was als plausibel ausgegeben wird, muss aber nicht notwendigerweise auch evident sein. Deshalb ist es nicht stets möglich, \textit{halt} durch \textit{eben} zu ersetzen. D.h. die Bedeutung von \textit{eben} schließt die Bedeutung von \textit{halt} ein, die Bedeutung von \textit{halt} schließt aber nicht die Bedeutung von \textit{eben} ein. Es besteht also zwischen den Bedeutungen von \textit{eben} und \textit{halt} ein Im\-plikationsverhältnis \is{Implikation}: Die Bedeutung von \textit{eben} impliziert die Bedeutung von \textit{halt}. Dieses Verhältnis, das Thurmair auf der Basis ihrer deskriptiven Erfassung der MPn \textit{eben} und \textit{halt} beschreibt, sollte aus allen Bedeutungsbeschreibungen der beiden Partikeln resultieren. 

Wie ich im Folgenden zeigen werde, bildet auch meine Modellierung im formalen Diskursmodell nach \citet{Farkas2010} dieses Verhältnis ab. Dazu möchte ich betrachten, in welchen Komponenten p/!p und p $>$ q bzw. q $>$ p/!p vor und nach einer \textit{halt}- bzw. \textit{eben}-Äußerung enthalten sind.  

Bei einer \textit{halt}-Äußerung ist p $>$ q (bzw. q $>$ p/!p) Teil von DC$_{\textrm{A}}$. Nach der \textit{halt}-Äußerung ist p in DC$_{\textrm{A}}$ (bzw. !p in TDL$_{\textrm{B}}$) (vgl. (\ref{655})).
\pagebreak
\begin{exe}
        \ex\label{655} \textit{halt}\\[-0.4em]
    \begin{tabular}[t]{lll}
    & \underline{vor der MP-Äußerung} & \underline{nach der MP-Äußerung}\\
    a. & Assertion - Begründung & {}\\
    {} & i. p $>$ q in DC$_{\textrm{A}}$ & i. p in DC$_{\textrm{A}}$\\
	{} & ii. q in DC$_{\textrm{A/B}}$ & ii. q ggf. in cg\\
	{} & {} & {}\\
	b. & Assertion - Folge & {}\\
	{} & i. q $>$ p in DC$_{\textrm{A}}$ & i. p in DC$_{\textrm{A}}$\\
	{} & ii. q in DC$_{\textrm{B}}$ & ii. q in cg\\
	{} & {} & {}\\
	c. & Direktiv & {}\\
	{} & i. q $>$ !p in DC$_{\textrm{A}}$ & a. !p in TDL$_{\textrm{B}}$\\
	{} & ii. q in DC$_{\textrm{B}}$ & ii. q in cg\\
    \end{tabular}
\end{exe}
Für eine \textit{eben}-Äußerung habe ich in Abschnitt~\ref{sec:kontexte} angenommen, dass entscheidenderweise im Kontextzustand vor der Äußerung p $>$ q (bzw. q $>$ p/!p) im cg ist und dass p in DC$_{\textrm{B}}$ (bzw. !p in TDL$_{\textrm{B}}$) enthalten ist. Nach der \textit{eben}-Äußerung ist p im cg (bzw. !p in TDL$_{\textrm{B}}$).

\begin{exe}
        \ex\label{656} \textit{eben}\\[-0.5em]
    \begin{tabular}[t]{lll}
    & \underline{vor der MP-Äußerung} & \underline{nach der MP-Äußerung}\\
    a. & Assertion - Begründung & {}\\
    {} & i. p $>$ q in cg & i. p in cg\\
	{} & ii. p in DC$_{\textrm{B}}$ & ii. q in cg\\
	{} & iii. q in DC$_{\textrm{A/B}}$ & {}\\	
	{} & {} & {}\\
	b. & Assertion - Folge & {}\\
	{} & i. q $>$ p in cg & i. q in cg\\
	{} & ii. q in DC$_{\textrm{B}}$ & ii. p in cg\\
	{} & p in DC$_{\textrm{B}}$ & {}\\
	{} & {} & {}\\
	c. & Direktiv & {}\\
	{} & i. q $>$ !p in cg & q in cg\\
	{} & ii. q in DC$_{\textrm{B}}$ & {}\\
	{} & iii. !p in TDL$_{\textrm{B}}$\\
    \end{tabular}
\end{exe}
Im Falle der \textit{eben}-Begründung ist p $>$ q im cg, p wird cg, während bei der \textit{halt}-Begründung p $>$ q in DC$_{\textrm{A}}$ ist und p Teil von DC$_{\textrm{A}}$ wird. Da der cg die DC-Systeme impliziert (es kann nichts im cg sein, das nicht auch in den DC-Systemen ist), bildet meine Modellierung das Implikationsverhältnis zwischen \textit{eben} und \textit{halt}, das Thurmair mit den Beschreibungen \textit{evident} und \textit{plausibel} erfasst, ebenfalls ab. 

Dies gilt genauso für die Diskursmodellierung, wenn \textit{eben} und \textit{halt} in Direktiven auftreten. Beim \textit{eben}-Direktiv ist nach meiner Analyse q $>$ !p im cg, !p steht bereits auf Bs TDL. Beim \textit{halt}-Direktiv ist q $>$ !p Teil von DC$_{\textrm{A}}$ und !p wird zum Bestandteil der TDL von B. 

Da sowohl in \textit{halt}- als auch \textit{eben}-Assertionen (Begründungen) q mindestens in DC$_{\textrm{A/B}}$ sein muss bzw. in assertiven Folgen und Direktiven p in DC$_{\textrm{B}}$ enthalten ist, nehmen diese Füllungen keinen Einfluss auf das Implikationsverhältnis. 

Da der cg-Inhalt q $>$ !p gleichzeitig auch DC$_{\textrm{A}}$-Inhalt ist und in beiden Fällen !p schließlich auf TDL$_{\textrm{B}}$ steht, wird das Implikationsverhältnis abgebildet.
	
Und schließlich ergeben sich die parallelen Verhältnisse auch für assertive Folgen: Im Falle von \textit{halt} ist die Inferenzrelation nur unter As Bekenntnissen, während sie bei \textit{eben} im cg ist. q wird im Zuge der \textit{halt}-Äußerung zu einem Bekenntnis von A, wohingegen es in der \textit{eben}-Äußerung ein cg-Inhalt wird. Hinsichtlich des vorausgesetzten q bei B unterscheiden sich assertive \textit{halt}- und \textit{eben}-Folgen nicht. q wird im Dialog cg.

Aus meiner Modellierung des Diskursbeitrags von \textit{halt}- und \textit{eben}-Äußerungen folgt somit ebenso wie im Rahmen von Thurmairs deskriptiver Erfassung, dass die Bedeutung von \textit{eben} die Bedeutung von \textit{halt} impliziert. 

Auf der Basis dieser Implikationsrelation kann auch die Beobachtung, dass es wenige Kontexte gibt, in denen \textit{eben} nicht durch \textit{halt} ersetzt werden kann, m.E. eine natürliche Erklärung finden: Da das \textit{eben} das \textit{halt} impliziert, kann man annehmen, dass die beiden Partikeln sich skalar anordnen lassen, wobei \textit{eben} das stärkere Element ist (vgl. (\ref{657})).

\begin{exe}
        \ex\label{657} 
          \begin{tabular}[t]{ccc}
   		 \multicolumn{3}{c}{eben}\\
         & $\vert$ & \\
        \multicolumn{3}{c}{halt}
    	\end{tabular}
\end{exe}
Aufgrund des Implikationsverhältnisses kann \textit{halt} normalerweise in jedem Kontext auftreten, in dem auch \textit{eben} stehen kann. Es lässt sich annehmen, dass die Verwendung von \textit{halt} eine konversationelle (skalare) Implikatur \is{konversationelle Implikatur} auslöst: Die Auswahl des schwächeren Elementes auf der Skala implikatiert die Negation des stärkeren. Skalare Implikaturen basieren üblicherweise auf derartigen Implikationsskalen.

Wird \textit{halt} in einem Kontext wie (\ref{658}) verwendet, führt dies zu der unangemessenen Lesart, dass es nur plausibel (und nicht fakt/evident etc.) ist, dass der Wal ein Säugetier ist.
\pagebreak
\begin{exe}
	\ex\label{658} 
		\begin{xlist}	
			\ex\label{658a} Der Wal ist \textbf{eben} ein Säugetier.
			\ex\label{658b} ?Der Wal ist \textbf{halt} ein Säugetier.
		\end{xlist}
\end{exe}
Wenn \textit{halt} und \textit{eben} kombiniert werden, treten folglich zwei Elemente auf, von denen das eine der Implikationsauslöser ist und das andere eben diese ausgelöste Implikation repräsentiert.

In der Abfolge \textit{halt eben} steht somit erst das Element, das die Implikation darstellt und anschließend das Element, das die Implikation auslöst. In der Abfolge \textit{eben halt} tritt zunächst das Element auf, das die Implikation auslöst, es folgt das Element mit dem implizierten Inhalt.

\subsection{Verstärkung von Implikationen}
\label{sec:verstimpli}
Unabhängig von der Beschäftigung mit der Kombination von MPn gibt es Annahmen zur relativen Abfolge von Elementen, die Implikationen mitbringen, und dem durch sie implizierten Inhalt. Das hier relevante Phänomen ist die \textit{Verstärkung von Implikationen} \is{Implikationsverstärkung (reinforcement of entailments)}(\textit{reinforcement of entailments}). Schon früh haben Autoren formuliert, dass man implizierte \is{Implikation} (und auch präsupponierte) \is{Präsupposition} Inhalte nicht \textit{verstärken} kann. Derartige Inhalte zu verstärken heißt, nach Einführung des Elementes, das die Implikation \is{Implikation} (bzw. Präsupposition) \is{Präsupposition} auslöst, den Inhalt genau dieses Schlusses overt festzustellen. Aus diesem Grund sind die Sätze in (\ref{659}) bis (\ref{661}) markiert.

\begin{exe}
	\ex\label{659} 
	??The King of France is bald, and there is a King of France.
\end{exe}
\vspace{-0.65cm}
\begin{exe}
	\ex\label{660} 
	John ??knew/??regretted//It’s considered ??odd that Mary left, and indeed she did.
\end{exe}
\vspace{-0.65cm}
\begin{exe}
	\ex\label{661} 
	??Even John left, and one wouldn’t have expected it. 
	\hfill\hbox {\citet[64/64/66]{Horn1976}}
\end{exe}											         
Sie weisen die Präsuppositionen in (\ref{662}) bis (\ref{664}) auf.

\begin{exe}
	\ex\label{662} 
	\textbf{\textit{der}} König von Frankreich $>>$ es gibt einen König von Frankreich
\end{exe}
\vspace{-0.65cm}
\begin{exe}
	\ex\label{663} 
	Hans \textbf{\textit{weiß}}/\textbf{\textit{bedauert}}/es wird \textbf{\textit{für komisch gehalten}}, dass Maria gegangen ist. $>>$ Maria ist gegangen.
\end{exe}
\vspace{-0.65cm}
\begin{exe}
	\ex\label{664} 
	\textbf{\textit{Sogar}} Hans ist gegangen. $>>$ Man erwartet nicht, dass Hans gegangen ist.
\end{exe}
Der definite Artikel löst eine Existenzpräsupposition \is{Existenzpräsupposition} aus. Die faktiven Matrixverben \is{Faktivität} setzen die Wahrheit ihres Komplementsatzes voraus. Und die Fokuspartikel \is{Fokuspartikel} \textit{sogar} führt die Erwartungspräsupposition \is{Erwartungspräsupposition} ein, dass der Sachverhalt als unerwartet gilt. Tritt zuerst der Präsuppositionsauslöser in der Struktur auf und wird anschließend die Präsupposition genannt, resultieren markierte Äußerungen. (\ref{665}) bis (\ref{667}) zeigen parallele Fälle unter Beteiligung von Implikationen.

\begin{exe}
	\ex\label{665} 
	John$_{\textrm{i}}$ ??managed to leave, and (indeed) he$_{\textrm{i}}$ left.
\end{exe}
\vspace{-0.65cm}
\begin{exe}
	\ex\label{566} 
	John ??killed Alvin, and Alvin died.
\end{exe}
\vspace{-0.65cm}
\begin{exe}
	\ex\label{667} 
	??John$_{\textrm{i}}$ left too, and he$_{\textrm{i}}$ did.
	\hfill\hbox {\citet[64/64/66]{Horn1976}}
\end{exe}						
Die Ausdrücke \textit{schaffen}, \textit{umbringen} sowie \textit{auch} sind hier für die implizierten Inhalte verantwortlich (vgl. (\ref{668}) bis (\ref{670})).

\begin{exe}
	\ex\label{668} 
	Hans hat es geschafft, zu gehen. $\rightarrow$ Hans ist gegangen.
\end{exe}
\vspace{-0.65cm}
\begin{exe}
	\ex\label{669} 
	Hans hat Alvin umgebracht. $\rightarrow$ Alvin ist gestorben.
\end{exe}
\vspace{-0.65cm}
\begin{exe}
	\ex\label{670} 
	Hans ist auch gegangen. $\rightarrow$ Hans ist gegangen.
\end{exe}
Und genauso wie in (\ref{659}) bis (\ref{661}) sind die Sätze degradiert, wenn der implizierte Inhalt auf das Element, das die Implikation einführt, folgt.

Diese Beobachtung hat \citet{Horn1976} in einer Beschränkung festgehalten, die besagt, dass das zweite Konjunkt einer Koordination nicht redundant sein darf (vgl. (\ref{671})).

\begin{exe}
	\ex\label{671} 
	The second conjunct Q of a conjunction \textit{P and Q must} assert some propositional content which does not logically 		follow from the first conjunct P (i.e. P \& Q is anamalous if P $\vdash$ Q or \textit{a fortiori}, if P $>>$ Q).	
	\hfill\hbox {\citet[65]{Horn1976}}
\end{exe}	
Implikationen lassen sich folglich nicht verstärken. Dies führt mit sich, dass implizierte Information nicht nach dem Implikationsauslöser eingeführt werden kann. 

Nach meiner Bedeutungszuschreibung an das kombinierte Auftreten von \textit{eben} und \textit{halt} in Abschnitt~\ref{sec:interpretationkombi} ergibt sich die Bedeutung der MP-Kombination additiv aus dem Beitrag der beiden Einzelpartikeln. D.h. die Bedeutungen von \textit{eben} und \textit{halt} werden koordiniert, wenn natürlich auch gerade nicht durch eine Konjunktion – eine Art der Verknüpfung, die für MPn grundsätzlich ausgeschlossen ist. Dazu habe ich in Abschnitt~\ref{sec:impli} gezeigt, dass sich im Rahmen meiner Mo\-dellierung des Diskursbeitrags von \textit{halt} und \textit{eben} die (auch schon von \citealt{Thurmair1989}) gemachte Annahme widerspiegelt, dass die Bedeutung von \textit{eben} die Bedeutung von \textit{halt} impliziert.

Die Abfolge \textit{eben halt} ist m.E. nun deshalb markiert, weil der Sprecher die Informationen, die er mitteilen möchte, auf redundantem \is{Redundanz} Wege vermittelt. Wie in den Beispielen von \citet{Horn1976} führt dies nicht zu harter Ungrammatikalität, präferiert werden aber in jedem Fall die non-redundanten Fälle. (\ref{672}) bis (\ref{675}) zeigen, dass die umgekehrte Sequenzierung von impliziertem Inhalt und Implikationsauslöser völlig unproblematisch ist.

\begin{exe}
	\ex\label{672} 
	Hans ist gegangen, Hans hat es geschafft zu gehen.
\end{exe}
\vspace{-0.65cm}
\begin{exe}
	\ex\label{673} 
	Alvin ist gestorben, Hans hat Alvin umgebracht.
\end{exe}
\vspace{-0.65cm}	
\begin{exe}
	\ex\label{674} 
	Es gibt einen König von Frankreich und der König von Frankreich ist kahlköpfig.
\end{exe}
\vspace{-0.65cm}
\begin{exe}
	\ex\label{675} 
	Maria ist gegangen und Hans wusste/bedauerte/fand es komisch, dass Maria gegangen ist.
\end{exe}
Diese Äußerungen sind in keiner Weise markiert, was darauf zurückzuführen ist, dass die Informationen non-redundant präsentiert werden. Es liegt hier eine Informations\underline{zunahme} (vs. Informations\underline{abnahme} \is{Informationszunahme/-abnahme} in (\ref{659}) bis (\ref{661}) und (\ref{665}) bis (\ref{667})) vor.

Da im Falle der Abfolge implizierter Inhalt $>$ Implikationsauslöser eine Informationszunahme erfolgt, ist auch die MP-Abfolge \textit{halt eben} problemlos möglich. Das \textit{halt} steuert den implizierten Inhalt bei, \textit{eben} führt die Implikation in die Struktur ein. Auch für die MPn in dieser MP-Kombination argumentiere ich folg\-lich, dass sich ihre Abfolge aus der Interpretation der MP-Sequenz motivieren lässt. Die Grundannahme ist somit auch hier, dass Form und Funktion zusammenhängen. Das Kriterium der Motiviertheit der Abfolge ist das der Informati\-onszunahme bzw. -abnahme, von der man die non-redundante Variante, d.h. die Zunahme, bevorzugt. Meine Annahme ist, dass die Präferenz gegenüber dieser Darstellung der Sachverhalte den Präferenzen entspricht, die Sprecher im Falle von Kohärenzrelationen \is{Kohärenzrelation} aufweisen. In Beispielen wie in (\ref{676}) und (\ref{677}) bevorzugen Sprecher auch dann die Abfolge von Sachverhalten, die den temporalen bzw. kausalen Zusammenhang abbilden, wenn kein Konnektor auftritt, der diese Interpretation einführt.

\begin{exe}
	\ex\label{676} 
	Maren öffnet das Fenster und der Blumentopf fällt herunter.\\
	vs.\\
	\#Der Blumentopf fällt herunter und Maren öffnet das Fenster.
\end{exe}

\begin{exe}
	\ex\label{677} 
	Paula bringt die Kleine um 8 Uhr morgens in den Kindergarten, sie holt sie um 4 Uhr nachmittags ab.\\
	vs.\\
	\#Paula holt die Kleine um 4 Uhr nachmittags ab, sie bringt sie um 8 Uhr morgens in den Kindergarten.
\end{exe}
Das Ergebnis der Sprecherurteile in Abschnitt~\ref{sec:spu} war, dass die Abfolge \textit{halt eben} in Direktiven genauso bevorzugt wird wie in Assertionen. Das Kriterium der Informationszunahme ist derart allgemein, dass es gleichermaßen auf Assertionen und Direktive angewendet werden kann. Egal, ob eine Aufforderung oder eine Mitteilung gemacht wird, bevorzugt wird die non-redundante Version dieser sprachlichen Handlung. Auch in (\ref{678}), wo zwischen \textit{kaltem Bier} und \textit{Bier} ein Implikationsverhältnis besteht, ist die Version, in der die Information (von Bier zu kaltem Bier) zunimmt (vgl. (\ref{678a})) völlig akzeptabel, während (\ref{678b}), wo die Implikation verstärkt wird, als markiert einzustufen ist.
	
\begin{exe}
	\ex\label{678} 
		\begin{xlist}	
			\ex\label{678a} Hol das Bier aus dem Keller, hol das kalte Bier aus dem Keller!
			\ex\label{678b} \#Hol das kalte Bier aus dem Keller, hol das Bier aus dem Keller!
		\end{xlist}
\end{exe}
D.h. anders als bei meiner Ableitung der (un)markierten Abfolgen von \textit{ja} und \textit{doch} in ihren kombinierten Vorkommensweisen, nimmt das Kriterium, das ich für den Zusammenhang  von Form (Abfolge) und Funktion (Interpretation) verantwortlich mache, nicht Bezug auf einen bestimmten Illokutionstyp. Da sich die Kombinationen aus \textit{halt} und \textit{eben} in Assertionen und Direktiven gleich verhalten (vgl. Abschnitt~\ref{sec:spu}), kann die Abfolgebeschränkung plausiblerweise auch nicht auf spezielle Eigenschaften eines bestimmten Sprechaktes eingehen, sondern muss von allgemeinerer Natur sein. Bei der Informationszunahme bzw. -abnahme \is{Informationszunahme/-abnahme} handelt es sich um ein entsprechend abstraktes Kriterium.\\

\noindent
Wie in Abschnitt~\ref{sec:spu} beschrieben, präferieren Sprecher im direkten Vergleich sowohl in Direktiven als auch Assertionen \textit{halt eben} gegenüber \textit{eben halt}. Ziel der beiden Experimente aus Abschnitt~\ref{sec:spu} war es ebenfalls, zu überprüfen, ob sich die auf Horn zurückgehenden Annahmen, wie formuliert in seiner Beschränkung in (\ref{671}), auch in Sprecherurteilen niederschlagen. 

Sollten Sprecher Unterschiede entlang dieser Bedingung u.U. gar nicht bemerken, könnte man auch meine Erklärung für die Präferenz von \textit{halt eben} gegen\-über \textit{eben halt} in Zweifel ziehen.
		
\subsection{Informationszunahme vs. -abnahme im Experiment}
Neben den sechs Testitems zu \textit{halt eben}-/\textit{eben halt}-Assertionen (Experiment 1) bzw. \textit{halt eben}-/\textit{eben halt}-Direktiven (Experiment 2) enthielten beide Experimente deshalb auch jeweils sechs Testitems, in denen die Zu- bzw. Abnahme \is{Informationszunahme/-abnahme} an Information bewertet werden sollte. Teil dieser Sätze sind jeweils Ober- und Unterbegriffe wie z.B. \textit{Haustier} und \textit{Hamster}, \textit{Werkzeug} und \textit{Hammer} oder \textit{Spielzeug} und \textit{Teddy}, wobei der Unterbegriff den Oberbegriff impliziert. 
	
Die Sätze, die zur Bewertung gestellt wurden, sind in allen Items in einen pa\-rallelen Kontext eingebettet. Vorweg geht ein Einwand des ersten Sprechers, der vom zweiten Sprecher abgelehnt wird. Es folgt der Testsatz, der stets dem formalen Muster in (\ref{679}) entspricht.

\begin{exe}
	\ex\label{679} 
	Name (einsilbig) $+$ finites Verb (einsilbig) $+$ NP (\textit{ein} $+$ Kompositum) $[$erstes Nomen akzentuiert$]$, \textit{er} $+$ gleiches finites Verb $+$ NP (\textit{einen} $+$ Nomen) $[$zweisilbig, erste Silbe akzentuiert$]$
\end{exe}
(\ref{680}) und (\ref{681}) zeigen zwei Beispiele für Testitems.

\begin{exe}
	\ex\label{680}
	\begin{tabular}[t]{ll}
	\multicolumn{2}{l} {B2 Renovierungsarbeiten} \tabularnewline
	\multicolumn{2}{l} {Lisa: Dirk ist schon wieder weg. Und wir können die Arbeit machen.} \tabularnewline
	Sabine: Nein, nein. & \underline{Dirk holt ein Werkzeug, er holt einen Hammer.} \tabularnewline
	{} & \underline{Dirk holt einen Hammer, er holt ein Werkzeug.}
    \end{tabular}
\end{exe}

\begin{exe}
	\ex\label{681}
	\begin{tabular}[t]{ll}
	\multicolumn{2}{l} {B6 Am Messestand} \tabularnewline
	\multicolumn{2}{l} {Julia: Für manche Objekte scheint sich keiner zu interessieren. Arndt} 					
	\tabularnewline
	\multicolumn{2}{l} {hat nichts zu tun.} 					
	\tabularnewline
	Thorsten: Nein, nein. & \underline{Arndt zeigt ein Sportboot, er zeigt einen Achter.} \tabularnewline
	{} & \underline{Arndt zeigt einen Achter, er zeigt ein Sportboot.}
    \end{tabular}
\end{exe}
Da die Hälfte der Testitems dieses Phänomen betraf, das nicht i.e.S. auf eine grammatische Fragestellung Bezug nimmt, thematisierte die Hälfte der Filleritems (d.h. 18 Items) Kohärenzrelationen \is{Kohärenzrelation} der Art in (\ref{676}) und (\ref{677}).

Die Ergebnisse beider Experimente (getestet wurden jeweils die gleichen Sätze) sind sehr deutlich (vgl. (\ref{682}) und (\ref{683})).

\begin{exe}
	\ex\label{682} Ergebnisse\\[-1em]
	\begin{tabular}[t]{|l|l|l|}
	\hline
	\textbf{Item} & Exp 1 & Exp 2 \\
	\hline
	Fahrzeug/Trecker & 28:0 & 32:0\\
	\hline
	Werkzeug/Hammer & 28:1 & 26:4\\
	\hline
	Spielzeug/Teddy & 26:3 & 30:2\\
	\hline
	Haustier/Hamster & 25:1 & 31:1\\
	\hline
	Schriftstück/Ausweis & 24:5 & 27:5\\
	\hline
	Sportboot/Achter & 21:7 & 28:4\\
	\hline
    \end{tabular}
\end{exe}
Zur statistischen Auswertung wurde (ebenfalls $[$s.o.$]$) jeweils ein log-lineares gemischtes Modell gerechnet (\citealt{Baayen2008}) mit Informationszunahme bzw. -abnahme als abhängiger Variable und Teilnehmern und Items als Zufallsvariablen (N = 169, log-Likelihood = $\minus$53,25 $[$Experiment 1$]$) bzw. (N = 190, log-Likelihood = $\minus$51,15 $[$Experiment 2$]$). Das signifikante Interzept ($\beta$ =  2,659 , SE = 0,483, Wald z = 5,50, p $<$ 0,001) bzw. ($\beta$ =  3,433, SE = 0,518, Wald z = 6,63, p $<$ 0,001) zeigt, dass signifikant häufiger die Sequenzierung unter Beteiligung der Informationszunahme als der Informationsabnahme verwendet wird.

Die Ergebnisse demonstrieren, dass die Fälle, die gegen Horns Prinzip in (\ref{671}) verstoßen, in Sprecherurteilen im direkten Vergleich tatsächlich schlechter abschneiden als ihre non-redundante Variante.

Die Redundanzbedingung \is{Redundanzbedingung} schlägt sich in Sprecherurteilen nieder und zwar unabhängig davon, ob die dispräferierte Implikationsverstärkung in der MP-Abfolge oder in Ober-/Unterbegriffen abgebildet wird.

In Bezug auf die Bewertungen der Sätze, die die Ober-/Unterordnungen beinhalten, halte ich es für erwähnenswert, dass je nach Prototypizität \is{Prototypizität} des Unterbegriffes in der Klasse der Oberbegriffe die Entscheidung zwischen den zwei Urteilsmöglichkeiten ggf. unterschiedlich deutlich ausfällt. D.h. ein Hamster ist z.B. mit Sicherheit ein prototypischeres Exemplar eines Haustieres, als ein Ausweis es für die Menge der Schriftstücke ist. Bzw. die auftretenden Oberbegriffe eröffnen nicht alle gleichermaßen typische Ober-/Unterordnungen. Fahr\-zeuge stellen hierbei z.B. eine prototypischere Klasse von Elementen dar als Sportboote. Dieser Nebeneffekt ist sicherlich nicht unerwartet, er bietet aber auch schöne Evidenz für die Sensitivität von Sprechern gegenüber dem untersuchten Phänomen. Abhängig davon, wie deutlich sie den Verstoß gegen das Redundanzprinzip wahr\-nehmen, fallen auch ihre Präferenzen unterschiedlich stark ge\-genüber der redundanten und non-redundanten Darstellung der Information aus. In Test 1 schneiden die Zuordnungen \textit{Schriftstück} und \textit{Ausweis} sowie \textit{Sportboot} und \textit{Achter} und in Test 2 ebenfalls \textit{Schriftstück} und \textit{Ausweis} im Vergleich zu ty\-pischeren Verbindungen von \textit{Haustier} und \textit{Hamster} oder \textit{Fahrzeug} und \textit{Trecker} am schlech\-testen ab (wenn\-gleich der Unterschied natürlich immer noch sehr deutlich ist) (vgl. (\ref{682})).\footnote{Warum in Test 2 das Paar \textit{Hammer} $\rightarrow$ \textit{Werkzeug} verhältnismäßig schlecht abschneidet, kann ich mir nicht erklären.}

\subsection{Interpretation oder Rhythmus? - Der Ausschluss von \textit{stress clash}}
\label{sec:stressclash}
Als Evidenz für die Annahme, dass \textit{halt eben} die unmarkierte und \textit{eben halt} die markierte Abfolge der beiden MPn darstellt, dient in meiner Argumentation das frequentere Auftreten der ersten gegenüber der zweiten Anordnung sowie die in Abschnitt~\ref{sec:spu} beschriebenen Ergebnisse zweier Akzeptabilitätsstudien. (\ref{683}) und (\ref{684}) zeigen erneut je ein Testitem aus den beiden Experimenten.

\begin{exe}
	\ex\label{683} Umgangsformen\\
	Verena: Warum ist dein neuer Freund eigentlich immer so höflich?\\
	Sara: \underline{Er ist halt eben Brite.}/\underline{Er ist eben halt Brite.}
\end{exe}

\begin{exe}
	\ex\label{684} Arbeitsbeginn\\
	Susanne: Ich bin jeden Tag immer viel zu früh im Büro.\\
	Daniela: \underline{Dann fahr halt eben später ab!}/\underline{Dann fahr eben halt später ab!}
\end{exe}
Wie in Abschnitt~\ref{sec:verstimpli} ausgeführt, mache ich einen interpretatorischen Unterschied zwischen den beiden Abfolgen für die Markiertheit von \textit{eben halt} verantwortlich. Einen (potenziellen) Einwand, den man gegenüber dieser Ausdeutung der sich in den Experimenten zeigenden deutlichen Präferenz der Sprecher gegenüber \textit{halt eben} anführen kann, ist, dass die Testsätze zwar strukturell pa\-rallel konstruiert wurden, dass die \textit{halt eben}- und \textit{eben halt}-Versionen aber rhythmisch jeweils nicht identisch sind. Dies ist bei ansonsten vorliegender segmentaler Identität auch nicht zu erreichen, wenn die beiden Partikeln eine unterschiedliche Sil\-benanzahl aufweisen und die erste Silbe von \textit{eben} zudem Wortakzent \is{Wortakzent} trägt. Wie (a) und (b) in (\ref{685}) und (\ref{686}) zeigen, liegen folglich jeweils unterschiedliche Akzentmuster vor. (\textit{u} steht für eine unakzentuierte, / für eine akzentuierte Silbe.)

\begin{exe}
	\ex\label{685} 
	\begin{xlist} 
	\ex\label{685a}
	\begin{tabular}[t]{lllllll}
	u & / & u & / & u & / & u\\
	Er & ist & halt & e & ben & Bri & te.
    \end{tabular}
    \ex\label{685b}
	\begin{tabular}[t]{lllllll}
	u & / & / & u & u & / & u\\
	Er & ist & e & ben & halt & Bri & te.
    \end{tabular}
\end{xlist}    
\end{exe}

\begin{exe}
	\ex\label{686} 
	\begin{xlist} 
	\ex\label{686a}
	\begin{tabular}[t]{llllllll}
	u & / & u & / & u & / & u & /\\
	Dann & fahr & halt & e & ben & spä & ter & ab!
    \end{tabular}
    \ex\label{686b}
	\begin{tabular}[t]{llllllll}
	u & / & / & u & u & / & u & /\\
	Dann & fahr & e & ben & halt & spä & ter & ab!
    \end{tabular}
\end{xlist}    
\end{exe}			
Eine bekannte Annahme in der Phonetik und Phonologie ist, dass Sprachen eine Präferenz gegenüber rhythmischer Alternation, d.h. des Wechsels akzentuierter und unakzentuierter Silben zeigen (vgl. z.B. \citealt[1]{Wagner2002}, \citealt[141]{Gussenhoven2004}, \citealt[18-24]{Schlueter2005}, \citealt[332]{Bohn2011}). Ein viel untersuchtes Phänomen vor diesem Hintergrund ist der sogenannte \textit{Ak\-zentzusammenstoß} (\textit{stress clash}) zusammen mit den Strategien, die Sprachen einsetzen, um ihn zu umgehen. Solche (potenziellen) Zusammenstöße können im Deutschen auf Wort- und Phrasenebene auftreten (vgl. (\ref{687}) und (\ref{688})) und werden dadurch vermieden, dass der Akzent entweder verschoben wird oder Deakzentuierung einer der Silben eintritt. (Je niedriger die Zahl, desto stärker der Akzent.)
\pagebreak
\begin{exe}
	\ex\label{687} 
	\begin{xlist} 
	\ex\label{687a}
	\begin{tabular}[t]{ll}
	1 & 2 \\
	an & ziehen
    \end{tabular}
    \ex\label{687b}
	\begin{tabular}[t]{llll}
	{} & 1 & 3 & 2 \\
	den & Rock & an & ziehen
    \end{tabular}
\end{xlist}    
\end{exe}

\begin{exe}
	\ex\label{688} 
	\begin{xlist} 
	\ex\label{688a}
	\begin{tabular}[t]{ll}
	1 & 2 \\
	Hoch & deutsch
    \end{tabular}
    \ex\label{688b}
	\begin{tabular}[t]{lll}
	1 & 3 & 2 \\
	Alt & hoch & deutsch
    \end{tabular}
    \newline
    \hbox{}\hfill\hbox {\citet[1]{Wagner2002}}
\end{xlist}    
\end{exe}
In Isolation ist die erste Silbe von \textit{anziehen} stärker akzentuiert als die zweite Silbe. Tritt es in der Phrase in (\ref{687b}) auf, wird hingegen die zweite Silbe von \textit{anziehen} stärker akzentuiert als die erste. Würde diese Strategie nicht eingesetzt, käme es zu einem Akzentzusammenstoß \is{Akzentzusammenstoß (stress clash)}der Akzente von \textit{Rock} und \textit{an}. Genauso verändert sich die Gewichtung der Akzente von (\ref{688a}) zu (\ref{688b}). Zwischen \textit{alt} und \textit{hoch} würde es zu einem Zusammenstoß der Akzente kommen. Dies führt dazu, dass \textit{hoch} in (\ref{688b}) weniger und \textit{deutsch} stärker akzentuiert ist als in (\ref{688a}) (vgl.  zu Beispielen im Französischen \citealt{Mazzola1992}, zum Englischen \citealt{Vogel1995}, zum Niederländischen \citealt[141]{Gussenhoven2004} und zum Katalanischen \citealt{Prieto2010}). Wir sehen, dass Sprecher Strategien anwenden, um zwei aufeinander folgende akzentuierte Silben in der Akzentstruktur zu vermeiden.

Die \textit{halt eben}- und \textit{eben halt}-Testsätze (vgl. (a) und (b) in (\ref{685}) und (\ref{686})) unterscheiden sich nun in sofern rhythmisch voneinander, als dass in (a) das präferierte akzentalternierende Muster vorliegt, während es in (b) zwischen dem finiten Verb und der ersten Silbe von \textit{eben} zu einem Akzentzusammenstoß kommt. Aufgrund dieser Beschaffenheit der Testsätze ist folglich nicht auszu\-schließen, dass die \textit{halt eben}-Sätze allein aufgrund ihres präferierten Rhythmus gegenüber den \textit{eben halt}-Sätzen besser abgeschnitten haben.\footnote{Diesen Hinweis sowie die Anregung, den Faktor auf die Art wie unten beschrieben in ein Experiment einzubauen, verdanke ich Ralf Vogel.} In \citet[32-33]{Franck1980} findet sich ein Hinweis darauf, dass MPn an sich auch mit prosodischen Verhältnissen interagieren können bzw. ihr Auftreten aufgrund solcher sogar bedingt ist. Mir ist allerdings kein Autor bekannt, der sich für eine Erklärung der MP-Reihungen über rhythmische Regeln ausspricht. Dies ist vermutlich darauf zurückzuführen, dass MPn in der Regel einsilbig und unbetont sind und sich deshalb bei verschiedenen Abfolgen hinsichtlich des prosodischen Faktors der Akzentuierung kein Unterschied einstellen kann. 

Um diesen möglichen intervenierenden Faktor auszuschließen, habe ich des\-halb ein drittes Experiment zur Abfolge von \textit{halt} und \textit{eben} durchgeführt (vgl. auch \citealt[161-165]{Mueller2016b}). Der Aspekt des Akzentzusammenstoßes \is{Akzentzusammenstoß} wurde derart in das Experiment aufgenommen, dass in den Testsätzen neben der Abfolge auch der NP-Typ variiert. Da die Subjekte in den Sätzen in Experiment 1 jeweils durch einsilbige Personalpronomen realisiert wurden, die – außer in bestimmten Fokuskontexten – stets unakzentuiert sind, führt dies dazu, dass das finite Verb akzentuiert wird. Um den stress clash zwischen dem finiten Verb und \textit{eben} zu vermeiden, gilt es, die Akzentuierung des finiten Verbs zu verhindern. Um dies zu erreichen, machen einsilbige Eigennamen die Subjekte aus (wie z.B. in (\ref{689})).

\begin{exe}
	\ex\label{689}
	\begin{tabular}[t]{llllllll}
	\multicolumn{8}{l} {Umgangsformen} \tabularnewline
	\multicolumn{8}{l} {Verena: Warum ist dein neuer Freund eigentlich immer so höflich?} \tabularnewline
	Sara: & / & u & u & / & u & / & u\\
	{} & Tom & ist & halt & e & ben & Bri & te.\\
	{} & / & u & / & u & u & / & u\\
	{} & Tom & ist & e & ben & halt & Bri & te.
    \end{tabular}
\end{exe}
Bevorzugen die Testanten auch dann die \textit{halt eben}-Variante, wenn kein Akzentzu\-sammenstoß \is{Akzentzusammenstoß} auftritt, kann der Einfluss dieses prosodischen Faktors (zumindest unter der von mir gewählten Kodierung) ausgeschlossen werden und weiterhin an einer auf der Interpretation der beiden Abfolgen basierenden Erklärung festgehalten werden.

Anstatt der sechs Testitems aus Experiment 1 bewertet jeder Testant in diesem dritten Experiment drei dieser Items mit dem NP-Typ \glq Pronomen\grq {} und drei (andere) mit dem NP-Typ \glq Eigenname\grq {}.


\begin{exe}
	\ex\label{690} Testitems pro Testant \textit{halt eben} vs. \textit{eben halt}\\[-1em]	
	\begin{tabular}[t]{|l|l|}
	\hline
	Pronomen & Eigenname\\
	\hline
	3 & 3\\
	\hline
    \end{tabular}
\end{exe}
Ebenfalls waren erneut die gleichen sechs Testitems zur Informationszunahme und -abnahme Teil des Experiments sowie die gleichen 36 Filler. Die Aufgabe bestand wiederum darin, zu entscheiden, ob Satz a) besser ist als Satz b) oder ob Satz b) besser ist als Satz a). Es gab acht verschiedene Versionen des Tests, in denen die Antworten a) und b) sowohl bei den Testsätzen als auch den Fillern ausbalanciert sind und zwischen den Tests wechseln.

In die Auswertung genommen wurden die Bewertungen von 61 deutschen Muttersprachlern\footnote{Ausgeschlossen habe ich die Bewertungen von acht ausländischen Studierenden.}, bei denen es sich um Germanistikstudierende an der Universität Bielefeld (31) sowie der Universität Göttingen (30) im WS 2013/14 handelte.\footnote{Ich bedanke mich herzlich bei Beate Lingnau, Sören Olhus und Jeanine Wein, dass sie die Bögen in ihren Kursen haben bearbeiten lassen.} Die Testanten waren zwischen 19 und 38 Jahre ($\diameter 24$) alt. Die Herkunft der Sprecher lässt sich entlang der Datenpunkte von \citet{Eichhoff1978} folgendermaßen verorten: 93\% der Sprecher stammen aus der oberen Landeshälfte (A – Mitte/Norden D), wobei der größte Teil (56\%) aus Bereich C stammt (dazu: 14xD, 7xB, 4xA). Die übrigen 7\% haben ihre Schulzeit in den Bereichen E (2x), F (1x) und G (1x) verbracht. Diese Informationen sind von daher relevant, als dass es sich wiederum (wie in den Experimenten 1, 2 und 4) um Sprecher handelt, die aus Gegenden stammen, für die man (früher) angenommen hat, dass dort der Gebrauch von \textit{eben} vorherrschend war. Damit hängt zusammen, dass dort nach Elspaß' Spekulation (\citeyear[17, Fn 41]{Elspass2005}) auch die Sequenz \textit{eben halt} zu verankern sein müsste.

Die Entscheidung zwischen der Abfolge \textit{halt eben} und \textit{eben halt} fällt bei jedem Testitem sowohl mit dem NP Typ \glq Eigenname\grq {} (kein Akzentzusammenstoß bei \textit{eben halt}) als auch mit dem NP-Typ \glq Pronomen\grq {}  (Akzentzusammenstoß bei \textit{eben halt}) deutlich zugunsten der Abfolge \textit{halt eben} aus (vgl. (\ref{691})).

\begin{exe}
	\ex\label{691} Häufigkeit Experiment 3 alle Testitems\\[-1em]
	\begin{tabular}[t]{|l|l|cx{1pt}l|l|l|}
	\hline
	{} & \multicolumn{2}{l|}{\textbf{Eigenname}} & \multicolumn{2}{l|}{\textbf{Pronomen}}\\
	\hline
	{} & \textit{halt eben} & \textit{eben halt} & \textit{halt eben} & \textit{eben halt}\\
	\hline
	Waage & 29 & 3 & 26 & 3\\
	\hline
	Schwabe & 26 & 3 & 28 & 4\\
	\hline
	Kölner & 28 & 4 & 26 & 3\\
	\hline
	Maurer & 27 & 5 & 24 & 5\\
	\hline
	Brite & 26 & 3 & 26 & 5\\
	\hline
	Moslem & 25 & 4 & 28 & 3\\
	\hline
    \end{tabular}
\end{exe}
Es wurde ein log-lineares gemischtes Modell gerechnet (\citealt{Baayen2008}) mit Abfolge als abhängiger Variable, NP-Typ als Effekt-kodierte unabhängige Variable und Testanten und Items als Zufallsvariablen mit dem Faktor NP-Typ in der Steigung (\citealt{Barr2013}) (N = 364, log-Likelihood = $\minus$129,9). Das signifikante Interzept ($\beta$ = 2,518, SE = 0,252, Wald z = 9,97, p $<$ 0,001) zeigt, dass signifikant häufiger \textit{halt eben} als \textit{eben halt} verwendet wird. Es findet sich keine Evidenz für einen Einfluss des NP-Typs ($\beta$ = 0,079, SE = 0,178, Wald z = 0,45, p = 0,66).\footnote{Die Verteilungen der Antworten bei den übrigen sechs Testitems zur Informationszunahme bzw. -abnahme entsprechen den Ergebnissen aus Experiment 1, 2 und 4. (i) fasst die Ergebnisse zusammen.
\begin{exe}
	\ex\label{692} Informationszunahme/-abnahme in Experiment 3\\[-1em]
	\begin{tabular}[t]{|l|l|}
	\hline
	Item & Ergebnis\\
	\hline
	Fahrzeug/Trecker & 56:3\\
	\hline
	Werkzeug/Hammer & 55:6\\
	\hline
	Spielzeug/Teddy & 54:7\\
	\hline
	Haustier/Hamster & 59:2\\
	\hline
	Schriftstück/Ausweis & 48:12\\
	\hline
	Sportboot/Achter & 50:9\\
	\hline
    \end{tabular}
\end{exe}
Es wurde erneut ein log-lineares gemischtes Modell gerechnet (\citealt{Baayen2008}) mit Informationszunahme bzw. -abnahme als abhängiger Variable und Teilnehmern und Items als Zufallsvariablen (N = 361, log-Likelihood = $\minus$119,5). Das signifikante Interzept ($\beta$ = 2,568, SE = 0,301, Wald z = 8,54, p $<$ 0,001) zeigt, dass signifikant häufiger die Sequenzierung unter Beteiligung der Informationszunahme als der Informationsabnahme verwendet wird.}

Da der Akzentzusammenstoß in meinen Testsätzen folglich nicht auf die Bewertung der Testsätze durch die Sprecher Einfluss zu nehmen scheint, möchte ich an meiner in Abschnitt~\ref{sec:verstimpli} ausgeführten Auffassung und Analyse festhalten, dass sich der Markiertheitsunterschied zwischen \textit{halt eben} und \textit{eben halt} unter Bezug auf die Interpretation, genauer den Diskurseffekt, der MP-Äußerungen ableiten lässt. 

\setcounter{equation}{0}
\section{Gibt es Gebrauchsunterschiede von \textit{halt eben} und \textit{eben halt}?}
\label{sec:gebrauchheeh}
Eine Frage, die sich aus der Auffassung, dass das Vorkommen von \textit{halt eben} und \textit{eben halt} ein Markiertheitsphänomen ist, ergibt, ist, unter welchen Umständen \textit{eben halt} von Sprechern gebraucht wird. Trotz seiner Markiertheit tritt die umgekehrte Abfolge schließlich durchaus auf. Sie macht in den drei untersuchten Korpora (vgl. (\ref{693})) ca. 1/7 bzw. 1/3 der Kombinationen aus. Von einem spora\-dischen Gebrauch kann nicht die Rede sein (vgl. auch schon Abschnitt~\ref{sec:häufko}). 

\begin{exe}
	\ex\label{693} Häufigkeiten \textit{halt eben}/\textit{eben halt} in allen Korpora\\[-1em]
	\begin{tabular}[t]{|l|l|l|}
	\hline
	{} & \textit{halt eben} & \textit{eben halt}\\
	\hline
	DeReKo & 715 & 117\\
	\hline
	DGD2 & 63 & 10\\
	\hline
	DECOW2012 & 7328 & 2291\\
	\hline
    \end{tabular}
\end{exe}
Es stellt sich die Frage nach Verwendungsunterschieden. Aus der Sicht meiner Annahme, dass es sich bei den zwei Vorkommensweisen um einen Markiertheits\-unterschied handelt, wäre die Feststellung unterschiedlicher Gebrauchsbedingungen auch wünschenswert. In diesem Fall ließe sich sagen, dass \textit{eben halt} aus bestimmten Kontexten ausgeschlossen ist bzw. auf bestimmte Kontexte beschränkt ist, während die unmarkierte Struktur eine weitere Verwendung hat. Noch wünschenswerter wäre es, wenn sich im Rahmen meiner Ableitung über das Konzept der Implikationsverstärkung eine Erklärung für die Kontexte anbieten würde, in denen \textit{eben halt} (nicht) auftritt.

\subsection{Kontexte für Implikationsverstärkung}
Interessanterweise gibt \citet{Horn1991} Kontexte an, in denen trotz Vorliegen der Situation aus der Redundanzbedingung die Äußerungen völlig akzeptabel sind. Diese Fälle werden im Folgenden charakterisiert (vgl. Abschnitt~\ref{sec:kontrast}). Wenn\-gleich der Schluss für die Argumentation meiner Arbeit sein wird, dass genau diese Kontexte nicht mit den MPn in Verbindung gebracht werden können (vgl. Abschnitt~\ref{sec:disk}), so dienen sie dennoch dem Fortschritt der Argumentation. Zeigen sie doch, dass es ein lohnenswerter Zugang ist, über zulässige Kontexte für die Implikationsverstärkung nachzudenken. Wie diese genau aussehen können, thematisiert der Folgeabschnitt~\ref{sec:weiterekon}.

\subsubsection{Rhetorischer Kontrast}
\label{sec:kontrast}
\citet{Horn1991} führt Sätze wie in (\ref{694}) als Beispiele für unmarkierte Strukturen mit redundantem zweiten Konjunkt an.

\begin{exe}
	\ex\label{694} 
		\begin{xlist}	
			\ex\label{694a} It’s odd that dogs eat cheese, but they do (eat cheese).
			\ex\label{694b} Only Hercules can lift this rock, but he can (lift it).	
			\ex\label{694c} The milk train doesn’t stop here anymore, but it used to.
			\hfill\hbox {\citet[322]{Horn1991}}
		\end{xlist}
\end{exe}	
In (\ref{695}) liegen jeweils Präsuppositionen \is{Präsupposition} vor (die auch die meisten Beispiele ausmachen $[$aber s.u. für Implikationen$]$). Horns Punkt ist, dass die Verstärkung der Präsupposition zulässig ist, wenn das zweite Konjunkt durch \textit{but} angefügt wird und nicht durch \textit{and}. Die ansonsten parallel konstruierten Fälle in (\ref{695}) sind – wie der Redundanzbedingung \is{Redundanzbedingung} nach zu erwarten – markiert.
\begin{exe}
	\ex\label{695} 
		\begin{xlist}	
			\ex\label{695a} \#It’s odd that dogs eat cheese, and they do.
			\ex\label{695b} \#The king of France is bald and there is one.	
			\hfill\hbox {\citet[318/321]{Horn1991}}
		\end{xlist}
\end{exe}						
(\ref{696}) und (\ref{697}) zeigen Beispiele unter Beteiligung von implizierter Information.
\begin{exe}
	\ex\label{696} 
	While she was dying, and I knew she was dying, I wrote my best book. I wrote it in agony, but I wrote it. 		
	\hfill\hbox {(Zitat Raymond Chandler)}
	\newline
	\hbox{}\hfill\hbox {\citet[322,Fn 12]{Horn1991}}	
\end{exe}
\begin{exe}
	\ex\label{697} 
	Tony Fernandez, the Blue Jays' outstanding shortstop, has been playing with stretched ligaments in his left knee, but he 		has been playing.
	\newline
	\hbox{}\hfill\hbox {(N.Y. Times 9/10/87, B14)}
	\newline
	\hbox{}\hfill\hbox {\citet[327]{Horn1991}}	
\end{exe}
Horn zufolge weisen diese Sätze ein Bedeutungsmoment von einem Zugeständnis zu P auf. Das Zugeständnis zu P ist der erste Satzteil, der den Schluss auslöst (in (\ref{698}) fett markiert), auf den die Bestätigung von Q folgt. Die Bestätigung von Q ist der zweite Teil des Satzes, der die Präsupposition darstellt (in (\ref{698}) unterstrichen).
\begin{exe}
	\ex\label{698} 
	\begin{tabular}[t]{cc}
	\textbf{It's odd that dogs eat cheese}, & but \underline{they do (eat cheese)}.\\
	P & Q
    \end{tabular}
\end{exe}
Q kann logisch aus P folgen, kontrastiert aber mit ihm. Oftmals seien die beiden Konjunkte dann Argumente für verschiedene Schlüsse (vgl. \citeyear[325]{Horn1991}). Mit dieser Lesart kompatibel sei das Auftreten adversativer Adverbiale wie \textit{nonetheless}, \textit{just the same}, \textit{be that as it may} oder \textit{despite (that)} in Q.

Warum genau das Vorliegen eines Kontrastes (Horn spricht von einer \textit{rhetori\-schen Kontrastrelation}) \is{rhetorische Kontrastrelation} dazu führt, dass diese Abfolgen sinnvoll werden, lässt sich seinen Ausführungen nicht so recht entnehmen. Man könnte seine Beispiele so ausdeuten, dass sich die Redundanz auflöst, wenn beide Konjunkte zum Inhalt verschiedener Argumentationen beitragen. Unter diesen Umständen leisten sie ihren Beitrag gar nicht beide hinsichtlich des gleichen Aspektes. Wenngleich ich die Kontexte, die Horn anführt, für einschlägig halte, würde der Ansatz m.E. an Attraktivität gewinnen, wenn sich zeigen ließe, \underline{warum} durch diesen speziellen Kontext, diese Absichten etc. diese – eigentlich markierte – Abfolge möglich wird.

\subsubsection{Diskussion}
\label{sec:disk}
Die Sätze aus \citet{Horn1991} werden weitestgehend konzessiv interpretiert. Schon \citet[227-232]{Ward1985} hat angenommen, dass die Komponente \glq Unerwartet\-heit\grq {} beteiligt ist. Horn schreibt allerdings, seine Bedingung sei weiter; Unerwartet\-heit (und auch  \glq Überraschung\grq {} bei Ward) sei nicht durchweg beteiligt. M.E. erfasst man auf diese Art aber die meisten Fälle und gewinnt einen Aspekt, unter Bezug auf den sich die Zulässigkeit der Abfolge Implikationsauslöser $>$ Implikation er\-klären lässt.

In \citet[2293]{Zifonun1997} wird das konzessive Verhältnis folgendermaßen definiert: \glqq $[$...$]$ eine Koinzidenz, die nach einer angenommenen Regularität eigentlich hätte eintreten müssen, $[$ist$]$ entgegen den Erwartungen nicht eingetreten $[$...$]$\grqq{}.

Meiner Meinung nach interpretiert man die Beispiele in (\ref{696}) und (\ref{697}) im Sinne von: \glq Wenn man ein Buch unter Qualen schreibt, schreibt man es nicht wirklich.\grq {} (weil ein Buch zu schreiben normalerweise nicht mit Qualen verbunden ist) und \glq Wenn man mit gedehnten Bändern spielt, spielt man nicht richtig/wirklich.\grq {} (weil Fußballspielen i.d.R. nicht mit gedehnten Bändern geschieht). Wider Erwarten hat derjenige aber geschrieben bzw. gespielt.

Das Konzessive verursacht somit die Lesart des Zugeständnisses: \glq Er hat zwar mit gedehnten Bändern gespielt/unter Qualen geschrieben, aber dennoch...\grq {} . 

Der Grund für die Akzeptabilität von Strukturen der Art in (\ref{696}) und (\ref{697}) liegt dann weniger darin, dass motiviert ist, warum der Sprecher in diesen Fällen die implizierte Information nachliefert oder warum es ihm erlaubt ist, sie nachzulie\-fern (unter Qualen schreiben $\rightarrow$ schreiben), sondern darin, dass die Implikation hier nicht salient vorliegt. Es lässt sich nicht behaupten, dass die Implikation aufgelöst ist, da es sich um einen logischen Zusammenhang handelt, der auf der Wortbedeutung basiert. Salient ist aber gerade ein ganz anderer Zusammenhang, nämlich: Wenn man unter Qualen schreibt, ist das normalerweise kein Schreiben (unter Qualen schreiben $>$ kein Schreiben). Und von eben diesem Zusammenhang wird mit dem \textit{aber}-Satz abgewichen.

Nach Anführen der Horn-Beispiele für akzeptable verstärkte Implikationen bzw. Präsuppositionen und einem Abwägen der Rolle dieser Kontexte, stellt sich die Frage, wie diese Erkenntnisse für den von mir untersuchten Fall einer Implikationsverstärkung nutzbar gemacht werden können. MPn lassen sich aus unabhängigen Gründen natürlich nicht kontrastieren. Es scheint mir auch nur schwer vorstellbar, wie die Implikation zwischen \textit{eben} und \textit{halt} weniger salient gemacht werden könnte zugunsten eines kontextuell prominenten Zusammenhangs, bei dem auf den entgegengesetzten Bedeutungsbeitrag von \textit{halt} geschlossen wird. Ebenfalls scheidet die Möglichkeit aus, dass die beiden Partikeln an unterschiedli\-chen Argumentationen teilhaben. Gewinnbringend ist die Untersuchung von Horn aber insofern, als sich aus ihr für die weitere Argumentation die folgende prinzipiellere Überlegung ergibt, die Horn gar nicht nutzbar macht: Welche Grün\-de kann es geben, implizierte Information nach der sie Implizierenden anzuführen bzw. unter welchen Umständen ist die Darbietung von Information in dieser Abfolge erlaubt?

\subsubsection{Weitere Kontexte zulässiger Implikationsverstärkung: Die Dominanz des implizierten Inhalts}
Ein weiterer Aspekt, der für die Akzeptabilität der Abfolge Implikationsauslöser $>$ Implikation eine Rolle spielen kann, ist, wie dominant die Implikation selbst vorliegt. Implikationen lassen sich dieser Ansicht nach (erst recht) nicht verstärken, wenn der implizierte Inhalt einen hohen Mitteilungswert aufweist \is{Mitteilungswert}(vgl. auch \citealt[165-166]{Mueller2016a}).
	
Ich schließe mich im Folgenden der allgemeineren Vorstellung an, dass sich bei der durch eine Äußerung vermittelten Information prinzipiell ein gewichtiger und weniger gewichtiger Teil unterscheiden lässt und es sprachliche Mittel gibt, die dieser Kodierung dienen. Ich spreche im ersten Fall von \textit{Äußerungsteilen mit hohem Mitteilungswert} und im letzteren von \textit{geringerem Mitteilungswert}. Diese Überlegung wird in anderen Arbeiten unter Bezeichnungen wie \is{Reliefgebung} \textit{Reliefgebung}, \textit{Informationsvordergrund} \is{Informationsvordergrund/-hintergrund} und \textit{-hintergrund}(\citealt{Hartmann1984}), \textit{kommunikative Gewichtung} \is{kommunikative Gewichtung} (\citealt{Brandt1994}) oder \textit{kommunikatives Gewicht} (\citealt{Hoffmann2002, Hoffmann2003} vertreten (vgl. auch \citealt{Reis1993}). Unter hohem Mitteilungswert verstehe ich Haupt- und Vordergrundinformation, die der Sprecher dem Hörer mitzuteilen beabsichtigt. Es handelt sich deshalb i.d.R. um neue Information, die nicht für beide Diskursteilnehmer ableitbar ist und mit der der cg zudem in einem geraden Diskursverlauf angereichert werden soll. Geringerer Mitteilungswert stellt sich dann ein, wenn genau gegensätzliche Verhältnisse vorliegen (zu Beispielen s.u.).

(\ref{700}) ist im Gegensatz zu (\ref{699}) als markiert einzustufen, wenn mit der Äußerung des Satzes über beide Sachverhalte gleichermaßen informiert werden soll – wovon zunächst auszugehen ist, wenn man die plausible Annahme zugrunde legt, dass Assertionen \is{Assertion} im Standardfall einen hohen Mitteilungswert haben.

\begin{exe}
	\ex\label{699} 
	Kai hat ein Haustier gekauft, er hat einen Hamster gekauft.	
\end{exe}
\vspace{-0.65cm}
\begin{exe}
	\ex\label{700} 
	\#Kai hat einen Hamster gekauft, er hat ein Haustier gekauft.	
\end{exe}
Die Mitteilung des implizierten Inhalts nach Mitteilung des Implikationsauslösers ist  hingegen akzeptabel, wenn der zweite Satz beispielsweise im Sinne einer Erläuterung, einer Erklärung oder Klarstellung gelesen wird, die das Verständnis sichern oder vielleicht auch der Erinnerung dienen. Denkbar sind auch Verallgemeinerungen oder Wiederholungen (vgl. die Beispiele in (\ref{701}) bis (\ref{704})), die alle die Konstellation der Implikationsverstärkung aufweisen und für die keinerlei Akzeptabilitätsverlust anzunehmen ist.

\begin{exe}
	\ex\label{701} 
	Kai hat einen Hamster gekauft, \textbf{\textit{d.h.}/\textit{damit will ich sagen}/\textit{m.a.W.}} hat er ein Haustier 		gekauft.
\end{exe}
\vspace{-0.65cm}
\begin{exe}
	\ex\label{702} 
		\begin{xlist}	
			\ex\label{702a} Arndt zeigt einen Achter \textbf{(}Sportboot\textbf{)}.
			\ex\label{702b} Knut hält einen Ausweis \textbf{(}Schriftstück\textbf{)}.
		\end{xlist}
\end{exe}
\vspace{-0.65cm}
\begin{exe}
	\ex\label{703} 
	Dirk holt einen Hammer, einen Akkuschrauber, eine Wasserwaage, \textbf{\textit{allgemeiner}} Werkzeug.
\end{exe}
\vspace{-0.65cm}
\begin{exe}
	\ex\label{704} 
	Karl schenkt einen Teddy, \textbf{\textit{ja}} ein Spielzeug.
\end{exe}
Ohne die entsprechenden Einstufungen der Informationen (hier verdeutlicht durch die jeweiligen Funktionslexeme) gelten für diese Sätze die gleichen Abstufungen wie zwischen (\ref{699}) und (\ref{700}) (vgl. (\ref{705}) bis (\ref{708})).

\begin{exe}
	\ex\label{705} 
		\begin{xlist}	
			\ex\label{705a} Arndt zeigt ein Sportboot, er zeigt einen Achter.
			\ex\label{705b} \#Arndt zeigt einen Achter, er zeigt ein Sportboot.
		\end{xlist}
\end{exe}

\begin{exe}
	\ex\label{706} 
		\begin{xlist}	
			\ex\label{706a} Knut hält ein Schriftstück, er hält einen Ausweis.
			\ex\label{706b} \#Knut hält einen Ausweis, er hält ein Schriftstück.
		\end{xlist}
\end{exe}

\begin{exe}
	\ex\label{707} 
		\begin{xlist}	
			\ex\label{707a} Dirk holt ein Werkzeug, er holt einen Hammer.
			\ex\label{707b} \#Dirk holt einen Hammer, er holt ein Werkzeug.
		\end{xlist}
\end{exe}

\begin{exe}
	\ex\label{708} 
		\begin{xlist}	
			\ex\label{708a} Karl schenkt ein Spielzeug, er schenkt einen Teddy.
			\ex\label{708b} \#Karl schenkt einen Teddy, er schenkt ein Spielzeug.
		\end{xlist}
\end{exe}	
Mit der Implikationsverstärkung in (\ref{701}) bis (\ref{704}) gehen keine Akzeptabilitätseinbußen einher, da die implizierte Information hier aufgrund der ihr zugeschriebenen Funktion einen geringeren Mitteilungswert aufweist. Die genannten Funktionen wie Klarstellung, Erinnerung und Wiederholung sind dem eigentlichen Mitteilungsziel im Diskurs untergeordnet. Sie tragen nicht unmittelbar zu einem geraden Diskursverlauf bei, d.h. machen keine Hauptinformation aus im Zuge der Absicht, das bewusst geteilte Wissen zu erweitern. Umgekehrt sind Implikationsverstärkungen erst recht ausgeschlossen, wenn die implizierte Information noch hervorgehoben wird. Funktionslexeme wie \textit{vor allem}, \textit{genauer}, \textit{insbesondere} oder \textit{z.B.} sind deshalb nicht zulässig.

\begin{exe}
	\ex\label{709} 
	\#Dirk holt einen Hammer, \textbf{\textit{insbesondere}} holt er ein Werkzeug.
\end{exe}
\vspace{-0.65cm}
\begin{exe}
	\ex\label{710} 
	\#Karl schenkt einen Teddy, \textbf{\textit{vor allem}} schenkt er ein Spielzeug.
\end{exe}
\vspace{-0.65cm}
\begin{exe}
	\ex\label{711} 
	\#Arndt zeigt einen Achter, \textbf{\textit{genauer}} zeigt er ein Sportboot.
\end{exe}
\vspace{-0.65cm}
\begin{exe}
	\ex\label{712} 
	\#Knut hält einen Ausweis, \textbf{\textit{z.B.}} hält er ein Schriftstück.
\end{exe}
Präzisierungen, Hervorhebungen und die Nennung von Beispielen scheinen direkt relevant und mitteilungswürdig im Zuge eines geraden Diskursverlaufs. Es handelt sich nicht um Information, bei der man Grund zur Annahme hätte, dass sie für beide Diskursteilnehmer etwa schon bekannt oder ableitbar sein müsste. Es sind Informationen, mit denen der cg erweitert werden soll (zu den hier angeführten Funktionen und den Funktionslexemen vgl. vor allem \citealt{Schindler1990} in einer Arbeit zu \is{Zusatz} \textit{Zusätzen}, vgl. auch \citealt{Auer1991}, \citealt{Freienstein2008}).

Es lässt sich folglich festhalten, dass wenn ein Implikationsauslöser und seine Implikation gemeinsam auftreten, die redundante Abfolge nicht zulässig ist, wenn die beiden Informationen als \glq gleich wichtig\grq {} präsentiert werden bzw. wenn die redundante Information gerade hervorgehoben wird. Ein hoher Mitteilungswert bzgl. der Implikation ist somit ausgeschlossen. Die Abfolge Implikationsauslöser $>$ Implikation führt hingegen zu keinen markierten Strukturen, wenn die Implikation nicht dominant ist, weil ihr Informationswert im Diskurs in gewissem Sinne zurückgestuft ist. Ein Kontext, in dem derartige ansonsten redundanten Äußerungen auch zulässig scheinen, sind appositive Relativsätze \is{appositiver Relativsatz} (aRSe) (vgl. (\ref{713})).
\begin{exe}
	\ex\label{713} 
	Kai hat einen Hamster gekauft, der bekanntlich ein Haustier ist.
\end{exe}
In restriktiven Relativsätzen \is{restriktiver Relativsatz} (rRSen) kann diese Information hingegen nicht vor\-kommen (vgl. (\ref{714})).
\begin{exe}
	\ex\label{714} 
	\#Kai hat einen Hamster gekauft, der ein Haustier ist.
\end{exe}
Sicherlich haben diese Verhältnisse ihre Ursache auch in den beteiligten Extensionen, aufgrund derer sich die Ausdrücke überhaupt implizieren, und sind nicht nur in den unterschiedlichen Informationsstatus zu suchen, die ich zuschreibe. Gleiches gilt auch für die oben angeführten Fälle.

Vorausgesetzt, dass es sich bei der in aRSen enthaltenen Information, die be\-kanntlich nicht der Identifikation des Referenten des Bezugsnomens dient und des\-halb i.d.R. auch weglassbar ist, ebenfalls um Hintergrundinformation handelt, deren Mitteilung somit nicht direkt relevant für die cg-Erweiterung ist, bietet sich für die Akzeptabilität von (\ref{713}) dieselbe Erklärung an wie für die anderen Fälle, denen ich weiter oben einen geringeren Mitteilungswert zugeschrieben habe. Die Implikation selbst erhält unter den vermittelten Inhalten nur geringes Gewicht. (\ref{714}) ist genauso wie (\ref{709}) bis (\ref{712}) markiert, weil rRSen – vor dem Hintergrund ihrer Rolle, notwendige Angaben für die Identifikation des Bezugselements beizusteuern, – ein hoher Mitteilungswert zukommt. Das durch die Implikationsverstärkung auftretende Missverhältnis tritt dann auch hier verschärft auf, da der Inhalt der Implikation als besonders mitteilungswürdige Information ausgegeben wird (zum Zusammenhang zwischen Restriktivität/Appositivität und Vorder-/Hintergrundinformation s. ausführlicher Abschnitt~\ref{sec:rs}).

Ermöglicht eine geringere Dominanz der implizierten Information die Abfolge Implikationsauslöser $>$ Implikation und somit die eigentlich ausgeschlossene Verstärkung der Implikation, ist es eine Überlegung wert, ob es eine Möglichkeit gibt, den Beitrag von \textit{halt} in den Hintergrund treten zu lassen und dadurch einen akzeptablen Kontext für die Abfolge \textit{eben halt} zu schaffen.

Ich bin der Meinung, dass dies möglich ist, da MPn je nach Äußerungstyp unterschiedliches Gewicht haben können. Die MP \textit{hal}t wurde in Abschnitt~\ref{sec:kontexte} mit informativen Assertionen assoziiert, \textit{eben} anders mit ableitbarer Information. Aus Perspektive der cg-Anreicherung lässt sich dies derart fassen, dass \textit{halt} mit Äußerungsteilen mit höherem Mitteilungswert verbunden ist als \textit{eben}, das gerade auf cg-Inhalt verweist bzw. auf Inhalt, den man als bekannt unterstellt/auf den man sich schon geeinigt hat/hinsichtlich dessen kein Widerspruch zu erwarten ist. Ich nehme weiter an, dass wenn \textit{halt} und \textit{eben} in Äußerungen mit hohem Mitteilungswert gemeinsam auftreten, \textit{halt} dominant ist, sowie dass \textit{halt} in Äußerungen mit geringerem Mitteilungswert weniger dominant ist. Daraus ergibt sich im Zusammenhang mit den obigen Ausführungen die Hypothese, dass die Abfolge \textit{eben halt} in Kontexten mit hohem Mitteilungswert tendenziell nicht verwendet wird, da unter diesen Umständen die (abweichende) Implikationsverstärkung aufgrund des hervorgehobenen implizierten Inhalts in verschärfter Form auftritt. Konkreter Testboden für diese Hypothese sind im folgenden Abschnitt Relativsätze, in denen die Verteilung der Einzelpartikeln und ihrer Kombinationen untersucht werden.

\subsection{Relativsätze}
\label{sec:rs}
\subsubsection{Interpretation von appositiven und restriktiven Relativsätzen}\is{appositiver Relativsatz} \is{restriktiver Relativsatz}
\label{sec:interrs}
In \citet[15]{Bluehdorn2007} wird der Unterschied zwischen den beiden RStypen folgendermaßen gefasst: 

\begin{quotation}
Ein restriktiver RS liefert Information, die für die Akkomodierung der Haupt\-satz-Proposition im Kontext unentbehrlich ist, während die durch einen nicht-restriktiven RS gelieferte Information über den für die Akkomodierung der Hauptsatz-Proposition erforderlichen Mindestbedarf hinausgeht.
\end{quotation}
Die klassische Definition von rRSen und aRSen (hier entlang von \citealt[193]{Schaffranietz1997} formuliert) ist, dass erstere die Referenz des Bezugselements bzw. seiner Klasse einschränken und aufgrund dessen der Identifikation des Bezugselements dienen, während letztere keine derartige Klasseneinschränkung vornehmen, sondern zusätzliche Information beisteuern, die zur Identifikation des Bezugselements nicht notwendig ist. Ähnliche Formulierungen finden sich in \citet[12-13]{Buscha1983}, \citet[268-272]{Eisenberg2004}, \citet[18, 32]{Birkner2008}, \citet[42]{Zifonun1997} und \citet[99-100]{Fritsch1990}.  Etwas anders fasst \citet[399]{Huddleston1984} die Unterscheidung auf: 

\begin{quotation}
The most general account, I believe, involves initially a distinction of thematic meaning: in the non-restrictive construction, the information encoded in the relative clause is presented as separate from, and secondary to, that encoded in the remainder of the superordinate clause. In the restrictive construction on the other hand, the information contained in the relative clause forms an integral part of the message conveyed by the larger construction.
\end{quotation}
Vergleichbar mit \citet{Bluehdorn2007} bezieht sich seine Definition weniger auf ein Verhältnis von Mengen zueinander bzw. die Bildung von Untermengen. Er hebt vielmehr den unterschiedlichen Informationsstatus der RSe hervor. \citet[85-86]{Abraham2012} bringt ähnlich rRSe mit Vorder– und aRSe mit Hintergrundinformation in Verbindung. Erstere fasst er als vordergrundierte Attribute, letztere als hintergrundierte Modifikatoren auf.

\begin{quotation}
Während die restriktiven Relativsätze etwas hervorheben, was in der Text- und Redeführung noch nicht zum gemeinsamen Rede- und Texthintergrund (\textit{common ground}) gehört, so beziehen sich die nichtrestriktiven Re\-lativsätze auch auf den Sprecher und Hörer gemeinsamen Wissensstand.
\end{quotation}
Eindeutige rRSe finden sich in (\ref{715}) und (\ref{716}).

\begin{exe}
	\ex\label{715} 
	Der \underline{\textbf{\textit{Mann}}}, \underline{\textit{\textbf{der über uns wohnt}}}, ist 88 Jahre alt.
\end{exe}
\vspace{-0.65cm}
\begin{exe}
\ex\label{716} 
	Er ist ein \underline{\textit{\textbf{Typ}}}, \underline{\textit{\textbf{der sich vieles anhört}}}, bevor er ungeduldig 			wird.
	\newline
	\hbox{}\hfill\hbox{\citet[15-16]{Bluehdorn2007}}	
\end{exe}	
Ohne die RSe wäre es nicht möglich, die intendierte Menge an Referenten auszumachen.
Eindeutig appositive Fälle zeigen (\ref{717}) und (\ref{718}).
\begin{exe}
	\ex\label{717} 
	\underline{\textit{\textbf{Der Vater meiner Frau}}}, \underline{\textit{\textbf{der über uns wohnt}}}, ist 88 Jahre alt.
\end{exe}
\vspace{-0.65cm}
\begin{exe}
\ex\label{718} 
	So eine Säge ist \underline{\textbf{\textbf{ein gefährliches Werkzeug}}}, \underline{\textbf{\textit{mit dem vor allem Kinder}}} 			\underline{\textit{\textbf{auf\-passen müssen}}}.	
	\hfill\hbox{\citet[17]{Bluehdorn2007}}	
\end{exe}	
Hier geht die durch den RS vermittelte Information über die Information hi\-naus, die benötigt wird, um das Bezugsnomen zu beschreiben. Die RSe führen zusätzliche Informationen ein, sie sind in der Identifikation der Referenten aber entbehrlich.

Freie RSe \is{freier Relativsatz} werden in vielen Arbeiten zu den rRSen gezählt (vgl. \citealt[831-832]{Heidolph1981}, \citealt[62, 66-67], \citealt[1036]{Duden2009}, \citealt[294]{Lehmann1984}, \citealt[47]{Staffeldt1987}). Es handelt sich hierbei um RSe ohne ein explizites Bezugs\-nomen (vgl. (\ref{719}) bis (\ref{721})).

\begin{exe}
	\ex\label{719} 
	Wer zuerst kommt, gewinnt.
	\hfill\hbox {\citet[16]{Birkner2008}}
\end{exe}
\vspace{-0.65cm}
\begin{exe}
	\ex\label{720} 
	Hans kocht, was ihm schmeckt.
\end{exe}

\begin{exe}
	\ex\label{721} 
	Hans kocht das, was ihm schmeckt.
	\hfill\hbox {\citet[62]{Rothweiler1993}}
\end{exe}
Freie RSe treten auch als Teil von so genannten Pseudo-Cleft-Sätzen \is{Pseudo-Cleft-Satz} auf, wie in (\ref{722}) bis (\ref{724}).

\begin{exe}
	\ex\label{722} 
	Was du machst, ist bloß die Soße. (= das, was du machst, ist...)
\end{exe}
\vspace{-0.65cm}
\begin{exe}
	\ex\label{723} 
	Was ich will, ist ein Pferd.
\end{exe}
\vspace{-0.65cm}
\begin{exe}
	\ex\label{724} 
	Was er fand, waren zwei Männer.
	\hfill\hbox {\citet[360]{Lehmann1984}}
\end{exe}
Ich fasse diese RSe als rRSe auf. Ich werde diese (nicht ganz klare) Zuordnung an späterer Stelle wieder aufgreifen. Bei (\ref{722}) bis (\ref{724}) handelt es sich um \is{kanonischer Pseudo-Cleft-Satz} \textit{kanonische Pseudo-Cleft-Sätze}. Nachgestellte Varianten der Art in (\ref{725}) bis (\ref{727}) sind \is{invertierter Pseudo-Cleft-Satz} \textit{invertierte Pseudo-Cleft-Sätze}. 

\begin{exe}
	\ex\label{725} 
	Die Soße ist was du machst.
\end{exe}
\vspace{-0.65cm}
\begin{exe}
	\ex\label{726} 
	Ein Pferd ist was ich will.
\end{exe}
\vspace{-0.65cm}
\begin{exe}
	\ex\label{727} 
	Zwei Männer waren was er fand.
\end{exe}	
Wie (\ref{728}) zeigt, gibt es derartige Strukturen auch unter Ellipse.

\begin{exe}
	\ex\label{728} 
	Was der Frau des Catchers so sehr gefällt: seine friedfertige, häusliche Art. 
	\newline
	\hbox{}\hfill\hbox{(ZM 7/73: 17), \citet[74]{Dyhr1978}}	
\end{exe}
Basierend auf diesem semantisch-pragmatischen Unterschied zwischen rRSen und aRSen vertrete ich im Folgenden, dass rRSe einen höheren Mitteilungswert vorweisen als aRSe – aus der Perspektive, dass sie in der Situation notwendige Information vermitteln und nicht eine Zusatzinformation darstellen, auf die an dieser Stelle prinzipiell auch verzichtet werden könnte (vgl. auch \citealt[166-167]{Mueller2016a}). Bei rRSen hat der Sprecher ein größeres Interesse, ihren Inhalt zu vermitteln als bei aRSen. Ohne dass die Information aus dem rRS cg wird, kann auch die Gesamtassertion nicht in den cg gelangen. Hier fügt sich die Auffassung von \citet[86]{Abraham2012} gut ein, dass rRSe Vorder- und aRSe Hintergrundinformation beisteuern: \glqq Die $[$...$]$ nichtrestriktiven Relativsätze setzen den gemeinsamen Redehintergrund, während $[...]$ die Restriktiva Vordergründiges kodieren.\grqq{}  

Es handelt sich hier sicherlich um eine lokale Version von hohem Mitteilungs\-wert, da i.d.R. für beide Typen von RSen angenommen wird, dass sie keine Hauptinformation für den Gesamtkontext beisteuern (vgl. \citealt[38-42]{Antomo2015}.\footnote{Ausnahmen sind hier nomenbezogene weiterführende RSe (auf die ich in Abschnitt~\ref{sec:nwrs} zurückkommen werde) und indefinite, spezifisch interpretierte rRSe (vgl. \citealt[41-42]{Antomo2015}).}) Sicherlich ist die Zuordnung von Restriktiva zu hohem Mitteilungswert und Appositiva zu einem niedrigen Mitteilungswert dazu eine Vereinfachung, da es Fälle gibt, die sich dieser Assoziationen entziehen. Es sind dies auch letztlich Fälle, die sich der Unterscheidung in restriktive und appositive RSe überhaupt entziehen. Es kann z.B. schwierig werden, redundante RSe in einer derartigen Klassifikation unterzubringen oder RSe, die sehr wenig informativ sind. Beispiele (aus \citealt[14/17/33]{Weinert2004}, vgl. auch \citeyear[28-38]{Weinert2004}) finden sich in (\ref{729}).

\begin{exe}
	\ex\label{729} 
		\begin{xlist}	
			\ex\label{729a} die Teilnehmer, die du hast
			\ex\label{729b} die Dinge, die ich gefunden habe
			\ex\label{729c} Jeder, der dagewesen ist, liebt es.
			\ex\label{729d} Es gibt so viele Bücher, die man lesen kann.
		\end{xlist}
\end{exe}	
In meinen Daten sind mir derartige redundante RSe nicht prominent aufgefallen, man sollte aber darauf verweisen, dass eine Kategorisierung entlang von \textit{restriktiv} und \textit{appositiv} nicht ohne Probleme ist. Nicht umsonst sind alternative RS-Klassifikationen vorgeschlagen worden (vgl. z.B. \citealt[301-302]{Fox1990}). Analog zu derartigen wenig informativen rRSen (wenn man sie denn als Restriktiva auffassen möchte), gibt es auf Seiten der aRSe auch informative Verwendungen (vgl. z.B. die Funktionen, die aRSe nach \citealt{Loetscher1998} haben können). Insbesondere sind hier auch \textit{nomenbezogene weiterführende RSe} \is{nomenbezogener weiterführender Relativsatz} (\textit{d}-wRSe in \citealt{Holler2005} zu nennen:

\begin{exe}
	\ex\label{730} 
	Die Kinder wollten ihre Lehrerin besuchen, \textbf{die aber nicht zu Hause war}.
	\newline
	\hbox{}\hfill\hbox{\citet[85]{Holler2005}}
\end{exe}
Diese RSe sind zwar appositiv, es wird für sie aber gerade angenommen, dass sie eine Fortführung des Gesamtdiskurses bewirken. In diesem Sinne wäre hier nicht davon auszugehen, dass sie Hintergrundinformation oder Nebensächliches ausdrücken. \citet[272-273]{Lehmann1984} spricht hier z.B. von einem \textit{kontinuativen RS}, der \glqq textsemantisch wie ein Hauptsatz\grqq{} (\glq und er brachte es zur Bibliothek\grq {}) $[$fungiert$]$, \glqq etwas zum übergeordneten Ziel des Textes bei$[$trägt$]$\grqq{}  und  \glqq den Dis\-kurs voran$[$bringt$]$\grqq{}. Auch \citet[70]{Brandt1990} hält derartige RSe für  \glqq kommunikativ gleichrangig mit einem Hauptsatz\grqq{}. Ich werde später auf Sätze dieser Art zurückkommen. Auch diese RSe haben in den Korpusdaten nur einen eher geringen Anteil. Für den Moment fallen sie unter die aRSe. Ich werde das Bild später differenzieren und etwaige Auswirkungen auf die Zählung untersuchen.

\subsubsection{Grammatische Eigenschaften}
In der Literatur zu RSen wird angenommen, dass bestimmte grammatische Merkmale Hinweise darauf geben, ob ein restriktiver oder appositiver RS vorliegt (zu einem Überblick der Kriterien vgl. \citealt[182-183]{Schaffranietz1997}, \citealt{Becker1978}, \citealt[25-40]{Holler2005}, \citealt[32-51]{Birkner2008}, \citealt[263-267]{Lehmann1984}, \citealt[62-69]{Zifonun2001}). Diese grammatischen Eigenschaften stellen somit eine Hilfe für die Entscheidung der Zuordnung zu einem der beiden RStypen dar. Gleichzeitig besteht hier aber ggf. auch die Gefahr der voreiligen Zuordnung (s.u.).

Zu diesen klassischerweise aufgezählten Merkmalen gehört z.B. der Skopus \is{Skopus} des Determinans: Ein rRS und sein Bezugsnomen \is{Bezugsnomen} stehen im Skopus des glei\-chen Determinans, während aRSe außerhalb des Skopus des Determinans stehen. Diese verschiedenen Bezüge werden auch mit unterschiedlichen syntaktischen Verkettungen in Verbindung gebracht. Im Falle eines rRSes bilden die NP und der RS zusammen die komplexe NP, die von D selegiert wird. Beim aRS bildet der RS mit der DP (D NP) zusammen eine komplexe DP. Personalpronomen der 1./2. Person, Eigennamen sowie definite und generische Bezugsnomen können nur durch aRSe angeschlossen werden. Auf Pronomen wie \textit{diese}, \textit{jene}, \textit{jeder}, \textit{keiner}, \textit{derjenige}, \textit{niemand}, \textit{jemand}, \textit{wer} oder \textit{was} können nur rRSe folgen. Glei\-ches gilt für prädikative Bezugsnomen. Ein weiteres Unterscheidungskriterium ist die Adjazenz \is{Adjazenz} zum Bezugsnomen: rRSe erlauben eine non-adjazente Position zum Bezugsnomen einfacher als aRSe. Prosodisch sollen rRSe \is{prosodische Inregration/Desintegration} integriert, aRSe desintegriert sein. Im restriktiven Fall bilden das Bezugselement und der RS eine Intonationseinheit ohne abfallenden Tonhöhenverlauf vor dem RS. Im appositiven Fall liegt vor dem RS eine Pause vor und der Tonhöhenverlauf fällt vor dem RS ab. Weiteres Material, das sich nur in den aRS einfügen lässt, sind die Ausdrücke \textit{übrigens} und \textit{bekanntlich}. Im Gegensatz zum rRS kann ein aRS in eine Satzkoordination aus dem Hauptsatz und dem aRS umgeformt werden.

\citet[22]{Bluehdorn2007} führt zu vielen dieser Kriterien Gegenbeispiele und Mo\-difikationen oder Relativierungen an. Er hält viele zu Recht nur für \glqq Tendenzen\grqq{} oder \glqq sogar Fiktionen\grqq{} (\citeyear[30]{Bluehdorn2007}). \citet[191]{Schaffranietz1997} zeigt auf, dass die intonatorischen Kriterien nicht derart verlässlich sind und Intonation und Interpretation durchaus nicht einhergehen können. Interessant für meine spätere Studie ist das Beispiel in (\ref{731}).

\begin{exe}
	\ex\label{731} 
	\underline{\textit{\textbf{Das}}}, \underline{\textbf{\textit{was Maria sagt}}}, leuchtet mir ein.
	\hfill\hbox {\citet[27]{Bluehdorn2007}}
\end{exe}										           
Blühdorn führt an diesem Beispiel vor, dass Restriktivität vs. Appositivität auch allein kontextabhängig entscheidbar sein kann: Ist der Referent von \textit{das} ohne den RS identifizierbar, liegt ein aRS vor. Ist dies nicht möglich, handelt es sich um einen rRS. Auch \citet[193-194]{Schaffranietz1997} weist darauf hin, dass der pragmatische Aspekt oftmals vernachlässigt wird. Es scheint folglich unbedingt notwen\-dig, RSe im Kontext zu betrachten, um eine möglichst zuverlässige Zuordnung zu erzielen. Die Betrachtung konkreter Beispiele wird zeigen, dass die Interpretation der Sätze im Kontext das verlässlichste Kriterium darstellt, um zwischen Restriktivität und Appositivität zu unterscheiden, wenngleich es – auf eine gewisse Art widersprüchlich – gleichzeitig auch das unsicherste der angeführten Kriterien ist, weil es der Interpretation (und damit am ehesten der Varianz) unterliegt.

Wie oben erläutert, haben rRSe meiner Argumentation nach in dem Sinne einen höheren Mitteilungswert als aRSe, weil sie Vordergrundinformation kodie\-ren. Meine Hypothese ist, dass \textit{halt} dominant ist, wenn \textit{halt} und \textit{eben} in dieser Umgebung auftreten, und dass \textit{eben halt} aus diesem Grund in rRSen weniger auftreten sollte als \textit{halt eben}. Der aRS sollte anders der Kontext sein, in dem \textit{eben halt} auftritt (wenn \textit{eben halt} im RS auftritt), weil der Mitteilungswert in diesem Äußerungsteil nicht hoch ist. Der aRS ist als Umgebung für \textit{eben halt} geeigneter, weil durch das wenig dominante \textit{halt} das Problem der verstärkten Implikation weniger deutlich auftritt, als wenn das \textit{halt} noch dominant ist, wie im rRS. Aus meiner Hypothese und meinem Modell heraus gibt es zudem keinen Grund, warum \textit{halt eben} auch eine Präferenz für den einen oder anderen RS haben sollte.

\subsubsection{Modalpartikeln in Relativsätzen} 
\label{sec:mpnrs}
Meine Hypothese zur unterschiedlichen Verteilung von \textit{halt eben} und \textit{eben halt} in den zwei RStypen setzt die (wie ich im Folgenden zeigen werde) kontroverse Annahme voraus, dass MPn überhaupt in diesen beiden Satzkontexten auftreten können. Diese wird in der gängigen Sicht sowohl in deskriptiven als auch theoretischen Arbeiten verneint.

Generell wird von vielen Autoren angenommen, dass MPn in rRSen nicht stehen können. Diese Verteilung wird oftmals sogar als Test verwendet, um zu argumentieren, dass ein RS appositiv ist (vgl. z.B. \citealt[3]{Becker1978}, \citealt[2007]{Zifonun1997}, \citealt[30]{Holler2005}). 

Werden RS-Kontexte für das (unmögliche) Auftreten von MPn angeführt, werden m.E. i.d.R. sehr eindeutig restriktive bzw. appositive RSe genannt, der Art in (\ref{732}) und (\ref{733}).

\begin{exe}
	\ex\label{732} 
		\begin{xlist}	
			\ex\label{732a} \underline{\textit{\textbf{Diese großen Autos}}}, die \textbf{doch} mehr als 20l Benzin 						verbrauchen, sind unpraktisch.
			\hfill\hbox {\citet[166]{Helbig1994}}
			\ex\label{732b} \underline{\textbf{\textit{Peter}}}, der \textbf{ja} sonst immer zu spät kommt, kam dieses Mal 					überraschenderweise pünktlich.		  
			\hfill\hbox {\citet[135]{Dahl1988}}
		\end{xlist}
\end{exe}
\vspace{-0.65cm}
\begin{exe}
	\ex\label{733} 
	\underline{\textit{\textbf{Diejenigen}}}, die \textbf{(*ja/*doch)} politisch interessiert sind, gehen auch zur Wahl.
	\newline
	\hbox{}\hfill\hbox{\citet[30]{Holler2005}}	
\end{exe}
Wenn MPn in \glq rRSen\grq {} auftreten, wird davon ausgegangen, dass diese zu aRSen uminterpretiert werden (vgl. (\ref{734}) und (\ref{735})).

\begin{exe}
	\ex\label{734} 
	\underline{\textit{\textbf{Autos}}}, die laut sind, sollten mit einer geschlossenen Motorkapsel versehen werden.
\end{exe}
\vspace{-0.65cm}
\begin{exe}
	\ex\label{735} 
	\underline{\textit{\textbf{Autos}}}, die \textbf{ja} laut sind, sollten mit einer geschlossenen Motorkapsel versehen werden.
	\hfill\hbox {\citet[151]{Hartmann1986}}
\end{exe}
Bei (\ref{734}) kann es sich prinzipiell um einen restriktiven (= diejenigen Autos, die laut sind) oder appositiven (= alle Autos) RS handeln. (\ref{735}) erlaubt allerdings nur noch die appositive Interpretation (= alle Autos). Bei dieser Generalisierung, dass MPn nur in appositiven RSen stehen können, handelt es sich einerseits um einen deskriptiven Befund, bei dem es vor allem ältere Arbeiten dann auch belassen. Andererseits gibt es aber auch neuere Betrachtungen (der letzten 10 Jahre), in denen dieser Aspekt zum Bestandteil theoretischer Arbeiten gemacht wurde.

Das (angenommene) beschränkte Auftreten von MPn in aRSen tritt auf in Diskussionen zu so genannten \textit{Wurzelphänomenen} \is{Wurzelphänomen} (vgl. auch schon Kapitel~\ref{chapter:jud}, Abschnitt~\ref{sec:eingkon}). Hierunter fallen Phänomene, die eigentlich auf Hauptsätze be\-schränkt sind, aber prinzipiell auch in Nebensätzen vorkommen können (z.B. Verbzweiteinbettung, Topikalisierung, VP-Voranstellung, Topikmarkierungen $[$vgl. \citealt{Heycock2005} für einen Überblick$]$). Die entscheidende Annahme bzw. Beobachtung ist, dass diese Phänomene nur in bestimmten Nebensätzen auftreten können. Dies sind – nach Ansicht der Literatur – Nebensätze, für die angenommen wird, dass ihnen – trotz ihrer syntaktischen Abhängigkeit – illokutiv eine gewisse Eigenständigkeit zukommt. Sie werden in den Arbeiten von \citet{Haegeman2002, Haegeman2004, Haegeman2006} \textit{periphere} Nebensätze \is{peripherer Nebensatz} (pNSe) genannt. Das \glq Pendant\grq {}, d.h. die Nebensätze, die keine zulässige Domäne für Wurzelphänomene ausmachen, sind \textit{zentrale} Nebensätze \is{zentraler Nebensatz}(zNSe). Diese Unterscheidung hat strukturelle Reflexe in der internen und externen Syntax solcher Nebensätze. ZNSe haben eine reduzierte Struktur; in den Arbeiten von und nach Haegeman fehlt ihnen die \is{Force-Projektion} \textit{Force-Projektion}, in der syntaktisch die Illokution venkert ist. PNSe weisen diese entspre\-chend auf. Auch sind die zNSe tiefer in der Struktur verkettet, während die pNSe eine losere Verknüpfung mit der übrigen Struktur eingehen.

Relativsätze spielen in dieser Diskussion insofern eine Rolle, als dass sie auch eine derartige Trennung zeigen: rRSe erlauben keine Wurzelphänomene, aRSe erlauben sie aber. Beispielsweise können sich Tag-Fragen \is{Tag-Fragen} auf aRSe, aber nicht auf rRSe beziehen (vgl. (\ref{736})).

\begin{exe}
	\ex\label{736} 
		\begin{xlist}	
			\ex\label{736a} I just ran into Susan, who was your roommate at Radcliffe, wasn’t she?
			\ex\label{736b} *I just ran into the girl who was your roommate ar Radcliffe, wasn’t she?
			\hfill\hbox{\citet[490]{Hooper1973}}
		\end{xlist}
\end{exe}
Es gibt Autoren (z.B. \citealt{Coniglio2011}, \citealt{Frey2011, Frey2012}, \citealt{Abraham2012}), die annehmen, dass MPn zu den Wurzelphänomenen zählen. In diesen Arbeiten wird deshalb entsprechend die Annahme vertreten, dass sie in aRSen (= pNSe), jedoch nicht in rRSen (= zNSe) stehen können.

Nimmt man an, dass MPn unter die Wurzelphänomene fallen und aufgrund dessen auf pNSe beschränkt sind, ist die Betrachtung von MPn in rRSen abgeschlos\-sen. Unter dieser Ansicht können MPn in diesem Kontext als Klasse nicht vorkommen. Es gibt allerdings auch Stimmen in der Literatur, die für den Ausschluss von einzelnen MPn Begründungen anführen. Hinzu kommt, dass Daten angeführt werden, in denen MPn durchaus auch in rRSen auftreten.

\citet[160]{Hentschel1986} hat sich wie folgt dazu geäußert, warum \textit{ja} nicht gut im rRS stehen kann (vgl. auch \citealt[107]{Hartmann1977}, \citealt[210]{Rinas2006}, \citealt[52]{Kwon2005}):

\begin{quotation}
Restriktive Relativsätze sind stets zugleich notwendige Relativsätze, da sie ein Attribut enthalten, mit dessen Hilfe das Beziehungswort erst wirklich identifiziert werden kann. $[...]$ Informationen, die zur Identifizierung des vom Sprecher Gemeinten unabdingbar notwendig sind, können aber nicht gleichzeitig als \glq bekannt\grq {} markiert werden.
\end{quotation}
Vor diesem Hintergrund spricht nichts dagegen, dass derartige Information als aus Sprechersicht plausibel oder klar ausgegeben wird, die Begründung für einen anderen Sachverhalt ist oder aus einer anderen (gegebenen) Information folgt und deshalb als (für beide Diskurspartner) ableitbar ausgegeben wird. Es scheint mir somit nichts dagegen zu sprechen, dass \textit{halt} im rRS auftritt. Da \textit{eben} (im Gegensatz zu \textit{ja}) mehr als den Bedeutungsanteil \glq Bekanntheit\grq {} transportiert (nämlich Begründung/Folge, Ableitbarkeit), trifft die obige Erklärung zum Ausschluss von \textit{ja} auch auf diese Partikel nicht direkt zu. Ich halte es auch für keinen Zufall, dass in den Beispielen, die in der Literatur angeführt werden, fast ausschließlich \textit{ja} auftritt. Ein Grund dafür ist sicherlich, dass diese MP sehr viel und ausführlich untersucht worden ist. Vermutlich liegt dieser Umstand aber auch darin begründet, dass genau diese MP im rRS einfach nicht stehen kann. Weitere Forschung müsste hier zeigen, inwiefern diese Beschränkung für MPn als Klasse Gültigkeit hat. Für die Argumentation meiner Arbeit ist an dieser Stelle wichtig, dass in der Literatur auch Beispiele angeführt werden, in denen MPn in rRSen auftreten (vgl. (\ref{737}) bis (\ref{739})).

\begin{exe}
	\ex\label{737} 
	(?)\underline{\textbf{\textit{Die von euch}}}, \underline{{\textit{\textbf{die}} \textbf{JA} \textit{\textbf{keinen Ärger wollen}}}}, kommen am besten gar \\nicht erst mit zur Demo.
	\hfill\hbox {\citet[202]{Hentschel1986}}
\end{exe}

\begin{exe}
	\ex\label{738} 
	Es muß anerkannt werden, daß an unseren Schulen \underline{\textit{\textbf{Kinder}}} sind, \underline{\textbf{\textit{die 		}}} \underline{\textbf{\textit{Deutsch}} \textbf{eben \textit{nicht als Muttersprache sprechen}}}. 		
	\hfill\hbox {(TAZ, 15.06.1998, 23)}
\end{exe}	

\begin{exe}
	\ex\label{739} 
	Da nehmen wir mal \underline{\textit{\textbf{Dinge}}} mit, \underline{\textbf{\textit{die man} halt \textit{so braucht}}}.
	\newline
	\hbox{}\hfill\hbox {(TAZ, 19.08.1995, 32)}
	\newline
	\hbox{}\hfill\hbox{\citet[68]{Kwon2005}}	
\end{exe}					
Neben der noch gar nicht nachgewiesenen generellen Unverträglichkeit von MPn und Restriktivität verwundert dieser kategorische Ausschluss auch aus dem Grund, dass man MPn auch in erweiterten Attributen finden kann, die restriktiv zu lesen sind (vgl. (\ref{740}) und (\ref{741})).

\begin{exe}
	\ex\label{740} 
	Man muß \underline{\textbf{\textit{die} halt \textit{richtigen und verantwortlichen Leute}}} wegschicken, dann klappt's 		auch mit dem dritten Wiederaufstieg.		
	\newline
	\hbox{}\hfill\hbox {(DECOW2012-05X: 309288190)}
\end{exe}
\vspace{-0.65cm}
\begin{exe}
	\ex\label{741} 
	Yep, es ist zwar kein Film, der einen voll aus dem Sessel reißt, aber es ist \underline{\textbf{\textit{ein} einfach 			\textit{schöner Film}}} ... 	
	\hfill\hbox {(DECOW2012-06: 1177669381)}
\end{exe}	              
In beiden Sätzen liegt ein indefinites Bezugsnomen vor, das zur korrekten Identifikation die attributiv beigefügten Charakterisierungen benötigt. In (\ref{740}) geht es gerade darum, die richtigen und verantwortlichen Leute zurückzuschicken. In (\ref{741}) soll zum Ausdruck gebracht werden, dass es ein Film ist, der schön ist, und nicht, dass es überhaupt ein Film ist. Dies ist hier – auch unter Hinzunahme des durch den ersten Teilsatz entstehenden Kontrasts – die einzig zulässige Interpretation. Formuliert man die erweiterten Attribute in RSe um (vgl. (\ref{742}) und (\ref{743})) liegen eindeutig restriktive RSe vor.

\begin{exe}
	\ex\label{742} 
	Man muss \underline{\textit{\textbf{die Leute}}} wegschicken, \underline{\textbf{\textit{die} halt \textit{die richtigen 		und verantwort}}}\- \underline{\textit{\textbf{lichen sind}}}.
\end{exe}
\vspace{-0.65cm}
\begin{exe}
	\ex\label{743} 
	Es ist \underline{\textbf{\textit{ein Film}}}, \underline{\textbf{\textit{der} einfach \textit{schön ist}}}.
\end{exe}
M.E. lässt sich nicht vertreten, dass MPn aus dem Kontext des rRSes kategorisch auszuschließen sind.

\subsubsection{\textit{Halt}, \textit{eben}, \textit{halt eben} und \textit{eben halt} in Relativsätzen}
Um die Hypothese zu testen, dass die MP-Kombinationen aus \textit{halt} und \textit{eben} nicht identisch verwendet werden, habe ich mir im Korpus DECOW2012 alle RSe, in denen \textit{halt eben} (231 Treffer) und \textit{eben halt} (59 Treffer) auftreten sowieso für ein Teilkorpus RSe mit den Einzelpartikeln (\textit{halt}: 119 Treffer, \textit{eben}: 266 Treffer)\footnote{Berücksichtigt wurden (nach Bereinigung der gefundenen irrelevanten Strukturen) alle \\ Treffer für Relativsätze der folgenden Suchanfrage im Teilkorpus DECOW2012-C06X7M:
$[$word= \glqq der$\vert$das$\vert$den$\vert$die$\vert$der$\vert$dem$\vert$dessen$\vert$was$\vert$wo$\vert$wen$\vert$wer$\vert$wohin$\vert$welcher$\vert$welche$\vert$welches$\vert$\\wofür$\vert$wo\-durch$\vert$womit$\vert$woran\grqq{}$][]\lbrace$0,4$\rbrace[$word= \glqq halt\grqq{}$][]\lbrace$0,4$\rbrace[$pos= \glqq VVFIN\grqq{}$]$} in allen Fällen im Kontext angeschaut. (\ref{744}) bis (\ref{751}) zeigen einige Beispiele ((\ref{744}), (\ref{746}), (\ref{748}), (\ref{750}) restriktiv; (\ref{745}), (\ref{747}), (\ref{749}), (\ref{751}) appositiv).

\begin{exe}
	\ex\label{744} 
		\begin{xlist}
		\ex\label{744a} 
		\scriptsize
		ich habe die Stoffe von Buttinette hier schon sehr lange liegen weil man diese Stoffe 											höchstens für eine Tasche, Wandbild oder \underline{\textbf{\textit{Sachen}}} \underline{\textbf{\textit{die man} halt 			\textit{so gut wie nie wäscht}}}, verwenden kann. 
		\newline	
	 	\hbox{}\hfill\hbox{(DECOW2012-C06X7M: 46297826)}
	 	\newline
	 	\hbox{}\hfill\hbox{\citet[168]{Mueller2016a}}
	 	\ex\label{744b} 
	 	\scriptsize
	 	Ben: Ich muss zu meiner Schande gestehen, dass Ich darüber kaum etwas weiß, nur 												\underline{\textbf{\textit{das}}} \underline{\textbf{\textit{was man}}} \underline{\textbf{halt \textit{liest}}} und 			das klingt nach einer großen Sache und super Werbestrategie.
	 	\newline
	 	\hbox{}\hfill\hbox{(DECOW2012-C06X7M: 49949863)}
		\end{xlist}
\end{exe}

\begin{exe}
	\ex\label{745} 
		\begin{xlist}	
		\ex\label{745a} 
		\scriptsize
		Na, ich denk, wenn mal wieder ein RL unserem Reiten beiwohnen würde, wäre das nicht mehr
	 	so positiv :lach : – da werden sich schon \underline{\textbf{\textit{einige Fehler}}} eingeschlichen haben, 					\underline{\textbf{\textit{die ich aber grad} halt}} \underline{\textit{\textbf{nicht merke}}} ...
	 	\hfill\hbox{(DECOW2012-C06X7M: 100768300)}
	 	\newline
	 	\hbox{}\hfill\hbox{\citet[168]{Mueller2016a}}
	 	\ex\label{745b} 
	 	\scriptsize
	 	Ich pachte z.B. kein Fabrik-Gebäude und reklamier dann beim Verpächter, dass mir meine Post 									\underline{\textbf{\textit{aus dem verrosteten Briefkasten}}} geklaut wird, \underline{\textbf{\textit{der} halt 				\textit{neben der Eingangstür hängt}}} ... Der Vergleich hinkt, zeigt aber was ich meine ...                  
		\hfill\hbox{(DECOW2012-C06X7M: 25072949)}
		\end{xlist}
\end{exe}			
	
\begin{exe}
	\ex\label{746} 
		\begin{xlist}	
		\ex\label{746a} 
		\scriptsize
		Sie lernten sich zwischen Geldanlagen und Spekulationen kennen – also genau
	 	an dem Ort, \underline{\textbf{\textit{wo sich Körpermusiker} eben \textit{kennen lernen}}}. 
	 	\hfill\hbox{(DECOW2012-C06X7M: 78832114)}
	 	\newline
	 	\hbox{}\hfill\hbox{\citet[168]{Mueller2016a}}
	 	\ex\label{746b} 
	 	\scriptsize
	 	Nur doof, wenn man sich immer von \underline{\textit{\textbf{Sachen}}} bedient, \underline{\textbf{\textit{die der Fan} 		eben \textit{kennt}}}.     
	 	\newline
	 	\hbox{}\hfill\hbox{(DECOW2012-C06X7M: 305992539)}
		\end{xlist}
\end{exe}
 					                     
\begin{exe}
	\ex\label{747} 
		\begin{xlist}	
		\ex\label{747a} 
		\scriptsize
		Bei Artefakten gibt es dann teilweise auch \underline{\textbf{\textit{mehrere verschiedene}}}, 									\underline{\textbf{\textit{die} eben \textit{bei häufiger Benut}}}\-\underline{\textbf{\textit{zung öfter vorkommen}}}.       
	 	\hfill\hbox{(DECOW2012-C06X7M: 70963084)}
	 	\newline
	 	\hbox{}\hfill\hbox{\citet[168]{Mueller2016a}}
	 	\ex\label{747b} 
	 	\scriptsize
	 	Die Romulaner in TOS etwa waren \underline{\textbf{\textit{absolut dreidimensionale Gegenspieler}}}, 							\underline{\textbf{\textit{die} eben \textit{auf einer}}} \underline{\textit{\textbf{anderen Seite standen als Kirk}}}, 		aber nichtsdestotrotz auf ihre Weise ehrenhaft. 	              
		\newline
	 	\hbox{}\hfill\hbox{(DECOW2012-C06X7M: 45833929)}
		\end{xlist}
\end{exe}

\begin{exe}
	\ex\label{748} 
		\begin{xlist}	
		\ex\label{748a} 
		\scriptsize
		Liebe Mitchristen, das großartige Geschenk der Taufe wird bei uns größtenteils
		als reine Familienfeier angesehen, oder als \underline{\textbf{\textit{etwas}}}, \underline{\textbf{\textit{was man} 			halt eben \textit{macht}}}. 
	 	\hfill\hbox{(DECOW2012-01: 763979523)}
	 	\newline
	 	\hbox{}\hfill\hbox{\citet[168]{Mueller2016a}}
	 	\ex\label{748b} 
	 	\scriptsize
	 	Aber ich denke es reicht, ganz allgemein zu sagen, dass das hier meine Primärempfehlung für 									\underline{\textbf{\textit{alle}}} ist, \underline{\textbf{\textit{die} halt eben \textit{kein Japanisch können}}} und 			deshalb nicht in den Genuss des Originals gelangen.                                                                      
		\hfill\hbox{(DECOW2012-05: 898872102)}
		\end{xlist}
\end{exe}

\begin{exe}
	\ex\label{749} 
		\begin{xlist}	
		\ex\label{749a} 
		\scriptsize
		Zug um Zug werden \underline{\textbf{\textit{diese Gebiete}}}, \underline{\textbf{\textit{die} halt eben 						\textit{Glasfaser-Gebiete sind}}}, mit T-DSL ausgebaut – natürlich immer unter der Prämisse, daß es eben auch 					wirtschaftlich ist.
		\newline		
	 	\hbox{}\hfill\hbox{(DECOW2012-01: 73171502)}
	 	\newline
	 	\hbox{}\hfill\hbox{\citet[168]{Mueller2016a}}
	 	\ex\label{749b} 
	 	\scriptsize
	 	Ich denke mal, in meinem Sohn steckt oft noch \underline{\textbf{\textit{eine ganz andere Person}}}, 							\textbf{\textbf{\textit{die}} halt eben \textit{\textbf{durch diese autistische Behinderung natürlich oft so nicht 				agieren kann}}} und so rauskommt wie man es eigentlich sich wünscht oder wie er es sich vielleicht wünschen würde. 
		\newline		
	 	\hbox{}\hfill\hbox{(DECOW2012-02: 197799301)}
		\end{xlist}
\end{exe}

\begin{exe}
	\ex\label{750} 
		\begin{xlist}	
		\ex\label{750a} 
		\scriptsize
		Es hat schon einen Sinn, du wirst das was du gibst eines Tages zurück bekommen, da bin ich
	 	mir ganz sicher! Du bist einfach \underline{\textbf{\textit{ein Mensch}}} \underline{\textbf{\textit{der} eben halt 			\textit{gerne helfen tut}}}, $[$...$]$ und dabei hälst du dich selber an der kurzen leine, du must auch an dich denken.	
	 	\hfill\hbox{(DECOW2012-03: 959102858)}
	 	\newline
	 	\hbox{}\hfill\hbox{\citet[168]{Mueller2016a}}
	 	\ex\label{750b} 
	 	\scriptsize
	 	Das ärgerte Jim zwar, aber er schimpfte deswegen nicht, weil es eh keinen Sinn hatte. Alicia war 								\underline{\textbf{\textit{jemand}}}, \underline{\textbf{\textit{der} eben halt \textit{liebend gern feierte anstatt 			hart zu arbeiten}}}, weshalb Jim auf Delilah zurückgriff, die ja ohnehin viel tüchtiger war, als ihre jüngere 					Schwester.
		\newline		
	 	\hbox{}\hfill\hbox{(DECOW2012-04: 779263058)}
		\end{xlist}
\end{exe}								                  

\begin{exe}
	\ex\label{751} 
		\begin{xlist}	
		\ex\label{751a} 
		\scriptsize
		Letztendlich ist das aber viel zu wenig. Normalerweise, ähnlich wie es in Naturparks ist, gibt es 								\underline{\textbf{\textit{Ranger-Abteilungen}}}, \underline{\textbf{\textit{die} eben halt \textit{mit einfachen 				Mitteln dafür sorgen, dass der Tierbe}}}- \\ \underline{\textit{\textbf{stand in vernünftigem Maß reguliert wird}}}. 
	 	\hfill\hbox{(DECOW2012-00: 766341958)}
	 	\newline
	 	\hbox{}\hfill\hbox{\citet[168]{Mueller2016a}}
	 	\ex\label{751b} 
	 	\scriptsize
	 	Das gilt auch für die Bahn, die müsste rechtzeitig dann halt zusätzlich \underline{\textbf{\textit{Arbeitskräfte}}} 			einstellen, \underline{\textbf{\textit{die} eben halt \textit{auch mal räumen können}}} oder eine Weiche freifegen 				können.
		\newline		
	 	\hbox{}\hfill\hbox{(DECOW2012-02: 87071840)}
		\end{xlist}
\end{exe} 
Bevor Abschnitt~\ref{sec:ergeb} aufzeigt, dass diese Studie Evidenz für die obige Hypothese liefert, sollen in Abschnitt~\ref{sec:ergeb} zunächst einige Punkte angeführt werden, die Entscheidungen über vorgenommene Kategorisierungen betreffen.

\subsubsubsection{Berücksichtigte Strukturen}
\label{sec:berstruk}
In (\ref{752}) bis (\ref{755}) finden sich Beispiele für \is{freier Relativsatz} freie RSe, die ich zu den rRSen zähle.
	
\begin{exe}
	\ex\label{752} 
	\scriptsize
	Bei zuviel Futter wählerisch geworden ... davor musste es essen \underline{\textbf{\textit{was} halt \textit{grad kam}}} 		... aber jetzt?
	\newline
	\hbox{}\hfill\hbox {(DECOW2012-C06X7M: 47856279)}
\end{exe}
\vspace{-0.65cm}
\begin{exe}
	\ex\label{753} 
	\scriptsize
	Und \underline{\textbf{\textit{wer sich es} eben \textit{nicht leisten kann}}} bleibt eben draußen.
	\newline		
	\hbox{}\hfill\hbox{(DECOW2012-C06X7M: 78983194)}
\end{exe}			       

\begin{exe}
	\ex\label{754} 
	\scriptsize
	Auch beim folgenden \glqq Summertime\grqq{}-Song waren die Zuhörer vollauf begeistert und nahmen die Technikaussetzer 			größtenteils sportlich hin, zumal die drei Jungs auf der Bühne das Letzte an Power investierten, um zu retten, 					\underline{\textbf{\textit{was man als Künstler unter solchen Bedingungen} eben halt \textit{retten kann}}}.                                               
	\newline		
	\hbox{}\hfill\hbox{(DECOW2012-04: 252720554)}
\end{exe}
 
\begin{exe}
	\ex\label{755} 
	\scriptsize
	Ich bewerte diese Aussagen einfach so, dass die Gemeinde zum einen einfach nur deutlich hervorgehoben hat, 						\underline{\textbf{\textit{was} halt eben \textit{sowieso verboten ist}}} und nicht nur, weil es so in der Verordnung steht 	und zum anderen wohl auch noch eigene Verbote verhängt hat. 
	\hfill\hbox{(DECOW2012-00: 592344716)}
\end{exe} 
Die restriktive Lesart scheint mir hier unstrittig.

Problematischer sind Fälle wie in (\ref{756}) bis (\ref{759}). 

\begin{exe}
	\ex\label{756} 
	\scriptsize
	\underline{\textbf{\textit{was mir} halt \textit{am meisten spaß macht}}} ist das leute quälen in den bodyshape kursen zb 			hihihi wenn se rauskriechen.                                                                            
	\hfill\hbox{(DECOW2012-C06X7M: 35154336)}
\end{exe} 

\begin{exe}
	\ex\label{757} 
	\scriptsize
	Jedenfalls kann man sagen was man will die Musik aus der Richtung is schon ziemlich geil und ein paar von den Leuten sehen 		auch ganz niedlich aus ... \underline{\textbf{\textit{was} eben \textit{ins Auge fällt}}} is wie üblich sowas, was für ein 		lebender Fail! 	                                                                            
	\hfill\hbox{(DECOW2012-C06X7M: 62382868)}
\end{exe} 	 
	
\begin{exe}
	\ex\label{758}
	\scriptsize 
	\underline{\textbf{\textit{Was mir} halt eben \textit{verstärkt zur Zeit auffällt}}} sind die blauen Finger und Lippen wenn 		ich mich anstrenge und kalter Schweiß ...  	                                                                            
	\hfill\hbox{(DECOW2012-03: 221281302)}
\end{exe} 				         
			       					                      
\begin{exe}
	\ex\label{759} 
	\scriptsize
	\textbf{\textit{Was} eben halt \textit{untypisch bei deinem hier wäre}}, wären die ganzen bunten Schatten da bei der Hose 			und auch dieses stark bläuliche beim Fell im Nacken	                                                                            
	\hfill\hbox{(DECOW2012-07: 268741617)}
\end{exe} 
 Es handelt sich hierbei um \is{Pseudo-Cleft-Satz} \textit{Pseudo-Cleft-Sätze}. In Arbeiten, die diese Struktur näher betrachten (vgl. z.B. \citealt[360]{Lehmann1984}, \citealt[Kapitel 8]{Birkner2008}), ließen sich wenige bis keine detaillierten Aussagen zur Frage nach Restriktivität vs. Appositivität finden, die darüber hinausgehen, dass die freien RSe zu den restriktiven gezählt werden. In der \citet[1036]{Duden2009} heißt es, die RSe würden hier spezifizieren, und auch in alten Arbeiten (vgl. \citealt[95, 100]{Motsch1970}, \citealt[81]{Valgard1971}) wird davon ausgegangen, dass ein rRS beteiligt ist. Betrachtet man z.B. die Interpretation von (\ref{760}) näher, bietet sich m.E. nur die restriktive Lesart an.

\begin{exe}
	\ex\label{760} 
	\underline{\textbf{\textit{Was mir} halt eben \textit{verstärkt zur Zeit auffällt}}} sind die blauen Finger und Lippen.
\end{exe} 
Der Sprecher bringt zum Ausdruck, dass er im Folgenden über die Menge der Dinge reden wird, die ihm verstärkt zur Zeit auffallen. Lässt man den RS weg, resultiert nur noch eine Identitätsaussage.

\begin{exe}
	\ex\label{761} 
	Das sind die blauen Finger und Lippen.
\end{exe} 
Die Bedeutung von (\ref{760}) lässt sich wiedergeben entlang von (\ref{762}) bis (\ref{765}).

\begin{exe}
	\ex\label{762} 
	$[$Das, was mir verstärkt zur Zeit auffällt$]$ sind die blauen Finger und Lippen.		
\end{exe}
\vspace{-0.65cm}
\begin{exe}
	\ex\label{763} 
	Die blauen Finger und Lippen sind $[$das, was mir zur Zeit verstärkt auffällt$]$.		
\end{exe}
\vspace{-0.65cm}
\begin{exe}
	\ex\label{764} 
	Zu $[$den Dingen, die mir zur Zeit verstärkt auffallen$]$, gehören die blauen Finger und Lippen.		
\end{exe}	
\vspace{-0.65cm}
\begin{exe}
	\ex\label{765} 
	Die blauen Finger und Lippen sind $[$etwas, das mir zu Zeit verstärkt auffällt$]$.		
\end{exe}	
(\ref{760}) wird aber nicht aufgefangen durch (\ref{766}) und (\ref{767}).

\begin{exe}
	\ex\label{766} 
	Das$_{1}$ – es$_{1}$ fällt mir zur Zeit verstärkt auf – sind die blauen Finger und Lippen.		
\end{exe}
\vspace{-0.65cm}
\begin{exe}
	\ex\label{767} 
	Das$_{1}$ sind die blauen Finger und Lippen und das$_{1}$ fällt mir zur Zeit verstärkt auf.		
\end{exe}
Derartige Umformungen sind mit aRSen aber normalerweise möglich, wie (\ref{768}) bis (\ref{770}) zeigen.

\begin{exe}
	\ex\label{768} 
	Der Vater meiner Frau, der über uns wohnt, ist 88 Jahre alt.		
\end{exe}
\vspace{-0.65cm}
\begin{exe}
	\ex\label{769} 
	[Der Vater meiner Frau]$_{1}$ – er$_{1}$ wohnt über uns – ist 88 Jahre alt.		
\end{exe}	
\vspace{-0.65cm}
\begin{exe}
	\ex\label{770} 
	[Der Vater meiner Frau]$_{1}$ ist 88 Jahre alt und er$_{1}$ wohnt über uns.	
\end{exe}
Ein Grund, weshalb man die freien RSe hier möglicherweise nicht als restriktiv einstufen möchte, ist, dass \textit{bekanntlich} und \textit{übrigens} durchaus auftreten können, die als die klassischen Marker für aRSe angesehen werden (vgl. (\ref{771})).

\begin{exe}
	\ex\label{771} 
	$[$Was mir bekanntlich/übrigens zur Zeit auffällt$]$ sind die blauen Finger und Lippen.
\end{exe}	
M.E. schließt sich das Vorkommen dieser Adverbien aber gar nicht mit der Annahme aus, dass die RSe die Menge, über die geredet wird, einschränken und in diesem Sinne notwendig sind. Es spricht nichts dagegen, dass ein Sprecher mitteilt, dass er über diejenigen Dinge spricht, die ihm zur Zeit auffallen und dass er es als bekannt oder Nebeninformation ausgibt, dass sie ihm zur Zeit auffallen. 

M.E. führt das Auftreten von \textit{bekanntlich} bzw. \textit{übrigens} in Pseudo-Cleft-Sätzen folglich nicht dazu, dass die Einschränkung auf die Menge der Dinge, die dem Sprecher auffallen, ausbleibt, und der RS deshalb keinen lokal relevanten Beitrag leistet. Den relevanten Beitrag leistet der RS nicht unbedingt für den Gesamtdiskurs, er tut dies aber für die Referenz des impliziten \textit{das}. Ohne den RS wäre die relevante Menge nicht zu identifizieren.

Derartige Sätze können auch nachgeordnet vorkommen (vgl. (\ref{772}) bis (\ref{774})).

\begin{exe}
	\ex\label{772} 
	Die Liederlichkeit ist was er nicht verträgt.
	\hfill\hbox {\citet[327]{Birkner2008}}
\end{exe}
\vspace{-0.65cm}
\begin{exe}
	\ex\label{773} 
	Die Frauen sind ?wer/was Verwirrung stiftet.
	\hfill\hbox {\citet[13]{Altmann2009}}
\end{exe}
\vspace{-0.65cm}
\begin{exe}
	\ex\label{774} 
	Champagner ist was ich mag.
	\hfill\hbox {nach \citet[467]{Lambrecht2001}}
\end{exe}
Ebenfalls möglich sind elliptische Varianten (vgl. (\ref{775}) und (\ref{776})).

\begin{exe}
	\ex\label{775} 
	\scriptsize
	Was noch wichtiger war: Ministerpräsident Kohl hat klargemacht, daß die Union bereit ist, \glqq zukünftig auf der Grundlage 	der Verträge mit der Bundesregierung zusammenzuarbeiten\grqq{}. 	
	\newline
	\hbox{}\hfill\hbox{(ZEIT 7/73:1), \citet[74-75]{Dyhr1978}}	
\end{exe}

\begin{exe}
	\ex\label{776} 
	\scriptsize
	Der alte Weinhändler liegt auf dem Totenbett und hat seine Söhne um sich versammelt: \glqq Was ich euch noch sagen wollte: 		Wein kann man auch aus Trauben machen\grqq{}. 	
	\newline
	\hbox{}\hfill\hbox{(STERN 45/73:13), \citet[93]{Dyhr1978}}	
\end{exe}
Vor diesem Hintergrund scheint es mir auch denkbar, dass die invertierte Variante elliptisch auftritt. Als so zu klassifizierende Strukturen fasse ich Sätze der Art in (\ref{777}) auf.

\begin{exe}
	\ex\label{777} 
	\scriptsize
	Langsam steckt er Teil für Teil ein. Zuerst einen mir völlig normal erscheinenden Damenslip, dann einen roten Seidentanga, 		eine handvoll BH's und zu guter letzt Damenstrümpfe, \textbf{\textit{was} halt eben \textit{noch so reingeht, in seinen 		Mantel}}. 
	\hfill\hbox {(DECOW2012-00: 458299540)}
\end{exe}
\citet[95]{Higgins1976} schreibt, die spezifizierende Funktion von Pseudo-Cleft-Sätzen würde Auflistungen ähneln, wie sie auch durch Sätze, die \textit{folgendes} ent\-halten und denen sich eine Liste anschließt, ausgedrückt werden (vgl. (\ref{778}) und (\ref{779})).
	
\begin{exe}
	\ex\label{778} 
	What I bought was a punnet of strawberries and a pint of clotted cream.
\end{exe}	
\vspace{-0.65cm}	
\begin{exe}
	\ex\label{779} 
	I bought the following things: a punnet of strawberries and a pint of clotted cream.
\end{exe}	
Die Fälle in meinen Daten, die ich als invertierte Pseudo-Cleft-Sätze kategorisiere, sind gerade derart beschaffen, dass zunächst die Exemplare einer Liste genannt werden und der Pseudo-Cleft-Satz als eine Art Zusammenfassung oder Abstraktion über die zuvor angeführten Exemplare folgt. 

Die nachgestellten Varianten wirken genauso einschränkend wie die vorange\-stellten und lassen sich am ehesten durch rRSe wiedergeben, wie in (\ref{780}).

\begin{exe}
	\ex\label{780} 
	Das, was er mitnimmt (Slips, BH, Strümpfe), ist $[$das, was noch in seinen Mantel reingeht$]$.
\end{exe}
Es ist bekannt, dass nicht stets eindeutig zu entscheiden ist, ob ein RS restriktiv oder appositiv ist. Es gibt zwar Anzeichen in Form grammatischer Eigenschaften (s.o.), doch wie bereits angemerkt, handelt es sich selbst bei diesen vermeintlich \glq  harten\grq {} grammatischen Kriterien auch nur um Tendenzen. Eine Betrachtung der Sätze im Kontext scheint mir deshalb unabdingbar. Dennoch bleiben aber Fälle, in denen die Zuordnung immer noch nicht völlig eindeutig ist. Aufgrund dieser sich mitunter einstellenden Unschärfe des Phänomenbereichs halte ich es für sehr wichtig, für die Klassifikationsentscheidungen der Daten klare Kriterien festzulegen und diese transparent zu machen. Im vorliegenden Fall habe ich die RSe defensiv bewertet, d.h. ich habe Restriktivität angenommen, wenn ich Appositivität für ausge\-schlossen halte. Das führt in Einzelfällen natürlich ggf. auch zu Entscheidungen, die man anders hätte fällen können. Da alle Datensätze \\ gleich defensiv betrachtet wurden, kann man m.E. aber gar nicht direkt sagen, ob es zu grundsätzlich anderen Ergebnissen käme, wenn man das Kriterium änderte.

Unter appositive RSe fallen in meiner Klassifikation deshalb auch z.B. Fälle, in denen das Bezugsnomen aufgrund seiner unmittelbaren Vorerwähntheit salient und damit identifizierbar ist. In (\ref{781}) z.B. ist \textit{den Zeitrau}m salient, da er ja unmittelbar vorher beschrieben wird, d.h. er kann gut als identifizierbar gelten.
	
\begin{exe}
	\ex\label{781} 
	\scriptsize
	wenn ein vampir also den schalter umlegt, dass er nichts mehr fühlt, also auch keine schuld mehr, gilt das dann nur für 		\underline{\textbf{\textit{den zeitraum}}}, \underline{\textbf{\textit{in dem der schalter} halt eben \textit{umgelegt 			ist}}}? und wenn der vampir dann den schalter wieder zurück umlegt auf fühlen, erwischt ihn dann die ganze geballte zurück-		liegende verursachte schuld wie son schlag ins gesicht?                  
	\hfill\hbox {(DECOW2012-01: 83801655)}
\end{exe}	       
Auf solche Beispiele weist auch \citet[115]{Mikame1998} hin. In (\ref{782}) ist die Bezugs-NP des RSes anaphorisch identifizierbar, weshalb man den RS nicht als restriktiv einstufen kann.
	
\begin{exe}
	\ex\label{782} 
	\scriptsize
	In dem Schreibwarengeschäft \textit{suchte er sich aus dem Geburtstagsalbum eine Gratulationskarte aus}. 						\underline{\textbf{\textit{Die Karte}}}, \underline{\textbf{\textit{die er wählte}}}, war wundervoll.               
	\hfill\hbox {(Kästner, Pünktchen und Anton, S. 95), \citet[115]{Mikame1998}}
\end{exe}	
Unter solche nicht zwangsweise restriktiven Fälle fallen auch Sätze, bei denen eine deiktische Interpretation des Bezugsnomens denkbar ist, wie in (\ref{783}).

\begin{exe}
	\ex\label{783} 
	\scriptsize
	Genauer geht's von meiner Seite im Augenblick nicht! Sorry! – Mehr als \textbf{\textit{das}}, \underline{\textbf{\textit{was ihr} halt eben \textit{unten}}} \underline{\textbf{\textit{seht}}}, habe ich einfach nicht mehr hinbekommen.              
	\hfill\hbox {(DECOW2012-00: 546390411)}
\end{exe}
Ähnlich werden in (\ref{784}) im Vorkontext gerade Betriebe mit mehreren hundert Mitarbeitern genannt, so dass man das \textit{dort} recht einfach anaphorisch interpretieren kann.

\begin{exe}
	\ex\label{784} 
	\scriptsize
	Personenwahl setzt voraus, dass mir alle Bewerber bekannt sind und ich ihre Fähigkeiten beurteilen kann und ihre Ziele 			kenne. Das dürfte bei \textbf{\textit{Dienststellen mit mehreren Hundert Beschäftigten}} kaum durchfürbar sein. Die 			Listenwahl und insbesondere die Listen der Gewerk\-schaften geben dem Wähler die Möglichkeit, Ziele zu vergleichen. Das ist gerade \underline{\textit{\textbf{dort}}} wichtig, \underline{\textbf{\textit{wo ich} halt eben \textit{nicht jeden}}} \underline{\textbf{\textit{der 700 wählbaren Kollegen an 		einer Dienststelle persönlich kenne}}}. Sinn und Zweck der Ge\-werkschaften ist die Vertretung der Beschäftigten in allen 			Belangen rund um deren Berufsausübung.                                                 
	\newline
	\hbox{}\hfill\hbox{(DECOW2012-02: 902604791)}
\end{exe}
Solche Verwendungen findet man selbst da, wo die Bezugsnomen sehr bedeutungsarm sind (vgl. (\ref{785}) und (\ref{786})).

\begin{exe}
	\ex\label{785} 
	\scriptsize
	Ich kann z.B. Volksmusik auch nicht besonders Leiden. Aber ich Tolleriere das es \textit{viele Menschen mögen} und belasse 		es dabei. Und gehe nicht \underline{\textbf{\textit{denjenigen}}} auf die Nerven \underline{\textbf{\textit{die es} eben 		halt \textit{Mögen}}}.                                                    
	\newline
	\hbox{}\hfill\hbox{(DECOW2012-01: 1020522671)}
\end{exe}

\begin{exe}
	\ex\label{786} 
	\scriptsize
	Aber in einer Nation wie Deutschland die sehr Tolerant geworden ist bringen diese \glqq Protestzüge\grqq{} so gut wie gar 		nichts mehr, im Gegenteil sie sind doch für andere Menschen eher störend und nervend geworden. es ist nur noch 					\textbf{\textit{eine Minderheit von Homophoben}} da und wie bei den Nazis auch wird es immer 									\underline{\textit{\textbf{eine}}} $[$sic!$]$ $[$\textbf{\textit{einige}} S.M.$]$ geben \underline{\textbf{\textit{die} 		eben halt \textit{nur dumme ignorante Sturköpfe sind}}}, da hilft keine Protestbewegung der Welt mehr, erst recht keine die 	schlimmer aussieht als Karneval.                                                                                                                                                               
	\newline
	\hbox{}\hfill\hbox{(DECOW2012-01: 582843178)}
\end{exe}
Man kann die Bezugsnomen hier durchaus anaphorisch lesen: Menschen, die Volksmusik mögen, bzw. einige Homophobe. Teilt man meine Einschätzung der Belege, wird deutlich, wie entscheidend der Einbezug des Kontextes ist. \textit{Diejenigen} gilt in der Literatur als Marker für Restriktivität, wobei auch \citet[302]{Bluehdorn2007} anhand des Beispiels in (\ref{787}) schon darauf hingewiesen hat, dass diese Lesart nicht zwingend ist.
	
\begin{exe}
	\ex\label{787} 
	\scriptsize
	Du musst auf deinem PC einen neuen Benutzer einrichten. Wenn du dich dann als \textbf{\textit{derjenige Benutzer}} 				anmeldest, \underline{\textbf{\textit{dessen Account du übrigens}}} \underline{\textbf{\textit{nicht durch ein Passwort zu 		sichern brauchst}}}, dann kannst du auf das gesperrte Verzeichnis wieder zugreifen.
\end{exe}								
Unklar kann die Zuordnung auch bei indefiniten Bezugsnomen sein. (\ref{788}) präsentiert einen recht langen Kontext, ohne den der folgende Punkt allerdings nicht nachvollziehbar ist.

\begin{exe}
	\ex\label{788} 
	\scriptsize
	So kam es dann zum Jungsheft! Nicole Rüdiger: Das bekommt man über unsere Internetseite\\ www.Jungsheft.de oder über verschiedene Geschäfte – es sind zur Zeit fünfzehn Geschäfte, die unser Heft verkaufen – weil wir es halt nicht über den normalen Kiosk verkaufen dürfen. Nicole Rüdiger: Weil es das noch gar nicht gibt! Damit Mädchen endlich mal die Möglichkeit haben, Jungs nackt zu sehen und zwar \underline{\textbf{\textit{Jungs}}}, \underline{\textbf{\textit{auf die wir} eben halt \textit{stehen}}}. Natürlich gibt es Frauen, die gerne den Feuerwehrmann nackt sehen wollen, der voller Ruß neben seinem Schlauch steht – das gibt es bestimmt auch schon – aber, das ist nicht das, war wir auch sehen wollen. Ich glaube auch, es gibt einen ganz großen Markt für Mädels, die einfach was ganz anderes sehen wollen. Es ist halt heute immer noch nicht selbstverständlich, Männer nackt zu sehen. Es ist selbstverständlich, Frauen nackt zu sehen, die sehe ich ja schon, wenn ich morgens mein Müsli esse, aber für Frauen gibt es das einfach noch nicht und ich glaube, das war jetzt einfach mal Zeit!
	\hfill\hbox{(DECOW2012-03: 717848465)}
\end{exe}
Nimmt man für die Interpretation des RSes den ganzen Kontext hinzu, bietet sich eine generische Lesart von \textit{Jungs} an. Es werden keine Männer gezeigt, sondern Jungs. Deshalb ist durchaus eine Interpretation plausibel, unter der mit dem RS nicht die Jungs als Klasse reduziert werden auf diejenigen Jungs, auf die Mädchen stehen, sondern es wird betont, dass es um Jungs geht, weil Mädchen auf Jungs stehen und nicht auf Männer, stellvertretend für die der Feuerwehrmann genannt wird. D.h. man kann m.E. nicht sagen, dass die restriktive Interpretation (als Untergruppe von Jungs) notwendig ist. 

Ein ähnlich gelagerter Fall ist (\ref{789}).
	
\begin{exe}
	\ex\label{789} 
	\scriptsize
	 Dann kamen leider gleich einige Fragen und Hinweise die mir deutlich zeigten, dass meine Charlie für den TA nur \underline{\textbf{\textit{ein Tier}}} war \underline{\textbf{\textit{was} eben halt \textit{leider verstorben ist}}}. 
	\hfill\hbox{(DECOW2012-05: 864200016)}
\end{exe}	    					   
Hier liegt eine appositive Interpretation nahe. Charlie war für den Tierarzt nur ein Tier wie alle seine Patienten (anders war Charlie für die Familie mehr als ein Tier). Die dazutretende Information, dass er verstorben ist, dient nicht der Identifikation des Tieres als Untergruppe in der Obermenge der Tiere an sich. Wertende Ausdrücke wie \textit{leider} sollten in rRSen auch eher nicht zu finden sein.

\subsubsubsection{Verteilungen}
\label{sec:ergeb}
Der grundsätzliche Punkt, für den die Daten Evidenz liefern, ist, dass \textit{halt}, \textit{eben} sowie \textit{halt eben} und \textit{eben halt} meiner Meinung nach in den beiden RStypen nicht gleich verwendet werden (vgl. auch \citealt[169-171]{Mueller2016a}). (\ref{790}) zeigt die ermittelten Verteilungen.

\begin{exe}
	\ex\label{790} Verteilung \textit{halt}, \textit{eben}, \textit{halt eben} und \textit{eben halt} in Relativsätzen\\[-1em]
     \begin{tabular}[t]{|l|l|l|}
     \hline
	  & \textbf{restriktiv} & \textbf{appositiv}\\
	 \hline\hline
	 \textit{halt} & 72 & 47\\
	 \hline
	 \textit{eben} & 99 & 167\\
	 \hline\hline
	 \textit{eben halt} & 22 & 37\\	 
	 \hline
	 \textit{halt eben} & 109 & 112\\
	 \hline       
     \end{tabular}
\end{exe}
Die Verteilungen geben nur bedingt Auskunft, bevor ein Wert bekannt ist, wie restriktive und appositive RSe in den Daten überhaupt verteilt sind. 

Dass es sich bei diesem Erwartungswert für die Verteilung von restriktiven und appositiven RSen um einen Wert handelt, den es für das jeweilige Korpus zu ermitteln gilt, zeigen andere Untersuchungen, in denen ganz verschiedene Verhältnisse festgestellt worden sind (vgl. die Übersicht in (\ref{791})).

\begin{exe}
	\ex\label{791} Verteilung rRSe und aRSe in verschiedenen Datentypen \\[-1em]		
 		\begin{tabular}[t]{|c|c|c|} 
 		\hline 	
   	 	\textbf{restriktiv} & \textbf{appositiv} & \textbf{Arbeit und} \textbf{Datentyp}\\ 
  		\hline\hline
  		65\% (96) & 35\% (51) & \citet[191]{Schaffranietz1997} \\ 
  		& & gesprochene Daten (Instruktionen) \\
   		\hline
   		41\% (292) & 59\% (428) & \citet[363]{Ravetto2009} 363)\\
   		& & geschriebene Daten (Literatur) \\
   		\hline
   		83\% (716) & 17\% (147) & \citet[240]{Birkner2008}\\
   		& & gesprochene Daten (Dialoge) \\
   		\hline
   		62,5\% (45) & 37,5\% (27) & \citet[172]{Grawunder2012}  \\
   		& & gesprochene Daten (Nachrichten) \\
   		\hline
   		73\% (52) & 27\% (19) & \citet[406]{Hirschberg2014}  \\
   		& & gesprochene Daten (\textit{Lindenstraßen-Korpus}) \\
   		\hline
   		66\% (322) & 34\% (168) & \citet[406-407]{Hirschberg2014}  \\
   		& & gesprochene Daten  \\
   		& & (Audioversion \textit{Der Vorleser})\\
  		\hline      
 		\end{tabular}
\end{exe}
Man sieht hier auch gut, dass Faktoren wie Medium, Konzeption, Register oder Textsorte (welcher Aspekt genau müsste letztlich eine eigene Untersuchung zeigen) Einfluss auf die Verteilung nehmen. Leider wird in den wenigsten Arbeiten genauer aufgeschlüsselt, welche Typen von RSen in die Betrachtung aufgenommen wurden, so dass die Ergebnisse auch aus diesem Grund schlecht vergleichbar sind.

Ich habe das Verhältnis für DECOW2012 anhand einer Stichprobe von 1924 RSen ohne MPn mit einer möglichst guten Näherung ($\pm$ 2,26\% bei einer Stichprobe dieser Größe) bestimmt.\footnote{Konkret habe ich dazu mit der Anfrage $[$word=\glqq  das$\vert$der$\vert$den$\vert$die$\vert$der$\vert$dem$\vert$dessen$\vert$was$\vert$wo$\vert$wen$\vert$\\wer$\vert$wohin$\vert$welcher$\vert$welche$\vert$welches$\vert$wofür$\vert$wodurch$\vert$womit$\vert$woran\grqq{}$][]\lbrace$1,8$\rbrace[$pos=\glqq VVFIN\grqq{}] 5000 Treffer ausgeben lassen, diese per Einzeldurchsicht um nicht relevante Strukturen reduziert und die verbliebenen 1924 RSe um ihre Kontexte ergänzt (da mir nur für die Kombinationen das zur Verfügungstellen der Sätze im Kontext gewährt wurde).} (\ref{792}) zeigt, dass die Verteilung auf die beiden RStypen sehr ausgeglichen ist.

\begin{exe}
	\ex\label{792}Erwartungswert Verteilung rRSe und aRSe in DECOW-2012\\[-1em]		
 		\begin{tabular}[t]{|c|c|c|} 
 		\hline 	
   	 	& \textbf{restriktiv} & \textbf{appositiv} \\
   	 	\hline 
  		absolute Zählung & 968 & 956\\ 
   		\hline
   		Anteil & \textbf{50,31\%} & \textbf{49,69\%}\\
   		\hline
   		95\%-Konfidenzintervall & $[$48,05\% ... 52,57\%$]$ & $[$47,43\% ... 51,95\%$]$ \\
   		\hline
 		\end{tabular}
\end{exe}
Mit diesem Wert im Hintergrund lassen sich die Verteilungen erst einschätzen und es wird deutlich, dass die Partikeln in den beiden RSen nicht gleich verteilt sind.

Die Partikel \textit{halt} überwiegt in restriktiven Kontexten, während \textit{eben} in entgegengesetzter Tendenz den appositiven RS bevorzugt:

\begin{exe}
	\ex\label{793} Verteilung \textit{halt} und \textit{eben} in RSen\\[-1em]
     \begin{tabular}[t]{|l|l|l|}
     \hline
	 {} & \textbf{restriktiv} & \textbf{appositiv}\\
	 \hline
	 \textit{halt} & 72 & 47\\
	 \hline
	 & 60\% & 40\% \\
	 \hline\hline 
	 \textit{eben} & 99 & 167\\
	 \hline
	 & 37\% & 63\%\\
	 \hline       
     \end{tabular}
\end{exe}
Bei \textit{halt eben} ist das Verhältnis etwas ausgeglichener als bei \textit{eben halt}. Die Kombination \textit{eben halt} hat – wie \textit{eben} – ein größeres Bedürfnis im aRS aufzutreten als \textit{halt eben}:

\begin{exe}
	\ex\label{794} Verteilung \textit{halt eben} und \textit{eben halt} in RSen\\[-1em]
     \begin{tabular}[t]{|l|l|l|}
     \hline
	 {} & \textbf{restriktiv} & \textbf{appositiv}\\
	 \hline
	 \textit{halt eben} & 109 & 112\\
	 \hline
	 & 49\% & 51\%\\
	 \hline\hline
	 \textit{eben halt} & 22 & 37\\
	 \hline
	 & 37\% & 63\%\\
	 \hline       
     \end{tabular}
\end{exe}
Wie die statistischen Angaben zeigen, stellen sich die Asymmetrien bei \textit{halt} ($\chi_{2}$(1, n = 119) = 4,9469, p $<$ 0,05, V = 0,2), \textit{eben} ($\chi_{2}$(1, n = 266) = 18,2376, p $<$ 0,001, V = 0,26) und \textit{eben halt} ($\chi_{2}$(1, n = 59) = 4,002, p $<$ 0,05, V = 0,26) auch als statistisch signifikant heraus, wenngleich sich nur geringe bis mittlere Effekte einstellen. Das ausgeglichenere Verhältnis bei \textit{halt eben} stellt sich nicht als signifikant he\-raus ($\chi_{2}$(1, n = 221) = 0,0864, p = 0,7688). Stellt man die Ergebnisse jeweils im Vierfeldertest gegenüber, ergibt sich für die Einzelpartikeln ein signifikanter Unterschied ($\chi_{2}$(1, n = 385) = 18,0582, p $<$ 0,001, V = 0,22), im Falle der Kombinationen bleibt der Chi-Wert leicht unter dem Wert, der ein signifikantes Ergebnis darstellen würde ($\chi_{2}$(1, n = 280) = 2,7083, p = 0,09983).

Generell denke ich, dass es sich bei diesen Verteilungen um Tendenzen handelt, für die sich im Rahmen meiner Theorie eine Erklärung anbietet. Bei statis\-tischen Auswertungen dieser Daten werden kleine Werte produziert, die sich durch wenige Umsortierungen verändern würden. Da die Betrachtung an verschiedenen Stellen Unsicherheiten mit sich bringt (eine gewisse zuzugebende Unschärfe des Phänomens an sich, ein Erwartungswert, für den man trotz einer hoher Stichprobe nur vom 95\%-Konfidenzintervall ausgehen kann), würde ich nicht zu sehr auf die statistischen Ergebnisse bauen. Dies mag sich merkwürdig anhören, zumal die Ergebnisse meine Hypothese auch statistisch durchaus bestätigen, m.E. sollte man auf diesen Punkt aber hinweisen. Möglicherweise wendet man hier Verfahren an, die eine Präzision suggerieren, die der Gegenstandsbe\-reich gar nicht zu leisten vermag. Im Rahmen der Möglichkeiten der Untersuchung bleiben aber stets die aus (\ref{793}) und (\ref{794}) zu entnehmenden Tendenzen. Es wird sich im Folgenden zeigen, dass sich daran auch unter weiteren Differenzierungen nichts Grundsätzliches ändert.\\

\noindent
Die beobachteten Verteilungsunterschiede lassen sich unter Bezug auf meine Annahmen zu den Einzelpartikeln und ihren Kombinationen erklären (vgl. auch \citealt[172-173]{Mueller2016a}). Die MP \textit{halt} ist in der Modellierung mit Assertivität \is{Assertivität} assoziiert und damit allgemeiner mit hohem Mitteilungswert, Rhematizität \is{Rhematizität} oder \is{Vordergrund-/Hintergrundinformation} Vordergrundinformation. Die MP \textit{eben} ist hingegen mit einer Präsupposition assoziiert und in diesem Sinne mit \is{Mitteilungswert} niedrigerem Mitteilungswert, Thematizität \is{Thematizität} oder Hintergrundinformation. Das Anzeigen einer Information als bekannt und einer anderen als ableitbar für beide Gesprächsteilnehmer passt dann sicherlich besser zum Informationsbeitrag des aRSes als dem des rRSes. Genau im Sinne der Er\-klärung von \citet{Hentschel1986} für den Ausschluss von \textit{ja} aus rRSen (vgl. Abschnitt~\ref{sec:mpnrs}) ist es plausiblerweise als merkwürdig einzustufen, Information, die zur Identifikation notwendig ist, als bekannt auszugeben.\footnote{Aufgrund des im Vergleich zu \textit{ja} zusätzlichen Bedeutungsanteils der Ableitbarkeit gehe ich allerdings davon aus, dass \textit{eben} immer noch frequenter im rRS ist (immerhin 37\%) als \textit{ja}.}

Für die Kombination stellt sich dann die Frage, warum auch \textit{eben halt} den Kontext der Vordergrundinformation meidet. Meine Überlegung ist, dass MPn je nach Struktur, in der sie auftreten, unterschiedliches Gewicht haben können. \textit{Halt} ist mit informativen Beiträgen assoziiert, \textit{eben} mit ableitbarer Information und \textit{halt} somit mit Äußerungsteilen mit höherem Mitteilungswert als \textit{eben}. Die unterschiedliche Gewichtung der MPn realisiert sich derart, dass wenn \textit{halt} und \textit{eben} in Strukturen mit hohem Mitteilungswert zusammen auftreten, \textit{halt} do\-minant ist. In Äußerungsteilen mit geringerem Mitteilungswert ist \textit{halt} weniger dominant. \textit{Eben halt} wird in den Kontexten mit hohem Mitteilungswert dann tendenziell deshalb nicht gebraucht, weil sich das Missverhältnis zwischen dem Implikationsauslöser und der Implikation gerade verstärkt, wenn die Information, die impliziert wird, noch hervorgehoben wird. Die umgekehrte Abfolge \textit{halt eben} kann auch in Kontexten mit geringerem Mitteilungswert stehen, weil es für die Abfolge Implikation $>$ Implikationsauslöser unerheblich ist, ob die Implikation dominant ist oder nicht.

Am Beispiel der RSe bestätigt sich folglich meine Hypothese, dass \textit{halt eben} und \textit{eben halt} nicht identisch verwendet werden. Gleiches gilt für das Einzelauftre\-ten der Partikeln. Hier bestätigt sich auch die Anmerkung von \citet[80, Fn 104]{Thurmair1989}, dass sie zwar rRSe mit \textit{halt} und \textit{eben} in ihrem Korpus findet, ihr \textit{halt} in dieser Umgebung aber akzeptabler erscheint. Zumindest scheint es in diesem Kontext häufiger verwendet zu werden. Ausgehend von der Annahme, dass die isolierten Partikeln und die Kombinationen verschiedene Diskursbeiträge leisten (vgl. Abschnitt~\ref{sec:kontexte} und \ref{sec:interpretationkombi}), bietet meine Analyse somit eine Erklärung für die beobachteten Verteilungsunterschiede an. 

Wie in Abschnitt~\ref{sec:interrs} erläutert, gehe ich von der Zuordnung Restriktivität und Vordergrundinformation sowie Appositivität und Hintergrundinformation aus. Dass diese Assoziationen Idealisierungen sind, habe ich dort bereits angemerkt. In den folgenden beiden Abschnitten sollen aus dieser Perspektive zwei Typen von Relativsatzstrukturen genauer betrachtet werden, bei denen diese Zuordnung unpassend sein kann, weil die Sätze mit dem Gesamtdiskurs interagieren. Die RSe sind dann zwar im lokalen Sinne (wie in meinen Ausführungen intendiert) restriktiv oder appositiv, gleichzeitig entspricht der lokalen Restriktivität ein niedriger oder hoher Mitteilungswert im Gesamtdiskurs bzw. leistet der appositive Satz einen relevanten Beitrag im Diskursverlauf. Da man davon ausgeht, dass RSe in der Regel generell keinen globalen Mitteilungswert im Gesamtdiskurs haben (vgl. \citealt[38-42]{Antomo2015}), kommt diesen RSen hinsichtlich dieses Kriteriums ein besonderer Status zu. Die beiden Strukturen sind \is{Pseudo-Cleft-Satz} \textit{Pseudo-Cleft-Sätze} (Abschnitt~\ref{sec:pcs}) und \textit{weiterführende nomenbezogene appositive RSe} \is{nomenbezogener weiterführender Relativsatz} (Abschnitt~\ref{sec:nwrs}). Zweck der Betrachtung ist, festzustellen, ob die Abweichung von der obigen Assoziation einen Einfluss auf das Auftreten der MPn nimmt. Es sei an dieser Stelle vorweg genommen, dass sich an den Ergebnissen aus Abschnitt 6.2.4.2 nichts Grundsätzliches ändern wird.

\subsubsection{Pseudo-Cleft-Sätze}
\label{sec:pcs}
In Abschnitt~\ref{sec:interrs} wurden alle freien RSe \is{freier Relativsatz} als restriktive RSe behandelt. Unter diesen freien RSen befinden sich auch \textit{kanonische} \is{kanonischer Pseudo-Cleft-Satz} bzw. \textit{invertierte Pseudo-Cleft-Sätze} \is{invertierter Pseudo-Cleft-Satz} (PCs). (\ref{795}) bis (\ref{798}) zeigen Beispiele für die vier Typen.

\begin{exe}
	\ex\label{795} 
	Was du machst, ist bloß die Soße. (= das, was du machst, ist...) (kanonisch)
	\newline
	\hbox{}\hfill\hbox{\citet[360]{Lehmann1984}}	
\end{exe}
\vspace{-0.65cm}
\begin{exe}
	\ex\label{796} 
	Champagner ist was ich mag. (invertiert)
	\hfill\hbox{nach \citet[467]{Lambrecht2001}}	
\end{exe}
\vspace{-0.65cm}
\begin{exe}
	\ex\label{797} 
	Was der Frau des Catchers so sehr gefällt: seine friedfertige, häusliche Art. 
	\hfill\hbox{(kanonisch + elliptisch)}
	\newline
	\hbox{}\hfill\hbox{(ZM 7/73: 17), \citet[74-75]{Dyhr1978}}		
\end{exe}
\vspace{-0.65cm}
\begin{exe}
	\ex\label{798} 
	Maria hat einen Großeinkauf gemacht mit Nudeln, Schokolade, Eis und Fanta – was Kinder so mögen.
	\hfill\hbox{(invertiert + elliptisch)}\footnote{In den Daten in \citet[84]{Guenthner2006} weisen 39\% der kanonischen PCs (14 aus 36) keine Kopula auf.}	
\end{exe}			         	                 
Aus meinen Daten gehören Belege wie in (\ref{799}) bis (\ref{802}) in diese Klasse. 

\begin{exe}
	\ex\label{799} 
	\scriptsize
	Das Sie sich ein Bild von machen können. Ich regel zumindest den einen Block. \textbf{\textit{Was mich} halt eben 				\textit{wundert}}, die anderen 3 Blöcke müssten genau so verkorkst sein, d.h. der Block bei jemand sein, die Ips aber nicht 	auf dessen Server geroutet sondern immer noch die Confixx Seiten kommen.                                                                                             
	\newline
	\hbox{}\hfill\hbox{(DECOW2012-02: 1035004163)}
\end{exe}

\begin{exe}
	\ex\label{800} 
	\scriptsize
	\textbf{\textit{Was mir} halt eben \textit{verstärkt zur Zeit auffällt}} sind die blauen Finger und Lippen wenn ich mich 		anstrenge und kalter Schweiß ...	
	\hfill\hbox{(DECOW2012-03: 221281302)}
\end{exe}					    	

\begin{exe}
	\ex\label{801} 
	\scriptsize
	Des Weiteren richte ich das Aqua mit allerlei Röhren ein, außerdem Holz, Pappe, Ton, Eierkartons, Reiskartons, 					\textbf{\textit{was} halt eben \textit{in der Küche so an Pappe anfällt}}. Vieles eignet sich prima als 						Rennmauszeitvertreib.    	
	\hfill\hbox{(DECOW2012-02: 608534381)}
\end{exe}

\begin{exe}
	\ex\label{802} 
	\scriptsize
	Langsam steckt er Teil für Teil ein. Zuerst einen mir völlig normal erscheinenden Damenslip, dann einen roten Seidentanga, 		eine handvoll BH's und zu guter letzt Damenstrümpfe, \textbf{\textit{was} halt eben \textit{noch so reingeht, in seinen 		Mantel}}.                      	
	\hfill\hbox{(DECOW2012-00: 458299540)}
\end{exe}
Eindeutig um kanonische PCs handelt es sich in (\ref{799}) und (\ref{800}). (\ref{801}) und (\ref{802}) fasse ich als invertierte PCs auf.

Wie in Abschnitt~\ref{sec:interrs} unter Bezug auf die Interpretation der Sätze und die Paraphrasen, die ihnen in der Literatur zugeschrieben werden, erläutert, sollten sie in meiner Untersuchung der Verteilung der Partikeln zu den rRSen gezählt werden, wie ich es in der obigen Betrachtung auch getan habe. Ein ggf. intervenierender Faktor für die Assoziation von Restriktivität und (lokal) hohem Mitteilungswert ist bei dieser Struktur, dass ihr unterschiedliche diskursstruktu\-rierende Funktionen im Sinne von Thema-/Fokus-/Topik-Auszeichnung zugeschrie\-ben wird. Aufgrund meines Zugangs zu MPn im Allgemeinen und der konkreten Modellierung von \textit{halt} und \textit{eben}, bei der Informationsstatus entlang der Dimension \glq bekannt-neu\grq {} eine Rolle spielen, gehe ich prinzipiell davon aus, dass das Auftreten von (diesen) MPn mit informationsstrukturellen Aspekten interagiert. Ob sie dies im konkreten Fall tun oder nicht, soll an dieser Stelle gar nicht entschie\-den werden. Dafür wären klare Annahmen zur Funktion dieser vier Typen vonnöten sowie die Betrachtung einer genügend großen Menge derartiger Sätze im Kontext.

Es gibt zahlreiche Untersuchungen zum Englischen, weniger zum Deutschen. Im Folgenden seien einige Aspekte angeführt, um die Argumentation zu stützen, dass bei diesen Strukturen vor allem verschiedene diskursstrukturierende Funktionen beteiligt sind, die gerade mit Status wie Vorerwähntheit/Präsupposition etc. zusammenhängen, auf die die hier betrachteten MPn Bezug nehmen.

Eine gängige Annahme in der Literatur ist, dass die Struktur der PCs der Fokusmarkierung \is{Fokusmarkierung} dient, wobei dabei der w-Teil die gegebene/präsupponierte Information beisteuert und der Kopulasatz den Fokus beinhaltet (vgl. z.B. \citealt{Prince1978}, \citealt{Collins1991}, \citealt{Lambrecht2001}). \citet[465-466]{Huddleston1984} schreibt, die Information im w-Teil sei zwar nicht stets gegeben, weise aber niedrigen kommunikativen Wert auf. \citet[65]{Guenthner2006} führt allerdings Belege an, die diese Verteilung nicht aufweisen. Es gibt z.B. Fälle mit Doppelfokus (S. 65) oder einer persönlichen Bewertung im w-Teil (S. 68), und die Autorin verweist auch auf weitere Beispiele, bei denen der w-Teil prosodisch hervorgehoben ist und aus interpretatorischer Sicht nicht gegeben sein kann (S. 69-70). Dass der Inhalt des w-Teils wichtig sein kann, stellen auch schon \citet{Weinert1996} fest.

Andere informationsstrukturelle Faktoren, die mit PCs in Verbindung gebracht werden, sind z.B. die thematische Umfokussierung innerhalb eines globalen Themas (\citealt[75]{Guenthner2006}) oder die Auszeichnung/Einführung von Topiks (\citealt[186, 196]{Weinert1996}). Die angeführten Funktionen wurden – soweit aus den Arbeiten ersichtlich – in Bezug auf kanonische PCs gemacht. Zum Deutschen ist mir keine Analyse von invertierten PCs bekannt. Die Ausführungen zum Eng\-lischen weisen darauf hin, dass es durchaus Unterschiede zu den kanoni\-schen Strukturen gibt. \citet[190]{Weinert1996} gehen davon aus, dass beide Teile der Struktur wichtige/neue Information aufnehmen. Die Sätze weisen ihnen zufolge eine enge Anbindung an den vorangehenden Diskurs auf (S. 191), was sich beispielsweise darin äußere, dass im non-w-Teil deiktische Elemente enthalten seien (S. 191). Die Sätze dienten der Reassertierung des Topiks (S. 192) und würden zusammenfassend interpretiert werden (S. 192). Generell schreiben sie kanoni\-schen PCs eine Vorwärtsorientierung, invertierten PCs eine Rückorientierung zu (vgl. \citealt[199]{Weinert1996}). Auch \citet[120, 145, Kapitel 6.1, 6.2]{Collins1991} äußert sich ausführlich zur Rolle der beiden Cleft-Satztpen im Diskurs im Eng\-lischen. Unterschiede sieht er bei der Verteilung von neuer und alter Information auf die beiden Teilstrukturen und bei der Diskursfunktion. Unter Inversion beinhalte der Hauptsatz typischerweise gegebene Information, im kanonischen Fall hingegen neue Information. Im RS befänden sich bei den kanonische PCs eher alte Inhalte (die neu nur im Sinne von Kontrast seien) und bei den invertierten Sätzen neue Informationen, die aber dennoch hintergrundierend/unkontrovers seien. Invertierte PCs treten \citet{Collins1991} zufolge am Ende von Abschnitten auf und haben zusammenfassende, abschließende Funktion. Kanonische PCs würden das Thema anzeigen und Hintergrundwissen ausdrücken, das der Hörer haben sollte, bevor im Hauptsatz die eigentliche Mitteilung gemacht wird (vgl. S. 213 \glqq interpersonal \glq tracking\grq {} device within the flow of discourse\grqq{}). Dieser kurze Blick auf den informationsstrukturellen und diskursfunktionalen Status von Pseudo-Cleft-Sätzen zeigt, dass zwar durchaus Uneinigkeit hinsichtlich des genauen Status besteht, dass aber von einer Interaktion dieser Sätze mit dem Gesamtdiskurs auszugehen ist. Die Dimensionen, um die es hierbei geht, sind vor allem Neuheit vs. Bekanntheit/mitgeteilte vs. vorausgesetzte Information. Diese Interaktion schließt natürlich nicht aus, dass den RSen lokal (aufgrund ihrer Restriktivität) ein hoher Mitteilungswert zukommt. Da die obigen Dimensionen aber genau die sind, mit denen die MPn \textit{halt} und \textit{eben} assoziiert werden, ist nicht auszuschließen, dass es zu überlappenden Kodierungen kommt, weil der RS z.B. – trotz seiner Restriktivität – als bekannt ausgegeben werden soll. Unter den Einzelpartikeln befinden sich 18 \textit{halt}-PCs und 6 \textit{eben}-PCs. Im Falle der Kombinationen treten 23 \textit{halt eben}-PCs und 8 \textit{eben halt}-PCs auf. Nimmt man diese Werte aus den Verteilungen, resultieren die Verhältnisse in (\ref{803}) und (\ref{804}).

\begin{exe}
	\ex\label{803} Verteilung \textit{halt} und \textit{eben} im RS\\[-1em]
     \begin{tabular}[t]{|c|c|c|}
     \hline
	 {} & restriktiv & appositiv\\
	 \hline
	 \textit{halt} & 54 & 47\\
	 \hline	 
	 \textit{eben} & 93 & 167\\
	 \hline    
     \end{tabular}
\end{exe}                    				

\begin{exe}
	\ex\label{804} Verteilung \textit{halt eben} und \textit{eben halt} im RS\\[-1em]
     \begin{tabular}[t]{|c|c|c|}
     \hline
	 {} & restriktiv & appositiv\\
	 \hline
	 \textit{halt eben} & 86 & 112\\
	 \hline	 
	 \textit{eben halt} & 14 & 37\\
	 \hline    
     \end{tabular}
\end{exe}                    				
Um zu entscheiden, ob die in Abschnitt~\ref{sec:ergeb} vorgestellten Ergebnisse durch die PCs beeinflusst werden bzw. um auszuschließen, dass die Ergebnisse nur durch sie zustande kommen, ist es zunächst nötig, den Erwartungswert für die generelle Verteilung von restriktiven und appositiven RSen in den Daten anzupassen. Unter den 1924 MP-losen RSen befinden sich 34 kanonische PCs. Der Erwartungswert ändert sich deshalb leicht (vgl. (\ref{805})).

\begin{exe}
	\ex\label{805}Erwartungswert Verteilung rRSe und aRSe in DECOW-2012 ohne PCs\\[-1em]		
 		\begin{tabular}[t]{|c|c|c|} 
 		\hline 	
   	 	& \textbf{restriktiv} & \textbf{appositiv} \\
   	 	\hline 
  		absolute Zählung & 934 & 956\\ 
   		\hline
   		Anteil & \textbf{49,42\%} & \textbf{50,58\%}\\
   		\hline
   		95\%-Konfidenzintervall & $[$47,14\% ... 51,70\%$]$ & $[$48,30\% ... 52,86\%$]$ \\
   		\hline
 		\end{tabular}
\end{exe}
Vor dem Hintergrund des neuen Erwartungswertes ergeben sich für \textit{eben} ($\chi_{2}$ = 19,3823, p $<$ 0,001, V = 0,27 (36\%–64\%)), \textit{eben halt} ($\chi_{2}$ = 4,002, p $<$ 0,05, V = 0,26 (37\%–63\%)) und \textit{halt eben} ($\chi_{2}$ = 0,0864, p = 0,7688 (49\%–51\%)) keine Veränderungen in ihrer Verteilung auf die beiden RS-Typen. Der Vergleich der beiden Kombinationen im Vierfeldertest liefert weiterhin einen Chi-Quadrat Wert, der knapp unter einer Signifikanz liegt ($\chi_{2}$ = 2,7083, p = 0,09983). Der Vergleich der Verteilungen von \textit{halt} und \textit{eben} stellt sich als signifikant heraus ($\chi_{2}$ = 9,4368, p $<$ 0,05, V = 0,16), wobei die Effektstärken in allen Fällen mittel bis gering sind. Eine Veränderung ergibt sich allerdings im Falle von \textit{halt}. Die Gerichtetheit hinsichtlich des präferierten Auftretens in rRSen verschiebt sich zu einer ausgeglicheneren Verteilung (54\%–46\%), die sich nicht als signifikant herausstellt ($\chi_{2}$ = 0,6612, p = 0,4161). 

Die Präsentation dieser Ergebnisse soll an dieser Stelle weniger dem Hinweis auf statistisch bedeutsame Unterschiede dienen, als vielmehr aufzeigen, dass sich an den Ergebnissen durch die differenziertere Betrachtung und den Ausschluss der PCs nichts ändert.

Die Haupterkenntnis aus der Studie ist die Beobachtung, dass \textit{eben halt} (genauso wie \textit{eben}) aus dem Kontext mit hohem Mitteilungswert ausgeschlossen ist. Für \textit{halt eben} stellte sich schon in Abschnitt~\ref{sec:ergeb} eine ausgeglichenere Verteilung ein. Ob \textit{halt} nun den rRS bevorzugt oder ebenfalls eher gleichverteilt ist, lässt sich nicht entscheiden. Erklären ließen sich beide Verteilungen. Arbeiten wie \citet{Thurmair1989} und \citet{Ickler1994}, die davon ausgehen, dass zwischen \textit{eben} und \textit{halt} ein Implikationsverhältnis besteht, würden vorhersagen, dass \textit{halt} eine weitere Verwendung hat als \textit{eben}, weil es in allen \textit{eben}-Kontexten stehen kann, während \textit{eben} nicht in allen \textit{halt}-Kontexten stehen kann. \textit{Halt} kann aus dieser Sicht genauso gut in alten wie neuen Kontexten stehen. Selbst wenn die Auftretenskontexte von \textit{halt} prinzipiell weiter sind als die von \textit{eben}, bedeutet dies aber natürlich noch nicht, dass sie auch tatsächlich im generellen Austausch verwendet werden. \textit{Halt} könnte den aRS auch meiden, um (über eine Implikatur) nicht anzuzeigen, dass der Bedeutungsbeitrag von \textit{eben} unpassend ist, der aber in aRSen ja gerade der vorliegende ist. Da es von den beiden Partikeln zudem die einzige ist, die neue Information markieren kann, ist es auch nicht unplausibel, dass sie derartige Kontexte präferiert – obwohl sie auch in alten Kontexten auftreten kann.

Im letzten Abschnitt wurde aufgezeigt, dass der RS in Pseudo-Spaltsätzen, der als restriktiv eingestuft werden kann, sowohl Vorder- als auch Hintergrundinformation kodieren kann und sich in dem Sinne von anderen RSen unterscheidet, die global i.d.R. keinen informativen Beitrag leisten. Prinzipiell könnte es hier zu einer Interaktion zwischen dem lokal (Vordergrund) und global (Vorder-/Hintergrund) bedingten MP-Auftreten kommen. Ein ähnliches Missverhältnis ergibt sich – in entgegengesetzter Richtung – für sogenannte \is{nomenbezogener weiterführender Relativsatz} \textit{nomenbezogene weiterführende RSe} (\textit{d}-wRSe).

\subsubsection{Nomenbezogene weiterführende Relativsätze}
\label{sec:nwrs}
Diese RSe (zwei Beispiele finden sich in (\ref{806}) und (\ref{807})) sind zweifellos appositiv zu interpretieren

\begin{exe}
	\ex\label{806} 
	Er suchte eine Telefonzelle, die er schließlich auch fand. 
	\newline
	\hbox{}\hfill\hbox{zitiert nach\citet[4]{Brandt1990}}	
	\newline
	\hbox{}\hfill\hbox{ursprünglich \citet[672]{Duden1984}}
\end{exe}

\begin{exe}
	\ex\label{807} 
	Ich traf einen Bauern, bei dem ich mich nach dem Wege erkundigte.
	\newline
	\hbox{}\hfill\hbox{zitiert nach\citet[5]{Brandt1990}}	
	\newline
	\hbox{}\hfill\hbox{ursprünglich \citet[28]{Jung1971}}
\end{exe}				
Es gibt Stimmen in der Literatur (z.B. \citealt[272-273]{Lehmann1984}, \citealt[67-68]{Brandt1990}), die annehmen, dass derartige RSe dem Diskursfortschritt dienen. Sie steuern folglich relevante und informative Inhalte bei, die plausiblerweise neu sind. Es sollte somit im Interesse des Sprechers liegen, dem Hörer diese Inhalte zu vermitteln und ihn von ihnen zu überzeugen.

Die Tatsache, dass diese RSe somit auch am Gesamtdiskurs teilnehmen, wäre ein Faktor, der prinzipiell mit dem MP-Vorkommen von \textit{halt} und \textit{eben} interagieren könnte. Insbesondere wäre für die tendenzielle Präferenz von \textit{eben halt} für den aRS nicht mehr plausibel zu argumentieren, wenn sich herausstellte, dass die \textit{d}-wRSe einen großen Anteil unter den aRSen ausmachen. Diese Kombination würde dann in RSen auftreten, die gerade nicht cg-bezogen, nebensächlich oder hintergrundierend sind. Um diesen Typ von RS in den Daten ausfindig zu machen, sind zunächst Kriterien vonnöten, anhand derer er vom üblichen aRS unterschieden werden kann. Derartige Charakteristika werden in \citet[158-163]{Holler2005} entworfen. Sie formuliert drei Bedingungen, die Bezug nehmen auf temporale Relationen und zulässige Situationsbeschreibungen.

Die \textit{Topikzeit} \is{Topikzeit} des Bezugssatzes muss vor der Topikzeit des RSes liegen. Die Topikzeit ist ein Konzept aus der Tempustheorie von \citet{Klein1994} und bezeichnet die Zeit, für die eine Aussage gemacht wird. Als Heuristik dient die Möglichkeit, die Frage zu stellen \textit{Was passierte als nächstes?}. Dieses Kriterium grenzt z.B. (\ref{808}) und (\ref{809}) ab.

\begin{exe}
	\ex\label{808} 
	Otto gab Emil ein Buch, das er dann in die Bibliothek brachte. (\textit{d}-wRS)
\end{exe}
\vspace{-0.65cm}
\begin{exe}
	\ex\label{809} 
	Otto gab Emil ein Buch, das übrigens einen roten Einband hatte. (aRS)	
	\newline
	\hbox{}\hfill\hbox{ursprünglich\citet[158/159]{Holler2005}}
\end{exe}
Nur der \textit{d}-wRS eignet sich in (\ref{810}) als Fortführung.

\begin{exe}
	\ex\label{810} 
	A: Otto gab Emil ein Buch.\\
	B: Was passierte als nächstes?\\
	A: Er brachte es in die Bibliothek.\\
	A': \#Es hatte einen roten Einband.
\end{exe}
Aufschluss bieten hier auch Adverbien wie \textit{dann}, \textit{danach}, \textit{nun} oder \textit{schließlich}, die diese temporale Ordnung mit abbilden.

Ferner gilt für \textit{d}-wRSe eine Beschränkung über zulässige Prädikate im Bezugs- und RS, die die vorliegenden Situationsbeschreibungen betreffen. In den beiden Sätzen müssen \is{1-state-Inhalt} 1-state- (z.B. \textit{schlafen}, \textit{suchen}, \textit{auf den Händen stehen} $[$\citealt[81-95]{Klein1994}$]$) oder \is{2-state-Inhalt} 2-state-Inhalte (z.B. \textit{töten}, \textit{das Fenster öffnen}/\textit{schließen} $[$\citealt[81-95]{Klein1994}$]$) ausgedrückt werden, d.h. eine Zustandsänderung muss beteiligt sein, die bei \is{0-state-Inhalt} 0-state-Inhalten (z.B. \textit{in Afrika sein}, \textit{rot sein}, \textit{der Sohn sein} $[$\citealt[81-95]{Klein1994}$]$) nicht vorliegt. Der zweite Zustand kann dabei lexikalisch kodiert sein, wie z.B. bei \textit{jemandem ein Buch geben}, wo sich ein Ausgangs- und Zielzustand ausmachen lässt (zunächst hat der Geber das Buch, anschließend der Empfänger). Er kann aber auch offen sein, wie z.B. bei \textit{am Fenster stehen}, wo es mit Sicherheit auch Zeitpunkte gibt, zu denen man nicht mehr am Fenster steht. Es muss also ein Kontrast zur Topikzeit gegeben sein, d.h. es existiert eine Topikzeit t$_{\textrm{top}}$, die im Kontrast steht zu t$_{\textrm{top'}}$, da an ihr im Vergleich zu t$_{\textrm{top}}$ das Gegenteil gilt. Dies ist bei 0-state-Inhalten (z.B. auch \textit{in russischer Sprache geschrieben sein}) nicht gegeben. Von einer Zustandsänderung ist nicht auszugehen: Liegt das Buch einmal in russischer Sprache vor, tut es das für immer. Es ist nicht möglich, eine alternative Topikzeit zu finden, zu der dies nicht gilt. Aus diesen Verhältnissen ergeben sich die Urteile/Interpretationsmöglichkeiten für (\ref{811}) bis (\ref{814}).

\begin{exe}
	\ex\label{811} 
	Der Nil liegt in Afrika, wo es heute noch wilde Tiere gibt. (aRS) (0 \& 0)
\end{exe}
\vspace{-0.65cm}
\begin{exe}
	\ex\label{812} 
	*Der Nil liegt in Afrika, wo dann die Sonne unterging/untergeht. (\textit{d}-wRS) (0 \& 2)
\end{exe}

\begin{exe}
	\ex\label{813} 
	Seine ersten Bücher, die noch in russischer Sprache geschrieben waren, liebte Solschenizyn am meisten. (aRS) (0 \& 0)
\end{exe} 	
\vspace{-0.65cm}
\begin{exe}
	\ex\label{814} 
	*Otto gab Emil ein Buch, das dann in russischer Sprache geschrieben war. 
	\newline
	\hbox{}\hfill\hbox{(\textit{d}-wRS) (2 \& 0)}	
	\newline
	\hbox{}\hfill\hbox{\citet[160]{Holler2005}}
\end{exe}
Damit ein \textit{d}-wRS vorliegt, müssen beide Sätze 1- bzw. 2-state-Inhalte ausdrücken (vgl. (\ref{815}) und (\ref{816})).

\begin{exe}
	\ex\label{815} 
	Er suchte eine Telefonzelle, die er schließlich auch fand. (1 \& 2)
\end{exe} 	
\vspace{-0.65cm}
\begin{exe}
	\ex\label{815} 
	Ich traf einen Bauern, bei dem ich mich nach dem Wege erkundigte. 
	\newline
	\hbox{}\hfill\hbox{(1 \& 1))}	
\end{exe} 
Die dritte Anforderung, die Holler an \textit{d}-wRSe stellt, ist, dass die im Bezugssatz beschriebene Situation entweder abgeschlossen ist (bevor der RS geäußert wird) oder durch den RS abgeschlossen wird. In (\ref{816}) ist dieses Kriterium erfüllt, da vor der Äußerung des RSes der Zielzustand des Bezugssatzes erreicht ist (= $<$Buch bei Emil$>$).

\begin{exe}
	\ex\label{816} 
	Otto gab Emil ein Buch, das er dann in die Bibliothek brachte.
\end{exe} 
In (\ref{817}) steuert der RS den Zielzustand des Bezugssatzes bei und schließt die beschriebene Situation somit ab.

\begin{exe}
	\ex\label{817} 
	Er suchte eine Telefonzelle, die er schließlich auch fand.
\end{exe} 
Der Zielzustand des Suchens einer Telefonzelle ist das zur Verfügungstehen einer Telefonzelle. Der Bezugssatz ist hier nicht abgeschlossen, da der Zielzustand ge\-rade noch nicht erreicht ist. Der RS drückt aber das Erreichen dieses Zustandes aus.

Anhand dieser Kriterien habe ich alle aRSe in meinen Daten betrachtet. Es stellt sich heraus, dass Sätze, für die sich diese Interpretation anbietet, sehr selten in den Daten auftreten. Unter den 968 aRSen, die in die Bestimmung des Erwartungswertes eingegangen sind, gibt es 64 Sätze, für die man annehmen kann, dass die Kriterien für \textit{d}-wRSe nach \citet{Holler2005} zutreffen. (\ref{818}) und (\ref{819}) zeigen zwei Beispiele.

\begin{exe}
	\ex\label{818} 
	\scriptsize
	In der 20 min. profitierte Kisser vom Fehlpass eines Südstern Spielers und gab eine kluge Vorlage für Anton Hoffmann, 				\underline{\textbf{\textit{der die Kugel zur 1:0 Führung im Tor versenkte}}}.	
	\newline
	\hbox{}\hfill\hbox{(DECOW2012-06X: 1005903522)}
\end{exe}

\begin{exe}
	\ex\label{819} 
	\scriptsize
	Durch einfaches Öffnen des Absperrhahnes entsteht eine entsprechend große Sogwirkung, \underline{\textbf{\textit{die 				nor}}}\- \underline{\textbf{\textit{malerweise den sich am Boden abgesetzten Mulm und Dreck entfernt}}} und zudem das 				Filtermaterial der 1. Kammer schonend reinigt.                                       	
	\hfill\hbox{(DECOW2012-06X: 1165147548)}
\end{exe} 		            							            
In (\ref{818}) geht die Topikzeit des Bezugssatzes der Topikzeit des RSes voran (Vorlage geben $>$ Tor schießen). Die Inhalte beider Sätze sind 2-state-Inhalte (\textit{eine Vorlage geben} $[$Ball bei Vorlagengeber, Ball bei Mitspieler$]$), \textit{Ball im Tor versenken} $[$Ball beim Schützen, Ball im Tor$]$). Die Situation im Bezugssatz ist abgeschlossen. Ohne die erfolgte Vorlage wäre der Ball schließlich nicht beim Mitspieler, der ihn im Tor versenkt. In (\ref{819}) geht die Entstehung der Sogwirkung der Entfernung des Drecks ebenfalls voran. Beide Inhalte sind 2-state-Inhalte. Das Entstehen der Sogwirkung sollte zudem abgeschlossen sein, wenn dies das Mittel ist, um den Dreck zu entfernen. Man sieht an Beispielen wie (\ref{819}) aber auch, dass über das Vorliegen eines \textit{d}-wRSes nicht immer klar entschieden werden kann, da hier nicht auszuschließen ist, dass der RS nur eine Beschreibung der Sogwirkung abgibt und nicht den Fortgang der Ereignisse beschreibt. 

(\ref{820}) zeigt die Werte für den angepassten Erwartungswert ohne \textit{d}-wRSe.

\begin{exe}
	\ex\label{820}Erwartungswert Verteilung rRSe und aRSe in DECOW-2012 ohne \textit{d}-wRSe\\[-1em]		
 		\begin{tabular}[t]{|c|c|c|} 
 		\hline 	
   	 	& \textbf{restriktiv} & \textbf{appositiv} \\
   	 	\hline 
  		absolute Zählung & 968 & 892\\ 
   		\hline
   		Anteil & \textbf{52,04\%} & \textbf{47,96\%}\\
   		\hline
   		95\%-Konfidenzintervall & $[$49,74\% ... 54,33\%$]$ & $[$45,67\% ... 50,26\%$]$ \\
   		\hline
 		\end{tabular}
\end{exe}
Wie im Falle der Pseudo-Cleft-Sätze nimmt der Ausschluss der \textit{d}-wRSe keinen entscheidenden Einfluss auf das Gesamtergebnis der Verteilungen der Einzelpartikeln bzw. der Kombinationen. Unter den appositiven \textit{halt}-RSen befinden sich sechs, unter appositiven \textit{eben}-RSen zehn \textit{d}-wRSe. (\ref{821}) und (\ref{822}) zeigen zwei Beispiele für Sätze, die ich hier eingeordnet habe.

\begin{exe}
	\ex\label{821} 
	\scriptsize
	Zwangsmaßnahmen eines Rechenzentrums (hier z.B. U Siegen) sind zwangsläufig pauschal und tref\-fen nicht unbedingt die Verursacher, \underline{\textbf{\textit{die sich} halt \textit{andere Tricks ausdenken}}}.                                        
	\newline
	\hbox{}\hfill\hbox{(DECOW2012-C06X7M: 70309394)}
\end{exe} 	

 \begin{exe}
	\ex\label{822} 
	\scriptsize
	Dabei gehen auch Zuordnungen von Gebern/Schaltern flöten, was womöglich im derzeitigen Mo\-dellspeicher eine spezielle Flugphase aufruft, \underline{\textbf{\textit{in welcher die Servos} eben \textit{dahin laufen}}} ... 	 
	\newline
	\hbox{}\hfill\hbox{(DECOW2012-C06X7M: 9065122)}
\end{exe}                                                           				                         
Statistisch stellen sich keine relevanten Veränderungen zu den Angaben aus Abschnitt~\ref{sec:ergeb} ein.

\begin{exe}
	\ex\label{823} Verteilung \textit{halt} und \textit{eben} im RS (ohne \textit{d}-wRS)\footnote{\textit{halt}: $				\chi_{2}$ = 6,1732, p $<$ 0,05, V = 0,23; \textit{eben}: $\chi_{2}$ = 18,3301, p $<$ 0,001, V = 0,27}\\[-1em]
     \begin{tabular}[t]{|c|c|c|}
     \hline
	 {} & restriktiv & appositiv\\
	 \hline
	 \textit{halt} & 72 (62\%) & 41 (38\%)\\
	 \hline	 
	 \textit{eben} & 99 (37\%) & 157 (63\%)\\
	 \hline    
     \end{tabular}
\end{exe}    
\textit{Halt} präferiert nach wie vor den rRS, während \textit{eben} den aRS bevorzugt. Auch der Vergleich der beiden Verteilungen im Vierfelder-Chi-Quadrattest ergibt wei\-terhin einen signifikanten Unterschied zwischen den beiden Verteilungen ($\chi_{2}$ = 19,7753, p $<$ 0,001, V = 0,23).

Unter den aRSen mit Kombinationen aus \textit{halt} und \textit{eben} befinden sich sechs (\textit{halt eben}) bzw. zwei (\textit{eben halt}) \textit{d}-wRSe (vgl. (\ref{824}) und (\ref{825})).

\begin{exe}
	\ex\label{824} 
	\scriptsize
	Wahrscheinlich hauen sie 30 Stück jedes Artikels raus und locken dabei zigtausende Leute auf die Seite, 							\underline{\textbf{\textit{die dann} halt eben \textit{mal was anderes bestellen}}}.  
	\hfill\hbox{(DECOW2012-00: 607525412)}
\end{exe} 

\begin{exe}
	\ex\label{825} 
	\scriptsize
	Ich kann diese Maus nicht mehr leiden sehen. Zur Not bekommt sie Schmerzmittel, \underline{\textbf{\textit{die dann}}} 			\underline{\textbf{eben halt \textit{auf die Nieren gehen}}}, aber so ist das auch kein Zustand.  		
	\hfill\hbox{(DECOW2012-04: 677418619)}
\end{exe}  	                                                                                     
Der Ausschluss der \textit{d}-wRSe führt zu der neuen Verteilung in (\ref{823}).

\begin{exe}
	\ex\label{823} Verteilung \textit{halt eben} und \textit{eben halt} im RS (ohne \textit{d}-wRS)\footnote{\textit{eben halt}: $\chi_{2}$ = 4,1275, p $<$ 0,05, V = 0,27; \textit{halt eben}: $\chi_{2}$ = 0,1552, p = 0,6936}\\[-1em]
     \begin{tabular}[t]{|c|c|c|}
     \hline
	 {} & restriktiv & appositiv\\
	 \hline
	 \textit{eben halt} & 22 (37\%) & 35 (63\%)\\
	 \hline	 
	 \textit{halt eben} & 109 (49\%) & 106 (51\%)\\
	 \hline    
     \end{tabular}
\end{exe}                        
Der Vergleich der beiden Verteilungen produziert nach wie vor einen Chi-Quadrat-Wert, der knapp unter einer Signifikanz liegt ($\chi_{2}$ = 2,6427, p = 0,104).\\

\noindent
Nomenbezogene weiterführende RSe sind neben Pseudo-Cleft-Sätzen ein wei\-terer Fall, für den potenziell zu erwarten wäre, dass das Auftreten von \textit{halt} und \textit{eben} (bzw. ihrer Kombinationen), deren unterschiedliche Verteilung ich an die Vordergrundierung in rRSen und Hintergrundierung in aRSen binde, beeinflusst wird durch den Diskursbezug derartiger RSe. Aufgrund ihrer Relevanz im glo\-balen Diskursverlauf könnte es hier zu Verschiebungen im Partikelauftreten kommen, weil lokale und globale Vorder-/Hintergrundierung abweichen. Die Berücksichtigung dieser Strukturen weist aber keinen derartigen Einfluss nach.\footnote{Die grundsätzlichen Verhältnisse bleiben ebenfalls konstant unter Ausschluss sowohl der Pseudo-Clefts als auch der \textit{d}-wRSe. Auf der Basis des Erwartungswertes von 51,15\% (r) $[$48,83\% ... 53,47\%$]$ – 48,85\% (a) $[$46,53\% ... 51,17\%$]$ ergeben sich die folgenden Verhältnisse:\\

\noindent
\textit{halt}: 54 (56\%) (r) – 41 (44\%) (a) ($\chi_{2}$ = 1,2319, p = 0,267)\\
\textit{eben}: 93 (36\%) (r) – 157 (64\%) (a) ($\chi_{2}$ = 19,4705, p $<$ 0,001, V = 0,27)\\
Vergleich der Verteilungen im Vierfeldertest: $\chi_{2}$ = 10,8612, p $<$ 0,001, V = 0,18\\
\textit{eben halt}: 14 (28\%) (r) – 35 (72\%) (a) ($\chi_{2}$ = 9,9972, p $<$ 0,05, V = 0,45)\\
\textit{halt eben}: 86 (44\%) (r) – 106 (56\%) (a) ($\chi_{2}$ = 3,1065, p $<$ 0,05, V = 0,13)\\
Vergleich der Verteilungen im Vierfeldertest: $\chi_{2}$ = 4,2307, p $<$ 0,05, V = 0,13\\
\noindent
Eine Veränderung, die sich einstellt, ist, dass sich die Verteilung der \textit{halt eben}-RSe ebenfalls als signifikant herausstellt. Gleichzeitig entsteht nun ein signifikanter Wert beim Vergleich der beiden Verteilungen untereinander.

M.E. beeinflussen auch diese etwas veränderten Ergebnisse nicht den generellen Eindruck, für den ich argumentiere, dass \textit{eben halt} (und auch \textit{eben}) aus rRSen eher ausgeschlossen sind als \textit{halt eben} (und \textit{halt}).


} Zudem zeigt sich, dass \textit{d}-wRSe (zumindest in dieser Sorte Daten) nur eine untergeordnete Rolle spielen. Sie machen jeweils ca. 5\% (\textit{halt eben}, \textit{eben halt}), 6\% (\textit{eben}), 7\% (ohne MP) und 13\% (\textit{halt}) der aRSe aus.\footnote{Eine interessante unabhängige Frage ist, wo mit derartigen RSen zu rechnen ist. Obwohl gar nicht Thema seines Aufsatzes, finden sich zahlreiche Fälle in \citet{Mikame1998}, der ausschließlich literarische Beispiele behandelt.}


	






\chapter{Kombinationen aus \textit{doch} und \textit{auch}}
\label{chapter:dua} 
\section{Die Präferenz für \textit{doch auch} - Annahmen aus der Literatur und Korpusfrequenzen}
\label{sec:präferenz}
Für die Kombination von \textit{doch} und \textit{auch} ist zunächst einmal festzuhalten, dass die Abfolge \textit{doch auch} gegenüber der Reihung \textit{auch doch} klar die bevorzugte ist. (\ref{824}) bis (\ref{827}) zeigt einige Beispiele.

\begin{exe}
	\ex\label{824} 
	\scriptsize
	B: \glqq Sie wissen dass sie mir meinen Job nicht gerade leicht machen?\grqq{}\\
	A: \glqq \textbf{Na sie müssen sich ihr Geld \underline{doch auch} verdienen Lucius!} Wenn sie mich dann entschuldigen würden, 	ich muss noch einige Einkäufe tätigen und den organisatorischen Kram erledigen.\grqq{}
	\newline
	\hbox{}\hfill\hbox{(DECOW14AX01)}
	\newline
	\hbox{}\hfill\hbox{(http://www.tabletopwelt.de/index.php?/topic/92424-40k-rpg-20/)}
\end{exe}

\begin{exe}
	\ex\label{825} 
	\scriptsize
	Also das was Guido Knopp macht, sind bestimmt keine 100 prozentigen wissenschaftlich/historisch korrekten Dokumentationen. \textbf{Will er \underline{doch auch} gar nicht.} Sowas kann man auf Arte sehen. 
	\newline
	\hbox{}\hfill\hbox{(deWac: 1624)}
\end{exe}
	
\begin{exe}
	\ex\label{826} 
	\scriptsize
    \begin{tabular}[t]{ll}
	0956 BS	& $[$ähm$]$ schule hier her isst nur dann geht er zur therapie dann zum judo und um sechs\\
	{} & uhr kommt er heim und da muss er noch hausaufgaben machen\\
	{} & $^{o}$h hat er (.)$[$s so gesagt \emph{dass es ihm zu viel is}$]$\\
	0957 HM & $[$hm\_hm$]$\\
	0958 SZ & \hspace{1cm}\textbf{$[$(is \underline{doch auch}) (.) ä programm$]$}\\
	0959 NG & $[$hm$]$ \_hm\\
	0960 BS & und da hab ich gsagt	
	\hfill\hbox{(FOLK\_E\_00026\_SE\_01\_T\_01)} 					 
    \end{tabular}   
\end{exe}

\begin{exe}
	\ex\label{827} 
	\scriptsize
    \begin{tabular}[t]{ll}
	0089 S1 & Äh, hat, was haben Sie denn zu Weihnachten für Kuchen gebacken? Und \emph{haben Sie auch}\\
 	& \emph{Mohnklöße gehabt zu Weihnachten?}\\
	0090 S2 & \emph{Ja, selbstverständlich.}\\
	0091 S1	& Ja, erzählen Sie mal, wie Sie die\\
	0092 S2 & \textbf{Ich bin \underline{doch auch} Schlesier.}			
	\hfill\hbox{(OS--\_E\_00349\_SE\_01\_T\_01; DGD)}	 
    \end{tabular}       
\end{exe}	
Die Präferenz von \textit{doch auch} wird auch in der Literatur vertreten, wenn sich andere Autoren zu dieser Kombination äußern. Anders als bei \textit{ja} \& \textit{doch} und \textit{halt} \& \textit{eben} ist mir hier keine detaillier\-tere Untersuchung bekannt. \textit{Auch doch} gilt in anderen Arbeiten als ungrammatisch oder nicht belegt (vgl. \citealt[227, 230]{Dahl1988}, \citealt[356]{Lemnitzer2001}, \citealt[196]{Kwon2005}). Gleiches lässt sich aus Abfolgelisten (vgl. (\ref{828})) oder Klassenbildungen (vgl. (\ref{829})) ablesen (vgl. auch \citealt[91-94]{Engel1968}, \citealt[42]{Helbig1981}).

\begin{exe}
	\ex\label{828}Aussagemodus\\
	\scriptsize
	ja $>$ denn $>$ eben $>$ halt $>$ doch $>$ eben $>$ halt $>$ wohl $>$ einfach $>$ auch $>$ schon $>$ auch $>$ mal\\
	\newline
	\hbox{}\hfill\hbox{\citet[908, 1542]{Zifonun1997}}
\end{exe}
\vspace{-0.5cm}
\begin{exe}
	\ex\label{829} 
    \begin{tabular}[t]{lll}
	1. Gruppe & $>$ & 4. Gruppe\\
	\textit{doch} & {} & \textit{auch}\\
	(Konjunktionen, & {} & (Fokuspartikeln)\\
	Diskurspartikeln) & &			 
    \end{tabular}   
    \newline
	\hbox{}\hfill\hbox{\citet[31]{Thurmair1991}}   
\end{exe}
	
Modelle, die über eine reine Benennung der Partikelabfolgen hinausgehen, beabsichtigen ebenfalls, die Reihung \textit{doch auch} zu erfassen. 

Thurmair (1989: 221-222) erwähnt die Kombination nur im Zuge der Betrachtung der Kombinationen mit \textit{doch}, nicht unter den Kombinationen mit \textit{auch}. Sie argumentiert, dass ihre Hypothese 2 (vgl. Kapitel~\ref{chapter:hintergrund}, Abschnitt~\ref{sec:katalog} zu einer detaillier\-teren Darstellung von Thurmairs Ansatz) hier greife (vgl. \citeyear[288]{Thurmair1989}). Diese lautet, dass MPn, die Bezug auf die momentane Äußerung nehmen, vor MPn stehen, die eine qualitative Bewertung des Vorgängerbeitrags vornehmen. \textit{Doch} beschreibt sie durch die Merkmale BEKANNT$_{\textrm{H}}$, KORREKTUR, d.h. die Proposition ist dem Hörer aus Sicht des Sprechers bekannt und drückt eine Aufforderung aus, seine Ansichten zu ändern. \textit{Auch} zeigt an, dass die Vorgängeräuße\-rung aus Sprechersicht erwartet war (KONNEX, ERWARTET$_{\textrm{V/S}}$). Es liegt folglich sowohl der Bezug auf die aktuelle Äußerung vor (durch \textit{doch}) als auch auf die Vorgängeräußerung (durch \textit{auch}), die qualitativ bewertet wird. 

Wie in Kapitel~\ref{chapter:hintergrund}, Abschnitt~\ref{sec:ri} ausgeführt, vertritt \citet{Rinas2007}, dass MPn Skopus übereinander nehmen und die Abfolge die Skopusverhältnisse spiegelt. Unter Bezug auf das Beispiel in (\ref{830}) sieht er die Interpretation einer \textit{doch auch}-Assertion durch die Paraphrase in (\ref{831}) erfasst (vgl. auch (\ref{832})).

\begin{exe}
	\ex\label{830}
	Ein Vater beklagt sich, daß seine Tochter so frech und unverschämt ist. Die Großmutter: Was regst du dich denn auf? \textbf{Du hast ihr \underline{doch auch} immer alles durchgehen lassen.}
	\hfill\hbox{\citet[221]{Thurmair1989}}
\end{exe}

\begin{exe}
	\ex\label{831}
	\glq Die Einschätzung, dass der Sachverhalt q nicht überraschend ist angesichts von p, steht im Widerspruch zu einer Auffassung r und besagte 				Einschätzung oder die Auffassung r ist dem Hörer bekannt.\grq {}
\end{exe}

\begin{exe}
	\ex\label{832}
	DOCH(AUCH(p) $>>$ NICHT-ÜBERRASCHEND(q) WEIL(p))\\
	$>>$ WIDERSPRICHT((AUCH(p) $>>$ NICHT-ÜBERRASCHEND(q)\\
	 WEIL(p)), r) \& (KENNT(H, (AUCH(p) $>>$ NICHT-ÜBERRASCHEND(q)\\ WEIL(p))) $\vee$ KENNT(H,r))
\end{exe}	
Des besseren Verständnisses wegen führen (\ref{833}) und (\ref{834}) die Modellierung der Einzelpartikeln nach \citet{Rinas2007} an.
\begin{exe}
	\ex\label{833}
	DOCH(p) $>>$ WIDERSPRICHT(p, q) \& (KENNT(H, p) $\vee$ KENNT(H, q))
\end{exe}	
\vspace{-0.65cm}	
\begin{exe}
	\ex\label{834}
	AUCH(p) $>>$ NICHT-ÜBERRASCHEND(q) WEIL(p)
	\hfill\hbox{\citet[134/136]{Rinas2007}}	
\end{exe}
Die Propositionen aus (\ref{832}) entsprechen in (\ref{830}) p = der Vater hat das Verhalten der Tochter immer durchgehen lassen, q = die Tochter ist frech und unverschämt, r = der Vater ist über das Verhalten seiner Tochter empört. Die Interpretation von (\ref{830}) ist dann: Die Großmutter drückt aus, dass es nicht überraschend ist, dass die Tochter frech ist, angesichts der Tatsache, dass der Vater ihr Verhalten hat durchgehen lassen. Diese Bewertung steht im Widerspruch zur Einschätzung des Vaters, dass das Verhalten der Tochter empörend ist. Dem Vater ist entweder -- trivialerweise -- seine eigene Empörung bekannt oder der Inhalt der Bewertung der Großmutter.

Über die umgekehrte Abfolge schreibt \citet[149]{Rinas2007}:
\begin{quotation}
Angesichts der Skopus-Verhältnisse in dieser Kombination sollte eine Um\-kehrung der APn-Abfolge nicht möglich sein. Dies ist zutreffend: *Du hast ihr \textbf{auch doch} immer alles durchgehen lassen.
\end{quotation}
Und schließlich gibt es einige wenige Arbeiten, in denen authentische Daten angeführt werden. \citet[233]{Rath1975} findet unter 180 \textit{doch}-Äußerungen zwei Mal \textit{doch auch} und ein Mal \textit{doch aber auch}. \citet[53]{Rudolph1983} listet basierend auf einer Korpusuntersuchung die Mehrfachkombinationen in (\ref{835}) mit je einem Beleg.
			
\begin{exe}
	\ex\label{835}
	\textit{doch aber auch}, \textit{doch auch ganz}, \textit{doch auch nur}, \textit{doch eben auch}, \textit{eben doch auch}, \textit{ja doch auch}
\end{exe}					
\citet[254]{Hentschel1986} macht in ihren Daten ein Mal \textit{doch auch} und ein Mal \textit{auch doch} aus. In den gesprochenen Daten von \citet{Moellering2004} lässt sich bei allen von ihr hinsichtlich der MP-Funktion bereinigten \textit{doch}-Belegen ein \textit{doch auch}-Treffer finden (\citeyear[256]{Moellering2004}). Ihre \textit{auch}-Daten (\citeyear[450]{Moellering2004}) kann man nicht einsehen, weil sie nicht alle Belege im Kontext disambiguiert hat, um den MP-Gebrauch auszusondern. Die Korpusuntersuchung in \citet{Braber2010} fördert zwei \textit{doch auch} und einen \textit{auch doch}-Treffer zu Tage.

Die Annahme in allen deskriptiven oder theoretischen Arbeiten ist, dass die grammatische Abfolge \textit{doch auch} ist. Für mich bedeutet dies nicht, dass die E\-xistenz von \textit{auch doch} abzusprechen ist und nicht betrachtet werden sollte. Eine Frage, die es aber in jedem Fall zu klären gilt, ist, warum das \textit{doch} dem \textit{auch} präferiert vorangeht. Die Arbeiten, die Korpusdaten benutzt haben, erlauben keine Aussage zur (Nicht-)Existenz der umgekehrten Abfolge. Es werden zwar nur zwei \textit{auch doch}-Belege angeführt, man sieht aber, dass die Belegzahlen auch für \textit{doch auch} sehr niedrig sind.

Eine Problematik, die die Beschäftigung mit MPn zwar generell begleitet, die im Falle der MP-Kombination aus \textit{doch} und \textit{auch} aber verschärft auftritt, ist, dass es nötig ist, sich die MP-Äußerungen im Kontext anzuschauen, um das Risiko zu minimieren, es mit einer gleichlautenden Form einer anderen Wortart zu tun zu haben. Beide Partikeln sind hier für diese Fehlinterpretation anfällig. Die Angaben aus den angeführten Studien sind auch aus diesem Grund schwierig zu bewerten. In \citet{Braber2010} werden z.B. gar keine Belege angeführt. \citet{Rudolph1983} präsentiert einige Beispiele im Kontext, bei \citet{Moellering2004} ist der Kontext sehr knapp gehalten. Aus meinen eigenen Datenuntersuchungen (s.u.) weiß ich, wie heikel die Entscheidung sein kann, insbesondere, ob \textit{auch} als MP (oder nicht als Gradpartikel \is{Gradpartikel} oder \is{Konjunktionaladverb} Konjunktionaladverb) vorliegt. Dass die Betrachtung großer Datenmengen hier bisher ausgeblieben ist ($[$s.o.$]$ M{öllering passt beispielsweise aufgrund dieses Aspektes in ihrer \textit{auch}-Untersuchung), verwundert nicht, da ein entsprechender Aufwand zur Beantwortung dieser Frage erforderlich ist. Wenngleich sicherlich davon auszugehen ist, dass \textit{doch auch} deutlich überwiegen wird, bin ich dieser Frage in DeReKo und DGD2 nachgegangen, um herauszufinden, in welchem Verhältnis \textit{doch auch} und \textit{auch doch} in großen Datenmengen stehen. Da selbst in einem Teilkorpus von DECOW zu viele Treffer ausgegeben werden, um zu jedem Beispiel den Kontext zu suchen (anders als in DeReKo und DGD2 ist er nicht Teil der extrahierten Daten) und zu entscheiden, ob sowohl \textit{doch} als auch \textit{auch} als MP auftreten, kann ich hier nur eine sehr grobe Schätzung angeben. Es ist davon auszugehen, dass die Anzahl de facto niedriger ist. Höher ist sie in keinem Fall. Die Angabe ist unpräzise, vermittelt aber einen Eindruck der Größenordnung. Findet man die umgekehrte Abfolge, ermöglicht sich so auch ein Vergleich zum Verhältnis von \textit{halt eben} und \textit{eben halt}, bei dem die markierte Reihung gut belegt ist.

Da jeder Beleg im Kontext angeschaut werden muss, habe ich mir für die Daten in DeReKo eine Möglichkeit der Hochrechnung der Ergebnisse einer Stichprobe zu Nutze gemacht und \glq nur\grq {} ein Teilkorpus von DECOW betrachtet. Obwohl die Entscheidung für jeden Beleg einzeln vorgenommen wurde, gilt dennoch die Einschränkung, dass es sich sicherlich nicht um absolute Urteile handelt. Das Kriterium, das ich angesetzt habe, ist, ob \textit{doch} und \textit{auch} als MP interpretiert werden \underline{können}. Ich teste, ob die Partikeln jeweils alleine als MPn gebraucht werden \underline{könnten}, nicht ob sie es \underline{müssen} (wenn dies überhaupt zu entscheiden ist). Bei der Abfolge \textit{doch auch} gilt es, auszuschließen, dass \textit{auch} als Adverb \is{Adverb} engen Skopus nimmt; bei der Abfolge \textit{auch doch} dass \textit{doch} betont auftritt. Ein völliger Ausschluss des Vorkommens einer der \glq Dubletten\grq {} ist in manchen Äußerungen allerdings nahezu unmöglich. (\ref{836}) zeigt die Ergebnisse.

\begin{exe}
	\ex\label{836} Verteilung \textit{doch auch} - \textit{auch doch} in Korpora\\[-1em]
	\begin{tabular}[t]{|c|c|c|}
	\hline
	& \textit{doch auch} & \textit{auch doch}\\
	\hline
	DeReKo & 59 (\scriptsize{auf 500}) & 4\\
	& 6654 ... 11258 \scriptsize{(gesamt: 95\%-Konfidenzintervall}) & -\\
	& 8864 & \\
	\hline
	DGD2 & 60 & 2\\
	\hline
	DECOW14AX01 & 6552 \scriptsize{(Schätzung)} & 8\\
	\hline				 
    \end{tabular}    
\end{exe}	
In (\ref{837}) bis (\ref{842}) führe ich einige Belege an.

\begin{exe}
	\ex\label{837}
	\scriptsize
	Bei den Diskussionen über eine Ausbildungsplatzabgabe nennen Sie immer die Bauindustrie als positives Beispiel und behaupten, dass das dortige 				Umlagesystem wunderbar funktioniere.\\
	(Willi Brase $[$SPD$]$: \textbf{Das funktioniert \underline{doch auch}!})
	\newline
	\hbox{}\hfill\hbox{(PBT/W15.00074 Protokoll der Sitzung des Parlaments Deutscher Bundestag am 12.11.2003)}
\end{exe}

\begin{exe}
	\ex\label{838}
	\scriptsize
	Es wurde natürlich immer von den Heidelbeeren ziemlich äh Marmelade gekocht, die sofort weggegessen wurde aufs Brot, \textbf{weil sie \underline{doch 		auch} hinten und vor nicht gereicht hat}, beziehungsweise immer ge\-spart werden mußte, nicht!	         
	\hfill\hbox{(OS--\_E\_00065\_SE\_01\_T\_01)}
\end{exe}

\begin{exe}
	\ex\label{839}
	\scriptsize
	0350 S2	ich meine s$+$ $+$g$+$ das ist eben ne Schachtel Zigaretten im Monat im höchsten Falle $+$s . ich mein s$+$ die sollte doch wohl jeder über 		haben $+$s . $+$g$+$ \textbf{wird ihm \underline{doch auch} einiges für geboten.}/ und der z$+$ NN $+$z hat vor i$+$ so ne $+$g$+$ diese 					Kaminabende(nich?)	         
	\hfill\hbox{(FR--\_E\_00064\_SE\_01\_T\_01)}
\end{exe}			

\begin{exe}
	\ex\label{840}
	\scriptsize
	0142 S2 ... $\plus$p$\plus$ ( ja und?) $\plus$p$\plus$ was hätte da jetzt in diesem Augenblick anders sein können?.\\
	0143 S1	$\plus$p$\plus$ ... s is schon gut so.\\
	0144 S2	( f$\plus$ nei $\plus$f )( na) was is?. $\plus$g$\plus$ es is doch gar nichts gewesen./ \textbf{es war ja \underline{auch doch}} $\plus$g$			\plus$ war keine gar keine so Absicht i$\plus$ so irgendein $\plus$g$\plus$ durch irgendein Verhalten irgendwas zu bewirken $\plus$i ./ ich mußte 		ja hereinplatzen ,$\plus$ nachdem ich dich zehn Minuten gesucht hab $\plus$p$\plus$ und gar nicht damit gerechnet( doch) zum Schluß damit gerechnet 	hab $\plus$, ,$\plus$ daß du noch oben bist $\plus$, ,$\plus$ weil da wieder s Licht an war $\plus$, ./ aber da hattest du schon die...                              
	\hfill\hbox{(FR--\_E\_00106\_SE\_01\_T\_01)}
\end{exe}											     				

\begin{exe}
	\ex\label{841}
	\scriptsize
	Hallo Atze,\\
	$>$ '/bin/bash': No such file or directory\\
	gibt es die Datei /mnt/hdb2/knx/source/KNOPPIX/bin/bash\\
	Tschuess\\
	Karl\\
	
	\noindent
	Hey Karl,\\
	nee, \textbf{kann es \underline{auch doch} gar nicht geben}, oder?\\
	das ist doch der Inhalt der Knoppix CD den ich da rüberkopieren muss oder?\\
	BZW. oder hab ich hier was falsch verstanden? 
	\hfill\hbox{(DECOW14AX01)}
	\newline
	\hbox{}\hfill\hbox{(http://www.knoppixforum.de/knoppix-forum-deutsch/remastern/}
	\newline
	\hbox{}\hfill\hbox{thread2133/probleme-beim-remastern-von-knoppix-5-0.html)}
\end{exe}

\begin{exe}
	\ex\label{842}
	\scriptsize
	Da mein Lipobestand langsam immer größer wird \textbf{und man \underline{auch doch} immer wieder Horrorgeschich\-ten von Lipobränden hört} wollte ich mal 	fragen ob den Interesse besteht eine Sammelbestellung für Munitionskisten zu organisieren. Ich hab selber schon so eine Box und mir ist es mittlerweile echt wohler wenn meine Akkus da drin liegen. Aber die eine Kiste reicht nicht mehr daher brauch ich Nachschub.						         
	\hfill\hbox{(DECOW14AX01)}
	\newline
	\hbox{}\hfill\hbox{(http://www.modellbauvideos.de/board/wbb/sonstiges/sammelbestellungen}
	\newline
	\hbox{}\hfill\hbox{/2031-aufbewahrungsbox-fuer-lipos-munitionskiste/)}
\end{exe}								
Nach der Bereinigung einer Zufallsauswahl von 500 Treffern bleiben 59:4 Belege übrig. Die Hochrechnung auf die Gesamtdatenmenge (unter Berücksichtigung der Gesamttextwörterzahl im Korpus sowie der für die Anfragen ausgegebenen Gesamttrefferzahl) (zum genauen Vorgehen vgl. \citealt[Kapitel 5.6]{Perkuhn2012}) ergibt, dass mit 95\%iger Wahrscheinlichkeit mit 6654 bis 11258 \textit{doch auch}-Belegen zu rechnen ist. Für \textit{auch doch} lässt sich dieses Intervall nicht berechnen, weil die Bedingung nicht erfüllt wird, dass im Korpus mindestens neun Treffer vorliegen. Die Schätzung entlang des Anteils wäre, dass ca. 7 Treffer vorhanden sind.

Anhand der Verwendung der beiden MP-Kombinationen bestätigt sich folg\-lich (wie erwartet) die sehr klare Präferenz für \textit{doch auch}, die der Intuition entspricht, die aus den geringen Belegzahlen bestehender Arbeiten aber nicht geschlossen werden konnte. Wenngleich die Belegzahlen für \textit{auch doch} im Vergleich sehr gering aussehen und diese Abfolge natürlich unterrepräsentiert ist, ist dies für mich kein Grund, die \textit{auch doch}-Treffer nicht näher zu untersuchen und in ihnen nach Mustern zu suchen. Wählt man eine größere Datenmenge (das DECOW14-Gesamtkorpus), steht eine größere Datensammlung zur Verfügung. Ohne Zweifel ist die Lupe sehr groß. Es steht fest, dass es die zentrale Aufgabe der Analyse ist, die deutliche Präferenz von \textit{doch auch} zu erklären. Dennoch sollte es erlaubt sein und sollte die Analyse in der Lage sein -- sofern sich eine Systematik feststellen lässt -- Gründe zu benennen, warum die Umkehr gerade dort stattfindet. Auch in Kapitel~\ref6{chapter:jud} basiert meine Argumentation, dass man die Abfolge \textit{doch ja} nicht komplett ausschließen sollte, nicht auf Frequenzen, sondern der Tatsache, dass sie auf bestimmte Kontexte beschränkt zu sein scheint.

In den DeReKo-Daten machen die \textit{doch auch}-Treffer nur 0,1\% der Kombinationen aus \textit{doch} und \textit{auch} aus. Im Falle von \textit{halt eben} und \textit{eben halt} (715-117) nimmt das in meiner Argumentation markierte \textit{eben halt} immerhin 14\% der kombinierten Fälle ein. Als Vergleichswert habe ich in (\ref{843}) die Verteilung von \textit{ja doch} und \textit{doch ja} in den DeReKo-Daten kalkuliert. 
\pagebreak
\begin{exe}
	\ex\label{843} Verteilung \textit{ja doch} - \textit{doch ja} in DeReKo\\[-1em]
	\begin{tabular}[t]{|c|c|c|}
	\hline
	& \textit{ja doch} & \textit{doch ja}\\
	\hline
	DeReKo & 135 (\scriptsize{auf 500}) & 8 (auf 337)\\
	& 4049 ... 5715 \scriptsize{(gesamt: 95\%-Konfidenzintervall}) & -\\
	\hline
	& 4813 & \\
	\hline				 
    \end{tabular}    
\end{exe}	
Die \textit{doch ja}-Treffer machen 0,2\% der Kombinationen aus. Die Verteilung von \textit{doch auch} und \textit{auch doch} ähnelt in der Größenordnung somit der Verteilung von \textit{ja doch} und \textit{doch ja}. Wohlgemerkt stammt auch da die größte Menge von Belegen, die Anlass für meine Argumentation gegeben hat, aus Webdaten. 

\section{Distribution von \textit{doch}, \textit{auch} und \textit{doch auch}}
\label{sec:distributionda}
Verfolgt man die Absicht, eine Erklärung für die Präferenz von \textit{doch auch} gegenüber \textit{auch doch} zu finden, ist es entscheidend, zu wissen, welche Satzmodi \is{Satzmodus} bzw. Äuße\-rungstypen \is{Illokutionstyp} erfasst werden können müssen. Wie bereits in Kapitel~\ref{chapter:jud}, Abschnitt~\ref{sec:distributionjd} und Kapitel~\ref{chapter:hue}, Abschnitt~\ref{sec:bedhe} angeführt, können sich die gleichen MPn nicht in allen (Satz)kontexten kombinieren. Die etablierte Annahme, auf die ich mich hier verlasse, ist die Schnittmengenbedingung \is{Schnittmengenbedingung} aus \citet{Thurmair1989, Thurmair1991}: MPn können sich nur in den Satzmodi kombinieren, in denen sie auch in Isolation auftreten können.\\

\noindent
\textit{Doch} weist eine sehr weite Distribution auf (vgl. (\ref{844}) bis (\ref{848})).

\begin{exe}
	\ex\label{844} Deklarativsatz\\
	Die Strecke über HH-Harburg ist \textbf{doch} länger.
\end{exe}

\begin{exe}
	\ex\label{845} E-Interrogativsatz\\
	*Hast du \textbf{doch} am Wochenende Zeit?
\end{exe}	
	
\begin{exe}
	\ex\label{846} w-Interrogativsatz\\
	Wie heißt \textbf{doch} (gleich) der Platz in Nippes, wo täglich Markt ist?
\end{exe}		
	
\begin{exe}
	\ex\label{847} Imperativsatz\\
	Mach \textbf{doch} die Heizung an!
\end{exe}	
		
\begin{exe}
	\ex\label{848} Optativsatz\\
	Hätte ich \textbf{doch} am Gewinnspiel teilgenommen!
\end{exe}		
Die Auftretensmöglichkeiten in Exklamativsätzen unterscheiden sich je nach Typ von Exklamativsatz. Aus Satzexklamativsätzen mit $[\minus$w, V2$]$-Stellung ist \textit{doch} ausgeschlossen (vgl. (\ref{849})).
	
\begin{exe}
	\ex\label{849}Satzexklamativsätze\\[-1.25em]
		\begin{xlist}	
			\ex\label{849a} *DER hat \textbf{doch} einen Bart!
			\ex\label{849b} DER hat \textbf{aber}/\textbf{vielleicht} einen Bart!	
			\hfill\hbox {\citet[218]{Rinas2006}}
		\end{xlist}
\end{exe}	
	
\begin{exe}
	\ex\label{850}\textit{dass}-Exklamativsatz\\
	Daß der mir \textbf{doch} die Vorfahrt nimmt!
	\newline
	\hbox{}\hfill\hbox {\citet[152]{Zaefferer1988}}
\end{exe}
	
\begin{exe}
	\ex\label{851}w-Exklamativsatz\\[-1.25em] 
		\begin{xlist}	
			\ex\label{851a} Was BIST du \textbf{doch} bloß für ein Mensch!
			\ex\label{851b} Wie SCHÖN du \textbf{doch} bist!	
			\hfill\hbox {\citet[218-219]{Rinas2006}}
		\end{xlist}
\end{exe}		
Hat man diese \textit{doch}-Kontexte bestimmt, lässt sich überprüfen, in welcher dieser Umgebungen \textit{auch} ebenfalls auftreten kann. Ist nur eine oder keine Partikel akzeptabel, sollten auch beide nicht kombiniert möglich sein.

An (\ref{852}) sieht man, dass der Deklarativsatz ein zulässiger Kontext ist.

\begin{exe}
	\ex\label{852}
	Die Strecke über HH-Harburg ist \textbf{doch}/\textbf{auch}/\textbf{doch auch} länger.
\end{exe}
Da bei \textit{auch} die Unterscheidung zum Konjunktionaladverb \is{Konjunktionaladverb} und zur Gradpartikel \is{Gradpartikel} schwierig sein kann, benötigt man hier Kontext, um die MP-Lesart (eindeutig(er)) nahezulegen.\footnote{Es ist wichtig, die MPn in geeigneten Kontexten hinsichtlich ihrer strukturellen Auftretensmäglichkeiten zu überprüfen, da auch die ausgedrückten Sachverhalte bereits Möglichkeiten des Gebrauchs mitsteuern. Stellt man fest, dass eine Partikel in einer bestimmten strukturellen Umgebung in einer isolierten Äußerung nicht akzeptabel erscheint, sollte man erst überprüfen, ob ein geeigneter Kontext fehlt. Ich gebe deshalb meist typische Kontexte für beide MPn an, auch wenn sie dann nicht in ein und demselben Satz vorkommen.} (\ref{853}) zeigt einen typischen Verwendungskontext.

\begin{exe}
	\ex\label{853}
	A: (Eine alte Frau ist auf der Straße ausgerutscht und hat sich verletzt.)\\
	B: Es ist \textbf{auch} furchtbar glatt auf der Straße.
	\hfill\hbox {\citet[88]{Helbig1990}}
\end{exe}
Im Rahmen der Untersuchung dieser beiden MPn sind \is{V1-Deklarativsatz} der V1- und \textit{Wo}-VL-Deklara\-tivsatz \is{Wo-VL-Deklarativsatz} von besonderem Interesse (vgl. auch meine gesonderte Betrachtung in Abschnitt~\ref{sec:Rand}).

\begin{exe}
	\ex\label{854}
	\scriptsize
	Mit zwei Koffern in der Hand, ganz neu wollte ich anfangen in Berlin. Die erste Zeit schlief ich auf einem Feldbett, umringt von Bauschutt und Zement. 		Daß um mich herum saniert wurde, störte mich damals nicht. \textbf{\textit{Wollte} ich \underline{doch auch} mein eigenes Leben sanieren.}  	
	\hfill\hbox {(TAZ, 28.03.1991, 25)}
	\newline
	\hbox{}\hfill\hbox {\citet[74-75]{Kwon2005}}
\end{exe}
   
\begin{exe}
	\ex\label{855}
	\scriptsize
	Auf eine kostspielige Asbest-Voruntersuchung wurde gutgläubig verzichtet. \textbf{\textit{Wo} \underline{doch auch} die Bundesbahn Gegenteiliges 			versichert habe.} 
	\hfill\hbox {(TAZ, 14.09.1996, 34)}
	\newline
	\hbox{}\hfill\hbox {\citet[75]{Kwon2005}}
\end{exe}  
Aus Perspektive der Schnittmengenbedingung \is{Schnittmengenbedingung} stellen sie eine Besonderheit dar, da \textit{auch} in Isolation in ihnen nicht verwendet zu werden scheint. Sie enthalten \textit{doch} oder die Kombination \textit{doch auch}. Die Partikel \textit{doch} wiederum ist für diese Sätze sehr typisch (\textit{Wo}-VL) bzw. sogar obligatorisch (V1).\\

\noindent
Wie oben bereits gesehen, kann \textit{doch} im E-Interrogativsatz nicht stehen. Obwohl \textit{auch} zulässig ist, ist das kombinierte Auftreten ausgeschlossen (vgl. (\ref{856})). (\ref{857}) zeigt einen typischen Kontext für \textit{auch} im \is{E-Interrogativsatz} E-Interrogativsatz.

\begin{exe}
	\ex\label{856}E-Interrogativsatz\\
	Hast du \textbf{*doch/auch/*doch auch} am Wochenende Zeit?
\end{exe}

\begin{exe}
	\ex\label{857}
	(Der Nikolaus fragt die Kinder:) Wart ihr \textbf{auch} brav gewesen? 
	\newline
	\hbox{}\hfill\hbox {\citet[57]{Dahl1988}}
\end{exe}
Aus dem gleichen Grund (mit anderer (in)akzeptabler Verteilung) ist eine Kombination auch im Optativsatz \is{Optativsatz} nicht möglich.

\begin{exe}
	\ex\label{858}Optativsatz\\
	Hätte ich \textbf{doch/*auch/*doch auch} am Gewinnspiel teilgenommen!
\end{exe}
Im w-Interrogativsatz \is{w-Interrogativsatz} können beide Partikeln stehen (vgl. (\ref{859a}) und (\ref{859b})), das gemeinsame Auftreten führt aber zu einer inakzeptablen Struktur (s.u. für meine Erklärung).
	
\begin{exe}
	\ex\label{859}w-Interrogativsatz\\[-1.25em]
		\begin{xlist}	
			\ex\label{859a} A: Mir ist furchtbar kalt.\\
			B: Warum ziehst du dich \textbf{auch} so leicht an?
			\hfill\hbox {\citet[89]{Helbig1990}}
			\ex\label{859b} Warum ziehst du dich \textbf{doch} (gleich) so leicht an? (Du hast es mir schon mal erklärt.)	
	 		\ex\label{859c} *Warum ziehst du dich \textbf{doch auch} so leicht an?			
		\end{xlist}
\end{exe}		
Da sowohl \textit{doch}- als auch \textit{auch}-Imperativsätze möglich sind, steht im Einverneh\-men mit der syntaktischen Schnittmengenbedingung auch der Kombination nichts im Wege (vgl. (\ref{860})). In (\ref{861}) findet sich ein typischer Kontext für einen \textit{auch}-Imperativsatz.

\begin{exe}
	\ex\label{860}Imperativsatz\\
	Mach \textbf{doch/auch/doch auch} die Heizung an!
\end{exe}

\begin{exe}
	\ex\label{861}
	Schreibe \textbf{auch} ordentlich!
	\hfill\hbox {\citet[90]{Helbig1990}}
\end{exe}
In Exklamativsätzen \is{Exklamativsatz} beobachtet man Unterschiede je nach Exklamativsatztyp. Aus dem Satzexklamativsatz \is{Satzeklamativsatz} scheint \textit{doch} (den Beispielen aus der Literatur folgend) (zumindest in Isolation) ausgeschlossen (vgl. (\ref{862})).

\begin{exe}
	\ex\label{862}Satzexklamativsatz\\[-1.25em]
		\begin{xlist}	
			\ex\label{862a} *DER hat \textbf{doch} einen Bart! vs. DER hat \textbf{aber/vielleicht} einen Bart!
			\newline
			\hbox{}\hfill\hbox {\citet[218]{Rinas2006}}
			\ex\label{862b} *DER ist \textbf{doch} alt geworden!
	 		\ex\label{862c} *Ist DER \textbf{doc}h alt geworden!		
	 		\hfill\hbox {\citet[224]{Kwon2005}}	
		\end{xlist}
\end{exe}	
\textit{Auch} (in Kombination mit \textit{aber}) ist aber akzeptabel.

\begin{exe}
	\ex\label{863}
		\begin{xlist}	
			\ex\label{863a} Donnerwetter, DAS ist \textbf{aber auch} ein Busen!	
			\hfill\hbox {(TAZ, 30.11.1988, 14)}
	 		\ex\label{863b} Ist DAS \textbf{aber auch} ein Busen!		
	 		\hfill\hbox {\citet[224]{Kwon2005}}	
		\end{xlist}
\end{exe}	
Nach der syntaktischen Schnittmengenbedingung sollte die Kombination nicht auftreten. Äußerungen wie (\ref{863}) werden für mein Befinden aber nicht inakzep\-tabel, wenn das \textit{doch} hinzutritt. Eindeutige Belege habe ich für die Sequenz \textit{doch aber auch} oder \textit{aber doch auch} weder in DeReKo noch DECOW finden können. Exklamativsätze sind (in diesen Daten) aber generell nur sehr schwierig zu belegen. Die \textit{aber auch}-Treffer, für die sich die Interpretation als Satzexklamativsatz anbietet, stört das Hinzufügen von \textit{doch} allerdings nicht (vgl. z.B. (\ref{864})).

\begin{exe}
	\ex\label{864}
	\scriptsize
	Schön, dass alle wieder gut nach Hause gekommen sind! \textbf{Das war aber auch ein Sch...wetter!} Nur gabba störte das wenig, der hatte eine 				passende Mütze auf!   
	\newline
	\hbox{}\hfill\hbox {(www.tt-board.de/forum/archive/index.php/t-15413.html)}
	\newline
	\hbox{}\hfill\hbox {(DECOW14AX)}
\end{exe}   													  
\textit{Dass}- und w-Exklamativsätze zeigen ein eindeutiges Bild: Beide Partikeln können in Isolation und gemeinsam auftreten (vgl. (\ref{865}) bis (\ref{868})).

\begin{exe}
	\ex\label{865} \textit{dass}-Exklamativsatz\\
	Daß der mir \textbf{doch/auch/doch auch} die Vorfahrt nimmt!  
	\newline
	\hbox{}\hfill\hbox {nach \citet[152]{Zaefferer1988}}
\end{exe} 
\vspace{-0.5cm}
\begin{exe}
	\ex\label{866} 
	Daß der Zug \textbf{auch} gerade heute so viel Verspätung hat!
	\newline
	\hbox{}\hfill\hbox {nach \citet[90]{Helbig1990}}
\end{exe} 	

\begin{exe}
	\ex\label{867}w-Exklamativsatz\\[-1.25em]
		\begin{xlist}	
			\ex\label{867a} Was war das \textbf{doch} für ein Fußballspiel!	
			\hfill\hbox {\citet[116]{Helbig1990}}
	 		\ex\label{867b} Wie SCHÖN du \textbf{doch} bist!		
	 		\hfill\hbox {\citet[218-219]{Rinas2006}}
		\end{xlist}
\end{exe}

\begin{exe}
	\ex\label{868}
		\begin{xlist}	
			\ex\label{868a} Was war das \textbf{auch} für ein Erfolg!
			\hfill\hbox {\citet[90]{Helbig1990}}
	 		\ex\label{868b} Was der Kerl \textbf{auch} für Einfälle hat!		
	 		\hfill\hbox {\citet[177]{Schubiger1977}}
		\end{xlist}
\end{exe}	
Im Falle der w-Exklamativsätze ist es dabei unerheblich, ob ein V2- oder VE-Satz vorliegt und \textit{auch} muss auch nicht mit \textit{aber} kombiniert sein.

Die Betrachtung dieser Satzkontexte zeigt, dass man es in Deklarativ-, Impera\-tiv-, w-, \textit{dass}- und ggf. auch Satzexklamativsätzen mit der Sequenz \textit{doch auch} zu tun hat. Diese Feststellung ist für die intendierte Ableitung der Reihenfolgebe\-schränkung bzw. -präferenz hochrelevant, da sie mitbeeinflusst, welcher Art die Reihungsbeschränkung sein kann. In Kapitel~\ref{chapter:jud} musste die Analyse der Abfolge von \textit{ja} und \textit{doch} nur für Assertionen aufkommen. Die dort vorgeschlagene Beschränkung bezog sich deshalb auch auf ein Charakteristikum von Assertionen. Im Falle von \textit{halt} und \textit{eben} (vgl. Kapitel~\ref{chapter:hue}) ergab es sich bereits, dass die vorgeschlagene Ableitung einen weiteren Anwendungsbereich aufweisen musste, weil es notwen\-dig war, Assertionen und Direktive zu erfassen. In der aktuellen Betrachtung von \textit{doch} und \textit{auch} sollte die Restriktion nun so beschaffen sein, dass sie prinzipiell assertive, direktive und exklamative Äußerungen auffangen kann. Mit Ausnahme der Randtypen des \is{Wo-VL-Deklarativsatz} \textit{Wo}-VL- und V1-Deklarativsatzes \is{V1-Deklarativsatz} (dazu s. Abschnitt~\ref{sec:Rand}) lässt sich die Akzeptabilität der MP-Kombination in Deklarativ-, Impe\-rativ- und Ex\-klamativsätzen sowie die Inakzeptabilität derselben im E-Interrogativ- und Optativsatz recht einfach über die syntaktische Schnittmengenbedingung aus \citet{Thurmair1989, Thurmair1991} ableiten. Zum Auftreten von \textit{doch} und \textit{auch} in w-Interroga\-tivsätzen gibt es hingegen mehr zu sagen: Die syntaktische Schnittmengenbedingung, die auf Satzmodi Bezug nimmt, kann nicht entscheidend sein, da die beiden Partikeln hier prinzipiell stehen können (vgl. (\ref{869})).

\begin{exe}
\ex\label{869}
Warum ziehst du dich \textbf{doch} (gleich)/\textbf{auch}/\textbf{*doch auch} so leicht an?
\end{exe}	
\citet[281]{Thurmair1989}; (\citeyear[27]{Thurmair1991}) und auch schon \citet[218, 222, 224-225]{Dahl1988} führen andere Beispiele an, in denen die Satzmodusbedingung eigentlich erfüllt ist, die Kombination aber trotzdem nicht zulässig ist. Sie erklären diese Fälle über eine semantisch-pragmatische Schnittmengenbedingung. Die beiden MPn bzw. die Äußerungen, die sie enthalten, dürfen auch nicht inkompatible Interpretationen oder Verwendungsbedingungen aufweisen. 	
	
Zu \textit{auch}-w-Fragen heißt es in der deskriptiven Literatur, dass als Reaktion auf eine solche Frage vom Sprecher eine negative oder gar keine Antwort erwartet wird (vgl. (\ref{869}), (\ref{870})). 

\begin{exe}
	\ex\label{869}
	A: Ich bin heute sehr müde.\\
	B: Warum gehst du \textbf{auch} immer so spät ins Bett?		
	\hfill\hbox {\citet[89]{Helbig1990}}\\
	(= Du sollst nicht so spät ins Bett gehen \& es ist klar, dass du müde bist, wenn du so spät ins Bett gehst.)		
\end{exe}	

\begin{exe}
	\ex\label{870}
	A: Ich friere so.\\
	B: Warum ziehst du dich \textbf{auch} so leicht an bei so nem nasskalten Wetter?
	\newline	
	\hbox{}\hfill\hbox {\citet[218]{Franck1980}}\\
	(= Du sollst dich nicht so leicht anziehen \& es ist klar, dass du frierst, wenn du dich so anziehst.)		
\end{exe}	
Man hat es folglich weniger mit einer echten Frage, d.h. einer \is{Informationsfrage} Informationsfrage, sondern einer rhetorischen Frage \is{rhetorische Frage} zu tun. Es handelt sich eher um einen Kommentar/eine Begründung der Vorgängeräußerung bzw. eine Bewertung der Proposition der Frage (vgl. \citealt[218-219]{Franck1980}, \citealt[51-54]{Dahl1988}, \citealt[158-159]{Thurmair1989}, \citealt[89]{Helbig1990}, \citealt[231]{Karagjosova2004}, \citealt[77, 202]{Kwon2005}).

In \textit{auch}-w-Interrogativsätzen tritt oft ein kausales w-Pronomen auf, möglich sind aber auch andere w-Ausdrücke (vgl. (\ref{871}), (\ref{872})).

\begin{exe}
	\ex\label{871}
	Der Jochen muß 4.000 Mark Kaution bezahlen! Aber wer unterschreibt \textbf{auch} einen Mietvertrag, ohne ihn vorher genau durchzulesen?	
	\newline
	\hbox{}\hfill\hbox {\citet[159]{Thurmair1989}}
\end{exe}
\vspace{-0.65cm}	
\begin{exe}
	\ex\label{872}
	Was wollte sie \textbf{auch} hier?
	\hfill\hbox {\citet[51]{Dahl1988}}	
\end{exe}
\textit{Doch}-Fragen, wie in (\ref{873}) und (\ref{874}), werden so charakterisiert, dass der Sprecher nach einer Information fragt, die er eigentlich kennt, die er aber vergessen hat/an die er sich im Moment nicht erinnern kann. Der Sprecher will die Antwort vom Hörer deshalb erneut erfahren, wobei es keine Voraussetzung für eine solche Äußerung ist, dass der Hörer die Frage beantworten kann. Sie erfragt beispiels\-weise nicht allgemein Bekanntes (\citealt[88]{Dahl1988}, \citealt[117]{Thurmair1989}, \citealt[114]{Helbig1990}, \citealt[204]{Kwon2005}).

\begin{exe}
	\ex\label{873}
	Wie heißt \textbf{doch} euer Hund?
	\hfill\hbox {\citet[114]{Helbig1990}}
\end{exe}
\vspace{-0.65cm}	
\begin{exe}
	\ex\label{874}
	Wer war \textbf{doch} der berühmte Feuerfresser im Zirkus Krone?
	\hfill\hbox {\citet[88]{Dahl1988}}	
\end{exe}
Auf der Basis dieser Eindrücke und Charakterisierungen müsste eine w-Frage, die die Sequenz \textit{doch auch} beinhaltet, gleichzeitig eine rhetorische Frage sein, deren Antwort beiden Diskursteilnehmern als bekannt vorausgesetzt wird, und eine Frage, mit der der Sprecher sich an etwas erinnern möchte, das er eigentlich weiß. Der Sprecher weiß die Antwort folglich wirklich, nimmt an, dass er und der Hörer sie wissen (\textit{auch}) und er weiß sie nur eigentlich und fragt deshalb nach/möchte sich erinnern (\textit{doch}). Er würde somit ausdrücken, dass er in der konkreten Situation die Antwort weiß (\textit{auch}) und nicht weiß (\textit{doch}). Der Hörer muss die Frage beantworten können (\textit{auch}) und er muss sie nicht beantworten können (\textit{doch}). Diese Verwendungsbedingungen sind schlichtweg unvereinbar bzw. meine Vorhersage wäre, dass die Kombination möglich wird, wenn ein Kontext vorliegt, in dem dieser Konflikt nicht entsteht.

Wenngleich es bei einer Ableitung der Abfolgepräferenz von \textit{doch} und \textit{auch} folglich Deklarativ-, Imperativ- und Exklamativsätze zu berücksichtigen gilt, be\-schränke ich mich im Folgenden zunächst auf \is{Deklarativsatz} Deklarativsätze. Der Beitrag der MPn (besonders gilt dies für \textit{auch}) ist am besten für Deklarativsätze bzw. Assertionen \is{Assertion} untersucht worden. Viele Ansätze zu \textit{auch} berücksichtigen nicht alle drei Auftretenskontexte (vgl. z.B. \citealt{Franck1980}, \citealt[100-105]{Burkhardt1982} und \citealt{Ickler1994}, die Imperativ- und Exklamativsätze nicht behandeln, \citealt{Dahl1988}, der w-Exklamative nicht beücksichtigt, \citealt[222]{Karagjosova2004}, die Imperativsätze aus\-klammert). Dies hat vor allem damit zu tun, dass unklar ist, ob einer MP in verschiedenen Satzmodi dieselbe Interpretation zugeschrieben werden kann. Diese Frage ist Teil der in dieser Arbeit schon mehrfach angesprochenen Grundsatzdiskussion zur Konkret-/Abstraktheit von \is{Bedeutungsminimalismus/-maximalismus} MP-Beschreibungen (Bedeutungs\-minimalismus/-maximalismus): Ist für eine Form von verschiedenen, abweichenden Bedeutungen auszugehen oder gibt es eine invariante Grundbedeutung und die Variation ist auf andere Faktoren zurückzuführen? Dass MP-Beschreibungen i.d.R. anhand von Deklarativsätzen erfolgen, ist sicherlich auch darauf zurückzuführen, dass generell unterschiedlich viel Klarheit über die Satzmodi besteht. Zu Exklamativsätzen wird zudem sogar die Diskussion geführt, ob es sich hier überhaupt um einen eigenen Satzmodus \is{Satzmodus} handelt (vgl. \citealt{Naef1987}, \citealt{Rosengren1992, Rosengren1997}). Und auch im Rahmen der \is{Kontextwechseltheorie} Kontextwechseltheorie, die den Hintergrund fär meine Analyse des Beitrags des Satzkontextes ausmacht, sind vor allem Deklarativsätze bzw. Assertionen behandelt worden. Ich halte deshalb das Vorgehen für legitim, sich für die Formulierung einer Ableitung zunächst auf Deklarativsätze zu beschränken, und anschließend zu überlegen, wie sich auch andere Satzmodi in das entworfene Bild einfügen lassen. 

Der folgende Abschnitt~\ref{sec:V2} beschäftigt sich deshalb zunächst nur mit Standard-Deklarativsätzen. Wie schon erwähnt, verkompliziert sich auch im Bereich der Deklarativsätze mit den Randtypen (\textit{Wo}-VL- und V1-Sätze) das Bild. Diese assertiven Äußerungen sind Gegenstand von Abschnitt~\ref{sec:Rand}. In Abschnitt~\ref{sec:direktive} werde ich die Analyse auf Imperativsätze übertragen. 

\section{V2-Deklarativsätze}
\label{sec:V2}
\subsection{Das Einzelauftreten von \textit{doch} und \textit{auch}}
\subsubsection{\textit{doch}}
\label{sec:doch}
Für \textit{doch} habe ich in Kapitel~\ref{chapter:jud}, Abschnitt~\ref{sec:doch1} schon eine Analyse vorgeschlagen, die ich an dieser Stelle beibehalte. Im Folgenden wiederhole ich diese Modellierung der besseren Lesbarkeit halber kurz. Unter Berufung auf \citet{Diewald1998} gehe ich davon aus, dass \textit{doch} eine konzessive Relation indiziert. Der \textit{doch}-Äußerung geht eine Situation voran, in der die Frage offen ist, ob das, was die Assertion ausdrückt, gilt oder nicht gilt. Vor dem Hintergrund zweier im Kontext bestehender Alternativen vertritt der Sprecher eine der beiden Möglichkeiten. Der Sprecher entscheidet sich für die in seiner Äußerung ausgedrückte Proposition, obwohl die gegenteilige Annahme ebenfalls kontextuell aktiviert ist. Aus \textit{es steht die Frage im Raum, ob p oder non-p gilt}  aus \citet{Diewald1998} wird in meiner Modellierung, dass die Frage schon auf dem Tisch liegt, bevor die \textit{doch}-Assertion gemacht wird. Die Assertion steuert ferner völlig regulär das Bekenntnis des Sprechers zur Proposition bei.

Typischerweise beziehen sich \textit{doch}-Äußerungen auf Implikaturen \is{Implikatur} aus der Vor\-gängeräußerung. Für den Dialog in (\ref{875}) kann man annehmen, dass die erste Äußerung das Gegenteil des zweiten Beitrags implikatiert (vgl. (\ref{875})).

\begin{exe}
	\ex\label{875}
	A: Patrick ist nicht zu Hause.\\
	B: Aber sein Auto ist \textbf{doch} da.
	\hfill\hbox {\citet[83]{Ormelius-Sandblom1997}}
\end{exe}
\vspace{-0.65cm}	
\begin{exe}
	\ex\label{876}
	Patrick ist nicht zu Hause. $>$ Patricks Auto ist nicht da.\\
	(Wenn Patrick nicht zu Hause ist, ist normalerweise sein Auto auch nicht da.)
	\hfill\hbox {\citet[83]{Ormelius-Sandblom1997}}
\end{exe}
Nach As Äußerung besteht folglich der Kontextzustand in (\ref{877}), wobei es sich um den Ausgangskontext der folgenden MP-Äußerung handelt.
\pagebreak
\newcolumntype{C}[1]{>{\centering}p{#1}} 
\begin{exe}
	\ex\label{877} Kontext nach As Äußerung\\
	A: Patrick ist nicht zu Hause. (= $\neg$p) $>$ Patricks Auto ist nicht da. (= $\neg$q)\\[-1em]	
 \begin{tabular}[t]{|C{6em}|C{6em}|C{6em}|}
 \hline 	
   $\textrm{DC}_{\textrm{A}}$ & {Tisch} & $\textrm{DC}_{\textrm{B}}$ \tabularnewline
  \hline
    $\neg$p & p $\vee$ $\neg$p & \tabularnewline
    {} & q $\vee$ $\neg$q & \tabularnewline
  \hline      
   \multicolumn{3}{|l|}{cg s$_{1}$} \tabularnewline   
   \hline
 \end{tabular}
\end{exe}
A hat ein Bekenntnis zu $\neg$p, weshalb sich auf dem Tisch die Frage eröffnet, ob p oder $\neg$p gilt. $\neg$p weist die \is{Implikatur} Implikatur $\neg$q auf, so dass auch die Frage, ob q oder $\neg$q zutrifft, auf dem Tisch landet. Auf diese durch die Implikatur bedingte offene Frage reagiert B mit seinem öffentlichen Bekenntnis zu q (vgl. (\ref{878})).

\begin{exe}
	\ex\label{878} Kontext nach Bs Äußerung\\
	B: Aber sein Auto ist \textbf{doch} da.(= q)\\[-1em]	
 \begin{tabular}[t]{|C{6em}|C{6em}|C{6em}|} 
 \hline 	
   $\textrm{DC}_{\textrm{A}}$ & {Tisch} & $\textrm{DC}_{\textrm{B}}$ \tabularnewline
  \hline
    $\neg$p & p $\vee$ $\neg$p & \tabularnewline
    {} & q $\vee$ $\neg$q & q\tabularnewline
  \hline      
   \multicolumn{3}{|l|}{cg s$_{2}$ = s$_{1}$} \tabularnewline   
   \hline
 \end{tabular}
\end{exe}
Eine \textit{doch}-Äußerung setzt dieser Ansicht nach einen Kontextzustand voraus, in dem der durch sie ausgedrückte Sachverhalt bereits zur Diskussion steht. Man kann auch sagen, dass eine \textit{doch}-Assertion einen instabilen Kontextzustand \is{instabiler Kontextzustand} voraussetzt. Die Proposition ist unentschieden vor der Äußerung, und weil die Assertion nur den \glq halben Beitrag\grq {} leistet, um eine der beiden Propositionen in den cg zu befördern, bleibt die Proposition auch nach der Äußerung im Kontext unentschieden.

\subsubsection{\textit{auch}}
\label{sec:auch}
Für die MP \textit{auch} sind in verschiedenen Arbeiten eine Reihe von deskriptiven Generalisierungen festgehalten worden, die ich im Folgenden darstelle, um anschließend meine Modellierung von \textit{auch} im Diskursmodell nach \citet{Farkas2010} auszuführen, in der sich diese Beschreibungen und Eindrücke natürlich wiederfinden sollen.

Über \textit{auch}-Assertionen wird gesagt, dass die MP-Äußerung einen Zusammenhang mit der Vorgängeräußerung herstellt, in dem Sinne, dass der Inhalt der \textit{auch}-Äußerung Voraussetzung für den Inhalt der vorweggehenden Äußerung ist. Sie fungiert als die Erklärung der ersten Äußerung (vgl. \citealt[47]{Dahl1988}, \citealt[160]{Thurmair1989}, \citealt[1226]{Zifonun1997}, \citealt[343]{Karagjosova2004}, \citealt[222]{Moellering2004}). Die Beispiele in (\ref{879}) und (\ref{880}) illustrieren, wie sich zwi\-schen der \textit{auch}-Äußerung und dem vorherigen Beitrag ein Begründungszusammenhang einstellt.

\begin{exe}
	\ex\label{879}
	A: Das Essen war ausgezeichnet.\\
	B: Es war \textbf{auch} die teuerste Speise, die es in diesem Hotel gibt.	
\end{exe}

\begin{exe}
	\ex\label{880}
	A: Er ist zum Direktor ernannt worden.\\
	B: Er hat \textbf{auch} die meisten Erfahrungen auf unserem Gebiet.	 
	\newline
	\hbox{}\hfill\hbox {\citet[88]{Helbig1990}}
\end{exe}
In (\ref{879}) erklärt der hohe Preis, warum das Essen sehr gut war; in (\ref{880}) dient die Information, dass er die meiste Erfahrung hat, als Erklärung für seine Ernennung zum Direktor. Es handelt sich hierbei um plausible, aber keineswegs notwendige Zusammenhänge. Denn natürlich kann ein teures Essen auch schlecht sein oder jemand, obwohl er der erfahrenste Kandidat ist, einen Posten nicht bekommen. Evidenz für die Tatsache, dass \textit{auch} in Assertionen einen Begründungszusammenhang kodiert, liefern auch Beispiele wie in (\ref{881}), in denen sich diese Relation nicht anbietet. Dass der Wein billig ist, erklärt schlecht, warum er so gut ist.

\begin{exe}
	\ex\label{881}
	A: Der Wein ist ja ausgezeichnet!\\
	B: *Ja, das war \textbf{auch} der billigste Wein im Handel.	 
	\hfill\hbox {\citet[211]{Franck1980}}
\end{exe}
Eine andere Beobachtung ist, dass durch eine \textit{auch}-Assertion die vorweggehende Äußerung (implizit) bestätigt/anerkannt wird (vgl. \citealt[212]{Franck1980}, \citealt[160]{Thurmair1989}, \citealt[88]{Helbig1990}, \citealt[221-222]{Moellering2004}, \citealt[343]{Karagjosova2004}). Es ist plausibel, dass ein Sprecher den Sachverhalt, den er begründet, auch annimmt. Evidenz für dieses Verhältnis ist, dass einer \textit{auch}-Äußerung kein \textit{Nein}. vorangehen kann (vgl. \citealt[212]{Franck1980}) (vgl. (\ref{882})).

\begin{exe}
	\ex\label{882}
	A: Der Wein ist ziemlich dürftig im Geschmack.\\
	B: *Nein, das war \textbf{auch} der billigste Wein bei Lichdi.	 
	\hfill\hbox {\citet[211]{Franck1980}}
\end{exe}
Es lassen sich auch weitere Aussagen über die bestätigte Information machen. Sie geht der \textit{auch}-Äußerung präferiert vorweg. Für mich stellt sich zwischen (\ref{883}) und (\ref{883}) für \textit{auch} ein Akzeptabilitätsunterschied ein, der für \textit{halt} und \textit{eben}, die ebenfalls einen Begründungszusammenhang herstellen können (vgl. Kapitel~\ref{chapter:hue}, Abschnitt~\ref{sec:bedhe} und \ref{sec:kontexte}), nicht gleichermaßen gilt.

\begin{exe}
	\ex\label{883}
	Susanne war \textbf{halt}/\textbf{eben}/\#\textbf{auch} müde. (Nur) DEShalb hat sie so wenig gesagt.	 
	\newline
	\hbox{}\hfill\hbox {nach \citet[89]{Autenrieth2002}}
\end{exe}

\begin{exe}
	\ex\label{884}
	A: Susanne hat sehr wenig gesagt.\\
	B: Sie war \textbf{halt}/\textbf{eben}/\textbf{auch} müde.
\end{exe}
Dies liegt nicht etwa an der Tatsache, dass \textit{auch} nicht monologisch verwendet werden kann (s.u.). 

Die bestätigte Relation kann auch nicht präsupponiert \is{Präsupposition} sein (vgl. (\ref{885}) vs. (\ref{886})).

\begin{exe}
	\ex\label{885}
	A: Warum hast du denn so schlechte Laune?\\
	B: Ich hab \textbf{halt}/\textbf{eben}/\#\textbf{auch} Hunger.
\end{exe}

\begin{exe}
	\ex\label{886}
	A: Du hast heute echt schlechte Laune.\\
	B: Ich hab \textbf{halt}/\textbf{eben}/\textbf{auch} Hunger.
\end{exe}
(\ref{887}) zeigt, dass die bestätigte Information aber aus dem Kontext abgeleitet sein kann.

\begin{exe}
	\ex\label{887}
	Arzt: Na, was fehlt dir denn?\\
	Sabine: Mein Finger blutet und mein Fuß tut weh.\\
	Arzt: Die werden \textbf{auch} wieder heilen.	 
	\hfill\hbox {\citet[118]{Bublitz1978}}
\end{exe}
Begründet wird vom Arzt ein Beitrag wie \textit{Du musst dir keine Sorgen machen.}, was aber nicht explizit vom Arzt mitgeteilt wird.

Darüber hinaus wird der beteiligte Zusammenhang als allgemein gültig angesehen (vgl. \citealt[47]{Dahl1988}). \citet[103]{Burkhardt1982} formuliert diese Beobachtung folgendermaßen: 

\begin{quotation}
Es zeigt sich nach alledem insgesamt, daß es allen exemplifizierten Varianten von \textit{auch} als Abtönungspartikel gemeinsam ist, die \textbf{Erwartbarkeit} eines Sachverhalts aus der Sicht des Sprechers aufgrund einer \textbf{allgemeinen Norm}, einer \textbf{Gesetzmäßigkeit} oder von \textbf{Erfahrungswerten} zu präsupponie\-ren.
\hfill\hbox {(Hervorhebungen S.M.)}
\end{quotation}
Die Beispiele, die man in der Literatur findet, involvieren typischerweise allgemein bekannte Zusammenhänge. Dies trifft auf die oben angeführten Fälle genauso zu wie auf die weiteren Beispiele in (\ref{888}) bis (\ref{890}).

\begin{exe}
	\ex\label{888}
	A: Peter sieht sehr schlecht aus.\\	 
	B: Er ist \textbf{auch} sehr lange krank gewesen.
	\hfill\hbox {\citet[88]{Helbig1990}}\\
	\glq Wenn man lange krank war, sieht man schlecht aus.\grq {}
\end{exe}

\begin{exe}
	\ex\label{889}
	A: Der Wein ist aber ziemlich dürftig im Geschmack.\\	 
	B: (Ja), das war \textbf{auch} der billigste Weisswein bei Lichdi.
	\newline
	\hbox{}\hfill\hbox {\citet[211]{Franck1980}}\\
	\glq Wenn der Wein billig ist, schmeckt er nicht gut.\grq {}
\end{exe}

\begin{exe}
	\ex\label{890}
	A: Das Boot sieht ja wieder fast wie neu aus!\\	 
	B: Ich hab \textbf{auch} seit Ostern jedes Wochenende daran rumgebastelt.
	\newline
	\hbox{}\hfill\hbox {\citet[211]{Franck1980}}\\
	\glq Wenn man viel am Boot bastelt, sieht es aus wie neu.\grq {}
\end{exe}
Ein weiterer Punkt ist, dass der Relevanzwert der Äußerung des ersten Sprechers vom zweiten Sprecher weniger hoch eingeschätzt wird als vom ersten Sprecher (vgl. \citealt[47-48]{Dahl1988}). Anders gefasst, wird der Vorgängeräußerung das Erstaunliche oder Fragwürdige (vgl. \citealt[211-212]{Franck1980}, \citealt[88]{Helbig1990}, \citealt[74]{Kwon2005}) bzw. ihre Informativität (\citealt[223-224]{Karagjosova2004}) genommen. Diese Einschätzung kann auch expliziert werden, wie in (\ref{891}).

\begin{exe}
	\ex\label{891}
	(nach einer gemeinsamen Autofahrt:)\\	 
	A: Wir haben heute nur zwei Stunden gebraucht bis nach Hause.\\
	B: \textbf{Das wundert mich nicht.} Du bist \textbf{auch} gefahren wie ein Weltmeister.
	\newline
	\hbox{}\hfill\hbox {nach \citet[101]{Burkhardt1982}}
\end{exe}	
Im Falle eines Sprecherwechsels ist der Inhalt der A-Äußerung für B ableitbar. In (\ref{892}) ist allgemein bekannt, dass wenn man nicht arbeitet, man wahrscheinlich die Prüfung nicht schafft. Und wenn B darüber hinaus annimmt, dass Peter sich nicht vorbereitet hat, folgt für B, dass Peter die Prüfung vermutlich nicht schafft. Es ist für ihn deshalb nicht verwunderlich oder neue Information, dass er die Prüfung nicht geschafft hat.

\begin{exe}
	\ex\label{892}
	A: Peter hat die Prüfung nicht bestanden.\\	 
	B: Er hat sich \textbf{auch} nicht vorbereitet.
\end{exe}
Im monologischen Gebrauch schätzt der Sprecher die Relevanz der ersten Proposition nicht so hoch ein, weil er sie aus der zweiten ableiten kann.

\begin{exe}
	\ex\label{893}
	Peter hat die Prüfung nicht bestanden. Er hatte sich \textbf{auch} nicht vorbereitet.
	\newline
	\hbox{}\hfill\hbox {nach \citet[227]{Karagjosova2004}}
\end{exe}
Der letzte deskriptive Punkt, der sich in anderen Arbeiten findet, ist, dass der Inhalt der \textit{auch}-Assertion (im Gegensatz zur beteiligten kausalen Relation) neue Information ist, d.h. er ist nicht bekannt oder wird nicht als bekannt vorausgesetzt (vgl. \citealt[215]{Franck1980}, \citealt[156]{Thurmair1989}, \citealt[71]{Koenig1997}, \citealt[343]{Karagjosova2004}, \citealt[73]{Kwon2005}). Wenn der Sprecher von Bekanntheit ausgeht, verwendet er die Kombination \textit{ja auch}, die \citet[156]{Thurmair1989} zufolge häufiger auftritt als nur \textit{auch}. Als Beispiel, das diesen Punkt illustriert, kann (\ref{894}) fungieren.

\begin{exe}
	\ex\label{894}
	A: Ich bin in der Stadtbücherei und ... interessiere mich  n bisschen für Malerei.\\
	B: Malen Sie selbst?\\
	A: Nein, ich male nicht selbst, ich hab \textbf{auch} überhaupt kein Geschick dazu.
	\newline
	\hbox{}\hfill\hbox {(PFE/BRD cc008), \citet[69]{Kwon2005}}
\end{exe}
Der Gesprächspartner kann hier nicht wissen, dass A kein Geschick hat. Beiden sollte jedoch bekannt sein, dass wenn man kein Geschick hat fürs Malen, man es normalerweise nicht macht. Dass A kein Geschick hat, ist seine Begründung für das Nichtmalen.

Die vier relevanten Punkte, die auch durch die diskursstrukturelle Modellierung abgedeckt sein sollten, finden sich in (\ref{895}) (p = \textit{auch}-Proposition, q = vorweggehende Proposition).

\begin{exe}
	\ex\label{895} 
		\begin{xlist}	
			\ex\label{895a} kausale Relation: p $>$ q
			\ex\label{896b} Bekanntheit von p $>$ q
			\ex\label{896c} Relevanzeinschränkung von q
			\ex\label{896d} Unbekanntheit von p			
		\end{xlist}
\end{exe}
Diese Beobachtungen, die vornehmlich aus deskriptiven Arbeiten stammen, übersetze ich auf die folgende Art in das Diskursmodell aus \citet{Farkas2010}. (\ref{896}) dient der Illustration der Modellierung.

\begin{exe}
	\ex\label{896}
	B: Der Wein ist aber ziemlich dürftig im Geschmack. (= q)\\
	A: (Ja), das war \textbf{auch} der billigste Wein bei Lichdi. (= p)
	\hfill\hbox {\citet[211]{Franck1980}}
\end{exe}
Nach Bs Äußerung, die den Ausgangskontext für die \textit{auch}-Assertion darstellt, besteht der Kontextzustand in (\ref{897}). Fett markiert sind die diskursstrukturellen Gegebenheiten, die auf \textit{auch} zurückgehen bzw. von \textit{auch} gefordert werden.

\begin{exe}
	\ex\label{897} Kontextzustand vor der \textit{auch}-Assertion\\[-1em]	
 	\begin{tabular}[t]{|C{6em}|C{6em}|C{6em}|} 
 	\hline 	
   	$\textrm{DC}_{\textrm{A}}$ & {Tisch} & \textbf{$\textrm{DC}_{\textrm{B}}$} \tabularnewline
 	 \hline
     & q $\vee$ $\neg$q & \textbf{q}\tabularnewline
  	\hline      
   	\multicolumn{3}{|l|}{\textbf{cg s$_{1}$ = $\lbrace$p $>$ q$\rbrace$}} \tabularnewline   
   \hline
 \end{tabular}
\end{exe}
B teilt mit, dass er den Wein für nicht gut im Geschmack empfindet, d.h. er bekennt sich zu q. Dadurch stellt sich die Frage, ob beide Gesprächsteilnehmer dies so sehen: Auf dem Tisch liegt q $\vee$ $\neg$q. Im cg befindet sich der Zusammenhang p $>$ q, d.h. es ist allgemein bekannt, dass billiger Wein in der Regel nicht schmeckt. Vor diesem Hintergrund erfolgt die \textit{auch}-Assertion. Zunächst tritt die Kontextsituation wie in (\ref{898a}) ein. A bekennt sich durch die Assertion zu p. Dadurch wird das Thema eröffnet, ob p gilt. Die Proposition q wird implizit bestätigt, d.h. A bekennt sich auch zu q (denn für A und B gilt p $>$ q und für A gilt auch p). Da B schon sein Diskursbekenntnis zu q gegeben hat, wird q zu einem cg-Inhalt (vgl. (\ref{898b})). Dieses Thema ist folglich entschieden. Das Thema rund um p ist hingegen noch offen, denn B könnte den Inhalt der Begründung für sich auch ablehnen.

\begin{exe}
	\ex\label{898} Kontextzustand nach der \textit{auch}-Assertion\\[-1.25em]	
	\begin{xlist}	
		\ex\label{898a} Teil 1\\[-1em]
 		\begin{tabular}[t]{|C{6em}|C{6em}|C{6em}|} 
 		\hline 	
   		$\textrm{DC}_{\textrm{A}}$ & {Tisch} & $\textrm{DC}_{\textrm{B}}$ \tabularnewline
 	 	\hline
     	q & q $\vee$ $\neg$q & q\tabularnewline
     	p & p $\vee$ $\neg$p & \tabularnewline
  		\hline      
   		\multicolumn{3}{|l|}{cg s$_{2}$ = s$_{1}$} \tabularnewline   
   		\hline
 		\end{tabular}
 		
 		\ex\label{898b} Teil 2\\[-1em]
 		\begin{tabular}[t]{|C{6em}|C{6em}|C{6em}|} 
 		\hline 	
   		$\textrm{DC}_{\textrm{A}}$ & {Tisch} & $\textrm{DC}_{\textrm{B}}$ \tabularnewline
 	 	\hline
     	p & p $\vee$ $\neg$p & \tabularnewline
  		\hline      
   		\multicolumn{3}{|l|}{cg s$_{3}$ = $\lbrace$ s$_{2}$ $\cup$ $\lbrace$q$\rbrace\rbrace$} \tabularnewline   
   		\hline
 		\end{tabular} 		
 	\end{xlist}
	\end{exe}
Meine Modellierung von \textit{auch} fängt die deskriptiven Beiträge aus der Literatur auf: Der Begründungszusammenhang zwischen der MP-Äußerung und der vorweggehenden Äußerung findet Eingang durch die Inferenzrelation p $>$ q (vgl. auch \citealt[341-342]{Karagjosova2003}; \citeyear[220-235]{Karagjosova2004} sowie meine Ausführungen zum kausalen Bedeutungsmoment bei \textit{halt} und \textit{eben} in Kapitel~\ref{chapter:hue}, Abschnitt~\ref{sec:kontexte}). Die Bekanntheit dieses Zusammenhangs ist dadurch garantiert, dass p $>$ q im cg ent\-halten ist. A kommt unter der Verankerung von p $>$ q im cg und seiner Annahme von p selbst zum Schluss von q. q ist folglich keine neue Information für ihn. Er hält q nicht für mitteilungswürdig, was der Proposition den Status von Neuigkeit und in diesem Sinne auch Erstaunen nimmt: In seinen Augen ist klar, dass wenn der Wein billig war (p), er nicht schmeckt (q). Die Relevanzeinschränkung des Inhalts der Vorgängeräußerung wird durch meine Beschreibung folglich ebenfalls aufgefangen. Und schließlich wird auch der Inhalt p der MP-Äußerung nicht als bekannt vorausgesetzt: Es ist B nicht wirklich bekannt oder wird ihm als bekannt unterstellt, dass der Wein billig war.

Wie schon bei den Modellierungen der anderen Partikeln liegt der reguläre Beitrag der Assertion vor: Der Sprecher bekennt sich zu ihrem Inhalt. Indem durch die Assertion die Proposition eingeführt wird, eröffnet sich das Thema. Zusätzlich müssen aber bestimmte Verhältnisse vorliegen, damit die \textit{auch}-Asser\-tion angemessen geäußert werden kann. Die minimale Anforderung an den Vorgangskontext einer \textit{auch}-Assertion ist meiner Meinung nach, dass p $>$ q im cg enthalten ist, und dass B (im Dialog) bzw. A selbst (im Monolog) ein Bekenntnis zu q hat. Es muss nicht die Frage auf dem Tisch liegen, ob q gilt, wie in (\ref{896}). Es kann beispielsweise auch der Fall sein, dass die Diskursteilnehmer sich schon auf q geeinigt haben, wie in (\ref{899}). Auch unter diesen Umständen hat der Gesprächspartner aber trotzdem ein Bekenntnis zu q.

\begin{exe}
	\ex\label{899}
	A: Das Boot sieht \textbf{ja} wieder fast wie neu aus! (= q)\\
	B: Ich hab \textbf{auch} seit Ostern jedes Wochenende daran rumgebastelt. (= p)
	\newline
	\hbox{}\hfill\hbox {\citet[211]{Franck1980}}
\end{exe}
Minimal muss q in Bs Diskursbekenntnissen enthalten sein bzw. -- im Falle eines Monologs - A von q ausgehen.

Ich gehe deshalb von (\ref{900}) als dem kontextuellen Vorzustand aus, der gegeben sein muss, damit die \textit{auch}-Assertion angemessen geäußert werden kann.
\begin{exe}
	\ex\label{900} Kontextzustand vor der \textit{auch}-Assertion\\[-1em]	
 	\begin{tabular}[t]{|C{6em}|C{6em}|C{6em}|} 
 	\hline 	
   	$\textrm{DC}_{\textrm{A}}$ & {Tisch} & $\textrm{DC}_{\textrm{B}}$ \tabularnewline
 	 \hline
     (q) & & (q)\tabularnewline
  	\hline      
   	\multicolumn{3}{|l|}{cg s$_{1}$ = $\lbrace$p $>$ q$\rbrace$} \tabularnewline   
   \hline
 \end{tabular}
\end{exe}
Da es sich um eine minimalistische Bedeutungsmodellierung \is{Bedeutungsminimalismus/-maximalismus} handelt, ist natürlich nicht ausgeschlossen, dass weitere Felder besetzt sind (s.o.).
\subsubsection{Der Vergleich des Gebrauchs von \textit{halt}, \textit{eben} und \textit{auch}}
Vergleicht man die Kontextanforderungen aus (\ref{900}) mit denen, die ich für \textit{halt} und \textit{eben} formuliert habe (vgl. (\ref{901}) und (\ref{902})), zeigt sich, dass diese sich nur minimal voneinander unterscheiden.
\pagebreak
\begin{exe}
	\ex\label{901} Kontextzustand vor der \textit{halt}-Assertion\\[-1em]	
 	\begin{tabular}[t]{|C{6em}|C{6em}|C{6em}|} 
 	\hline 	
   	$\textrm{DC}_{\textrm{A}}$ & {Tisch} & $\textrm{DC}_{\textrm{B}}$ \tabularnewline
 	 \hline
     p $>$ q & & \tabularnewline
     (q) & & (q) \tabularnewline
  	\hline      
   	\multicolumn{3}{|l|}{cg s$_{1}$} \tabularnewline   
   \hline
 \end{tabular}
\end{exe}

\begin{exe}
	\ex\label{902} Kontextzustand vor der \textit{eben}-Assertion\\[-1em]	
 	\begin{tabular}[t]{|C{6em}|C{6em}|C{6em}|} 
 	\hline 	
   	$\textrm{DC}_{\textrm{A}}$ & {Tisch} & $\textrm{DC}_{\textrm{B}}$ \tabularnewline
 	 \hline
     & & p \tabularnewline
     (q) & & (q) \tabularnewline
  	\hline      
   	\multicolumn{3}{|l|}{cg s$_{1}$ = $\lbrace$p $>$ q$\rbrace$} \tabularnewline   
   \hline
 \end{tabular}
\end{exe}
Die kausale Komponente -- hier erfasst in Form der Inferenzrelation \is{Inferenzrelation} p $>$ q -- liegt in allen drei Beschreibungen vor: \textit{Auch} ähnelt \textit{eben} darin, dass die kausale Relation \is{kausale Relation} als gesetzt (und deshalb als Teil des cg) angenommen wird. Die Gründe für die Verankerung im cg sind dabei ggf. verschiedene: Bei \textit{auch} handelt es sich um Normen und Erfahrungswerte, bei \textit{eben} wird die kausale Relation als kategorisch/Fakt angesetzt. Mit \textit{halt} teilt es den Umstand, dass der Inhalt der MP-Äußerung nicht als bekannt angesehen/vorausgesetzt wird. Da die Beschreibungen in (\ref{900}) bis (\ref{902}) sehr präzise Aussagen zur Verankerung der Information in den verschiedenen Diskurskomponenten machen, bietet sich die Möglichkeit, die unterschiedliche Verteilung der drei Partikeln in Kontexten zu erklären. Es finden sich Kontexte, in denen alle drei MPn akzeptabel sind, aber auch solche, in denen ihr Gebrauch unterschiedlich angemessen ist. Durch die jeweils angelegten Szenarien lässt sich ableiten, welche/r Gemeinsamkeit/Unterschied für die Verteilung verantwortlich ist.

In dem Dialog in (\ref{903}) beispielsweise können alle drei Partikeln gut auftreten.

\begin{exe}
	\ex\label{903}
	A: Peter sieht sehr schlecht aus. (= q)\\
	B: Er war \textbf{halt}/\textbf{eben}/\textbf{auch} lange krank. (= p) 
	\hfill\hbox {nach \citet[340]{Karagjosova2003}}
\end{exe}
In allen drei Fällen wird As Beitrag besätigt. Unter der Verwendung von \textit{halt} hält B den Zusammenhang zwischen p und q für plausibel, teilt p mit und leitet q allein ab. Gebraucht B \textit{eben}, wird p $>$ q als evidente Relation ausgegeben, p wird ebenfalls als bekannt angenommen, weshalb q für beide abzuleiten ist. Beide Interpretationen bieten sich an, genauso wie die Einordnung von p $>$ q als Norm sowie die Mitteilung der neuen Information p und die Ableitbarkeit von q durch B (\textit{auch}). 

Neben Kontexten, in denen die Äußerung aller drei MPn angemessen ist, weil sich Interpretationen entlang der drei Füllungen der Komponenten in (\ref{900}) bis (\ref{902}) anbieten, gilt für andere Dialoge, dass nur manche Partikeln akzeptabel sind. In den Dialogen in (\ref{904}) und (\ref{905}), die Thurmair anführt, um nachzuweisen, dass \textit{halt} und \textit{eben} nicht identisch gebraucht werden, kann \textit{auch} ebenfalls nicht gut stehen.

\begin{exe}
	\ex\label{904}
	Du kannst deine Freunde schon mitbringen. (= q)\\
	Wir haben \textbf{halt}/\#\textbf{eben}/\#\textbf{auch} kein Bier mehr. (= p)
\end{exe}

\begin{exe}
	\ex\label{905}
	Er: Ich muß noch Rosinen kaufen. Für den Obstsalat.\\
	Sie: Brauchts das denn?\\
	Er: Ja, paßt dir das nicht?\\
	Sie: Naja, in einen Obstsalat gehören \textbf{halt}/\#\textbf{eben}/\#\textbf{auch} keine Rosinen, find ich.
	\hfill\hbox {nach \citet[124]{Thurmair1989}}
\end{exe}
Für (\ref{904}) habe ich schon in Kapitel~\ref{chapter:hue}, Abschnitt~\ref{sec:untersch} angenommen, dass denkbar ist, dass allein der Sprecher davon ausgeht, dass wenn kein Bier mehr vorhanden ist, keine weiteren Gäste eingeladen werden. Da dies nur seine Sicht ist und dazu im Diskurs auch noch nicht bekannt ist, dass es kein Bier gibt, ist das Zugeständnis möglich und auch die Anfrage, die vorausgegangen sein wird, sinnvoll. (\textit{halt}) Wäre der Zusammenhang evident, d.h. beide wären sich einig, dass ohne Bier keine weiteren Leute eingeladen werden, und wäre bekannt, dass kein Bier mehr vorhanden ist, könnte das Zugeständnis schwieriger gemacht werden bzw. wäre die Anfrage schon unangemessen.  Beide Diskursteilnehmer wären sich dann einig, dass keine weiteren Leute eingeladen werden. (\#\textit{eben}) Ein Zugeständnis zu machen, ist auch schwieriger, wenn der Zusammenhang eine Norm ist und man sich gegen sie entscheidet, obwohl man annimmt, dass die Diskursteilnehmer sie teilen. (\#\textit{auch}) Die Information, dass das Bier aus ist, wurde vom Fragenden allerdings in diesem Fall noch nicht gewusst. 

Eine Rolle in der Ableitung des inakzeptablen Gebrauchs von \textit{auch} kann auch spielen, dass hier mit dem Zugeständnis ein Sachverhalt bestätigt wird, der selbst nur eingeschränkt vertreten wird und bei dem es sich deshalb auch nicht eindeutig um eine positive Aussage handelt, die eine \textit{auch}-Äußerung im vorangehenden Kontext aber benötigt (vgl. Abschnitt~\ref{sec:auch}). 

Den Dialog in (\ref{905}) habe ich bereits in Kapitel~\ref{chapter:hue}, Abschnitt~\ref{sec:untersch} so analysiert, dass sich die beiden Gesprächspartner hier sicherlich nicht einig sind, wann der Salat als gut einzustufen ist. Sie nimmt an, dass es ein guter Obstsalat ist, wenn keine Rosinen enthalten sind; er vertritt die Ansicht, dass es ein guter Salat ist, wenn sie enthalten sind. Es bietet sich folglich keine Interpretation an, unter der p $>$ q im cg enthalten ist, was die Inakzeptabilität von \textit{eben} erklärbar macht. \textit{Halt} kann hingegen problemlos gebraucht werden, da sie ihre Version des Zusammenhangs vertreten kann. Da zwei unterschiedliche Positionen vertreten werden, kann der Zusammenhang zwischen einem (guten) Obstsalat und (keinen) Rosinen auch nicht als allgemeine Norm aufgefasst werden. \textit{Auch} ist somit aus dem gleichen Grund ausgeschlossen wie \textit{eben}.

Das zu den Verteilungen aus (\ref{904}) und (\ref{905}) gespiegelte Verhältnis liegt in (\ref{906}) vor.

\begin{exe}
	\ex\label{906}
	Tim: Mensch du bist ja ganz trocken!\\
	Hans: Das ist ?\textbf{halt}/\textbf{eben}/\textbf{auch} Goretex.	
	\hfill\hbox {nach \citet[125]{Thurmair1989}}
\end{exe}
Dass Goretex Wasserschutz bietet, darf als kategorisch, evident/Fakt (\textit{eben}) und auch als allgemeine Norm, erwarteter Erfahrungswert (\textit{auch}) angesehen werden und kann hier demzufolge sehr plausibel im cg verankert sein. Die fragwürdige Verwendung von \textit{halt} ist auf die Art abzuleiten, dass die Aussage, dass nur der Sprecher vom Zusammenhang zwischen Goretex und Trockensein ausgeht, zu schwach ist.

In den bisher angeführten Dialogen haben sich \textit{eben} und \textit{auch} parallel verhalten. Dies ist u.a. darauf zurückzuführen, dass beide Partikeln (wenn auch etwas unterschiedlich motiviert) die cg-Zugehörigkeit der kausalen Relation voraussetzen. Wenn sich diese Verankerung im Dialog nicht anbietet, sind beide Partikeln ausgeschlossen. Kontexte, für die sich argumentieren lässt, dass von einem faktischen Zusammenhang, aber keiner Norm auszugehen ist (eine Norm, die kein Faktum ist, ist m.E. auszuschließen), habe ich bisher nicht finden können. Da sich \textit{eben} und \textit{auch} (neben diesem subtilen Unterschied) auch hinsichtlich des nur bei \textit{eben} beim Hörer vorausgesetzten Bekenntnisses zu p unterscheiden, gibt es Kontexte, in denen diese beiden Partikeln nicht gleichermaßen angemessen gebraucht werden können. Dies trifft für meine Begriffe auf (\ref{907}) zu.

\begin{exe}
	\ex\label{907}
	Arzt: Na, was fehlt dir denn?\\
	Sabine: Mein Finger blutet und mein Fuß tut weg.\\
	Arzt: Die werden \textbf{auch}/\textbf{halt}/\#\textbf{eben} wieder heilen.	
	\hfill\hbox {nach \citet[118]{Bublitz1978}}
\end{exe}
Die Einschätzung der Unangemessenheit des Auftretens von \textit{eben} ändert sich auch nicht, wenn die begründete Aussage expliziert wird.

\begin{exe}
	\ex\label{908}
	Du musst dir keine Sorgen machen. (= q) Die werden \textbf{auch}/\#\textbf{eben}/\textbf{halt} wieder heilen. (= p)
\end{exe}	
Die Differenzierung der Akzeptabiliät ist hier nicht auf den in den obigen Beispielen auftretenden Kontrast der Bewertung der kausalen Relation als cg- vs. Sprecherannahme zurückzuführen. Der Zusammenhang p $>$ q kann als Fakt (\textit{eben}), Norm (\textit{auch}) oder auch nur als plausibler Zusammenhang in den Augen des Sprechers (\textit{halt}) angesehen werden. Die Inakzeptabilität von \textit{eben} ist vielmehr darauf zurückzuführen, dass es in diesem Szenario unangemessen scheint, p als bekannt anzunehmen/zu unterstellen. Dies führt dazu, dass bei bekanntem Zusammenhang p $>$ q und bekanntem p q als für beide Diskursteilnehmer ableitbar dargestellt wird. Folglich teilt der Arzt dem Patienten mit, dass sie beide wissen, dass seine Sorgen (und damit möglicherweise sein ganzes Erscheinen) unnötig sind. \textit{Halt} kann angemessen geäußert werden, weil der Arzt ausdrücken kann, dass er selbst die Sorgen für unbegründet hält (p $>$ q in DC$_{\textrm{Arzt}}$ und Assertion von p). \textit{Auch} ist ebenfalls akzeptabel, weil der Zusammenhang zwischen Heilen und unnötigen Sorgen gut (und wahrscheinlich auch plausibler als nicht) als allgemeine Norm eingeschätzt werden kann. Da p assertiert wird, wird q als für den Sprecher abgeleitet bewertet. In beiden Fällen wird (anders als bei \textit{eben}) keine Einschätzung hinsichtlich der Bekanntheit von p vorgenommen. Auch hier lässt sich unter Bezug auf die verschiedenen Füllungen der Diskurskomponenten im Rahmen meiner Modellierung der ggf. abweichende Gebrauch der drei sehr ähnlichen Partikeln ableiten.

\subsection{Das kombinierte Auftreten von \textit{doch} und \textit{auch}}
Wie schon zuvor in Kapitel~\ref{chapter:jud} und \ref{chapter:hue} stellt sich auch hier als nächstes die Frage, wie eine Äußerung, in der \textit{doch} und \textit{auch} auftreten, interpretiert wird. Und es wiederholt sich die zentrale Frage, ob und, wenn ja, wie, die Skopoi \is{Skopus} der Einzelpartikeln interagieren. Angenommen, die beiden MPn nehmen jeweils Skopus über die Proposition p (= dass Peter krank ist) wie in (\ref{909}), ergeben sich für die Sequenz in (\ref{910}) die vier möglichen Skopusverhältnisse in (\ref{911}) und (\ref{912}).

\begin{exe}
	\ex\label{909} 
		\begin{xlist}	
			\ex\label{909a} Peter ist \textbf{doch} krank. doch(p)
			\ex\label{909b} Peter ist \textbf{auch} krank. auch(p)
		\end{xlist}
\end{exe}

\begin{exe}
	\ex\label{910} 
		Peter ist \textbf{doch auch} krank.
\end{exe}
	
\begin{exe}
	\ex\label{911} Verschiedener Skopus\\[-1em]
		\begin{xlist}
			\ex\label{911a} doch(auch(p))
			\ex\label{911b} auch(doch(p))
		\end{xlist}
\end{exe}

\begin{exe}
	\ex\label{912} Gleicher Skopus\\[-1em]
		\begin{xlist} 	
			\ex\label{912a} doch(p) \& auch(p) ((doch \& auch)(p))
			\ex\label{912b} auch (p) \& doch(p)	((auch \& doch)(p)) 
		\end{xlist}
\end{exe}
Wenngleich eine gängige Erklärung für die (in den Augen der Literatur) feste Abfolge ist, dass sie das asymmetrische Skopusverhältnis \is{Skopus} widerspiegelt, meine ich, dass eine Äußerung, in der \textit{doch} und \textit{auch} gereiht vorkommen, die passendste Interpretation unter gleichem Skopus erhält. Ich gehe somit von einer Interpretation entlang von (\ref{912}) aus.

\subsubsection{Additive Bedeutungskonstitution}
Dass sich für eine \textit{doch auch}-Äußerung die passendste Interpretation ergibt, wenn die beiden MPn den gleichen Skopus nehmen, zeigt die Analyse von Korpusbelegen. Ein erstes Beispiel findet sich in (\ref{913}).

\begin{exe}
	\ex\label{913} 
	\scriptsize
	B: \glqq Sie wissen dass sie mir meinen Job nicht gerade leicht machen?\grqq{}\\
	A: \glqq \textbf{Na sie müssen sich ihr Geld \underline{doch auch} verdienen Lucius!} Wenn sie mich dann entschuldigen würden, ich muss noch einige Einkäufe tätigen und den organisatorischen Kram erledigen.\grqq{}
	\newline
	\hbox{}\hfill\hbox{(DECOW14AX01)}
	\newline
	\hbox{}\hfill\hbox{(http://www.tabletopwelt.de/index.php?/topic/92424-40k-rpg-20/)}
\end{exe}
In diesem Kontext können beide Partikeln sehr plausibel als MPn verwendet sein. (\ref{914}) und (\ref{915}) wiederholen die Zustände, die meiner Modellierung nach im Kontext vor den Partikeläußerungen vorliegen müssen.

\begin{exe}
	\ex\label{914} Kontext vor einer \textit{doch}-Assertion\\[-1em]
 	\begin{tabular}[t]{|C{6em}|C{6em}|C{6em}|}
 	\hline 	
 	$\textrm{DC}_{\textrm{A}}$ & {Tisch} & $\textrm{DC}_{\textrm{B}}$ \tabularnewline
  	\hline
    & p $\vee$ $\neg$p & \tabularnewline
 	\hline      
   	\multicolumn{3}{|l|}{cg s$_{1}$} \tabularnewline   
   	\hline
 	\end{tabular}
\end{exe}

\begin{exe}
	\ex\label{915} Kontextzustand vor der \textit{auch}-Assertion\\[-1em]	
 	\begin{tabular}[t]{|C{6em}|C{6em}|C{6em}|} 
 	\hline 	
   	$\textrm{DC}_{\textrm{A}}$ & {Tisch} & $\textrm{DC}_{\textrm{B}}$ \tabularnewline
 	 \hline
     (q) & & (q)\tabularnewline
  	\hline      
   	\multicolumn{3}{|l|}{cg s$_{1}$ = $\lbrace$p $>$ q$\rbrace$} \tabularnewline   
   \hline
 \end{tabular}
\end{exe}
Wenn meine Annahme zur additiven Bedeutungskonstitution korrekt ist, muss deshalb zwischen den beiden Äußerungen die kausale Verbindung \is{kausale Relation} aufzufinden sein. Darüber hinaus muss sich mindestens einer der Diskurspartner zu dem Sachverhalt, der in Folge begründet wird, bekennen (\textit{auch}). Und die Proposition, die als Begründung fungiert, muss zur Debatte stehen (\textit{doch}). Im konkreten Fall kann und wird das genaue Zustandekommen dieser Verhältnisse durchaus variieren. Für meine Analyse sind nur die Inhalte von Bedeutung, die ich an das Partikelvorkommen von \textit{doch} und \textit{auch} und somit auch \textit{doch auch} binde.

Die beiden relevanten Propositionen sind in (\ref{913}) p = B muss sein Geld wert sein und $\neg$q = A macht Bs Job für B nicht leicht. Die relevante Relation ist: \glq Wenn B sein Geld wert sein muss, macht A Bs Job für B nicht leicht.\grq {} Dieser Zusammenhang ist in (\ref{913}) im cg enthalten. Darüber hinaus präsupponiert \is{Präsupposition} Bs Frage (faktives \textit{wissen}), dass A B den Job nicht leicht macht. Aus diesem Grund ist auch $\neg$q im cg und deshalb ist $\neg$q ebenfalls unter As und Bs Diskursbekenntnissen. Die Anforderungen, die \textit{auch} an den Kontext stellt, sind somit erfüllt (vgl. (\ref{914})). Da die Frage m.E. vorwurfsvoll oder negativ erstaunt hinsichtlich der Tatsache $\neg$q wirkt, denke ich, dass sich aus der Tatsache, dass B diese Frage stellt, ableiten lässt, dass B sich zu $\neg$p bekennt, d.h. dazu, dass er sein Geld nicht wert sein muss. Wenn er annähme, dass er sein Geld wert sein muss, wäre er von seiner harten Zeit nicht überrascht. Auf diesem Wege eröffnet sich in diesem Dialog das Thema \textit{Muss B sein Geld wert sein?}.\footnote{Dass aus dem Überraschtsein von $\neg$q ein Bekenntnis zu $\neg$p ableitbar ist, ließe sich ggf. auch als Implikatur im cg verankern.}

\begin{exe}
	\ex\label{916} Kontext vor der \textit{doch auch}-Assertion\\[-1em]	
 	\begin{tabular}[t]{|C{6em}|C{6em}|C{6em}|} 
 	\hline 	
   	$\textrm{DC}_{\textrm{A}}$ & {Tisch} & $\textrm{DC}_{\textrm{B}}$ \tabularnewline
 	 \hline
     & p $\vee$ $\neg$p & $\neg$p\tabularnewline
  	\hline      
   	\multicolumn{3}{|l|}\textbf{cg s$_{1}$ = $\lbrace$p $>$ $\neg$q, $\neg\textrm{q}\rbrace$} \tabularnewline   
   \hline
 \end{tabular}
\end{exe}
Wenn die \textit{doch auch}-Assertion gemacht wird (vgl. (\ref{917})), führt A p ein. Diese Proposition dient der Erklärung von $\neg$q. Für A folgt $\neg$q, weil p $>$ $\neg$q Teil des cg ist. Dieser Bedeutungseffekt geht auf \textit{auch} zurück. Dazu kommt, dass A auf das offene Thema p $\vee$ $\neg$p reagiert, wofür \textit{doch} verantwortlich ist.

\begin{exe}
	\ex\label{917} Kontext nach der \textit{doch auch}-Assertion\\[-1em]	
 	\begin{tabular}[t]{|C{6em}|C{6em}|C{6em}|} 
 	\hline 	
   	$\textrm{DC}_{\textrm{A}}$ & Tisch & $\textrm{DC}_{\textrm{B}}$ \tabularnewline
 	\hline
    p & p $\vee$ $\neg$p & $\neg$p\tabularnewline
  	\hline      
   	\multicolumn{3}{|l|}{cg s$_{2}$ = s$_{1}$} \tabularnewline   
   \hline
 \end{tabular}
\end{exe}
Nach der MP-Äußerung weiß B, dass A p annimmt und dass dies in As Augen die Erklärung für $\neg$q ist. Je nach dem weiteren Kontextverlauf kann B seine eigene Ansicht gegenüber $\neg$p revidieren oder beibehalten. Es ist folglich möglich, zu motivieren, warum die Anforderungen, die \textit{doch} und \textit{auch} in Isolation benötigen, in diesem Dialog, in dem eine \textit{doch auch}-Assertion verwendet wird, erfüllt sind.

Bevor ich alternative Interpretationen einer \textit{doch auch}-Assertion durchspiele, sollen zwei weitere authentische Beispiele zeigen, dass die additive Bedeutungszu\-schreibung den Effekt der MP-Äußerung adäquat auffängt. Zunächst sei zu diesem Zweck (\ref{918}) betrachtet, in dem die Verhältnisse etwas komplexer sind und indirekter zustande kommen als in dem vorherigen Beispiel.

\begin{exe}
	\ex\label{918}
	\scriptsize
	Also das was Guido Knopp macht, sind bestimmt keine 100 prozentigen wissenschaftlich/historisch korrekten Dokumentationen. \textbf{Will er 					\underline{doch auch} gar nicht.} Sowas kann man auf Arte sehen. 
	\hfill\hbox {(deWac: 1624)}
\end{exe}
Um nachzuweisen, dass eine Analyse, in der \textit{doch} und \textit{auch} sich jeweils auf die gleiche Proposition beziehen, zutreffend ist, bietet es sich an, zunächst das Ein\-zelauftreten der beiden MPn zu motivieren. Die MP-Äußerung könnte auch lauten \textit{Will er \textbf{doch} gar nicht.} oder \textit{Will er \textbf{auch} gar nicht.} Die beiden relevanten Propositionen sind in (\ref{919}) festgehalten.
	
\begin{exe}
	\ex\label{919} 
		\begin{xlist}	
			\ex\label{919a} p = Guido Knopp (GK) will keine hundertprozentigen wissenschaftlich/ historisch korrekten Dokumentationen
			\ex\label{919b} q = GK macht keine hundertprozentigen wissenschaftlich/historisch korrekten Dokumentationen
			\hfill\hbox {\citet[42]{Helbig1981}}
		\end{xlist}
\end{exe}
Die für \textit{auch} benötigte Inferenzrelation ist: \glq Wenn GK keine hundertprozentigen wissenschaftlich/historisch korrekten Dokumentationen will (= p), sind seine Dokumentationen nicht hundertprozentig wissenschaftlich/historisch korrekt (= q).\grq {} D.h. die \textit{auch}-Äußerung bestätigt q, die Aussage, die der Sprecher selber macht. Da der Zusammenhang zwischen gewollter und tatsächlicher Gestaltung als allgemeingültig angesehen werden kann, kann hier p $>$ q sehr plausibel im cg enthalten sein. Dass er entsprechende Dokumentationen nicht beabsichtigt, kann in diesem Dialog darüber hinaus auch als neue Information gewertet werden. Im weiteren Vorgangskontext wird nicht darüber gesprochen, dass GK dies will - wenngleich von einer anderen Person beigetragen wird, dass die Produktionen an sich populärwissenschaftlich sind.

Nach meiner Analyse von \textit{doch} muss auf dem Tisch liegen \textit{Will GK derartige Dokus oder will er sie nicht?} D.h. p $\vee$ $\neg$p steht zur Debatte und der Sprecher entscheidet sich mit der Assertion für p. Dieses offene Thema kann in diesem Kontext ebenfalls nachgewiesen werden, über eine Inferenzrelation \is{Inferenzrelation} und eine Standardannahme, von der auszugehen ist. Letztere ist, dass normalerweise davon ausgegangen wird, dass Dokumentationen wissenschaftlich korrekt sind (t $>$ k $[$\glq Wenn eine TV-Sendung eine Doku ist, ist sie normalerweise hundertprozentig korrekt.\grq {}$]$). Wenn eine Abweichung von diesem Standard auftritt (r) (es gelten t und $\neg$k), stellt sich die Frage, ob dies gewollt ist (p $\vee$ $\neg$p) (r $>$ (p $\vee$ $\neg$p)).

Basierend auf diesen zwei Teilanalysen interpretiere ich die Szene in (\ref{918}) des\-halb folgendermaßen: Im Normalfall sind Dokumentationen wissenschaftlich/his\-torisch korrekt (t $>$ k). GK produziert eine nicht-korrekte Dokumentation ($\neg$k trotz t). Da die Doku von der Norm abweicht (r), stellt sich ein unerwarteter Umstand ein. Man wundert sich. Auch stellt sich ein potenzieller Vorwurf oder eine potenzielle Kritik ein. Aufgrund der Normabweichung (r), die erst einmal immer unerwartet ist, stellt sich die Frage, ob GK eine korrekte Doku (nicht) wollte (p $\vee$ $\neg$p). Diese Frage beantwortet der Sprecher derart, dass GK eine korrekte Doku nicht wollte ($\neg$p). Dadurch rechtfertigt er zwar nicht diese eher unübliche Art der Dokumentation. Er nimmt aber die Spannung/potenzielle Kritik, dass dies ein Fehler ist oder GK etwa zu keiner korrekten Dokumentation fähig sei. (\ref{920}) zeigt den Kontextzustand vor der \textit{doch auch}-Assertion.

\begin{exe}
	\ex\label{920} Kontext vor der \textit{doch auch}-Assertion\\[-1em]	
 	\begin{tabular}[t]{|C{6em}|C{6em}|C{6em}|} 
 	\hline 	
   	$\textrm{DC}_{\textrm{A}}$ & Tisch & $\textrm{DC}_{\textrm{B}}$ \tabularnewline
 	\hline
 	q & q $\vee$ $\neg$q & \tabularnewline
    & p $\vee$ $\neg$p & \tabularnewline
  	\hline      
   	\multicolumn{3}{|l|}{cg s$_{1}$ = $\lbrace$p $>$ q, t $>$ k, r $>$ (p $\vee$ $\neg$p$)\rbrace$} \tabularnewline   
   \hline
 \end{tabular}
\end{exe}
Der Sprecher hat ein Bekenntnis zu q (= GK macht nicht hundertprozentig wissenschaftlich/historisch korrekte Dokus). Dieses Thema gelangt standardmäßig auch auf den Tisch. Für die MP-Verwendung relevant ist aber vielmehr, dass aufgrund der Annahme, dass Standarddokumentationen normalerweise hundertprozentig wissenschaftlich/historisch korrekt sind (t $>$ k), der Zusammenhang r $>$ (p $\vee$ $\neg$p) greift, d.h. sich die Frage stellt, ob GK dies so will. Aus diesem Grund liegt auch p $\vee$ $\neg$p auf dem Tisch. Es gibt hier zwar keinen Diskurspartner, der q explizit bestätigt, im Vorkontext wurde aber schon angemerkt, dass GKs Produktionen populärwissenschaftlich sind, d.h. diese Annahme kann im Kontext als etabliert gelten (die Fragen \textit{ob q} und \textit{ob r} werden deshalb sofort zugunsten von q und r aufgelöst). Wenn dies nicht gegeben wäre, müsste man annehmen, dass nur der Sprecher die Offenheit von p annimmt (weil nur er die Abweichung von der Norm t $>$ k sieht). Da q aber schon eingeführt ist, kann p $\vee$ $\neg$p für alle Beteiligten zur Diskussion stehen. Dieses offene Thema adressiert die \textit{doch auch}-Assertion (\textit{doch}) und löst die offene Frage nach p auf. Da im cg zudem die Inferenzrelation \is{Inferenzrelation} enthalten ist, dass GKs Absicht die Erklärung für diese Art von Dokus ist (p $>$ q), begründet der Sprecher q (dass GK derartige Dokus macht) damit, dass GK solche Dokus machen will. Für den Sprecher ist deshalb nicht verwunderlich, dass GKs Filme so beschaffen sind (vgl. (\ref{921})).

\begin{exe}
	\ex\label{921} Kontext nach der \textit{doch auch}-Assertion\\[-1em]	
 	\begin{tabular}[t]{|C{6em}|C{6em}|C{6em}|} 
 	\hline 	
   	$\textrm{DC}_{\textrm{A}}$ & Tisch & $\textrm{DC}_{\textrm{B}}$ \tabularnewline
 	\hline
    p & p $\vee$ $\neg$p & \tabularnewline
  	\hline      
   	\multicolumn{3}{|l|}{cg s$_{2}$ = $\lbrace$p $>$ q, r, t $>$ k, r $>$ (p $\vee$ $\neg$p$)\rbrace$} \tabularnewline   
   \hline
 \end{tabular}
\end{exe}
Auch in diesem komplexeren Beispiel gehe ich davon aus, dass sich \textit{auch} und \textit{doch} auf die Proposition p beziehen und sich die Interpretation aus dem additiven Zusammenschluss beider MP-Beiträge ergibt.

Ein weiteres m.E. unkompliziertes Beispiel findet sich in (\ref{922}).

\begin{exe}
	\ex\label{922} 
	\scriptsize
    \begin{tabular}[t]{ll}
	0956 BS	& $[$ähm$]$ schule hier her isst nur dann geht er zur therapie dann zum judo und um sechs\\
	{} & uhr kommt er heim und da muss er noch hausaufgaben machen\\
	{} & $^{o}$h hat er (.)$[$s so gesagt \emph{dass es ihm zu viel is}$]$\\
	0957 HM & $[$hm\_hm$]$\\
	0958 SZ & \hspace{1cm} \textbf{$[$(is \underline{doch auch}) (.) ä programm$]$}\\
	0959 NG & $[$hm$]$ \_hm\\
	0960 BS & und da hab ich gsagt	
	\hfill\hbox{(FOLK\_E\_00026\_SE\_01\_T\_01)} 					 
    \end{tabular}   
\end{exe}
Die beteiligten Propositionen finden sich in (\ref{923}).

\begin{exe}
	\ex\label{923} 
		\begin{xlist}	
			\ex\label{923a} p = es (Schule, Therapie, Judo, Hausaufgaben) ist ein Programm
			\ex\label{923b} q = es ist zu viel
		\end{xlist}
\end{exe}
Die Inferenzrelation \is{Inferenzrelation} ist: \glq Wenn etwas als Programm bezeichnet wird, ist es zu viel.\grq {}. SZs Äußerung fungiert in diesem Dialog als Begründung für seine (unausgesprochene) Zustimmung zur Einschätzung des Sohnes, dass es ihm zu viel ist: p wird assertiert, weshalb es – vor dem Hintergrund der Relation im cg – in SZs Augen klar ist, dass das Pensum zu viel ist. Diese Verhältnisse legitimieren die Verwendung von \textit{auch}. Die Offenheit von p  lässt sich dadurch motivieren, dass unter der im cg enthaltenen Relation p $>$ q q thematisiert wird. Dadurch kommt die Frage auf, ob p tatsächlich der Grund ist. Auch die Kontextanforderung, die ich \textit{doch} zuschreibe (p $\vee$ $\neg$p liegt auf dem Tisch), kann somit als erfüllt angesehen werden.

Unter Bezug auf die Gebrauchsweise der \textit{doch auch}-Assertionen im Kontext in Dialogen wie (\ref{913}), (\ref{918}) und (\ref{922}) gehe ich davon aus, dass das Skopusverhältnis in (\ref{924}) vorliegt.

\begin{exe}
	\ex\label{924} 
	doch(p) \& auch(p)
\end{exe}	
Die beiden Partikeln beziehen sich auf dieselbe Proposition und weisen demnach einen identischen Skopusbereich auf. 

Es ist \underline{eine} Komponente der Argumentation, zu zeigen, dass die additive Bedeutungskonstitution für Äußerungen im Kontext passende Interpretationen leistet. Im Folgenden möchte ich aber ebenfalls durchspielen, welche Ergebnisse resultieren, wenn man von einer Skopuslesart ausgeht. Die entstehenden Lesarten sind zum einen nicht zutreffend und zum anderen auch aus generelleren Gründen nicht sinnvoll anzunehmen.

Eine Alternative zu einer Aufaddierung der beiden Partikelbeiträge ist, dass \textit{doch} Skopus über \textit{auch} nimmt. Das Ergebnis für die Verhältnisse im Kontextzustand vor der MP-Assertion zeigt (\ref{925}).

\begin{exe}
	\ex\label{925} Kontext vor der \textit{doch auch}-Assertion (doch(auch(p)))\\[-1em]	
 	\begin{tabular}[t]{|C{7em}|C{12em}|C{7em}|} 
 	\hline 	
   	$\textrm{DC}_{\textrm{A}}$ & Tisch & $\textrm{DC}_{\textrm{B}}$ \tabularnewline
 	\hline
     & (q $\in$ $\textrm{DC}_{\textrm{A/B}}$ \& cg = $\lbrace$p $>$ q$\rbrace$) $\vee$ & \tabularnewline
     & $\neg$(q $\in$ $\textrm{DC}_{\textrm{A/B}}$ \& cg = $\lbrace$p $>$ q$\rbrace$) \tabularnewline
  	\hline      
   	\multicolumn{3}{|l|}{cg s$_{1}$} \tabularnewline   
   \hline
 \end{tabular}
\end{exe}
Ich denke nicht, dass (\ref{925}) die Situation passend auffängt, die die Ausgangslage für die MP-Äußerung ist. Unter Bezug auf (\ref{913}) ginge es in dem Dialog um die Frage, ob A oder B annehmen, dass A Bs Job für B nicht einfach macht und ob der kausale Zusammenhang besteht zwischen dem Sachverhalt, dass A Bs Job für B nicht leicht macht und dass B sein Geld wert sein muss. Diese Interpretation trifft schlichtweg nicht zu: Dass A und B sich zu $\neg$q bekennen (A macht Bs Job nicht leicht für B) steht hier sicherlich nicht zur Diskussion, diese Diskursbekenntnisse liegen eindeutig vor (Sie wissen, dass $\neg$q?). Es ist auch nicht Thema des Gesprächs, ob p $\neg$q begründet; auf diese Relation bezieht A sich einfach. Genauso wenig wird thematisiert, ob die beiden Aspekte zusammen gelten oder nicht. In (\ref{922}) begründet SZ zwar nur eine implizite Zustimmung (es ist zu viel), es steht in diesem Dialog aber dennoch nicht zur Diskussion, ob er dies vertritt. Es wird auch keine Diskussion geführt, ob die Kennzeichnung eines Tagesablaufs als \glqq Programm\grqq{} die Aktivitäten als \glqq zu viel\grqq{} einschätzt. In (\ref{918}) ist offensichtlich, dass der Sprecher vertritt, dass Guido Knopp nicht wissenschaftlich/historisch hundertprozentig korrekte Dokumentationen macht. Dies steht nicht zur Frage. Die MP-Äußerung scheint mir auch nicht auf die Situation zu reagieren, dass unklar ist, ob von dem Zusammenhang auszugehen ist, dass aus Guido Knopps gewollter Art der Dokumentation auf die Beschaffenheit dieser geschlossen werden kann. 

Die Alternative ist, dass das umgekehrte Skopusverhältnis vorliegt: auch(doch(p)). Der Kontext vor der \textit{doch auch}-Assertion ist in diesem Fall beschaffen wie in (\ref{926}).

\begin{exe}
	\ex\label{926} Kontext vor der \textit{doch auch}-Assertion (auch(doch(p)))\\[-1em]	
 	\begin{tabular}[t]{|C{6em}|C{6em}|C{6em}|} 
 	\hline 	
   	$\textrm{DC}_{\textrm{A}}$ & Tisch & $\textrm{DC}_{\textrm{B}}$ \tabularnewline
 	\hline
 	(q) & & (q) \tabularnewline
  	\hline      
   	\multicolumn{3}{|l|}{cg s$_{1}$ = $\lbrace$((p $\vee$ $\neg$p) $\in$ T) $>$ q$\rbrace$} \tabularnewline   
   \hline
 \end{tabular}
\end{exe}
Hier ist der Bedeutungsaspekt beteiligt, dass es eine Hintergrundannahme gibt, der zufolge aus einem zur Debatte stehenden Thema q normalerweise folgt. Dies scheint mir auch ohne den Bezug auf konkrete Beispiele nicht besonders sinnvoll zu sein.

Für (\ref{913}) hieße dies, dass man sich einig ist, dass wenn im Diskurs unentschieden ist, ob B sein Geld wert sein muss oder nicht, normalerweise folgt, dass A Bs Job für B nicht leicht macht. In (\ref{925}) läge die Situation vor, dass bekannt ist, dass aus der Diskussion, ob das Tagespensum ein \glqq  Programm\grqq{} ist, plausiblerweise folgt, dass es für den Sohn zu viel ist. In (\ref{918}) wäre aus der Offenheit des Themas, ob Guido Knopp derartige Produktionen will, abzuleiten, dass die Dokus derart beschaffen sind. 

Wenngleich ich diese Verhältnisse nicht für völlig abwegig halte, glaube ich nicht, dass der nach dieser Lesart vorliegende Kontextzustand die Verwendung der MP-Äußerung motiviert. p $\vee$ $\neg$p müsste dann gar nicht wirklich zur Debatte stehen, was in meinen Augen aber in allen drei Dialogen der Fall ist. Prinzipiell dürfte es, wenn dies die zutreffende Interpretation ist, keine Kontexte geben, in denen hinsichtlich q gegensätzliche Annahmen bestehen, wenn gleichzeitig p $\vee$ $\neg$p auf dem Tisch liegt. Wenn das Thema zur Debatte steht, müssen beide die Proposition vertreten, die der Sprecher der MP-Äußerung begründet. Es wäre z.B. ausgeschlossen, dass der Gesprächspartner $\neg$q vertritt, während gleichzeitig  p $\vee$ $\neg$p verhandelt wird, bzw. es müsste dann eine Einigung hinsichtlich q klar sein.

Auf (\ref{913}) und (\ref{918}) scheint mir diese Konstellation zuzutreffen. $\neg$q bzw. q ist jeweils cg und p $\vee$ $\neg$p steht zur Diskussion. In (\ref{922}) ist q aber noch unentschieden, obwohl die Diskussion um p/$\neg$p zur Verhandlung auf dem Tisch liegt. Ob das Pensum zu viel ist, entscheidet sich im Dialog nicht. In (\ref{918}) muss es nicht der Fall sein, dass der Sprecher der Vorgängeräußerung q akzeptiert und die Proposition dadurch cg wird. Nur für SZ ist in jedem Fall klar, dass q gilt.

Auch die umgekehrte Skopusrelation scheint folglich nicht stets zu einer sinn\-vollen Interpretation zu führen, auch wenn sie – je nach Szenario – mehr oder weniger problematisch erscheint. In allen drei Fällen erfasst sie meiner Meinung nach aber nicht die Situation, die die MP-Äußerung motiviert.

\subsubsection{Erklärung für die unmarkierte Abfolge}
Obwohl die Annahme, dass die Präferenz für eine Partikelabfolge auf das Skopusverhältnis zurückgeführt werden kann, plausibel erscheint, denke ich, dass es wenig von Nutzen ist, wenn die resultierende Interpretation in Dialogen, in denen derartige Äußerungen angemessen verwendet werden können, nicht zu\-treffend ist. Meine Ableitung der präferierten Reihung baut deshalb auf der additiven Interpretation auf. Die Grundidee, die ich in meiner Arbeit verfolge, ist, dass man die Abfolge der MPn in Kombinationen über ihre Interpretation motivieren kann. Ich vertrete somit eine Form von Ikonizität: Die Form entspricht der MP-Sequenz (\textit{doch} vor \textit{auch}), die Funktions-/Bedeutungsseite ist dem Diskursbeitrag der MP-Äußerung zugeordnet. Die Überlegung ist, dass die unmarkierte Abfolge dem diskursiven Ziel direkter nachkommt als die markierte Reihung. Ferner gehe ich auch davon aus, dass die markierte Anordnung (wenn es sie denn gibt) zulässig ist, weil sie unter bestimmten Umständen auftritt, in denen im Vergleich zur unmarkierten Abfolge geänderte Bedingungen vorliegen. Inwieweit anzunehmen ist, dass diese Sicht auch auf Kombinationen aus \textit{doch} und \textit{auch} zutrifft, diskutiere ich in Abschnitt~\ref{sec:distributionad}.

(\ref{927}) und (\ref{928}) wiederholen erneut die Kontexte, die meiner Modellierung nach vorliegen müssen, damit eine \textit{doch}- bzw. \textit{auch}-Assertion angemessen verwendet werden kann.

\begin{exe}
	\ex\label{927} Kontext vor einer \textit{doch}-Assertion\\[-1em]
 	\begin{tabular}[t]{|C{6em}|C{6em}|C{6em}|}
 	\hline 	
 	$\textrm{DC}_{\textrm{A}}$ & {Tisch} & $\textrm{DC}_{\textrm{B}}$ \tabularnewline
  	\hline
    & p $\vee$ $\neg$p & \tabularnewline
 	\hline      
   	\multicolumn{3}{|l|}{cg s$_{1}$} \tabularnewline   
   	\hline
 	\end{tabular}
\end{exe}

\begin{exe}
	\ex\label{928} Kontextzustand vor der \textit{auch}-Assertion\\[-1em]	
 	\begin{tabular}[t]{|C{6em}|C{6em}|C{6em}|} 
 	\hline 	
   	$\textrm{DC}_{\textrm{A}}$ & {Tisch} & $\textrm{DC}_{\textrm{B}}$ \tabularnewline
 	 \hline
     (q) & & (q) \tabularnewline
  	\hline      
   	\multicolumn{3}{|l|}{cg s$_{1}$ = $\lbrace$p $>$ q$\rbrace$} \tabularnewline   
   \hline
 \end{tabular}
\end{exe}
\textit{Doch} verweist meiner Auffassung nach auf die Offenheit der Proposition, d.h. reagiert auf das Thema, das gerade ausgehandelt wird bzw. noch zur Verhandlung steht. \textit{Auch} begründet eine andere Annahme.

Neben diesen Beiträgen der MPn nimmt meine Erklärung zudem Bezug auf die zwei Aspekte in (\ref{929}), die auch an anderer Stelle bereits in die Analyse eingebunden waren (vgl. Kapitel~\ref{chapter:jud}, Abschnitt~\ref{sec:markiert}).

\begin{exe}
	\ex\label{929} Antriebe für Konversation\\[-1em]
		\begin{xlist}	
			\ex\label{929a} Erweiterung des cg
			\ex\label{929b} Herstellung eines stabilen Kontextzustands
			\newline
			\hbox{}\hfill\hbox {\citet[87]{Farkas2010}}
		\end{xlist}
\end{exe}
Zum einen folgen Teilnehmer dem Drang, den cg anzureichern. Aus diesem Grund legen sie Elemente auf den Tisch. Zum anderen streben sie danach, einen stabilen Kontextzustand zu erreichen, d.h. einen Zustand, in dem kein Thema offen ist. Aufgrund dieser Absichten entfernen sie die Elemente so vom Tisch, dass der cg angereichert wird.

Die Idee, die ich vertreten möchte, ist, dass die Abfolge \textit{doch auch} das Diskurs\-ziel direkter abbildet als \textit{auch doch}, weil es für den Gang der Konversation oder das Ziel von Kommunikation im Sinne von (\ref{929}) direkter relevant ist, das aktuelle Thema zu adressieren (was \textit{doch} leistet), als einen Grund für einen anderen Sachverhalt anzuführen (was für \textit{auch} gilt). Für den Diskurs ist es dringlicher, zu erfahren, dass die assertierte Proposition Teil des Themas der Diskussion ist, als dass der Sprecher annimmt, dass die Proposition eine andere Proposition begründet.

Den Diskurs vorwärts zu bringen, ist dieser Argumentation nach einer qualitativen Bewertung übergeordnet. Natürlich treten sowohl die Adressierung des Themas als auch das Anführen des Grundes ein, aus Perspektive der Diskursabsichten geht die Themaadressierung der Begründung aber voran.

Meine Erklärung bietet einen Anknüpfungspunkt zu Hypothese 2 aus \citet[288]{Thurmair1989}. Sie führt sehr wenig aus, welche MP-Reihungen welche in den Hypothesen 1–5 formulierten Verhältnisse spiegeln. Für die Abfolge von \textit{doch} und \textit{auch} kommt H2 i.E. in Frage (\citeyear[288]{Thurmair1989}). H2 lautet, dass MPn, die Bezug auf die momentane Äußerung nehmen, vor MPn stehen, die eine qualitative Bewertung des Vorgängerbeitrags vornehmen. 

\textit{Doch} vereint in ihrer Modellierung die Merkmale BEKANNT$_{\textrm{H}}$ und KORREKTUR. Die ausgedrückte Proposition ist dem Hörer bekannt und die Äußerung fordert den Hörer auf, seine Ansicht zu ändern.

\textit{Auch} wird durch die Merkmale KONNEX und ERWARTET$_{\textrm{V/S}}$ charakterisiert. Die Partikel zeigt an, dass die Vorgängeräußerung aus Sprechersicht erwartet war.

Mit BEKANNT$_{\textrm{H}}$, KORREKTUR liegt der Bezug auf die aktuelle Äußerung vor, mit KONNEX und ERWARTET$_{\textrm{V/S}}$ auf die Vorgängeräußerung, die dadurch, dass sie als erwartet ausgegeben wird, qualitativ bewertet wird.

Natürlich arbeitet Thurmair mit einer anderen Modellierung der Einzelbedeutungen. Die durch Hypothese 2 abstrakter gefasste Konstellation findet sich in meiner Ableitung aber dennoch wieder: \textit{Doch} leistet aus Diskurssicht den dringlicheren Beitrag, das offene Thema anzugehen, während \textit{auch} im Vergleich eine untergeordnetere Angabe macht, dass diese Proposition für den Sprecher als Begründung für einen anderen Sachverhalt herhält. Im Gegensatz zu Thurmairs Hypothese beschreibe ich nicht allein die Verhältnisse, die sich bei dieser Partikelabfolge einstellen, sondern biete einen Erklärungsvorschlag an. Bei ihrer Untersuchung der MP-Kombinationen bleibt im Grunde bei jeder der fünf Hypothesen die Frage offen, \underline{warum} die Partikelabfolgen die beschriebenen Verhältnisse spiegeln. Warum gehen Partikeln, die sich auf die momentane Äußerung beziehen, Partikeln, die eine qualitative Bewertung der Vorgängeräußerung vor\-nehmen, voran? Dies beantworte ich im vorliegenden Fall mit der direktesten Ableitung gewünschter Diskursziele. Das oberste Ziel von Kommunikation ist der Vorstellung entlang von (\ref{929}), Themen vom Tisch zu entfernen und den cg anzureichern. Die Voraussetzung dafür ist es, die zur Diskussion stehenden Themen zu adressieren, womit einhergeht, sich auf die momentane Äußerung zu beziehen. Diskursstrukturell sekundär ist die Einschätzung kausaler Zusammenhänge, wobei es sich um die qualitative Bewertung handelt. 

Meine Erklärung dient somit auch der Stützung der von Thurmairs recht allgemein formulierten und wenig an konkreten Partikelkombinationen durchgespielten Hypothese.

Im folgenden Abschnitt, der sich mit den assertiven Randtypen der \textit{Wo}-VL- und V1-Deklarativsätze beschäftigt, werde ich mir die Interaktion von zu begründender Proposition und Offenheit der Folgeassertion, die die Begründung darstellt, in meiner Argumentation zu Nutze machen.
\setcounter{equation}{0}
\section{\textit{Wo}-Verbletzt- und Verberst-Deklarativsätze}
\label{sec:Rand}
Die hier betrachteten \textit{Wo}-VL- und V1-Sätze sind Deklarativsätze, in denen \textit{doch auch} auftreten kann (vgl. (\ref{930}), (\ref{931})). Welche interpretatorischen Besonderheiten diese Satztypen aufweisen, wird in den Folgeabschnitten noch detailliert ausgeführt. Für den Moment soll der Hinweis genügen, dass sie kausal (bzw. konzessiv) gelesen werden.

\begin{exe}
	\ex\label{930}
	\scriptsize
	Henry befand sich indes in einem tiefen Schlaf in er von abstruden Dingen träumte.\\
	Er träumte unter anderem davon, wie er auf einem Tisch lag und eine Decke anstarrte.\\
	\emph{Seltsam!} \textbf{\textit{Wo} er sich \underline{doch auch} auf einem solchen befand.}\\
	Ihm gefiel dieser Traum nicht. Er wollte etwas anderes träumen. 				         
	\hfill\hbox{(DECOW14AX)}
	\newline
	\hbox{}\hfill\hbox{(http://www.kurzgeschichten.de/vb/archive/index.php?t-4329.html)}
\end{exe}

\begin{exe}
	\ex\label{931}
	\scriptsize
	Die Direktorin der Zentralmusikschule Eisenstadt, Renate Bedenik, \emph{war sichtlich stolz} über das gelungene Konzert ihrer Musikschüler. 				\textbf{\textit{Legten} \underline{doch auch} zwei davon}, Hans Peter Gradwohl am Klavier und Martin Gruber am Schlagwerk, \textbf{dabei ihre 				öffentliche Abschlussprüfung ab.}				         
	\newline
	\hbox{}\hfill\hbox{(BVZ09/NOV.01705 Burgenländische Volkszeitung, 25.11.2009)}
\end{exe}										      
In (\ref{930}) und (\ref{931}) wären auch \textit{denn}-Sätze denkbar. Um Verwechslungen mit anderen Strukturen zu vermeiden, ist es wichtig, dass die \textit{Wo}-Sätze nicht lokal interpretiert werden (was ggf. die erste Assoziation mit diesem Einleiter ist) und dass die V1-Sätze nicht vorfeldelliptisch sind, d.h. aus Sicht ihrer Argumentstruktur sind sie vollständig.

Die Frage, die diese Strukturen im Rahmen meiner Untersuchung zu einem re\-levanten Betrachtungsgegenstand macht, ist, ob sich die von mir vorgeschlagene Erklärung für die Präferenz der Abfolge \textit{doch auch} aufrechterhalten lässt, wenn man Beobachtungen, die für derartige Randtypen unabhängig gemacht worden sind, hinzunimmt. Sie weisen nämlich einige Besonderheiten auf.
																
\subsection{Obligatorizität, Konzessivität/Kausalität und Ausbleichung}
\label{sec:eig}
Beispielsweise heißt es, \textit{doch} sei in diesen Sätzen obligatorisch bzw. in V1-Sätzen obligatorisch (z.B. \citealt[36]{Winkler1992}, \citealt[1020]{Altmann1993}, \citealt[155/157]{Oennerfors1997}, \citealt[2299]{Zifonun1997}, \citealt[157]{Pittner2011}, \citealt[40/42]{Oppenrieder2013}, \citealt[640]{Thurmair2013}). 

Wie eingangs erwähnt, werden Sätze der Art in (\ref{930}) und (\ref{931}) kausal bzw. konzessiv interpretiert, wobei sich diese Interpretationen auf verschiedenen Ebenen einstellen.

Für beide Satztypen gilt, dass die kausale Interpretation auf epistemischer \is{epistemischer Kausalsatz}oder illokutionärer Ebene \is{illokutionärer Kausalsatz} wirkt (modale Lesart) \is{modaler Kausalsatz} und es sich nicht um Sachverhaltsbegründungen handelt (vgl. auch Kapitel~\ref{chapter:jud}, Abschnitt~\ref{sec:markiert}). Es werden Annahmen, Sprechakte oder Einstellungen begründet, wie z.B. in (\ref{930}), warum es seltsam ist, oder in (\ref{931}), wie der Eindruck beim Sprecher entsteht, dass Renate Bedenik stolz ist. Für die \textit{Wo}-Sätze ist angenommen worden, dass sie auch eine konzessive Komponente haben können (vgl. \citealt[2312-2313]{Zifonun1997}, \citealt{Pasch1999}, \citealt{Guenthner2002}). Was die Konzessivität betrifft, haben allerdings nicht alle Autoren erkannt, was \citealt{Pasch1999} am klarsten beschreibt: Die konzessive Lesart spielt sich nicht auf der modalen, sondern auf der \is{propositionaler Konzessivsatz} propositionalen \is{modaler Konzessivsatz} Ebene ab und betrifft in diesem Sinne eine tiefere Interpretationsschicht (vor allem contra \citealt[2312-2313]{Zifonun1997}, \citealt{Guenthner2002}).

\begin{exe}
	\ex\label{932}
	\scriptsize
	\emph{Es ist schon komisch.} Über Wochen kann der Bär nicht mit einem Betäubungsgewehr überrascht werden. \textbf{\textit{Wo} er \underline{doch} 			angeblich kaum scheu ist und der geneigte Wanderer von ihm als Appetithappen angesehen wird. 		}		         
	\hfill\hbox{(BRZ06/JUL.00738 Braunschweiger Zeitung, 03.07.2006)}
\end{exe}	       											   
Die in (\ref{932}) beteiligte konzessive Relation ist beispielsweise: \glq Obwohl der Bär kaum scheu ist, kann er nicht mit dem Gewehr überrascht werden.\grq {}, die kausale: \glq Weil der Bär kaum scheu ist, wundere ich mich.\grq {}. Die konzessive Lesart ergibt sich sehr klar nicht auf modaler Ebene. Unter Konzessivität auf modaler Ebene fallen Beispiele wie in (\ref{933}).

\begin{exe}
	\ex\label{933}
	Ich will diesen Rock kaufen. Obwohl: Er hat ein Loch.	     
	\newline    
	\hbox{}\hfill\hbox{\citet[436]{Antomo2013}}
\end{exe}
Es liegt eine Infragestellung, Einschränkung oder Zurücknahme der vorherigen Äußerung vor. Ggf. will der Sprecher den Rock auch gar nicht mehr kaufen. In (\ref{934}) hingegen (propositional konzessiv) \is{propositionaler Konzessivsatz} besteht der Wille klar entgegen der Erwartung, dass man Röcke, die Löcher haben, normalerweise nicht kauft (vgl. \citealt[436]{Antomo2013}).

\begin{exe}
	\ex\label{934}
	Ich will diesen Rock kaufen, obwohl er ein Loch hat.	   
	\newline      
	\hbox{}\hfill\hbox{\citet[436]{Antomo2013}}
\end{exe}
In (\ref{932}) wird der Sachverhalt, dass der Bär nicht mit einem Gewehr überrascht werden kann, eindeutig nicht in Frage gestellt, etwa wie in (\ref{935}).

\begin{exe}
	\ex\label{935}
	\scriptsize
	Es ist schon komisch. Über Wochen kann der Bär nicht mit einem Betäubungsgewehr überrascht werden. \#Obwohl: Angeblich ist er kaum scheu $[$...$]$.	         
\end{exe}
Das Pendant zu (\ref{933}) ist in diesem Fall sogar unsinnig, weil der Sachverhalt (Der Bär kann nicht mit einem Gewehr überrascht werden.) vom Sprecher als präsupponiert \is{Präsupposition} ausgegeben wird. Er wundert sich schließlich über ihn.

Für alle hier betrachteten \textit{Wo}-Sätze gilt, dass sie auf modaler Ebene kausal sind. Ob sie zusätzlich ein propositional-konzessives Bedeutungsmoment aufweisen, ist meiner Meinung nach allein von der Art der begründeten Einstellung abhängig. Konzessivität ist beteiligt, wenn z.B. Haltungen wie \textit{erstaunt sein}, \textit{wundern}, \textit{komisch finden} auftreten. Die Einstellungen werden begründet, und sie ergeben sich aber entscheidenderweise aus einem unerwarteten Zusammenstoß von Ereignissen. (\ref{936}) und (\ref{937}) zeigen zwei Beispiele für einen \textit{Wo}-Satz, bei dem keine konzessive Relation im Spiel ist. Begründet wird allerdings die vom Sprecher ausgedrückte Annahme der Möglichkeit, dem Gefallen nachzukommen und die saliente Sache einzurichten, bzw. die aus der rhetorischen Frage abzuleitende Annahme, dass außerhalb von Schlesien niemand Karpfen mit brauner Soße aß.	

\begin{exe}
	\ex\label{936}
	\scriptsize
	\emph{Vielleicht} tut man dir trotzdem den Gefallen und ermöglicht es. \textbf{\textit{Wo} du \underline{doch} so nett darum bittest.}       
	\newline  
	\hbox{}\hfill\hbox{(DECOW2014) (http://www.crystals-dsa-foren.de/archive/index.php/thread-2885.html)}
\end{exe}
	
\begin{exe}
	\ex\label{937}
	\scriptsize
	\emph{Wer} goutierte bis dahin diesseits von Schlesien \emph{schon} Karpfen mit brauner Soße? \textbf{\textit{Wo} es selbst Karpfen blau bei Umfragen 		gerade auf zwei Prozent Zustimmung bringt.	     }
	\newline  
	\hbox{}\hfill\hbox{(HA09/DEZ.02971 Hannoversche Allgemeine, 19.12.2009)}
\end{exe}	
Bei \citet[161]{Oennerfors1997} heißt es, der V1-Satz könne nur kausal gelesen werden, Konzessivität sei nie beteiligt. Als Evidenz führt er den Kontrast in (\ref{938}) und (\ref{939}) an.

\begin{exe}
	\ex\label{938}
	A: Max ist jetzt endgültig ans Bett gefesselt.\\
	B: \textbf{\textit{Wo}} er \textbf{doch} immer so gesund war.	
	\hfill\hbox{\citet[203]{Oppenrieder1989}}
\end{exe}	

\begin{exe}
	\ex\label{939}
	A: Max ist jetzt endgültig ans Bett gefesselt.\\
	B: *\textbf{\textit{War}} er \textbf{doch} immer so gesund.	
	\hfill\hbox{\citet[161]{Oennerfors1997}}
\end{exe}
Önnerfors Annahme zur Interpretation dieses Typs von V1-Satz ist m.E. nicht korrekt. Ich teile zwar sein Urteil in (\ref{939}), halte aber andere Gründe für verantwortlich. Auch meine ich, dass sich der generelle Eindruck, dass V1-Sätze nicht konzessiv gebraucht werden können, begründen lässt.

Wie (\ref{940}) und (\ref{941}) nachweisen, findet man in den Korpusbelegen durchaus V1-Sätze, die neben der modal-kausalen Lesart auch die propositional-konzessive aufweisen.

\begin{exe}
	\ex\label{940}
	\scriptsize
	Die Brand- und Verletzungsgefahr des Elta-Geräts \emph{erstaunte die Berliner Tester sehr}. \textbf{\textit{Trägt} der Elta \underline{doch} wie alle 		anderen Haartrockner das CE-Zeichen und dazu das GS-Zeichen (Geprüfte Sicherheit).   	}
	\hfill\hbox{(HAZ09/OKT.02744 Hannoversche Allgemeine, 19.10.2009)}
\end{exe}
In (\ref{940}) ist der kausale Zusammenhang: \glq Weil der Fön das Zeichen trägt, sind die Tester erstaunt, dass Brand- und Verletzungsgefahr des Gerätes besteht.\grq {} und die konzessive Relation: \glq Obwohl der Fön das Zeichen trägt, besteht Brand- und Verletzungsgefahr.\grq {}

\begin{exe}
	\ex\label{941}
	\scriptsize
	Ich bin klar enttäuscht über das Resultat der FDP. Das schlechte Abschneiden \emph{ist sehr überraschend}. \textbf{\textit{Führten} die Freisinnigen 		\underline{doch} einen super Wahlkampf} – ganz im Gegensatz zu den anderen Parteien. 
	\newline  
	\hbox{}\hfill\hbox{(A08/SEP.09380 St. Galler Tagblatt, 29.09.2008)}
\end{exe}
In (\ref{941}) sind die analogen Relationen: \glq Weil die Freisinnigen einen super Wahlkampf führten, ist das schlechte Abschneiden für den Sprecher sehr überraschend.\grq {} und \glq Obwohl die Freisinnigen einen super Wahlkampf führten, haben sie schlecht abgeschnitten.\grq {} Die konzessive Interpretation derartiger V1-Sätze ist ebenfalls möglich.

Allerdings gibt es durchaus Verwendungsunterschiede, die die Inadäquatheit von Bs Reaktion in (\ref{939}) ableiten: V1-Sätze treten sehr wenig im mündlichen Sprachgebrauch oder dialogisch auf (wenngleich auch dies nicht ausgeschlossen ist $[$vgl. (\ref{942}) bis (\ref{945})$]$), sondern sind dem Schriftmedium zugeordnet (vgl. auch \citealt[157]{Oennerfors1997}).
	
\begin{exe}
	\ex\label{942}
	\scriptsize
	Auf den Vorwurf, viele Schulabgänger seien heute nicht mehr ausbildungsfähig, konterte Frans Thön\-nes kompetent mit der These, dass die Unternehmer den 		Kontakt zu den Schulen suchen sollten. \glqq Holen Sie sich nicht nur Schüler, sondern auch die Lehrer als Praktikanten in die Betriebe. Bringen Sie 		ihnen Wirtschaft bei! Bei den rückläufigen Geburten können wir es uns nicht mehr leisten, dass nur ein einziger Schulabgänger auf der Strecke bleibt. 		\textbf{\textit{Sind} es \underline{doch} die Neugeborenen von heute, die morgen unser Sozialsystem bezahlen müssen.}\grqq{}, so der Staatssekretär 		$[$...$]$.                                                                                           
	\newline  
	\hbox{}\hfill\hbox{(HMP09/MAI.00413 Hamburger Morgenpost, 06.05.2009)}
\end{exe}	
	
\begin{exe}
	\ex\label{943}
	\scriptsize
	Danke für den link Thomas. \textbf{\textit{Räumt} er \underline{doch} mit der hier geäußerten Meinung auf, der Dollar wäre unterbewertet.                                                                                           	}
	\hfill\hbox{(DECOW2014)}
	\newline  
	\hbox{}\hfill\hbox{(http://www.computerbase.de/forum/archive/index.php/t-394561-p-2.html)}
\end{exe}		
	 
\begin{exe}
	\ex\label{944}
	\scriptsize
	Jeden Monat lassen wir eine prominente Person zu Wort kommen, diesen Monat Art Furrer (76), Bergführer, Skilehrer und Hotelier auf der Riederalp.\\
	\newline
	\noindent
	Wie häufig trifft man Sie am Postschalter?\\
	\newline
	\noindent
	Recht oft. \textbf{\textit{Liegt} \underline{doch} die Post in der Bergstation der Grosskabinenbahn}, die zu uns auf die Alp führt. Hat man etwas 			vergessen, helfen die Pöstler immer.			      
	\hfill\hbox{(DECOW2014)}
	\newline  
	\hbox{}\hfill\hbox{(http://personalzeitung.post.ch/de/leute/promis-ueber-die-post/art-furrer-201201)}
\end{exe}

\begin{exe}
	\ex\label{945}
	\scriptsize
	Zitat von: Hans Bergman\\
	18.06.2011 16:49 \#74622\\
	\newline
	\noindent
	Warum so bald? ich würde Ihnen etwas mehr Zeit für die Weiterentwicklung geben.\\
	\newline
	\noindent
	Nee, 500 Jahre reicht. \textbf{\textit{Kommen} diese Herrschaften \underline{doch} alle aus einer Wel}t, in der das Rad und auch der Computer bereits 		erfunden sind. Dementsprechend müssen bestimmnte Dinge nicht erst noch erarbeitet werden.  			      
	\hfill\hbox{(DECOW2014)}
	\newline  
	\hbox{}\hfill\hbox{(http://181209.homepagemodules.de/t29f2-Michael-Schnarch-31.html)}
\end{exe}																	           
Gegen (\ref{942}) könnte man noch einwenden, dass unklar ist, inwieweit es sich hier um eine geplante Rede handelt, die damit konzeptionell schriftlicher Sprache nahekommt, trotz ihrer medialen Mündlichkeit. Die Belege in (\ref{943}) bis (\ref{945}) sind hingegen als konzeptionell mündlich einzustufen.

Unter Bezug auf denselben Aspekt ist auch der Eindruck zu erklären, dass im Falle der V1-Sätze konzessive Interpretationen generell ausbleiben. Dieser Satztyp ist fast ausschließlich schriftsprachlich zu finden und damit auch viel in Zeitungstexten. Die DeReKo-Verteilungen zeigen in Abschnitt~\ref{sec:korp} auch, dass die \textit{Wo}-Sätze weit weniger auftreten als die V1-Sätze, während sich dieses Verhältnis in den DECOW-Daten deutlich annähert. In den DeReKo-Daten treten in meinen Augen schlicht weniger die Einstellungen auf, deren Begründung den konzessiven Aspekt einbringt, sondern eher Annahmen oder rhetorische Fragen, bei denen die Konzessivität keine Rolle spielt (s.o.). Im prinzipiellen Potenzial unterscheiden sich die \textit{Wo}-VL- und V1-Sätze hier aber nicht. Für ihre Interpretation gilt folglich, dass sie auf modaler Ebene kausal wirken (Begründung einer Einstellung) und ggf. (je nach Einstellung) auf propositionaler Ebene konzessiv.

Der Aspekt, der mich im Rahmen meiner Untersuchung zur Abfolge von \textit{doch} und \textit{auch} vor allem interessiert, ist, dass angenommen wurde, dass \textit{doch} in den obigen V1-Sätzen nicht transparent verwendet wird. Genauer vertritt \citet[167]{Oennerfors1997}, dass der Partikel das Element des Widerspruchs fehlt. Er beruft sich auf die \textit{doch}-Zuschreibung von \citet{Ormelius-Sandblom1997} (vgl. (\ref{946})), nach der die ausgedrückte Proposition ein	Fakt ist und sich als eine konventionelle Implikatur \is{konventionelle Implikatur} gegen (eventuelle) Einwände wendet.

\begin{exe}
	\ex\label{946} 
		$\lambda \textrm{p[FAKTp}]$\\
		\textsc{Implikatur}$[\exists \textrm{q[q} \rightarrow \neg \textrm{p}]]$
			\hfill\hbox {\citet[83]{Ormelius-Sandblom1997}}
\end{exe}
\citet[167]{Oennerfors1997} vertritt, die Implikatur liege in den V1-Sätzen nicht vor. Der Sprecher wende sich nicht gegen eine andere Proposition, die ggf. implizit ableitbar ist. Seine Lösung des Problems ist, zu sagen, dass die Implikatur im V1-Satz streichbar ist, weshalb er sie auch für eine \underline{konversationelle} Implikatur \is{konversationelle Implikatur} hält.

Ich denke, dass diese Erklärung aus dem Grund wenig attraktiv ist, da \textit{doch} auch durch \glq härtere\grq {} Bedeutungsaspekte lizensiert werden kann, wie konventionelle Implikaturen, Implikationen \is{Implikation} oder \is{Sprechaktbedingung} Sprechaktbedingungen (vgl. meine Ausführungen in Abschnitt~\ref{sec:doch} dieses Teils sowie in Kapitel~\ref{chapter:jud}, Abschnitt~\ref{sec:doch1}), die man i.d.R. nicht streichen kann. Darüber hinaus werden Implikaturen normalerweise auch eher durch kontextuelle Informationen in Dialogen gestrichen und nicht aufgrund grammatischer Gegebenheiten. Wenngleich ich die Ableitung von Önnerfors nicht teile, lässt sich seine Beobachtung, dass es schwierig scheint, in diesen Sätzen $\neg$p zu motivieren, aber dennoch zunächst einmal annehmen. 

Man sieht in diesem Kontext bereits, dass die Beantwortung dieser Fragen immer auch stark von der jeweils zugrundegelegten Bedeutung von \textit{doch} abhängt. In diesem Sinne ist dieser Satzkontext deshalb auch generellerer Testboden für die Geeignetheit einer \textit{doch}-Modellierung. Es werden Überlegungen dazu angestellt, warum \textit{doch} so wichtig für diese Satztypen ist. Mit der Klärung dieser Frage hängt zusammen, dass man besondere Eigenschaften dieser Sätze ausgemacht hat, die man mit \textit{doch} in Zusammenhang gebracht hat. Aus diesem Grund habe auch ich mich mit Eigenschaften dieser Sätze beschäftigt. Je nach \textit{doch}-Auffassung kommt man für diese Eigenschaften (nicht) auf bzw. ist mit ihnen ggf. auch gar nicht einverstanden. Ich verfolge darüber hinaus auch das Ziel, die \textit{Wo}-VL- und V1-Sätze in diesen Fragen parallel zu behandeln. In der Literatur werden sie zwar zusammen erwähnt (vgl. z.B. \citealt[161]{Oennerfors1997}), detaillier\-tere Untersuchungen beschäftigen sich aber immer nur mit einem von beiden. Ich halte es für lohnenswert, beide Sätze parallel zu betrachten, was nicht heißen soll, dass sie sich nicht in manchen Aspekten voneinander unterscheiden.

Die folgenden Abschnitte beleuchten einige der schon angeführten sowie unerwähnte Eigenschaften, mit denen die beiden Satztypen in Verbindung gebracht worden sind.
	
\subsection{Obligatorizität/Typizität von \textit{doch}}
\label{sec:korp}
Der erste Aspekt, den ich genauer untersucht habe, ist, wie deutlich \textit{doch} in \textit{Wo}-VL- und V1-Sätzen tatsächlich vertreten ist.\footnote{Wenn ich im Folgenden Verteilungen angebe, gilt die Einschränkung, dass ich nur Fälle in die Betrachtung einbezogen habe, in denen die Sätze auch orthografisch selbständige Sätze darstellen, wie in den bisher angeführten Beispielen. Die relevanten Strukturen kommen auch in Nebensatzform vor. Ich glaube nicht, dass für diese Vorkommensweisen anderes gilt, Aufschluss würde hier aber nur ihr Einbezug geben. Die Beschränkung auf die auch orthografisch als solche erkennbaren selbständigen Sätze hat allein den praktischen Grund, die Menge der zu betrachtenden Daten zu reduzieren.} (\ref{947}) zeigt, ob/welche MPn in \textit{Wo}-VL-Sätzen in DeReKo vorzufinden sind.

\begin{exe}
	\ex\label{947} \textit{Wo}-VL-Sätze (nachgestellt) in DeReKo (Tagged C) (exhaustiv)\footnote{Jede Suche ist ggf. auch durch ihre Suchanfrage beschränkt. In diesem Fall sind \textit{Wo}-VL-Sätze ausgeschlossen, die von einem Fragezeichen beendet werden (Anfrage: Wo /s0 MORPH(V IND -INF -PCP) /w0 $<$se$>$ \%$+$w0 \textbackslash?). Die DECOW-Daten zeigen, dass es solche Sätze gibt. Ich habe sie allein aus praktischen Gründen ausgeschlossen, um keine w-Interrogativsätze unter den Ergebnissen zu haben.}\\[-1em]
    \begin{tabular}[t]{|l|l|l|l|}
    \hline
    \textit{doch} & \textit{schon} & \textit{doch sowieso} & keine Partikel\\
    \hline
    129 & 1 & 2 & 7\\
    \hline	 
    \end{tabular}   
\end{exe}
(\ref{948}) bis (\ref{950}) zeigen die Ergebnisse der parallelen Betrachtung von funktional ähnlichen kausalen Nebensätzen.

\begin{exe}
	\ex\label{948} \textit{Denn}-Sätze in DeReKo (Tagged C) (Zufallsstichprobe 300 bereinigt)\footnote{reduziert aus 284702 Treffern}\\[-1em]
    \begin{tabular}[t]{|l|l|l|l|l|l|l|l|}
    \hline
    keine MP & MP & & & & & &  \\
    \hline
    287 & 12 & & & & & & \\
    \hline
    & \textit{ja} & \textit{doch} & \textit{auch} & \textit{wohl} & \textit{ja} & \textit{eben} & \textit{einfach}\\
    \hline
    & 4 & 2 & 1 & 1 & 1 & 1 & 2\\
    \hline	 
    \end{tabular}   
\end{exe}

\begin{exe}
	\ex\label{949} \textit{da}-Sätze (nachgestellt) in DeReKo (Tagged C)\\
	(Zufallsstichprobe 500 bereinigt)\footnote{reduziert aus 260671 Treffern}\\[-1em]
    \begin{tabular}[t]{|l|l|l|l|l|}
    \hline
    keine MP & MP & & & \\
    \hline
    300 & 9 & & & \\
    \hline
    & \textit{ja} & \textit{doch} & \textit{eben} & \textit{sowieso}\\
    \hline
    & 6 & 1 & 1 & 1\\
    \hline	 
    \end{tabular}   
\end{exe}

\begin{exe}
	\ex\label{950} \textit{Zumal}-Sätze in DeReKo (Tagged C) (exhaustiv) \\[-1em]
    \begin{tabular}[t]{|l|l|l|l|l|l|l|}
    \hline
    keine MP & MP & & & & & \\
    \hline
    & \textit{ja} & \textit{doch} & \textit{eben} & \textit{wohl} & \textit{ja auch} & \textit{auch}\\
    \hline
    & 3 & 1 & 2 & 2 & 1 & 22\\
    \hline	 
    \end{tabular}   
\end{exe}
Funktionale Ähnlichkeit meint hier, dass für diese Konnektoren angenommen wurde, dass sie modale Interpretationen \is{modaler Kausalsatz} aufweisen. \textit{Denn} kann (so \citealt[320]{Volodina2010}) nur modal gelesen werden bzw. favorisiert diese Interpretationsweise (\citealt[270]{Bluehdorn2006}; \citeyear[29]{Bluehdorn2008}). Ebenso wird das (nachgestellte) \textit{da} mit dieser Lesart assoziiert (vgl. z.B. \citealt[335]{Pasch1983}, \citealt[182]{Rosengren1987}, \citealt[2303]{Zifonun1997}, \citealt[397]{Pasch2003}; in \citealt[411, 415]{Frey2012} zählen \textit{da}-Sätze zu den \textit{peripheren Nebensätzen}, \is{peripherer Nebensatz} mit denen $[$wenn auch eher unausgesprochen$]$ die modalen Lesarten in Verbindung gebracht werden). Zu \textit{zumal}-Sätzen finden sich in der Literatur sehr wenige Äußerungen. In \citet[6]{Bluehdorn2014} zählen sie zu den \textit{peripheren Nebensätzen}. \citet[397]{Pasch2003} ordnen \textit{zumal} als nicht-propositionalen Konnektor ein, der folglich oberhalb der Sachverhalts\-ebene verknüpft. In \citet[81]{Heidolph1981} wird \textit{zumal}-Sätzen die gleiche Funktion zugeschrieben wie \textit{wo}+\textit{doch}-Sätzen (dagegen vgl. \citealt[78-79]{Borst1985}).

Im Vergleich zu (\ref{947}) sind die Verteilungen in (\ref{948}) bis (\ref{950}) deutlich verschieden. In \textit{denn}-,\textit{ da}- und \textit{zumal}-Sätzen hat man es hinsichtlich der Verteilung \textit{MP} vs. \textit{keine MP} quasi mit genau gespiegelten Verhältnissen zu tun. Partikeln scheinen generell eher wenig aufzutreten und diese Verteilungen spiegeln vermutlich genau die Verhältnisse, mit denen MPn überhaupt in Assertionen vorkommen. Vor diesem Hintergrund müssen die Verhältnisse im \textit{Wo}-Satz erst recht als besonders gelten.

Aufgrund anderer Abfragemöglichkeiten habe ich in den Webdaten in DECOW2014 verglichen, welche MPn auftreten (über exhaustive Suchen nach den Sätzen mit den Partikeln). Angaben für \textit{Wo}-Sätze ohne MPn liegen deshalb nicht vor.

\begin{exe}
	\ex\label{951} \textit{Wo}+VL-Sätze (nachgestellt) in DECOW14AX\footnote{ Anfrage: $[$word= \glqq Wo\grqq{} $][]\lbrace$0,4$\rbrace[$word= \glqq ja\grqq{}$][ ]\lbrace$0,4$\rbrace[$pos= \glqq VVFIN\grqq{}$]$} \\
	\scriptsize
    \begin{tabular}[t]{|l|l|l|l|l|l|l|l|l|l|l|}
    \hline
    \textit{doch} & \textit{halt} & \textit{eben} & \textit{auch} & \textit{doch einfach} & \textit{doch eh} & \textit{doch sowieso} & \textit{doch auch} & 	\textit{ja} & \textit{ja auch} & \textit{ja doch} \\
	\hline
    605 & - & 1 & 5 & 1 & 6 & 2 & 20 & 17 & 3 & 1\\
    \hline	 
    \end{tabular}   
\end{exe}
Die Webdaten geben hier folglich kein anderes Bild ab als die Daten aus DeReKo. Es ist davon auszugehen, dass \textit{doch} sowieso häufiger vorkommt als die Kombinationen aus (\ref{951}). Auf deren Unterrepräsentiertheit sollte man folglich nicht schließen. Auch vermute ich, dass \textit{ja} häufiger verwendet wird als \textit{doch}, was den Kontrast noch verstärkt. Genauso ist davon auszugehen, dass MP-lose Assertionen normalerweise überwiegen (was auch (\ref{948}) bis (\ref{950}) nahelegen). Für \textit{Wo}-VL-Sätze bestätigt sich in der Verwendung folglich, dass \textit{doch} zwar nicht obligatorisch ist, aber sehr typisch, z.T. in Kombination, auftritt.

Ein Vergleich der Trefferzahl einer exhaustiven Suche nach \textit{doch} und \textit{ja} in V1-Sätzen in DeReKo ergibt ein unmissverständliches Übergewicht von \textit{doch}.

\begin{exe}
	\ex\label{951a} V1-Sätze in DeReKo (Tagged C) (exhaustiv)\footnote{ Anfrage: (MORPH(V IND -INF -PCP) /w0 $<$sa$>$) /s0 doch}\\[-1em]
    \begin{tabular}[t]{|l|l|}
    \hline
    \textit{doch} & \textit{ja}\\
    \hline	 
    3685 (57 \textit{doch auch}) & 22 (6 x \textit{ja auch}, 2 x \textit{ja eh}, 1 x \textit{ja sowieso})\\    
    \hline
    \end{tabular}   
\end{exe}
Gleiches gilt für eine exhaustive Suche nach V1-Sätzen in einem Teilkorpus von DECOW2014 (vgl. (\ref{952})).

\begin{exe}
	\ex\label{952} V1-Sätze in DECOW2014AX (Teilkorpus) (exhaustiv)\footnote{Anfrage: $<$s$>[$pos= \glqq VVFIN\grqq{} $][]\lbrace$0,4$\rbrace[$word= \glqq doch\grqq{} 		$]$}\\[-1em]
    \begin{tabular}[t]{|l|l|}
    \hline
    \textit{doch} & \textit{ja}\\
    \hline	 
    698 (2 x \textit{doch wohl}, 2 x \textit{ja doch}, 15 x \textit{doch auch}) & 16 (3 x \textit{ja auch}, 2 x \textit{ja doch})\\
    \hline
    \end{tabular}   
\end{exe}
Vorausgesetzt, \textit{doch} und \textit{ja} stehen nicht sowieso in dem Frequenzverhältnis zuein\-ander, wie es sich hier für \textit{doch}- und \textit{ja}-V1-Sätze einstellt, scheint behauptbar, dass \textit{doch} in diesem Satztyp deutlich überwiegt. Die \textit{ja}-Treffer, die auftreten, können aber nicht durchweg als älteren Sprachstufen zugehörig angenommen werden (vgl. (\ref{953}) und (\ref{954})) (contra \citealt[158]{Oennerfors1997}) (vgl. auch schon meine Belege in Kapitel~\ref{chapter:jud}, Abschnitt~\ref{sec:nonkan}).

\begin{exe}
	\ex\label{953}
	\scriptsize
	Grundsätzlich könnt das durchaus Sinn machen. \textbf{\textit{Geht} es bei Monopoly \underline{ja} darum}, Gebäude und Orte zu kaufen, zu erhalten und 		dafür Miete zu kassieren. 			      
	\hfill\hbox{(DECOW2014AX)}
	\newline  
	\hbox{}\hfill\hbox{(http://locationmarketing.at/)}
\end{exe}
								          
\begin{exe}
	\ex\label{954}
	\scriptsize
	Neben den bereits genannten Lutzmannsburg, Großwarasdorf und Sieggraben würden mit Rattersdorf, Mannersdorf, Steinberg und Tschurndorf gleich weitere 		vier Teams mittendrin statt nur dabei sein. \textbf{\textit{Trennen} den Neunten, Rattersdorf, \underline{ja} lediglich sechs Punkte vom Vorletzten 		Großwarasdorf.} 			      
	\newline  
	\hbox{}\hfill\hbox{(BVZ08/DEZ.00744 Burgenländische Volkszeitung, 03.12.2008, S. 71}
\end{exe} 
Bei beiden Satztypen sprechen die Verteilungen folglich dafür, dass es berechtigt ist, die Frage zu stellen, warum die MP \textit{doch} für diese Sätze so wichtig ist und damit auch die nächste Frage aufzuwerfen, welchen Beitrag sie in ihnen leistet.
	
\subsection{Unkontroverse/Thematizität}
\label{sec:unkontr}
Für beide Satztypen ist angenommen worden, dass sie den Sachverhalt, auf den sie sich beziehen, als unkontrovers, bekannt, ein Faktum oder Hintergrundinformation markieren,  ihn m.a.W. präsupponieren \is{Präsupposition} (wenngleich nicht alle Arbeiten diesen konkreten Begriff verwenden): Bei \citet[90]{Kwon2005} heißt es z.B. über \textit{Wo}-VL- und V1-Sätze, der Sachverhalt sei unkontrovers. Über erstere schreibt \citet[43]{Winkler1992}, ihr Inhalt sei bekannt. \citet[145]{Pasch1999} schreibt, er sei eine Tatsache, ein Faktum, evident, er werde als dem Adressaten bekannt ausgegeben. Ähnlich liest man bei \citet[236]{Eroms2000} von einem \glqq diskursiv akzeptierten Tatbestand\grqq{} und bei \citet[315]{Guenthner2002} von Hintergrundinformation, die als evident ausgelegt wird, evidentem/präsupponiertem Inhalt (S. 325) und der Unmöglichkeit der Rhematizität des Begründungsinhalts (S. 325) (vgl. ähnliche Attribute auch bei \citealt[134, 135, 148]{Guenthner2007}). Über die V1-Sätze schreibt \citet[1020]{Altmann1993}, der mit ihnen ausgedrückte Sachverhalt sei unkontrovers und akzeptiert. Auch \citet[171]{Pittner2011} hält ihn für einen allgemein akzeptierten Grund.

Mein im Folgenden ausgeführter Punkt ist, dass ich mich für beide Satztypen gegen diese Einschätzungen aussprechen möchte. Ich werde meine Argumente für beide Typen getrennt darlegen. Meine Argumentation gegen den präsupponierten Status des Sachverhalts ist auch durch die Sicht bedingt, dass in meiner Modellierung der \textit{doch}-Bedeutung der Aspekt von Präsupponiertheit nicht vorhanden ist. Ich bin der Meinung, dass man ihn nicht zur Bedeutung von \textit{doch} machen muss. 

\subsubsection{\textit{Wo}-Verbletzt-Sätze}
Beispielsweise frage ich mich, ob die Eigenschaft, unkontrovers, bekannt, hintergrundierend, evident, präsupponiert zu sein, nicht zu einem gewissen Grad auf jeden kausalen Nebensatz zutrifft. Sicherlich sind Unterschiede in der Interpretation und Verwendung verschiedener Kausalsätze gemacht worden, vermutlich würde aber keiner Fragliches oder Spekulatives zum Inhalt eines Satzes machen, der als Begründung intendiert ist.

Ferner geht die Zuschreibung dieser Eigenschaft nicht gut mit der Annahme einher, dass die \textit{Wo}-Sätze modal gelesen werden und somit subjektiviert sind. Man geht davon aus, dass sich die Kausalität dieser Sätze immer auf epistemi\-scher oder illokutionärer Ebene abspielt. Sie können Annahmen, Einstellungen oder Sprechakte begründen. Sachverhaltsbegründungen sollten hingegen nicht möglich sein. Ein \textit{Wo}-Satz kann deshalb z.B. nicht die Antwort auf eine Ergänzungsfrage sein (vgl. (\ref{955})).

\begin{exe}
	\ex\label{955}
	A: Warum kommt du nicht mit essen?\\
	B: *\textbf{Wo} ich keinen Hunger hab.	 			      
	\hfill\hbox{\citet[325]{Guenthner2002}}\footnote{Günthner selbst vertritt eine andere Erklärung der Inakzeptabilität des \textit{Wo}-Satzes in diesem 		Kontext.}
\end{exe} 
Gibt man einen Sachverhalt stets als Faktum aus, würde es auch naheliegen, damit einen anderen Sachverhalt zu begründen. Ich sehe keinen Grund, warum mit einem als real ausgegebenen Sachverhalt ausschließlich subjektive Einschätzungen begründet werden sollten. Von den vier Möglichkeiten der Verteilung von Objektivität und Subjektivität auf den begründeten und begründenden Sachverhalt erscheint mir die objektive Begründung einer subjektiven Wahrnehmung am unplausibelsten, weshalb ich die Beschränkung des kausalen \textit{Wo}-Satzes auf dieses Verhältnis nicht akzeptieren kann.

\citet{Guenthner2002} leitet aus der Annahme, der Inhalt der Sätze sei stets präsupponiert und evident, ab, warum \textit{Wo}-Sätze sich nicht als Antwort auf eine w-Frage eignen (vgl. (\ref{955})). Der Kontext würde die Rhematizität des Inhalts nahelegen. Auf dieselbe Art erklärt sie die Ungrammatikalität von (\ref{956}) und (\ref{957}).

\begin{exe}
	\ex\label{956}
	*ich heirate ihn, wo ich ihn liebe und nicht, wo er Geld hat.
	\newline 	 			      
	\hbox{}\hfill\hbox{\citet[325]{Guenthner2002}}
\end{exe}
\vspace{-0.65cm}
\begin{exe}
	\ex\label{957}
	*Ich habe deshalb das Fenster geschlossen, \textbf{\textit{wo}} es \textbf{doch} so verbrannt gerochen hat. 			      
\end{exe}
Hier zeigt sich, dass \textit{wo}-Sätze nicht im Skopus der Negation stehen können und es auch keine \textit{deshalb-wo}-Konstruktion geben kann.

Der Grund für den Status der Daten in (\ref{955}) bis (\ref{957}) ist m.E. ein anderer als die unmögliche Kodierung von Rhematizität. (\ref{955}) habe ich schon derart beschrieben, dass der Kontext die propositional-kausale Lesart fordert, die \textit{wo} nicht aufweist. Und genauso lassen sich die beiden weiteren Beispiele erklären. Die \textit{deshalb}-Struktur wird immer propositional interpretiert und auch in (\ref{956}) liegt eine Sachverhaltsbegründung vor (vgl. \citealt[143]{Pasch1999} für weitere Evidenz der ausgeschlossenen propositionalen Lesart). Mit der Bekanntheit der Proposition muss/kann der Ausschluss auch nicht begründet werden, weil \textit{denn}, für das angenommen wird, dass es sich auf neue Information bezieht, hier ebenfalls nicht auftreten kann. Auch \textit{denn} scheint aber immer/favorisiert nicht-propositional interpretiert zu werden – was folglich die Datenlage erklärt.

Weitere Evidenz für die Annahme, dass der \textit{wo}-Satz Hintergrundinformation beisteuert, sieht \citet{Guenthner2002} im Dialog in (\ref{958}).

\begin{exe}
	\ex\label{958} HAUSRENOVATION (Bodensee)\\
	43Dora: mhm. des geht ECHT langsam. (–)\\
	44 \hspace{0.5cm}nobwohl wir VIE:L zeit rein$[$stecken.$]$\\
	45Ute:\hspace{0.5cm}			         $[$(           )$]$\\
	46Dora: JE.DES. WOCHENENDE machen wir dran rum! 
	\newline	 				      
	\hbox{}\hfill\hbox{\citet[331]{Guenthner2002}}
\end{exe}
Sie geht davon aus, dass \textit{wo}-Konstruktionen durch \textit{obwohl}-Konstruktionen zu ersetzen sind, aber die umgekehrte Ersetzbarkeit nicht gegeben ist. I.E. gibt es folg\-lich weitere Restriktionen für konzessive \textit{wo}-Strukturen, und zwar, wenn ihr Inhalt neue Information ist. (\ref{958}) ist Günthner zufolge ein Fall, in dem anstelle des \textit{obwohl}-Satzes kein \textit{wo}-Satz eingesetzt werden kann. Als Grund führt sie an, der \textit{obwohl}-Satz könne hier keine Hintergrundinformation liefern. Wiederum liegt die Inakzeptabilität des \textit{wo}-Satzes in meinen Augen aber nicht daran, dass er neue Information kodieren muss, sondern ist darauf zurückzuführen, dass sich die modale Begründung einer Einstellung/Annahme hier nicht gut anbietet. Der \textit{wo}-Satz wird aber akzeptabel, wenn man den ersten Satz so liest, dass sich die Person über die Langsamkeit z.B. aufregt oder wundert, und der \textit{wo}-Satz diese Haltung begründet. Die \glq konzessiven\grq {} \textit{wo}-Sätze können prinzipiell nicht verwendet werden, wenn man die modale Begründung nicht motivieren kann.

Betrachtet man \textit{Wo}-VL-Daten, finden sich keine Beispiele, die die Ansicht Günth\-ners zum Status des Sachverhalts im \textit{Wo}-Satz in dem Sinne widerlegen, dass man sagen müsste, ein ausgedrückter Sachverhalt kann kein Faktum/keine Hintergrundannahme etc. sein. Dies liegt aber daran, dass jede assertierte Information akkommodiert \is{Akkommodation} werden kann. Es ist ebenso schwierig, assertive Kontexte zu finden, in denen \textit{ja} nicht stehen kann. Auf der Basis von Belegen halte ich die Annahme deshalb für schwer widerlegbar. Es gibt aber andererseits auch keinen Grund, die Inhalte stets als faktisch zu lesen. In den \textit{Wo}-Sätzen treten nicht ausschließlich absolute Fakten oder unumstößliche Tatsachen auf. Es kann sich genauso plausibel nur um eine Ansicht oder Einschätzung des Sprechers handeln, die im gegebenen Kontext neu ist.

\begin{exe}
	\ex\label{959}
	\scriptsize
	Warum?\\
	Warum ist es Wichtig, das sich die Eheleute \glqq lieb haben\grqq{} und viele viele gemeinsame Kinder zeugen?\\
	Wo ist der Nutzen?\\
	Reicht es nicht, wenn se einfach nur nebeneinander in einem Haus leben, ohne im extrem mit einander zu Reden?\\
	Und Warum ist dafür die Liturgie nötig? \textbf{\textit{Wo} es \underline{doch} reicht, wenn man den beiden einfach ihre Freiheiten lässt (die ihnen 		die Liturgie nimmt)?} 	
	\hfill\hbox{(DECOW2014)}
	\newline
	\hbox{}\hfill\hbox{(http://www.ulisses-forum.de/archive/index.php/t-8220.html)}
\end{exe}
 
\begin{exe}
	\ex\label{960}
	\scriptsize
	Auf die Gefahr hin hier geschlagen zu werden: boah, sind die beide häßlich\\
	\textbf{\textit{Wo} es \underline{doch} so endlos geile Käfer gibt}\\
	*duckundwech*			
	\hfill\hbox{(DECOW2014)}
	\newline
	\hbox{}\hfill\hbox{(http://www.passat35i.de/archive/index.php/t-16223.html)}
\end{exe}					                 

\begin{exe}
	\ex\label{961}
	\scriptsize
	Anrufe!! Er hatte heute gar nicht angerufen...\\
	Warum hatte er mich nicht angerufen??? \textbf{\textit{Wo} ich \underline{doch} anfing ihn ganz nett zu finden!!!!!!!!!!} Ich konnte es mir schon 			denken warum:\\
	Nach den 2 Tagen wo wir miteinander geredet hatten, hatte er mich schon satt...\\
	War ja klar, dass er mich verarscht hatte... Wer würde mich schon lieben??\\
	NIEMAND!!!!!!! 			
	\hfill\hbox{(DECOW2014)}
	\newline
	\hbox{}\hfill\hbox{(http://www.rockundliebe.de/liebesgeschichten/liebesgeschichten\_1474\_m.php)}
\end{exe}										  
Ein anderer Punkt, der für mich gegen die vorgenommenen Eigenschaftszuschreibungen (Hintergrund, Bekanntheit, Faktum, Unkontroversität) spricht, ist auch, dass man im \textit{Wo}-Satz das Vorkommen von \textit{ja} erwarten würde, als beliebte Partikel an der Stelle, weil diese unmissverständlich Bekanntheit, Faktenorientierung etc. kodiert. Die Verteilungen in (\ref{947}) und (\ref{951}) zeigen aber, dass \textit{ja} hier nicht auffällig beliebt ist.

Zudem bin ich nicht der Meinung, dass \textit{doch} überhaupt inhärent diese Bedeutungsaspekte mitbringt. Nach meiner Modellierung setzt \textit{doch} einen Kontextzustand voraus, in dem der Sachverhalt, auf den sich die MP-Äußerung bezieht, schon zur Diskussion steht. Die von einigen Autoren ins Feld geführten Bedeutungsaspekte des \textit{Wo}-Satzes sind mit dieser Bedeutung zwar nicht inkompatibel, aber in ihr auch nicht explizit vertreten. 

In einer empirischen Studie, in der \citet{Doering2014} die Korrelation von be\-stimmten Diskursrelationen und MPn untersucht, findet sich keine Evidenz für eine Assoziation von \textit{doch} mit Relationen wie HINTERGRUND und EVIDENZ. 

In einer Korpusstudie (Protokolle von Parlamentsreden von Helmut Kohl) be\-stimmt sie die Häufigkeit des Auftretens einer MP in Verbindung mit einer be\-stimmten Diskursrelation relativ zu einem Erwartungswert für das unabhängige Auftreten der untersuchten \is{Diskursrelation} Diskursrelationen. 

Sie stellt fest, dass \textit{ja} häufiger als erwartet mit den Relationen HINTERGRUND, EVIDENZ und GRUND auftritt, und \textit{doch} in den Relationen JUSTIFY, EVALUATION, INTERPRETATION, MOTIVATION und EVIDENZ. Während die Korrelationen von \textit{ja} zu der dieser Partikel zugeschriebenen Bedeutung (Döring zufolge Bekanntheit und Unkontroverse) passen, hält sie die auftretenden Relationen bei \textit{doch} (Bedeutung: Bekanntheit, Unkontroverse, Kontrast) für unerwartet. Sie erklärt sie über manipulative Verwendungen, was sie aber nur für EVIDENZ ausführt. 

Das Ergebnis spricht für mich gerade dafür, dass \textit{doch} die Komponente von Bekanntheit/Unkontroverse nicht aufweisen muss: Während ein evidenter Sach\-verhalt plausiblerweise auch als bekannt/unkontrovers ausgegeben wird, ist eine Rechtfertigung stets sprechergebunden und muss nicht Einigkeit voraussetzen. Eine Evaluation muss erst recht keine Einigkeit voraussetzen: Es handelt sich um eine subjektive Einschätzung und es ist auch gar nicht beabsichtigt, dass der Adressat diese Ansicht teilt. Nimmt der Sprecher eine Ausdeutung des zuvor Gesagten vor (INTERPRETATION), muss diese Information auch keineswegs bekannt oder unkontrovers sein. Die Bedeutungskomponenten, die gerade die Partikel \textit{ja} auszeichnen, scheinen mir nicht das Auftreten von \textit{doch} mit den ge\-nannten Diskursrelationen zu motivieren. 

Die Frage ist, ob der Aspekt von Widerspruch/Kontrast, der in meiner Modellierung durch das geforderte offene Thema vertreten ist, mit diesem Ergebnis zu motivieren ist. Im Falle der Relation JUSTIFY ist diese Argumentation denkbar: Eine Rechtfertigung hat einen Auslöser, d.h. dass das Thema (aus irgendwelchen Gründen) schon im Raum steht, scheint sehr plausibel. Bei der Relation MOTIVATION soll der Adressat zu einer Handlung bewegt werden. In der Erklärung von \citet[89]{Doering2014} tritt auch hier der Aspekt von Bekanntheit/Unkontroverse nicht auf. Dies bietet sich auch wieder nicht an. Sie schreibt hier vor allem über die Verwendung in Imperativen, bei denen generell qua Sprechaktbedingung davon auszugehen ist, dass ihr Inhalt nicht bereits bekannt ist. Für die Imperative, die über die Relation der Motivation in den Diskurs eingebunden sind, lässt sich gut annehmen, dass die Handlungsaufforderung verstärkt wird, indem sie (expliziter als Direktive überhaupt) voraussetzen, dass fraglich ist, ob der Angesprochene die Aktion ausführen wird (vgl. auch meine Ausführungen in Abschnitt~\ref{sec:direktive}). Für die anderen drei Relationen, bei denen sie Korrelationen mit \textit{doch} feststellt (EVALUATION, INTERPRETATION, EVIDENZ), scheint mir ein vorausgesetztes offenes Thema auch nicht direkt zu motivieren. Kompatibel ist die Situation sicherlich, ausgeschlossen sind Bekanntheit/Unkontroverse aber auch nicht.

Zu ihrer eigenen Verwunderung stellt Döring keine Verbindung zwischen dem Vorkommen von \textit{doch} und den Relationen KONTRAST, CONCESSION und ANTITHESE fest (\citeyear[89]{Doering2014}). Sie erklärt dies tentativ über das vorliegende Genre: In den Parlamentsreden sei sowieso klar, dass der Hörer eine andere Meinung vertrete.\footnote{Als potenziell beeinflussenden Faktor führt sie auch an, dass die Daten zum einen auf einen einzigen Sprecher zurückgehen, und zum anderen, dass die Entscheidungen über die vorliegenden Diskursrelationen eine subjektive Einschätzung bleiben (vgl. \citeyear[90]{Doering2014}).}

In einer weiteren experimentellen Studie bestätigt sich der Aspekt der Korpusstudie, dass \textit{doch} keine Korrelation mit Hintergrundinformation eingeht. In als BACKGROUND- bzw. JUSTIFY-Kontexten angelegten Umgebungen wie in (\ref{962}) hatten die Testanten die Wahl zwischen \textit{ja}, \textit{doch} und (als Filler) betontem \textit{schon}.

\begin{exe}
	\ex\label{962}
	Wenn Ganztagsschulen eingeführt werden, verlieren Musikschulen und Sportvereine viele Mitglieder.\\
	\newline
	\noindent
	BACKGROUND: In Musikschulen machen Kinder \hrulefill \ die größte Gruppe \\
	der Mitglieder aus.\\
	JUSTIFY: Dieser Aspekt muss \hrulefill  \ mal in 
	den Vordergrund gerückt werden.\\
	\newline
	\noindent
	Ein solcher Mitgliederschwund ist für diese Einrichtungen verheerend!	
	\newline
	\hbox{}\hfill\hbox{\citet[90-91]{Doering2014}}
\end{exe}		
Das Ergebnis ist, dass \textit{ja} häufiger als erwartet im Hintergrundkontext und weniger häufig als erwartet im Rechtfertigungskontext gewählt wurde. \textit{Doch} wurde genau entgegengesetzt öfter als erwartet zusammen mit der Relation der Rechtfertigung und seltener als erwartet mit der Relation des Hintergrundes ausgewählt. Die beiden Partikeln werden folglich jeweils in einem der beiden Kontexte bevorzugt. Wie ich oben bereits ausgeführt habe, sehe ich keinen Grund, davon auszugehen, dass Rechtfertigungen stets mit den Bedeutungsaspekten Bekanntheit und Unkontroverse einhergehen.
 
Ein letztes Argument gegen die Annahme, der \textit{Wo}-Satz verweise stets auf bekannte Information, ist für mich die Tatsache, dass Strukturen der Art in (\ref{963}) bis (\ref{965}) ein sehr typisches Muster in den Belegen sind.

\begin{exe}
	\ex\label{963}
	\scriptsize
	Wieso müssen Agenturpartys immer am Donnerstag sein? \textbf{\textit{Wo} \underline{doch} jeder weiß, dass Donnerstags die neuen Filme anlaufen.}	
	\hfill\hbox{(DECOW2014AX)}
	\newline
	\hbox{}\hfill\hbox{(http://www.ankegroener.de/anke1/pasdeblog/blogarchiv/september2002.html)}
\end{exe}

\begin{exe}
	\ex\label{964}
	\scriptsize
	Vampire Hunter D:\\
	Das war jetzt aber sehr fies. \textbf{\textit{Wo} \underline{doch} jeder weiß, daß das gar nicht zu schaffen ist!}		
	\hfill\hbox{(DECOW2014AX)}
	\newline
	\hbox{}\hfill\hbox{(http://www.comicforum.de/archive/index.php/t-88976.html)}
\end{exe}				               

\begin{exe}
	\ex\label{965}
	\scriptsize
	Kein leichtes Thema: Paranoia ist ein Hirngespinst; und die Schwierigkeit des Autors besteht deshalb insbesondere darin, den Leser bei der Stange zu 		halten. \textbf{\textit{Wo} der \underline{doch} weiß, dass die Dinige, die geschehen, lediglich einen innerpsychischen Wahrheitswert für die 				Betroffenen haben.}
	\newline
	\hbox{}\hfill\hbox{(NUN08/JUL.00391 Nürnberger Nachrichten, 04.07.2008)}
\end{exe}				               		                            
In den DECOW-Daten machen diese Strukturen 15\% der \textit{Wo-doch}-Belege aus (93 x eine Form von \textit{wissen} + Nebensatz, davon 63 x eingeleitet durch \textit{jeder weiß}). In einer Stichprobe von 500 Sätzen mit finiten, lexikalischen Verben treten in DECOW2014AX sechs Vorkommen von \textit{wissen} auf. Es ist folglich nicht anzunehmen, dass der Anteil der Strukturen mit \textit{wissen} auf die analoge Häufigkeit dieses Verbs zurückzuführen ist. Handelt es sich um ein Muster, finde ich es auch in diesem Unterfall der Verwendung von \textit{Wo}-Sätzen schwierig, abzuleiten, dass \textit{doch} hier Hintergrund/Bekanntheit \is{Hintergrund} markiert. Der Inhalt der Sätze transportiert dann schließ\-lich genau die Bedeutung, die die MP beisteuern soll. Es gibt andere Beispiele von MP-Charakterisierungen, bei denen derart vorgegangen worden ist (zur Kritik vgl. \citealt[380]{Ickler1994}) (vgl. (\ref{966}) und (\ref{967})).

\begin{exe}
	\ex\label{966}
	Sie wissen \textbf{ja}, daß er nächste Woche operiert wird.\\
	\textit{ja}: Bekanntheit des Sachverhalts für den Hörer
\end{exe}	

\begin{exe}
	\ex\label{967}
	Da kann man \textbf{halt} nichts machen.\\
	\textit{halt}: Einsicht des Sprechers in die Unabänderlichkeit des geäußerten Sachverhalts
	\hfill\hbox{\citet[380]{Ickler1994}}
\end{exe}
Es darf zumindest als ungeschickt gelten, anhand derartiger Beispiele die Bedeutung der Partikeln motivieren zu wollen.\\
\newline
\noindent
Auch für die V1-Sätze gilt, dass einige Aspekte die Annahme in Frage stellen, dass ihr Inhalt stets unkontrovers, thematisch und präsupponiert ist.

\subsubsection{Verberst-Sätze}
Für diese Sätze gilt ebenfalls der Einwand, dass unklar ist, ob \textit{doch} die besagten Bedeutungsanteile tatsächlich kodiert und sein typisches Auftreten durch ihr Vorliegen deshalb motiviert ist.

Schon für die \textit{Wo}-Sätze habe ich festgestellt, dass es schwierig erscheint, explizit zu widerlegen, dass die Sätze nicht auf Fakten verweisen. Bei den V1-Sätzen ist dies noch weiter erschwert, da sie fast ausschließlich in schriftsprachlichen Kontexten vorkommen. In Zeitungskontexten beispielsweise werden sowieso meist Fakten berichtet. Ich sehe allerdings auch keinen Grund anzunehmen, dass die Inhalte deshalb stets als bekannt oder hintergründig eingestuft werden müssen. Für meine Begriffe handelt es sich bei (\ref{968}) und (\ref{969}) nicht um Inhalte, die der Leser in jedem Fall bereits weiß und deshalb unzweifelbar hinnimmt oder die längst akzeptiert sind. Es bietet sich ebenso die Interpretation an, dass die Inhalte neu sind ((\ref{968})) oder es sich lediglich um eine Annahme des Schreibers handelt ((\ref{969})).
	
\begin{exe}
	\ex\label{968}
	\scriptsize
	Rund um die Weihnachtsfeiertage gab es im Land um Hollabrunn Büro noch einen weiteren Grund zum Feiern. \textbf{\textit{Konnte} man \underline{doch} 		dem Regionalmanager Didi Jäger zum Geburtstag gratulieren.}
	\newline
	\hbox{}\hfill\hbox{(NON09/JAN.01979 Niederösterreichische Nachrichten)}
\end{exe}	

\begin{exe}
	\ex\label{969}
	\scriptsize
	Aber warum eigentlich eine Himbeere? Tut man ihr damit nicht Unrecht und kratzt an ihrem Image? \textbf{\textit{Schmeckt} sie \underline{doch} wirklich 	lecker} – und süß. Wie Erfolg. Walle, äh ... \glqq WALL-E\grqq{} weiß, wovon ich rede.
	\hfill\hbox{(BRZ09/FEB.12094 Braunschweiger Zeitung, 25.02.2009)}
\end{exe}
Ein anderer Einwand betrifft die vollkommentarische Natur \is{vollkommentarisch} von V1-Deklarativ\-sätzen. 

Es gibt Arbeiten, die sich mit den generellen Eigenschaften von V1-Deklara\-tivsätzen beschäftigen. Ich bin der Meinung, dass die Annahme, es sei die Natur der begründenden V1-Sätze, stets unkontrovers, bekannt etc. zu sein, einigen grundsätz\-lichen Überlegungen zu V1-Deklarativsätzen zuwiderläuft.

In einer Typologie deutscher V1-Deklarativsätze reiht sich dieser Typ in die Fälle in (\ref{970}) ein. 

\begin{exe}
	\ex\label{970} V1-Deklarativsatztypen im Deutschen\\[-1em]
		\begin{xlist}	
			\ex\label{970a} narrative V1-Deklarative \is{narrativer V1-Deklarativsatz}
				\begin{xlist}
					\ex\label{970x} Hab ich ihr ganz frech noch en Kuß gegeben.
					\ex\label{970y} $[$Die Tübinger mögen sowas$]$. Sprach die Künstlerin hinterher erfreut-verwundert: ...					
				\end{xlist}
			\ex\label{970b} aufzählend-reihende V1-Deklarative \is{aufzählend-reihender V1-Deklarativsatz}
				\begin{xlist}
					\ex\label{970s} Bleibt ein dritter Einwand, nicht weniger gravierend.
					\ex\label{970t} $[$Die Tendenz / geht ... / nach unten.$]$ Kommt noch hinzu, ...				
				\end{xlist}
			\ex\label{970c} inhaltlich-begründende V1-Deklarative \is{inhaltlich-begründender V1-Deklarativsatz}\\
				$[$Sein Tod bewegt viele.$]$ Hatte doch seine Ära den Wiederaufstieg / ... / begründet.   	
				\hfill\hbox {\citet[216]{Reis2000} [nach \citet[99-184]{Oennerfors1997}]}
		\end{xlist}
\end{exe}
In diversen Arbeiten ist angenommen worden, dass V1-Sätze (und zwar auch nicht-deklarative/assertive) vollrhematisch \is{vollrhematisch} bzw. vollfokussiert \is{vollfokussiert} sind (vgl. \citealt{Rosengren1992} (vgl. zu einem Überblick über derartige Erwähnungen \citealt[71-86]{Oennerfors1997}). Auch typologisch (vgl. \citealt{Sasse1995} und diachron (vgl. \citealt{Coniglio2012} zum Mittelhochdeutschen, vgl. auch die dort genannten Arbeiten zum Althochdeut\-schen (\citeyear[10]{Coniglio2012}) ist dies eine gängige Annahme. Deklarative V1-Sätze sind aufgrund dieser Charakteristik mit thetischen Sätzen assoziiert worden.

Önnerfors argumentiert, dass man nicht annehmen kann, dass die Sätze vollrhematisch sind in dem Sinne, dass sie kein Element beinhalten, das im Kontext nicht bekannt ist (z.B. Pronomen). Auch argumentiert er gegen Vollfokus (weil z.B. Scrambling \is{Scrambling} möglich ist) (\citeyear[76-82]{Oennerfors1997}). Er geht davon aus, dass V1-Deklarativsätze keine Topik-Kommentar-Gliederung \is{Topik-Kommentar-Gliederung} haben (\citeyear[84]{Oennerfors1997}) und des\-halb z.B. keine Topikalisierung \is{Topikalisierung} im Mittelfeld erlauben. Sie sind vollkommentarisch. Die klassische \is{Topikposition} Topikposition (das Vorfeld) ist in diesen Sätzen nicht vorhanden. In dieser Hinsicht ähneln sie thetischen Sätzen (vgl. (\ref{971})) und V2-Sätzen mit initialem \textit{es} (vgl. (\ref{972})).

\begin{exe}
	\ex\label{971}
	PEter hat angerufen.
\end{exe}
\vspace{-0.65cm}
\begin{exe}
	\ex\label{972}
	Es war einmal ein König.
	\hfill\hbox{\citet[305]{Oennerfors1997a}}
\end{exe}
(\ref{971}) weist keine Topik-Kommentar-Gliederung auf. Es liegt auch Vollfokus vor, d.h. \is{Satzfokus} Satzfokus. Der Satz kann out-of-the-blue \is{out-of-the-blue-Äußerung} geäußert werden und die Antwort auf eine sehr offene Frage darstellen. In vielen Sprachen liegt in Sätzen dieser Art Verb-Subjekt-Abfolge vor. Sätze wie (\ref{972}) führen ein Topik ein, \textit{es} kann aber kein Topik sein. Önnerfors unterscheidet zwar die drei V1-Satztypen in (\ref{970}), nimmt aber auch an, dass sie – trotz unterschiedlicher Funktionen – diesen \glqq informations\-strukturelle$[$n$]$ Kern\grqq{} teilen.

Wenn es so ist, dass die begründenden V1-Sätze die Eigenschaft aufweisen, vollkommentarisch zu sein, halte ich die Annahme für merkwürdig, dass \textit{doch} in diesen Sätzen anzeigen soll, dass der Sachverhalt präsupponiert \is{Präsupposition} ist. Önnerfors lehnt ab,  dass die Sätze auch vollfokussiert und rhematisch sind (s.o.). Der Kommentar, den die Sätze ausschließlich aufweisen, kann natürlich vollfokussiert und rhematisch sein, er muss es aber nicht. Derartige Sätze können z.B. erneut geäußert werden, um jemanden zu erinnern (vgl. (\ref{973})).

\begin{exe}
	\ex\label{973}
	Wie du bereits weißt: Es regnet.
\end{exe}
Folgte man den Annahmen aus der Literatur, läge ein Satztyp vor, in dem der Kommentar immer auf hintergrundierende, bekannte und akzeptierte Information verweist. Dies halte ich für unplausibel. \citet[223, Fn 22]{Reis2000} verweist darauf, dass \textit{doch} die vollkommentarische Eigenschaft außer Kraft setzen kann, weil I-Topikalisierung \is{I-Topikalisierung} möglich ist.

\begin{exe}
	\ex\label{974}
	$[$Das geht schon,$]$ weinen ALLe (/) Mädchen doch NICHT (\textbackslash) so leicht.
\end{exe}
Nimmt man diesen Aspekt hinzu, greift mein Argument gegen den präsupponier\-ten \is{Präsupposition} Status unter Bezug auf die vollkommentarische Natur dieses Subtyps nicht mehr. In (\ref{974}) sind \textit{alle Mädchen} das Topik \is{Topik} und der Rest des Satzes, auf den sich auch \textit{doch} bezieht, der Kommentar. Dann würde man allerdings nach wie vor davon ausgehen, dass der Kommentar \is{Kommentar} in diesen Sätzen immer alte Information beinhaltet. Ich glaube nicht, dass man dies verallgemeinern kann.

Neben Önnerfors Annahmen, die für mich gegen den stets präsupponierten Status der Sätze sprechen, halte ich auch Reis' Beispiel (vgl. (\ref{974})) für real. In meinen Daten gibt es ohne Zweifel Fälle, in denen Referenten aus dem Vorgänger\-kontext in den V1-Sätzen aufgegriffen werden und in diesen Sätzen eine Interpretation als Topik erfahren können (vgl. (\ref{975}) bis (\ref{977})).

\begin{exe}
	\ex\label{975}
	\scriptsize
	@Melvin: Danke für \emph{die Antwort}. \textbf{\textit{Zeigt} \textsc{diese} \underline{doch}, daß auch von dieser Seite noch in vernünftigen Bahnen 		gedacht wird.}	
	\hfill\hbox{(DECOW14AX)}
	\newline
	\hbox{}\hfill\hbox{(http://foren.waffen-online.de/lofiversion/index.php/t380704.html)}
\end{exe}
			 
\begin{exe}
	\ex\label{976}
	\scriptsize
	Aber \emph{Peter Sch.} hätte vor dem Auftritt über diese Besonderheit aufgeklärt werden sollen. \textbf{\textit{Lamentierte} \textsc{er} 					\underline{doch} darüber, dass seine Strumpfhose nur ein Bein habe}.....kicher......	
	\hfill\hbox{(DECOW14AX)}
	\newline
	\hbox{}\hfill\hbox{(http://home.arcor.de/abschlusshakemicke93/geschichten.html)}
\end{exe}						       

\begin{exe}	
	\ex\label{977}
	\scriptsize
	Es ist, so nehme ich mal an, eine \emph{der klassischen Armbanduhren oder Chronografen}, die zwar nicht mit ihrer Ganggenauigkeit, aber jederzeit mit 		ihrer Ausstrahlung der modernen Konkurrenz Paroli bieten können. \textbf{\textit{Signalisieren} \textsc{sie} \underline{doch}, dass ihr Besitzer mit 		einen individuellen Geschmack ausgestattet ist} und aller Wahrscheinlichkeit nach eine berufliche Position begleitet, die nicht mehr am sekundengenauen 	Erscheinen gemessen wird. 
	\hfill\hbox{(DECOW14AX)}
	\newline
	\hbox{}\hfill\hbox{(http://www.freunde-alter-wetterinstrumente.de/61news08.htm)}
\end{exe}						            						
Hier werden in den V1-Sätzen weitere Informationen über \textit{die Antwort}, \textit{Peter Sch.} und \textit{die Uhren} präsentiert. Es sind folglich keine auffälligen Strukturen wie die I-Topikalisierung \is{I-Topikalisierung} in Reis' Beispiel vonnöten. 

Nach Betrachtung einer großen Menge von \textit{doch}-V1-Sätzen (aus DECOW) bin ich der Meinung, dass Reis' Beispiel nicht gegen Önnerfors Auffassung spricht, sondern die Daten seine Einschätzung stützen: Fälle der Art in (\ref{975}) bis (\ref{977}) gibt es zwar, sie sind aber eher unüblich. Evidenz für diese Annahme liefert eine Untersuchung der auftretenden Subjekte in den 698 \textit{doch}-V1-Sätzen. Mit hohem Anteil (und zwar mit höherem als erwartet $[$s.u.$]$) kommen nicht referierende, semantisch leere Subjekte vor. Es treten folglich Subjekte auf, die sich nicht als Topik \is{Topik} eignen. Konkret handelt es sich hierbei um non-anaphorische \textit{es}-Subjekte (zu weiteren Details s.u.). Natürlich muss das Subjekt nicht das Topik im Satz sein (vgl. (\ref{978})), man geht aber davon aus, dass diese beiden Status in der Regel zusammenfallen. Das Subjekt gilt als das unmarkierte Topik (vgl. z.B. \citealt[132]{Lambrecht1994}).

\begin{exe}
	\ex\label{978}
	A: Erzähle mir was \textit{von Hans}!\\
	B: Maria hat \textit{ihn} verlassen.
\end{exe}
In den besagten V1-Sätzen mit \textit{es}-Subjekten scheint mir ebenfalls nicht das Verhältnis vorzuliegen, dass stets andere Elemente als Topik fungieren. Strukturen, um die es geht, sind solche der Art in (\ref{979}) bis (\ref{981}).

\begin{exe}
	\ex\label{979}
	\scriptsize
	Johnny Cash, der Mann in Schwarz, er bekam insgeheim den meisten Applaus. \textbf{\textit{Gilt} es \underline{doch} die Musiker zu ehren}, die sich um 		die Country Music verdient gemacht haben.
	\hfill\hbox{(DECOW14AX)}
	\newline
	\hbox{}\hfill\hbox{(http://www.musikansich.de/ausgaben/1203/\_cma.html)}
\end{exe}


\begin{exe}
	\ex\label{980}
	\scriptsize
	Als seine Frau Nancy sich von ihm scheiden lässt, erscheint ihm sein größtes Kinoidol Humphrey Bogart und gibt ihm gute Ratschläge für die Zukunft. 		Allan schöpft wieder Hoffnung. \textbf{\textit{Gibt} es \underline{doch} mehrere Millionen Frauen allein in New York} und eine davon wird er bestimmt 		erobern. 
	\newline
	\hbox{}\hfill\hbox{(DECOW14AX)}
	\newline
	\hbox{}\hfill\hbox{(http://www.volkstheater.at/home/spielplan/1494/Spiel\%5C\%27s$+$nochmal\%2C$+$Sam)}
\end{exe}							          
		   
\begin{exe}
	\ex\label{981}
	\scriptsize
	Für den neuen Handwerkskammerpräsidenten Hans Peter Wollseifer sind die Themen Aus- und Wei\-terbildung von besonderer Bedeutung.
	\textbf{\textit{Geht} es \underline{doch} darum, die jungen Menschen mit einer bestmöglichen Ausbildung auszustatten}, um Sie für das Berufsleben 			optimal auszurüsten.
	\hfill\hbox{(DECOW14AX)}
	\newline
	\hbox{}\hfill\hbox{(http://www.kunststoff-magazin.de/epaper/km091010/km\_091010/}
	\newline
	\hbox{}\hfill\hbox{blaetterkatalog/blaetterkatalog/html/wirtschaftsminister\_guntram\_schn.html)}
\end{exe}
Legt man eine \is{Aboutness-Topik} Aboutness-Topik-Auffassung (im Sinne von (\ref{982})) mit dem Topik als Satzgegenstand, über den eine Aussage gemacht wird, zugrunde, kommen die Subjekte hier nicht als Topiks in Frage.

\begin{exe}
	\ex\label{982} 
		\begin{xlist}	
			\ex\label{982a} What about x?
			\hfill\hbox {\citet[32]{Gundel1977}}
			\ex\label{982b} Ich sage dir über NP, daß S.
			\hfill\hbox {\citet[68-69]{Sgall1974}}
			\ex\label{982c} Er sagt über x, dass ...
			\hfill\hbox {\citet[65]{Reinhart1981}}
		\end{xlist}
\end{exe}
Die entscheidende Anforderung an einen potenziellen Topik-Referenten \is{Topik-Referent} ist, dass er referieren kann (vgl. z.B. \citealt{Reinhart1981}, \citealt[331]{Frey2007}). Aus diesem Grund eignen sich die Vorfeldkonstituenten in (\ref{983}) beispielsweise nicht als Topik.

\begin{exe}
	\ex\label{983} 
		\begin{xlist}	
			\ex\label{983a} \textbf{\textit{Keiner}} mag den Film.
			\ex\label{983b} \textbf{\textit{Leider}} hat Paul verschlafen.
			\ex\label{983c} \textbf{\textit{Wer}} hat Maria heute gesehen?
			\hfill\hbox {\citet[331]{Frey2007}}
		\end{xlist}
\end{exe}
Die Sätze können nicht als Aussage über \textit{keinen}, \textit{leider} oder \textit{wer} interpretiert werden. In Sätzen der Art in (\ref{979}) bis (\ref{981}) kommt \textit{es} folglich ohne Zweifel nicht als Topik in Frage. Für diese Sätze gilt somit, dass das Subjekt nie die Funktion des Topiks übernimmt.

Konkret befinden sich unter den 698 \textit{doch}-V1-Sätzen 116 Sätze mit strukturellem \textit{es}-Subjekt. Diese Strukturen machen auf den ersten Blick folglich 17\% der Sätze aus. Bewertungen solcher Verteilungen sind aber – wie ich schon an verschiedenen Stellen gezeigt habe – sehr von dem Erwartungswert für das Vorkommen des besagten Kontextes an sich abhängig. Im vorliegenden Fall ist es notwendig, zu wissen, mit wie vielen \textit{es}-Subjekten innerhalb von Subjekten überhaupt zu rechnen ist. Liegt dieser vor, kann man feststellen, wie das Auftreten des \textit{doch}-V1-Satzes zu diesem Erwartungswert steht.	

In einer Stichprobe von 500 Sätzen mit lexikalischen finiten Verben des glei\-chen Teilkorpus, aus dem die 698 \textit{doch}-V1-Sätze stammen, sind lediglich 35 derartige Subjekte enthalten. Zu erwarten sind non-anaphorische \textit{es}-Subjekte demzufolge nur mit 7\%. (\ref{984}) gibt auch das 95\%-Konfidenzintervall bei dieser Stichprobengröße an.

\begin{exe}
	\ex\label{984}Erwartungswert Verteilung \textit{es}-Subjekte in DECOW2014AX\\[-1em]		
 		\begin{tabular}[t]{|c|c|c|} 
 		\hline 	
   	 	& \textbf{\textit{es}-Subjekt} & \textbf{kein \textit{es}-Subjekt} \\
   	 	\hline 
  		absolute Zählung & 35 & 465\\ 
   		\hline
   		Anteil & \textbf{7\%} & \textbf{93\%}\\
   		\hline
   		95\%-Konfidenzintervall & $[$4,991\% ... 9,693\%$]$ & $[$90,31\% ... 95,01\%$]$ \\
   		\hline
 		\end{tabular}
\end{exe}
Der Unterschied zwischen 116 non-referenziellen \textit{es}-Subjekten und 582 anderen Subjekten stellt sich auf der Basis des ermittelten Erwartungswertes als signifikant heraus ($\chi^{2}$(1, n = 698) = 101,8247, p $<$ 0,001, V = 0,38). Vor dem Hintergrund des Erwartungswertes entspricht der Anteil der non-referenziellen \textit{es}-Subjekte quasi 73\%. Vor der Folie des niedrigeren Erwartungswertes als des Wertes, mit dem die Strukturen in den V1-Sätzen auftreten, kommen diese Subjekte folglich sehr häufig vor. Es tauchen somit überwiegend Subjekte auf, die sich nicht als Topik eignen.

Die non-referenziellen \textit{es}-Vorkommen in Subjektfunktion sind vor dem Hintergrund typologisch interessierter Arbeiten wie z.B. \citet{Askedal1990} und \citet{Zitterbart2002} nicht alle als gleich einzustufen. \citet{Speyer2009}, dessen Dreiteilung mir für meine Argumentation ausreichend erscheint, unterscheidet die drei Typen, die durch (\ref{985}) bis (\ref{987}) illustriert werden. \is{Vorfeld-es} \is{expletives es} \is{Korrelat-es}

\begin{exe}
	\ex\label{985} 
	\textbf{\textit{Es}} folgten mit weitem Abstand Uller und Karl Zaible.
	\hfill\hbox {(Vorfeld-\textit{es})}
\end{exe}
\vspace{-0.65cm}
\begin{exe}
	\ex\label{986} 
	\textit{\textbf{E}s} handelt sich um Ullers Verhältnis zu Rathkolb.
	\hfill\hbox {(Expletives \textit{es})}
\end{exe} 	
\vspace{-0.65cm}
\begin{exe}
	\ex\label{987} 
	\textbf{\textit{Es}} macht richtig Spaß, Kajak zu fahren.
	\hfill\hbox {(Korrelat-\textit{es})}
	\newline
	\hbox{}\hfill\hbox{\citet[324/325/326]{Speyer2009}}
\end{exe} 
Das expletive \textit{es} tritt als strukturelles Subjekt von Verben auf, die kein logisches Subjekt aufweisen. Im Gegensatz zum Vorfeld-\textit{es} ist es nicht auf diese Position beschränkt. Das Korrelat-\textit{es} ist in meinen Fällen mit einem Subjektsatz bzw. einer Verbativergänzung im Nachfeld koindiziert.

Unter den 116 non-referenziellen \textit{es}-Subjekten sind 65 expletive \textit{es} und 51 Korrelatauftreten. In dem Erwartungswert gibt es nur ein Vorkommen als Korrelat. Transparenter ist es m.E. deshalb, von einem Erwartungswert von 7\% für die expletiven \textit{es} auszugehen. Die Korrelate in struktureller Subjektposition/-funktion scheinen generell noch viel seltener zu sein. Nur 0,2\% der finiten Verben der 500er Stichprobe weisen ein solches Subjekt auf. Die 51 Vorkommen in den Daten sind somit extrem auffällig. Schließt man die Korrelate aus der Kalkulation aus, ergibt sich für die Verteilung des expletiven \textit{es} (65:582) nach wie vor ein signifikanter Unterschied ($\chi^{2}$(1, n = 647) = 9,2233, p $<$ 0,05, V = 0,12). Der Effekt ist allerdings schwach.

Die Darstellung konzentriert sich im Folgenden auf die expletiven \textit{es}. Die 65 Belege werden weitestgehend erfasst durch das Auftreten der Prädikate \textit{handeln um} (12x), \textit{geben} (30x) und \textit{gehen um} (16x) (vgl. (\ref{988}) bis (\ref{990})). Hierbei handelt es sich um ein \glqq  allgemeines Existentialprädikat\grqq{} und zwei \glqq allgemeine kommunikative Themaprädikate\grqq{} (\citealt[218]{Askedal1990} zu \textit{geben} und \textit{handeln um}).\footnote{Ferner treten auf: \textit{geben als}, \textit{ankommen auf}, \textit{brauchen}, \textit{stehen}, \textit{laufen}, \textit{fehlen} und \textit{gehen mit}.}

\begin{exe}
	\ex\label{988} 
	\scriptsize
	Auch die Nachricht, die wir an diesem Samstagabend auf die Ticker gegeben haben, ist rechtlich nicht zu beanstanden. \textbf{\textit{Handelt} es sich 		\underline{doch} um eine objektive Tatsache:} Die Staatsanwaltschaft Hamburg ermittelt gegen Gregor Gysi – und zwar mit ausdrücklicher Billigung des 		Immunitätsausschusses des Deutschen Bundestags, der am 31. Januar über den Fall beraten hat. 
	\hfill\hbox {(DECOW14AX)}
	\newline
	\hbox{}\hfill\hbox{(http://investigativ.welt.de/2013/02/10/gregor-gysi-und-die-wahrheit/)}
\end{exe} 

\begin{exe}
	\ex\label{989} 
	\scriptsize
	Auch stellen die Autoren zu wenig ihren eigenen Ansatz in Frage. \textbf{\textit{Gibt} es \underline{doch} schon lange Jahre die Diskussion}, ob eine 		Vorklassifizierung und –strukturierung von Information oder der effektive und intelligente Einsatz von Suchwerkzeugen erfolgversprechender sind.
	\hfill\hbox {(DECOW14AX)}
	\newline
	\hbox{}\hfill\hbox{(http://www.b-i-t-online.de/archiv/1999-04/rezen6.htm)}
\end{exe} 							 

\begin{exe}
	\ex\label{990} 
	\scriptsize
	Axel Köhler-Schnura vom Vorstand der CBG und einer ihrer Gründer ergänzt: \glqq Das Verfahren hat grundsätzliche Bedeutung und wird bundesweit mit 			Aufmerksamkeit verfolgt. \textbf{\textit{Geht} es \underline{doch} um die im Rahmen von Deregulierung und entfesseltem Kapitalismus überall zunehmende 		Unterwerfung von Forschung und Lehre unter wirtschaftliche Interessen und Konzernprofite.}\grqq{}
	\hfill\hbox {(DECOW14AX)}
	\newline
	\hbox{}\hfill\hbox{(http://www.nrhz.de/flyer/beitrag.php?id=18513)}
\end{exe}						                                      
Derartige Strukturen lassen sich alle als \textit{präsentative Konstruktionen} \is{Präsentativkonstruktion} einordnen. \textit{Es gibt x.} stellt als \textit{Existentialkonstruktion} \is{Existentialkonstruktion} einen Unterfall der Präsentativkonstruktion dar. Sie dient dem Einführen von Referenten (wenn auch nicht notwendigerweise – wie der Name suggeriert – um die Existenz an sich zu behaupten, sondern um einen Referenten in die Szene einzuführen $[$vgl. \citealt[179]{Lambrecht1994}$]$). Auch Strukturen, die durch \textit{es handelt sich um} oder \textit{es geht um} eingeleitet werden, ordne ich den präsentativen Konstruktionen zu. Sie führen auch gerade einen Gegenstandsbereich, ein Thema oder auch Referenten ein, über das/die in der Folge plausiblerweise weiter berichtet wird. Präsentative Strukturen gelten als Konstruktionen, die Topiks einführen. Dazu werden sie als thetisch \is{thetisch} behandelt, weisen also auch Vollfokus \is{vollfokussiert} auf (vgl. \citealt[144, 177]{Lambrecht1994}). Das Subjekt eignet sich folglich nicht als Topik (was nach \citealt[144-145]{Lambrecht1994} auch die definierende Eigenschaft thetischer Sätze ist). Dies bedeutet aber nicht, dass nicht andere Einheiten diese Funktion innehaben können. In Bezug auf die Sätze mit expletivem \textit{es} stellt sich somit die Frage, ob man andere Topiks in den Sätzen ausmachen kann. Meiner Meinung nach gibt es in ca. einem Drittel der Fälle Konstituenten, die sich prinzipiell eignen, wobei ich es auch in diesen Fällen für fraglich halte, dass sie wirklich als Topik fungieren.

In (\ref{991}) bis (\ref{995}) sind mit allquantifizierten NPs, Eigennamen, Pronomen und generischen NPs Einheiten in den Sätzen enthalten, die als Topik in Frage kommen. Man kann sich vorstellen, dass über diese Einheiten eine Aussage gemacht wird. Im Sinne der Vorstellung (wie in \citealt{Reinhart1981} vertreten), dass ein Topik einer Karteikarte entspricht, auf der durch den Kommentar Einträge gemacht werden, ist vertretbar, dass es für diese Einheiten eine Karte gibt.

\begin{exe}
	\ex\label{991} 
	\scriptsize
	Es stellt sich als schwierig heraus eine festgelegte Zielgruppe für dieses Buch zu definieren. \textbf{\textit{Gibt} es \underline{doch} \textsc{in 		allen Fachbereichen}, die sich mit dem Personenkreis von Menschen mit Behinderung befassen, solche die sich philosophischen Gedankengängen eher öffnen 		als andere.}
	\hfill\hbox {(DECOW14AX)}
	\newline
	\hbox{}\hfill\hbox{(http://www.socialnet.de/rezensionen/8130.php)}
\end{exe}

\begin{exe}
	\ex\label{992} 
	\scriptsize
	Sie versteht sich zugleich als Forum für eine geschlechtergerechtere Gesellschaft. \glqq Nicht gegen die Männer können wir uns emanzipieren, sondern 		nur in Auseinandersetzung mit ihnen. \textbf{\textit{Geht} es \textsc{uns} \underline{doch} um die Loslösung von den alten Geschlechterrollen}, um die 		menschliche Emanzipation überhaupt.	\grqq{} 	
	\hfill\hbox {(DECOW14AX)}
	\newline
	\hbox{}\hfill\hbox{(http://www.gemeinsamlernen.de/laufend/geschlechterrollen/}
	\newline
	\hbox{}\hfill\hbox{literatur70/gutenmorgen/margit\_kurz.htm)}
\end{exe}

\begin{exe}
	\ex\label{993} 
	\scriptsize
	Es gibt aber auch sehr viele Menschen, die diesem Hype sehr kritisch gegenüber stehen. \textbf{\textit{Handelt} es sich \underline{doch} \textsc{bei 		Schönheits-Operationen} um einen sehr persönlichen, medizinischen Eingriff}, der durchaus mit Gesundheitsrisiken behaftet ist.  		
	\hfill\hbox {(DECOW14AX)}
	\newline
	\hbox{}\hfill\hbox{(http://www.abazo-plastische-chirurgie.de/)}
\end{exe}							               
						                			                      
\begin{exe}
	\ex\label{995} 
	\scriptsize
	Hier wird Improvisieren zum Abenteuer jenseits von Konzepten und Klischees. \textbf{\textit{Geht} es \textsc{Thomas Ruëckert} doch keineswegs um 			vordergründige Virtuosität}, sondern um das Ausloten neuer (energetisch pulsierender) Klangdimensionen.		 		
	\hfill\hbox {(DECOW14AX)}
	\newline
	\hbox{}\hfill\hbox{(http://www.altes-pfandhaus.de/veranstaltungen/archiv.html)}
\end{exe}								                       						  	
In manchen Sätzen finden sich auch spezifische Adverbiale, die prinzipiell Topik sein können (vgl. (\ref{996}) und (\ref{997})), im Gegensatz zu unspezifischen Angaben wie in (\ref{998}) und (\ref{999}).

\begin{exe}
	\ex\label{996} 
	\scriptsize
	Wie schon beim Perikles wird auch hier wieder eine hochgesteigerte Schönheit verschmolzen mit persönlichen Zügen. \textbf{\textit{Gab} es 					\underline{doch} \textsc{in Griechenland} immer Kunstrichter}, die es rühmten, wenn die Künstler edle Männer noch edler darstellten.	 		
	\hfill\hbox {(DECOW14AX)}
	\newline
	\hbox{}\hfill\hbox{(http://www.lexikus.de/bibliothek/Antike-Portraets/Bemerkungen-}
	\newline
	\hbox{}\hfill\hbox{zu-den-Tafeln/Griechische-und-Roemische-Portraets)}
\end{exe}

\begin{exe}
	\ex\label{997} 
	\scriptsize
	Doch keinem der Indios war es im ersten Eifer der Eroberung und im kindlichen Besitzerglück eingefallen, einen Streifen Neuland als festen und 				dauernden Familienbesitz abzugrenzen. \textbf{\textit{Gab} es \underline{doch} \textsc{rings um das Geviert des ersten Rodungsbrandes} noch genug 			freies Land}, um auch im nächsten und übernächsten Jahr nochmals einen Flecken Urwald abzuholzen und urbar zu machen.	
	\newline 		
	\hbox{}\hfill\hbox {(DECOW14AX)}
	\newline
	\hbox{}\hfill\hbox{(http://www.payer.de/bolivien2/bolivien0222.htm)}
\end{exe}	
	
\begin{exe}
	\ex\label{998} 
	\scriptsize
	Auch stellen die Autoren zu wenig ihren eigenen Ansatz in Frage. \textbf{\textit{Gibt} es \underline{doch} \textsc{schon lange Jahre} die Diskussion}, 		ob eine Vorklassifizierung und 	–strukturierung von Information oder der effektive und intelligente Einsatz von Suchwerkzeugen erfolgversprechender 		sind.	 		
	\hfill\hbox {(DECOW14AX)}
	\newline
	\hbox{}\hfill\hbox{(http://www.b-i-t-online.de/archiv/1999-04/rezen6.htm)}
\end{exe}								                  

\begin{exe}
	\ex\label{999} 
	\scriptsize
	P.S. Den Bassdruck wirst du nur mit großen Resonanzkörper erzeugen können. \textbf{\textit{Gab} es \underline{doch} \textsc{vor kurzem} einen 				gehörlosen der zur 	Musik bei einer dieser Talentshows getanzt hat} ... Wir erinnern uns? 	 		
	\hfill\hbox {(DECOW14AX)}
	\newline
	\hbox{}\hfill\hbox{(http://www.mtb-news.de/forum/archive/index.php?t-484264.html)}
\end{exe}				                    
Für meine Begriffe interpretiert man die Adverbiale in (\ref{996}) und (\ref{997}) allerdings nicht als Satzgegenstand. Die Sätze beabsichtigen keine Aussage über Griechenland oder das genannte Gebiet. In 13 Sätzen kommen Ausdrücke vor, die prinzi\-piell als Topik \is{Topik} in Frage kommen. In 10 weiteren Fällen findet man adverbiale Situations- oder Umweltbezüge, die z.T. auch konkreter lokal oder temporal sind. (\ref{1000}) und (\ref{1001}) dienen der Illustration.

\begin{exe}
	\ex\label{1000} 
	\scriptsize
	So gesehen, könnte den beiden Lindenau-Grafen eigentlich der Titel \glqq Tierarzt ehrenhalber\grqq{} zugeordnet und ihr Wiegenort Machern als eine Art 		\glqq Wallfahrtsort\grqq{} für Tierärzte und Pferdezüchter gepriesen werden. \textbf{\textit{Gibt} es \underline{doch} \textsc{hier} viele Andenken an 		beide:} das Schloss mit der Ritterstube, die drei Grafen-Wappen und die Lindenau-Ausstellung wie auch den Landschaftsgarten mit seinen Parkbauten und 		Plastiken. 		
	\hfill\hbox {(DECOW14AX)}
	\newline
	\hbox{}\hfill\hbox{(http://www.uni-leipzig.de/\~{}mielke/MachernH/vetarzte.htm)}
\end{exe}

\begin{exe}
	\ex\label{1001} 
	\scriptsize
	Robbins-Fans werden sich über dieses neue Buch freuen. Nicht so unsere Wende-Politiker. \textbf{\textit{Handelt} es sich \underline{doch} hierbei um ein einziges 				enthusiastisches Plädoyer für den Terrorismus:} \glqq Die Terroristen sind die Büchsenöffner im Supermarkt des Lebens\grqq{}.		
	\hfill\hbox {(DECOW14AX)}
	\newline
	\hbox{}\hfill\hbox{(http://blogs.taz.de/hausmeisterblog/2007/01/page/5/)}
\end{exe}
Ich frage mich, ob diese Einheiten nicht eher die Funktion haben, einzuordnen, in welche Szene der Referent/das Thema eingeführt wird, als das Prädikat mit seinen Argumenten auf einen Satzgegenstand zu beziehen. Dies gilt deutlich für Fälle der Art in (\ref{1000}) und (\ref{1001}), aber auch für (\ref{996}) und (\ref{997}) und z.T. auch für Beispiele der Art in  (\ref{991}) bis (\ref{995}). Mit Ausnahme von fünf Fällen (wobei in zweien grammatische Gründe vorliegen) könnten die \glq Topikkonstituenten\grq {} auch weggelassen werden, ohne dass die Sätze anders interpretiert würden (vgl. z.B. (\ref{995}) und (\ref{1002}) vs. (\ref{1004})).

\begin{exe}
	\ex\label{1002} 
	\scriptsize
	Sogar Diesel durften jetzt sportlich sein, wie der Ibiza Cupra zum Modelljahr 2004 	zeigte. \textbf{\textit{Gab} es \textsc{ihn} \underline{doch} 			erstmals als 160 PS starken 1,9-Liter-TDI.} 		
	\hfill\hbox {(DECOW14AX)}
	\newline
	\hbox{}\hfill\hbox{(http://ww2.autoscout24.at/test/renault-zoe/so-wird-das-}
	\newline
	\hbox{}\hfill\hbox{nichts-frau-merkel/4319/418106/mrb-mz\_home?article=}
	\newline
	\hbox{}\hfill\hbox{420574\%26intcidm=mrb-mz\_home)}	
\end{exe}	
 		 						                            
\begin{exe}
	\ex\label{1004} 
	\scriptsize
	Aber so richtig neu ist Irish Tour '74 natürlich nicht. \textbf{\textit{Handelt} es sich (\textsc{hierbei}) \underline{doch} um den be\-kannten Tourfilm 		zum gleichnamigen klassischem Livealbum}, welches Rorys Ruf als außergewöhnlichen Musiker für immer zementierte.      		
	\hfill\hbox {(DECOW14AX)}
	\newline
	\hbox{}\hfill\hbox{(http://www.musikansich.de/ausgaben/0511/reviews/rory\_gallagher.html)}
\end{exe}								                      
Dieses Verhältnis ist nicht darauf zurückzuführen, dass die potenziellen Topiks weitestgehend Adverbiale \is{Adverbial} sind. Es gibt andere Adverbiale, die als Topik fungieren können und die nicht ohne Veränderung der Interpretation auslassbar sind. Dies gilt für rahmensetzende Ausdrücke wie in (\ref{1005}).

\begin{exe}
	\ex\label{1005} 
		\begin{xlist}	
			\ex\label{1005a} \textbf{\textit{Finanziell}} hat er keine Probleme.
			\ex\label{1005b} \textbf{\textit{Gesundheitlich}} hat er keine Probleme.
			\ex\label{1005c} \textbf{\textit{Privat}} hat er keine Probleme.
			\hfill\hbox {\citet[42]{Helbig1981}}
		\end{xlist}
\end{exe}
Wenngleich in ca. einem Drittel der expletiven \textit{es}-Subjekte Einheiten auftreten, die aufgrund ihrer referenziellen Eigenschaften als Topiks fungieren können, bezweifle ich, dass sie diese Aufgabe in den V1-Sätzen auch tatsächlich überneh\-men. Selbst mit den auftretenden referierenden Ausdrücken gelingen Tests, die dem Nachweis des Topikstatus dienen (vgl. \citealt[28-29]{Musan2010}), nur bedingt (vgl. (\ref{1006})).

\begin{exe}
	\ex\label{1006} 
		\begin{xlist}	
			\ex\label{1006a} ?Ich erzähl dir was über \textbf{\textit{Schönheits-OPs}}. Es handelt sich um einen persönlichen Eingriff.
			\ex\label{1006b} ?Hast du das über \textbf{\textit{Schönheits-OPs}} schon gehört? Es handelt sich um...
			\ex\label{1006c} ?Was \textbf{\textit{Schönheits-OPs}} betrifft, so handelt es sich um...
			\ex\label{1006d} ?Es sind \textbf{\textit{Schönheits-OPs}}, bei denen es sich um einen persönlichen Eingriff handelt.
		\end{xlist}
\end{exe}
Den Bezug auf einen Satzgegenstand halte ich auch daher für schwierig, weil die beteiligten Prädikate sehr abstrakte Inhalte vermitteln, d.h. lexikalisch recht blass sind. \citet[68]{Reinhart1981} weist unter Bezug auf \citet{Kuno1972} darauf hin, dass sich Entitäten besser als Topiks eignen, wenn sie deutlich affiziert werden.

Man täte besser daran, die Topiks in den V1-Sätzen als \textit{Diskurs}- und \is{Diskurstopik} nicht als \textit{Satztopiks} \is{Satztopik} aufzufassen. Die Sätze werden als Information über einen vorerwähn\-ten Film, Operation, einen Ort etc. gewertet, sie unterliegen selbst aber in der Regel keiner Topik-Kommentar-Gliederung. Diskurstopiks hat man auch in den Fällen, in denen weder das Subjekt Topik ist (weil es dies kategorisch hier nicht sein kann) noch eine (adverbiale) Einheit diese Funktion übernimmt. In (\ref{1007}) und (\ref{1008}) könnte man Ausdrücke einbauen wie \textit{dort}, \textit{dabei}, \textit{in dieser Diskussion}, \textit{hierbei} oder \textit{bei dieser Nachricht}. Man liest solche Einheiten mit, weil die Informationen, die die Sätze vermitteln, natürlich einen Bezug benötigen.

\begin{exe}
	\ex\label{1007} 
	\scriptsize
	Grau oder Sandsteinfarben? So lautete beim Meinungsaustausch der Immobilien- und Standortgemeinschaft (ISG) am Samstagmorgen in der Gaststätte \glqq 		Alt Ochtrup\grqq{} die Frage aller Fragen. \textbf{\textit{Ging} es \underline{doch} um das künftige Bild der Innenstadt} – genauer die Pflasterung.   		\hfill\hbox {(DECOW14AX)}
	\newline
	\hbox{}\hfill\hbox{(http://www.azonline.de/Muensterland/Kreis-Steinfurt/Ochtrup/(offset)/225)}
\end{exe}	

\begin{exe}
	\ex\label{1008} 
	\scriptsize
	Auch die Nachricht, die wir an diesem Samstagabend auf die Ticker gegeben haben, ist rechtlich nicht zu beanstanden. \textbf{\textit{Handelt} es sich 		\underline{doch} um eine objektive Tatsache:} Die Staatsanwaltschaft Hamburg ermittelt gegen Gregor Gysi – und zwar mit ausdrücklicher Billigung des 		Immunitätsaussschusses des Deutschen Bundestags, der am 31. Januar über den Fall beraten hat.    		
	\hfill\hbox {(DECOW14AX)}
	\newline
	\hbox{}\hfill\hbox{(http://investigativ.welt.de/2013/02/10/gregor-gysi-und-die-wahrheit/)}
\end{exe}	
Ich fasse diese Einheiten nicht als Satztopik \is{Satztopik} auf. Man hat es m.E. mit Diskurstopiks \is{Diskurstopik} zu tun. Ein solches müssen sie haben. Auch wenn die Sätze vollkommentarisch sind, steuern sie natürlich zu irgendeiner Sache eine Information bei. Da die \textit{doch}-V1-Sätze nie diskursinitial auftreten (Sie begründen schließlich stets eine Annahme.), kann man davon ausgehen, dass das Diskurstopik stets bereits etabliert ist. Aus meiner Annahme ergibt sich, dass das Diskurstopik im Satz repräsentiert werden kann, es aber nicht realisiert sein muss. Ich halte dies nicht für unplausibel, da es Sätze gibt, die ein anderes Satz- als Diskurstopik aufweisen. Der Paragraph in (\ref{1009}) kann als Ganzes als Information über die Stadt Kamp-Lintfort gewertet werden. Dennoch können (neben dem Diskurstopik) auch andere Einheiten als Satztopik fungieren.

\begin{exe}
	\ex\label{1009} 
	\scriptsize
	Ich erzähle dir mal was von meiner Heimatstadt: \textbf{\textit{Kamp-Lintfort}} liegt nördlich von Düsseldorf am linken Niederrhein. 						\textbf{\textit{Die Stadt}} liegt acht Kilometer von Moers entfernt und sechs Kilometer von Rheinberg. \textbf{\textit{Sie}} gliedert sich in elf 			Stadtteile. Über die Stadtgrenze hinaus ist \textbf{\textit{Kamp-Lintfort}} bekannt für das Kloster Kamp. \textbf{\textit{Dieses Zisterzienserkloster}} 	wurde 1123 gegründet. Bekannt ist \textbf{\textit{Kamp-Lintfort}} auch aus dem Radio. \textbf{\textit{Das Kreuz Kamp-Lintfort}} ist anfällig für 			Staus. Seit 2014 hat \textbf{\textit{die Stadt}} zudem eine Hochschule. \textbf{\textit{Die Hochschule Rhein-Waal}} ist mit der Fakultät \glqq 				Kommunikation und Umwelt\grqq{} ansässig.
\end{exe}
Wenn das Diskurstopik in den V1-Sätzen realisiert ist, ist eine vorerwähnte Konstituente enthalten. \citet{Oennerfors1997} (s.o.) hält gerade dies für möglich und er nimmt deshalb nicht an, dass die Sätze vollfokussiert oder rhematisch sind. Trotz vorerwähnten und damit bekannten Einheiten kodiert der Gesamtsatz den Kommentar. 

Eine Konstellation wie ich sie für die \textit{doch}-V1-Sätze mit expletivem \textit{es} als strukturellem Subjekt ansetze, liegt in (\ref{1010}) unter Beteiligung eines analogen V2-Satzes vor.

\begin{exe}
	\ex\label{1010} 
	\scriptsize
	\textbf{\textit{Bielefeld}} hat 330000 Einwohner. \textbf{\textit{Die Stadt}} liegt eine Stunde von Hannover entfernt an der ICE-Strecke Köln-Berlin. 		\textbf{\textit{Bielefeld}} ist die größte Stadt in der Region Ostwestfalen-Lippe. \textbf{Es gibt \underline{in Bielefeld} \textit{14 Gymnasien}.} 		\textbf{\textit{Sie}} verteilen sich auf die 10 Stadtteile.
\end{exe}	
(\ref{1010}) kann so interpretiert werden, dass der globale Diskurs Bielefeld als Thema hat. Der \textit{es gibt}-Satz ist thetisch \is{thetisch} und hat damit kein \is{Satztopik} Satztopik. Er dient der Einführung der Gymnasien. Dennoch ist das Diskurstopik auch in diesem Satz \textit{Bielefeld}. Es wird hier versprachlicht, könnte aber genauso gut ausgelassen werden. An der Interpretation, dass dies eine weitere Information über Bielefeld ist, würde sich dann nichts ändern.\footnote{Eine andere Strategie, die in der Literatur verfolgt wird, um thetischen Sätzen nicht jeglichen Bezugspunkt abzusprechen, ist, dass für diese Sätze eine Art von Situationstopik angenommen wird. \citet[16]{Erteschik-Shir2007} geht z.B. von so genannten \textit{stage topics} aus, die die spatio-temporalen Parameter (das Hier und Jetzt) des Diskurses bezeichnen (vgl. auch \citealt{Back1995} zur \textit{diffusen Deixis}). Diese Modellierungen \glqq retten\grqq{} die Ansicht, dass jeder Satz ein Topik aufweist. Über ein derartiges stage topic verfügt dann aber auch jeder Satz (ggf. zusätzlich zu seinem \glq eigentlichen\grq{} Topik), weil jeder Satz auch in Raum und Zeit verankert ist. Das Problem, dass die Sätze nicht wirklich einen Satzgegenstand beinhalten, löst man meiner Meinung nach auf diesem Wege nicht.}

Neben den 65 Sätzen mit expletivem \textit{es} treten 51 strukturelle \textit{es}-Subjekte auf, die als Korrelat \is{es-Korrelat} fungieren. Da unter den 500 lexikalischen finiten Verben der Stichprobe nur eines mit einem Subjekt-Korrelat auftritt, ist für diese Strukturen generell eine extrem niedrige Auftretenswahrscheinlichkeit anzunehmen. Umso bedeutsamer ist ihr Vorkommen in den \textit{doch}-V1-Strukturen. (\ref{1011}) bis (\ref{1013}) zeigen einige Beispiele.
	
\begin{exe}
	\ex\label{1011} 
	\scriptsize
	Vor mehr als 100 Jahren war man auch im Waldeck-Frankenberger Land mit dem Eisenbahnbau fieberhaft beschäftigt. \textbf{\textit{Galt} es 					\underline{doch}, mit dem Anschluss an die 1850 eröffnete Main-Weser-Bahn zwi\-schen Kassel, Marburg und Frankfurt/ Main dem schleichenden Niedergang der 	ohnehin spärlichen Wirtschaft und der damit verbundenen Verarmung der Bevölkerung in unserer Region entgegenzuwirken.  	}	  		
	\hfill\hbox {(DECOW14AX)}
	\newline
	\hbox{}\hfill\hbox{(http://regiowiki.hna.de/Frankenberg)}
\end{exe}	
	
\begin{exe}
	\ex\label{1012} 
	\scriptsize
	Sie sollten wissen, dass wir sie in das Verfahren mit einbeziehen werden, denn sie als Tarifpartner sind für uns Mitbeteiligter. \textbf{\textit{Geht} 		es \underline{doch} auch darum, nachzuweisen, dass wir richtige Verhandlungen geführt haben.  }	  		
	\hfill\hbox {(DECOW14AX)}
	\newline
	\hbox{}\hfill\hbox{(http://www.judicialis.de/Th\%C3\%BCringer-}
	\newline
	\hbox{}\hfill\hbox{Landesarbeitsgericht\_2-BV-3-00\_Beschluss\_17.10.2002.html)}
\end{exe}	
										           
\begin{exe}
	\ex\label{1013} 
	\scriptsize
	Bascha Mika wirft den Frauen in ihrem Buch \glqq Die Feigheit der Frauen\grqq{} vor, sie würden sich freiwillig den uralten Rollenbildern unterwerfen. 		\textbf{\textit{Scheint} es \underline{doch} so verlockend, das bequeme Leben zu wählen: einen Ernährer suchen, Kinder bekommen, daheim bleiben.} 		
	\hfill\hbox {(DECOW14AX)}
	\newline
	\hbox{}\hfill\hbox{(http://www.christundwelt.de/themen/detail/artikel/unsere-preussische-kinderstube/)}
\end{exe}	
Anders als in anderen Haupt-/Nebensatzstrukturen, die sich ebenfalls unter den V1-Sätzen befinden (vgl. \textit{der Vorwurf} und \textit{er} in (\ref{1014}) und (\ref{1015})), gibt es im Hauptsatz kein Subjekt, das als Topik herhalten könnte. 

\begin{exe}
	\ex\label{1014} 
	\scriptsize
	Nun hat Magnus-Essay-Enzensberger bereits 1957 dem deutschen Nachrichtenmagazin vorgeworfen, die Nachricht der Story zu opfern. Ein schrecklicher Befund, träfe er den Sigmaringer Volksboten oder den Zillertaler Almdudler. \textbf{\textit{Besagt} \textsc{der Vorwurf} \underline{doch} nichts anderes, als dass die Primärtugenden der Journalisten-Schule, \glqq wer, was, wann, wo, warum, wie\grqq{} zu fragen, darauf wahrheitsgemäß und konzise zu beantworten, der spannenden Story im Stahlnetzfieber des Lesers geopfert werden.		}
	\newline
	\hbox{}\hfill\hbox {(DECOW14AX)}
	\newline
	\hbox{}\hfill\hbox{(http://www.goedartpalm.de/spie.html)}
\end{exe}	

\begin{exe}
	\ex\label{1015} 
	\scriptsize
	Wer Paul Gerhard kennt, den überrascht das nicht. \textbf{\textit{Sagt} er \underline{doch} in dem Lied \glqq Befiehl du deine Wege\grqq{}:} Mit Sorgen 	und mit Grämen und mit selbsteigner Pein lässt Gott sich gar nichts nehmen, es muss erbeten sein. 		
	\hfill\hbox {(DECOW14AX)}
	\newline
	\hbox{}\hfill\hbox{(http://www.predigtpreis.de/predigtdatenbank/newsletter/}
	\newline
	\hbox{}\hfill\hbox{article/liedpredigt-zu-dem-lied-wie-soll-ich-dich-empfangen-eg-11.html)}	
\end{exe}									         
Ähnlich wie bei den expletiven \textit{es} weist ca. ein gutes Drittel der Sätze (19x) ein anderes potenzielles Topik auf, d.h. eine Phrase, die das \is{Topik} Topik sein könnte. Ob man diese Einheiten tatsächlich als das Topik der Sätze liest, ist wiederum eine andere Frage. In den meisten Fällen scheint mir auch hier eher eine situative Einordnung vorgenommen zu werden und weniger einem Referenten eine Eigenschaft zugeschrieben zu werden (vgl. (\ref{1016}) und (\ref{1017})).					                     
		
\begin{exe}
	\ex\label{1016} 
	\scriptsize
	Die scheibchensweise und nebulöse Informationspolitik der Bürgermeisterin Pfordt geht ungehindert für die Öffentlichkeit weiter. Von Transparenz kann weiter keine Rede sein. \textbf{\textit{Hieß} es \underline{doch} noch \textsc{im Artikel des KSTA v. 16.08.2006}:} Hamacher bestätigte, dass das Grundstück inzwischen der Ehefrau eines ehemaligen Rista-Gesellschafters gehöre, der gleichzeitig auch Liquidator des Grundstückes sei. 			
	\newline
	\hbox{}\hfill\hbox {(DECOW14AX)}
	\newline
	\hbox{}\hfill\hbox{(http://www.glessen-gazette.de/2009\_05\_05\_spielplatz.htm)}	
\end{exe}		

\begin{exe}
	\ex\label{1017} 
	\scriptsize
	Es ist eine der schönsten Aufgaben, die man als Vertreterin der Landesregierung wahrnehmen kann.
	\textbf{\textit{Gilt} es \underline{doch} \textsc{bei einer solchen Gelegenheit} Menschen zu würdigen, die sich in besonderem Maß für die Gemeinschaft 		eingesetzt haben.}				
	\hfill\hbox {(DECOW14AX)}
	\newline
	\hbox{}\hfill\hbox{(http://www.melinaev.de/blog/tags....Inzest/?view=30)}	
\end{exe}						   
Ob der Auftretensanteil dieser Einheiten wirklich bedeutet, dass eher selten überhaupt Einheiten auftreten, die Topiks sein können, lässt sich erst beurteilen, wenn ein Richtwert zum Auftreten solchen Materials in derartigen Sätzen, die keine begründenden V1-Sätze sind, vorliegt.

Dass auch in den Sätzen mit Korrelat-\textit{es} zusätzliche Elemente auftreten, die in der Regel eine situative Verankerung vornehmen, ist ebenfalls dem Umstand zuzuschreiben, dass die Aussagen verankert werden müssen. Ein durch \textit{es heißt} vermittelter Inhalt (vgl. (\ref{1018}) und (\ref{1019})) wird natürlich jemandem oder einem Schriftstück zugeschrieben bzw. in Bezug auf irgendetwas/einen Zeitpunkt verstanden. Der Inhalt kann nicht im luftleeren Raum stehen. Das, was hier ggf. angeführt wird, kann weggelassen werden, oder wird – wenn es nicht auftritt – mitgedacht.

\begin{exe}
	\ex\label{1018} 
	\scriptsize
	Zudem ist noch anzumerken, dass der Scheck ein Sparkassenscheck war – also von der Konkurrenz\-veranstaltung – was natuerlich die Nicht-Berechnung der 3 		DM Einloesungsgebuehr in einem noch strengeren Licht erscheinen laesst. \textbf{\textit{Heisst} es \underline{doch} \textsc{bei Raiffeisen}} – \glqq 		Einer fuer alle, alle fuer einen\grqq{}. Aber nicht fuer die Sparkasse! 					
	\hfill\hbox {(DECOW14AX)}
	\newline
	\hbox{}\hfill\hbox{(http://www.hoeflichepaparazzi.de/forum/archive/index.php/t-20765.html)}	
\end{exe}

\begin{exe}
	\ex\label{1019} 
	\scriptsize
	Trotz aller Fortschritte der Digitaltechnik, trotz neuer Daten- und Tonträgerformate ist die Schallplatte klanglich das Maß aller Dinge. 					\textbf{\textit{Heißt} es \underline{doch} \textsc{in Bezug auf DVD und SACD}:} \glqq Nie war digital so analog wie heute!\grqq{} 	 					
	\hfill\hbox {(DECOW14AX)}
	\newline
	\hbox{}\hfill\hbox{(http://verein365.de/firma/profil/212/analogue-audio-association-ev.html)}	
\end{exe}						         
Genauso gilt immer für irgendetwas, dass es um x geht (vgl. (\ref{1020})).

\begin{exe}
	\ex\label{1020} 
	\scriptsize
	Die Ernährung eines Sportlers mit intensivem oder regelmäßigem Trainingsaufwand ist grundsätzlich von den alltäglichen Essgewohnheiten eines 				Normalverbrauchers zu unterscheiden. \textbf{\textit{Geht} es \underline{doch} \textsc{bei einer gesunden, sinnvollen und vor allem sportlich 				orientierten Ernährung} darum, gewisse persönliche Ziele zu erreichen.}	
	\hfill\hbox {(DECOW14AX)}
	\newline
	\hbox{}\hfill\hbox{(http://www.bodybuilding-ironbody.de/sportlernahrung.html)}	
\end{exe}
Man interpretiert den Satz aber auch nicht anders, wenn diese Einheit nicht genannt wird (vgl. (\ref{1021})).

\begin{exe}
	\ex\label{1021} 
	\scriptsize
	Sie sollten wissen, dass wir sie in das Verfahren mit einbeziehen werden, denn sie als Tarifpartner sind für uns Mitbeteiligter. \textbf{\textit{Geht} 		es \underline{doch} auch darum, nachzuweisen, dass wir richtige Verhandlungen geführt haben.}
	\hfill\hbox {(DECOW14AX)}
	\newline
	\hbox{}\hfill\hbox{(http://www.judicialis.de/Th\%C3\%BCringer-Landesarbeitsgericht\_}
	\newline
	\hbox{}\hfill\hbox{2-BV-3-00\_Beschluss\_17.10.2002.html)}
\end{exe}
Auch für diese Sätze möchte ich deshalb in gleicher Argumentation wie bei den Sätzen mit expletivem \textit{es} vertreten, dass es sich nicht um das Satztopik \is{Satztopik} handelt, wenn eine sprachliche Realisierung des \glq Topiks\grq {} vorliegt, sondern um das Diskurstopik.

Wie bei den expletiven \textit{es} gilt auch für die Korrelatstrukturen, dass die auftretenden Verben eine eher abstrakte und lexikalisch arme Semantik aufweisen: Die Prädikate \textit{gelten} (18x), \textit{heißen} (17x) $[$1x im Sinne von \textit{gelten}$]$ und um \textit{etw. gehen} (7x) machen den Großteil der Verben aus. Dazu kommen \textit{scheinen} (2x), \textit{gelingen} (2x), \textit{liegen}, \textit{geschehen}, \textit{auf etw. hinauslaufen}, \textit{an etw. kratzen} und \textit{aussehen}. Dass diese semantisch blassen Verben vorkommen, spricht für mich dafür, dass große Teile des Satzes mitteilungswürdig sind. Hierzu passt auch die Vorstellung aus \citet[153]{Zitterbart2002}, die die Funktion des Korrelats als \glqq Progressionsindikator\grqq{} und \glqq Rhemaexponent für den extraponierten Nebensatz\grqq{} ansieht. Das Korrelat \glqq kündigt an, dass die im Nebensatz enthaltene Hauptinformation noch zu kommen hat\grqq{}. Da \textit{doch} sich auf das ganze Satzgefüge bezieht, würde man dann zum Ausdruck bringen, dass der Gesamtsatz bekannt ist, obwohl der Nebensatzinhalt hochrhematisch \is{Rhematizität} ist.

\subsection{Kausalität}
Nachdem ich gegen die von anderen Autoren vertretene Annahme argumentiert habe, dass \textit{doch} in V1-/\textit{Wo}-VL-Sätzen Bekanntheit/Unkontroverse/Hintergrund kodiert, schließt sich als nächste Frage an, ob/inwiefern die kausale Interpretation der Sätze mit der Bedeutung dieser MP zusammenhängt. 

\subsubsection{Ist \textit{doch} direkt für Kausalität verantwortlich?}	
Es gibt Arbeiten, in denen die kausale Interpretation der V1-Sätze direkt an das \textit{doch} gebunden wird. In \citet[59]{Koenig1990} wird \textit{doch} als \glqq kausale konjunktional gebrauchte Partikel\grqq{} eingestuft, in \textit{grammis 2.0}\footnote{http://hypermedia.ids-mannheim.de/call/public/gramwb.ansicht?v\_app=g\&v\_kat=Konnek\-tor\&v\_id=2058} wird es als kausa\-ler Adverbkonnektor gehandelt. \citet[168]{Oennerfors1997} und \citet[170]{Pittner2011} vertre\-ten demgegenüber, dass \textit{doch} nicht an sich kausal ist. Alle Äußerungen zu dieser Frage in der Literatur beziehen sich auf die V1-Sätze. Dies ist sicherlich darauf zurückzuführen, dass die Partikel in dieser Satzumgebung obligatorisch ist. Wie oben gezeigt, ist ihr Auftreten in den \textit{Wo}-Sätzen zwar nicht notwendig, aber dennoch sehr typisch. Seltsamerweise hat man sich bei der Betrachtung dieser Sätze nicht die Frage gestellt, ob das \textit{doch} mehr mit Kausalität zu tun hat. Dass das Auftreten von \textit{doch} nicht i.e.S. mit Kausalität verbunden ist, zeigen die Verteilungen aus Abschnitt~\ref{sec:korp}. Gäbe es diesen direkten Zusammenhang, sollte \textit{doch} schließlich auch in anderen kausalen Sätzen beliebt sein. Insbesondere sollte dies gelten für solche Kausalsätze, die ähnlich verwendet werden wie die kausal interpretierten V1- und \textit{Wo}-Sätze, d.h. die \is{modaler Kausalsatz} modalen Kausalsätze. Aus den Korpusdaten ist aber abzuleiten, dass der typische Kausalsatz gar keine Partikel aufweist und wenn eine Partikel auftritt, dann ist dies nicht auffällig häufig \textit{doch}. 

Die in Abschnitt~\ref{sec:unkontr} angeführte Korpusstudie von \citet{Doering2014} zur Interaktion von MPn und Diskursrelationen deckt keine Interaktion von \textit{doch} und der Relation \textit{CAUSE} auf. Die Anzahl der \textit{doch}-Äußerungen in dieser Diskursrelation entspricht hier in etwa der Häufigkeit, mit der die Relation überhaupt vorkommt (vgl. \citeyear[88]{Doering2014}). Einen schwachen Effekt stellt sie hier für \textit{ja} fest. Wenn die Kausali\-tät in den beiden Satztypen direkt auf die Partikel zurückzuführen sein soll, wäre folglich – wie schon beim Kriterium der Präsupposition – noch eher mit dem präferierten Auftreten von \textit{ja} zu rechnen. 

Natürlich können Sätze auch kausal aufeinander bezogen werden, wenn das \textit{doch} nicht in ihnen enthalten ist (vgl. \citealt[168]{Oennerfors1997}, \citealt[171]{Pittner2011}). In diesem Kontext ist die Erkenntnis über die Existenz von \is{konzeptuelles Deutungsmuster} grundlegenden konzeptu\-ellen Deutungsmustern (vgl. \citealt[228-229]{Linke2001}, \citealt[26-28]{Averintseva-Klisch2013}, die sich auf \citet{Hume1955[1748]} bezieht, vgl. auch \citealt[6]{Sanders1992}, \citealt[339-340]{Fabricius-Hansen2000}, \citealt{Kehler2002, Kehler2004}) von Bedeutung. Diese sind aus der Sicht zu sehen, dass alles Wahrgenommene nicht als chaotische Menge betrachtet wird, sondern – in einem abgesteckten Rahmen – gewisse Möglichkeiten von Bezügen bestehen. Für Sprache heißt dies, dass Sprecher auch dann Kohärenzrelationen \is{Kohärenzrelation} herstellen, wenn keine Kohäsionsmarker \is{Kohäsionsmarker} vorliegen. Die drei basalen Relationen sind koordinative, temporale und kausale Bezüge.

\begin{exe}
	\ex\label{1022} 
	koordinativ $<$ temporal $<$ kausal
\end{exe}
Die Stärke der Beziehungen nimmt in (\ref{1022}) von links nach rechts zu. Zusätz\-lich gilt, dass die linke Relation immer auch Voraussetzung für die rechte ist. Es ist auch angeführt worden, dass Sprecher – wenn möglich – in ihrer Interpretation von der engsten Verbindung ausgehen, d.h. den kausalen Zusammenhang auswählen (vgl. z.B. \citealt[58, Fn 6, 61-62]{Breindl2006}).

Unabhängig davon, ob in (\ref{1023}) und (\ref{1024}) \textit{doch} verwendet wird oder nicht, wird der zweite Satz als Begründung des ersten gedeutet.

\begin{exe}
	\ex\label{1023} 
	Hans kommt nicht. Er ist krank.
\end{exe}
\vspace{-0.65cm}
\begin{exe}
	\ex\label{1024} 
	Hans kommt nicht. Er ist \textbf{doch} krank.
	\hfill\hbox {\citet[168]{Oennerfors1997}}
\end{exe}
Bei den \textit{Wo}-Sätzen sieht man, dass die Sätze auch kausal gelesen werden können, wenn kein \textit{doch} auftritt, obwohl dies in meinen Daten wenig eintritt, wie man an der Verteilung in Abschnitt~\ref{sec:korp} sieht.

\begin{exe}
	\ex\label{1025}
	 \scriptsize 
	\glqq Wieso hat ein amischer Mann Lautsprecherkabel?\grqq{} denke ich laut. \glqq \textbf{\textit{Wo} er weder Radio noch Fernseher besitzt.} Er benutzt ja 			nicht mal eine Melkmaschine oder einen Generator zur Milchverarbeitung.\grqq{} 
	\newline
	\hbox{}\hfill\hbox {\citet[53]{Castillo2011}}
\end{exe}
Unter den nachgestellten \textit{doch}-Sätzen ist keiner, für den sich nicht eine kausale Interpretation anbietet. Prinzipiell ist eine nicht-kausale Interpretation für \textit{Wo}-\textit{doch}-Sätze aber nicht ausgeschlossen (vgl. den modifizierten Beleg in (\ref{1026}), in dem ursprünglich keine Partikel auftritt).
	
\begin{exe}
	\ex\label{1026}
	 \scriptsize 
	Was wäre der Sport ohne seine Funktionäre – und umgekehrt? Eine interessante Frage \emph{in Zeiten} \emph{wie diesen}. \textbf{\textit{Wo} \underline{doch} 		tagtäglich über neue Highlights aus der \glqq Königsklasse des Sports\grqq{} – der olympischen Bewegung samit ihren Funktionären – berichtet wird.}
	\newline
	\hbox{}\hfill\hbox {(BVZ09/SEP.00997 Burgenländische Volkszeitung, 09.09.2009) $[$verändert S.M.$]$}
\end{exe}	
Fehlt das \textit{doch}, ist man allerdings eher bereit, den Satz nicht kausal zu lesen. 

\begin{exe}
	\ex\label{1027}
	\scriptsize 
	Was wäre der Sport ohne seine Funktionäre – und umgekehrt? Eine interessante Frage \emph{in Zeiten} \emph{wie diesen}. \textbf{\textit{Wo} tagtäglich über 		neue Highlights aus der \glqq Königsklasse des Sports\grqq{} – der olympischen Bewegung samit ihren Funktionären – berichtet wird.}
	\newline
	\hbox{}\hfill\hbox {(BVZ09/SEP.00997 Burgenländische Volkszeitung, 09.09.2009)}
\end{exe}                                    
Für (\ref{1027}) bietet sich auch gut die Interpretation als Relativsatz an. Dies ist selbst dann möglich, wenn kein explizites Bezugselement vorhanden ist:

\begin{exe}
	\ex\label{1028}
	 \scriptsize 
	Und endlich mal Dusche, Badewanne und die Kloschüsseln mit der Bürste von unschönen Kalkrändern befreien. Und natürlich noch schnell den Garten 			entlauben, umgraben und mit Mulch bedecken. \textbf{\textit{Wo} es gerade so trocken ist.}    
	\hfill\hbox {(BRZ05/NOV.18215 Braunschweiger Zeitung, 19.11.2005)}
\end{exe}                                                                        
Beim \textit{Wo}-Satz vereindeutigt das \textit{doch} die Interpretation. Bei den V1-Sätzen sieht man, dass eine der drei Eigenschaften, die die drei deklarativen V1-Sätze aus\-zeichnet (vgl. Abschnitt~\ref{sec:unkontr}), bzw. ein segmental identischer anderer V1-Satz vorliegen muss, da die Struktur ansonsten nicht interpretierbar ist. Es ist aber auch bei den V1-Sätzen nicht so, dass sie kausal gelesen werden müssen, sobald \textit{doch} vorkommt. (\ref{1029}) kann beispielsweise auch adhortativ \is{Adhortativ} gelesen werden. In (\ref{1030}) liegt ein \is{emphatischer V1-Deklarativsatz} emphatischer V1-Deklarativsatz vor, den man (anders als (\ref{1029})) auch gar nicht kausal verknüpfen kann.

\begin{exe}
	\ex\label{1029}
	 \scriptsize 
	Eine positive Einstellung und Wertschätzung ihnen gegenüber lohnt sich auf jeden Fall. \textbf{Bauen wir \underline{doch} Brücken auf und Vorurteile 		ab.}                  		
	\hfill\hbox {(A08/NOV.01067 St. Galler Tagblatt, 05.11.2008)}
\end{exe}

\begin{exe}
	\ex\label{1030}
	 \scriptsize 
	Wenn die Maus sich darunter zum Schlemmern niederlässt, Deckel runter, peng. \textbf{Bricht \underline{doch} der Knauf der Dose ab!} Mitten in der 			Inszenierung.    
	\newline              		
	\hbox{}\hfill\hbox {(BRZ05/DEZ.05511 Braunschweiger Zeitung, 05.12.2005)}
\end{exe}
Ich glaube deshalb, dass es für \textit{Wo}- und V1-Sätze sehr wichtig ist, dass die Sachverhalte bzw. Annahmen, die kausal aufeinander bezogen werden, diese Verbindung überhaupt erlauben (vgl. auch \citealt[87-128]{Gohl2000} zur Relevanz von situativem und konzeptuellem Wissen bei der kausalen Interpretation \is{Asyndese} asyndetischer Strukturen). Ist dies gegeben und tritt dann \textit{doch} in diesen Sätzen auf, werden sie i.d.R. auch kausal gelesen. Auf die Annahme, dass die Konstellation der Sachverhalte in dem vorweggehenden Satz und dem V1-Satz relevant ist, bauen auch \citet{Oennerfors1997} und \citet{Pittner2011}. Die Erklärung des Zustandekommens der kausalen Interpretation wird in den beiden Arbeiten deshalb als Zusammenspiel des MP-Beitrags, der Verbstellung, des inhaltlichen Bezugs der beiden Sätze und der Nachstellung des V1-Satzes gesehen. Sie unterscheiden sich allerdings darin, dass Önnerfors (anders als Pittner) davon ausgeht, dass die eigentliche \textit{doch}-Bedeutung nicht vorhanden ist.

\subsubsection{Ist \textit{doch} indirekt für Kausalität verantwortlich?}
\label{sec:kausalind}
Den Beitrag von \textit{doch} fassen die beiden Autoren wie folgt:

\begin{quotation}
Durch die V1-Stellung im bV1-DS erfolgt eine besonders enge Anknüpfung an den unmittelbar vorausgehenden Kotext. Durch diese Anknüpfung, die durch die inhaltliche Nähe der beiden – in begründender Weise aufeinander bezogenen – Propositionen noch unterstrichen wird, wird gewissermaßen signalisiert, daß im Falle des bV1-DS die \glq Bezugsdomäne\grq {} des rückverweisenden \textit{doch} nicht, wie im Standardfall, der Kontext ist, sondern der Kotext, genauer: die Proposition des Bezugssatzes. Dieser obligatorische Bezug auf den unmittelbar vorausgehenden Kotext blockiert die Möglichkeit einer auf den Kontext zugreifenden Widerspruchsimplikatur des \textit{doch}.                            
\newline              		
\hbox{}\hfill\hbox {\citet[170]{Oennerfors1997}}
\end{quotation}

\begin{quotation}
Indem der Rezipient auf einen Sachverhalt hingewiesen wird, der als unkontrovers, aber in seinem momentan aktualisierten Wissen nicht präsent gekennzeichnet wird, kann in Zusammenhang mit der engen Anbindung an den Bezugssatz durch die V1-Stellung die Relation der stützenden Begründung erschlossen werden.
\hfill\hbox {\citet[170]{Pittner2011}}
\end{quotation}
Interessanterweise erachten beide die V1-Stellung als sehr relevant, sie sehen in ihr das Potenzial im Beitrag dieser Sätze. Die \textit{doch}-Bedeutung selbst (er geht von (\ref{1031}) aus) spielt bei Önnerfors keine Rolle. Sie ist getilgt (s.o.).

\begin{exe}
	\ex\label{1031} 
		$\lambda \textrm{p[FAKTp}]$\\
		\textsc{Implikatur}$[\exists \textrm{q[q} \rightarrow \neg \textrm{p}]]$
		\hfill\hbox {\citet[83]{Ormelius-Sandblom1997}}
\end{exe}
Es geht ihm nur um die generelle Funktion des Rückverweises im Kontext des sowieso verfügbaren kausalen Zusammenhangs und der durch die Verbstellung angezeigten Verbindung zwischen den Sätzen. Bei Pittner spielt die \textit{doch}-Bedeu\-tung hingegen eine Rolle: Sie modelliert sie über eine Anweisung an den Hörer, wie in (\ref{1032}), und unterscheidet zudem zwischen dem \textit{cg des Dialogs} \is{Dialog-cg} und einem \is{genereller cg} \textit{generellen cg}. Der generelle cg umfasst einen \is{persönlicher cg} persönlichen cg, der zwischen Individuen in der Interaktion zustande kommt (z.B. durch gemeinsame Erfahrungen, Handlungen) und \is{kultureller cg} einen \textit{kulturellen cg}, der zwischen Mitgliedern bestimmter Gruppen entsteht (z.B. Nation, Sprache).

\begin{exe}
	\ex\label{1032} 
	Ersetze $\neg$p durch p.	
	\hfill\hbox {\citet[167]{Pittner2011}}
\end{exe}
Eine \textit{doch}-Äußerung nimmt ihr zufolge Bezug auf eine Situation, in der der Hörer p eigentlich weiß, es aktuell aber nicht berücksichtigt, so dass die MP-Äußerung ihn auffordert, den aktuellen Wissensstand (Dialog-cg) aus dem generellen cg zu aktualisieren. 

Das Beispiel in (\ref{1033}) deutet Pittner so, dass der Sprecher der \textit{doch}-Äußerung den Adressaten an ihren Inhalt erinnern möchte, d.h. davon ausgeht, dass er ihn eigentlich weiß. Entlang von (\ref{1032}) müsste folglich gelten, dass der Angespro\-chene im aktuellen Diskurs von $\neg$p ausgeht, obwohl im generellen cg p gilt, und er wird aufgefordert, p anzunehmen (vgl. \citealt[167-168]{Pittner2011}).

\begin{exe}
	\ex\label{1033} 
	$[$Ein Junge will in Gegenwart eines Erwachsenen etwas aus einer Flasche trinken.$]$
	Du bist noch nicht groß genug. Du kannst \textbf{doch} nicht eine Flasche Wein allein austrinken.			
	\hfill\hbox {\citet[168]{Pittner2011}}
\end{exe}
Ich halte die von Pittner angesetzte \textit{doch}-Bedeutung unabhängig der Betrachtung von V1-Sätzen für problematisch, da sie in jeder Verwendung nachweisen können müsste, dass der Gesprächspartner die gegenteilige Annahme vertritt. Es ist sicherlich für jede Bedeutungszuschreibung eine Herausforderung, in jedem Kontext die formulierten Bedeutungsaspekte nachzuweisen. Auf dieses Bedeutungsmoment lässt sich aber gut verzichten. Hinzu kommt, dass sie immer annehmen muss, dass Gesprächsteilnehmer angesprochen werden. Sie gehen schließlich von $\neg$p aus und sollen diese Annahme revidieren. Für mich hat diese Vorstellung etwas sehr Aktives. Insbesondere die V1-Sätze, mit denen Pittner sich beschäftigt, treten so gut wie gar nicht dialogisch auf. Die gegenteilige Annahme müsste dann immer dem Leser unterstellt werden und er müsste angesprochen werden, $\neg$p durch p zu ersetzen, weil p im kulturellen cg enthalten ist. Es ist eher nicht davon auszugehen, dass ein Autor und seine anonyme Leserschaft einen persönlichen cg teilen bzw. dass die V1-Sätze nur unter diesen Umständen verwendet werden. Man müsste dann annehmen, dass dieses Verhältnis stets vorgegeben ist – was ich nicht für besonders wünschenswert halte.

Für (\ref{1034}) z.B. halte ich es nicht für passend, anzunehmen, dass der Leser vertritt, dass es nicht König Dagobert I. war und dass er durch die \textit{doch}-Äußerung angehalten wird, diese Ansicht zu ersetzen durch die ihm eigentlich bekannte Annahme, dass KD der Metzer Domkirche ein Weingut in Neef schenkte.

\begin{exe}
	\ex\label{1034}
	\scriptsize 
	Schon seit dem frühen Mittelalter hat der Wein dem Ort Bedeutung verliehen. \textbf{\textit{War} es \underline{doch} König Dagobert I., der der Metzer 		Domkirche ein Weingut in Neef schenkte.}    
	\newline              		
	\hbox{}\hfill\hbox {(RHZ09/OKT.24515 Rhein-Zeitung, 28.10.2009)}
\end{exe}
Ich verstehe an Pittners Ausführungen auch nicht, wie man aus ihrer \textit{doch}-Model\-lierung ableiten kann, dass die Sätze, die aufgrund der V1-Stellung eng aufeinander bezogen werden, kausal verknüpft werden. Sie führt diesen Aspekt nicht aus. Warum begünstigt \glq Ersetze $\neg$p durch p.\grq{} eine kausale Verbindung zwischen den Sätzen?

Önnerfors schreibt der Partikel nur rückverweisende Funktion zu. Ich halte die Annahme dieses allgemeinen Beitrags dieser speziellen Partikel für zu schwach, weil der Rückverweis der Grundbeitrag einer jeden MP ist. \textit{Doch} zählt nicht einmal zu den Partikeln, die den Rückverweis am deutlichsten vorweisen. Sie kann auch diskursinitial verwendet werden, anders als z.B. \textit{halt}, \textit{eben} oder \textit{auch}, für deren adäquate Verwendung ein Beitrag vorweggehen muss. Es gibt Partikeln, die sich besser eignen (würden), den Rückverweis anzuzeigen, wenn dies der Grund für das obligatorische Vorkommen der Partikel in diesen Sätzen sein soll.

Beide Ansätze lassen den Anteil der Partikel an der kausalen Relation offen. Da ich gegen die stets vorliegende Bekanntheit des Inhalts des V1-Satzes argumentiere und zudem nicht davon ausgehe, dass \textit{doch} Unkontroverse markiert, stellt sich für mich umso mehr die Frage, welchen Beitrag \textit{doch} in diesen und in den \textit{Wo}-VL-Sätzen leistet. 

An beiden Ansätzen stört mich darüber hinaus, dass sie so viel Potenzial in der Verberststellung sehen. Sie bewirkt den beiden Autoren zufolge die enge Kontextanbindung. Da in den \textit{Wo}-VL-Sätzen, die ich unter der gleichen Erklärung aufzufangen beabsichtige, keine V1-Stellung vorliegt, finde ich es schwierig, sie positiv in meine Analyse zu integrieren. Man bräuchte dann in jedem Fall eine andere Erklärung für die \textit{Wo}-Sätze.

Ich möchte deshalb die gegenteilige Perspektive einschlagen und den Blick auf die beiden Satztypen wählen, zu sagen, dass sie beide gewisse \glq Defekte\grq {} (und weniger Potenziale) aufweisen. Aus diesem Grund müssen sie auf ganz be\-stimmte Art eingebunden und sprachlich ausgestattet werden, um im Diskurs überhaupt Verwendung zu finden. Ich halte hier die recht einfache Aussage von \citet[250]{Scheutz2009} zum V1-Satz für entscheidend. Er schreibt, der V1-Satz drücke \glqq textuelle Unselbständigkeit\grqq{} aus. Dem schließe ich mich an und nehme an, dass es einen \glq neutralen\grq {} V1-Deklarativsatz nicht gibt. Jeder V1-Deklarativsatz \is{V1-Deklarativsatz} muss eine bestimmte Beschaffenheit haben, um grammatisch lizensiert und textuell/diskurs\-strukturell verwendbar zu sein. Er weist kein Vorfeld und eine besetzte linke Satzklammer auf, d.h. zwei Positionen, in denen sich sonst textuelle Anknüpfungen maßgeblich abspielen, scheiden für derartige Kodierungen aus. Der \textit{Wo}-Satz bringt nun ein ähnliches Problem mit sich, wenn auch weniger verschärft. Er ist prinzipiell ambig, weil \textit{wo} temporal, adversativ, lokal, kausal und konzessiv interpretiert werden kann. In diesem Sinne ist seine inhärente Bedeutung sehr arm/unspezifisch und könnte abstrakt etwa als \glq Gleichheit\grq {} gefasst werden. Die textuelle Verknüpfung ist folglich auch für diesen Satztyp nötig, aber ebenfalls schwierig, da auch hier das Vorfeld und die linke Satzklammer nicht verfügbar sind. Für beide Satztypen bleibt somit nur das Mittelfeld, um die textuelle Anbindung, die sie zur Interpretation benötigen, zu gewährleisten. Für die Kodierung kommen MPn und Adverbien im Mittelfeld in Frage.

Neben dieser \glq defekten\grq {} Ausgangslage geht in meine Analyse der \textit{Wo}-VL- und V1-Sätze auch die oben ausgeführte Annahme zu konzeptuellen Grundmustern \is{konzeptuelles Grundmuster} ein. Der V1/\textit{Wo}-Satz und der vorweggehende Satz sind prinzipiell kausal aufeinander zu beziehen. Ich habe gezeigt, dass dies ein wichtiger Aspekt ist. Das \glq rei\-chere\grq {} \textit{Wo} ermöglicht auch allein (d.h. ohne \textit{doch}) die kausale Interpretation. Wenn die kausale Relation nicht nahe liegt (aber alternative Lesarten zulässig sind), tritt sie auch bei den V1-Sätzen nicht zwingend ein. Die kausale Relation kann folglich stets als bestehend angesehen werden, entweder, weil sie tatsächlich gilt, oder, weil eine Default-Interpretation in Kraft tritt, der zufolge Sprecher sehr geneigt sind, da, wo es möglich ist, Kausalität hineinzulesen, wenn zwei Sätze aufeinander folgen.\footnote{Im Rahmen meiner Analyse sind folglich weder die \textit{Wo}-VL- noch die V1-Sätze inhärent kausal oder konzessiv. Ein Gutachter argumentiert, dass \textit{Wo}-VL-Sätze alleine, d.h. unabhängig der vorweggehenden Einstellung, konzessiv gelesen würden. Er vertritt, dass ein V1-Satz in einem Kontext wie in (\ref{1037}), in dem das Erstaunen ausgeschlossen werden muss und sich die Begründung einer Annahme nicht anbietet, unangemessen ist, während ein \textit{Wo}-Satz gut stehen könnte. 

\begin{exe}
	\ex\label{1037}
	Das Elta-Gerät hat einen hohe Brand- und Verletzungsgefahr. \#Trägt der Elta doch wie alle anderen Haartrockner das CE-Zeichen und dazu das GS-Zeichen 		(GeprüfteSicherheit).   
\end{exe}

\begin{exe}
	\ex\label{1038}
	Das Elta-Gerät hat einen hohe Brand- und Verletzungsgefahr. Wo der Elta doch wie alle anderen Haartrockner das CE-Zeichen und dazu das GS-Zeichen 		(GeprüfteSicherheit).  
\end{exe}
Es wird sich später zeigen, dass es tatsächlich entlang der Interpretationen \textit{konzessiv} vs. \textit{non-konzessiv} Verwendungsunterschiede zwischen den beiden Satztypen gibt. Ich bin aber weiterhin der Ansicht, dass die Sätze nicht inhärent konzessiv (und auch nicht inhärent kausal) sind.
}

Ferner meine ich, dass \textit{doch} diese enge Kontextanbindung, die prinzipiell vor\-handen ist, forciert. Die von mir angesetzte \textit{doch}-Bedeutung erlaubt es, zu motivieren, warum sich genau diese Partikel für die enge Kontextanbindung gut eignet. Nach meiner Modellierung zeigt \textit{doch} an, dass die Äußerung auf ein offenes Thema reagiert (vgl. (\ref{1035})).
\newcolumntype{C}[1]{>{\centering}p{#1}} 
\begin{exe}
	\ex\label{1035} Kontext vor einer \textit{doch}-Assertion\\[-1em]
 	\begin{tabular}[t]{|C{6em}|C{6em}|C{6em}|}
 	\hline 	
 	$\textrm{DC}_{\textrm{A}}$ & {Tisch} & $\textrm{DC}_{\textrm{B}}$ \tabularnewline
  	\hline
    & p $\vee$ $\neg$p & \tabularnewline
 	\hline      
   	\multicolumn{3}{|l|}{cg s$_{1}$} \tabularnewline   
   	\hline
 	\end{tabular}
\end{exe}
Auf ein offenes Thema zu reagieren, verweist auf eine enge Kontextanbindung bzw. schafft diese. Eine Vorstellung, die in Diskursmodellen steckt (vgl. z.B. \citealt{Roberts1996}, \citealt{Buering2003}), ist, dass sich ein Diskurs idealerweise durch Frage-Antwort-Sequenzen gestaltet, wobei die Fragen (wie in meiner Darstellung des Modells von \citealt{Farkas2010} gezeigt) auch durch Assertionen eröffnet werden können. Tatsächlich ist ein Diskurs i.d.R. nicht durch strikte Frage-(komplette) Antwort-Sequenzen strukturiert, sondern es werden Teilantworten gegeben, wes\-halb auch Modelle davon ausgehen, dass sich ein Diskurs in Unterfragen splittet. Diese Subfragen bilden aber entscheidenderweise mit ihren Antworten ebenfalls Frage-Antwort-Paare (vgl. (\ref{1035a}) und (\ref{1035b}), \citealt[515-516]{Buering2003}).

\begin{exe}
\ex \label{1035a}
        \begin{jtree}
        \! = {Diskurs}
                :[scaleby=3.75 1]{Frage}!a [scaleby=3.75 1]{Frage}
                <vert>{\ldots}.
        \!a = <wideleft>[scaleby=2.75 1]{Subfrage}(<vert>{Antwort}) ^<left>[scaleby=1.75 1]{Subfrage}(<vert>{Antwort}) ^<right>[scaleby=1.75 1]{Subfrage}!b ^<wideright>[scaleby=2.75 1]{Subfrage}(<vert>{Antwort}).
        \!b = :[scaleby=1.25 1]{Subfrage}(<vert>{Antwort}) [scaleby=1.25 1]{Subfrage}(<vert>{Antwort}).
        \end{jtree}
\end{exe}

\begin{exe}
        \ex \label{1035b}
        \begin{tabbing}
        Wie war \= War die \= Wie war der Schlagze \= Es war \= \kill
        Wie war das Konzert?\\
                \> War die Musik gut?                \>\> Nein, sie war schrecklich.\\
                \> Wie war das Publikum?        \>\> Es war enthusiastisch.\\       
                \> Wie war die Band?\\
                        \>\> Wie war der Schlagzeuger?        \>\> Fantastisch.\\
                        \>\> Wie war der Sänger?        \>\> Besser den je!\\
                \> Wurden alte Lieder gespielt? \>\> Nein, kein einziges.\\
        Was hast du nach dem Konzert gemacht? \ldots
        \end{tabbing}
\end{exe}
Jede Antwort auf eine \glq Frage\grq {} leistet somit eine enge Kontextanbindung. Sie bringt den Diskurs weiter, auch dann, wenn man dadurch nur zur nächsten Frage gelangt. Es ist unmöglich, zu sagen, dass ein Sprecher mit seiner Assertion auf eine offene Frage reagiert und gleichzeitig damit anzeigt, dass er keinen direkt relevanten Diskursbeitrag leistet.

Ich gehe davon aus, dass die Sätze aufgrund der ausgedrückten Sachverhalte bzw. einer generellen Erwartung zur kausalen Default-Interpretation zweier auf\-einander folgender Sätze kausal aufeinander bezogen werden. Wären die Sätze nicht unterspezifiziert und bräuchten deshalb eine Interpretationshilfe, könnte Kausa\-lität über Kohärenz auch ohne die Partikel hergestellt werden. In diesen Satzkontexten braucht man aber ein Element, das diese Lesart stützt. \textit{Wo} kann diese Interpretation prinzipiell auch alleine erreichen, \textit{doch} begünstigt sie allerdings. In meinen Daten finden sich sehr wenige \textit{Wo}-Sätze ohne \textit{doch}. V1-Sätze hingegen brauchen einen speziellen Marker, um ihre Interpretation sicherzustellen. In den kausalen V1-Deklarativsätzen ist dies das \textit{doch}. Wie in Abschnitt~\ref{sec:unkontr} gesehen, weisen auch die anderen deklarativen V1-Sätze sprachliche Besonderheiten auf. 

Wenn \textit{doch} in diesen Sätzen indirekt einen Beitrag zur kausalen Lesart leistet, schließt sich die Frage an, warum nicht eine andere Partikel verwendet wird, die Kausalität besser kodiert. \textit{Doch} wird inhärent nicht kausal interpretiert. Ginge es in diesen Sätzen nur darum, die sowieso schon vorhandene kausale Verknüpfung zwischen den Sätzen zu forcieren, würden sich durchaus andere MPn anbieten, die dies nicht indirekt und nur auf Umwegen leisten können. Schließlich gibt es MPn, die Kausalität direkt anzeigen, und zwar \textit{halt}, \textit{eben} und \textit{auch}, die in Abschnitt~\ref{sec:kontexte} in Kapitel~\ref{chapter:hue} und Abschnitt~\ref{sec:auch} behandelt werden. Den Ausschluss dieser Partikeln motiviere ich im Folgenden in den Abschnitten~\ref{sec:transdoch} und \ref{sec:litdoch} unter Bezug auf zwei weitere Aspekte dieser Sätze. Einen solchen Ausschluss leisten Önnerfors und Pittner nicht. Beide gehen vom Vorliegen von Unkontroverse/Bekanntheit/ Präsupponiertheit aus. Nach meiner Bedeutungsmodellierung schließen diese Bedeutungsaspekte die Verwendung von \textit{halt} und \textit{auch} aus, da sie die Proposition, auf die sie sich beziehen, assertieren. \textit{Eben} käme allerdings in Frage, da es sowohl Präsupponiertheit als auch Kausa\-lität kodiert. Da ich gegen den präsupponierten Status des Satzinhaltes argumentiere, kommen im Rahmen meiner Ableitung bisher auch \textit{halt} und \textit{auch} in Frage.

\subsection{Die Transparenz der \textit{doch}-Bedeutung}
\label{sec:transdoch}
Der erste Aspekt, der die Eignung von \textit{doch} unterstreicht, ist, dass ich nicht glaube, dass die Partikel ausschließlich dem Anzeigen von Kontextbezug dient. Geht man von der Modellierung aus, dass sie auf ein vorausgesetztes, offenes Thema reagiert, lässt sich von einer transparenten Verwendung ausgehen. Die Tatsache, dass Önnerfors vertritt, die \textit{doch}-Bedeutung sei in den V1-Sätzen ausgeblichen, ist folglich auf die Bedeutung zurückzuführen, auf die er sich stützt. Auch Pittner, die prinzipiell für das Vorliegen der regulären \textit{doch}-Bedeutung argumentiert, zeigt an keinem V1-Satz auf, inwiefern sie die Anweisung \glq Ersetze $\neg$p durch p!\grq {} vorliegen sieht.

Schaut man sich die Kontexte genauer an, scheint es nicht abwegig, die \textit{doch}-Bedeutung zu rekonstruieren, und zwar fern der indirekten Funktion im kausalen Zusammenhang. (\ref{1039}) zeigt einen \textit{Wo}-VL-Satz und es lässt sich motivieren, warum nach der vorweggehenden Frage das Thema offen ist, ob jeder weiß, dass donnerstags die neuen Filme anlaufen.

\begin{exe}
	\ex\label{1039} 
	\scriptsize
	Wieso müssen Agenturpartys immer am Donnerstag sein? \underline{\textbf{\textit{Wo} doch \textit{jeder weiß}}},\\ 
	\underline{\textbf{dass Donnerstags die neuen Filme anlaufen}}.                                      	
	\newline              		
	\hbox{}\hfill\hbox {(http://www.ankegroener.de/anke1/pasdeblog/blogarchiv/september2002.html)}
\end{exe}
Wenn die Leute, die die Agenturpartys planen, diese auf Donnerstag legen, kann man sich fragen, ob sie nicht wissen, dass an diesem Tag das Kino stattfindet, da sie die Partys ansonsten nicht auf diesen Termin legen würden. Die Ableitung der offenen Frage, ob sie nicht um das Kino wissen, geschieht also vor dem Hintergrund, dass man eine Agenturparty normalerweise nicht auf den Donnerstag legen würde, wenn man wüsste, dass dann Kino ist.  Die Leute, die die Partys planen, wissen es scheinbar nicht. Der Sprecher kann aber nicht so recht glauben, dass sie dies nicht wissen. Es kann somit auch als auf dem Tisch liegend ausgegeben werden, dass fraglich ist, ob sie p wissen. Diese Frage beantwortet der Sprecher dahingehend, dass sie es s.E. wissen müssen, indem er sagt, dass es jeder weiß (also auch diese Leute).

Es lässt sich aus den konzessiven Fällen immer die Offenheit von p ableiten. Die vorkommenden Einstellungen unter dieser Interpretation (Verwunderung, Erstauntsein, Sichfragen) kommen überhaupt zustande, weil der beteiligte Sachverhalt normalerweise anders zu erwarten wäre. Und in diesem Sinne kann der Sachverhalt, der der Grund für diese Einstellung ist, immer in Frage stehen, weil der Sprecher/Schreiber von der gegenteiligen Annahme ausgeht. Unter Bezug auf (\ref{1039}) bedeutet dies: Obwohl in den Augen des Sprechers jeder weiß, dass donnerstags die Filme anlaufen, finden dann die Partys statt. $\rightarrow$ Weiß es nicht jeder? $\rightarrow$ Wissen die Organisatoren es nicht? $\rightarrow$ Dass in den Augen des Sprechers jeder weiß, dass donnerstags die Filme anlaufen, begründet seine Verwunderung.

(\ref{1040}) zeigt einen V1-Satz, für den eine parallele Analyse möglich ist.

\begin{exe}
	\ex\label{1040} 
	\scriptsize
	Ich bin klar enttäuscht über das Resultat der FDP. Das schlechte Abschneiden ist sehr überraschend. \textbf{\textit{Führten} die Freisinnigen 				\underline{doch} einen super Wahlkampf} – ganz im Gegensatz zu den anderen Parteien. 
	\hfill\hbox {(A08/SEP.09380 St. Galler Tagblatt, 29.09.2008)}
\end{exe}
In (\ref{1040}) ist die konzessive Relation \glq Obwohl sie einen tollen Wahlkampf geführt haben, haben sie schlecht abgeschnitten.\grq {} und die kausale \glq Ich wundere mich über das schlechte Abschneiden, weil sie einen tollen Wahlkampf hatten.\grq {} Aufgrund des schlechten Abschneidens kann man sich fragen, ob sie keinen guten Wahlkampf geführt haben, weil ein guter Wahlkampf normalerweise zu gutem Abschneiden führt bzw. andersherum aus schlechtem Abschneiden ein schlechter Wahlkampf abzuleiten ist. Die Frage, ob der Wahlkampf schlecht war, wird negativ beantwortet und erklärt die Verwunderung: Sie hatten einen super Wahlkampf, aber sie haben schlecht abgeschnitten.

Und auch wenn Konzessivität nicht beteiligt ist, bin ich der Meinung, dass sich die Offenheit des Themas (anders als das Vorliegen der der \textit{doch}-Äußerung entgegengesetzten Proposition) motivieren lässt, so dass \textit{doch} auch in diesem Fall eine völlig transparente Verwendung zuzuschreiben ist.

In (\ref{1041ab}) begründet der Sprecher seine Einschätzung, dass Markus Haag stolz sein kann damit, dass er 99\% der Stimmen erhalten hat.

\begin{exe}
	\ex\label{1041ab} 
	\scriptsize
	Wattwils Gemeindammann Markus Haag hingegen darf auf sein Wahlresultat überaus stolz sein. \textbf{\textit{Sprachen} \underline{doch} 99 Prozent all 		derer, die an die Urne gingen, ihm ihr Vertrauen aus und gaben ihm die Stimme.}
	\hfill\hbox {(A00/SEP.65537 St. Galler Tagblatt, 25.09.2000)}
\end{exe}
Wie in Abschnitt~\ref{sec:kausalind} bereits angeführt, ist davon auszugehen, dass Sprecher – bemüht um Kohärenz – zu einer kausalen Default-Interpretation \is{kausale Default-Interpretation} tendieren. Wenn es sich anbie\-tet, beziehen sie aufeinander folgende Sätze kausal aufeinander. Der Adressat rechnet folglich damit, dass der Folgesatz als Begründung des vorweggehenden Satzes fungieren kann. Für (\ref{1041ab}) würde dies bedeuten, dass der Rezipient nach Abschluss der Aufnahme des ersten Satzes die Erwartungshaltung aufweist, dass eine der denkbaren Begründungen, derer es sicherlich verschiedene gibt (vgl. z.B. p, r und s in (\ref{1041})), die zutreffende Begründung ist und der Sprecher sie somit im Folgesatz vertreten könnte.

\begin{exe}
	\ex\label{1041} 
		\begin{xlist}	
			\ex\label{1041a} Wenn man 99\% der Stimmen bekommt (p), kann man stolz sein (q).
			\ex\label{1041b} Wenn man als sehr junger Mensch kandidiert (r), kann man stolz sein (q).
			\ex\label{1041c} Wenn man gegen einen Ortsansässigen antritt (s), kann man stolz sein (q).
		\end{xlist}
\end{exe}
Wenn q (die Folge) assertiert wird, stellt sich dieser Überlegung nach für p, r und s die Frage, ob dies der tatsächliche Grund ist. Prinzipiell angeführt werden könnten sie alle. Und da sie alle als Grund denkbar wären, stellt sich auch für alle diese in Frage kommenden Propositionen – als Voraussetzung, als Grund zu fungieren – ob sie im Diskurs Gültigkeit haben. Bevor der V1-Satz in (\ref{1041ab}) geäußert wird, steht in diesem Sinne bereits im Raum, ob p/r/s gelten. Diese Überlegung baut dann natürlich darauf, dass sich der Inhalt des V1/\textit{Wo}-Satzes plausibel als Begründung des ersten eignet. Von dieser Konstellation gehen auch schon \citet{Oennerfors1997} und \citet{Pittner2011} (für die V1-Sätze) aus. Diese Annahme, dass in diesem Sinne nach jeder Äußerung prinzipiell denkbare Begründungen in Frage stehen, muss nicht heißen, dass \textit{doch} deshalb präferiert in kausal interpretierten Sätzen auftritt (ob durch Konnektoren ausgezeichnet oder asyndetisch \is{Asyndese} verbunden). Das Aufdecken der kodierten Interpretation, die im Kontext vorliegt, ist schließlich kein Muss. \textit{Doch} sollte allerdings in derart interpretierten Sätzen stehen können. 

Eine Frage, die man an dieser Stelle berechtigterweise aufwerfen kann, ist, ob die Vorstellung, dass jede Äußerung Möglichkeiten ihrer Begründung evoziert, nicht zu redundant ist. Ich meine jedoch, dass in Verbindung mit dieser Art von kausalen Relation, die bei den V1- und \textit{Wo}-VL-Sätzen vorliegt, in größerem Maße von der Erwartungshaltung des Adressaten, eine Begründung zu erhalten, ausgegangen werden kann. Wie zu Beginn der Diskussion dieser Satztypen ausgeführt, begründen \textit{Wo}-VL- und V1-Sätze stets auf \is{epistemischer Kausalsatz} der epistemischen oder \is{illokutionärer Kausalsatz} illokutionären Ebene. 

\citet{Gohl2000} untersucht asyndetische kausal interpretierte Strukturen, die vor\-herige Konversationszüge erklären oder rechtfertigen, wie z.B. in (\ref{1042}), wo die unterstrichenen Beiträge Vorschläge bzw. dispräferierte Reaktionen auf solche (s.u.) begründen.
		
\begin{exe}
	\scriptsize
	\ex\label{1042} 
	\begin{tabular}[t]{ll}
	\multicolumn{2}{l}{MEAT (Schwab 8; 15:18)}\\
	1 Erik:	& tausche mer $<<$p$>$ \underline{deins isch kleiner}.$>$\\
	2 Kai: & gar net –\\
	3 Uwe: & noi nemm du \underline{i will bloße schtü$[$ck};=\\
	 & \hspace{5cm}$[$nein (2 Silben) esse.\\
	4 Kai: & i will doch net solche brocken.
	\hfill\hbox{\citet[100]{Gohl2000}} 						 
    \end{tabular}       
\end{exe}														      
Die untersuchten Strukturen entsprechen am ehesten epistemischen oder illokutionären Begründungen, da höhere Einheiten als Sachverhalte begründet werden. \citet[103]{Gohl2000} fragt sich, in Reaktion auf welche Beiträge derartige Begründungen angeführt werden und kommt zu dem Schluss, dass die vorherige Äußerung ihrer bedarf: 

\begin{quotation}
An utterance is much more likely to be interpreted as an account, i.e. as an utterance explaining and/or justifying a previous conversational move, if the preceding utterance, by virtue of its sequential and social implications, calls for accounting.                                                                                             
\end{quotation}
\citet[58, Fn 7]{Breindl2006} gehen davon aus, dass im Grunde jede nicht-rituelle Äußerung strittig sein kann. Gohl benennt allerdings einige konkrete Fälle, auf die erklärende oder rechtfertigende Begründungen in ihren Daten folgen. Sie nennt hier die unerwünschte Reaktion auf eine vorherige Aktion (z.B. ein abgelehnter Rat, die Unmöglichkeit, eine Frage zu beantworten), Bewertungen, Aufforderungen, Beschwerden und Vorwürfe. \citet[289-297]{Ford2000} führt als weiteren Fall einen vorweggehenden Kontrast an (vgl. z.B. (\ref{1043})), wobei das Konzept sehr weit gefasst wird (explizit gegensätzliche Propositionen, Negation präsupponierter Information, Uneinigkeit zwischen Sprechern):

\begin{exe}
	\ex\label{1043} 
    \begin{tabular}[t]{ll}
	& S: I have a check.\\
	& S: Eight fifty.\\
	contrast & J: Nah. I won t take – I don t take second-party checks.\\
	& S: eh huh huh huh huh\\
	explanation & J: I don't got no way of telecheking  em,						 
    \end{tabular} \\
    \hbox{}\hfill\hbox{\citet[294]{Ford2000}} 
\end{exe}	
Das typische Muster contrast + explanation erklärt \citet[289]{Ford2000} auf die Art, dass die Diskursteilnehmer Kontraste als Problem darstellen, die eine Erklärung erforderlich machen. \citet[103]{Gohl2000} nimmt an, dass im Falle der uner\-wünschten Reaktion auf die vorherige Aktion sowie bei Aufforderungen, Be\-schwerden und Vorwürfen gesichtsbedrohende, konversationell sensible Züge vorliegen, die deshalb einer Stützung bedürfen. Und diese fördere auch eine subjektive Sprechereinschätzung.

\citet{Gohl2000} und \citet{Ford2000} benennen folglich konkrete Anlässe für Erklärungen und Rechtfertigungen, die der kausalen Relation auf der epistemischen und illokutionären \is{epistemischer Kausalsatz} Ebene \is{illokutionärer Kausalsatz} entsprechen. Man könnte sagen, dass es sich um sprachliche Handlungen handelt, an die die Erwartung einer Erklärung (aus Adressaten\-perspektive) bzw. die Annahme des Bedarfs (aus Sprecherperspektive) geknüpft ist.

Für meine Betrachtung ist interessant, dass \citet{Ford1993, Ford2000} und \citet{Gohl2000} derartige Kontexte benennen. Ich gehe davon aus, dass sich aus der vorweggehenden Äußerung ableiten lässt, dass der Sachverhalt, der als Begründung angeführt wird, tatsächlich offen ist und nicht nur als solcher vorgegeben wird. Er ist dies meiner Argumentation nach deshalb, weil mit einer Erklärung zu rechnen ist. Die Inhalte der plausiblerweise in Frage kommenden Begründungen stehen zur Einigung im Raum, da ihre jeweilige Akzeptanz Voraussetzung dafür ist, dass sie auch die Erklärung ausmachen können. (\ref{1044}) zeigt, mit welchen Anteilen die Kontexte aus Gohl und Ford in einer Zufallsstichprobe der Größe 100 aus der exhaustiven Menge aller \textit{Wo}- bzw. V1-Sätze aus dem Archiv Tagged C in DeReKo vorkommen.

\begin{exe}
	\ex\label{1044} 
    \begin{tabular}[t]{|l|l|l|}
    \hline
	& V1 & \textit{Wo}-VL\\
	\hline
	Kontrast & 14 & 10\\
	\hline
	Bewertung & 35 & 32\\
	unerwartete Reaktion & - & -\\
	\hline
	Aufforderung & - & -\\
	\hline
	Beschwerde & - & 8\\
	\hline
	Vorwurf & - & 34\\
	\hline
	anderes & 51 & 16\\
	\hline	
    \end{tabular}   
\end{exe}
Schon in Bezug auf die Arbeiten von Gohl und Ford ist zu sagen, dass es schwierig ist, mehr festzustellen, als dass diese Kontexte mit Erklärungen  auftreten (können), weil dort keine Angaben über ihr prinzipielles Vorkommen gemacht werden. \citet[548]{Ford1993} schreibt, in 53\% ihrer Fälle handle es sich um den Kontrastkontext. Ich halte Kontrast \is{Kontrast} allerdings für eine derart unspezifische/typische Kategorie, dass man den Grad von Besonderheit erst einschätzen kann, wenn man weiß, welchen Anteil derartige Kontexte in ihrem Korpus haben. Das glei\-che gilt für Bewertungen, die \citet{Ford1993} auch schon anführt. Da in Gohl gar keine Zahlen genannt werden, sind die Annahmen in dieser Hinsicht auch schwer einzuschätzen. 

Bei den V1-Sätzen fällt ungefähr die Hälfte der Belege in die Kategorien \textit{Be\-wertung} \is{Bewertung} und \textit{Kontrast} \is{Kontrast} und diese Kategorien sind bei den \textit{Wo}-Sätzen ähnlich häufig vertreten. Nicht unproblematisch ist es, die Grenze zu anderen Kategorien zu ziehen. Unter den V1-Daten finden sich viele Annahmen, Vermutungen und Einschätzungen, die ich nicht im engen Sinne als Bewertungen aufgefasst habe. Tut man dies, erhöht sich die Zahl hier. Kontrast ist natürlich auch bei Einstellungen wie Erstaunen und Wundern beteiligt, bei denen es sich letztlich aber auch um Bewertungen handelt. Man könnte auch denken, dass Kontrast bei\-spielsweise anzeigt/begünstigt, dass etwas als Annahme verstanden wird. Begründet wird nämlich in keinem Fall die Kontrastrelation selbst. Wenn sich Kategorien überlappen, ist die Frage, welche Kategorie die wirklich relevante ist. Interessant ist, dass sich zwischen den beiden Satztypen eine Arbeitsteilung he\-rauskristallisiert: Beschwerden und Vorwürfe sind in den Vorgängeräußerungen der \textit{Wo}-Sätze dominant vertreten, in dieser Stichprobe aber absent in den V1-Daten. Dieser Tatbestand hat damit zu tun, dass die \textit{Wo}-Sätze häufiger als die V1-Sätze zusätzlich zu ihrer kausalen Interpretation das konzessive Bedeutungsmoment aufweisen (vgl. die Verteilung in (\ref{1045})).

\begin{exe}
	\ex\label{1045} 
    \begin{tabular}[t]{|l|l|l|}
    \hline
	& V1 & \textit{Wo}-VL\\
	\hline
	kausal & 87 & 16\\
	\hline
	kausal + konzessiv & 13 & 84\\
	\hline
    \end{tabular}   
\end{exe}
Wie ich oben ausgeführt habe, ist die Konzessivität auf die Begründung be\-stimmter Einstellungen zurückzuführen. Da Beschwerden/Vorwürfe durch die Beurteilung eines als abweichend empfundenen Verhaltens zustandekommen, ist nicht verwunderlich, dass in diesem Zusammenhang Konzessivität eine Rolle spielt (vgl. (\ref{1046}) bis (\ref{1049})).

\begin{exe}
	\ex\label{1046} 
	\scriptsize
	$[$...$]$ Warum denn ins Konzert gehn, wenn ich so schöne CDs zu Hause habe. Somit hört der Tag auf, wie er angefangen hat. Binär.\\
	Binär? Wissen sie, was das heisst? Nein? \textbf{\textit{Wo} \underline{doch} heute alles binär ist.} 0 oder 1, keine andere Zahl.
	\newline              		
	\hbox{}\hfill\hbox {(A98/APR.22936 St. Galler Tagblatt, 11.04.1998, Ressort: RT-PIA (Abk.); gast)}
\end{exe}
			                     
\begin{exe}
	\ex\label{1047} 
	\begin{tabular}[t]{ll}
	Zustandekommen des Vorwurfs: & Obwohl alles binär ist, wissen Sie \\
	& nicht, was \textit{binär} bedeutet.
	\end{tabular}
\end{exe}
	
\begin{exe}
	\ex\label{1048} 
	\scriptsize
	Der junge Arzt, der mich jetzt betreut, macht mir klar, dass Magen und Darm eingehend endoskopisch untersucht werden müssen. Und jetzt hält unser 			Gesundheitswesen eine faustdicke Überraschung für mich parat: Diese Untersuchungen laufen nicht mehr in der Klinik, dafür muss ich mir in einer 			Facharztpraxis einen Termin geben lassen. Das kann Wochen dauern. Ist das jetzt eine sinnvolle und kostengünstige Form der Diagnostik? 						\textbf{\textit{Wo} ich \underline{doch} schon im Krankenhaus bin.}
	\newline              		
	\hbox{}\hfill\hbox {(RHZ08/NOV.25632 Rhein-Zeitung, 29.11.2008)}
\end{exe}	

\begin{exe}
	\ex\label{1049} 
	\begin{tabular}[t]{ll}
	Zustandekommen der Beschwerde: & Obwohl ich schon im Krankenhaus \\
	& bin, muss ich zur Untersuchung \\
	& zum Facharzt in eine Praxis.
	\end{tabular}
\end{exe}
Die häufiger festzustellende konzessive Lesart der \textit{Wo}-Sätze geht somit einher mit der Begründung bestimmter Einstellungen und dem Vorliegen bestimmter Äußerungstypen. V1-Sätze werden eher verwendet, um Argumente und Annahmen zu stützen, \textit{Wo}-Sätze treten eher in Kontexten auf, in denen Ärger, Entsetzen, gegenteilige Erwartungen oder abweichendes Verhalten beteiligt sind. Letz\-teres macht den Anteil begründeter Beschwerden/Vorwürfe nachvollziehbar. Eine Beobachtung ist, dass für die Mehrheit der V1-Sätze in meiner Stichprobe gilt, dass sie nur schwer auszulassen sind (68\%). Dies spricht für die gesteigerte Erwartung ihres Vorkommens. Der Eindruck ist aber nicht nur auf das Auftreten der von Ford und Gohl angeführten Äußerungstypen zurückzuführen. Es gehen den V1-Sätzen mitunter auch Annahmen voraus, die aufgrund ihres Inhalts einer weiteren Erklärung bedürfen, weil sie z.B. Ungenauigkeiten oder Andeutungen beinhalten und meiner Mei\-nung nach schlecht ohne die folgenden Erklärungen stehen bleiben können. (\ref{1050}) bis (\ref{1053}) zeigt einige Beispiele.

\begin{exe}
	\ex\label{1050} 
	\scriptsize
	Engelburg. Die SVP und Bruno Stump, zwei Namen, die oft im gleichen Atemzug genannt werden. Zumindest in der Region St. Gallen–Gossau. Dort ist der 		66jährige Kantonsrat bekannt wie kaum ein anderer SVPler. \textbf{\textit{Gründete} er \underline{doch} 1995 die SVP des Bezirks Gossau}, danach die 		SVP Waldkirch und im Jahr 2000 die SVP Gaiserwald, die heute 60 Mitglieder zählt. Dieser stand er seither als Präsident vor.        
	\hfill\hbox {(A09/MAR.06564 St. Galler Tagblatt, 20.03.2009)}
\end{exe}

\begin{exe}
	\ex\label{1051} 
	\scriptsize
	Der Teufel steckt allerdings im Detail. \textbf{\textit{Hatten} \underline{doch} Union und SPD recht eigenwillige Einzelheiten auf der Latte}, zu denen 	noch eine Fülle von Expertenmeinungen außerhalb der Koalitionsrunde kamen.                 
	\hfill\hbox {(NUZ09/JAN.00384 Nürnberger Zeitung, 06.01.2009)}
\end{exe}                                              

\begin{exe}
	\ex\label{1052} 
	\scriptsize
	Den grössten Clou landete aber die Gattin des OK-Präsidenten Karin Hanselmann aus Oberriet. \textbf{\textit{Zeigte} sie \underline{doch} mit ihrem 			souveränen Ritt in der höchst dotierten Prüfung des Anlasses}, was in ihrem Pferd Canetta und in ihr steckt.                      
	\hfill\hbox {(A08/MAR.03654 St. Galler Tagblatt, 10.03.2008)}
\end{exe}	                                     

\begin{exe}
	\ex\label{1053} 
	\scriptsize
	Die Erzbahn, in der Ausstellung braun gestaltet, stellte die Ingenieure der schwedischen Eisenbahn vor besondere Herausforderungen. 						\textbf{\textit{Mussten} sie ihre Gleise und Bahnhöfe \underline{doch} ins \glqq buchstäbliche Nichts\grqq{} setzen}, erklärt Rainer Merten:                     
	\hfill\hbox {(NUZ08/DEZ.00406 Nürnberger Zeitung, 04.12.2008)}
\end{exe}
Mein Eindruck ist, dass dies weniger deutlich auf die \textit{Wo}-Sätze zutrifft (55\% sind schlecht weglassbar). Ich finde Gohls Überlegung, dass Beschwerden und Vorwürfe aus sozialen Gründen die Erwartung einer Begründung mit sich führen, sehr nachvollziehbar. Anscheinend führt diese Erwartungshaltung aber nicht so sehr zu dem Eindruck eines (bei Auslassung der Begründung) unvollständigen Diskurses, wie er sich meiner Meinung nach bei den V1-Sätzen in vielen Fällen einstellt.

Die Annahme der erwarteten Begründung ist in meiner Argumentation wichtig, um die Offenheit des Sachverhalts, der mit dem V1-/\textit{Wo}-VL-Satz ausgedrückt wird, in den Fällen zu motivieren, in denen kein konzessiver Bedeutungsanteil vorliegt. Dies betrifft – wie die Zahlen oben belegen – mehr die V1- als die \textit{Wo}-Sätze. Der Nachweis der Erwartung der Begründung ist im Rahmen meiner Argumentation allerdings in den \textit{Wo}-Sätzen auch weniger relevant als bei den V1-Sätzen. Ist Konzessivität beteiligt, ist die Offenheit über die nicht erfüllte/in Frage gestellte Erwartung zu motivieren. 

Insgesamt führen die Überlegungen und empirischen Untersuchungen zu den Gebrauchsweisen der beiden Satztypen dazu, dass ich unter der von mir vertretenen \textit{doch}-Bedeutung keinen Grund sehe, sagen zu müssen, dass nur eine uneigentliche Verwendung vorliegt, über die (durch den als solchen zu deutenden engen Kontextbezug) die kausale Verbindung zwischen den Sätzen forciert wird.

Wird in \textit{Wo}-VL- und V1-Sätzen tatsächlich die Offenheit des Themas ausgedrückt, erklärt sich auch der Ausschluss des (isolierten) Auftretens von \textit{eben}, \textit{halt} und \textit{auch}.

Der nächste Abschnitt untersucht einen letzten Aspekt, der \textit{Wo}-VL- und V1-Sätzen zugeschrieben worden ist und der leichter mit dem Beitrag von \textit{doch} als dem der hinsichtlich anderer Aspekte ggf. ebenfalls geeigneten Partikeln in Einklang zu bringen ist: eine gewisse emotionale Involviertheit.

\subsection{Expressivität/Emotionalität}
\label{sec:litdoch}
Über beide Satztypen ist gesagt worden, dass sie expressiv verstärkt sind (vgl. \citealt[204]{Oppenrieder1989} über \textit{wo}-VL- und V1-Sätze, \citeyear[42]{Oppenrieder2013} über \textit{wo}-VL-Sätze). Für V1-Sätze (insbesondere für den narrativen deklarativen Typ) wird dazu generell für Expressivität argumentiert (vgl. \citealt[218]{Reis2000}). Wenngleich man die Formulierung dieses Eindrucks findet, konnte ich weiter nicht aufdecken, was genau hinter dieser Auffassung steckt. Inwiefern diese Sätze expressiver \is{Expressivität} sind als andere und inwiefern dies möglicherweise mit ihrem Gebrauch einhergeht, wird nicht ausgeführt.

Obwohl der folgende Aspekt noch genauerer empirischer Beschäftigung bedarf, möchte ich die Annahme vertreten, dass sich sowohl in den Zeitungsdaten als auch in literarischen Beispielen Verwendungskontexte der beiden Satztypen ausmachen lassen, denen ohne Weiteres Expressivität zugeschrieben werden kann.

In DeReKo treten die \textit{Wo}-Sätze vornehmlich in Texten auf, die man dem Feuilleton zuordnen würde, wie z.B. Kolumnen. In diesem Sinne liegen recht \glq lockere\grq {} Kontexte vor. Typisch sind Fälle wie in (\ref{1048}) und (\ref{1054}).

\begin{exe}
	\ex\label{1054} 
	\scriptsize
	Was ich denn so von den Spielerfrauen auf der Tribüne halte, insbesondere von Frau Beckham? Unnatürlich und eindeutig fit gespritzt, lautet meine 			schlaue Antwort. Okay, die Klippe wäre umschifft. Warum ich denn kein Autogramm von Lukas Podolski mitgebracht habe, mischt sich unsere Tochter ein. 		Die hat es nötig, denke ich. \textbf{\textit{Wo} sie \underline{doch} sonst jeden oberpeinlich-uncool findet, der nur das Wort Fußball in den Mund 			nimmt.}
	\hfill\hbox {(BRZ06/JUL.01343 Braunschweiger Zeitung, 04.07.2006)}
\end{exe}
Ein Ich-Erzähler berichtet von einem Erlebnis und empört sich oftmals. Dies geht einher mit der Beobachtung aus dem letzten Abschnitt, dass \textit{Wo}-Sätze oftmals auf Beschwerden und Vorwürfe folgen. Diese Einstellung kann auch durchaus performativ zum Ausdruck gebracht werden (vgl. (\ref{1055}), (\ref{1056})).

\begin{exe}
	\ex\label{1055} 
	\scriptsize
	Und wenn wir ihr irgendetwas ganz Besonderes zum Valentinstag schenken möchten, dann ist das nicht weniger als die Bereitschaft unserer Arme, sie in 		diese zu nehmen und zu schützen – vor der bösen Männer-Welt mit ihrer herzlosen Hämoglobin-Häme.\\
	\emph{16,3!} \textbf{\textit{Wo} \underline{doch} jeder weiß, wie dieser Wert zustande kam} – durch den Überschuss an roten Valentinsherzchen, die so 		viel Liebe nicht mehr ertrugen und sich deshalb alle in rote Blutkörperchen verwandelten. Wie die Valentinsherzchen in Evis süßen Körper gelangten? 
	\newline              		
	\hbox{}\hfill\hbox {(M06/FEB.12331 Mannheimer Morgen, 14.02.2006; Blutherzchen)}
\end{exe}

\begin{exe}
	\ex\label{1056} 
	\scriptsize
	Ein Hockey-Trainer im DFB, \emph{igitigitt!}\\
	\textbf{\textit{Wo} dieser Verband \underline{doch} Koryphäen der Ausbildung wie den von Sportdirektor Matthias Sammer, welcher statt Peters das Rennen 	machte, aus dem Ruhestand reaktivierten Erich Rutemöller vorzuweisen hat.}   
	\hfill\hbox {(M06/SEP.76347 Mannheimer Morgen, 29.09.2006)}
\end{exe}
Es finden sich auch \textit{Wo}-Sätze, deren Interpunktion zusätzlich eine expressive Intonation nahelegt (vgl. (\ref{1057}), (\ref{1058})).

\begin{exe}
	\ex\label{1057} 
	\scriptsize
	Julia Fischer, gerade mal 25 Jahre alt, ist die bestgelaunte Geigerin der Welt und strahlt mit dem charmantesten Lächeln. Zu diesem positiven Befund 		kommen wir nach ihrem Gastspiel, das jüngst und angeblich im Heidelberger Frühling stattfand. \textbf{\textit{Wo} \underline{doch} noch tiefer Winter 		ist!} 	  
	\newline              		
	\hbox{}\hfill\hbox {(M09/JAN.07195 Mannheimer Morgen, 28.01.2009)}
\end{exe}

\begin{exe}
	\ex\label{1058} 
	\scriptsize
	Sein Pech: Zugleich setzt sich sein Sohn samt Braut per Ballon in den Westen ab, und der Vater hatte noch ahnungslos die Materialien beschafft. Jetzt 		steht er als Fluchthelfer da, im Westen gefeiert, im Osten verdammt. \textbf{\textit{Wo} er \underline{doch} so gern zurück zu seiner Braut will!} 	  
	\newline              		
	\hbox{}\hfill\hbox {(M07/OKT.00333 Mannheimer Morgen, 02.10.2007)}
\end{exe}	
Wenn die Einstellung, die durch den \textit{Wo}-Satz begründet wird, nicht performativ vorliegt, treten Ausdrücke auf, die \is{exklamatives Prädikat} unter \textit{exklamative} (\citealt[39-40]{Avis2001}) (z.B. \textit{überrascht}/\textit{fasziniert sein}, \textit{komisch}/\textit{nicht normal}/\textit{erschreckend finden}) oder auch \is{emotiv-faktives Prädikat} \textit{emotiv faktive} (\citealt[363]{Kiparsky1970}) (z.B. \textit{sich freuen}/\textit{ärgern}, \textit{ge\-nervt}/\textit{traurig sein}) Prädikate fallen. Letztere sind affektiv aufgeladen, weil eine subjektive Einschätzung der Proposition vorgenommen wird. Bei Exklamativsätzen kommt die angenommene Expressivität durch die Wahrnehmung des Unerwarte\-ten/der Normabweichung/nicht erfüllten Erwartung zustande. Da genau diese Einstellung in den konzessiven \textit{Wo}-Fällen beteiligt ist, treten auch entsprechende Prädikate häufig auf.

In anderen Fällen wird der Leser direkt und im Falle hier auch typischerweise auftretender rhetorischer Fragen dazu provokativ angesprochen (vgl. (\ref{1046}) und (\ref{1059})).

\begin{exe}
	\ex\label{1059} 
	\scriptsize
	Nun stellt sich die Frage, wie herzlos Seehund-Mamas sein müssen, um ihre erst wenige Tage alten Sprösslinge zurückzulassen? Klar, so ein Junges 			überfordert eine allein erziehende Seehund-Mama schon mal. Dauernd das Genörgel nach frischem Fisch, das fiese Kitzeln der kleinen Barthaare beim 			Säugen. Aber sind das Gründe, ein Baby auszusetzten? \textbf{\textit{Wo} es \underline{doch} (noch) keine Seehund-Babyklappen gibt.}  
	\hfill\hbox {(HMP06/JUN.00808 Hamburger Morgenpost, 09.06.2006)}
\end{exe}
Die vorweggehenden, zu begründenden Einstellungen sind im Falle der \textit{Wo}-Sätze folglich selbst deutlich emotional aufgeladen. Ist die begründete Einstellung affektiv aufgeladen, scheint es unnatürlich, dass eine rationale, neutrale oder indifferente Begründung folgt.

Wie in Abschnitt~\ref{sec:transdoch} gesehen, sind die Begründungskontexte der V1-Sätze nicht identisch mit denen der \textit{Wo}-Sätze. Exklamative Prädikate oder entsprechende performative Varianten treten eher wenig auf. Emotive Einstellungen sind auch vertreten. Das Gros der Fälle machen aber epistemische Einschätzungen aus (d.h. Annahmen, Vermutungen, Hypothesen, Überlegungen) (wie z.B. in (\ref{1060}) und(\ref{1061})).

\begin{exe}
	\ex\label{1060} 
	\scriptsize
	Der 73-Jährige, der über 19 Jahre lang an der Spitze des Olympischen Comités fungierte und auch Österreichs einziges Mitglied im Internationalen 			Olympischen Comité (IOC) ist, hätte sich seinen Abgang aber \emph{sicher} anders vorgestellt. \textbf{\textit{Galt} \underline{doch} gerade der 			ehemalige Casino-General seit Jahren als die graue Eminenz im heimischen Sport} – als ein Sir schlechthin. 
	\newline              		
	\hbox{}\hfill\hbox {(NON09/SEP.03193 Niederösterreichische Nachrichten, 07.09.2009)}
\end{exe}
				 
\begin{exe}
	\ex\label{1061} 
	\scriptsize
	Früher ein Duell mit unsicherem Ausgang, \emph{dürfte} die Sache für die Öko-Ritter aus Güssing auch auswärts kein Problem sein. \textbf{\textit{Liegt} 	die Vogler-Truppe \underline{doch} klar auf Platz eins der Tabelle in der Hauptrunde zwei.}
	\hfill\hbox {(BVZ07/APR.00677 Burgenländische Volkszeitung, 11.04.2007)}
\end{exe}	
Die begründeten Einstellungen, die in meiner obigen Argumentation generell einen emotionaleren Kontext schaffen, sind hier folglich weniger deutlich expressiv aufgeladen als bei den \textit{Wo}-Sätzen. Doch denke ich, dass der Sprecher bei jeder modalen Begründung \is{modaler Kausalsatz} spürbarer ist als bei einer propositionalen, da diese immer subjektiviert \is{Subjektivierung} ist. Und da auch mit den V1-Sätzen keine Sachverhalte, sondern Einstellungen/Ansichten begründet werden, ist Subjektivität und damit emotionale Involviertheit sowohl in der Begründung als auch in dem Begründeten ebenfalls beteiligt.

Der Eindruck der verstärkten Expressivität der beiden Satztypen lässt sich meiner Meinung nach folglich in dem Sinne bestätigen, dass der gesamte Kontext (begründete Einstellung + Begründung) aufgrund der Tatsache, dass Einstellungen motiviert werden, eine gewisse affektive Aufladung vorweist. Man kann sich schlecht vorstellen, dass sie vor der Begründung abbricht. Die Geeignetheit der Partikel \textit{doch} lässt sich vor diesem Hintergrund so erklären, dass das als offen vorausgesetzte Thema Involviertheit beim Sprecher/Schreiber schafft, da dieser in eine bereits bestehende Diskussion einsteigt. In den kolumnenartigen Texten scheint er mir in den Dialog mit dem Leser treten zu wollen und ihn somit involvieren zu wollen. Je nach begründeter Einstellung wird er auch schon in der vorangehenden Äußerung (oder im weiteren Kontext) direkt angesprochen. Durch die Verwendung von \textit{doch} ist der Leser sofort beteiligt, weil Schreiber und Leser ein gemeinsames Konversationsthema aufweisen, zu dessen Auflösung der Sprecher/Schreiber den \glq halben\grq {} Beitrag geleistet hat und die Reaktion des Lesers erwünscht ist (auch wenn hier natürlich kein direkter Dialog zustande kommt).

Betrachtet man das Auftreten der beiden Strukturen in literarischen Daten, fällt auf, dass sie häufig (und auch dies gilt es, empirisch stichhaltiger zu machen) in Kontexten auftreten, in denen die Erzählung Einblick in das Innenleben einer Figur gewährt. Generell begegnet man sowohl V1- als auch \textit{Wo}-VL-Sätzen sehr selten in Erzähltexten. Wenn sie vorkommen, findet man sie aber m.E. auffällig in diesen Kontexten.\footnote{ Aufgrund der Seltenheit ihres Vorkommens führe ich hier teilweise auch orthografische Varianten an, die ich in den anderen Datenmengen nicht beachtet habe.} (\ref{1062}) bis (\ref{1067}) zeigt einige Beispiele.

\begin{exe}
	\ex\label{1062} 
	\scriptsize
	Steiger beobachtete die Gruppe, und wie schon oft fiel ihm auf, dass man in den Männern, ohne sich groß bemühen zu müssen, noch die Jungen sah, die sie 	einmal gewesen waren, vor allem, wenn sie sich so hemmungslos ausschütteten wie jetzt. \emph{Erstaunlich}, \textbf{\textit{wo} \underline{doch} das 		Leben und die Nachtdienste nicht spurlos an ihren Gesichtern vorbeigegangen waren} und sie in Wahrheit fette, alte Kerle waren. Die meisten machte das 		sympathischer, fand Steiger, nur Benno Krone nicht.
	\newline              		
	\hbox{}\hfill\hbox {\citet[187-188]{Horst2011}}
\end{exe}

\begin{exe}
	\ex\label{1063} 
	\scriptsize
	\glqq Erst einmal vielen Dank, Herr Richter, dass Sie sich die Zeit genommen haben, heute hierher zu kommen\grqq{}, eröffnete Kluftinger das Gespräch 		und \emph{ärgerte sich} sofort darüber, dass er den Mann vor ihm \glqq Richter\grqq{} genannt hatte, \textbf{\textit{wo} dieser \underline{doch} schon 		längst aus dem Staatsdienst ausgeschieden war.}	
	\hfill\hbox {\citet[269]{Kluepfel2012}}
\end{exe}

\begin{exe}
	\ex\label{1064} 
	\scriptsize
	\glqq Ich wünschte, es wäre nicht so.\grqq{} Ich zeige auf das Haus. \glqq Da drin sind vier Kinder, die ihre Eltern nie wiedersehen werden.\grqq{}
	\glqq In so einem Moment fragt man sich doch, wie gütig Gott eigentlich ist.\grqq{}
	\glqq Ich \emph{frage mich} da noch eine Menge anderer Dinge.\grqq{} Wie zum Beispiel \emph{warum ich immer noch Polizistin bin}, \textbf{\textit{wo} 		\underline{doch} die beiden letzten Fälle mir so schwer zugesetzt haben}. Dazu muss man wissen, dass ich meine Arbeit liebe. Tief im Innersten bin ich 		Idealistin und mag die Vorstellung, etwas bewirken zu können.	
	\hfill\hbox {\citet[40]{Castillo2012}}
\end{exe}

\begin{exe}
	\ex\label{1065} 
	\scriptsize
	\glqq Schalalala, schalalala, heey DSC!\grqq{}, hallte das Echo ohrenbetäubend von den Wänden wider, als der Zug der Linie 4 einfuhr. Bröker schätzte, 		dass mehr als 100 Menschen versuchten, sich in die Waggons der Stadtbahn zu drängen. Er wusste selbst nicht, wie er unter diejenigen geriet, die es ins 	Wageninnere schafften. Die Freude darüber dauerte jedoch nicht lange an. \textbf{Wurde er \underline{doch} von der} \textbf{Masse der anderen Fahrgäste 	derart zusammengedrückt, dass er kaum Luft bekam.}	
	\newline              		
	\hbox{}\hfill\hbox {\citet[7]{Glauche2014}}
\end{exe}

\begin{exe}
	\ex\label{1066} 
	\scriptsize
	Paradol-Kammer? Was war das? Während Maupertuis weitersprach, \emph{prägte sich} Sherlock den seltsamen Begriff genau \emph{ein}. 							\textbf{\textit{Könnte} sich das \underline{doch} noch als wichtiger und verhängnisvoller Versprecher erweisen}, der Mycroft sicher brennend 				interessieren würde. 	
	\hfill\hbox {\citet[337]{Lane2014}}
\end{exe}

\begin{exe}
	\ex\label{1067} 
	\scriptsize
	Er hatte gegen die klaren Anweisungen seines Onkels verstoßen, und er hatte das dumpfe Gefühl, dass jeder Versuch, das Ganze mit einem Hinweis auf ein 		vermeintliches Treffen mit Amyus Crowe zu erklären, mit rigoroser Härte beantwortet werden würde. Schlimmer noch: Er war in einen ordinären Faustkampf 		verwickelt worden. Und sogar schlimmer noch als das: Er hatte verloren. Das würde zwar Sherrinford und Anna Holmes wahrscheinlich nicht sonderlich 			berühren, aber wenn Sherlocks Vater jemals etwas davon mitbekäme, \emph{würde er außer sich vor Zorn sein}. \textbf{\textit{War} \underline{doch} eine 		seiner beliebtesten Redensarten:} Ein Gentleman beginnt niemals einen Kampf, sondern beendet ihn stets. 	
	\hfill\hbox {\citet[220-221]{Lane2014}}
\end{exe}
Die \textit{Wo}-Sätze werden – anders als die V1-Sätze – allerdings auch in den fiktiven Kontexten ebenfalls in direkter Rede verwendet (vgl. (\ref{1068}) und (\ref{1069})).

\begin{exe}
	\ex\label{1068} 
	\scriptsize
	\glqq Und nach uns muss noch einer da gewesen sein\grqq{}, sagt er dann weiter und gar nicht mehr so motzig wie gerade. Nein, voller Begeisterung tut 		er das kund. \glqq Ein Riesendepp muss das gewesen sein. Weil: der hat nämlich das Kellerfenster eingeschlagen. \textbf{\textit{Wo} \underline{doch} 		die Tür auf war!}\grqq{} 	
	\hfill\hbox {\citet[68]{Falk2010}}
\end{exe}

\begin{exe}
	\ex\label{1069} 
	\scriptsize
	\glqq Ja. Und danke, Mütze. Du hast mir sehr geholfen! \grqq{}\\
	\glqq Wobei auch immer, \textbf{\textit{wo} du \underline{doch} gar nicht auf Detektivpfaden wandelst}.\grqq{}
	\hfill\hbox {\citet[56]{Glauche2014}}
\end{exe}
In Fällen wie in (\ref{1062}), (\ref{1063}) und (\ref{1065}) bis (\ref{1067}) hat man es mit einer \textit{personalen Erzählweise} \is{personaler Erzähler} zu tun (vgl. \citealt[89-102]{Eicher2001}, \citealt[264-283]{Spoerl2004}, \citealt[49-50]{Klarer2009}, \citealt[114-119]{Nuenning2015} zu Erzählsi\-tuationen). Die Distanz zwischen Erzähler und Figur wird dadurch aufgegeben, dass der Erzähler hinter die Figur tritt. Das Ergebnis ist, dass der Leser durch die Figur Zugriff auf die erzählte Welt hat. Somit hat er Teil an ihren Handlungen und ihrer Wahrnehmung. In der Darstellung der Figuren überwiegt die Innen\-perspektive. Je nach Ausgeprägtheit dieser Annäherung von Erzähler und Figur, ist der Erzähler noch mehr oder weniger präsent. In allen obigen Beispielen ((\ref{1062}), (\ref{1063}) und (\ref{1065}) bis (\ref{1067})) entsteht der Eindruck, dass das \glq Geschehen\grq {} aus der Perspektive von Bröker, Kluftinger, Steiger bzw. Holmes präsentiert wird. In (\ref{1062}), (\ref{1063}) und (\ref{1065}) ist der Erzähler aber dennoch als Mittlerinstanz deutlicher zu spüren als in (\ref{1066}) und (\ref{1067}). Die beiden Textstellen lassen sich als \textit{erlebte Rede} \is{erlebte Rede} oder \textit{Gedankenbericht} \is{Gedankebericht}  auffassen. Diese Techniken der Bewusstseinsdarstellung sind typisch für eine personale Erzählweise, da der Erzähler die Gedanken einer Figur in ihrer Sprache (erlebte Rede) oder in seiner Sprache (Gedankenbericht) präsentiert. Dadurch, dass das fiktive Geschehen bei der personalen Erzählsituation aus Sicht einer Reflektorfigur \is{Reflektorfigur} wahrgenommen und verarbeitet wird, ergibt sich sehr eindeutig eine Situation, in der sowohl diese Figur als auch der Leser hochgradig involviert sind: Durch den Fokus auf die Innenperspektive nimmt die Figur ihre Umwelt wahr, denkt und fühlt: In den obigen Beispielen ärgert sie sich, staunt, hat Befürchtungen, bringt sich einen Begriff ins Bewusstsein, um ihn nicht zu vergessen bzw. entwickelt eine schlechte Laune. Die sich anschließende Begründung dieser Haltungen wird m.E. verstanden als Begründung der Figur und nicht als Bewertung eines außenstehenden Erzählers, der hier ja gerade (anders als bei auktorialem \is{auktorialer Erzähler} oder \is{neutraler Erzähler} neutralem Erzählen) mit der Figur zusammenfällt/hinter sie tritt. Wie für das Auftreten der Begründungen in den Zeitungstexten ist es auch hier nicht verwunderlich, dass in diesen sowieso schon expressiven Kontexten auch expressive Erklärungen angeführt werden. Das Anzeigen des schon offenen Themas durch \textit{doch} fügt sich gut in diese vermittelte Stimmung ein, da dadurch ausgedrückt wird, dass die Figur bzw. Erzählinstanz bereits an einer bestehenden Diskussion Teil hat und somit in die Lösung/Entscheidung des Sachverhalts eingebunden ist. Gleiches gilt auch für den Leser, für den es in dieser Erzählform ein Leichtes ist, sich mit der Figur zu identifizieren. Durch das vorausgesetzte gemeinsame offene Thema wird er als beteiligt ausgegeben, da auch ihm Anteil an der Diskussion zugeschrieben wird.

In (\ref{1064}) tritt eine \textit{Ich-Erzählsituation} auf. Der \is{Ich-Erzähler} Ich-Erzähler nimmt als Figur (als Kommissarin Kate Burkholder) an der erzählten Welt teil. Wenngleich beschränkt auf diese eine Figur, sind natürlich auch in dieser Erzählsituation Bewusstseinsdarstellungen möglich, die – aufgrund der Tatsache, dass es sich um psychische Zustände handelt – als affektiv und damit expressiv eingestuft werden können. Der Leser erhält unvermittelten Einblick in die Gedanken und Gefühle der Figur, was seine Identifikation fördert. Die oben als mit der personalen Erzählweise verbunden angeführte Stimmung der Involviertheit von Figur und Leser gilt hier folglich gleichermaßen und wird durch das durch \textit{doch} vorausgesetzte offene Thema, an dem Erzähler/Figur und Leser beteiligt sind, gefördert.

Ich gehe folglich davon aus, dass diese beiden Satztypen auch gewisse stilistische Effekte mit sich bringen (Involviertheit, Expressivität, Affektivität), die sich in der Schriftsprache im Auftreten in Zeitungsdaten in bestimmten Genres bzw. im literarischen Bereich in bestimmten Erzählsituationen niederschlagen. Wie erläutert, eignet sich \textit{doch} – setzt man den Bedeutungsbeitrag des schon zur Diskussion stehenden Themas an – gut, um die vorliegende Involviertheit und affektive Atmosphäre sowohl aus Sprecher-/Schreiber- als auch Lesersicht zu motivieren. Die anderen Partikeln, die für die Kodierung von Kausalität sogar besser in Frage kämen, können diesen Aspekt nicht auffangen. \textit{Ja} setzt Zustimmung beim Gegenüber voraus, genauso wie \textit{eben}. Der Gesprächspartner wird übergangen, weil ihm keine Einspruchsmöglichkeit gewährt wird. Eine gesteigerte Beteiligung des Sprechers/Schreibers scheint mir auch nicht ableitbar. \textit{Auch} ko\-diert – unter Annahme einer Norm/eines gesetzten Zusammenhangs – allein Kausalität und ist demnach ebenfalls kein besserer Kandidat, um den expressiven Eindruck aufzufangen. \textit{Halt} ist zwar enger sprechergebunden (da der kausale Zusammenhang allein Annahme des Sprechers ist), die Involviertheit des Hörers folgt allerdings nicht. Auch aus der Perspektive einer vorliegenden gesteigerten Expressivität dieser Sätze scheint das \textit{doch} gut geeignet. Dieser Aspekt bietet somit einen weiteren Beitrag zu der Erklärung seines obligatorischen bzw. sehr typi\-schen Vorkommens in diesen Sätzen.
	
Der grundsätzliche Punkt, der aus meiner Untersuchung von \textit{Wo}-VL- und V1-Sätzen resultiert, ist folglich, dass \textit{doch} in diesen Satztypen transparent verwendet wird. Das Thema wird als offen angezeigt. Dies hat zwar indirekt die Funktion, die kausale Lesart zu forcieren (enger Kontextbezug durch Reaktion auf offene Frage). Man muss aber nicht annehmen, dass dies nur eine völlig uneigentliche Verwendung von \textit{doch} ist, die hier ausschließlich indirekt wirkt. Aus der Konzessivität und Kausalität, die aufgrund der beteiligten Sachverhalte/Ein\-stellungen vorliegen, lässt sich die Offenheit des Themas in der Begründung motivieren. Und schließlich lässt sich über das offene Thema, das zwischen Sprecher/Schreiber und Leser besteht, auch die nachzuweisende Expressivität kodieren.

Für meine Ableitung der Abfolge von \textit{doch} und \textit{auch} bedeutet dies, dass sie auch in diesen Satztypen gelten kann. Prinzipiell mache ich im Rahmen meiner Erklärung keinen Unterschied zwischen verschiedenen assertiven Satztypen. Auch wenn der Gesamtsatz wie hier ein Kausalsatz ist, gehe ich davon aus, dass wenn diese beiden MPn auftreten, ihr Gebrauch auch beabsichtigt ist, so dass meine Erklärung die gleiche sein kann wie in allen anderen Assertionen. Die Adressierung des Themas steht über einer qualitativen Bewertung. Welche weitere Interpretation die Äußerung hat, nimmt auf meine Ableitung zunächst keinen Einfluss.\footnote{Wie auch schon in Abschnitt~\ref{sec:markiert} in Kapitel~\ref{chapter:jud} argumentiert, gehe ich allerdings davon aus, dass spezielle interpretatorische Eigenschaften von Äußerungen Einfluss auf die Abfolgen nehmen können, da sich für bestimmte Bedeutungsanteile, auf denen meine Ableitung der Abfolgepräferenzen basiert, unterschiedliche Gewichtungen einstellen können und sich somit Verhältnisse verschieben können. Ich werde dies in Abschnitt~\ref{sec:distributionad} auch für Kausalsätze annehmen.}

Ausgangspunkt der Betrachtung dieser Deklarativsätze war in Abschnitt~\ref{sec:eig} gerade, dass – auf der Basis einer anderen Bedeutungszuschreibung an \textit{doch} – angenommen worden ist, dass die \textit{doch}-Bedeutung hier nicht transparent vorliegt. Es ist somit dieser Aspekt, der diese beiden Satztypen interessant macht, und nicht die Tatsache, dass sie kausal interpretiert werden. Da das Thema meiner Argumentation nach aber auch in diesen Satzkontexten tatsächlich offen ist, gehe ich auch hier davon aus, dass für die Partikeln untereinander gilt, dass das Anzeigen der Thema-Adressierung dem Anzeigen von Kausalität übergeordnet ist, und zwar auch, wenn sich diese Kodierungen innerhalb eines kausalen Satzes abspielen. Dies ließe sich schwieriger annehmen, wenn \textit{doch} seine Bedeutung hier gar nicht aufwiese. Die Adressierung des Themas hat in diesem speziellen Fall sogar mehr Relevanz als sonst, da man über diese auch eine andere Eigenschaft der Sätze ableiten kann (Expressivität). Den Sätzen ist es wichtig, dass das Thema adressiert wird, die Hinzufügung von \textit{auch} ist anschließend möglich, da die weitere Auszeichnung des Zusammenhangs zwischen den Sätzen als erwartet natürlich denkbar ist. Für die Relevanz des Anzeigens des offenen Themas spricht auch, dass \textit{auch} zwar hinzutreten kann, es aber sehr schlecht (wenn nicht gar nicht) allein auftreten kann. Wie erläutert, ist es nicht das einzige Ziel dieser Sätze, kausal interpretiert zu werden, was darüber hinaus eben auch schon gegeben ist, wenn \textit{doch} allein vorkommt. 

Vor dem Hintergrund meiner Ausführungen in Kapitel~\ref{chapter:hue} ist mir wichtig, dass \textit{doch} nur indirekt für die Kausalität verantwortlich ist. Andernfalls hätte man es auf der Ebene von durch MPn kodierter Kausalität mit einem Fall von Redundanz zu tun, den ich an anderer Stelle in MP-Kombinationen ausschließe. Ein Aspekt kann zunächst indirekt (\textit{doch}) und anschließend erneut direkt (\textit{auch}) kodiert werden. Die umgekehrte bzw. doppelt direkte Vermittlung wird hingegen als abwei\-chend eingeschätzt, da die Darstellung redundant erfolgt.\\

\noindent
Abschnitt~\ref{sec:distributionda} hat gezeigt, dass die Kombination \textit{doch auch} eine weite Verwendung in verschiedenen Äußerungstypen hat. Der folgende Abschnitt beschäftigt sich genauer mit Direktiven.
\setcounter{equation}{0}
\section{Direktive}
\label{sec:direktive}
Im Anschluss an die These um \is{Bedeutungsminimalismus/-maximalismus} Bedeutungsminimalismus (wie in Kapitel~\ref{chapter:hue} auch schon zu \textit{halt} und \textit{eben} vertreten) gehe ich – sofern keine dem widersprechenden Verhältnisse festzustellen sind – davon aus, dass in Direktiven der gleiche Beitrag für die MPn anzusetzen ist wie in Assertionen. Im Rahmen meiner Modellierung bedeutet dies, dass der geforderte vorweggehende Kontextzustand der gleiche ist. Unter Berücksichtigung von deskriptiven Eindrücken aus der Literatur und authentischen Belegen werde ich aufzeigen, wie eine diskursstrukturelle Modellierung die Interpretation von \textit{doch}-, \textit{auch}- und \textit{doch auch}-Direktiven abbilden kann. Wenngleich ich dafür argumentiere, dass der Partikelbeitrag beibehalten wird, ändert sich natürlich der mit dem Sprechakt ausgeführte Kontextwechsel. Dies übersetzt sich in meine Modellierung in die Füllung anderer Diskurskomponenten. 

In Abschnitt~\ref{sec:dirdm} in Kapitel~\ref{chapter:hue} habe ich die Diskursmodellierung aus \citet{Farkas2010} erweitert, um direktive Äußerungen auffangen zu können. Die entscheidende Komponente ist in dieser Erweiterung die To-Do-Liste (TDL) \is{To-Do-Liste} eines jeden Diskursteilnehmers, die die Propositionen enthält, zu deren Realisierung er sich bekannt hat. Notationell habe ich die unterschiedliche Qualität der Inhalte der Diskursbekenntnismengen (DC$_{\textrm{X}}$) \is{Diskursbekenntnismenge} bzw. des cg \is{Common Ground} und TDL$_{\textrm{X}}$ dadurch aufgefangen, dass !p in TDL enthalten ist. (\ref{1070}) bis (\ref{1072}) zeigt die Kontextveränderungen, die im Zuge der Äußerung eines Direktivs eintreten. 
\pagebreak
\newcolumntype{C}[1]{>{\centering}p{#1}}
\begin{exe}
\ex\label{1070} K$_1$: Kontextzustand K$_{1}$ nach Äußerung eines Direktivs\\[-0.6em]
\begin{tabular}[t]{|C{6em}|C{12em}|C{6em}|}
\hline
$\textrm{DC}_{\textrm{A}}$ & Tisch &  $\textrm{DC}_{\textrm{B}}$ \tabularnewline
\hline
!p $\in$ $\textrm{TDL}_{\textrm{B}}$ & p $\vee$ $\neg$p & {}  \tabularnewline
\cline{1-1}\cline{3-3}
$\textrm{TDL}_{\textrm{A}}$ & {} & $\textrm{TDL}_{\textrm{B}}$  \tabularnewline
\cline{1-1}\cline{3-3}
{} & !p $\in$ $\textrm{TDL}_{\textrm{B}}$ $\vee$ $\neg$(!p $\in$ $\textrm{TDL}_{\textrm{B}}$) & {}  \tabularnewline
\hline
\multicolumn{3}{|l|}{cg s$_{1}$} \tabularnewline
\hline
\end{tabular}
\end{exe}
				                   
\begin{exe}
\ex\label{1071} K$_1$: Kontextzustand K$_{2}$ nach Äußerung eines Direktivs\\[-0.6em]
\begin{tabular}[t]{|C{6em}|C{12em}|C{6em}|}
\hline
$\textrm{DC}_{\textrm{A}}$ & Tisch &  $\textrm{DC}_{\textrm{B}}$ \tabularnewline
\hline
!p $\in$ $\textrm{TDL}_{\textrm{B}}$ & p $\vee$ $\neg$p & !p $\in$ $\textrm{TDL}_{\textrm{B}}$  \tabularnewline
\cline{1-1}\cline{3-3}
$\textrm{TDL}_{\textrm{A}}$ & {} & $\textrm{TDL}_{\textrm{B}}$  \tabularnewline
\cline{1-1}\cline{3-3}
{} & !p $\in$ $\textrm{TDL}_{\textrm{B}}$ $\vee$ $\neg$(!p $\in$ $\textrm{TDL}_{\textrm{B}}$) & !p  \tabularnewline
\hline
\multicolumn{3}{|l|}{cg s$_{2}$ = s$_{1}$} \tabularnewline
\hline
\end{tabular}
\end{exe}		

\begin{exe}
\ex\label{1072} K$_1$: Kontextzustand K$_{3}$ nach Äußerung eines Direktivs\\[-0.6em]
\begin{tabular}[t]{|C{6em}|C{12em}|C{6em}|}
\hline
$\textrm{DC}_{\textrm{A}}$ & Tisch &  $\textrm{DC}_{\textrm{B}}$ \tabularnewline
\hline
{} & p $\vee$ $\neg$p & {}  \tabularnewline
\cline{1-1}\cline{3-3}
$\textrm{TDL}_{\textrm{A}}$ & {} & $\textrm{TDL}_{\textrm{B}}$  \tabularnewline
\cline{1-1}\cline{3-3}
{} & {} & !p  \tabularnewline
\hline
\multicolumn{3}{|l|}{cg s$_{3}$ = $\lbrace$s$_{2}$ $\cup \ \lbrace$!p $\in$ $\textrm{TDL}_{\textrm{B}}\rbrace\rbrace$} \tabularnewline
\hline
\end{tabular}
\end{exe}	
Äußert A einen Direktiv \is{Direktiv} adressiert an B, geht A davon aus, dass !p zum Element von Bs TDL wird. Da B ablehnen kann, !p in seine TDL aufzunehmen, steht zunächst die Frage im Raum, ob !p auf seine TDL gelangt.\footnote{Streng genommen geht es hier jeweils um den nächsten Zustand von TDL$_{\textrm{B}}$}.
Dies gilt parallel zur Einspruchsmöglichkeit bei der Äußerung einer Assertion, von deren Inhalt der Sprecher den Adressaten zu überzeugen beabsichtigt. Nach der Äußerung des Direktivs steht ebenfalls im Raum, ob p eintritt (vgl. (\ref{1070})). Da p das erfüllte !p ist, hängt die Entscheidung der Frage p $\vee$ $\neg$p davon ab, ob B p/$\neg$p realisiert. Die Auflösung dieser Unentschiedenheit ist somit unabhängig von Bs Reaktion zum Direktiv. D.h. p $\vee$ $\neg$p verbleibt auch auf dem Tisch, wenn B !p nicht in seine TDL aufnimmt. Dies wäre gleichbedeutend damit, dass er !$\neg$p in seine TDL einfügt. Akzeptiert B den Direktiv, gelangt !p auf seine TDL und er bekennt sich dazu, dass !p in seiner TDL enthalten ist (vgl. (\ref{1071})). Als Folge wird es ein cg-Inhalt, dass !p auf Bs TDL steht (vgl. (\ref{1072})). Die Frage, ob p zutrifft, entscheidet sich, wenn B !p realisiert und p damit wahr wird. Wenn sich eine Proposition in TDL befindet, steht folglich auch immer ihre Realisierung im Raum.\footnote{Für Assertionen habe ich nicht angenommen, dass immer das Thema zur Debatte steht, wenn in einem DC-System eine Proposition enthalten ist. Der Grund hierfür ist, dass man sich in diesem Fall auch einig sein kann, sich nicht einig zu sein. Solange TDL gefüllt ist, hat der Diskursteilnehmer diese Proposition noch nicht realisiert. Deshalb gehe ich davon aus, dass ihre Realisierung offen ist und demnach noch keine Einigkeit dahingehend besteht, ob p gilt oder nicht.} 

Ich denke, die Modellierung in (\ref{1070}) bis (\ref{1072}) kann Gemeinsamkeiten und Unterschiede zwischen Assertionen \is{Assertion} und Direktiven \is{Direktiv} abbilden. In beiden Fällen macht der Sprecher einen Vorschlag, von dem er ausgeht und der im Diskurs generell gelten soll, der aber von der Akzeptanz durch den Hörer abhängt. Bei Assertionen handelt es sich hierbei um eine Annahme p, von der der Sprecher möchte, dass der Hörer sie bestätigt, damit p cg wird. Das Ziel ist erfüllt, sobald der Hörer p akzeptiert. Bei Direktiven ist !p beteiligt, das der Sprecher vom Hörer angenommen sehen will, um das letztliche Ziel zu erreichen, dass !p realisiert und p damit wahr wird, so dass p schließlich auch in den cg gelangt. Da p eine noch zu rea\-lisierende Proposition ist, hat man es hier mit einem zusätzlichen Schritt zu tun: Der Hörer akzeptiert zunächst einmal nur, dass er !p auf seine TDL setzt. p selbst entscheidet sich erst, wenn er die Handlung ausführt und der Sprecher dies als Realisierung ansieht. Sowohl bei Assertionen als auch Direktiven besteht eine Einspruchsmöglichkeit: Bei ersteren ist der Einspruch direkt damit verbunden, p nicht zuzustimmen, bei letzteren entscheidet die Weigerung, die TDL mit !p anzureichern, noch nicht über die Zukunft von p im Diskurs. 

Treten MPn in Direktiven auf, liegt ein beschränkterer Kontextzustand vor. Wie alle MPn in meiner Modellierung fordern \textit{doch} und \textit{auch} in Direktiven einen bestimmten Zustand im Kontext, damit die \textit{doch}-, \textit{auch}- und \textit{doch auch}-Direktive angemessen geäußert werden können. 

\subsection{Das Einzelauftreten von \textit{doch}}
\subsubsection{Der Diskursbeitrag von \textit{doch}}
Die speziellen Verwendungen, Effekte und Nuancen von \textit{doch}-Direktiven (s.u.) können meiner Meinung nach darauf zurückgeführt werden, dass in dem Kontext, in dem der Direktiv geäußert wird, p $\vee$ $\neg$p bereits auf dem Tisch liegt. Die potenzielle Realisierung des Sachverhalts, zu dem B im Anschluss aufgefordert wird, steht in Frage und ist damit salient/steht aktuell im Raum (vgl. (\ref{1073})).
\pagebreak
\begin{exe}
\ex\label{1073} Kontextzustand vor der Äußerung eines \textit{doch}-Direktivs\\[-0.6em]
\begin{tabular}[t]{|C{6em}|C{12em}|C{6em}|}
\hline
$\textrm{DC}_{\textrm{A}}$ & Tisch &  $\textrm{DC}_{\textrm{B}}$ \tabularnewline
\hline
{} & p $\vee$ $\neg$p & {}  \tabularnewline
\cline{1-1}\cline{3-3}
$\textrm{TDL}_{\textrm{A}}$ & {} & $\textrm{TDL}_{\textrm{B}}$  \tabularnewline
\cline{1-1}\cline{3-3}
{} & {} & {}  \tabularnewline
\hline
\multicolumn{3}{|l|}{cg s$_{1}$} \tabularnewline
\hline
\end{tabular}
\end{exe}
\textit{Doch} steuert hier folglich den gleichen Beitrag bei wie in Assertionen (vgl. auch \citealt[92]{Diewald1998}, die für \textit{doch} in Direktiven ebenfalls prinzipiell das gleiche Schema ansetzen wie in Assertionen und die Einstellung ändern zu \textit{Ich will: p} (vs. \textit{Ich glaube: p})) (vgl. auch Kapitel~\ref{chapter:jud}, Abschnitt~\ref{sec:mpn}). Ich bin der Meinung, dass dies als Minimalanforderung \is{Bedeutungsminimalismus/-maximalismus} ausreicht. Wie bei den assertiven Kontextwechseln kann es aber wiederum diverse Szenarien geben, \underline{warum} das Thema rund um diese zur Realisierung ausstehende Proposition schon offen ist (und nicht erst durch die Äußerung des Direktivs eingeführt wird) und der Sprecher B in diesem Zusammenhang zur Realisierung eines der beiden Sachverhalte anhält.

Es ist schwieriger, Belege für \textit{doch}-Direktive in Korpora zu finden als für \textit{doch}-Assertionen. Ich präsentiere im Folgenden drei Arten der Verwendung, die ich in Korpora und literarischen Texten habe ausmachen können. Ich erhebe keinen Anspruch auf Vollständigkeit, es scheint sich aber um typische Verwendungen zu handeln. Entscheidenderweise lässt sich der in (\ref{1073}) beschriebene Kontextzu\-stand als minimale Anforderung nachweisen, da er in allen drei Verwendungen vorliegt.

In der ersten Gebrauchsweise wird die Aufforderung zum wiederholten Male erteilt, was involviert, dass B dem erteilten Auftrag bisher nicht nachgegangen ist und ihn beispielsweise ablehnt oder verzögert. In (\ref{1074}) beantwortet Adam die erste Frage nach seinem Aufenthalt zunächst nicht und weist in diesem Sinne die erste Aufforderung, die Angabe zu machen, zurück. 

\begin{exe}
	\ex\label{1074} 
	\scriptsize
	Adam muss uns kommen gehört haben, denn er rollt sich gerade unter dem Traktor heraus und steht auf. „Chief Burkholder.“ Sein Blick gleitet zu Pickles 		und wieder zurück zu mir.\\ 
	\glqq Ich hatte nicht erwartet, Sie so schnell wiederzusehen. Ist alles in Ordnung?\grqq{}\\
	Ich sehe ihn offen an. \glqq Wo waren Sie letzte Nacht und heute Morgen?\grqq{}\\
	Er tritt einen Schritt zurück, als wolle er von etwas Unschönem Abstand gewinnen. \glqq Warum fragen Sie das?\grqq{}\\
	\glqq \textbf{Beantworten Sie \underline{doch} bitte einfach die Frage.}\grqq{}\\
	\glqq Ich war hier auf der Farm.\grqq{}\\
	\glqq War jemand bei Ihnen?\grqq{}	
	\hfill\hbox {\citet[105]{Castillo2012}}
\end{exe}
Nach der ersten Frage von Burkholder steht im Raum p $\vee$ $\neg$p (Beantwortet er die Frage oder nicht?). Durch Adams Reaktion auf die Frage (entgegen der Vorstellung des Fragenden) wird !$\neg$p in seiner TDL verankert, denn (zumindest zeitweise) weigert er sich, die Antwort zu geben (vgl. (\ref{1075})).

\begin{exe}
\ex\label{1075} Burkholder: \glqq Wo waren sie letzte Nacht und heute Morgen?\grqq{}\\
				Adam: \glqq Warum fragen Sie das?\grqq{}\\[-0.6em]
\begin{tabular}[t]{|C{6em}|C{14em}|C{6em}|}
\hline
$\textrm{DC}_{\textrm{Burkholder}}$ & Tisch &  $\textrm{DC}_{\textrm{Adam}}$ \tabularnewline
\hline
!p $\in$ $\textrm{DC}_{\textrm{Adam}}$ & p $\vee$ $\neg$p & !$\neg$p $\in$ $\textrm{DC}_{\textrm{Adam}}$  \tabularnewline
\cline{1-1}\cline{3-3}
$\textrm{TDL}_{\textrm{A}}$ & {} & $\textrm{TDL}_{\textrm{B}}$  \tabularnewline
\cline{1-1}\cline{3-3}
{} & !p $\in$ $\textrm{DC}_{\textrm{Adam}}$ $\vee$ $\neg$(!p $\in$ $\textrm{DC}_{\textrm{Adam}}$) & !$\neg$p  \tabularnewline
\hline
\multicolumn{3}{|l|}{cg s$_{1}$} \tabularnewline
\hline
\end{tabular}
\end{exe}
Burkholder gibt sich mit diesem Zustand nicht zufrieden. Wenn er bestehen bleibt, wird p im Kontext wahrscheinlich nicht wahr. Aus diesem Grund reagiert sie erneut auf das ohnehin schon offene Thema, indem sie die Aufforderung zur Beantwortung der Frage explizit ausspricht, und beabsichtigt, !$\neg$p auf Adams TDL durch !p zu überschreiben.

Ein ähnlich gelagertes Beispiel ist (\ref{1076}).

\begin{exe}
	\ex\label{1076} 
	\scriptsize
	Mit freundlicher Zurückhaltung bat sie die beiden Polizisten ins Haus. Diese aber machten keine Anstalten, auszusteigen.\\
	\glqq Bitte treten Sie ein, meine Herren!\grqq{}, wiederholte sie ihre Aufforderung.\\
	Wieder aber erkannte sie keine Regung bei den Beamten, die nun demonstrativ in Richtung des Hundes sahen.\\
	Hiltrud Urban aber schien nicht zu verstehen und bat nun, schon etwas ungeduldig: \glqq \textbf{Ja kommen Sie \underline{doch} herein!}\grqq{} 		
	\hfill\hbox {\citet[221]{Kluepfel2012}}
\end{exe}
Auch hier ist die Aufforderung schon erteilt worden, so dass p $\vee$ $\neg$p im Raum steht. Der Adressat zeigt Anzeichen dafür, dass er die gegenteilige Handlung beabsichtigt, weshalb der Direktiv wiederholt wird.

Dass der Adressat Gegenteiliges vorzuhaben scheint, muss allerdings nicht dadurch offenbart werden, dass der Direktiv bereits (mehrfach) nicht angenommen wurde. Die Offenheit des zu realisierenden Sachverhalts kann auch dadurch zustandekommen, dass sich der Adressat aktuell entgegengesetzt zum vom Spre\-cher gewünschten Zustand verhält, so dass nicht damit zu rechnen ist, dass der vom Sprecher gewünschte Sachverhalt eintreten wird bzw. (schwächer) unklar ist, ob der Sachverhalt realisiert werden wird. Belege, die ich unter diesen Fall fasse, zeigen, dass das Ausleben des gegenteiligen Verhaltens unterschiedlich deutlich sein kann.

In (\ref{1077}) beispielsweise gilt im Kontext vor dem \textit{doch}-Direktiv der Sachverhalt, dass die Mutter jodelt (diese Proposition q ist im cg enthalten $[$vgl. (\ref{1078})$]$).
		
\begin{exe}
	\ex\label{1077} 
	\scriptsize
	Wie typische Jodlerinnen kommen die 38jährige Claudia Städler und die 36jährige Andrea Haffa nicht daher. Wenn sie aber ihre Leidenschaft zum Jodeln 		beschreiben, geraten die beiden ins Schwärmen. Eine Herzenssache sei es, etwas, das tief aus der Seele komme, \glqq voll einfahre\grqq{} und dem 			Publikum die Tränen in die Augen treibe. Claudia Städler sah das als Mädchen noch nicht so und tat sich mit dem Jodeln schwer. \glqq Hör doch auf			\grqq{}, habe sie ihre Mutter jeweils angefleht, wenn diese jodelnd in die Stube gekommen sei. Später habe sie sich einen Walkman gewünscht, um bei 		Autofahrten mit der Familie der ungeliebten Jodelmusik zu entkommen.		
	\hfill\hbox {(A10/APR.00228 St. Galler Tagblatt, 01.04.2010)}
\end{exe}		
 	    
\begin{exe}
\ex\label{1078} Kontextzustand vor dem \textit{doch}-Direktiv\\[-0.6em]
\begin{tabular}[t]{|C{6em}|C{12em}|C{6em}|}
\hline
$\textrm{DC}_{\textrm{A}}$ & Tisch &  $\textrm{DC}_{\textrm{B}}$ \tabularnewline
\hline
{} & p $\vee$ $\neg$p & !$\neg$p $\in$ $\textrm{DC}_{\textrm{B}}$  \tabularnewline
\cline{1-1}\cline{3-3}
$\textrm{TDL}_{\textrm{A}}$ & {} & $\textrm{TDL}_{\textrm{B}}$  \tabularnewline
\cline{1-1}\cline{3-3}
{} & {} & !$\neg$p  \tabularnewline
\hline
\multicolumn{3}{|l|}{cg s$_{1}$ = $\lbrace$q, q $>$ (!$\neg$p $\in$ $\textrm{TDL}_{\textrm{B}}$)$\rbrace$} \tabularnewline
\hline
\end{tabular}
\end{exe}
Es ist davon auszugehen, dass wenn sie in der aktuellen Situation jodelt, sie im naheliegenden zukünftigen Zustand nicht nicht jodeln wollen wird, d.h. sie wird vorhaben, weiterzujodeln ((q $>$ (!$\neg$p $\in$ $\textrm{DC}_{\textrm{B}}$)) ist cg). Deshalb ist !$\neg$p in $\textrm{TDL}_{\textrm{B}}$ und !$\neg$p $\in$ $\textrm{TDL}_{\textrm{B}}$ in DC$_{\textrm{B}}$ enthalten. Folglich steht im Raum, ob p oder $\neg$p realisiert wird (mit einer Voreingenommenheit in Richtung $\neg$p). Die Tochter möchte p, weshalb sie zu seiner Verwirklichung aufruft.\\

\noindent
In (\ref{1079}) arbeitet der Sprecher an seiner Antwort. Der Adressat tut nichts/zeigt keine Reaktion, das zu tun, wozu er anschließend aufgefordert wird und was auch naheliegend ist, wenn jemand etwas allein nicht zu schaffen scheint. Ihm kann deshalb auch zugeschrieben werden, dass er das Gegenteil vorhat, so dass aus diesem Grund offen ist, was realisiert wird.

\begin{exe}
	\ex\label{1079} 
	\scriptsize
	Dann könnte ich zum Beispiel. Wenn man zurückrechnet. Wenn ich um halb Zwölf dort bin, viereinhalb Stunden, heijeijei, wäre? Halb zwölf? \textbf{Helfen 	Sie mir \underline{doch} bitte.}\\
	\noindent		
	Ja, da sollten Sie um sieben Uhr drei auf jeden Fall fahren.		
	\newline              		
	\hbox{}\hfill\hbox {(Tübinger Baumbank des Deutschen/Spontansprache)}
\end{exe}
Die Situation ist in diesem Verwendungsfall von \textit{doch}-Direktiven im Grunde die gleiche wie bei denjenigen \textit{doch}-Assertionen, bei denen aus einer anderen Proposition bzw. einem Verhalten des Angesprochenen auf sein Bekenntnis zu $\neg$p geschlossen werden kann. Aufgrund dessen steht p $\vee$ $\neg$p im Raum (mit einer Voreingenommenheit zu $\neg$p) und der Sprecher vertritt die entgegengesetzte Proposition, die er durch Umstimmung des Adressaten zu einem cg-Inhalt machen möchte. Im Falle des Direktivs ist aus Bs Verhalten abzuleiten, dass er ¬p beabsichtigt und deshalb die Realisierung von p $\vee$ $\neg$p offen ist. Sofern sich nichts ändert, wird aber vermutlich $\neg$p realisiert werden. Da der Sprecher Gegenteiliges wünscht, fordert er zu !p auf, um den Adressaten umzustimmen und die Realisierung von p herbeizuführen.

Die Zuschreibung der gegenteiligen Absicht kann auch konventionalisiert sein. 
	
\begin{exe}
	\ex\label{1080} 
	\scriptsize
    \begin{tabular}[t]{lll}
	0001 & & (1.95)\\
	0002 &	XM & fritzsche\\
	0003 & MF & ja\\
	0004 & XM &	\textbf{nehmen sie \underline{doch} ? $[$platz} (.) ihre$]$ prüfer kennen sie\\
	0005 & MF &	$[$schuldigung h$^{0}$ $]$\\
	0006 & & (0.24)\\
	0007 & MF & ja\\
	0008 & XM & ((schmatzt)) ja (.) ich bin tamara hackel bin die prüfungsvorsitzende
	\end{tabular}
	\newline
	\hbox{}\hfill\hbox{(FOLK\_E\_00059\_SE\_01\_T\_01)} 					        
\end{exe}
Ein Direktiv wie in (\ref{1080}) kann geäußert werden, wenn jemand wirklich zögert, sich hinzusetzen, was in Kontexten, in denen eine solche Äußerung gemacht wird, auch durchaus vorkommen kann (weil jemand z.B. unschlüssig im Raum steht, sich ziert, sich hinzusetzen). In den meisten Fällen wird er aber als Floskel verwendet werden, womit meiner Meinung nach aber nicht einhergeht, dass die \textit{doch}-Bedeutung nicht transparent zu erfassen ist: Der Sprecher geht davon aus, dass der Hörer sich nicht von allein hinsetzen wird (qua Konvention), obwohl im Kontext klar ist, dass einmal der sitzende Zustand bestehen soll. Beratungsgespräche, Prüfungen (wie in (\ref{1080})) und Besuche finden i.d.R. nicht im Stehen statt. Man kann hier aber nicht unterstellen, dass der Adressat wirklich beabsichtigt, sich \underline{nicht} hinzusetzen, so dass man sagen müsste, dass auf seiner TDL steht, dass er sich nicht hinsetzen will oder dass man ihm unterstellen müsste, dass er ein Verhalten zeigt, das dem (späteren) Hinsetzen entgegengesetzt ist. Der Adressat ist vielmehr unsicher, ob er sich hinsetzen kann oder nicht oder ob er aufgefordert werden muss. In diesem Sinne kann plausibel im Raum stehen, ob er sich hinsetzt oder nicht, ohne ihm eine Absicht zuzuschreiben. Für meine Begriffe kann eine Äußerung wie \textit{Setz dich \textbf{doch}!} gleichermaßen geäußert werden, wenn der Angesprochene beabsichtigt oder nicht beabsichtigt, sich hinzusetzen.

Ähnlich verhält es sich in Interviews, in denen der Interviewer den Interviewten auffordert, zu reden (\textit{Erzählen Sie \textbf{doch} mal...}, \textit{Sagen Sie \textbf{doch} mal...}) (vgl. (\ref{1081})).

\begin{exe}
	\ex\label{1081} 
	\scriptsize
    \begin{tabular}[t]{lll}
	0001 & S1 &	\textbf{Herr Sch., erzählen Sie \underline{doch} ? mal etwas über Ihren .. Lebenslauf.}\\
	0002 & S2 & Ich wurde im Jahre neunzehnhundertneununddreißig in dem damaligen Sudetenland \\
	& & geboren... Im Jahre neunzehnhundertfünfundvierzig, bei Kriegsende, mußten \\
	& & wir die Heimat verlassen.
	\end{tabular}
	\hfill\hbox{(PF--\_E\_00249\_SE\_01\_T\_01)} 					       
\end{exe}
Hier sollte dem Interviewten klar sein, dass er zu bestimmten Dingen etwas sagen soll. Qua Konvention fängt man aber nicht einfach an zu reden. Der Interviewte wird quasi \glq angeschoben\grq {}, auch, weil unklar ist, ob er von allein reden soll oder nicht. Hier lässt sich ebenfalls nicht sagen, dass der Adressat beabsichtigt, nichts zu sagen, oder dass er aktuell das entgegengesetzte Verhalten zeigt, nicht zu reden, und dies so bliebe, wenn man ihn nicht ermunterte. Es ist aber sicherlich fraglich, ob er von allein erzählen würde. In der durch (\ref{1080}) und (\ref{1081}) illustrierten Situation liegt eine andere Lage vor, als wenn jemand freiwillig wirklich nichts sagte, oder er demjenigen, der die Aufforderung ausspricht, zu wenig sagt (vgl. (\ref{1082}) und (\ref{1083})).

\begin{exe}
	\ex\label{1082} 
	\scriptsize
    \glq Äh, ach so, ja, ja, ja äh wir haben Karl, weißt du, der hat sein, sein großes Messer abgebrochen.\grq {} \glq Messer abgebrochen, wieso, wo, 			wobei?\grq {} \glq Ja, er, er, haa/ , hat da ein Pferd mit tot gestochen, ja, das hat er, ja. Und da ist das dann, ist ihm das dann dabei abgebrochen.		\grq {} \glq Pferd, was für ein Pferd tot stechen, was für ein Pferd denn?\grq {} \glq Na, unseren schwarzen Heng/ Hengst, den hat er tot gestochen.		\grq {} \glq Jan, Mensch, bist du des Teufels, unseren schwarzen Hengst, warum das denn?\grq {} \glq Der hatte sich da ein Bein gebrochen.\grq {} \glq 		Der hat sich ein Bein gebrochen, wo das denn bei?\grq {} \glq Beim Wasser fahren.\grq {} \glq Was habt ihr dann für Wasser zu fahren?\grq {} \glq Na 		nach dem Feuer, wir mußten doch löschen.\grq {} \glq Feuer, was für ein Feuer! Menschenkind, Jan, \textbf{nun ? erzähl \underline{doch} mal}, nun sei 		doch nicht so maulfaul!\grq {} \glq Ja nun äh, äh äh die Scheune ist doch abgebrannt!\grq {} \glq Die Scheune ist abgebrannt, was für eine Scheu/ , 		unsere Scheune?\grq {} 		  
	\hfill\hbox{(ZW--\_E\_05156\_SE\_01\_T\_01)} 					     
\end{exe}
					     
\begin{exe}
	\ex\label{1083} 
	\scriptsize
    Mutters Klavier (Loriot)\\
    Zwei Möbelträger stellen ein Klavier vor einer Wohnungstür ab und klingeln.\\
	VATI (die Tür öffnend) Aha!\\
	TRÄGER (Lieferschein ablesend) Ist das hier richtig bei...Panislowski?\\
	VATI (in die Wohnung zurückrufend) Thomas! Also, Netzschalter auf \glqq On\grqq{}, gleichzeitig die Tasten \glqq Start\grqq{} und \glqq Aufnahme\grqq{} 	drücken, aber die Kamera erst auslösen, wenn die beiden Herren mit dem Klavier in der Wohnzimmertür erscheinen...\\
	TRÄGER Sind wir hier richtig bei...Panislowski?\\
	VATI Thomas!\\
	THOMAS (von innen) Ja...\\
	VATI Hast du verstanden?\\
	THOMAS Ja...\\
	VATI Na, \textbf{dann sag \underline{doch} was!}\\
	TRÄGER Wir kommen von der Firma... 
    \hfill\hbox{(http://privat.flachpass.net/html/mutters\_klavier.html)}		
    \newline
    \hbox{}\hfill\hbox{(eingesehen am 07.12.2015)}			     
\end{exe}
In den Beispielen, die ich bisher angeführt habe, konnte man (abgesehen von den konventionalisierten Varianten von Fall 2) dem Adressaten gut das Vorhaben um das Gegenteil des vom Sprecher gewünschten Zustands zuschreiben und die Offenheit der Frage um die Rea\-lisierung von p $\vee$ $\neg$p daraus ableiten. Generell argumentiere ich aber gerade dagegen, dass \textit{doch} stets i.e.S. auf einen Widerspruch verweist (vgl. Kapitel~\ref{chapter:jud}, Abschnitt~\ref{sec:doch1}). Vor allem auf die diskursinitialen Fälle und \textit{Wo}-VL-/V1-Sätze scheint dies nicht zuzutreffen. Die letzte Gebrauchsweise von \textit{doch}-Direktiven, die ich habe ausfindig machen können, ist diejenige, die mir diese Annahme auch bei den Direktiven zu bestätigen scheint. Es liegen Kontexte vor, in denen die Realisierung von  p $\vee$ $\neg$p zwar fraglich ist und es deshalb nötig ist, den Adressaten in die gewollte Richtung zu lenken, man dem Adressaten aber nicht das gegenteilige Verhalten oder entgegengesetzte Absichten zuschreiben kann. Ich halte es für eine zu starke Annahme, in diesen Fällen davon auszugehen, dass der Adressat das Gegenteil in Planung hat. Natürlich vollzieht er die Handlung, zu der er aufgefordert wird, aktuell noch nicht. Dies ist eine Bedingung, die an jeden Direktiv gestellt wird. In diesen Beispielen hat man aber keine Anzeichen dafür, dass die gegenteilige Handlung beabsichtigt ist. Es besteht eher eine gewisse Ratlosigkeit. Illokutionär handelt es sich hierbei um Anregungen, Nahelegungen, Vorschläge, Ratschläge, Ermunterungen oder Einladungen. Beispiele, die ich hier zuordne, findet man z.B. in Ratgeberzeitungen oder Tippsektionen in Zeitungen. Und sie können auch Teil von Horoskopen sein (vgl. z.B. (\ref{1084}) bis (\ref{1086})).

\begin{exe}
	\ex\label{1084} 
	\scriptsize
	\textbf{Nehmen Sie diesbezüglich Ihren Arbeitsalltag \underline{doch} einmal genau unter die Lupe}, um unnötige Belastungen zu vermeiden.	
	\hfill\hbox{(DECOW 2014)}  
	\newline
	\hbox{}\hfill\hbox{(http://www.lambertus-apotheke.de/html/verschleiss.html)} 				     
\end{exe}

\begin{exe}
	\ex\label{1085} 
	\scriptsize
	Nun ist Zeit zum Entspannen und Faulenzen. Vielleicht auch zum Träumen? \textbf{Schmieden Sie mit dem Partner \underline{doch} ein paar Reise- und/oder 	Zukunftspläne.} 	
	\newline
	\hbox{}\hfill\hbox{(RHZ09/JAN.06468 Rhein-Zeitung, 10.01.2009; Zwillinge 21.5.)} 				     
\end{exe}
				        
\begin{exe}
	\ex\label{1086} 
	\scriptsize
	Sie sehen den Wald vor lauter Bäumen nicht. Sie haben Mühe damit, im Beruf alles in den Griff zu bekommen. \textbf{Reduzieren Sie \underline{doch} Ihr 		Pensum.} Auch im Privatleben ist Ärger möglich. Seien Sie toleranter.	
	\hfill\hbox{(NON09/APR.15465 Niederösterreichische Nachrichten, 27.04.2009)} 				     
\end{exe}
Der Adressat hat ein Interesse, ein bestimmtes Ziel zu erreichen, z.B. hier Ausgeglichenheit, Entspannung, Wohlbefinden. Er hat in dieser Hinsicht wahrscheinlich Defizite, d.h. diese Aspekte liegen nicht vollends vor. Anderenfalls würde er nicht solche Texte lesen, in denen er hofft, eine Lösung für sein Problem zu finden, bzw. möchte er sich auf die angesprochenen Ziele (z.B. in den Horoskopen) einlassen. Da der Adressat als Hilfesuchender auftritt, stellen sich vor diesem Hintergrund im Kontext dann Fragen wie: Wird der Adressat den Alltag auf Belastungen untersuchen (um ausgeglichener zu sein)? Tritt die Situation ein, dass er sein Pensum reduziert (damit er beruflich weniger eingebunden ist)? Damit es dem Adressaten besser geht, müssen diese Punkte in die Richtung des Direktivs aufgelöst werden. Ich gehe davon aus, dass im cg die Relation p $>$ q ent\-halten ist. Da B nach einem Weg sucht, den Zustand q herzustellen, obwohl er weiß, dass das Vorliegen von p diesen normalerweise mit sich bringen würde, lässt sich ableiten, dass verstärkt unklar ist, ob er von alleine p realisieren würde. Man kann annehmen, dass in diesen Situationen fraglich ist, ob diese Dinge für den Adressaten gelten (weil er anderenfalls keinen Beratungsbedarf hätte). Man kann aber z.B. nicht sagen, dass er explizit beabsichtigt, nicht das Pensum zu reduzieren (zumindest nicht mehr als dies sowieso nicht bereits gegeben sein kann, damit man den Direktiv äußern kann). Die Offenheit von p kann auch in diesen Fällen als Kontextanforderung angesetzt werden. Denkbar ist auch, dass der Hörer um den Zusammenhang noch gar nicht weiß und sich die Fraglichkeit der Realisierung deshalb nur in den Augen des Sprechers ergibt. Diese Konstellation habe ich auch bei \textit{doch} in Assertionen bereits für möglich gehalten. Mit der Auslegung der Beispiele aus (\ref{1084}) bis (\ref{1086}) in (\ref{1087}) geht einher, dass !p dem Adressaten bekannt sein sollte. Die Frage ist hier, ob man dies für alle diese Typen von \textit{doch}-Direktiven vertreten möchte. \textit{Doch} \underline{verweist} hier aber m.E. nicht auf die Bekanntheit des Rates, sondern auf die Fraglichkeit der Realisierung. Hier wird deutlich, dass verschiedene Partikeln in gleichen Kontextsituationen unterschiedliche Aspekte thematisieren. Als evident und bekannt zeichnet \textit{eben} eine Aufforderung aus. Es wäre aber äußerst unangemessen, in den obigen Beispielen eben zu verwenden und dem Hilfesuchenden somit anzuzeigen, dass er die Antwort selber weiß (auch wenn es de facto so ist!).
	
\begin{exe}
\ex\label{1087} Kontextzustand vor einem \textit{doch}-Direktiv\\[-0.6em]
\begin{tabular}[t]{|C{6em}|C{12em}|C{6em}|}
\hline
$\textrm{DC}_{\textrm{A}}$ & Tisch &  $\textrm{DC}_{\textrm{B}}$ \tabularnewline
\hline
{} & p $\vee$ $\neg$p & {}  \tabularnewline
\cline{1-1}\cline{3-3}
$\textrm{TDL}_{\textrm{A}}$ & {} & $\textrm{TDL}_{\textrm{B}}$  \tabularnewline
\cline{1-1}\cline{3-3}
{} & {} & {}  \tabularnewline
\hline
\multicolumn{3}{|l|}{cg s$_{1}$ = $\lbrace\neg$q, p $>$ q$\rbrace$} \tabularnewline
\hline
\end{tabular}
\end{exe}		 
Ähnliches wie für derartige Ratschläge gilt auch für Einladungen der Art in (\ref{1088}) und (\ref{10888}).

\begin{exe}
	\ex\label{1088} 
	\scriptsize
	Diskutiert wird mit den Gästen im Studio natürlich auch über aktuelle niedersächsische Themen. Ein Beleg dafür, dass Talkshows im Radio weitaus 			spannender als die im Fernsehen sind. \textbf{Hören Sie \underline{doch} mal rein.} Wir treffen uns hier wieder, an dieser Stelle, heute in einer 			Woche. Bis dahin viel Spaß mit Ihrem Radio! 		
	\hbox{}\hfill\hbox{(HAZ09/MAI.02715 Hannoversche Allgemeine, 16.05.2009)} 				     
\end{exe}

\begin{exe}
	\ex\label{10888} 
	\scriptsize
	Spielst du seit zwei Jahren ein Blasinstrument oder Schlagzeug, dann bist du bei uns genau richtig. Nebst dem Musizieren legen wir auch Wert auf das 		gemütliche Zusammensein (Sommer- oder Winterplausch usw.). Der Dirigent heisst Christoph Diem und wohnt in Flawil. $[$...$]$ Die offenen Proben sind am 	17. März und 21. April. \textbf{Komm \underline{doch} einfach mal vorbei} und schau, ob es dir gefallen würde!
	\newline
	\hbox{}\hfill\hbox{(A10/MAR.05294 St. Galler Tagblatt, 16.03.2010)} 				     
\end{exe}	               			           				  
Man kann hier nicht davon ausgehen, dass der Adressat das Gegenteil beabsichtigt. Der Sprecher nimmt in (\ref{1088}) natürlich nicht an, dass der Adressat sowieso vorhat, zu kommen oder sowieso schon Mitglied des Orchesters ist. Er hat aber auch nicht vertreten, dass er plant, nicht zu kommen (oder nicht reinhören zu wollen). Wenn er dies vorhätte, würde er die Anzeige wahrscheinlich auch nicht lesen.

Auch bei Vorschlägen wie in (\ref{1089}) kann man nicht davon ausgehen, dass der Adressat Gegenteiliges zu tun plant. In einem Gespräch über Terminvereinbarungen, in dem der Tag dazu schon feststeht, lässt sich aber ableiten, dass das Thema zur Diskussion steht, ob der Adressat früh morgens kommen wird. 

\begin{exe}
	\ex\label{1089} 
	\scriptsize
	Prima und dann noch vielleicht übermorgen ja da bleibt uns ja nur der Nachmittag.\\
	
	Ja guten Tag Frau Müller. Samstag der fünfzehnte passt mir sehr gut ja.\\
	
	Ja okay \textbf{dann kommen Sie \underline{doch} gleich um neun Uhr zu mir.}
	\newline
	\hbox{}\hfill\hbox{(Tübinger Baumbank des Deutschen/Spontansprache)} 				     
\end{exe}
Ob sich die Gesprächsteilnehmer um neun Uhr treffen, ist einfach salient, da sie sich zu irgendeiner Zeit treffen müssen. 

Ich gehe folglich davon aus, dass die kontextuelle Voraussetzung für einen \textit{doch}-Direktiv ist, dass die Frage, \textit{ob p} (wobei es sich bei p um die realisierte Proposition handelt, zu deren Erfüllung der im Anschluss geäußerte Direktiv auffordert), auf dem Tisch liegt, d.h. ein offenes, salientes Thema ist. Wie oben ausgeführt, kann es verschiedene Szenarien geben, wie sich dieser Kontextzu\-stand ergibt.

Ich halte die Annahme, die davon ausgeht, dass \textit{doch} immer auf einen Gegensatz oder Widerspruch verweist, was bei Direktiven dann bedeuten würde, dass der Adressat Gegenteiliges vorhat oder bereits ausführt, für zu stark. Als schwä\-chere Variante vertrete ich deshalb, dass die Realisierungsfrage um p offen ist. Die Erfüllung von p ist somit fraglich. Es ist aber dazu zu sagen, dass den beiden Alternativen nicht die gleiche Wahrscheinlichkeit ihrer Realisierung zukommt. Für \textit{doch} in Assertionen gilt ebenfalls, dass es nicht rein bestätigend verwendet werden kann (vgl. (\ref{1090})).

\begin{exe}
	\ex\label{1090} 
	B: Peter hat Maria mit einem Mann in der Stadt gesehen.\\
	A: \#Es war \textbf{doch} nicht ihr Ehemann.				     
\end{exe}
Für die Direktive bedeutet dies, dass \textit{doch} nicht vorkommen kann, wenn die Realisierung von p in keiner Hinsicht fraglich ist, und zwar über das Maß hinaus, mit dem die tatsächliche Realisierung der Handlung sowieso noch aussteht (auch nach Zustimmung der Realisierung). Nach meiner Modellierung steht nach der Äußerung eines jeden Direktivs bis zur tatsächlichen Ausführung der beteiligten (Nicht-)Handlung schließlich die Frage im Raum, ob p oder $\neg$p realisiert wird. Dieser Aspekt wird eine Rolle spielen in Abschnitt~\ref{sec:kombida}, in dem ich die Kombinationen aus \textit{doch} und \textit{auch} betrachte.

\subsubsection{Deskriptive Eindrücke aus der Literatur}
Meine Modellierung fängt auch die deskriptiven Eindrücke aus der Literatur zu \textit{doch}-Direktiven auf. 

Ein Aspekt, der in verschiedenen Arbeiten angeführt wird, ist, dass ein Widerspruch zwischen der geforderten Handlung und der aktuellen Situation bestehe. Natürlich liegt dieser zu einem gewissen Grad in jedem Kontext, in dem ein Direktiv geäußert wird, vor, er werde durch die Partikel aber zusätzlich thematisiert (\citealt[139]{Hentschel1986}; auch \citealt[118]{Thurmair1989}, \citealt[214]{Kwon2005}). Dieser Aspekt findet sich in meiner Modellierung auf die Art wieder, dass die Realisierung von p (vs. $\neg$p) je nach Verwendung vor der Äußerung aus verschiedenen Gründen bereits salient ist. D.h. anstatt von einem Widerspruch auszugehen, vertrete ich die \underline{Fraglichkeit} der Realisierung von p.

Weiter hat \citet[139]{Hentschel1986} den Eindruck formuliert, \textit{doch}-Direktive würden eine enge Anbindung an den Kontext leisten. Da die Offenheit von p aus dem aktuellen Handeln/Verhalten des Adressaten oder der Kontextsituation folgt, spie\-gelt meine Analyse auch diesen Aspekt.

Einig sind sich existierende Arbeiten auch dahingehend, dass zwei Arten von \textit{doch}-Direktiven zu unterscheiden sind: a) ein verstärkender Typ, dem Eigenschaften wie \glq ungeduldig\grq {}, \glq dringend\grq {}, \glq drängend\grq {}, \glq ärgerlich\grq {}, \glq vorwurfsvoll\grq {}, \glq unwirsch\grq {} zugeordnet werden, und b) ein abschwächender Typ, belegt mit Charakteristika wie \glq höflich‘, \glq freundlich‘, \glq beiläufig‘, \glq beruhigend‘. \citet[191]{Rinas2006} paraphrasiert die erste Verwendungsweise durch \glq auch wenn DU/Sie dich/sich offenbar nicht traust/trauen\grq {} (vgl. auch \citealt[188]{Franck1980}, \citealt[139]{Hentschel1986}, \citealt[113]{Helbig1990}, \citealt[91-92, 214]{Kwon2005}) und die zweite durch \glq auch wenn Du/Sie das nicht willst/wollen\grq {}. Mit dieser Zweiteilung gehen auch die verschiedenen denkbaren Illokutionstypen einher: Aufforderung, Befehl, Bitte, Empfehlung, Rat (vgl. z.B. \citealt[27]{Volmert1991}, \citealt[113]{Helbig1990}, \citealt[91-92, 214]{Kwon2005}). Welcher Illokutionstyp realisiert wird, ist abhängig von Kontext, Inhalt und dem Auftreten weiteren sprachlichen Materials (z.B. \textit{endlich} vs. \textit{bitte}, \textit{mal}). Es gibt Autoren, die vertreten, dass \textit{doch}-Direktive i.d.R. Ratschläge \is{Ratschlag} sind (\citealt[402]{Ickler1994}) und Befehle \is{Befehl} zu Bitten \is{Bitte} und Ratschlägen abschwächen (\citealt[111]{Bublitz1978}). Der Nachweis dieser Annahme, dass innerhalb der \textit{doch}-Direktive bestimmte Illokutionen bzw. einer der zwei angeführten Typen überwiegen, steht aber noch aus.

Für die \textit{doch}-Modellierung bedeuten derartige illokutive Unterscheidungen, dass sie für die beiden in ihrem Ton entgegengesetzt wirkenden Direktivtypen aufkommen können sollte. Ich denke, dass dies möglich ist: Der Verstärkungsfall (mit den einhergehenden Eigenschaften) liegt sinnvollerweise vor, wenn ein Direktiv wiederholt wird, weil ihm noch nicht nachgekommen worden ist/keine Bereitschaft zum Nachkommen angezeigt wird (Fall 1) oder der Adressat aktuell das Gegenteil ausführt und man ihn dazu bewegen möchte, im zukünftigen Zustand ein anderes Verhalten zu realisieren (Fall 2). Dass die Realisierungsfrage schon im Raum steht, kann Aspekte wie Dringlichkeit, Ungeduld und Vorwurf zur Folge haben. Auch der Effekt der Abschwächung und damit einhergehender Höflichkeit lässt sich im Rahmen meiner Modellierung erfassen: Prinzipiell sind Direktive immer gesichtsbedrohend für den Adressaten, da in sein Gebiet eingegriffen wird (negative \is{Gesichtsbedrohung} Gesichtsbedrohung) (\citealt[127]{Held1995}). Im Grunde besteht auch eine Bedrohung des positiven Gesichts des Sprechers, weil er es ist, der den Adressaten in seiner Handlungsfreiheit einschränkt. Diese Gesichtsbedrohungen werden durch das Auftreten von \textit{doch} etwas genommen: Die Direktive sind weniger aufdringlich (und damit auch beruhigend, freundlich), wenn angezeigt wird, dass die Realisierungsfrage bereits im Raum steht. Dadurch, dass sie Thema ist, weisen die Diskurspartner schon diese Gemeinsamkeit auf, was die Gesichtsbedrohung auf beiden Seiten entlastet. Vor diesem Hintergrund wird der Adressat in eine Richtung beeinflusst, es ist aber nicht der Direktiv, der die Realisierungsfrage mitbringt. Dass der Sachverhalt, der ihm als Sprecherwunsch mitgeteilt wird, eine Möglichkeit ist, ist ebenfalls schon gegeben. Der Inhalt des Auftrags kann somit nicht völlig überraschend sein. Auch über die diskursinitialen \textit{doch}-Assertionen habe ich in Abschnitt~\ref{sec:doch1} in Kapitel~\ref{chapter:jud} gesagt, dass sie höflicher wirken, weil sie der Assertion die \glq Aufgabe\grq {} abnehmen, das Thema zu eröffnen. Hier hätte man es unter diesen Betrachtungen folglich mit parallelen Effekten zu tun, die gerade auf die grundsätzlich parallele Modellierung zurückgehen.

Im Falle einer Einladung \is{Einladung} kann eine Gesichtsverletzung dadurch zustandekommen, dass der Adressat gar nicht an der Aktivität teilnehmen möchte, oder auch dadurch, dass er denkt, dass er sich aufdrängt, wenn er dazugebeten wird. Auch diese beiden Gründe für die Gesichtsbedrohung können überdeckt werden von der schon zur Debatte stehenden Realisierung. Beide Alternativen sind dann schon verfügbar (unabhängig davon, was der Sprecher möchte) und der Adressat kann sich auf eine Art einfacher für die Ablehnung und Realisierung von $\neg$p entscheiden, weil diese Möglichkeit bereits gleichermaßen im Raum steht wie p. Ablehnen kann er natürlich auch einen MP-losen Direktiv, aber unter diesen Umständen eröffnen sich die Alternativen hinsichtlich der weiteren Entwicklung erst im Zuge der Äußerung des Direktivs und im Zuge dessen, dass der Sprecherwille deutlich wird. Stehen die beiden Möglichkeiten schon im Vorfeld zur Verfügung, scheint mir dies dem Adressaten mehr Raum zu geben. Die Gefahr des Aufdrängens wird ebenfalls abgeschwächt, weil das potenzielle Mitkommen auch schon vor der Äußerung des Direktivs eine verfügbare Alternative ist.

In Ratschlägen \is{Ratschlag} wird gesichtsbedrohend \is{Gesichtsbedrohung} eingeschätzt, dass suggeriert wird, dass der Adressat ohne das Eingreifen des Sprechers nicht zurecht kommt und der Sprecherbeitrag somit als unerwünschtes Einmischen gedeutet wird (vgl. \citealt[114]{Frank2011} und die dort zitierte Literatur). Auch diese Bedrohung wird mit einem \textit{doch}-Direktiv genommen, da sowieso bereits fraglich ist, was der Adressat tun wird.

Ich denke, die Betrachtung der \textit{doch}-Direktive zeigt auch, dass die Bedeutungs\-komponente von Bekanntheit bei dieser MP nicht zutreffend ist bzw. nicht als kategorisch beteiligt angenommen werden kann. \citet[118-119]{Thurmair1989} hält sie zwar auch in diesem Äußerungstyp hoch (dem Adressaten sei der Sprecherwille klar) (vgl. ähnlich auch \citealt[111]{Bublitz1978}, \citealt[168]{Karagjosova2004}), aber man kann in meinen Augen nicht für alle der obigen Beispiele von Bekanntheit, Ableitbarkeit, Offensichtlichkeit ausgehen (vgl. auch \citealt[401]{Ickler1994}, der darauf hinweist, dass neue Vorschläge und Aufforderungen möglich sind). Bekanntheit lässt sich generell vertreten für Fall 1 (weil die Aufforderung zum wiederholten Male getätigt wird) sowie die konventionalisierten Varianten von Fall 2. Genauso ist der Vorschlag in (\ref{1089}) ableitbar und offensichtlich. Nicht zutreffend scheint mir eine solche Einschätzung allerdings für die Ratschläge. Ausgenommen, man sagt, dass der Inhalt von Ratschlägen (wie in Ratgebersektionen) sowieso bekannt ist, kann man für derartige Vorschläge generell nicht annehmen, dass sie offensichtlich sind. Wäre dies der Fall, würde der Adressat sie nicht lesen, sondern direkt sein Leben entsprechend ändern. Wenn dem Hörer hier bekannt sein sollte, was zu tun ist, verweist \textit{doch} für meine Begriffe nicht auf diese Einschätzung. Wenn Ratschläge dem Hörer stets als bekannt ausgegeben würden, hätte man es dazu mit sehr unhöflichen Kontexten zu tun. Wie oben schon angeführt, muss man unterscheiden zwischen der Kontextsituation und den Aspekten, auf die die Partikeln verweisen. Im gleichen Kontext können natürlich oftmals verschiedene Partikeln verwendet werden, die eben auf verschiedene Verhältnisse Bezug nehmen.

Auch Einladungen kann man nicht grundsätzlich als bekannt einstufen. In manchen Kontexten bietet sich eine Interpretation als Stereotyp an, etwa, dass der Angesprochene in Ankündigungen wie in (\ref{1088}) zu den beteiligten Aktivitäten (testweise) dazugebeten wird. 
 
Will man die Bekanntheitshypothese aufrecht erhalten, gibt es natürlich immer die Möglichkeit, dieses Bedeutungsmoment als vorausgesetzt anzusehen. Für die Verwendung als Vorschlag, Rat, Empfehlung oder Einladung halte ich eine solche Bewertung des Direktivs nicht für plausibel. Dieser Eindruck bestätigt sich auch, wenn man genauer in die (Bedingungen der) Illokutionstypen schaut. Für Vorschläge \is{Vorschlag} heißt es z.B., der Hörer habe dem Sprecher ein \glqq vages Ziel\grqq{} angegeben (\citealt[319]{Rehbein1977}). Es handle sich dabei nur um \glqq ein Grobziel\grqq{}, \glqq das so vage ist, daß er nicht weiß, wie er es erreichen soll\grqq{} (S. 317) (vgl. \citealt[186]{Rolf1997}). Für Ratschläge \is{Ratschlag} gilt nach \citet[186-187]{Rolf1997}, dass der Hörer ein (technisch- oder moralisch-praktisches) Problem hat, das er auch explizit an den späteren Sprecher adressiert haben kann. Das Vorliegen eines Problems gilt auch für Empfehlungen (vgl. ebd.). Vor dem Hintergrund dieser Charakterisierungen wäre es meiner Meinung nach merkwürdig, wenn ein Vorschlag, Rat, eine Empfehlung stets bekannt wäre. Unter diesen Umständen müsste sich der Adressat auch gar nicht in den Kontext begeben, dass er den/die jeweilige(n) Vorschlag, Rat oder Empfehlung erhält. Diese Illokutions\-typen hätten dann immer die Lesart, dass der Adressat die Lösung sowieso weiß. Diese Verhältnisse entsprechen nicht meiner Intuition hinsichtlich der Gründe, aus denen derartige Äußerungen gemacht werden.  

An den Ausführungen zu \textit{doch} in Direktiven sieht man, dass diese Partikel mit verschiedenen direktiven Illokutionstypen \is{Illokutionstyp} kompatibel ist und in diesem Sinne recht unspezifisch ist (vgl. \citealt[119]{Thurmair1989}). Für \textit{auch}, dessen Vorkommen in direktiven Äußerungen Gegenstand des nächsten Abschnitts ist, liegen spezifi\-schere Auftretensweisen vor.

\subsection{Das Einzelauftreten von \textit{auch}} 
\label{sec:dadir}
\subsubsection{Der Diskursbeitrag von \textit{auch}}
Da ich grundsätzlich von der Möglichkeit einer minimalistischen Bedeutungszu\-schreibung \is{Bedeutungsminimalismus/-maximalismus} ausgehe, verfolge ich auch hier das Ziel, \textit{auch} in Direktiven parallel zu seinem Vorkommen in Assertionen zu behandeln. Für assertive Verwendungen (vgl. z.B. (\ref{1091})) habe ich in Abschnitt~\ref{sec:auch} angenommen, dass die Relation p $>$ q, die auf einer Norm, Erwartung oder Erfahrung basiert, im cg enthalten ist, und sich q mindestens unter den Diskursbekenntnissen von A oder B befindet. Durch das Äußern der \textit{auch}-Assertion wird p eingeführt und fungiert als Erklärung für q, das als für den Sprecher ableitbar ausgegeben wird, weil es aus der cg-Relation folgt.
	
\begin{exe}
	\ex\label{1091} 
	A: Es ist so kalt.\\
	B: Wir haben \textbf{auch} November.				     
\end{exe}	
Generell sind \textit{auch}-Direktive in der Literatur bisher sehr wenig berücksichtigt worden. Wenn überhaupt, werden sie bei der Diskussion von \textit{auch}-Assertionen am Rande erwähnt. (\ref{1092}) zeigt einige Beispiele.

\begin{exe}
	\ex\label{1092} 
		\begin{xlist}	
			\ex\label{1092a} Sei \textbf{auch} brav!	
			\hfill\hbox {\citet[97]{Diewald1998}}
			\ex\label{1092b} Nun iß \textbf{auch} schön!
			\hfill\hbox {\citet[158]{Thurmair1989}}
			\ex\label{1092c} Jetzt geh \textbf{auch} (schön)!	
			\hfill\hbox {\citet[60]{Dittmann1980}}
			\ex\label{1092d} Schreibe \textbf{auch} ordentlich!		
			\hfill\hbox {\citet[90]{Helbig1990}}
			\ex\label{1092e} Kommt \textbf{auch} nicht zu spät nach Haus!		
			\hfill\hbox {\citet[58]{Dahl1988}}
			\ex\label{1092f} (A sieht, wie B den Autoschlüssel nimmt, er sagt:)\\
			A: Vergiß \textbf{auch} das Tanken nicht!		
			\hfill\hbox {\citet[50]{Dahl1988}}
		\end{xlist}
\end{exe}
In deskriptiven Erfassungen des \textit{auch}-Beitrags ist aus verschiedenen Ausführungen abzulesen, dass die Handlung ableitbar bzw. bekannt ist (vgl. z.B. \citealt[109]{Burkhardt1982}). \citet[158]{Thurmair1989} schreibt, die Handlung, zu der aufgefordert wird, ist eine \glqq allgemein gültige Norm\grqq{}. \citet[50]{Dahl1988} und \citet[78]{Kwon2005} vertreten, die Handlungsausführung sei bei B bekannt (weil beispielsweise vorher darüber gesprochen wurde). Angenommen wurde darüber hinaus, ein \textit{auch}-Befehl könne nicht der erste Befehl sein (\citealt[50]{Dahl1988}, \citealt[78]{Kwon2005}) und der Hörer sollte den Sachverhalt schon ausgeführt haben (\citealt[90]{Helbig1990}, \citealt[215]{Kwon2005}). Diese Eindrücke finden sich auch in der Paraphrase in (\ref{1093}).

\begin{exe}
	\ex\label{1093} 
	\glq Ich, der Sprecher, habe erwartet, daß du p tust/getan hast; du tust p nicht/\\hast p nicht getan. Tue jetzt p.\grq {} 	
	\hfill\hbox {\citet[60]{Dittmann1980}}
\end{exe}
Ferner wird vertreten, bei \textit{auch}-Direktiven sei eine asymmetrische Beziehung beteiligt, die sich darin äußert, dass sie typischerweise von Erwachsenen an Kinder gerichtet würden (\citealt[158]{Thurmair1989}, \citealt[78]{Kwon2005}). 

Illokutionär sind \textit{auch}-Direktive Vorwürfe \is{Vorwurf} oder \is{Ermahnung} Ermahnungen (\citealt[58]{Dahl1988}, \citealt[78]{Kwon2005}). Mit diesem Punkt hängt auch zusammen, in welchen Formtypen die \textit{auch}-Direktive möglich sind. Neben V1-Imperativsätzen, die ich hier aus\-schließlich betrachte, tritt diese Partikel auch in \textit{dass}-VL-Sätzen auf (vgl. (\ref{1094})).

\begin{exe}
	\ex\label{1094} 
    \begin{tabular}[t]{ll}
	Enkeltochter: & Oma, hast du ma was Geld? Meine Strümpfe sind kaputt, \\
	& ich muß mich doch vorstelln gehn.\\
	Großmutter: & (Seufzend kramt sie in ihrer Schürzentasche) \\
	& Daß du \textbf{auch} Strümpfe kaufst! Nicht wieder bloß Süßkram!
	\end{tabular}
	\newline
	\hbox{}\hfill\hbox{(TAZ, 24.09.1994, 40)}
	\newline
	\hbox{}\hfill\hbox{\citet[78]{Kwon2005}}				       
\end{exe}
\textit{Das}s-VL-Aufforderungen gelten als starke Aufforderungen (vgl. \citealt[54]{Thurmair1989}). 

Ich meine, dass man bei den \textit{auch}-Direktiven zwei grundsätzliche Fälle trennen muss: In der einen Verwendung hat der Adressat sich schon dazu bekannt, dass er p tun wird. Dieser Gebrauch fängt mit auf, dass es sich nicht um den ersten Befehl handelt, wobei m.E. hinzu kommt, dass die Bereitschaft zur Übernahme bereits angezeigt wurde. Nach Ablehnung kann ein \textit{auch}-Direktiv nicht die nächste Aufforderung sein (vgl. (\ref{1095}) und die Belege weiter unten).

\begin{exe}
	\ex\label{1095} 
 	A: Trink deinen Kaffee!\\
	B: Ne, ich möchte nicht.\\
	A: \#Trink ihn \textbf{auch}! Der ist wirklich gut.\\
	vs.\\
	A : Trink ihn \textbf{doch}! Der ist wirklich gut.
\end{exe}
Die zweite Verwendungsweise umfasst Belege, die im Kontext zum ersten Mal geäußert werden und für die gilt, dass die Handlungsaufforderung aus dem Kontext, Weltwissen oder einer erfüllten salienten Bedingung (s.u.) ableitbar ist.

Die beiden Verwendungsweisen können sich überlappen, dies muss aber nicht der Fall sein.

Im Folgenden wird zunächst Gebrauch 2 genauer untersucht, an dem ich meine Modellierung einführe. Es handelt sich hierbei um Kontexte, die Beispielen der Art \textit{Sei \textbf{auch} artig/brav!}, \textit{Benimm dich \textbf{auch}!} entsprechen (vgl. (\ref{1096}) bis (\ref{1100})).

\begin{exe}
	\ex\label{1096} 
	\scriptsize
 	Trixis Vater freute sich und wünschte ihr eine gute Reise. \glqq Grüß' mir Tante Ottilie \textbf{und sei \underline{auch} artig!} Ich rufe heute Abend 		an.\grqq{}\\	
	\glqq Mach  ich, Papa. Aber jetzt müssen wir zum Bahnhof fahren, sonst verpasse ich wirklich noch meinen Zug.\grqq{}, verabschiedete sich Trixi von 		ihrem Vater.
	\hfill\hbox{(Liebrecht 2002 Der Sonderbare Schaukelstuhl)}				 
  	\newline
	\hbox{}\hfill\hbox{(https://books.google.de/books?id=in9y1c-FPdgC\&pg=PA5\&lpg=PA}		
   	\newline
	\hbox{}\hfill\hbox{5\&dq=\%22sei+auch+artig\%22\&source=bl\&ots=3Fkfw7Of0F\&sig=k}
	\newline
	\hbox{}\hfill\hbox{WTagKqIg\_5j80JCD8WPnLv6FoI\&hl=de\&sa=X\&ved=0CCsQ6AE}
	\newline
	\hbox{}\hfill\hbox{wAmoVChMIvcbX7PzGyAIVyPJyCh2qoQHB\#v=onepage\&q=\%22s}
	\newline
	\hbox{}\hfill\hbox{ei\%20auch\%20artig\%22\&f=false)}
\end{exe}
	
\begin{exe}
	\ex\label{1097} 
	\scriptsize
 	\glqq Den Schlüssel hast du?\grqq{} fragte Papa.\\
	\glqq Ja, Papa.\grqq{}\\
	\glqq Bitte, schließ die Tür gut ab, wenn du die Wohnung verläßt. Und laß nachmittags, wenn Du allein zu Hause bist, ja keinen rein!\grqq{} ermahnte 		Papa sie.\\
	\glqq Ja, Papa.\grqq{}\\
	\glqq \textbf{Mach \underline{auch} deine Hausaufgaben!}\grqq{}\\
	\glqq Ich weiß schon selbst, daß ich meine Hausaufgaben machen muß!\grqq{} entgegnete Tina.
	\newline
	\hbox{}\hfill\hbox{(Hogger 2000, Tina und der Teddybär)}				 
  	\newline
	\hbox{}\hfill\hbox{(https://books.google.de/books?id=tyNR-EWT9qMC\&pg=PA109\&lp}		
   	\newline
	\hbox{}\hfill\hbox{g=PA109\&dq=\%22Mach+auch+deine+Hausaufgaben\%22\&source=b}
	\newline
	\hbox{}\hfill\hbox{l\&ots=Dcwv-lmPUy\&sig=FWyk77CetirGeSeOEIx4vME0UaI\&hl=de}
	\newline
	\hbox{}\hfill\hbox{\&sa=X\&ved=0CCUQ6AEwAWoVChMIxMiR7J7HyAIV5yVyCh27}
	\newline
	\hbox{}\hfill\hbox{7AKq\#v=onepage\&q=\%22Mach\%20auch\%20deine\%20Hausaufgabe}
	\newline
	\hbox{}\hfill\hbox{n\%22\&f=false)}
\end{exe}						                              
						
\begin{exe}
	\ex\label{1098} 
	\scriptsize
 	Anruf von Frau Mutter: \glqq \textbf{Trink \underline{auch} genug bei dem Wetter!}\grqq{}
	\newline
	\hbox{}\hfill\hbox{(https://twitter.com/herrhellmute/status/618429887519526913)}				 
  	\newline
	\hbox{}\hfill\hbox{(Google-Suche, eingesehen am 12.12.2015)}		
\end{exe}	
								                    
\begin{exe}
	\ex\label{1099} 
	\scriptsize
 	Mit Rauschebart schien der finstere Gesell' schon ziemlich alt, jedenfalls musste er sich hinsetzen und Marktmeisterin Monika Haudel die Geschenke aus 		dem Sack holen. Irgenwie schien er aber alle Kinder zu kennen und kannte sich auch sonst in der Stadt gut aus: \glqq Räum scheen dei Zimmer uff, und 		\textbf{putz dir \underline{auch} die Zähne}\grqq{} gab er den Kleinsten mit. 
	\newline
	\hbox{}\hfill\hbox{(http://www.heylive.de/index.php?id=96\&tx\_ttnews$[$pointer$]$}				 
  	\newline
	\hbox{}\hfill\hbox{=13\&tx\_ttnews$[$tt\_news$]$=104\&tx\_ttnews$[$backPid$]$=35\&cHa}		
	\newline
	\hbox{}\hfill\hbox{sh=f2ddcfb43b7771418de993f3915d72c5)}
	\newline
	\hbox{}\hfill\hbox{(Google-Suche, eingesehen am 12.12.2015)}
\end{exe}
									
\begin{exe}
	\ex\label{1100} 
	\scriptsize
 	\glqq \textbf{Sei \underline{auch} pünktlich!}\grqq{} ermahnt mich Kaspar jedesmal, wenn wir uns verabreden. Dabei müßte er längst wissen, daß ich 			geradezu superpünktlich bin.
	\newline
	\hbox{}\hfill\hbox{(http://www.abendblatt.de/archiv/1965/article200886539/Sin}				 
  	\newline
	\hbox{}\hfill\hbox{d-wir-wirklich-unpuenktlich.html)}		
\end{exe}
(\ref{1101}) zeigt meine Modellierung des Kontextzustandes, der besteht, bevor ein \textit{auch}-Direktiv geäußert wird.

\begin{exe}
\ex\label{1101} Kontextzustand vor einem \textit{auch}-Direktiv\\[-0.6em]
\begin{tabular}[t]{|C{6em}|C{12em}|C{6em}|}
\hline
$\textrm{DC}_{\textrm{A}}$ & Tisch &  $\textrm{DC}_{\textrm{B}}$ \tabularnewline
\hline
{} & {} & q  \tabularnewline
\cline{1-1}\cline{3-3}
$\textrm{TDL}_{\textrm{A}}$ & {} & $\textrm{TDL}_{\textrm{B}}$  \tabularnewline
\cline{1-1}\cline{3-3}
{} & {} & {}  \tabularnewline
\hline
\multicolumn{3}{|l|}{cg s$_{1}$ = $\lbrace$q $>$ !p$\rbrace$} \tabularnewline
\hline
\end{tabular}
\end{exe}									
Die Relation q $>$ !p ist im cg enthalten. Aus dem Kontext ist ersichtlich, dass mindestens B (der Adressat des späteren Direktivs), meistens A und B, von q ausgehen. Dass nur A von q ausgeht, scheidet meiner Meinung nach aus, da der Direktiv immer an B gerichtet wird. Wenn der Direktiv geäußert wird, teilt A B etwas mit, das für diesen nicht überraschend ist, weil er es aus der Relation im cg ableiten können sollte. Er wird folglich zur Realisierung von p angehalten und diese Anweisung ist erwartet, weil sie aus der aktuellen Situation (Eintritt/Vorliegen von q) folgt. B hat im Grunde keine Möglichkeit, !p nicht auf seine TDL aufzunehmen, da B q $>$ !p ebenfalls teilt. Die Möglichkeit der Ablehnung bei der Frage \textit{Wird B p auf seine TDL nehmen oder nicht?} wird somit übersprungen und es entsteht ein ähnlicher Effekt, wie ihn \textit{ja} und \textit{eben} in Assertionen bewirken oder auch \textit{eben} ihn in Direktiven auslöst (vgl. (\ref{1102})).
\pagebreak		
\begin{exe}							
\ex\label{1102} Kontextzustand nach einem \textit{auch}-Direktiv\\[-0.6em]
\begin{tabular}[t]{|C{6em}|C{12em}|C{6em}|}
\hline
$\textrm{DC}_{\textrm{A}}$ & Tisch &  $\textrm{DC}_{\textrm{B}}$ \tabularnewline
\hline
{} & {} & q  \tabularnewline
\cline{1-1}\cline{3-3}
$\textrm{TDL}_{\textrm{A}}$ & {} & $\textrm{TDL}_{\textrm{B}}$  \tabularnewline
\cline{1-1}\cline{3-3}
{} & {} & !p  \tabularnewline
\hline
\multicolumn{3}{|l|}{cg s$_{2}$ = $\lbrace$s$_{1}$ $\cup$ $\lbrace$!p $\in$ $\textrm{TDL}_{\textrm{B}}\rbrace$} \tabularnewline
\hline
\end{tabular}
\end{exe}										 
In Beispiel (\ref{1096}) ist dann die Relation in (\ref{1103}) beteiligt.

\begin{exe}
\ex\label{1103} 
	Wenn du Tante Ottilie besuchst, musst du artig sein.
\end{exe}	
Aus dem Kontext ist klar, dass Vater und Tochter sich einig sind, dass die Tochter die Tante besucht. Die Ermahnung des Vaters ist deshalb erwartbar. Da die Tochter weiß, dass dies das angemessene Verhalten ist, besteht für sie keine Möglichkeit, die Realisierungsabsichten hinsichtlich dieses Sachverhalts abzuleh\-nen. Die gleichen Verhältnisse stellen sich in den anderen Fällen in (\ref{1096}) bis (\ref{1100}) ein (vgl. die beteiligten Relationen in (\ref{1104})).

\begin{exe}
	\ex\label{1104} 
		\begin{xlist}	
			\ex\label{1104a} Wenn in der Schulzeit Nachmittag ist, musst du deine Hausaufgaben machen.
			\ex\label{1104b} Wenn es warm ist wie derzeit, musst du genug trinken.\footnote{In dem Beleg wird mit dieser Ermahnung eigentlich gespielt, 				indem vom Angesprochenen ein Bier hochgeladen wird, an das seine Mutter sicherlich nicht dachte.}
			\ex\label{1104c} Wenn der Nikolaus zufrieden sein soll, musst du dir die Zähne putzen.
			\ex\label{1104d} Wenn sie sich mit ihm verabredet, soll sie pünktlich sein.
		\end{xlist}
\end{exe}
In dieser Verwendung wird zu Verhaltensweisen aufgefordert, die normiert in bestimmten Situationen (die durch den aktuellen Kontext vorgegeben sind) zu realisieren sind, so dass es nicht verwunderlich ist, dass der Sprecher zu ihnen auffordert. Weil eine Norm beteiligt ist, kann auch gar kein Zweifel bestehen, dass der Adressat die Bereitschaft zur Ausführung der Handlung nicht ablehnt. Nicht umsonst handelt es sich hier illokutionär \is{Ermahnung} um Ermahnungen. \citet[179]{Rolf1997} nimmt an, dass \glqq der Adressat hinsichtlich dessen, was er tun oder unterlassen soll, bereits aufgefordert worden ist\grqq{}. Eine bereits explizit erfolgte Aufforderung liegt hier sicherlich nicht vor, aber der Adressat muss mit der Aufforderung rechnen. In meinen Augen entsteht der Eindruck von vorbeugenden Ermahnungen, die dazu auch etwas Belehrendes haben können. Denkbare Reaktionen sind beispielsweise \textit{Ja. Ja. Ich weiß./Natürlich.} Meiner Modellierung nach liegen grundsätzlich zu Assertionen parallele Verhältnisse vor, die aller\-dings gespiegelt sind, weil eine \textit{auch}-Assertion den Grund für die gegebene Folge und ein \textit{auch}-Direktiv die Folge aus der gegebenen Bedingung angibt. Im Falle von Assertionen ist die Relation p $>$ q im cg, q ist im Kontext und q wird über p begründet. Wenn der Adressat p akzeptiert, sollte das Vorliegen von q für ihn klar sein. Bei \textit{auch}-Direktiven ist die Relation q $>$ !p im cg, q trifft im Kontext zu, weshalb !p folgt.

Auch für die oben erwähnte zweite Gebrauchsweise (Gebrauch 1) eines \textit{auch}-Direktivs setze ich die Kontextbeschreibung in (\ref{1101}) an. Die Gebrauchsunterschiede ergeben sich aus der Gestalt von q. 

In dieser Verwendung hat der Adressat sich schon dazu bekannt, die Handlung auszuführen, sie ist bisher aber ausgeblieben, so dass für den Sprecher ein Grund entsteht, darauf hinzuweisen, dass ihr nachzukommen ist. Im Dialog in (\ref{1105}) ist z.B. aus dem Kontext klar, dass das Kind essen soll, und es gibt ein Hin und Her darum, wie es die Nudeln haben möchte, um sie zu essen. Schließlich sind sie so hergerichtet wie gefordert, und dennoch sieht es so aus, als ob es sie nicht essen wird.

\begin{exe}
	\ex\label{1105} 
	\scriptsize
 	K: Kei:ne Soße da:zu.\\
	V zu K: Du willst $\uparrow$keine Soße dazu?\\
	K zu V: Ne:e.\\
	V zu K: Sag mal, so langsam\\
	M zu K: spinnst du. (.) Saskia, stell dich nicht so an! ((zweiter Teil deutlich lauter))\\
	V zu K: Was machst du denn da?\\
	K zu V: Will keine So:ße $\downarrow$haben.\\
	V zu K: Keine Soße?\\
	K zu V: $^{0}\textrm{mhmh}^{0}$ ((verneinend)) ((K schüttelt vorsichtig den Kopf))\\
	V zu K: $\uparrow$Komm her, dann geb ich dir meine. (ein paar Wörter unverständlich) lass mal sehen. $\uparrow$Dann kriege ich die da mit der Soße. (.) 	So und du kriegst die hier $\uparrow$ohne Soße. (.) Bitteschön $>$ von Papa$<$ Ohne Soße. Bitteschön. Ohne Soße. Da! ((V nimmt sich Teller von K und 		schüttet Nudeln vom K auf seinen Teller und gibt K Nudeln ohne Soße zurück))\\
	((K fängt an zu quengeln))\\
	K: Der hat mir meine abgenom:men. ((zeigt auf seinen Teller und fängt an weinerlich zu schreien))\\
	V zu K: Du wolltest doch keine Soße haben. ((laut))\\
	((V guckt kurz vom Kind zu Mutter))\\
	V zu M: Also sowas! ((wieder bisschen leiser))\\
	V zu K: Du wolltest doch ohne Soße haben. (.) Ja, willst du wieder deine Soßendinger haben, oder $\uparrow$was? ((wieder laut))\\
	K zu V: Ja::a ((weinerlicher Ton, aber K hört auf zu quengeln))\\
	V zu K: Also doch mit Soße! ((laut))\\
	M zu K: Mm, eben hast gesagt \glqq ohne Soße\grqq{}, \textbf{$\uparrow$jetzt iss \underline{auch} ohne Soße.} ((K fängt wieder an zu quengeln))(.) (Ein 	paar Wörter unverständlich) zurück. ((auch laut))\\
	((K zeigt mit Finger auf die Schüssel Nudeln))
	\newline
	\hbox{}\hfill\hbox{(Keller 2015: 32-33., Die Entwicklung der Generation Ich Eine}				 
  	\newline
	\hbox{}\hfill\hbox{psychologische Analyse aktueller Erziehungsleitbilder, Springer}		
	\newline
	\hbox{}\hfill\hbox{(http://link.springer.com/book/10.1007\%2F978-3-658-10392-7)}
\end{exe}
Dem Kind kann zugeschrieben werden, dass es die Aufnahme von !p in seine TDL bereits akzeptiert hat. Wenn es zustimmt, die Nudeln ohne Soße zu essen, kann es nicht länger das Essen ablehnen, wenn der Zustand, unter dem es bereit ist zu essen, hergestellt ist.

In Beispiel (\ref{1106}) ist trotz des Versprechens die Handlung zur Heirat bisher noch nicht erfolgt. Auf der TDL von B steht somit das Vorhaben zu heiraten, die Rea\-lisierung ist aber noch nicht erfolgt bzw. es sieht nicht danach aus.
	
\begin{exe}
	\ex\label{1106} 
	\scriptsize
 	\glqq [...] Gleich nach der Hochzeit geht's nach Europa. Heute Abend noch. In ein paar Stunden. Lass uns
 	zu mir fahren. Der Priester wartet und die Gäste, und ...\grqq{}\\
	\glqq Nein, Stephen, es geht nicht. Ich kann dich nicht heiraten. Ich hab's mir anders ...\grqq{}\\
	\glqq Halt! Sag es nicht, Lorraine. Du hast behauptet, du liebst mich! Du willst mich heiraten, hast du versprochen. \textbf{Jetzt tu es 					\underline{auch}!}\grqq{}	
	\hfill\hbox{(Morgen 2009, Bis der Tod uns eint)}				 
  	\newline
	\hbox{}\hfill\hbox{(https://books.google.de/books?id=y1QOre76SQwC\&pg=PA235\&lpg=PA235}		
	\newline
	\hbox{}\hfill\hbox{\&dq=\%22Jetzt+tu+es+auch\%22\&source=bl\&ots=m9nDWXTBus\&sig=F8HC}	
	\newline
	\hbox{}\hfill\hbox{enpQ19vhLiP\_rsX7B7i8ZEk\&hl=de\&sa=X\&ved=0CCkQ6AEwA2oVChMItu}
	\newline
	\hbox{}\hfill\hbox{vHmZbHyAIVxSVyCh3wZQkS\#v=onepage\&q=\%22Jetzt\%20tu\%20es\%20au}
	\newline
	\hbox{}\hfill\hbox{ch\%22\&f=false)}
\end{exe}	
Ähnlich liegt in (\ref{1107}) die Situation vor, dass der Adressat gesagt hat, dass sie p tun werden. !p ist somit auf seiner TDL. Da die Handlung noch nicht ausgeführt wurde, scheint trotz der Ankündigung zweifelhaft, ob sie tatsächlich realisiert wird.
	
\begin{exe}
	\ex\label{1107} 
	\scriptsize
 	Meine Damen und Herren, ich habe bereits mehrfach gesagt, dass wir die Stellungnahmen, die eingegangen sind, insbesondere natürlich auch die der beiden 	kommunalen Landesverbände, ernst nehmen und den Gesetzentwurf deutlich überarbeiten. (Heiterkeit bei einzelnen Abgeordneten der CDU - Dr. Armin Jäger, 		CDU: \textbf{Dann machen Sie das \underline{auch}!} Dann machen Sie das!) Er wird mit anderen Worten nicht so in den Landtag eingebracht, wie er im 		November 2004 zur Anhörung gebracht wurde.	 	
	\newline
	\hbox{}\hfill\hbox{(PMV/W04.00054 Protokoll der Sitzung des Parlaments Landtag}
	\newline
	\hbox{}\hfill\hbox{Mecklenburg-Vorpommern am 10.03.2005. 54. Sitzung der 4. Wahlperiode}
	\newline
	\hbox{}\hfill\hbox{2002-2006. Plenarprotokoll, Schwerin, 2005)}
\end{exe}	
In dieser Gebrauchsweise von \textit{auch} in Direktiven liegt ebenso die Relation q $>$ !p im cg vor. Die Proposition q entspricht hier allerdings der komplexen Aussage !p $\in$ TDL$_{\textrm{B}}$ (vgl. (\ref{1108})).

\begin{exe}
\ex\label{1108} Kontextzustand vor dem \textit{auch}-Direktiv in Verwendung 1\\[-0.6em]
\begin{tabular}[t]{|C{6em}|C{12em}|C{6em}|}
\hline
$\textrm{DC}_{\textrm{A}}$ & Tisch &  $\textrm{DC}_{\textrm{B}}$ \tabularnewline
\hline
{} & {} & q  \tabularnewline
\cline{1-1}\cline{3-3}
$\textrm{TDL}_{\textrm{A}}$ & {} & $\textrm{TDL}_{\textrm{B}}$  \tabularnewline
\cline{1-1}\cline{3-3}
{} & p $\vee$ $\neg$p & !p  \tabularnewline
\hline
\multicolumn{3}{|l|}{cg s$_{1}$ = $\lbrace$q $>$ !p$\rbrace$} \tabularnewline
\hline
\end{tabular}
\end{exe}
Dieser Gebrauch wirkt vorwurfsvoller als der zuvor beschriebene. Bei Kontext 2 gibt es m.E. keine Anzeichen dafür, dass der Adressat der geforderten Handlung nicht nachkommen wird. Z.T. handelt es sich hierbei auch um völlig überflüssige Hinweise, die Reaktionen wie \textit{Ja. Ja.} oder \textit{Ich weiß.} hervorrufen (s.o.). In Kontext 1 besteht (trotz Bestätigung von !p) $[$das deshalb auf der TDL steht$]$ Zweifel, ob B p nachkommt.

Ich bin der Meinung, dass man diese beiden Kontexte, die in der Literatur in Ausführungen zu \textit{auch} in Direktiven in einem Zug erwähnt werden, trennen sollte. Der minimale Bedeutungsbeitrag \is{Bedeutungsminimalismus/-maximalismus} von \textit{auch} ist zwar derselbe (q $>$ !p im cg, q mindestens in DC$_{\textrm{B}}$), q nimmt jedoch jeweils eine andere Gestalt an, womit in Kontext 1 wiederum zusätzliche Füllungen der Komponenten einhergehen.
				
Neben diesen beiden Gebrauchsweisen, die in der Literatur Erwähnung finden, lässt sich dazu noch eine weitere Verwendung ausmachen: Es gibt Fälle, in denen die Aufforderung ableitbar ist, der Adressat ihr bisher noch nicht nachgekommen ist, er sich aber noch nicht dazu bekannt hat, p zu tun. Es handelt sich um Kontexte, in denen die Bedingung offen thematisiert ist (in Kontext 2 ist diese nur situativ gegeben), was mit sich bringt, dass der Aspekt der Ableitbarkeit noch salienter ist als in Kontext 2. Äußerungen wirken dadurch vorwurfsvoller als in Kontext 2. Anders als in Kontext 1 hat der Adressat !p aber nicht schon zuge\-stimmt. Beispiele für diesen Gebrauch finden sich in (\ref{1109}) und (\ref{1110}).

\begin{exe}
	\ex\label{1109} 
	\scriptsize
 	Bezeichnend ist der Ruf von Barisits nach einem Spieler wie Lukas Kulovits, der sicher nicht die fußballerische Qualität der Siegls hat, dafür in 			anderen Bereichen seine Vorzüge genießt. Gegen Winden muss jetzt ein Sieg her und diese Ausnahmekicker sind jetzt um so mehr gefordert, denn dass sie 		mit zu den besten im Land zählen, bleibt unbestritten. \textbf{Zeigt es \underline{auch}!}	
	\newline
	\hbox{}\hfill\hbox{(BVZ12/APR.00864 Burgenländische Volkszeitung, 12.04.2012)}
\end{exe}	

\begin{exe}
	\ex\label{1110} 
	\scriptsize
 	\glqq Er ist unser Sohn! Oder sag mir einen guten Grund, warum er es nicht mehr sein kann!\grqq{} \glqq Du würdest das nicht verstehen. Es ist eine 		Sache unter Männern.\grqq{} Der kleine Wagen ächzte, als Mechthild neben ihren Mann auf den schmalen Holzsitz stieg. \glqq Wenigstens erkennst du, dass 	Gerald ein Mann geworden ist. Dann behandle ihn auch so!\grqq{} Er funkelte sie an. \glqq Er muss den ersten Schritt tun!\grqq{}	
	\newline
	\hbox{}\hfill\hbox{(DIV/ERB.00001 Erwin, Birgit ; Buchhorn, Ulrich: Die Herren}
	\newline
	\hbox{}\hfill\hbox{von Buchhorn, $[$Roman$]$. - Meßkirch, 25.03.2011)}
\end{exe}
Die Relationen, die jeweils im cg enthalten sind, sind:

\begin{exe}
	\ex\label{1111} 
		\begin{xlist}	
			\ex\label{1111a} Wenn man zu den besten Spielern gehört, muss man es zeigen.
			\ex\label{1111b} Wenn Gerald ein Mann ist, musst du ihn wie einen behandeln.
		\end{xlist}
\end{exe}	
Der Direktiv ist hier ableitbar, wenn man die Relation teilt, die Bedingung vorerwähnt ist und sie als geteilte Information ausgegeben wird.

Da für alle \textit{auch}-Direktive gilt, dass die Relation q $>$ !p im cg enthalten ist und q im Kontext salient ist, muss es natürlich immer einen Grund geben, einen Direktiv zu äußern, um den der Adressat eigentlich weiß. Seine Äußerung kann aus Gründen der Versicherung erfolgen (Kontext 2 $[$(vorbeugende, belehrende) Ermahnung$]$). Sie kann auch dadurch bedingt sein, dass nicht klar ist, ob der Adressat p tatsächlich tut, obwohl er schon zugestimmt hat (Kontext 1 $[$Vorwurf$]$). Oder die Bedingung ist salient und es sieht aber nicht so aus, als ob der Adressat von allein die Folge erfüllt (Kontext 3 $[$Ermahnung und Vorwurf$]$). Die invariante Kontextanforderung von \textit{auch} ist in allen drei Gebrauchsweisen durch (\ref{1112}) erfasst.
 	
\begin{exe}
\ex\label{1112} Kontextzustand vor einem \textit{auch}-Direktiv\\[-0.6em]
\begin{tabular}[t]{|C{6em}|C{12em}|C{6em}|}
\hline
$\textrm{DC}_{\textrm{A}}$ & Tisch &  $\textrm{DC}_{\textrm{B}}$ \tabularnewline
\hline
{} & {} & q  \tabularnewline
\cline{1-1}\cline{3-3}
$\textrm{TDL}_{\textrm{A}}$ & {} & $\textrm{TDL}_{\textrm{B}}$  \tabularnewline
\cline{1-1}\cline{3-3}
{} & {} & {}  \tabularnewline
\hline
\multicolumn{3}{|l|}{cg s$_{1}$ = $\lbrace$q $>$ !p$\rbrace$} \tabularnewline
\hline
\end{tabular}
\end{exe}		
Die Relation ist Teil des cg und die Voraussetzung wird mindestens von B vertreten.
					
\subsubsection{Der Diskursbeitrag von \textit{auch}- und \textit{eben}-Direktiven}
Meine Modellierung für \textit{auch}- und \textit{eben}-Direktive ist im Sinne der vorliegenden Diskurszustände/-effekte dieselbe. Für letztere habe ich in Abschnitt~\ref{sec:kontexte} in Kapitel~\ref{chapter:hue} ebenfalls angenommen, dass dem Adressaten ein Auftrag erteilt wird, den er ableiten könnte (vgl. z.B. (\ref{1113})).

\begin{exe}
	\ex\label{1113} 
	A: Ich bin immer so müde.\\
	B: Dann geh \textbf{eben} früher ins Bett!
\end{exe}
Wenngleich anhand des Kontextzustandes, wie modelliert in Anlehnung an das Modell von \citet{Farkas2010}, eine Differenzierung zwischen \textit{auch}- und \textit{eben}-Direktiven nicht möglich ist, gibt es dennoch weitere Kriterien der Unterscheidung. Die \textit{eben}-Direktive haben für meine Begriffe eine eingeschränktere Verwendung, weil q ein Problem darstellt. Illokutiv sind sie deshalb \is{Ratschlag} Ratschläge. Dadurch, dass q als Problem ausgegeben wird und der eine Diskursteilnehmer den anderen um Hilfe bittet, signalisiert er, dass er nicht von allein auf !p gekommen wäre. Da die Relation im cg ist, müsste er dies eigentlich. Aus diesem Verhältnis folgt die harsche Wirkung, die \textit{eben}-Direktive im Gegensatz zu \textit{auch}-Direktiven aufweisen. Die \textit{auch}-Direktive werden hingegen i.d.R. verwendet, wenn gar nicht Gegenteiliges signalisiert wird hinsichtlich der Bereitschaft, p zu erfüllen. 							
\textit{Auch} kann in dem \textit{eben}-Kontext aus (\ref{1113}) auftreten (vgl. (\ref{1114})), d.h. das Vorliegen eines Problems interveniert nicht negativ mit dem \textit{auch}-Beitrag.

\begin{exe}
	\ex\label{1114} 
	A: Ich bin immer so müde.\\
	B: Dann geh \textbf{auch} früher ins Bett!
\end{exe}
Es handelt sich um die Verwendung aus Kontext 3. !p ist ableitbar, weil q $>$ !p im cg ist. Die Bedingung q ist zudem salient und nach dem Beitrag von A geteilte Information zwischen den Diskurspartnern.

\textit{Eben} kann hingegen nicht verwendet werden, wenn kein Problem besteht. In den Beispielen für den \textit{auch}-Kontext 2 ist die Ersetzung von \textit{auch} durch \textit{eben} nicht möglich (vgl. (\ref{1115}) bis (\ref{1119})).
	
\begin{exe}
	\ex\label{1115} 
	Trixis Vater freute sich und wünschte ihr eine gute Reise. \glqq Grüß  mir Tante Ottilie und \textbf{\#sei \underline{eben} artig!} Ich rufe heute 			Abend an.\grqq{}
\end{exe}	
	
\begin{exe}
	\ex\label{1116} 
	\scriptsize
	\glqq Bitte, schließ die Tür gut ab, wenn du die Wohnung verläßt. Und laß nachmittags, wenn Du allein zu Hause bist, ja keinen rein!\grqq{} ermahnte 		Papa sie.
	\glqq Ja, Papa.\grqq{}
	\glqq \textbf{\#Mach \underline{eben} deine Hausaufgaben!}\grqq{}
	\glqq Ich weiß schon selbst, daß ich meine Hausaufgaben machen muß!\grqq{} entgegnete Tina.
\end{exe}		

\begin{exe}
	\ex\label{1117} 
	\scriptsize
	Irgenwie schien er aber alle Kinder zu kennen und kannte sich auch sonst in der Stadt gut aus: \glqq Räum scheen dei Zimmer uff, \textbf{\#und putz dir 	\underline{eben} die Zähne}\grqq{} gab er den Kleinsten mit. 
\end{exe}	
		
\begin{exe}
	\ex\label{1118} 
	\scriptsize
	\glqq \textbf{\#Sei \underline{eben} pünktlich!}\grqq{} ermahnt mich Kaspar jedesmal, wenn wir uns verabreden. Dabei müßte er längst wissen, daß ich 		geradezu superpünktlich bin. 
\end{exe}		

\begin{exe}
	\ex\label{1119} 
	Anruf von Frau Mutter: \glqq \textbf{\#Trink \underline{eben} genug bei dem Wetter!}\grqq{}
\end{exe}	
Damit die Ersetzung zulässig wird, muss im Kontext ein Problem vorliegen. Am ehesten ist dies hier zu konstruieren für Beispiel (\ref{1120}):

\begin{exe}
	\ex\label{1120} 
	A: Oh, mir ist total schwindelig bei dieser Hitze.\\
	B: Trink \textbf{eben} genug bei dem Wetter!
\end{exe}	
Gleiches gilt für Kontext 1 und 3 (vgl. (\ref{1121}) bis (\ref{1126})).

\begin{exe}
	\ex\label{1121} 
	\scriptsize
 	Meine Damen und Herren, ich habe bereits mehrfach gesagt, dass wir die Stellungnahmen, die eingegangen sind, insbesondere natürlich auch die der beiden 	kommunalen Landesverbände, ernst nehmen und den Gesetzentwurf deutlich überarbeiten. (Heiterkeit bei einzelnen Abgeordneten der CDU - Dr. Armin Jäger, 		CDU: \textbf{Dann machen Sie das \underline{eben}!}) 	
\end{exe}

\begin{exe}
	\ex\label{1122} 
	\scriptsize
 	 \glqq Nein, Stephen, es geht nicht. Ich kann dich nicht heiraten. Ich hab s mir anders ...\grqq{}\\
	\glqq Halt! Sag es nicht, Lorraine. Du hast behauptet, du liebst mich! Du willst mich heiraten, hast du versprochen. \textbf{\#Jetzt tu es 					\underline{eben}!}\grqq{} 	
\end{exe}	

\begin{exe}
	\ex\label{1123} 
	\scriptsize
 	V zu K: Also \textit{doch} mit Soße! ((laut))\\
	M zu K: Mm, eben hast gesagt \glqq ohne Soße\grqq{}, \textbf{\#$\uparrow$jetzt iss \underline{eben} ohne Soße}. ((K fängt wieder an zu quengeln))(.) 		(Ein paar Wörter unverständlich) zurück. ((auch laut))\\
	((K zeigt mit Finger auf die Schüssel Nudeln))	
\end{exe}

\begin{exe}
	\ex\label{1124} 
	\scriptsize
 	Bezeichnend ist der Ruf von Barisits nach einem Spieler wie Lukas Kulovits, der sicher nicht die fußballerische Qualität der Siegls hat, dafür in 			anderen Bereichen seine Vorzüge genießt. Gegen Winden muss jetzt ein Sieg her und diese Ausnahmekicker sind jetzt um so mehr gefordert, denn dass sie 		mit zu den besten im Land zählen, bleibt unbestritten. \textbf{\#Zeigt es \underline{eben}!}	
\end{exe}	

\begin{exe}
	\ex\label{1125} 
	\glqq Wenigstens erkennst du, dass Gerald ein Mann geworden ist. \textbf{\#Dann behandle ihn \underline{eben} so!}\grqq{} Er funkelte sie an.
\end{exe}	

\begin{exe}
	\ex\label{1126} 
	Na, na, du behauptest mir sei zu hoch, was du sagen willst? \textbf{Bitte, dann erkläre dich \underline{eben}!}
\end{exe}
In (\ref{1126}) ist der Vorgangsbeitrag einigermaßen passabel als Problem zu deuten (vgl. (\ref{1127})).

\begin{exe}
	\ex\label{1127} 
	A: Ihr versteht mich ja alle nicht.\\
	B: Dann erkläre dich \textbf{eben}!
\end{exe}
In (\ref{1121}) verliert die Äußerung ihre direktive Interpretation, wenn \textit{eben} auftritt. 
	
Ich halte in allen diesen Beispielen \textit{auch} für adäquater. Damit \textit{eben} auftreten kann, müsste man beim Adressaten eine problematische Ausgangslage schaffen (vgl. die veränderten Kontexte in (\ref{1128}) bis (\ref{1130})). Diese kann ebenfalls bei \textit{auch} vorliegen (obwohl dies normalerweise nicht der Fall ist), sie ist aber nicht notwendig wie im Falle von \textit{eben}.

\begin{exe}
	\ex\label{1128} 
	A: Ich komme mit Gerald gar nicht mehr klar. Er behauptet, jetzt ein Mann zu sein. \\
	B: Dann behandle ihn \textbf{eben} so.
\end{exe}

\begin{exe}
	\ex\label{1129} 
	A: Ich will nicht, dass wir uns bei jeder Verabredung streiten.\\
	B: Sei \textbf{eben} pünktlich.
\end{exe}
		
\begin{exe}
	\ex\label{1130} 
	A: Meine Eltern machen schon Witze, dass ich die Hochzeit versprochen habe und immer noch nichts passiert ist.\\
	B: Dann tu es \textbf{eben}!
\end{exe}
Ich halte die Beobachtung, die Anlass zu diesem Abschnitt gibt, für relevant, weil sie aufzeigt, dass sich nicht jegliche Unterschiede zwischen MPn mit den durch das Diskursmodell vorgegebenen Kategorien erfassen lassen.

\subsubsection{Deskriptive Eindrücke aus der Literatur}
Nach meiner Betrachtung authentischer Belege bewerte ich die Annahmen aus der Literatur (s.o.) folgendermaßen: Die Handlungsaufforderung ist in dem Sinne bekannt, dass sie abzuleiten ist. Damit ist verbunden, dass beim Adressaten wenig Einspruchsmöglichkeit besteht. Aus diesen Verhältnissen wird zudem er\-klärbar, warum die Partikel auch in \textit{dass}-Direktiven auftreten kann, die als starke Direktive gelten.

Ein \textit{auch}-Direktiv kann der erste Befehl sein im Gespräch. Wiederholt wird der \textit{auch}-Direktiv nur in Szenarien entlang von Kontext 1 und da aber auch nur in dem Sinne, dass der Adressat !p schon zugestimmt hat. Nur wiederholen, weil der Aufforderung noch nicht nachgekommen wurde, kann man die \textit{auch}-Direktive für meine Begriffe nicht.

Die Eltern-Kind-Interaktion ist für Kontext 2 ein typisches Interaktionsverhältnis, es muss aber nicht vorliegen. Ich glaube, dass es ein beliebtes Schema ist, das sich deshalb gut eignet, weil a) eine asymmetrische Relation benötigt wird, b) eine zu realisierende Handlung als erwartet ausgegeben werden muss und c) kein Problemlösungsszenario im Raum zu stehen hat wie bei den \textit{eben}-Direktiven. Als Umstände bieten sich hier typischerweise Verhaltskontexte an, die wiederum oft von Eltern an Kinder gerichtet sind. Die obigen Belege zeigen bereits, dass dies nicht so sein muss (vgl. Nikolaus in (\ref{1099}), Freundin und Freund in (\ref{1100})).
 							 
Denkbar ist beispielsweise auch die Anweisung eines Herrchens an seinen Hund (vgl. (\ref{1131})) oder der Rat einer Freundin (vgl. (\ref{1132})).
	
\begin{exe}
	\ex\label{1131} 
	Komm, mach \textbf{auch} Platz!
\end{exe}	
\vspace{-0.5cm}	
\begin{exe}
	\ex\label{1132} 
	\scriptsize
 	\textbf{Dann sei \underline{auch} artig}, Babsi und höre darauf, was das Dokterchen sagt. Sonst hast du nachher wieder Beschwerden, das wäre doch auch 		schlecht. Also Geduld, auch wenn es schwer fällt.
	\newline
	\hbox{}\hfill\hbox{(http://www.forum-garten.de/was-habt-ihr-heute-}
	\newline
	\hbox{}\hfill\hbox{alles-im-garten-gemacht-t243395,start,1130.htm)}
	\newline
	\hbox{}\hfill\hbox{(Google-Suche, eingesehen am 13.12.2015)}
\end{exe}
Man könnte aufgrund der Beispiele denken, dass naive Kontexte dominant vertreten sind. Ihr Vorkommen liegt m.E. daran, dass man nur in wenigen Interaktions\-zusammenhängen das typische \textit{Sei \textbf{auch} artig/brav.} sagt.

Auch Kontexte wie in (\ref{1133}) und (\ref{1134}) halte ich für denkbar.

\begin{exe}
	\ex\label{1133} 
	(Arzt in Menschenmenge)\\
	A: Zum Glück. Endlich ein Arzt!\\
	B: (Muss Leute beiseite schieben.) Lassen Sie mich \textbf{auch} durch! 
\end{exe}

\begin{exe}
	\ex\label{1134} 
	(Polizei zu Menschenmenge):\\
	Jetzt machen Sie \textbf{auch} Platz!, Bilden Sie \textbf{auch} eine Gasse!
\end{exe}	
Meiner Meinung nach sind die \textit{auch}-Direktive in genau den Kontexten zulässig, in denen \textit{auch}-E- bzw. -w-Fragen bzw. -Assertionen auftreten können. Die beteiligte Relation ist stets die gleiche. Sofern in diesen Kontexten Direktive überhaupt geäußert werden können, sollten \textit{auch}-Direktive möglich sein (vgl. (\ref{1135}) bis (\ref{1138})).
								
\begin{exe}
	\ex\label{1135} 
	A: Der Nikolaus war nett zu uns. (q)\\
	B: Ihr wart \textbf{auch} artig dieses Jahr. (p)\\
	$[$p $>$ q \glq Wenn ihr artig wart, ist der Nikolaus nett zu euch!\grq {}$]$
\end{exe}	

\begin{exe}
	\ex\label{1136} 
		\begin{xlist}
			\ex\label{1136a} Seid \textbf{auch} artig im Jahr! (!p) (Sonst gibt es Probleme mit dem Nikolaus) (Kontext 2)
			\ex\label{1136b} A: Wir wollen, dass der Nikolaus nett zu uns ist. (q)\\
							 B: Dann seid das Jahr über \textbf{auch} artig! (!p)\\
		$[$q $>$ !p \glq Wenn ihr wollt, dass der Nikolaus nett ist, müsst ihr das Jahr über artig sein.\grq$]$					 
		\end{xlist}
\end{exe}	

\begin{exe}
	\ex\label{1137} 
	A: Der Nikolaus war gar nicht nett zu uns.\\
	B: Warum wart ihr \textbf{auch} nicht artig dieses Jahr?\\
	$[$p $>$ q \glq Wenn ihr artig wart, ist der Nikolaus nett zu euch.\grq\\
	$\neg$q $>$ $\neg$p ($\neg$p ist präsupponiert in der \textit{Warum}-Frage)$]$
\end{exe}

\begin{exe}
	\ex\label{1138} 
	Nikolaus: Wart ihr \textbf{auch} artig?\\
	(mit Antwortpräferenz zu p, weil sie wollen, dass der Nikolaus nett zu ihnen ist)\\
	$[$p $>$ q \glq Wenn ihr artig wart, ist der Nikolaus nett zu euch.\grq$]$
\end{exe}
In den anderen beiden Verwendungskontexten von \textit{auch}-Direktiven spielt die Annahme zu den Interaktionspartnern aus der Literatur keine Rolle.

Die Analyse der Beispiele und Belege zeigt, dass \textit{auch}-Äußerungen immer einen situativen oder dialogischen Kontext benötigen. Wie \textit{halt}- und \textit{eben}-Äuße\-rungen sind sie im engeren Sinne reaktiv. Diese Erkenntnis, dass q kontextuell verfügbar sein muss, passt zum Eindruck der \textit{auch}-Direktive aus \citet[60]{Dittmann1980}, dass sie sich \glqq auf die Situation selbst und einen Zeitraum unmittelbar nach dem Sprechzeitraum\grqq{} beziehen. Im Gegensatz zu den assertiven Fällen ist der monologische Gebrauch (in dem q nur vom Sprecher vertreten wird) auszuschließen, weil der Direktiv immer an den Adressaten gerichtet ist. Monologische Fälle gibt es, wenn man auch Vorhaben des Sprechers miteinbezieht. Hierbei handelt es sich um Fälle, in denen der Sprecher eine Handlungsabsicht bekundet und sich (bzw. einer größeren Gruppe) somit selbst eine zur Realisierung ausstehende Proposition auf die TDL legt. Beispiele sind hier für alle drei Kontexte zu konstruieren (vgl. (\ref{1139}) bis (\ref{1141})).

\begin{exe}
	\ex\label{1139} 
	Ich besuche morgen Oma. Ich werde \textbf{auch} artig sein.
\end{exe}

\begin{exe}
	\ex\label{1140} 
		\begin{xlist}
			\ex\label{1140a} 
				A: Du hast gesagt, unter diesen Umständen isst du es.\\
				B: Ja. Ich werde es \textbf{auch} essen.
			\ex\label{1140b} 
				A: Du hast versprochen, mich zu heiraten.\\
				B: Ich werde es \textbf{auch} machen.
		\end{xlist}
\end{exe}	

\begin{exe}
	\ex\label{1141} 
	A: Ihr seid Spitzenspieler.\\
	B: Wir werden es \textbf{auch} zeigen.
\end{exe}	
	
\subsubsection{Gibt es eine assertive Folge-Verwendung von \textit{auch}?}
Im Zusammenhang mit derartigen Verwendungen von \textit{auch} tritt noch ein weiterer Aspekt zu Tage. Ich habe eingangs schon geschrieben, dass in Darstellungen zu \textit{auch} meist nur Assertionen untersucht werden. Es gibt Autoren, die Direktive nicht betrachten, weil unklar sei, ob es sich beim Vorkommen von \textit{auch} in diesem Satzkontext überhaupt um die MP-Verwendung handelt (\citealt[222]{Karagjosova2004}). Ich denke, dass dieser Punkt beim Auftreten von \textit{auch} in Direktiven tatsächlich eine Rolle spielt.

Eindeutig als MP einstufen kann man \textit{auch} in Kontext 2. Bei den Beispielen handelt es sich um die klassischen Beispiele in der Literatur zum Thema. Zudem gibt es Kontext 1, der diejenigen Fälle auffängt, zu denen es in der Literatur heißt, dass der Direktiv wiederholt wird. Auf der Basis der Betrachtung von Belegen habe ich zudem Kontext 3 hinzugenommen.

Ich bin der Meinung, dass man mit der Reihung \textit{Kontext 2 – 1 – 3} die wörtliche Bedeutung eines Adverbs \textit{auch} immer deutlicher spürt, so dass die Verwendungen von \textit{auch} sich in Kontext 1 und 3 in die Richtung des Adverbs bewegen. Der Beitrag lässt sich zwar nicht durch \textit{ebenfalls} ersetzen, diese Bedeutung ist aber zunehmend spürbar. Da die Direktive aber auch in diesen Kontexten die Folge anzeigen, fügen sie sich gut in meine Ableitung.

Für die meisten MPn gilt, dass die Verbindung zu den Vormodalpartikellexemen \is{Vormodalpartikellexeme} wahrnehmbar ist. Dies trifft auch auf die Additivlesart von \textit{auch} zu: Bei erfüllter Bedingung tritt ebenfalls die Folge ein (Direktive) bzw. die Folge hat Gültigkeit und zusätz\-lich greift die Bedingung, die sie begründet (Assertionen). Dieses additive Moment scheint mir innerhalb der Direktiv-Verwendungen unterschiedlich stark durchzudringen.

In Beispielen der Art in (\ref{1142}) ist der additive Beitrag am wenigsten wahrnehmbar: Er liegt vor in dem Sinne, dass der Besuch bei der Tante mit dem Artigsein einhergeht. Die Tante kann nicht nur besucht werden, man muss zusätzlich/ebenfalls artig sein.

\begin{exe}
	\ex\label{1142} 
	Trixis Vater freute sich und wünschte ihr eine gute Reise. \glqq Grüß' mir Tante Ottilie und \textbf{sei \underline{auch} artig}! Ich rufe heute Abend 		an.\grqq{}
\end{exe}
In Beispiel (\ref{1143}), das für Kontext 1 steht, ist der Wille des Kindes erfüllt, weshalb der Sprecher davon ausgeht, dass es der Handlung jetzt nachkommen wird. Das \textit{auch} im Direktiv drückt aus, dass man nicht nur \underline{sagen} kann, dass man etwas tut (hier die Nudeln ohne Soße zu essen), man muss es zusätzlich/ebenfalls tun.

\begin{exe}
	\ex\label{1143} 
	\scriptsize
 	V zu K: Also \textit{doch} mit Soße! ((laut))\\
	M zu K: Mm, eben hast gesagt \glqq ohne Soße\grqq{}, \textbf{\#$\uparrow$jetzt iss \underline{eben} ohne Soße}. ((K fängt wieder an zu quengeln))(.) 		(Ein paar Wörter unverständlich) zurück. ((auch laut))\\
	((K zeigt mit Finger auf die Schüssel Nudeln))	
\end{exe}
In (\ref{1144}) wird ausgesagt, dass man nicht nur bester Spieler sein kann, sondern dass damit ebenfalls einhergehen muss, dass man es sehen kann. Deshalb werden sie aufgefordert, es zu zeigen.

\begin{exe}
	\ex\label{1144} 
	\scriptsize
	Gegen Winden muss jetzt ein Sieg her und diese Ausnahmekicker sind jetzt um so mehr gefordert, denn dass sie mit zu den besten im Land zählen, bleibt 		unbestritten. \textbf{Zeigt es \underline{auch}}! 
\end{exe}
Für meine Begriffe nimmt der erkennbare additive Beitrag von (\ref{1142}) über (\ref{1143}) zu (\ref{1144}) zu.

Mir ist keine Arbeit bekannt, in der für Assertionen angenommen worden ist, dass sie auch die Folgelesart haben können. Dies ist eigentlich merkwürdig, da diese bei den ansonsten ähnlichen MPn \textit{halt} und \textit{eben} möglich ist. Zählt man die Kontexte 1 und 3 zu MP-Verwendungen, wäre zu überlegen, ob Beispiele der folgenden Art nicht auch MP-Verwendungen sind, die eben nicht kausal sind. 

Am nächsten kämen einer MP \textit{auch} in assertiver Folge Äußerungen mit performativ gebrauchten \is{performatives Modalverb} Modalverben, wie z.B. in (\ref{1145}).

\begin{exe}
	\ex\label{1145} 
	\scriptsize
 	Sie machen sie dem Volk zugänglich und gewöhnen es daran. Ohne Gemeindeeinrichtungen kann sich ein Volk eine freie Regierung geben, aber den Geist der 		Freiheit besitzt es nicht. Für uns geht es darum, daß man auch aus der Armut eine Antwort sucht im Geist der Freiheit. Das kann man nur, wenn man eine 		politische Handlungsmöglichkeit findet. (Jan Ehlers SPD: \textbf{Das heißt aber nicht nur reden, Sie müssen \underline{auch} handeln!} Sie müssen den 		Tocqueville auch richtig verstehen!) - Darauf gehe ich gern gleich ein, daß das auch handeln heißt. 
	\newline
	\hbox{}\hfill\hbox{(PHH/W16.00001 Protokoll der Sitzung des Parlaments Hamburgische Bürgerschaft am}
	\newline
	\hbox{}\hfill\hbox{ 08.10.1997. 1. Sitzung der 16. Wahlperiode 1997-2001. Plenarprotokoll, Hamburg, 1997)}
\end{exe}
Die assertive MP-Äußerung entspricht hier Kontext 1: \textit{Dann handeln Sie \textbf{auch} so!}.

Als nächstes wären auf einer \glq Skala\grq {} von MP-Adverb-Qualität unpersönliche Strukturen mit Modalverben einzuordnen (vgl. (\ref{1146})).

\begin{exe}
	\ex\label{1146} 
	\scriptsize
 	\glqq Genauso wenig wie sich Märkte fair von alleine regulieren, trifft Deutschland von allein die richtige Entscheidung. Nichtwählen würde deshalb nie 	für mich infrage kommen – wer eine Stimme hat, \textbf{sollte sie \underline{auch} nutzen!} In diesem Jahr setze ich mein Kreuz bei der SPD und hoffe 		auf eine rot-grüne Koalition. $[$...$]$\grqq{}	
	\hfill\hbox{(HMP13/SEP.01762 Hamburger Morgenpost, 20.09.2013)}
\end{exe}
Würde eine Person direkt angesprochen, wäre ein \textit{auch}-Direktiv denkbar:

\begin{exe}
	\ex\label{1147} 
	Du hast eine Stimme? Nutze sie \textbf{auch}!
\end{exe}
Und noch tiefer auf der Skala stufe ich Fälle der Art in (\ref{1148a}) mit unpersönlichen Subjekten und indikativischen Verben, die hier m.E. performativ verwendet werden, ein.

\begin{exe}
	\ex\label{1148a} 
	\scriptsize
	\glqq Was man angefangen hat, \textbf{macht man \underline{auch} fertig}!\grqq{}, und so singt er von Abbrechern in der Schule, beim Klavierunterricht 		und in der Ausbildung. 
	\newline
	\hbox{}\hfill\hbox{(RHZ09/NOV.01056 Rhein-Zeitung, 02.11.2009)}
\end{exe}

\begin{exe}
	\ex\label{1148} 
	Du hast es angefangen. Jetzt mach es \textbf{auch} fertig!
\end{exe}
Die Beispiele in (\ref{1146}) und (\ref{1148a}) entsprächen meinem dritten Kontext. 

Diese Verwendungen kontrastieren mit anderen Fällen, in denen \textit{auch} in Folgen auftritt, in denen für meine Begriffe eindeutig das Adverb (mit weitem Skopus) vorliegt (vgl. z.B. (\ref{1149}) bis (\ref{1151})).

\begin{exe}
	\scriptsize
	\ex\label{1149} 	
	\scriptsize
	{Der Lohn war die frühe Führung: Ein Freistoß von Andac Güleryüz klatschte ans Aluminium – Nico Granatowski setzte nach und traf zum 0:1 (15.). 				\textbf{Und wenn der VfL II erst einmal führte, dann hatte er in der laufenden Saison \underline{auch} noch nicht verloren} – diese Serie hielt bis 		gestern Abend. 
	\newline
	\hbox{}\hfill\hbox{(BRZ13/MAI.05814 Braunschweiger Zeitung, 16.05.2013)}}\\
	$[$Wenn sie geführt haben, galt ebenfalls: Sie haben noch nicht verloren.$]$
\end{exe}

\begin{exe}
	\ex\label{1150} 
	\scriptsize
	{\glqq $[$...$]$ Etwas schade ist, dass wir dieses Jahr keine Halbfinalbegegnungen austragen konnten. Aber nächstes Jahr, wenn die Kinder am Samstag 		schulfrei haben, beginnen wir früher \textbf{und sind dann \underline{auch} in der Lage, die im Rahmen der sportlichen Fairness notwendigen Halbfinals 		auszutragen}.\grqq{}
	\newline
	\hbox{}\hfill\hbox{(A97/JUN.09605 St. Galler Tagblatt, 16.06.1997)}}\\
	$[$Wenn wir früher beginnen, gilt ebenfalls: Wir können die Halbfinals austragen.$]$
\end{exe}

\begin{exe}
	\ex\label{1151} 
	\scriptsize
	{\glqq $[$...$]$ Ich mag Feldschlösschen, aber wenn ich im Appenzellerland bin, \textbf{dann will ich \underline{auch} Bier von dort trinken}.\grqq{}   
	\hfill\hbox{(A97/AUG.19304 St. Galler Tagblatt, 20.08.1997)}}\\
	$[$Wenn ich im AL bin, dann gilt ebenfalls: Ich will ein Bier von dort trinken.$]$
\end{exe}
Ich denke, dass die Aspekte der Performativität und Handlungsorientierung eine Rolle spielen, wenn sich die \textit{auch}-Verwendung in assertiven Folgen der MP \textit{auch} in Direktiven, die immer Folgen darstellen, annähert. Im Gegensatz zu (\ref{1145}), (\ref{1146}) und (\ref{1148a}) sind in (\ref{1149}) bis (\ref{1151}) direktive Pendants gar nicht denkbar.

Das Vorkommen von \textit{auch} in Fall 1 und 3 bei den Direktiven stellt in meinen Augen den Übergang zur Adverbverwendung dar. Man hat es hier mit einer gewissen Grauzone zu tun. Wenn die Performativität ausbleibt, liegt auch in Kontext 2 eher das Adverb vor. 

Von (\ref{1152}) zu (\ref{1154}) nimmt die MP-Qualität zu.
	
\begin{exe}
	\ex\label{1152} 
	Wenn Paula Tante Anne besucht, muss sie \textbf{auch} artig sein.
\end{exe}	
\vspace{-0.6cm}	
\begin{exe}
	\ex\label{1153} 
	Du besuchst Tante Anne? Dann musst du \textbf{auch} artig sein!
\end{exe}	
\vspace{-0.6cm}
\begin{exe}
	\ex\label{1154} 
	Du besuchst Tante Anne? Dann sei \textbf{auch} artig!
\end{exe}	
Nachdem nun das Einzelauftreten von \textit{doch} und \textit{auch} in Direktiven untersucht wurde, geht es im folgenden Abschnitt um die Sequenz \textit{doch auch}. Neben der Klärung der Frage, in welchen Arten direktiver Äußerungen die Kombination möglich ist, gilt es vor allem, festzustellen, ob sich die glei\-che Erklärung der präferierten Sequenz \textit{doch auch} anbietet, wie ich sie in Abschnitt~\ref{sec:kombida} für Assertionen vertrete. 

\subsection{Das kombinierte Auftreten von \textit{doch} und \textit{auch}}
\label{sec:kombida}
Die Kombination der beiden Partikeln ist immer dann möglich, wenn die Kontextanforderungen, die ich in Abschnitt~\ref{sec:dadir} formuliert habe, beide erfüllt sind: Zum einen folgt die Handlung, zu der aufgefordert wird, aus der Situation/Handlung/dem Vorgängerbeitrag und dieser Zusammenhang stellt eine allgemein gültige Norm dar (\textit{auch}). Zum anderen ist es fraglich, ob der Adressat p realisieren wird (\textit{doch}). Ich gehe folglich wiederum davon aus, dass sich die Bedeutung der MP-Kombination additiv ergibt und die Partikeln nicht Skopus \is{Skopus} übereinander nehmen. Weiter unten spiele ich die Interpretation, die unter Skopus resultiert, ebenfalls durch.

\subsubsection{Vorkommensweisen}
Einfach belegen lässt sich \textit{doch auch} in den Kontexten 1 und 3, d.h. bei wiederholter Anweisung bzw. erfüllter, salienter Vorbedingung.

In (\ref{1155}) vertreten die Parteien das Ziel der Verbesserung der Bildungschancen. Bisher sind aber keine Taten gefolgt, weshalb fraglich ist, ob es tatsächlich umgesetzt wird. !p befindet sich auf der TDL, die Auflösung von p $\vee$ $\neg$p in Richtung p ist aber noch nicht erfolgt, woraus sich der verstärkte Zweifel ergibt, ob p noch realisiert werden wird. Die beteiligte Relation ist \glq Wenn man das Ziel hat, die Bildungschancen zu verbessern, muss dieses Vorhaben umgesetzt werden.\grq {}.

\begin{exe}
	\ex\label{1155} 
	\scriptsize
	Bei so viel Schimpfen kann man leicht das Ziel aus den Augen verlieren. Bei der jüngsten Bundestagswahl sind nahezu alle Parteien mit dem Ziel 				angetreten, die Bildungschancen zu verbessern. \textbf{Nun setzt das \underline{doch auch} endlich um!} Wer Bildungschancen für unsere Kinder nicht 		einschränken will, muss für alle Kinder, egal wo sie zur Schule gehen, die Beförderungskosten übernehmen – und zwar bis zum Abitur. 		
	\hfill\hbox{(RHZ09/OKT.16499 Rhein-Zeitung, 19.10.2009)}
\end{exe}
In (\ref{1156}) ist die beteiligte Relation \glq Wenn man sagt, dass man etwas tut, macht man es.\grq {} und die Offenheit von p entsteht hier durch die Information, dass die angesprochene Person schon wiederholt gesagt hat, dass sie zum Arzt gehen wird, es aber nicht getan hat, weshalb verstärkt fraglich ist, ob sie es tut. !p ist somit schon in TDL$_{\textrm{B}}$, im cg ist enthalten, dass !p ein Element von TDL$_{\textrm{B}}$ ist, p $\vee$ $\neg$p liegt auf dem Tisch.

\begin{exe}
	\ex\label{1156} 
	\scriptsize
	\glqq Ich dachte, dir geht's wieder besser ...\grqq{}, meinte sie.\\
	\glqq Ja, mir ging's auch wieder besser ... es hat nur gerade wieder angefangen ...\grqq{}\\
	\glqq Du solltest vielleicht mal zum Osteopathen gehen.\grqq{}\\
	\glqq Ich werde zum Osteopathen gehen. Édouard hat mir schon einen empfohlen.\grqq{}\\
	\glqq Naja, aber \textbf{dann geh \underline{doch auch} hin}. Du redest immer nur und tust dann doch nichts.\grqq{}\\	
	\glqq Ich geh ja hin ...\grqq{}
	\hfill\hbox{(Foenkinos 2013, Zum Glück Pauline)}
	\newline
	\hbox{}\hfill\hbox{(https://books.google.de/books?id=tfotAAAAQBAJ\&pg=PT83\&lpg=}
	\newline
	\hbox{}\hfill\hbox{PT83\&dq=\%22Dann+geh+doch+auch\%22\&source=bl\&ots=GgydR-}
	\newline
	\hbox{}\hfill\hbox{XHZZ\&sig=dbVCZbobwlBNDmIKhNFHduYaEAM\&hl=de\&sa=X\&}
	\newline
	\hbox{}\hfill\hbox{ved=0CC4Q6AEwA2oVChMIiJL3yrrPyAIVg1osCh369Ae6\#v=onep}
	\newline
	\hbox{}\hfill\hbox{age\&q=\%22Dann\%20geh\%20doch\%20auch\%22\&f=false)}
\end{exe}
Die Beispiele zeigen, dass die Kombination \textit{doch auch} im \textit{auch}-Kontext 1 gut stehen kann. Der Adressat hat den Auftrag schon akzeptiert, die Realisierung ist aber bisher ausgeblieben. p $\vee$ $\neg$p steht folglich im Raum. \textit{Doch} kann hinzutreten, wodurch die Frage, ob der Adressat p nachkommen wird, zusätzlich hervorgehoben wird.

Es lassen sich ebenfalls Belege nachweisen, die meinem dritten Kontext entspre\-chen. Dies gilt z.B. für (\ref{1158}).

\begin{exe}
	\ex\label{1158} 
	\scriptsize
	Warum sagen Sie den Leuten sogar nach der Bundestagswahl noch, wo doch nun alles gelaufen ist, Dinge, für die Sie nicht eine müde Mark im Haushalt 			werden aufbringen können? Warum machen Sie das? – Sie schaden nicht nur Ihrem Ansehen, sondern Sie schaden auch dem Ansehen der gesamten Politik. Sie 		wollen doch Staatsmann werden. \textbf{Dann verhalten Sie sich \underline{doch auch} entsprechend!} 
	\newline
	\hbox{}\hfill\hbox{(PNI/W14.00013 Protokoll der Sitzung des Parlaments Landtag Nieder-}
	\newline
	\hbox{}\hfill\hbox{sachsen am 29.10.1998. 13. Sitzung der 14. Wahlperiode 1998-2003.}
	\newline
	\hbox{}\hfill\hbox{Plenarprotokoll, Hannover, 1998)}
\end{exe}
Beteiligt ist die Relation: \glq Wenn man Staatsmann sein will, benimmt man sich wie ein Staatsmann.\grq {}. Die Bedingung ist offen thematisiert. Der Angespro\-chene will Staatsmann werden. Momentan zeigt er aber nicht das entsprechende Verhalten, weshalb fraglich ist, ob er sich in Zukunft so benehmen wird. Die Handlungsanweisung ist klar, wenn die obige Relation anerkannt wird.

Ähnlich kann man in (\ref{1159}) als Relation, auf die \textit{auch} Bezug nimmt, ansetzen: \glq Wenn die Verhältnisse auf bestimmte Art beschaffen sind, dann teilt diese mit.\grq {}. Bis zum Sprechzeitpunkt sagen die Angesprochenen dies scheinbar nicht, woraus sich ergibt, dass fraglich ist, ob sie es sagen werden. Auf diese offene Frage bezieht sich \textit{doch}.
	
\begin{exe}
	\ex\label{1159} 
	\scriptsize
	Das ist dann aber keine rentenrechtliche Frage, sondern eine sozialpolitische Frage. Dann geht es letzten Endes darum, dass ihr nur die Grundsicherung 		im Alter von dem jetzigen Betrag von 680 Euro auf 1 050 Euro anheben wollt. \textbf{Das sagt dann \underline{doch auch}! }
	\newline
	\hbox{}\hfill\hbox{(PBT/W17.00198 Protokoll der Sitzung des Parlaments Deutscher Bundestag am}
	\newline
	\hbox{}\hfill\hbox{s18.10.2012. 198. Sitzung der 17. Wahlperiode 2009-. Plenarprotokoll, Berlin, 2012)}
\end{exe}	
Im \textit{auch}-Kontext 2 steht überhaupt nicht zur Diskussion, ob p realisieren werden wird oder nicht. Eine Gegenreaktion liegt nicht bereits vor und mit ihr ist auch nicht zu rechnen. Die Handlungsaufforderung wird mitgeteilt und der Adressat verpflichtet sich dazu, ihr nachzukommen. Er würde sich vermutlich aber von allein nicht anders verhalten. Dies ist der Grund, aus dem \textit{doch} nicht gut zum \textit{auch} hinzutreten kann. In Kontext 1 scheint mir dies stets möglich zu sein, weil trotz Ankündigung, p zu tun, dies noch nicht erfolgt ist. Bei Kontext 3 scheint die Ausgangslage auch so zu sein, dass die Bedingung gerade deshalb thematisiert wird, weil die damit einhergehende (bekannte) Folge bisher nicht realisiert wurde oder fraglich ist, ob sie es wird.

In Kontext 2 kann \textit{doch} nicht ohne Weiteres hinzutreten. Man kann dem bei der Tante abgesetzten Kind nicht (\ref{1160}) mit auf den Weg geben.

\begin{exe}
	\ex\label{1160} 
	\#Sei \textbf{doch}/\textbf{doch auch} artig!
\end{exe}	
Man kann jemandem, der etwas mit Präzision schreiben muss, über die Schulter (\ref{1161}) zukommen lassen.

\begin{exe}
	\ex\label{1161} 
	Schreib \textbf{auch} ordentlich!
\end{exe}	
Wenn kein weiterer Grund zur Annahme besteht, dass der Adressat dies viel\-leicht nicht tut, ist (\ref{1162}) allerdings keine denkbare Äußerung in diesem Kontext.

\begin{exe}
	\ex\label{1162} 
	\#Schreib \textbf{doch}/\textbf{doch auch} ordentlich!
\end{exe}
Damit \textit{doch} hinzutreten kann, muss die Fraglichkeit der Realisierung von p in den Kontext gelangen.

Wie schon in Abschnitt~\ref{sec:doch} erwähnt, kann \textit{doch} nicht rein bestätigend verwendet werden. Assertiert B p, kann A zum Zwecke der Einigung auch nicht \textit{doch(p)} äußern. Bei den Direktiven bedeutet dies, dass nicht klar sein darf, dass von der Realisierung von !p sowieso auszugehen ist. Wenngleich ich gegen einen stets vorliegenden Widerspruch argumentiere, darf die Entscheidung zugunsten der Proposition in der \textit{doch}-Äußerung nicht schon gefallen sein bzw. dürfen nicht alle Anzeichen so stehen, dass mit ihrer Annahme/Realisierung sowieso zu rechnen ist.

Belege, die diese Situation (eine zusätzliche Fraglichkeit im \textit{auch}-Kontext 2) widerspiegeln, sind sehr schwer aufzufinden. (\ref{1163}) eignet sich für die Illustration.

\begin{exe}
	\ex\label{1163} 
	\scriptsize
	\textit{Frei(n)} sind Kinder, \glqq die nicht fremden\grqq{} (sich gegenüber Fremden ablehnend und unfreundlich benehmen)$^{3}$ \textit{sind frei mit enand! 	Bis au(ch) frei und schrei nüd eisig} (\textbf{sei \underline{doch auch} brav} und schrei nicht so) sagt die Mutter zu ihrem laut schreienden Kinde.
	\newline
	\hbox{}\hfill\hbox{(Zeitschrift für vergleichende Sprachforschung auf dem Gebie-}
	\newline
	\hbox{}\hfill\hbox{te der Indogermanischen Sprachen: Ergänzungshefte, Aus-}
	\newline
	\hbox{}\hfill\hbox{gaben 15–19, Vandenhoeck and Ruprecht, 1957)}
\end{exe}
Hier weiß man, dass die Äußerung in einer Situation gemacht wird, in der das Kind schreit. Es ist im unmittelbaren Kontext vor bzw. während der Äußerung nicht brav, so dass fraglich ist, ob es dies in Zukunft ist. Auf diese Frage kann \textit{doch} Bezug nehmen. Zusätzlich ist klar, dass es in dieser Situation brav sein soll (weil Kinder immer brav sein sollen). Die Kontextbedingung für \textit{auch} ist somit ebenfalls gegeben.

Wenngleich Belege kaum zu finden sind, lassen sich derartige Kontexte leicht konstruieren (vgl. (\ref{1164})).

\begin{exe}
	\ex\label{1164} 
	Philipp schreibt eine Hochzeitskarte.\\
	Melanie ist unzufrieden: \textbf{Schreib \underline{doch auch} ordentlich!} Dieses Gekrakel kann keiner lesen.
\end{exe}
Die Sequenz \textit{doch auch} ist immer dann möglich, wenn Zweifel daran besteht, dass die angesprochene Person der Handlung nachkommt, die der Sprecher als klar ausgibt.

So ist auch in (\ref{1165}) aufgrund des aktuellen Verhaltens der Leute fraglich, ob sie den Arzt im nächsten Moment durchlassen werden, da sie es gerade noch nicht tun, obwohl unmissverständlich ist, dass sie einen Arzt durchlassen müssen.

\begin{exe}
	\ex\label{1165} 
	(Ein Arzt wühlt sich durch eine Menschenansammlung.)\\
	Lassen Sie mich \textbf{doch auch} durch. Ich bin Arzt!
\end{exe}
Wenn sich die Fraglichkeit der Realisierung motivieren lässt, kann \textit{doch auch} im \textit{auch}-Kontext 2 verwendet werden. Die Interpretation ist dann stets: Obwohl klar ist, dass p zu tun ist, gibt es Anzeichen dafür, dass es fraglich ist, ob p realisiert wird.
						  
\subsubsection{Gegen ein Skopusverhältnis}
Ich denke, dass die Interpretation der \textit{doch auch}-Direktive in diesen Beispielen zeigt, dass die Partikeln nicht Skopus \is{Skopus} übereinander nehmen: Für die Vorgangskontextzustände ist nachzuweisen, dass die Anforderungen beider Einzelpartikeln vorliegen und auch vorliegen müssen: Es ist fraglich, ob p oder $\neg$p realisiert wird, d.h. diese Disjunktion liegt auf dem Tisch. Zusätzlich gilt, dass es nicht ausreicht, dass dieses Verhältnis allein aus dem Grund eintritt, weil die Realisierung der Handlung, zu der aufgefordert wird, noch nicht erfolgt ist. Dieses Verhältnis (das beispielsweise vorliegt, wenn \textit{auch} in Kontext 2 verwendet wird oder auch wenn der Angesprochene der Realisierung zugestimmt hat) reicht nicht aus, um die Realisierung von p als fraglich auszuzeichnen. Zu\-sätzlich ist die Inferenzrelation \is{Inferenzrelation} q $>$ !p Teil des cgs und q gilt im Diskurs. Der \textit{doch auch}-Direktiv thematisiert deshalb das Thema auf dem Tisch, indem er dazu anhält, eine der zur Diskussion stehenden Handlungen zu erfüllen, und er drückt aus, dass die Aufforderung zu dieser Handlung für den Adressaten ableitbar ist.

Angenommen, man ginge davon aus, dass \textit{doch} über \textit{auch} Skopus nimmt, läge vor einem angemessenen \textit{doch auch}-Direktiv die Kontextsituation in (\ref{1166}) vor.

\begin{exe}
\ex\label{1166} Kontextzustand vor einem \textit{doch auch}-Direktiv $[$doch(auch(p))$]$\\[-0.6em]
\begin{tabular}[t]{|C{6em}|C{12em}|C{6em}|}
\hline
$\textrm{DC}_{\textrm{A}}$ & Tisch &  $\textrm{DC}_{\textrm{B}}$ \tabularnewline
\hline
{} & (cg = $\lbrace$q $>$ !p$\rbrace$ \& q $\in$ $\textrm{DC}_{\textrm{B}}$) $\vee$ $\neg$(cg = $\lbrace$q $>$ !p$\rbrace$ \& q $\in$ $\textrm{DC}_{\textrm{B}}$) & {} \tabularnewline
\cline{1-1}\cline{3-3}
$\textrm{TDL}_{\textrm{A}}$ & {} & $\textrm{TDL}_{\textrm{B}}$  \tabularnewline
\cline{1-1}\cline{3-3}
{} & {} & {}  \tabularnewline
\hline
\multicolumn{3}{|l|}{cg s$_{1}$} \tabularnewline
\hline
\end{tabular}
\end{exe}					                       
Da das p auf dem Tisch das realisierte !p ist, würde das Verhältnis in (\ref{1166}) bedeuten, dass die Frage im Raum steht, ob der Adressat realisiert, dass cg und DC$_{\textrm{B}}$ so aussehen oder ob er dies nicht tut. Auf der Basis der oben durchgespielten Belege scheint mir dies nicht der Kontext zu sein, auf den ein \textit{doch auch}-Direktiv reagiert. Dazu kommt, dass ich diese Interpretation auch für ziemlich abwegig halte, da sie voraussetzt, dass der Adressat auf die Herbeiführung dieser Füllung der Komponenten Einfluss nehmen kann. Übertragen auf die Beispiele von oben bedeutet dies, dass zur Diskussion steht, ob der Adressat realisiert, dass a) im cg enthalten ist, dass wenn der Angesprochene Staatsmann werden will, er sich wie ein Staatsmann zu benehmen hat, und b) dass der Angesprochene vertritt, dass er Staatsmann werden will oder ob er dies nicht realisiert. Führen die Angesprochenen herbei, dass es Teil des cg wird, dass wenn ein Arzt an einen Unfallort kommt, sie normalerweise Platz machen, und dass sie davon ausgehen, dass der Arzt an den Unfallort kommt oder tun sie dies nicht? 							

Ich denke, in den Diskurssituationen, in denen \textit{doch auch}-Direktive getätigt werden, ist eindeutig, dass unklar ist, ob p realisiert wird, und nicht, ob der Adressat den einen oder anderen cg-Zustand bewirkt. Das Enthaltensein von q in DC$_{\textrm{B}}$ liegt zudem jeweils vor und muss nicht mehr herbeigeführt werden.

Unter dem umgekehrten Skopusverhältnis, in dem \textit{auch} Skopus über \textit{doch} nimmt, ergibt sich der Kontextzustand in (\ref{1167}).

\begin{exe}
\ex\label{1167} Kontextzustand vor einem \textit{doch auch}-Direktiv $[$auch(doch(p))$]$\\[-0.6em]
\begin{tabular}[t]{|C{6em}|C{12em}|C{6em}|}
\hline
$\textrm{DC}_{\textrm{A}}$ & Tisch &  $\textrm{DC}_{\textrm{B}}$ \tabularnewline
\hline
{} & {} & q \tabularnewline
\cline{1-1}\cline{3-3}
$\textrm{TDL}_{\textrm{A}}$ & {} & $\textrm{TDL}_{\textrm{B}}$  \tabularnewline
\cline{1-1}\cline{3-3}
{} & {} & {}  \tabularnewline
\hline
\multicolumn{3}{|l|}{cg s$_{1}$ = $\lbrace$q $>$ !((p $\vee$ $\neg$p) $\in$ T)$\rbrace$} \tabularnewline
\hline
\end{tabular}
\end{exe}
Im cg wäre folglich enthalten, dass normalerweise aus der mindestens von B vertretenen Bedingung q folgt, dass der Adressat bewirken soll, dass p $\vee$ $\neg$p auf dem Tisch liegt. Aus q folgt somit, dass vom Adressaten realisiert werden soll, dass offen ist, ob er p oder $\neg$p realisiert.

Ich halte diese Interpretation für recht unangebracht. In Kontext 1 führt diese Interpretation generell zu einer redundanten Situation: Da q entspricht, dass !p ein Element der TDL von B ist, steht p $\vee$ $\neg$p sowieso zur Diskussion. Die aus q folgende Aufforderung, p zur Diskussion zu machen, ist völlig überflüssig. Auch in Kontext 2 und 3 scheint diese Auslegung wenig überzeugend: Im cg wären z.B. die Relationen enthalten: Wenn der Adressat Staatsmann werden will, soll er realisieren, dass zur Diskussion steht, ob er sich so benehmen wird. Wenn der Arzt an einer Unfallstelle ankommt, soll der Hörer dafür sorgen, dass zur Diskussion steht, ob er Platz machen soll oder nicht.

Ich gehe somit erneut davon aus, dass \textit{doch} und \textit{auch} bei ihrem kombinierten Auftreten gleichen Skopus nehmen. Beide Partikeln beziehen sich nacheinander auf die gleiche Proposition, so dass sich die Bedeutung der Sequenz additiv ergibt. \textit{Doch} verweist auf die saliente Frage, ob der Adressat p realisieren wird, und die Äußerung beeinflusst diese zugunsten von p, indem der Sprecher ihm aufträgt, p zu bewirken. \textit{Auch} drückt zusätzlich aus, dass die Handlungsaufforderung aus der Situation/Vorgängeräußerung (q gilt mindestens für den späteren Adressaten des Direktivs) ableitbar ist, und deshalb klar sein sollte. 

\subsubsection{Die Erklärung der (un)markierten Abfolge}
Da ich als reine Partikelbedeutung in den Direktiven die gleiche Bedeutung ansetze wie in den Assertionen und sich die Unterschiede dadurch ergeben, dass man es mit einer anderen Äußerungsart zu tun hat, möchte ich auch eine sehr ähnliche Erklärung für die präferierte Reihung \textit{doch auch} anführen, die in den Direktiven gleichermaßen wie in den Assertionen gilt. Im Falle der Assertionen habe ich gesagt, dass das \textit{doch} dem \textit{auch} vorangeht, weil gespiegelt wird, dass es ein übergeordnetes Diskursziel ist, das zur Debatte stehende Thema zu adressieren. Diesem Bedeutungsbeitrag ist eine qualitative Bewertung, wie hier, dass es sich um die Begründung der Vorgängeräußerung handelt, nachgeordnet. Vor dem Hintergrund der Überlegung, dass die Anreicherung des cg das Hauptziel von Kommunikation ist, halte ich es für natürlicher, auszudrücken \glq Die vertretene Proposition adressiert das aktuelle Thema und stellt darüber hinaus eine Begründung für einen anderen Sachverhalt dar.\grq {} als diese beiden diskursiven Beiträge in die entgegengesetzte Reihung zu bringen: \glq Was mit dieser Äußerung mitgeteilt wird, ist eine Begründung für den Vorgängerbeitrag und es adressiert ebenfalls die aktuelle Diskursfrage.\grq {}

Aus dem einfachen Grund, dass Direktive auf die Realisierung von Sachverhalten abheben (und nicht auf (geteilte) Annahmen), die von der Erfüllung durch den Adressaten abhängig ist, ist eine direkte Übertragung meiner Auslegung der Situation in Assertionen nicht möglich. 

Der Sprecher des Direktivs leistet sicherlich nicht auf die gleiche direkte Art wie in Assertionen seinen Beitrag, das Thema zu lösen. Dies kann in Reaktion auf Direktive gerade nur durch den Angesprochenen erfolgen. Im Rahmen der Möglichkeiten des Sprechers, p realisiert zu sehen und den Kontext in diesem Sinne hinsichtlich p aufzulösen, lässt sich aber durchaus für seinen Beitrag zur Entscheidung des Themas argumentieren. 

Mit \textit{doch} adressiert er das Thema/die im Raum stehende Frage, m.a.W. das, was den Diskurs aktuell bewegt und für dessen Lösung der Sprecher sich einsetzt. Mit \textit{auch} nimmt der Sprecher, wie in Assertionen, eine Wertung vor, indem er sagt, dass diese Aufforderung ableitbar und in diesem Sinne auch erwartbar ist. Bei beiden Einstufungen, die \textit{auch} vornehmen kann, handelt es sich um qualitative Bewertungen. Da \textit{auch} in Direktiven die Einspruchsmöglichkeiten beschränkt, beeinflusst es die Realisierung von p in größerem Ausmaß zugunsten von p als \textit{doch}. Die Akzeptanz von !p, die durch \textit{auch} evoziert wird, bringt p seiner Rea\-lisierung näher als der Verweis von \textit{doch} auf das offene Thema, dem in dieser Hinsicht kein Einfluss zugeschrieben werden kann. Die Verhältnisse weichen aber ab von denen, die im Falle der Sequenz \textit{ja doch} vorliegen. Diese Abfolge habe ich in Kapitel~\ref{chapter:jud}, Abschnitt~\ref{sec:markiert} auf die Art erklärt, dass \textit{ja} präferiert vorangeht, weil es (im Gegensatz zu \textit{doch}) unmittelbar cg herstellen kann bzw. auf cg-Zugehörigkeit verweisen kann. Dieser Beitrag entspricht genau dem, was man sich von Assertionen erwünscht. \textit{Auch} kann in den Direktiven aber nicht analog bewirken, dass p wirklich realisiert wird, so dass p cg werden würde.

Ich halte die Abfolge der Beiträge, erst zu sagen, dass die Aufforderung ableitbar ist, und anschließend auszudrücken, dass sie sich auf das im Raum stehende Thema bezieht, für unnatürlich und weniger der Vorstellung von Diskursabläufen entsprechend, als zu vermitteln, dass mit dem Beitrag das aktuelle Thema adressiert wird, und anschließend der Handlungsaufforderung die weitere Qualität zuzu\-schreiben, dass sie ableitbar, klar bzw. erwartbar ist. Wie sich p $\vee$ $\neg$p auf dem Tisch auflöst, ist dann von der Handlung des Hörers abhängig. Die Beeinflussung der Realisierung betreffend ist die Überzeugung des Adressaten von der vom Sprecher gewünschten Handlung, die die Frage auflöst, das, was der Sprecher tun kann. 

\subsubsection{Ausblick auf weitere Satzmodi/Äußerungstypen}
Meine Modellierung des Bedeutungsbeitrags der beiden Einzelpartikeln und der MP-Kombination in Assertionen \is{Assertion} und Direktiven \is{Direktiv} ist im Prinzip die gleiche. Unterschiede stellen sich aufgrund der verschiedenen Äußerungstypen ein. Abschnitt~\ref{sec:distributionda} hat gezeigt, dass die Kombinationsmöglichkeiten von \textit{doch} und \textit{auch} (im Vergleich zu anderen Partikelkombinationen) recht weit sind. Ein weiterer Satzkontext, in dem das kombinierte Vorkommen ebenfalls möglich ist, ist der \is{Exklamativsatz} Exklamativsatz. \textit{Dass}- und w-Exklamativsätze \is{dass-Exklamativsatz} \is{w-Exklamativsatz} erlauben die Kombination (und entsprechend das isolierte Auftreten) in jedem Fall. Für meine Fragestellung wäre es zweifellos interessant, zu schauen, inwiefern die angenommene Modellierung auch auf diese Äußerungstypen zutrifft. Diese Untersuchung kann ich an dieser Stelle nur als Desiderat ausmachen, weil zu viele Unbekannte (s.u.) vorliegen, um dieser Frage nachzugehen. Einige Voraussetzungen müssten zunächst geschaffen werden. 

Ich habe schon an verschiedenen Stellen darauf hingewiesen, dass Assertionen und das Auftreten von MPn in Assertionen am besten bearbeitet worden sind. Wie die Darstellung gezeigt hat, stellen sich für die Behandlung von Direktiven neue Fragen. Dies setzt sich bei den Exklamativsätzen weiter fort. Beispiels\-weise ist zu klären, wie sich Exklamativsätze im Diskursmodell erfassen lassen. Einen vielversprechenden Beitrag, mit dem man in diese Frage einsteigen könn\-te, leistet hier \citet{Chernilovskaya2014}, die w-Exklamative (mit einem Ausblick auf andere Exklamativsätze $[$\citeyear[131]{Chernilovskaya2014} $]$) in das Modell aus \citet{Farkas2010} integriert. Andere Autoren (\citealt{CastroviejoMiro2008}) vertreten, dass ein einheitliches Kontextwechselpotential für die verschiedenen Exklamativsätze gar nicht formuliert werden kann. Für die deutschen Beispiele müsste man dann auch klären, ob sich die V2-Variante von der VL-Variante unterscheidet. Chernilovskaya untersucht englische und niederländische w-Exklamative. Eine Beschreibung der Diskurssemantik der verschiedenen Exklamativsätze im Deutschen steht noch aus. Es gibt zahlreiche Arbeiten zu Exklamativsätzen, auf die man hier bauen kann (z.B. \citealt{Roncador1977}, \citealt{Zaefferer1983}, \citealt{Rosengren1992}, \citealt{Avis2001}), wobei natürlich insbesondere eine Untersuchung der Pragmatik der Sätze vonnöten ist, im Sinne ihrer Diskursfunktion, von Wissensverteilungen oder Annahmenzuweisungen. 

Neben der diskursstrukturellen Erfassung des Äußerungstyps steht auch die Untersuchung des Auftretens von Partikeln in diesem Satzkontext noch aus. Die Arbeit mit Belegen ist hier fast aussichtslos, da in Korpusdaten Exklamativsätze sehr wenig frequent sind. Die Korpusuntersuchung von \citet{Naef1996} weist darüber hinaus nach, dass nur in einem geringen Anteil von w-Exklamativsätzen \is{w-Exklamativsatz} überhaupt MPn vertreten sind (5\%). Belege anzuführen und zu analysieren scheint folg\-lich nicht vielversprechend, um Unterschiede zwischen \textit{auch}- und \textit{doch}-Exkla\-mativsätzen aufzudecken und um zu entscheiden, wie die Kombination interpretiert wird. Erschwerend kommt hinzu, dass der Partikelbeitrag umso unklarer ist, je randständiger die Äußerungstypen sind. \citet[637]{Thurmair2013} schreibt hierzu z.B., dass die Bedeutungsunterschiede zwischen den verschiedenen MPn in w-Exklamativsätzen gering sind. 

Diese kurzen Ausführungen geben Einblick in die Aspekte, die in Isolation zu untersuchen sind, bevor eine Beschäftigung mit den Partikelkombinationen möglich wird. Ich möchte eine solche Untersuchung an dieser Stelle nicht anstellen. (Meine) weitere Forschung muss zeigen, inwiefern sich auch die Verwendungen von \textit{doch}, \textit{auch} sowie ihren Kombinationen in diesem Kontext in meine Modellierung und Analyse der Abfolgen integrieren lassen.

Der Vollständigkeit halber sei auch eine noch randständigere Verwendungsweise der beiden MPn erwähnt: Es gibt auch einen Gebrauch in w-Interrogativsätzen, \is{w-Interrogatuvsatz} unter dem die beiden Partikeln sich kombinieren lassen. (\ref{1168}) illustriert diesen Typ, der in Betrachtungen von \textit{auch} nahezu gar nicht erwähnt wird (anders als w-Interrogativsätze der Art in (\ref{1169}), aus denen \textit{doch} ausgeschlossen ist) (vgl. meine Erklärung zum Ausschluss der Kombination hier in Abschnitt~\ref{sec:distributionda}).

\begin{exe}
	\ex\label{1168} 
	Wie heißt \textbf{doch}/\textbf{auch}/\textbf{doch auch}/\#\textbf{auch doch} Gretes Kaninchen? Ich hab's vergessen.
\end{exe}
\vspace{-0.6cm}	
\begin{exe}
	\ex\label{1169} 
	Warum gehst du *\textbf{doch}/\textbf{auch}/*\textbf{doch auch} im Regen vor die Tür?
\end{exe}	
Bei den relevanten w-Interrogativsätzen wie in (\ref{1168}) handelt es sich um Fragen, deren Antwort der Sprecher einmal gewusst hat und an die er sich aktuell nicht mehr erinnern kann. In den Darstellungen zum Gebrauch von \textit{doch} werden sie in der Regel angeführt.

Auch hier steht die Aufgabe noch aus, den Partikelbeitrag zu bestimmen. Es wäre zwar wünschenswert, wenn sich ein bedeutungsminimalistischer Ansatz \is{Bedeutungsminimalismus/-maximalismus} durch alle Äußerungstypen hindurch verfolgen ließe, der Nachweis müsste aber erst einmal erbracht werden, dass auch hier die Kontextzustände vorliegen, die ich für das Einzelauftreten von \textit{doch} und \textit{auch} angenommen habe. An diese Entscheidung schließt sich die Überlegung an, ob die Erklärung der präferierten Abfolge die gleiche sein kann. 	

Weitere Äußerungstypen betrachte ich folglich nicht. Im nächsten Abschnitt gehe ich abschließend zu meiner Untersuchung von \textit{doch} und \textit{auch} der Frage nach, ob die umgekehrte Abfolge bei dieser Kombination kategorisch auszuschlie\-ßen ist oder ob sich nicht auch hier Fälle nachweisen lassen, die zeigen, dass diese nicht völlig abzulehnen ist. Ich werde für die letztere Ansicht argumentieren.

\section{Die Distribution von \textit{auch doch}}
\label{sec:distributionad}
Die Frequenzen, die ich in Abschnitt~\ref{sec:präferenz} angegeben habe, zeigen, dass selten auch die Sequenz \textit{auch doch} zu finden ist. In den zwei Korpora, für die ich die Anzahl der \textit{doch auch}-Treffer (näherungsweise) bestimmt habe, liegen ohne Zweifel sehr wenige Belege für diese Ordnung vor (vier in DeReKo, zwei in DGD2 vs. ca. 8664 bzw. 60 \textit{doch auch}-Bespiele). Aufgrund der großen Anzahl von \textit{doch auch}-Treffern habe ich für DECOW keine Häufigkeiten gegenüberstellen können. Gerade in dieser Datenmenge findet sich aber eine größere Menge von \textit{auch doch}-Belegen (s.u.).

\subsection{Der Ausschluss der \glq Dubletten\grq {}}
Ich gehe deshalb auch im Falle dieser Kombination davon aus, dass aus den seltenen Treffern in DeReKo und DGD2 nicht auf die Non-Existenz und Ungrammatikalität dieser Reihung zu schließen ist. Bei der Beschäftigung mit MPn hat man stets mit dem Umstand umzugehen, dass es sich bei einer betrachteten \glq MP\grq {} um eine ihrer \glq Dubletten\grq {}  handeln könnte. Vor diesem Hintergrund ist die Untersuchung des kombinierten Auftretens der Partikeln \textit{doch} und \textit{auch} deshalb besonders problematisch, da prinzipiell beide nicht in ihrer MP-Verwendung vorliegen können.

Die konkurrierenden Formen bei \textit{auch} sind das \is{Adverb} Adverb (meist mit engem Skopus, ggf. betont) (vgl. (\ref{1170}), (\ref{1171})) und das Konjunktionaladverb (unbetont, weiter Skopus) (vgl. (\ref{1172})). 

\begin{exe}
	\ex\label{1170} 
	Das Wasser des Sees war \textbf{\textit{auch}} dem abgehärtetesten Schwimmer zu kalt.\\
	$[$= \textit{sogar}, \textit{selbst}, \textit{ebenfalls}$]$
	\hfill\hbox{\citet[92]{Helbig1990}}
\end{exe}

\begin{exe}
	\ex\label{1171} 
	Der andere Lehrer hat \textbf{\textit{AUCH}} recht.\\
	$[$= \textit{ebenfalls}, \textit{gleichfalls}$]$
	\hfill\hbox{\citet[22]{Mueller2014b}}
\end{exe}							
			
\begin{exe}
	\ex\label{1172} 
	Peter macht Weihnachtseinkäufe. Beim Metzger besorgt er die Gans, beim Konditor Marzipan. 
		\begin{xlist}
			\ex\label{1172a} \textbf{\textit{Auch}} kauft er einige neue Tannenbaumkugeln im Baumarkt.
			\ex\label{1172b} Er kauft \textbf{\textit{auch}} einige neue Tannenbaumkugeln im Baumarkt.
			$[$= \textit{außerdem}, \textit{zusätzlich}$]$
		\hfill\hbox{\citet[22]{Mueller2014b}}	
		\end{xlist}		
\end{exe}
\textit{Doch} kann betontes Adverb sein (vgl. (\ref{1173}))).

\begin{exe}
	\ex\label{1173} 
	Peter wollte erst an Silvester zu Hause bleiben und ist dann \textbf{\textit{DOCH}} auf die Party gekommen. $[$= \textit{dennoch}, \textit{trotzdem}$]$
\end{exe}	
\citet[84]{Kwon2005} zieht auch in Betracht, dass \textit{doch} im Mittelfeld als Konjunktio\-naladverb \is{Konjunktionaladverb} vorkommen kann (vgl. (\ref{1174}) und (\ref{1175})).
	
\begin{exe}
	\ex\label{1174} 
	\scriptsize
	Denn die Semester habe ich erst später gemacht. Ich habe zunächst eine Stelle in Bochum an der Stadtverwaltung angetreten. Das war \textbf{doch} sehr 		schwierig, damals unterzukommen überhaupt. 
	\newline
	\hbox{}\hfill\hbox{(PFE/BRD.cf008)}
\end{exe}	
\vspace{-0.5cm}			
\begin{exe}
	\ex\label{1175} 
	\textbf{Doch} war das sehr schwierig... $[$= \textit{aber}, \textit{jedoch}$]$ 
\end{exe}	
Er schreibt allerdings, dass unklar ist, ob \textit{doch} im Mittelfeld als Konjunktionaladverb \is{Konjunktionaladverb} auftreten kann. In (\ref{1174}) hält er den MP-Gebrauch für wahrscheinlicher.

M.E. kann man die Konjunktionaladverbien kaum ausmachen. Mein Hauptau\-genmerk liegt auf dem Schluss der Adverbien.

Man hat bei den \textit{doch auch}-Treffern insbesondere damit zu tun, die Adverbverwendung von \textit{auch} herauszufiltern (vgl. (\ref{1176}) und (\ref{1177})) – was sich bei unbetontem Auftreten und weitem Skopus als schwierig erweist.

\begin{exe}
	\ex\label{1176} 
	\scriptsize
	{so angenehm die Börsengeschäfte für die Banken auch sind, \textbf{so lauern hier \underline{doch auch} mancherlei Gefahren}.
	\hfill\hbox{(H87/BM5.11485 Mannheimer Morgen, 25.04.1987, S. 05; Die fetten Jahre)}}\\
	$[$Adverb/MP \textit{doch}, Adverb \textit{auch}$]$
\end{exe}

\begin{exe}
	\ex\label{1177} 
	\scriptsize
	{Aber, liebe Leute vom ORF, \textbf{da hätte \underline{doch auch} Kärnten einiges zu bieten}: Die blaue Küste von Reifnizza, die Nocky Mountains, die 		Gegend zwischen Saualm und Klein St. Paul, kurz Sao Paolo genannt. 
	\hfill\hbox{(K97/JUN.42849 Kleine Zeitung, 08.06.1997, Ressort: Lokal)}}\\	
	$[$MP \textit{doch}, Adverb \textit{auch}$]$
\end{exe}	
In den \textit{auch doch}-Belegen gilt es andererseits insbesondere, das betonte \textit{doch} zu erkennen. 

Betrachtet man \textit{auch doch}-Treffer in Korpora, hat man in den meisten Fällen den Eindruck, dass nicht beide Bestandteile als MPn vorkommen. Zwei Beispiele finden sich in (\ref{1178}) und (\ref{1179}). 

\begin{exe}
	\ex\label{1178} 
	\scriptsize
	{Als der Empfang vorüber war, hatte er allerdings einen Job, mit dem er selbst nicht gerechnet hatte. Trippe hatte mit seiner ungeheuren 					Überzeugungskraft Lindbergh dazu gebracht, \glqq techni\-scher Berater\grqq{} bei Pan Am zu werden. \textbf{Mit derselben Überredungskunst machte Trippe 		seine Betty schließlich \underline{auch doch} noch zur Ehefrau.}
	\hfill\hbox{(Z04/405.04570 Die Zeit (Online-Ausgabe), 27.05.2004)}}\\	
	$[$Konjunktionaladverb \textit{auch}, Adverb \textit{doch}$]$
\end{exe}

\begin{exe}
	\ex\label{1179} 
	\scriptsize
	{Frage: kann man nicht alle nicht auf den Artikel bezogenen Beiträge raus nehmen?--G.E.M.A. 23:31, 16. Jan. 2010 (CET)
	Nein. (\textbf{Das gäbe schlimmstenfalls dann \underline{auch doch} nur weitere Bewertungs\-scherereien nach dem Motto \glqq Was ist artikelbezogen			\grqq{}}). 
	\newline
	\hbox{}\hfill\hbox{(WDD11/L43.00760: Diskussion:Lectorium Rosicrucianum/Archiv/2010/1. Teilarchiv)}}\\	
	$[$MP \textit{auch}, Adverb \textit{doch}$]$
\end{exe}
An den Kalkulationen in Abschnitt~\ref{sec:präferenz} sieht man, dass auch bei der unmarkierten Abfolge eine große Anzahl von Treffern nicht der MP-Abfolge \textit{doch auch} entspricht. Von 500 Zufallstreffern bleiben 59 übrig. Ich habe oben schon angemerkt, dass die Untersuchung dieser Kombination schwierig ist, weil prinzipiell bei beiden Bestandteilen ein Non-MP-Pendant vorliegen kann. Die obigen Beispiele zeigen darüber hinaus, dass die auszuschließenden Formen subtiler vom MP-Gebrauch abweichen als in anderen Fällen und deshalb nicht leicht auszumachen und von diesem abzugrenzen sind. 	

\subsection{Die zwei \textit{auch doch}-Kontexte}
Trotz der Situation, dass man bei sehr vielen Treffern den Eindruck hat, dass es sich nicht um zwei MP-Vorkommensweisen handelt, finden sich in DECOW2014 40 Fundstellen für \textit{auch doch}, für die meiner Meinung nach gilt, dass man beide Partikeln der Kombination gut als MPn lesen kann. (\ref{1180}) bis (\ref{1188}) zeigt einige Beispiele.

\begin{exe}
	\ex\label{1180} 
	\scriptsize
	Ja, das auch. Aber auch hier setzt die Politik den Rahmen, ich erinnere mich an ein Schreiben des Ministeriums, in dem uns in der üblichen sanften Art 		nahegelegt wurde, die Schüler, denen Gymnasialeignung attestiert wurde, \textbf{\underline{auch doch} gefälligst bis zum (erfolgreich bestandenen) 			Abitur zu führen.}
	\hfill\hbox{(DECOW2014)}	
	\newline
	\hbox{}\hfill\hbox{(http://www.herr-rau.de/wordpress/2012/03/}
	\newline
	\hbox{}\hfill\hbox{wie-geht-kacken-und-das-achteinhalbjaehrige-gymnasium.htm)}
\end{exe}

\begin{exe}
	\ex\label{1181} 
	\scriptsize
	Und zum Thema Hechte in der Nacht:\\
	Der Hecht ist ein Augentier, welches Nachts die Beute erschnüffelt. Sprich nen Köderfisch richt der Kollege trotzdem. Nen Jerk wird er schwerlich 			erschnüffeln können. \textbf{Zumal Hechte denn \underline{auch doch} zu 90\% Tagaktiv bzw. Dämmerungsaktiv sind...}Ich hab in all den Jahren 4-5 Hechte 	nach Einbruch der Dämmerung gefangen aber unendlich viele mehr Tagsüber oder bei Dämmerung...
	\newline
	\hbox{}\hfill\hbox{(DECOW2014)}	
	\newline
	\hbox{}\hfill\hbox{(http://barsch-alarm.de/Forums/viewtopic/t=23812/start=0.html)}
\end{exe}						  
						
\begin{exe}
	\ex\label{1182} 
	\scriptsize
	@Titus : Wenn ich mich richtig entsinne hast Du das hier mal als zusätzliche Funktion gewünscht (siehe Kommentar Nr. 42 weiter oben ) Ich bin aber noch 	nicht dazu gekommen eine solche Funktion einzubauen, \textbf{da es aufgrund der Wahlmöglichkeit die man dazu haben sollte \underline{auch doch} kein 		kleiner Aufwand ist.} 
	\hfill\hbox{(DECOW2014)}	
	\newline
	\hbox{}\hfill\hbox{(http://www.crazytoast.de/plugin-wordpress-blogroll-widget-with-rss-feeds.html)}
\end{exe}	

\begin{exe}
	\ex\label{1183} 
	\scriptsize
	Ich finde ohne Sattel reiten prima, \textbf{weil man \underline{auch doch} viel genauer merkt, was unter einem los ist.} 
	\hfill\hbox{(DECOW2014)}	
	\newline
	\hbox{}\hfill\hbox{(http://www.wege-zum-pferd.de/forum/archive/index.php?t-5461.html)}
\end{exe}	

\begin{exe}
	\ex\label{1184} 
	\scriptsize
	um mich vllt klarer auszudrücken, ich wollte nicht sagen, dass veigar op ist, aber seine verwendete kombo war sehr stark. und ich weiss, es gibt für 		alles einen konter oder zumindest sollte es. aber in normalen spielen sehe ich nicht, was ich für gegner habe und kann mich erst im spiel 					dem\-entsprechend anpassen. dennoch bin ich der meinung, dass es op kombos und chars gibt. ist in einem solchen spiel auch nicht vermeidbar.\\	
	\noindent
	\textbf{Das ist dann \underline{ja auch doch} deine Meinung} ob sie stimmt und ob du es mit deinem spielvermögen beurteilen kannst ist was anderes, 		nicht böse gemeint.	 
	\hfill\hbox{(DECOW2014)}	
	\newline
	\hbox{}\hfill\hbox{(http://www.computerbase.de/forum/archive/index.php/t-697528-p-14.html)}
\end{exe}

\begin{exe}
	\ex\label{1185} 
	\scriptsize
	Ich fragte: \glqq Kann ich denn Schuh mal anprobieren, da ich ihn mir woanders hollen will\grqq{}\\

	Da sagte die Frau \glqq Neeee,so gehts nicht junger Mann\grqq{}\\
	
	Manuel\\
	21.06.2009, 21:11\\
	Ja ... naja \textbf{das is ja \underline{auch doch} ziemlich dreist.} xD			
	\hfill\hbox{(DECOW2014)}	
	\newline
	\hbox{}\hfill\hbox{(http://forum.torwart.de/de/archive/index.php/t-62037-p-4.html)}
\end{exe}

\begin{exe}
	\ex\label{1186} 
	\scriptsize
	Tut mir leid für Dich, dass es schon wieder Probleme gibt! Also Du hast mit Deinem echt die A.R.S.C.H.-Karte gezogen! Da ist meiner ja noch harmlos, 		obwohl ICH mich schon genug ärgere. Hast nicht schon überlegt, ob sie ihn Dir wandeln sollten? \textbf{Es gibt ja \underline{auch doch} einige im 			Board, bei denen alles funktioniert.}
	\hfill\hbox{(DECOW2014)}	
	\newline
	\hbox{}\hfill\hbox{(http://www.der206cc.de/forum/archive/index.php/t-2177.html)}
\end{exe}

\begin{exe}
	\ex\label{1187} 
	\scriptsize
	vielen dank für eure schnellen antworten. das BP – Bipolar, NP – Nonpolar ist hatte ich schon vermutet, aber offensichtlich das falsche daraus 				geschlossen. ;)\\
	\textbf{die beiden verschiedenen bezeichnungen sind aber \underline{auch doch} ein wenig irreführend.} (naja jetzt weis ich ja bescheid)\\
	gruß tim 		
	\hfill\hbox{(DECOW2014)}	
	\newline
	\hbox{}\hfill\hbox{(http://forum.musikding.de/vb/archive/index.php?t-11974.html)}
\end{exe}

\begin{exe}
	\ex\label{1188} 
	\scriptsize
	Find ich blöd dass du gehst, \textbf{weil du ja \underline{auch doch} sehr aktiv warst.}	
	\hfill\hbox{(DECOW2014)}	
	\newline
	\hbox{}\hfill\hbox{(http://www.websitepark.de/forum/archive/index.php/t-4754.html)}
\end{exe}						
Im Vergleich zum Auftreten der unmarkierten Abfolge \textit{doch auch} in diesem Kor\-pus sind dies natürlich sehr wenige Belege. Ich kann den genauen Wert nicht angeben. Es ist aber völlig klar und steht nicht zur Diskussion, dass die \textit{auch doch}-Treffer auch in diesen Daten deutlich unterrepräsentiert sind. Ich möchte betonen, dass meine Argumentation hinsichtlich der umgekehrten Abfolgen nicht derart verläuft, nachweisen zu wollen, dass die markierten Reihungen ähnlich gebräuchlich sind wie die unmarkierten. Es liegt zweifellos ein deutlicher Markiert\-heitsunterschied \is{Markiertheit} vor. Dennoch halte ich die in meiner Argumentation markierten Fälle nicht für ungrammatisch und non-existent. Ich gehe vielmehr davon aus, dass sie nicht gänzlich ausgeschlossen sind und sie mit einer gewissen Systema\-tik auftreten. Insbesondere aus dem letzten Grund möchte ich den Treffern den Status von Performanzfehlern absprechen. Ich bin der Meinung, dass in den erwähnten 40 Fällen, die durch (\ref{1180}) bis (\ref{1188}) illustriert werden, – anders als in den Beispielen in (\ref{1178}) und (\ref{1179}) – plausibel in beiden Fällen die MPn vorliegen können.

In DECOW findet man folglich derartige Daten mit einer gewissen Häufigkeit, die für meine Begriffe über die sporadischen Vorkommen in DGD2 und DeReKo hinausgehen. Der Frequenzunterschied zu der unmarkierten Abfolge bleibt natürlich sehr groß (weil auch in DECOW entsprechend viel mehr \textit{doch auch}-Treffer vorhanden sind). Der Vorteil dieser Daten ist allerdings, dass eine Menge vorliegt, die sich auf Muster untersuchen lässt.

Zwei Muster kristallisieren sich auch im Falle dieser Kombination heraus. Es gibt zwei Kontexte, die die Interpretation als MP für beide Elemente der Sequenz stützen. Hierbei handelt es sich zum einen um kausale Nebensätzen und zum anderen um die Dreierkombination \textit{ja auch doch}. Letztere wird auch im einzigen Hinweis auf die Abfolge \textit{auch doch}, den ich finden konnte (vgl. \citealt[254]{Hentschel1986}), erwähnt. 

Unter den 40 Treffern, denen ich die MP-Verwendung beider Elemente der Kombination zuschreibe, befinden sich 14 durch eine Konjunktion bzw. die Verb\-stellung (V1) ausgezeichnete kausale Nebensätze, 11 Kombinationen mit \textit{ja auch doch} sowie zwei Verbindungen aus diesen beiden Kontexten.

Es ist nicht so, dass das Auftreten dieser Kontexte stets mit der Interpretation von \textit{doch} und \textit{auch} als MP einhergeht (vgl. (\ref{1189}) und (\ref{1190})).

\begin{exe}
	\ex\label{1189} 
	\scriptsize
	\textbf{\textit{Da} mein Triebwagen jetzt \underline{auch doch} besser geworden ist}, als ich dachte, will ich mal auf der Börse nächstes WE schauen, 		ob ich vielleicht nen guten 4000 er günstig bekomme, ohne Achsen udn Kupplungen oder so.			
	\hfill\hbox{(DECOW2014)}	
	\newline
	\hbox{}\hfill\hbox{(http://alte-modellbahnen.xobor.de/t14781f2-Schrott-wird-flott-3.html)}
\end{exe}	

\begin{exe}
	\ex\label{1190} 
	\scriptsize
	Sollen sich doch die Leute die Köppe einrennen, wenn's ihnen Spass macht. Und wie's scheint, \textbf{steckt \textit{ja} \underline{auch doch} sehr viel 	mehr Strategie und Taktik dahinter}, als es auf den ersten Blick aussieht.			
	\newline
	\hbox{}\hfill\hbox{(DECOW2014)}	
	\newline
	\hbox{}\hfill\hbox{(http://www.comicforum.de/archive/index.php/t-89772.html)}
\end{exe}					 
Meiner Meinung nach sind dies aber Kontexte, die die Lesart beider Elemente als MPn stützen. Wenn die Partikel-Lesart vorliegt, scheint dies in den beiden Kontexten zu erfolgen.

Sucht man gezielt nach \textit{auch doch} in diesen beiden Kontexten, findet man mit mäßigem Aufwand weitere Treffer, die ich nicht für auffällig abweichend halte (vgl. (\ref{1191a}) bis (\ref{1194})).
	
\begin{exe}
	\ex\label{1191} 
	\scriptsize
	... früher war mein Blog hauptsächlich ein Naturfotografieblog. Heute ist er hauptsächlich ein People-fotografie-Blog, ich hab von den Fotomotiven her 		einmal ne 180 Grad Wendung gemacht :D Das liegt zunächst einmal daran, dass ich kaum noch Natur fotografiere, \textbf{\textit{weil} ich \textit{ja} 		\underline{auch doch} keine 36 Stunden Tage habe} (leider :D) und daher dann neben den Shootings für so etwas keine Zeit mehr bleibt.				
	\hfill\hbox{(Google-Suche 25.06.2015)}	
	\newline
	\hbox{}\hfill\hbox{(http://www.lichtreflexe-blog.de/2014\_10\_01\_archive.html)}
\end{exe}	
	
\begin{exe}
	\ex\label{1191} 
	\scriptsize
	Ich sehe es schon deutlich vor meinem dritten Auge: Alle einstigen Zonenklubs werden teilnehmen, das Gewinnerteam darf zwei Wochen Urlaub machen. In 		Nordkorea. Bikini, Sandburg, Folterkeller. Aus Chile wird eine Honeckermumie eingeführt, die den FDGB-Pokal überreicht. Die Spieler und Funktionäre 		können sich ein bisschen heldenhaft fühlen. Wie man hört, will die deutsche Wirtschaft Asiens letztem originalem Diktator Beine machen. 					\textbf{\textit{Weil} \underline{auch doch} dort unten alles besser werden soll.} 						
	\hfill\hbox{(Google-Suche 25.06.2015)}	
	\newline
	\hbox{}\hfill\hbox{(http://www.tagesspiegel.de/sport/willmanns-kolumne-dresdner-fans-wollen-de}
	\newline
	\hbox{}\hfill\hbox{n-fdgb-pokal-wieder-einfuehren/7601988-2.html)}	
\end{exe}								
						   
\begin{exe}
	\ex\label{1192} 
	\scriptsize
	Noch mehr aber ward ich allzeit dadurch beruhigt, wenn ich bedachte, wie Du am Kreuze den Vater in Dir für alle Deine Feinde um Vergebung batest; und 		da konnte ich denn den armen Judas trotz seines Selbstmordes nicht ausschließen. \textbf{Dazu war \textit{ja} \underline{auch doch} offenbar an dieser 		seiner letzten Tat nach der Schrift der in ihn fahrende Teufel schuld.} Daher also möchte ich wohl auch diesen Apostel, wenn schon nicht hier, so aber 		doch wenigstens irgendwo nicht im höchsten Grade unglücklich wissen.							
	\hfill\hbox{(Google-Suche 25.06.2015)}	
	\newline
	\hbox{}\hfill\hbox{(http://www.j-lorber.de/jl/gso2/gso2-007.htm)}
	\end{exe}						            
							           
\begin{exe}
	\ex\label{1193} 
	\scriptsize
	Ich denk mal eher, dass die 11 Punkte fuer Multiplayer apps sind ... Is ja auch logisch groesseres Display als iPhone. Mag sein dass es im Moment keine 		apps gibt die es nutzen aber wer weiß was die Zukunft bringt ;)\\
	— Coolix\\

	\textbf{Ich will neben meinen Finger aufm iPhone \textit{ja} \underline{auch doch} was erkennen} ... das ne Simple erklareung… \\
	Auch wenn das touchpad vom MacBook 11 Finger wie das Opas erkennt – es ist nur zum Mauszeiger steuern und nicht um was anzuschauen…\\
	— Gtc-michel89 								
	\hfill\hbox{(Google-Suche 25.06.2015)}	
	\newline
	\hbox{}\hfill\hbox{(http://www.iphone-ticker.de/multitouch-punkte-ipa}
	\newline
	\hbox{}\hfill\hbox{d-unterstutzt-11-iphone-nur-funf-10833/)}
\end{exe}								

\begin{exe}
	\ex\label{1194} 
	\scriptsize
	Inszeniert wurde sie von Kai Grehn, der sich durch eine Vielzahl verschiedener Kunstprojekte im Hörspielgenre einen Namen machen konnte. Auch \glqq Die 	Frau in den Dünen\grqq{} lässt er nicht ohne besondere Note erklingen. Das könnte einigen Hörern vielleicht etwas zu verkünstelt sein, aber es geht 		aufgrund des Inhaltes, \textbf{der \textit{ja} \underline{auch doch} eher auf einer psychologischen Ebene seinen \\ Schwerpunkt hat.}						
	\hfill\hbox{(Google-Suche 25.06.2015)}	
	\newline
	\hbox{}\hfill\hbox{(http://www.hoerspieltipps.net/archiv/diefrauindenduenen.html)}
\end{exe}					            
M.E. findet man die Abfolge in diesen beiden Kontexten zu einfach, um behaupten zu wollen, dass sie ungrammatisch ist und nicht existiert.

\subsection{Erklärung der markierten Abfolge}
Ich möchte weiter unten einen Vorschlag vorstellen, warum sich möglicherweise genau diese Kontexte eignen. Die Überlegungen erfolgen vor dem Hintergrund meiner Annahmen zu den umgekehrten Abfolgen in den Kapiteln~\ref{chapter:jud} und \ref{chapter:hue} und meinem Eindruck, dass MPn in Kombinationen abhängig vom Satztyp unterschiedliches Gewicht haben können. Es ist aber zweifellos so, dass in Verbindung mit der umgekehrten Reihung \textit{auch doch} noch einige empirische Aufgaben ausstehen, die ich zunächst adressieren möchte.

Insbesondere gilt es, festzustellen, ob es sich bei den beiden genannten Kontexten, die sich in den Daten herauskristallisieren, wirklich um genuine \textit{auch doch}-Umgebungen handelt. Es ist klar, dass die Verwendung von \textit{doch auch} prinzi\-piell weiter ist. Diese Reihung tritt auch gut in anderen Kontexten als kausalen Nebensätzen und größeren Kombinationen mit \textit{ja} auf. In Analogie zu meiner Untersuchung zu \textit{doch ja} müsste man aber festzustellen versuchen, ob Sprecher \textit{auch doch} tatsächlich innerhalb dieser Kontexte als besser und außerhalb dieser Umgebungen als schlechter erachten. Ob sich dies empirisch testen lässt und mit welcher Methode, muss (meine) weitere Forschung zeigen.

In Bezug auf die Korpusdaten wäre es wieder interessant, zu wissen, ob diese beiden Kontexte auch für \textit{doch auch} bzw. das Korpus an sich ein häufiger Auftre\-tenskontext sind. Es wäre zu überlegen, zum Zwecke dieser Erkenntnis die Treffer für die beiden Kontexte vor dem Hintergrund eines Erwartungswertes für die beiden Abfolgen gegenüberzustellen. Dieser Erwartungswert wäre allerdings äußerst aufwändig zu bestimmen. Selbst das Teilkorpus DECOW2014AX, mit dem ich zu anderen Fragestellungen dieser Arbeit exhaustive Suchen durchgeführt habe, gibt hier eine viel zu große Treffermenge aus, als dass man die Belege sortieren könnte. Wie ich schon verschiedentlich deutlich gemacht habe, ist dies aber insbesondere bei dieser Kombination absolut notwendig. Ein solches Vorgehen wäre folglich sehr mühsam, (davon abgesehen, dass das Korpus nur einzelne Sätze ausgibt und die Kontexte erst gesucht werden müssen). 

Trotz dieser Fragen, die es zu klären gilt, gehe ich (bis Gegenteiliges nachgewie\-sen ist) davon aus, dass diese beiden Kontexte eine Rolle spielen, wenn Sprecher die Abfolge \textit{auch doch} verwenden. Und ich möchte einige Überlegungen anstellen, warum die Umkehr der Abfolge genau in diesen Kontexten besser möglich zu sein scheint. Schon in Kapitel~\ref{chapter:jud} und \ref{chapter:hue} habe ich angenommen, dass MPn in Kombinationen nicht immer gleich gewichtet sind, in dem Sinne, dass der Satzkontext Einfluss auf das Gewicht einer Partikel nehmen kann. 

Die präferierte Abfolge \textit{doch auch} habe ich derart erklärt, dass diese Reihung das kommunikative Ziel, den cg zu erweitern und (dafür) die Themen auf dem Tisch zu entfernen, direkter abbildet, als die umgekehrte Abfolge. Die Partikel \textit{doch} adressiert das Thema auf dem Tisch, \textit{auch} führt eine Begründung und somit qualitative Bewertung eines anderen Sachverhalts an. Letzterer Diskursbeitrag darf vor dem Hintergrund des Diskursmodells aus \citet{Farkas2010} und dem dort angelegten Ziel/Zweck von Kommunikation (vgl. (\ref{1195})) als dem ersteren nachgeordnet angesehen werden.

\begin{exe}
	\ex\label{1195} Zwei fundamentale Antriebe für Gespräche 
		\begin{xlist}	
			\ex\label{1195a} Erweiterung des cg
			\ex\label{1195b} Herstellung eines stabilen Kontextzustands
			\newline
			\hbox{}\hfill\hbox {\citet[87]{Farkas2010}}
		\end{xlist}
\end{exe}
In Kapitel~\ref{chapter:jud}, Abschnitt~\ref{sec:unmarkiert} habe ich die präferierte Reihung \textit{ja doch} derart abgeleitet, dass diese Abfolge des Partikelbeitrags das Ziel einer Assertion im Diskurs direkter abbildet als die Abfolge \textit{doch ja}, weil \textit{ja} genau das bewirkt, was eine Assertion anstrebt: zum Teil des cg zu werden. Diesem Ziel kommt man mit der Abfolge \textit{ja doch} direkter nach, wenn man die Partikel, die dies zu leisten imstande ist, als erstes zur Applikation bringt und nicht erst darauf verweist, dass auch ein zur Debatte stehendes Thema angesprochen wird (\textit{doch}).  

Tritt nun zu \textit{auch doch} die Partikel \textit{ja} hinzu, könnte man argumentieren, dass die Adressierung des Themas entsprechend weniger relevant wird, da ohnehin sofort cg hergestellt wird. Die Folge ist, dass diese Partikel auch erst spät zur Anwendung gebracht werden kann und deshalb in dieser Dreierkombination am rechten Rand erscheint. Die Konstellation, aufgrund derer \textit{doch} in der Kombination \textit{doch auch} meiner Argumentation nach vorne steht, löst sich in diesem Kontext durch die Hinzunahme von \textit{ja} folglich auf und ermöglicht die späte Applikation von \textit{doch}.\footnote{Diese Erklärung setzt die Annahme voraus, dass sich die MPn in einer Dreierkombination (genauso wie ich es für eine Zweierkombination annehme) alle unter gleichem Skopus auf dieselbe Proposition beziehen.}

Ähnlich lässt sich auch für den kausalen Kontext argumentieren, dass der Aspekt der Adressierung des Themas hier in den Hintergrund rückt. 

Generell gehe ich davon aus, dass wenn Partikeln auftreten, ihr Diskursbeitrag im jeweiligen Satzkontext auch zur Anwendung kommt und beabsichtigt ist. Aus diesem Grund habe ich auch in meinen Ausführungen zu den V1- und \textit{Wo}-VL-Sätzen angenommen, dass ihre kausale Lesart keinen Einfluss auf die Abfolge der Partikeln nimmt. Die Adressierung des Themas liegt dort transparent vor und ist der qualitativen, kausalen Bewertung übergeordnet. Dennoch verwundert es nicht, wenn sich die Umkehr gerade im kausalen Kontext einstellen kann, d.h. dieser Satzkontext anfällig für die andere Reihung ist: Zweck eines Kausalsatzes ist es gerade, eine Begründung zu leisten. Tritt in dieser Umgebung eine Partikel auf, die diese kausale Lesart stützt, und eine andere, die die Adressierung des aktuell diskutierten Themas anzeigt, lässt sich folglich plausibel annehmen, dass gerade dieser Kontext die Hintergrundierung der Themaadressierung erlaubt. Die Konsequenz ist, dass die Partikel, die die Adressierung des aktuellen Diskursthemas kodiert (was für (assertive) Äußerungen prinzipiell hochrelevant ist), in genau diesem Kontext zurücktritt und somit später zur Applikation gebracht werden kann. Da jeder Kausalsatz ein assertiver Kontext ist, ist die Abfolge \textit{doch auch} natürlich möglich und auch geläufiger.

Vor dem Hintergrund meiner Überlegung, dass die umgekehrten Abfolgen in speziellen Kontexten, in denen die zweiten Bestandteile der Kombination weniger dominant sind, möglich werden, bietet sich auch eine Erklärung für den (soweit man dies sagen kann) positiven Einfluss des \textit{ja}-Kontextes und kausaler Nebensätze an.

Da man nicht ausschließen kann, dass sowohl der kausale Kontext als auch die Kombination mit \textit{ja} sowieso häufig auftreten, möchte ich in Betracht ziehen, dass auch allgemeinere (Verarbeitungs-)Prozesse einen Einfluss auf die Verwendung/Akzeptabilität der umgekehrten Abfolgen nehmen. In den DECOW-Daten, für die ich den MP-Gebrauch von \textit{auch} und \textit{doch} ansetze, findet sich auch jeweils ein Beleg mit \textit{wohl} und \textit{halt} (\textit{wohl auch doch}, \textit{halt auch doch}). Informell erfragte Sprecherurteile ergeben, dass Sprecher dazu tendieren, auch diese Kombinationen besser zu bewerten als das isolierte \textit{auch doch}. Da die Kombination \textit{ja auch} äußerst frequent ist, ist nicht auszuschließen, dass \glq bekannte Bestandteile\grq {} einen Einfluss auf die Bewertung nehmen. Verarbeitet man die Sequenz \textit{ja auch} als akzeptabel, stört in diesem Sinne auch das Hinzufügen von \textit{doch} weniger. Die Sequenzen \textit{wohl auch doch} und \textit{halt auch doch} treten in den Daten weniger auf als \textit{ja auch doch}, weil auch schon die Reihung \textit{ja auch} die Abfolgen \textit{wohl auch} und \textit{halt auch} deutlich übertrifft.

In Kapitel~\ref{chapter:jud}, Abschnitt~\ref{sec:ort} habe ich gegen die Kritik aus einem Gutachten argumentiert, dass die Reihung \textit{doch ja} durch eingefrorene Konstruktionen zustande komme, bei denen \textit{X}+\textit{doch} und \textit{ja}+\textit{X} ein festes Muster darstellen. In dem Sinne der dortigen Argumentation lehne ich diese Überlegung in Bezug auf genau diese Beispiele ab. Die aufzufindenden Dreierkombinationen, in denen \textit{auch doch} integriert vorkommt, könnten hier eher als ein solcher Fall eines bekannten Musters fungieren. Weist man nach, dass nicht nur die Dreierkombination mit \textit{ja} am linken Rand das Auftreten von \textit{auch doch} begünstigt, kann die Erklärung nicht mehr am konkreten Beitrag dieser Partikel hängen, sondern verliefe entlang allgemeinerer Verarbeitungsaspekte. Es bietet sich im Rahmen meines generellen Zugangs eine Erklärung an, die Bezug nimmt auf den Diskurseffekt von \textit{ja}. Die andere Erklärung möchte ich beim jetzigen Stand aber nicht ausschließen. In Bezug auf \textit{ja} widersprechen sich die beiden Erklärungen noch nicht einmal. Anderes gilt für \textit{wohl} und \textit{halt}. Beide Partikeln stellen nicht cg her, so dass die Hintergrundierung der Themaadressierung nicht angenommen werden kann. Dazu folgen sie \textit{doch} im unmarkierten Fall. 

Eine ähnliche Erklärung über die Verarbeitung bekannterTeilstrukturen lässt sich auch für das Auftreten von \textit{auch doch} in den kausalen Nebensätzen andenken: Stellte sich heraus, dass \textit{auch} in kausalen Nebensätzen frequent wäre (meine Verteilungsangaben zu \textit{da}-, \textit{denn}- und \textit{zumal}-Sätzen legen dies in Abschnitt~\ref{sec:korp} für \textit{zumal}-Sätze nahe), könnte auch das Vorkommen von \textit{auch} als Muster gelten, dessen Bekanntheit/Erwartetheit ausreicht, um ein hinzutretendes \textit{doch}, das an dieser Stelle eigentlich nicht gewünscht ist, zu \glq tolerieren\grq {}.\\

\noindent
Auch dieser Teil der Arbeit schließt folglich mit der Annahme, dass auch die umgekehrte Abfolge \textit{auch doch} nicht gänzlich ausgeschlossen werden sollte. Diese Reihung ist in den Daten sicherlich unterrepräsentiert und wird als weniger akzeptabel bewertet. Liegt eine gewisse Datenmenge vor (wie es DECOW ermöglicht), können m.E. aber Muster aufgedeckt werden, die zeigen, dass die Umkehr nicht völlig unsystematisch erfolgt und deshalb als Performanzfehler abgetan werden muss. Sicherlich sind aber weitere (vor allem empirische) Untersuchungen nötig, um diesen Aspekt weiter zu verfolgen. Die Daten, die ich anführe, geben für meine Begriffe aber zunächst ein Bild der realen Partikelverwendung.














 










\chapter{Lexical sense relations}\label{sec:6}

\section{Meaning relations between words}\label{sec:6.1}

A traditional way of investigating the meaning of a word is to study the relationships between its meaning and the meanings of other words: which words have the same meaning, opposite meanings, etc. Strictly speaking these relations hold between specific senses, rather than between words; that is why we refer to them as sense relations. For example, one sense of \textit{mad} is a synonym of \textit{angry}, while another sense is a synonym of \textit{crazy}.



In \sectref{sec:6.2} we discuss the most familiar classes of sense relations: synonymy, several types of antonymy, hyponymy, and meronymy. We will try to define each of these relations in terms of relations between sentence meanings, since it is easier for speakers to make reliable judgments about sentences than about words in isolation. Where possible we will mention some types of linguistic evidence that can be used as diagnostics to help identify each relation. In \sectref{sec:6.3} we mention some of the standard ways of defining words in terms of their sense relations. This is the approach most commonly used in traditional dictionaries.


\section{Identifying sense relations}\label{sec:6.2}

Let’s begin by thinking about what kinds of meaning relations are likely to be worth studying. If we are interested in the meaning of the word \textit{big}, it seems natural to look at its meaning relations with words like \textit{large}, \textit{small}, \textit{enormous}, etc. But comparing \textit{big} with words like \textit{multilingual} or \textit{extradite} seems unlikely to be very enlightening. The range of useful comparisons seems to be limited by some concept of semantic similarity or comparability.



Syntactic relationships are also relevant. The kinds of meaning relations mentioned above (same meaning, opposite meaning, etc.) hold between words which are mutually substitutable, i.e., which can occur in the same syntactic environments, as illustrated in (\ref{ex:6.1}a). These relations are referred to as \textsc{paradigmatic} sense relations. We might also want to investigate relations which hold between words which can occur in construction with each other, as illustrated in (\ref{ex:6.1}b). (In this example we see that \textit{big} can modify some head nouns but not others.) These relations are referred to as \textsc{syntagmatic} relations.


\ea \label{ex:6.1}
\ea Look at that \textit{big/large/small/enormous/?\#discontinuous/*snore} mosquito!\\
\ex Look at that big \textit{mosquito/elephant/?\#surname/\#color/*discontinuous/*snore}!
                       \z
\z


We will consider some syntagmatic relations in \chapref{sec:7}, when we discuss selectional restrictions. In this chapter we will be primarily concerned with paradigmatic relations.


\subsection{Synonyms}\label{sec:6.2.1}

We often speak of synonyms as being words that “mean the same thing”. As a more rigorous definition, we will say that two words are synonymous (for a specific sense of each word) if substituting one word for the other does not change the meaning of a sentence. For example, we can change sentence (\ref{ex:6.2}a) into sentence (\ref{ex:6.2}b) by replacing \textit{frightened} with \textit{scared}. The two sentences are semantically equivalent (each entails the other). This shows that \textit{frightened} is a synonym of \textit{scared}.


\ea \label{ex:6.2}
\ea John \textit{frightened} the children.\\
\ex John \textit{scared} the children.
                       \z
\z


“Perfect” synonymy is extremely rare, and some linguists would say that it never occurs. Even for senses that are truly equivalent in meaning, there are often collocational differences as illustrated in (\ref{ex:6.3}--\ref{ex:6.4}). Replacing \textit{bucket} with \textit{pail} in (\ref{ex:6.3}a) does not change meaning; but in (\ref{ex:6.3}b), the idiomatic meaning that is possible with \textit{bucket} is not available with \textit{pail}. Replacing \textit{big} with \textit{large} does not change meaning in most contexts, as illustrated in (\ref{ex:6.4}a); but when used as a modifier for certain kinship terms, the two words are no longer equivalent (\textit{big} becomes a synonym of \textit{elder}), as illustrated in (\ref{ex:6.4}b).


\ea \label{ex:6.3}
\ea John filled the \textit{bucket}/\textit{pail}.\\
\ex John kicked the \textit{bucket}/??\textit{pail}.
                       \z
\z

\ea \label{ex:6.4}
\ea Susan lives in a \textit{big}/\textit{large} house.\\
\ex Susan lives with her \textit{big}/\textit{large} sister.\footnote{Adapted from \citet[66]{Saeed2009}.}
                       \z
\z

\subsection{Antonyms}\label{sec:6.2.2}

Antonyms are commonly defined as words with “opposite” meaning; but what do we mean by “opposite”? We clearly do not mean ‘as different as possible’. As noted above, the meaning of \textit{big} is totally different from the meanings of \textit{multilingual} or \textit{extradite}, but neither of these words is an antonym of \textit{big}. When we say that \textit{big} is the opposite of \textit{small}, or that \textit{dead} is the opposite of \textit{alive}, we mean first that the two terms can have similar collocations. It is odd to call an inanimate object \textit{dead}, in the primary, literal sense of the word, because it is not the kind of thing that could ever be \textit{alive}. Second, we mean that the two terms express a value of the same property or attribute. \textit{Big} and \textit{small} both express degrees of size, while \textit{dead} and \textit{alive} both express degrees of vitality. So two words which are antonyms actually share most of their components of meaning, and differ only with respect to the value of one particular feature.



The term \textsc{antonym} actually covers several different sense relations. Some pairs of antonyms express opposite ends of a particular scale, like \textit{big} and \textit{small}. We refer to such pairs as \textsc{scalar} or \textsc{gradable} antonyms. Other pairs, like \textit{dead} and \textit{alive}, express discrete values rather than points on a scale, and name the only possible values for the relevant attribute. We refer to such pairs as \textsc{simple} or \textsc{complementary} antonyms. Several other types of antonyms are commonly recognized as well. We begin with simple antonyms.


\subsubsection{Complementary pairs (simple antonyms)}\label{sec:6.2.2.1}
\begin{quote}\small
“All men are created equal. Some, it appears, are created a little more equal than others.”\hfill 
\footnotesize{[Ambrose Bierce, In \textit{The San Francisco Wasp} magazine, September 16, 1882]}
\end{quote}


Complementary pairs such as \textit{open/shut}, \textit{alive/dead}, \textit{male/female}, \textit{on/off}, etc. exhaust the range of possibilities, for things that they can collocate with. There is (normally) no middle ground; a person is either alive or dead, a switch is either on or off, etc. The defining property of simple antonyms is that replacing one member of the pair with the other, as in \REF{ex:6.5}, produces sentences which are \textsc{contradictory.} As discussed in \chapref{sec:3}, this means that the two sentences must have opposite truth values in every circumstance; one of them must be true and the other false in all possible situations where these words can be used appropriately.


\ea \label{ex:6.5}
\ea The switch is on.\\
\ex The switch is off.\\
\ex ??The switch is neither on nor off.
                       \z
\z


If two sentences are contradictory, then one or the other must always be true. This means that simple antonyms allow for no middle ground, as indicated in (\ref{ex:6.5}c). The negation of one entails the truth of the other, as illustrated in \REF{ex:6.6}.


\ea \label{ex:6.6}
\ea ??The post office is not open today, but it is not closed either.\\
\ex ??Your headlights are not off, but they are not on either.
                       \z
\z


A significant challenge in identifying simple antonyms is the fact that they are easily coerced into acting like gradable antonyms.\footnote{\citet[463]{Cann2011}.} For example, \textit{equal} and \textit{unequal} are simple antonyms; the humor in the quote by Ambrose Bierce at the beginning of this section arises from the way he uses \textit{equal} as if it were gradable. In a similar vein, zombies are often described as being \textit{undead}, implying that they are not dead but not really alive either. However, the gradable use of simple antonyms is typically possible only in certain figurative or semi-idiomatic expressions. The gradable uses in \REF{ex:6.7} seem natural, but those in \REF{ex:6.8} are not. The sentences in \REF{ex:6.9} illustrate further contrasts. For true gradable antonyms, like those discussed in the following section, all of these patterns would generally be fully acceptable, not odd or humorous.


\ea \label{ex:6.7}
\ea half-dead, half-closed, half-open\\
\ex more dead than alive\\
\ex deader than a door nail
                       \z
\z

\ea \label{ex:6.8}
\ea ?half-alive\\
\ex \#a little too dead\\
\ex \#not dead enough\\
\ex \#How dead is that mosquito?\\
\ex \#This mosquito is deader than that one.
                       \z
\z

\ea \label{ex:6.9}
\ea I feel fully/very/??slightly alive.\\
\ex This town/\#mosquito seems very/slightly dead.
                       \z
\z

\subsubsection{Gradable (scalar) antonyms}\label{sec:6.2.2.2}

A defining property of gradable (or scalar) antonyms is that replacing one member of such a pair with the other produces sentences which are \textsc{contrary}, as illustrated in (\ref{ex:6.10}a--b). As discussed in \chapref{sec:3}, contrary sentences are sentences which cannot both be true, though they may both be false (\ref{ex:6.10}c).


\ea \label{ex:6.10}
\ea My youngest son-in-law is extremely diligent.\\
\ex My youngest son-in-law is extremely lazy.\\
\ex My youngest son-in-law is neither extremely diligent nor extremely lazy.
                       \z
\z


Note, however, that not all pairs of words which satisfy this criterion would normally be called “antonyms”. The two sentences in \REF{ex:6.11} cannot both be true (when referring to the same thing), which shows that \textit{turnip} and \textit{platypus} are \textsc{incompatibles}; but they are not antonyms. So our definition of gradable antonyms needs to include the fact that, as mentioned above, they name opposite ends of a single scale and therefore belong to the same semantic domain.


\ea \label{ex:6.11}
\ea This thing is a turnip.\\
\ex This thing is a platypus.
                       \z
\z


The following diagnostic properties can help us to identify scalar antonyms, and in particular to distinguish them from simple antonyms:\footnote{Adapted from \citet[67]{Saeed2009}; \citet[204ff.]{Cruse1986}.}

{\sloppy
\begin{enumerate}[label=\alph*.]
\item Scalar antonyms typically have corresponding intermediate terms, e.g. \textit{warm, tepid, cool} which name points somewhere between \textit{hot} and \textit{cold} on the temperature scale.
\item Scalar antonyms name values which are relative rather than absolute. For example, a small elephant will probably be much bigger than a big mosquito, and the temperature range we would call hot for a bath or a cup of coffee would be very cold for a blast furnace.
\item As discussed in \chapref{sec:5}, scalar antonyms are often vague.
\item Comparative forms of scalar antonyms are completely natural (\textit{hotter}, \textit{colder}, etc.), whereas they are normally much less natural with complementary antonyms, as illustrated in (\ref{ex:6.8}e) above.
\item The comparative forms of scalar antonyms form a converse pair (see below).\footnote{\citet[232]{Cruse1986}.} For example, \textit{A is longer than B}  ↔   \textit{B is shorter than A}.
\item One member of a pair of scalar antonyms often has privileged status, or is felt to be more basic, as illustrated in \REF{ex:6.12}.
\end{enumerate}
}

\ea \label{ex:6.12}
\ea How old/??young are you?\\
\ex How tall/??short are you?\\
\ex How deep/??shallow is the water?
                       \z
\z

\subsubsection{Converse pairs}\label{sec:6.2.2.3}

Converse pairs involve words that name an asymmetric relation between two entities, e.g. \textit{parent-child, above}-\textit{below}, \textit{employer-employee}.\footnote{\citet[231]{Cruse1986} refers to such pairs as \textsc{relational} \textsc{opposites}.} The relation must be asymmetric or there would be no pair; symmetric relations like \textit{equal} or \textit{resemble} are (in a sense) their own converses. The two members of a converse pair express the same basic relation, with the positions of the two arguments reversed. If we replace one member of a converse pair with the other, and also reverse the order of the arguments, as in (\ref{ex:6.13}--\ref{ex:6.14}), we produce sentences which are semantically equivalent (paraphrases).


\ea \label{ex:6.13}
\ea Michael is my advisor.\\
\ex I am Michael’s advisee.
                       \z
\z

\ea \label{ex:6.14}
OWN(x,y) ↔  BELONG\_TO(y,x)\\
ABOVE(x,y) ↔  BELOW(y,x)\\
PARENT\_OF(x,y) ↔  CHILD\_OF(y,x)
\z

\subsubsection{Reverse pairs}\label{sec:6.2.2.4}

Two words (normally verbs) are called \textsc{reverses} if they “denote motion or change in opposite directions… [I]n addition… they should differ only in respect of directionality” \citep[226]{Cruse1986}. Examples include \textit{push/pull, come/go}, \textit{fill/empty}, \textit{heat/cool}, \textit{strengthen/weaken}, etc. Cruse notes that some pairs of this type (but not all) allow an interesting use of \textit{again}, as illustrated in \REF{ex:6.15}. In these sentences, \textit{again} does not mean that the action named by the second verb is repeated (\textsc{repetitive} reading), but rather that the situation is restored to its original state (\textsc{restitutive} reading).


\ea \label{ex:6.15}
\ea The nurse heated the instruments to sterilize them, and then cooled them \textit{again}.\\
\ex George filled the tank with water, and then emptied it \textit{again}.
                       \z
\z

\subsection{Hyponymy and taxonomy}\label{sec:6.2.3}

When two words stand in a generic-specific relationship, we refer to the more specific term (e.g. \textit{moose}) as the \textsc{hyponym} and to the more generic term (e.g. \textit{mammal}) as the \textsc{superordinate} or \textsc{hyperonym}. A generic-specific relationship can be defined by saying that a simple positive non-quantified statement involving the hyponym will entail the same statement involving the superordinate, as illustrated in \REF{ex:6.16}. (In each example, the hyponym and superordinate term are set in boldface.) We need to specify that the statement is positive, because negation reverses the direction of the entailments \REF{ex:6.17}.


\ea \label{ex:6.16}
\ea \textit{Seabiscuit was a \textbf{stallion}}  entails:  \textit{Seabiscuit was a \textbf{horse}}.\\
\ex \textit{Fred \textbf{stole} my bicycle}  entails:  \textit{Fred \textbf{took} my bicycle}.\\
\ex \textit{John \textbf{assassinated} the Mayor}  entails:  \textit{John \textbf{killed} the Mayor}.\\
\ex \textit{Arthur looks like a \textbf{squirrel}}  entails:  \textit{Arthur looks like a \textbf{rodent}}.\\
\ex \textit{This pot is made of \textbf{copper}}  entails:  \textit{This pot is made of \textbf{metal}}.
                       \z
\z

\ea \label{ex:6.17}
\ea \textit{Seabiscuit was not a \textbf{horse}}  entails:  \textit{Seabiscuit was not a \textbf{stallion}}.\\
\ex \textit{John did not \textbf{kill} the Mayor}  entails:  \textit{John did not \textbf{assassinate} the Mayor}.\\
\ex \textit{This pot is not made of \textbf{metal}}  entails:  \textit{This pot is not made of \textbf{copper}}.
  \z
\z


\textsc{Taxonomy} is a special type of hyponymy, a classifying relation. \citet[137]{Cruse1986} suggests the following diagnostic: X is a \textsc{taxonym} of Y if it is natural to say \textit{An X is a kind/type of Y}. Examples of taxonomy are presented in (\ref{ex:6.18}a--b), while the examples in (\ref{ex:6.18}c--d) show that other hyponyms are not fully natural in this pattern. (The word \textsc{taxonymy} is also used to refer to a generic-specific hierarchy, or system of classification.)


\ea \label{ex:6.18}
\ea \textit{A beagle is a kind of dog}.\\
\ex \textit{Gold is a type of metal}.\\
\ex ?\textit{A stallion is a kind of horse.}\\
\ex ??\textit{Sunday is a kind of day of the week}.
                       \z
\z


\textsc{Taxonomic sisters} are taxonyms which share the same superordinate term, such as \textit{squirrel} and \textit{mouse} which are both hyponyms of \textit{rodent}.\footnote{More general labels for hyponyms of the same superordinate term, whether or not they are part of a taxonomy, include \textsc{hyponymic sisters} and \textsc{cohyponyms}.} Taxonomic sisters must be incompatible, in the sense defined above; for example, a single animal cannot be both a squirrel and a mouse. But that property alone does not distinguish taxonomy from other types of hyponymy. Taxonomic sisters occur naturally in sentences like the following:

\ea \label{ex:6.19} \ea \textit{A beagle is a kind of dog, and so is a Great Dane}.\\
\ex \textit{Gold is a type of metal, and copper is another type of metal}.
\z \z


Cruse notes that taxonomy often involves terms that name \textsc{natural kinds} (e.g., names of species, substances, etc.). Natural kind terms cannot easily be paraphrased by a superordinate term plus modifier, as many other words can (see §3 below):


\ea 
\label{ex:6.20}
\ea “\textit{Stallion” means a male horse.}\\
\ex \textit{“Sunday” mean}\textit{s the first day of the week}.\\
\ex \textit{??“Beagle” means a {\longrule} dog}.\\
\ex \textit{??“Gold” means a {\longrule} metal}.\\
\ex \textit{??“Dog” means a {\longrule} animal}.
\z \z


We must remember that semantic analysis is concerned with properties of the object language, rather than scientific knowledge. The taxonomies revealed by linguistic evidence may not always match standard scientific classifications. For example, the authoritative \textit{Kamus Dewan} (a \ili{Malay} dictionary published by the national language bureau in Kuala Lumpur) gives the following definition for \textit{labah-labah} ‘spider’:


\ea \label{ex:6.21
}\textit{labah-labah: sejenis \textbf{serangga} yang berkaki lapan}\\
‘spider: a kind of \textbf{insect} that has eight legs’
\z


This definition provides evidence that in \ili{Malay}, \textit{labah-labah} ‘spider’ is a taxonym of \textit{serangga} ‘insect’, even though standard zoological classifications do not classify spiders as insects. (Thought question: does this mean that \textit{serangga} is not an accurate translation equivalent for the English word \textit{insect}?)



Similar examples can be found in many different languages. For example, in \ili{Tuvaluan} (a  {Polynesian} language), the words for ‘turtle’ and ‘dolphin/whale’ are taxonyms of \textit{ika} ‘fish’.\footnote{\citet[192]{Finegan1999}.} The fact that turtles, dolphins and whales are not zoologically classified as fish is irrelevant to our analysis of the lexical structure of \ili{Tuvaluan}.


\subsection{Meronymy}\label{sec:6.2.4}

A \textsc{meronymy} is a pair of words expressing a part-whole relationship. The word naming the part is called the meronym. For example, \textit{hand}, \textit{brain} and \textit{eye} are all meronyms of \textit{body}; \textit{door}, \textit{roof} and \textit{kitchen} are all meronyms of \textit{house}; etc.



Once again, it is important to remember that when we study patterns of mero\-nymy, we are studying the structure of the lexicon, i.e., relations between words and not between the things named by the words. One linguistic test for identifying meronymy is the naturalness of sentences like the following: \textit{The parts of an X include the Y, the Z, ...} \citep[161]{Cruse1986}.



A meronym is a name for a part, and not merely a piece, of a larger whole. Human languages have many words that name parts of things, but few words that name pieces. \citet[158--159]{Cruse1986} lists three differences between parts and pieces. First, a part has autonomous identity: many shops sell automobile parts which have never been structurally integrated into an actual car. A piece of a car, on the other hand, must have come from a complete car. (Few shops sell pieces of automobile.) Second, the boundaries of a part are motivated by some kind of natural boundary or discontinuity — potential for separation or motion relative to neighboring parts, joints (e.g. in the body), difference in material, narrowing of connection to the whole, etc. The boundaries of a piece are arbitrary. Third, a part typically has a definite function relative to the whole, whereas this is not true for pieces.


\section{Defining words in terms of sense relations}\label{sec:6.3}

Traditional ways of defining words depend heavily on the use of sense relations; hyponymy has played an especially important role. The classical form of a definition, going back at least to Aristotle (384–322 BC), is a kind of phrasal synonym; that is, a phrase which is mutually substitutable with the word being defined (same syntactic distribution) and equivalent or nearly equivalent in meaning.



The standard way of creating a definition is to start with the nearest superordinate term for the word being defined (traditionally called the \textit{genus proximum}), and then add one or more modifiers (traditionally called the \textit{differentia specifica}) which will unambiguously distinguish this word from its hyponymic sisters. So, for example, we might define \textit{ewe} as ‘an adult female sheep’; \textit{sheep} is the superordinate term, while \textit{adult} and \textit{female} are modifiers which distinguish ewes from other kinds of sheep.


\newpage 
This structure can be further illustrated with the following well-known definition by Samuel Johnson (1709--1784), himself a famous lexicographer. It actually consists of two parallel definitions; the superordinate term in the first is \textit{writer}, and in the second \textit{drudge}. The remainder of each definition provides the modifiers which distinguish lexicographers from other kinds of writers or drudges.


\ea \label{ex:6.22}
\textit{Lexicographer}: A writer of dictionaries; a harmless drudge that busies himself in tracing the [origin], and detailing the signification of words.
\z


Some additional examples are presented in \REF{ex:6.23}. In each definition the superordinate term is bolded while the distinguishing modifiers are placed in square brackets.


\ea \label{ex:6.23}
\ea \textit{fir} (N): a kind of \textbf{tree} [with evergreen needles].\footnote{\citet[62]{HartmannJames1998}.}\\
\ex \textit{rectangle} (N): a [right-angled] \textbf{quadrilateral}.\footnote{\citet[219]{Svensén2009}.}\\
\ex \textit{clean} (Adj): \textbf{free} [from dirt].\footnote{\citet[219]{Svensén2009}.}
                       \z
\z


However, as a number of authors have pointed out, many words cannot easily be defined in this way. In such cases, one common alternative is to define a word by using synonyms (\ref{ex:6.24}a--b) or antonyms (\ref{ex:6.24}c--d).


\ea \label{ex:6.24}
\ea \textit{grumpy}: moodily cross; surly.\footnote{\url{http://www.merriam-webster.com/dictionary/}}\\
\ex \textit{sad}: affected with or expressive of grief or unhappiness.\footnote{\url{http://www.merriam-webster.com/dictionary/}}\\
\ex \textit{free}: not controlled by obligation or the will of another;\\
  not bound, fastened, or attached.\footnote{\url{http://www.thefreedictionary.com/free}} \\
\ex \textit{pure}: not mixed or adulterated with any other substance or material.\footnote{\url{http://oxforddictionaries.com/us/definition/american_english/pure}} 
                       \z
\z


Another common type of definition is the \textsc{extensional} definition. This definition spells out the denotation of the word rather than its sense as in a normal definition. This type is illustrated in \REF{ex:6.25}.


\ea \label{ex:6.25}
Definitions from Merriam-Webster on-line dictionary:\\
\ea   \textit{New England}: the NE United States comprising the states of Maine, New Hampshire, Vermont, Massachusetts, Rhode Island, \& Connecticut
\ex  \textit{cat}: any of a family (Felidae) of carnivorous, usually solitary and nocturnal, mammals (as the domestic cat, lion, tiger, leopard, jaguar, cougar, wildcat, lynx, and cheetah)
\z \z

Some newer dictionaries, notably the COBUILD dictionary, make use of full sentence definitions rather than phrasal synonyms, as illustrated in \REF{ex:6.26}.

\ea \label{ex:6.26}
confidential: Information that is \textbf{confidential} is meant to be kept secret or private.\footnote{COBUILD dictionary, 3\textsuperscript{rd} edition (2001); cited in \citet{Rundell2006}.}
\z

\section{Conclusion}\label{sec:6.4}

In this chapter we have mentioned only the most commonly used sense relations (some authors have found it helpful to refer to dozens of others). We have illustrated various diagnostic tests for identifying sense relations, many of them involving entailment or other meaning relations between sentences. Studying these sense relations provides a useful tool for probing the meaning of a word, and for constructing dictionary definitions of words.



\furtherreading{



\citet[chapters 4--12]{Cruse1986} offers a detailed discussion of each of the sense relations mentioned in this chapter. \citet{Cann2011} provides a helpful overview of the subject.

}

\discussionexercises{
% \subsection*{Discussion exercise} %\label{sec:}
\paragraph*{}\noindent
Identify the meaning relations for the following pairs of words, and provide linguistic evidence that supports your identification:

\medskip 

\begin{tabular}{l >{\itshape}l >{\itshape}l}
a. & sharp & dull\\
b. & finite & infinite\\
c. & two & too\\
d. & arm & leg\\
\end{tabular}
\qquad
\begin{tabular}{l >{\itshape}l >{\itshape}l}
e. & hyponym &  hyperonym\\
f. & silver  & metal\\
g. & insert  & extract  \\
\\
\end{tabular}

}

\homeworkexercises{
% \subsection*{Homework exercises}
\paragraph*{Antonyms.\footnote{Adapted from \citet[82]{Saeed2009}, ex. 3.4.}}

Below is a list of incompatible pairs. (i) Classify each pair into one of the following types of relation: \textsc{simple antonyms, gradable antonyms, reverses, converses,} or \textsc{taxonomic sisters}. (ii) For each pair, provide at least one type of linguistic evidence (e.g. example sentences) that supports your decision, and where possible mention other types of evidence that would lend additional support.\\
 
\begin{tabular}{l l l}
a. & \textit{legal} & \textit{illegal}\\
b. & \textit{fat} & \textit{thin}\\
c. & \textit{raise} & \textit{lower}\\
d. & \textit{wine} & \textit{beer}\\
\end{tabular}
\qquad
\begin{tabular}{l l l}
e. & \textit{lend to} & \textit{borrow from}\\
f. & \textit{lucky} & \textit{unlucky}\\
g. & \textit{married} & \textit{unmarried}\\
~\\
\end{tabular}
}\vspace*{-33mm}


% % copy the lines above and adapt as necessary

%%%%%%%%%%%%%%%%%%%%%%%%%%%%%%%%%%%%%%%%%%%%%%%%%%%%
%%%                                              %%%
%%%             Backmatter                       %%%
%%%                                              %%%
%%%%%%%%%%%%%%%%%%%%%%%%%%%%%%%%%%%%%%%%%%%%%%%%%%%%

% \is{some term| see {some other term}}
% \il{some language| see {some other language}}
\ilsa{Kalaallisut}{West Greenlandic}
\ilsa{West Greenlandic}{Kalaallisut}
\ilsa{Malay}{Indonesian}
\ilsa{Indonesian}{Malay}
\ilsa{Chinese}{Mandarin}
\ilsa{Mandarin}{Chinese} 
% There is normally no need to change the backmatter section
\input{backmatter.tex} 
\end{document} 

% you can create your book by running
% xelatex main.tex
%
% you can also try a simple 
% make
% on the commandline

