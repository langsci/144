\chapter{The meaning of \textit{meaning}}\label{sec:1}

\section{Semantics and pragmatics}\label{sec:1.1}

The American author Mark Twain is said to have described a certain person as “a good man in the worst sense of the word.” The humor of this remark lies partly in the unexpected use of the word \textit{good}, with something close to the opposite of its normal meaning: Twain seems to be implying that this man is puritanical, self-righteous, judgmental, or perhaps hypocritical. Nevertheless, despite using the word in this unfamiliar way, Twain still manages to communicate at least the general nature of his intended message.



Twain’s witticism is a slightly extreme example of something that speakers do on a regular basis: using old words with new meanings. It is interesting to compare this example with the following famous conversation from \textit{Through the Looking Glass}, by Lewis Carroll:


\ea
{}[Humpty Dumpty speaking] “There’s glory for you!”

“I don’t know what you mean by ‘glory’,” Alice said.

Humpty Dumpty smiled contemptuously. “Of course you don’t — till I tell you. I meant ‘there’s a nice knock-down argument for you!’ ”

“But ‘glory’ doesn’t mean ‘a nice knock-down argument’,” Alice objected. 

“When I use a word,” Humpty Dumpty said, in rather a scornful tone, “it means just what I choose it to mean — neither more nor less.”

“The question is,” said Alice, “whether you can make words mean so many different things.”

“The question is,” said Humpty Dumpty, “which is to be master — that’s all.”
\z

Superficially, Humpty Dumpty’s comment seems similar to Mark Twain's:\linebreak both speakers use a particular word in a previously unknown way. The results, however, are strikingly different: Mark Twain successfully communicates (at least part of) his intended meaning, whereas Humpty Dumpty fails to communicate; throughout the ensuing conversation, Alice has to ask repeatedly what he means.



Humpty Dumpty’s claim to be the “master” of his words — to be able to use words with whatever meaning he chooses to assign them — is funny because it is absurd. If people really talked that way, communication would be impossible. Perhaps the most important fact about word meanings is that they must be shared by the speech community: speakers of a given language must agree, at least most of the time, about what each word means.



Yet, while it is true that words must have agreed-upon meanings, Twain’s remark illustrates how word meanings can be stretched or extended in various novel ways, without loss of comprehension on the part of the hearer. The contrast between Mark Twain’s successful communication and Humpty Dumpty’s failure to communicate suggests that the conventions for extending meanings must also be shared by the speech community. In other words, there seem to be rules even for bending the rules. In this book we will be interested both in the rules for “normal” communication, and in the rules for bending the rules.



The term \textsc{semantics} is often defined as the study of meaning. It might be more accurate to define it as the study of the relationship between linguistic form and meaning. This relationship is clearly rule-governed, just as other aspects of linguistic structure are. For example, no one believes that speakers memorize every possible sentence of a language; this cannot be the case, because new and unique sentences are produced every day, and are understood by people hearing them for the first time. Rather, language learners acquire a vocabulary (lexicon), together with a set of rules for combining vocabulary items into well-formed sentences (syntax). The same logic forces us to recognize that language learners must acquire not only the meanings of vocabulary items, but also a set of rules for interpreting the expressions that are formed when vocabulary items are combined. All of these components must be shared by the speech community in order for linguistic communication to be possible. When we study semantics, we are trying to understand this shared system of rules that allows hearers to correctly interpret what speakers intend to communicate.


\largerpage
The study of meaning in human language is often partitioned into two major divisions, and in this context the term \textsc{semantics} is used to refer to one of these divisions. In this narrower sense, semantics is concerned with the inherent meaning of words and sentences as linguistic expressions, in and of themselves, while \textsc{pragmatics} is concerned with those aspects of meaning that depend on or derive from the way in which the words and sentences are used. In the above-mentioned quote attributed to Mark Twain, the basic or “default” meaning of \textit{good} (the sense most likely to be listed in a dictionary) would be its semantic content. The negative meaning which Twain manages to convey is the result of pragmatic inferences triggered by the peculiar way in which he uses the word.



The distinction between semantics and pragmatics is useful and important, but as we will see in \chapref{sec:9}, the exact dividing line between the two is not easy to draw and continues to be a matter of considerable discussion and controversy. Because semantics and pragmatics interact in so many complex ways, there are good reasons to study them together, and both will be of interest to us in this book.


\section{Three “levels” of meaning}\label{sec:1.2}

In this book we will be interested in the meanings of three different types of linguistic units:


\begin{enumerate}
\item word meaning
\item sentence meaning
\item utterance meaning (also referred to as “speaker meaning”)
\end{enumerate}

The first two units (words and sentences) are hopefully already familiar to the reader. In order to understand the third level, “utterance meaning”, we need to distinguish between sentences vs. utterances. A sentence is a linguistic expression, a well-formed string of words, while an utterance is a speech event by a particular speaker in a specific context. When a speaker uses a sentence in a specific context, he produces an utterance. As hinted in the preceding section, the term \textsc{sentence meaning} refers to the semantic content of the sentence: the meaning which derives from the words themselves, regardless of context.\footnote{As we will see, this is an oversimplification, because certain aspects of sentence meaning do depend on context; see \chapref{sec:9}, §3 for discussion.} The term \textsc{utterance} \textsc{meaning} refers to the semantic content plus any pragmatic meaning created by the specific way in which the sentence gets used. \citet[27]{Cruse2000} defines utterance meaning as “the totality of what the speaker intends to convey by making an utterance.”



\citet[1]{Kroeger2005} cites the following example of a simple question in \ili{Teochew} (a Southern Min dialect of \ili{Chinese}), whose interpretation depends heavily on context.


\ea \label{ex:2}
\ea  \gll Lɯ  chyaʔ  pa  bɔy?\\
you  eat  full  not.yet\\
\glt ‘Have you already eaten?’  (tones not indicated)
\z \z


The literal meaning (i.e., sentence meaning) of the question is, “Have you already eaten or not?”, which sounds like a request for information. But its most common use is as a greeting. The normal way for one friend to greet another is to ask this question. (The expected reply is: “I have eaten,” even if this is not in fact true.) In this context, the utterance meaning is roughly equivalent to that of the English expressions \textit{hello} or \textit{How do you do?} In other contexts, however, the question could be used as a real request for information. For example, if a doctor wants to administer a certain medicine which cannot be taken on an empty stomach, he might well ask the patient “Have you eaten yet?” In this situation the sentence meaning and the utterance meaning would be essentially the same.


\section{Relation between form and meaning}\label{sec:1.3}

For most words, the relation between the form (i.e., phonetic shape) of the word and its meaning is arbitrary. This is not always the case. \textsc{Onomatopoetic} words are words whose forms are intended to be imitations of the sounds which they refer to, e.g. \textit{ding-dong} for the sound of a bell, or \textit{buzz} for the sound of a housefly. But even in these cases, the phonetic shape of the word (if it is truly a part of the vocabulary of the language) is partly conventional. The sound a dog makes is represented by the English word \textit{bow-wow}, the \ili{Balinese} word \textit{kong-kong}, the \ili{Armenian} word \textit{haf-haf}, and the \ili{Korean} words \textit{mung-mung} or \textit{wang-wang}.\footnote{\url{http://www.psychologytoday.com/blog/canine-corner/201211/how-dogs-bark-in-different-languages} (accessed 2018-01-22)} This cross-linguistic variation is presumably not motivated by differences in the way dogs actually bark in different parts of the world. On the other hand, as these examples indicate, there is a strong tendency for the corresponding words in most languages to use labial, velar, or labio-velar consonants and low back vowels.\footnote{Labial consonants such as /b, m/; velar consonants such as /g, ng/; or labio-velar consonants such as /w/. Low back vowels include /a, o/.} Clearly this is no accident, and reflects the non-arbitrary nature of the form-meaning relation in such words. The situation with “normal” words is quite different, e.g. the word for ‘dog’: \ili{Armenian} \textit{shun}, \ili{Balinese} \textit{cicin}, \ili{Korean} \textit{gae}, \ili{Tagalog} \textit{aso}, etc. No common phonological pattern is to be found here.



The relation between the form of a sentence (or other multi-word expression) and its meaning is generally not arbitrary, but \textsc{compositional}. This term means that the meaning of the expression is predictable from the meanings of the words it contains and the way they are combined. To give a very simple example, suppose we know that the word \textit{yellow} can be used to describe a certain class of objects (those that are yellow in color) and that the word \textit{submarine} can be used to refer to objects of another sort (those that belong to the class of submarines). This knowledge, together with a knowledge of English syntax, allows us to infer that when the Beatles sang about living in a \textit{yellow} \textit{submarine} they were referring to an object that belonged to both classes, i.e., something that was both yellow and a submarine.



This \textsc{principle of compositionality} is of fundamental importance to almost every topic in semantics, and we will return to it often. But once again, there are exceptions to the general rule. The most common class of exceptions are \textsc{idioms}, such as \textit{kick the bucket} for ‘die’ or \textit{X’s goose is cooked} for ‘X is in serious trouble’. Idiomatic phrases are by definition non-compositional: the meaning of the phrase is not predictable from the meanings of the individual words. The meaning of the whole phrase must be learned as a unit.



The relation between utterance meaning and the form of the utterance is neither arbitrary nor, strictly speaking, compositional. Utterance meanings are derivable (or “calculable”) from the sentence meaning and the context of the utterance by various pragmatic principles that we will discuss in later chapters. However, it is not always fully predictable; sometimes more than one interpretation may be possible for a given utterance in a particular situation.


\section{What does \textit{mean} mean?}\label{sec:1.4}

When someone defines semantics as “the study of meaning”, or pragmatics as “the study of meanings derived from usage”, they are defining one English word in terms of other English words. This practice has been used for thousands of years, and works fairly well in daily life. But if our goal as linguists is to provide a rigorous or scientific account of the relationship between form and meaning, there are obvious dangers in using this strategy. To begin with, there is the danger of circularity: a definition can only be successful if the words used in the definition are themselves well-defined. In the cases under discussion, we would need to ask: What is the meaning of \textit{meaning}? What does \textit{mean} mean?



One way to escape from this circularity is to translate expressions in the \textsc{object language} into a well-defined \textsc{metalanguage}. If we use English to describe the linguistic structure of  {Swahili},  {Swahili} is the object language and English is the metalanguage. However, both  {Swahili} and English are natural human languages which need to be analyzed, and both exhibit vagueness, ambiguities, and other features which make them less than ideal as a semantic metalanguage.



\newpage 
Many linguists adopt some variety of formal logic as a semantic metalanguage, and later chapters in this book provide a brief introduction to such an approach. Much of the time, however, we will be discussing the meaning of English expressions using English as the metalanguage. For this reason it becomes crucial to distinguish (object language) expressions we are trying to analyze from the (metalanguage) words we are using to describe our analysis. When we write “What is the meaning of \textit{meaning}?” or “What does \textit{mean} mean?”, we use italics to identify object language expressions. These italicized words are said to be \textsc{mentioned}, i.e., referred to as objects of study, in contrast to the metalanguage words which are \textsc{used} in their normal sense, and are written in plain font.



Let us return to the question raised above, “What do we mean by \textit{meaning}?” This is a difficult problem in philosophy, which has been debated for centuries, and which we cannot hope to resolve here; but a few basic observations will be helpful. We can start by noting that our interests in this book, and the primary concerns of linguistic semantics, are for the most part limited to the kinds of meaning that people intend to communicate via language. We will not attempt to investigate the meanings of “body language”, manner of dress, facial expressions, gestures, etc., although these can often convey a great deal of information. (In sign languages, of course, facial expressions and gestures do have linguistic meaning.) And we will not address the kinds of information that a hearer may acquire by listening to a speaker, which the speaker does not intend to communicate.



For example, if I know how your voice normally sounds, I may be able to deduce from hearing you speak that you have laryngitis, or that you are drunk. These are examples of what the philosopher Paul Grice called “natural meaning”, rather than linguistic meaning. Just as smoke “means” fire, and a rainbow “means” rain, a rasping whisper “means” laryngitis. \citet[15]{Levinson1983} uses the example of a detective questioning a suspect to illustrate another type of unintended communication. The suspect may say something which is inconsistent with the physical evidence, and this may allow the detective to deduce that the suspect is guilty, but his guilt is not part of what the suspect intends to communicate. Inferences of this type will not be a central focus of interest in this book.



An approach which has proven useful for the linguistic analysis of meaning is to focus on how speakers use language to talk about the world. This approach was hinted at in our discussion of the phrase \textit{yellow} \textit{submarine}. Knowing the meaning of words like \textit{yellow} or \textit{submarine} allows us to identify the class of objects in a particular situation, or universe of discourse, which those words can be used to refer to. Similarly, knowing the meaning of a sentence will allow us to determine whether that sentence is true in a particular situation or universe of discourse.



Technically, sentences like \textit{It is raining} are neither true nor false. Only an utterance of a certain kind (namely, a statement) can have a truth value. When a speaker utters this sentence at a particular time and place, we can look out the window and determine whether or not the speaker is telling the truth. The statement is true if its meaning corresponds to the situation being described: is it raining at that time and place? This approach is sometimes referred to as the \textsc{correspondence} theory of truth.



We might say that the meaning of a (declarative) sentence is the knowledge or information which allows speakers and hearers to determine whether it is true in a particular context. If we know the meaning of a sentence, the principle of compositionality places an important constraint on the meanings of the words which the sentence contains: the meaning of individual words (and phrases) must be suitable to compositionally determine the correct meaning for the sentence as a whole. Certain types of words (e.g., \textit{if}, \textit{and}, or \textit{but}) do not “refer” to things in the world; the meanings of such words can only be defined in terms of their contribution to sentence meanings.


\section{Saying, meaning, and doing}\label{sec:1.5}

The \ili{Teochew} question in \REF{ex:2} illustrates how a single sentence can be used to express two or more different utterance meanings, depending on the context. In one context the sentence is used to greet someone, while in another context the same sentence is used to request information. So this example demonstrates that a single sentence can be used to perform two or more different \textsc{speech acts}, i.e., things that people do by speaking.



In order to fully understand a given utterance, the addressee (= hearer) must try to answer three fundamental questions:


\begin{enumerate}
\item What did the speaker say? i.e., what is the semantic content of the sentence? (The philosopher Paul Grice used the term “What is said” as a way of referring to semantic content or sentence meaning.)
\item What did the speaker intend to communicate? (Grice used the term \textsc{implicature} for intended but unspoken meaning, i.e., aspects of utterance meaning which are not part of the sentence meaning.)
\item What is the speaker trying to do? i.e., what speech act is being performed?
\end{enumerate}

In this book we attempt to lay a foundation for investigating these three questions about meaning. We will return to the analysis of speech acts in \chapref{sec:10}; but for a brief example of why this is an important facet of the study of meaning, consider the word \textit{please} in examples (\ref{ex:3}a--b).


\ea \label{ex:3}
\ea \textit{Please} pass me the salt.\\
\ex Can you \textit{please} pass me the salt?
                       \z
\z


What does \textit{please} mean? It does not seem to have any real semantic content, i.e., does not contribute to the sentence meaning; but it makes an important contribution to the utterance meaning, in fact, two important contributions. First, it identifies the speech act which is performed by the utterances in which it occurs, indicating that they are \textsc{requests}. The word \textit{please} does not occur naturally in other kinds of speech acts. Second, this word is a marker of politeness; so it indicates something about the manner in which the speech act is performed, including the kind of social relationship which the speaker wishes to maintain with the hearer. So we see that we cannot understand the meaning of \textit{please} without referring to the speech act being performed.



The claim that the word \textit{please} does not contribute to sentence meaning is supported by the observation that misusing the word does not affect the truth of a sentence. We said that it normally occurs only in requests. If we insert the word into other kinds of speech acts, e.g. \textit{It seems to be raining, please}, the result is odd; but if the basic statement is true, adding \textit{please} does not make it false. Rather, the use of \textit{please} in this context is simply inappropriate (unless there is some contextual factor which makes it possible to interpret the sentence as a request).



The examples in \REF{ex:3} also illustrate an important aspect of how form and meaning are related with respect to speech acts. We will refer to the utterance in (\ref{ex:3}a) as a \textsc{direct} request, because the grammatical form (imperative) matches the intended speech act (request); so the utterance meaning is essentially the same as the sentence meaning. We will refer to the utterance in (\ref{ex:3}b) as an \textsc{indirect} request, because the grammatical form (interrogative) does not match the intended speech act (request); the utterance meaning must be understood by pragmatic inference.


\section{“More lies ahead” (a roadmap)}\label{sec:1.6}

As you have seen from the table of contents, the chapters of this book are organized into six units. In the first four units we introduce some of the basic tools, concepts, and terminology which are commonly used for analyzing and describing linguistic meaning. In the last two units we use these tools to explore the meanings of several specific classes of words and grammatical markers: modals, tense markers, \textit{if}, \textit{because}, etc.



The rest of this first unit is devoted to exploring two of the foundational concepts for understanding how we talk about the world: reference and truth. \chapref{sec:2} deals with reference and the relationship between reference and meaning. Just as a proper name can be used to refer to a specific individual, other kinds of noun phrase can be used to refer to people, things, groups, etc. in the world. The actual reference of a word or phrase depends on the context in which it is used; the meaning of the word determines what things it can be used to refer to in any given context.



\chapref{sec:3} deals with truth, and also with certain kinds of inference. We say that a statement is true if its meaning corresponds to the situation under discussion. Sometimes the meanings of two statements are related in such a way that the truth of one will give us reason to believe that the other is also true. For example, if I know that the statement in (\ref{ex:4}a) is true, then I can be quite certain that the statement in (\ref{ex:4}b) is also true, because of the way in which the meanings of the two sentences are related. A different kind of meaning relation gives us reason to believe that if a person says (\ref{ex:4}c), he must believe that the statement in (\ref{ex:4}a) is true. These two types of meaning-based inference, which we will call \textsc{entailment} and \textsc{presupposition} respectively, are of fundamental importance to most of the topics discussed in this book.


\ea \label{ex:4}
\ea John killed the wasp.\\
\ex The wasp died.\\
\ex John is proud that he killed the wasp.
                       \z
\z


\chapref{sec:4} introduces some basic logical notation that is widely used in semantics, and discusses certain patterns of inference based on truth values and logical structure.



Unit~\ref{unit:2} focuses on word meanings, starting with the observation that a single word can have more than one meaning. One of the standard ways of demonstrating this fact is by observing the ambiguity of sentences like the famous headline in \REF{ex:5}. Many of the issues we discuss in Unit~\ref{unit:2} with respect to “content words” (nouns, verbs, adjectives, etc.), such as ambiguity, vagueness, idiomatic uses, co-occurrence restrictions, etc., will turn out to be relevant in our later discussions of various kinds of “function words” and grammatical morphemes as well.


\ea \label{ex:5}
Headline: \textit{Reagan wins on budget, but more lies ahead}.
\z


Unit~\ref{unit:3} deals with a pattern of pragmatic inference known as \textsc{conversational implicature}: meaning which is intended by the speaker to be understood by the hearer, but is not part of the literal sentence meaning. Many people consider the identification of this type of inference, by the philosopher Paul Grice in the 1960s, to be the “birth-date” of pragmatics as a distinct field of study. It is another foundational concept that we will refer to in many of the subsequent chapters. \chapref{sec:10} discusses a class of conversational implicatures that has received a great deal of attention, namely indirect speech acts. As illustrated above in example (\ref{ex:3}b), an indirect speech act involves a sentence whose literal meaning seems to perform one kind of speech act (asking a question: \textit{Can you pass me the salt?}) used in a way which implicates a different speech act (request: \textit{Please pass me the salt}). \chapref{sec:11} discusses various types of expressions (e.g. sentence adverbs like \textit{frankly}, \textit{fortunately}, etc., honorifics and politeness markers, and certain types of “discourse particles”) whose meanings seem to contribute to the appropriateness of an utterance, rather than to the truth of a proposition. Some such meanings were referred to by Grice as a different kind of implicature.



Unit~\ref{unit:4} addresses the issue of compositionality: how the meanings of phrases and sentences can be predicted based on the meanings of the words they contain and the way those words are arranged (syntactic structure). It provides a brief introduction to some basic concepts in set theory, and shows how these concepts can be used to express the truth conditions of sentences. One topic of special interest is the interpretation of “quantified” noun phrases such as \textit{every person}, \textit{some animal}, or \textit{no student}, using set theory to state the meanings of such phrases. In Unit~\ref{unit:5} we will use this analysis of quantifiers to provide a way of understanding the meanings of modals (e.g. \textit{may}, \textit{must}, \textit{should}) and \textit{if} clauses.



Unit~\ref{unit:6} presents a framework for analyzing the meanings of tense and aspect markers. Tense and aspect both deal with time reference, but in different ways. As we will see, the use and interpretation of these markers often depends heavily on the type of situation being described.


\largerpage 
Each of these topics individually has been the subject of countless books and papers, and we cannot hope to give a complete account of any of them. This book is intended as a broad introduction to the field as a whole, a stepping stone which will help prepare you to read more specialized books and papers in areas that interest you.



\furtherreading{
For helpful discussions of the distinction between semantics vs. pragmatics, see \citet[ch. 1]{Levinson1983} and \citet[§1.2]{Birner20122013}. \citet[ch. 1]{Levinson1983} also provides a helpful discussion of Grice’s distinction between “natural meaning” vs. linguistic meaning.
}